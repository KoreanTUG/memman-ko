\chapter{For package users} \label{chap:packageusers}

    Many packages work just as well with \Pclass{memoir} as with any of the
standard classes. In some instances, though, there may be a problem. In other
instances the class and package have been designed to work together but
you have to be careful about any code that you might write.

\section{Class/package name clash} \label{sec:nameclash}

    A typical indication of a problem is an error message saying that a 
command has already been defined\index{Command \cs{...} already defined ...} 
(see \pref{alreadydefined}).

    When the class and a package both use the same name for a macro that
they define something has to give. For the sake of an example, assume that
memoir has defined a macro called \cs{foo} and a package \Ppack{pack} 
that is used in the
document also defines \cs{foo}. There are several options that you can 
choose which might resolve the difficulty.

\begin{enumerate}

\item Just keep the class definition:
\begin{lcode}
\documentclass{...}
\let\classfoo\foo   % save the class' definition
\let\foo\relax      % `undefine' \foo
\usepackage{pack}   % defines \foo
\let\foo\classfoo   % restore \foo to the class definition
...
\foo...             % the class \foo
\end{lcode}

\item Just keep the package definition:
\begin{lcode}
\documentclass{...}
\let\foo\relax      % `undefine' \foo
\usepackage{pack}   % defines \foo
...
\foo...             % the package \foo
\end{lcode}

\item Keep the class definition and rename the package definition:
\begin{lcode}
\documentclass{...}
\let\classfoo\foo   % save the class' definition
\let\foo\relax      % `undefine' \foo
\usepackage{pack}   % defines \foo
\let\packfoo\foo    % rename pack's \foo 
\let\foo\classfoo   % restore \foo to the class definition
...
\foo...             % the class \foo
\packfoo            % the package \foo
\end{lcode}

\item Keep the package definition and rename the class definition:
\begin{lcode}
\documentclass{...}
\let\classfoo\foo   % save the class' definition
\let\foo\relax      % `undefine' \foo
\usepackage{pack}   % defines \foo
...
\foo...             % the package \foo
\classfoo...        % the class \foo
\end{lcode}

\end{enumerate}

    A potential problem with these options can occur after the package is
loaded and there are class or package commands that you use that, 
knowingly or not, 
call \cs{foo} expecting to get the class or the package definition, one of
which is now not available (except under a different name).

    The \Ppack{memoir} class has been available since 2001. It seems likely
that if older packages clash with \Ppack{memoir} then, as eight years have
gone by, the authors are unlikely to do anything about it. If a newer 
package clashes then contact the author of the package. 

    If all else fails, ask for help on the \pixctt\ newsgroup.

\section{Support for bididirectional typesetting}

    The \Lpack{bidi} system~\cite{BIDI} provides means of bidirectional 
typesetting. The class has built in support for \Lpack{bidi} but this means 
that if you are defining your own macros there are some things you need 
to be aware of.

    When dealing with bidirectional texts the left-to-right\index{LTR} 
(LTR) direction
is the familiar one and \ltx\ is set up for this. When typesetting
right-to-left (RTL)\index{RTL} \Lpack{bidi} interchanges left and right. 
The support in \Ppack{memoir} consists of replacing many, but not all, of 
the right- and left-specific constructs. The replacement macros are:

\begin{syntax}
\cmd{\memRTLleftskip} \cmd{\memRTLrightskip} \\
\cmd{\memRTLvleftskip} \cmd{\memRTLvrightskip} \\
\cmd{\memRTLraggedright} \cmd{\memRTLraggedleft} \\
\end{syntax}
\glossary(memRTLleftskip)%
  {\cs{memRTLleftskip}}%
  {RTL (bidi) replacement for \cs{leftskip}}
\glossary(memRTLrightskip)%
  {\cs{memRTLrightskip}}%
  {RTL (bidi) replacement for \cs{rightskip}}
\glossary(memRTLvleftskip)%
  {\cs{memRTLvleftskip}}%
  {RTL (bidi) replacement for \cs{vleftskip}}
\glossary(memRTLrightskip)%
  {\cs{memRTLvrightskip}}%
  {RTL (bidi) replacement for \cs{vrightskip}}
\glossary(memRTLraggedleft)%
  {\cs{memRTLraggedleft}}%
  {RTL (bidi) replacement for \cs{raggedleft}}
\glossary(memRTLraggedright)%
  {\cs{memRTLraggedright}}%
  {RTL (bidi) replacement for \cs{raggedright}}

    In certain places, but not everywhere: \\
\cmd{\memRTLleftskip} is used instead of \cmd{\leftskip} \\
\cmd{\memRTLrightskip} is used instead of \cmd{\rightskip} \\
\cmd{\memRTLvleftskip} is used instead of \cmd{\vleftskip} \\
\cmd{\memRTLvrightskip} is used instead of \cmd{\vrightskip} \\
\cmd{\memRTLraggedleft} is used instead of \cmd{\raggedleft} \\
\cmd{\memRTLraggedright} is used instead of \cmd{\raggedright} \\

    So, if you are defining any macros that use the \cs{...skip} or
\cs{ragged...} macros you may have to use the \cs{memRTL...} version
instead. The \Ppack{memoir} definitions of these macros are simply:
\begin{lcode}
\newcommand*{\memRTLleftskip}{\leftskip}
\newcommand*{\memRTLrightskip}{\rightskip}
...
\newcommand*{\memRTLraggedleft}{\raggedleft}
\end{lcode}
The \Lpack{bidi} system redefines them to suit its purposes. If your
work will only be set LTR there is no need to use the \cs{memRTL...}
macros in any of your code. If you might ever be producing a bidirectional
document then you may have to use the \cs{memRTL...} versions of the standard
commands in your code. To determine where you might have to use them
you will have to consult the \Lpack{bidi} documentation as not every use
of the standard commands needs to be replaced. 




%#% extend
%#% extstart include an-example-book-design.tex

\svnidlong
{$Ignore: $}
{$LastChangedDate: 2013-04-24 17:14:15 +0200 (Wed, 24 Apr 2013) $}
{$LastChangedRevision: 442 $}
{$LastChangedBy: daleif $}


%%%%%%%%%%%%%%%%%%%%%%%%%%%%%%%%

\LMnote{2010/07/01}{This belongs here}
\clearpage

\makeatletter
\begin{comment}
%% Bringhurst chapter style
\makechapterstyle{bringhurst}{%
  \renewcommand{\chapterheadstart}{}
  \renewcommand{\printchaptername}{}
  \renewcommand{\chapternamenum}{}
  \renewcommand{\printchapternum}{}
  \renewcommand{\afterchapternum}{}
  \renewcommand{\printchaptertitle}[1]{%
    \raggedright\Large\scshape\MakeLowercase{##1}}
  \renewcommand{\afterchaptertitle}{%
    \vskip\onelineskip \hrule\vskip\onelineskip}
}
\end{comment}

\setsecheadstyle{\raggedright\scshape\MakeLowercase}
  \setbeforesecskip{-\onelineskip}
  \setaftersecskip{\onelineskip}

%%\setsubsecheadstyle{\renewcommand\@hangfrom[1]{\noindent ##1}\raggedright\itshape}
\setsubsecheadstyle{\sethangfrom{\noindent ##1}\raggedright\itshape}
  \setbeforesubsecskip{-\onelineskip}
  \setaftersubsecskip{\onelineskip}

%% Bringhurst page style
\makepagestyle{bringhurst}
\makeevenfoot{bringhurst}{\thepage}{}{}
\makeoddfoot{bringhurst}{}{}{\thepage}
\setlength{\pwlayi}{\headsep} % \verb?\setlength{\pwlayi}{\headsep}? headsep=\printlength{\headsep}, pwlayi = \printlength{\pwlayi}.
\addtolength{\pwlayi}{\topskip} % \verb?\addtolength{\pwlayi}{\topskip}? topskip=\printlength{\topskip}, pwlayi=\printlength{\pwlayi}.
\addtolength{\pwlayi}{7.3\onelineskip} % \verb?\addtolength{\pwlayi}{7.3\onelineskip}? onelineskip=\printlength{\onelineskip}, pwlayi=\printlength{\pwlayi}.
\newcommand{\bringpicr}[1]{%
  \setlength{\unitlength}{1pt}
  \begin{picture}(0,0)
    \put(\strip@pt\marginparsep, -\strip@pt\pwlayi){%
      \begin{minipage}[t]{\marginparwidth}
         \raggedright\itshape #1
      \end{minipage}}
  \end{picture}
}

\setlength{\pwlayii}{\marginparsep} % \verb?\setlength{\pwlayii}{\marginparsep}? marginparsep=\printlength{\marginparsep}, pwlayii=\printlength{\pwlayii}.
\addtolength{\pwlayii}{\marginparwidth} % \verb?\addtolength{\pwlayii}{\marginparwidth}?  marginparwidth=\printlength{\marginparwidth}, pwlayii=\printlength{\pwlayii}.
\newcommand{\bringpicl}[1]{%
  \setlength{\unitlength}{1pt}
  \begin{picture}(0,0)
    \put(-\strip@pt\pwlayii, -\strip@pt\pwlayi){%
      \begin{minipage}[t]{\marginparwidth}
        \raggedleft\itshape #1
      \end{minipage}}
  \end{picture}
}

\makepsmarks{bringhurst}{%
  \def\chaptermark##1{\markboth{##1}{##1}}
  \def\sectionmark##1{\markright{##1}}
}
\makeevenhead{bringhurst}{\bringpicl{\rightmark}}{}{}
\makeoddhead{bringhurst}{}{}{\bringpicr{\leftmark}}


\renewcommand{\cftchapterfont}{\normalfont}
\renewcommand{\cftchapterpagefont}{\normalfont}
\renewcommand{\cftchapterpresnum}{\bfseries}
\renewcommand{\cftchapterleader}{}
\renewcommand{\cftchapterafterpnum}{\cftparfillskip}
%%%\settocdepth{chapter}

\makeatother


\cleardoublepage
\pagestyle{bringhurst}
%%%\chapterstyle{bringhurst}
\headstyles{bringhurst}
%%%%%%%%%%%%%%%%%%%%%%%%%%%%%%%%
\PWnote{2009/07/05}{Added `book' to example design chapter title}
\chapter{An example book design} \label{chap:bringhurst}
 
%unitlength=\printlength{\unitlength}, \\
%pwlayi=\printlength{\pwlayi}, \\
%pwlayii=\printlength{\pwlayii}.

\section{Introduction}

    In this chapter I will work through a reasonably complete design
exercise. Rather than trying to invent something myself I am taking the design
of Bringhurst's 
\textit{The Elements of Typographic Style}~\cite{BRINGHURST99}
as the basis of the
exercise. This is sufficiently different from the normal LaTeX appearance
to demonstrate most of the class capabilities, and also it is a design by
a leading proponent of good typography.

    As much as possible, this chapter is typeset according to the results
of the exercise to provide both a coding and a graphic example.

\section{Design requirements}

    The \textit{Elements of Typographic Style} is typeset using 
Minion\facesubseeidx{Minion} as the text font and Syntax\facesubseeidx{Syntax} 
(a sans font) 
for the captions. 
%The page layout has been
%shown diagramatically in \fref{fb:1} on \pref{fb:1}, but further details need
%to be described for those not fortunate enough to have a copy of their own.

    The trimmed page size is 23 by 13.3cm. The foredge is 3.1cm and the
top margin\index{margin!upper} is 1.9cm.

    As already noted, the font for the main text is Minion\index{Minion}, 
with 12pt leading
on a 21pc measure with 42 lines per page. For the purposes of this exercise
I will assume that Minion can be replaced by the font used for this
manual. The captions to figures\index{figure} and tables\index{table} are 
unnamed and 
unnumbered and typeset in Syntax\index{Syntax}. The captions give the 
appearance of being
in a smaller font size than the main text, which is often the case. I'll
assume that the \cmd{\small}\cmd{\sfseries} font will reasonably do for the
captions. 

    The footer\index{footer} is the same width as the 
typeblock\index{typeblock} and the folio\index{folio} is placed 
in the footer\index{footer} at the \foredge. There are two blank lines between 
the bottom of the typeblock\index{typeblock} and the folio\index{folio}.

    There is no header\index{header} in the usually accepted sense of the 
term but the chapter title is put on recto pages and section titles are on 
verso pages. The running titles are placed in the \foredge\ 
margin\index{margin!foredge?\foredge} 
level with the seventh line of the text in the typeblock\index{typeblock}. 
The recto headers\index{header} are typeset raggedright
and the verso ones raggedleft.

Bringhurst also uses many marginal\index{marginalia} notes,
their maximum width being about 51pt, and typeset raggedright in a smaller
version of the textfont.

    Chapter titles are in small caps, lowercase, in a larger font than for 
the main text, and a rule is placed between the title and the 
typeblock\index{typeblock}. The total vertical space used by a chapter 
title is three text lines. Chapters are not numbered in the text but are 
in the \toc.

    Section titles are again in small caps, lowercase, in the same size as the
text font. The titles are numbered, with both the chapter and section number.

    A subsection title, which is the lowest subdivision in the book, is in
the italic form of the textfont and is typeset as a numbered non-indented
paragraph\index{paragraph}. These are usually multiline as Bringhurst 
sometimes uses them like an enumerated list, so on occasion there is a 
subsection title with no following text.

    Only chapter titles are put into the \toc, and these are set raggedright 
with the page numbers immediately after the titles. There is no \lof\ or
\lot.

    Note that unlike the normal \ltx\ use of fonts, essentially only three
sizes of fonts are used --- the textfont size, one a bit larger for the
chapter titles, and one a bit smaller for marginal\index{marginalia} notes 
and captions. Also, bold fonts are not used except on special occasions, 
such as when he is comparing font families and uses large bold
versions to make the differences easier to see.

\section{Specifying the page and typeblock}

    The first and second things to do are to specify the sizes of the page
after trimming and the typeblock\index{typeblock}. The trimmed size is 
easy as we have the dimensions.
\begin{lcode}
\settrimmedsize{23cm}{13.3cm}{*}
\end{lcode}
We want 42 lines of text, so that's what we set as the height of the 
typeblock\index{typeblock}; however, we have to remember to ask for 
\texttt{lines} as the optional \meta{algorithm} argument when we 
finally call \cmd{\checkandfixthelayout}.
\begin{lcode}
\settypeblocksize{42\onelineskip}{21pc}{*}
\end{lcode}

    To make life easier, we'll do no trimming of the top of the 
stock\index{stock}
\begin{lcode}
\setlength{\trimtop}{0pt}
\end{lcode}
but will trim the \foredge. The next set of calculations first sets the 
value of the \lnc{\trimedge} to be the \lnc{\stockwidth}; subtracting the
trimmed \lnc{\paperwidth} then results in \lnc{\trimedge} being the amount
to trim off the foredge.
\begin{lcode}
\setlength{\trimedge}{\stockwidth}
\addtolength{\trimedge}{-\paperwidth}
\end{lcode}

    The sizes of the trimmed page and the typeblock\index{typeblock} have 
now been specified. The typeblock\index{typeblock} is now positioned on 
the page. The sideways positioning is
easy as we know the \foredge\ margin\index{margin!foredge!\foredge} 
to be 3.1cm.
\begin{lcode}
\setlrmargins{*}{3.1cm}{*}
\end{lcode}
The top margin\index{margin!top} is specified as 1.9cm, which is very 
close to 
four and a half lines of text. Just in case someone might want to use a 
different font size, I'll specify the top margin\index{margin!top} 
so that it 
is dependent on the font size. The
\lnc{\footskip} can be specified now as well (it doesn't particularly matter
what we do about the header-related lengths as there isn't anything above
the typeblock\index{typeblock}).
\begin{lcode}
\setulmargins{4.5\onelineskip}{*}{*}
\setheadfoot{\onelineskip}{3\onelineskip}
\setheaderspaces{\onelineskip}{*}{*}
\end{lcode}

    Lastly define the dimensions for any marginal\index{marginalia} notes.
\begin{lcode}
\setmarginnotes{17pt}{51pt}{\onelineskip}
\end{lcode}

    If this was for real, the page layout would have to be checked and
implemented.
\begin{lcode}
\checkandfixthelayout[lines]
\end{lcode}

    It is possible to implement this layout just for this chapter but
I'm not going to tell you either how to do it, or demonstrate it. Except
under exceptional circumstances it is not good to make such drastic changes
to the page layout in the middle of a document. However, the picture on
\pref{fig:bplayout} illustrates
how this layout would look on US letterpaper\index{paper!size!letterpaper} 
stock\index{stock}. Looking at the illustration suggests that the layout 
would look rather odd unless the stock\index{stock} was trimmed down
to the page size --- another reason for not switching the layout here.

\begin{figure}
\captiontitlefont{\small\sffamily}
\captionstyle{\centerlastline}
\setstocksize{11in}{8.5in}
\settrimmedsize{23cm}{13.3cm}{*}
\settypeblocksize{41\onelineskip}{21pc}{*}
\setlength{\trimtop}{0pt}
\setlength{\trimedge}{\stockwidth}
\addtolength{\trimedge}{-\paperwidth}
\setlrmargins{*}{3.1cm}{*}
\setulmargins{4.5\onelineskip}{*}{*}
\setheadfoot{\onelineskip}{3\onelineskip}
\setheaderspaces{\onelineskip}{*}{*}
\setmarginnotes{17pt}{51pt}{\onelineskip}
\checkandfixthelayout
\currentstock
\oddpagelayouttrue
\twocolumnlayoutfalse
\drawmarginparstrue
\drawparametersfalse
\drawstock
\legend{An illustration of Bringhurst's page layout style when printed
on US letter paper stock. Also shown are the values used for the
page layout parameters for this design.} \label{fig:bplayout}
\end{figure}

\section{Specifying the sectional titling styles}

\subsection{The chapter style}

    Recapping, chapter titles are in small caps, lowercase, in a larger 
font than for the main text, and a rule is placed between the title and the 
typeblock\index{typeblock}.
The total vertical space used by a chapter title is three text lines.
Chapters are not numbered in the text but are in the \toc.
Titles in the \toc\ are in mixed case.

    The definition of the chapterstyle is remarkably simple, as shown below.
\begin{lcode}
%% Bringhurst chapter style
\makechapterstyle{bringhurst}{%
  \renewcommand{\chapterheadstart}{}
  \renewcommand{\printchaptername}{}
  \renewcommand{\chapternamenum}{}
  \renewcommand{\printchapternum}{}
  \renewcommand{\afterchapternum}{}
  \renewcommand{\printchaptertitle}[1]{%
    \raggedright\Large\scshape\MakeLowercase{##1}}
  \renewcommand{\afterchaptertitle}{%
    \vskip\onelineskip \hrule\vskip\onelineskip}
}
\end{lcode}

    Most of the specification consists of nulling the majority of the normal
LaTeX specification, and modifying just two elements. 

The chapter title (via \cmd{\printchaptertitle}) 
is typeset raggedright using the \cmd{\Large} smallcaps fonts. The 
\cmd{\MakeLowercase} macro is used to ensure that the entire title is 
lowercase before typesetting it. Titles are input in mixed case.

    After the title is typeset the \cmd{\afterchaptertitle} macro
specifies that one line is skipped, a horizontal rule
is drawn and then another line is skipped.

\subsection{Lower level divisions}

    Section titles are in small caps, lowercase, in the same size as the
text font. The titles are numbered, with both the chapter and section number.

The specification is:
\begin{lcode}
\setsecheadstyle{\raggedright\scshape\MakeLowercase}
  \setbeforesecskip{-\onelineskip}
  \setaftersecskip{\onelineskip}
\end{lcode}

    The macro \cmd{\setsecheadstyle} lowercases the title and typesets it
small caps. 

The default skips before and after titles are rubber lengths but this does
not bode well if we are trying to line something up with a particular line
of text --- the presence of section titles may make slight vertical 
adjustments to the text lines because of the flexible spacing. So, we have
to try and have fixed spacings.
A single blank line is used before (\cmd{\setbeforesecskip)}
and after (\cmd{\setaftersecskip}) the title text. 

    A subsection title, which is the lowest subdivision in the book, is in
the italic form of the textfont and is typeset as a numbered non-indented
paragraph\index{paragraph}. The code for this is below.

\begin{lcode}
\setsubsecheadstyle{\sethangfrom{\noindent ##1}\raggedright\itshape}
  \setbeforesubsecskip{-\onelineskip}
  \setaftersubsecskip{\onelineskip}
\end{lcode}

    As in the redefinition of the \cmd{\section} style, there are fixed
spaces before and after the title text. The title is typeset 
(\cmd{\setsubsecheadstyle}) raggedright in a normal sized italic font.
The macro \cmd{\sethangfrom} is used to to redefine the internal
\cmd{\@hangfrom} macro so that the title and number are typeset as a block 
paragraph\index{paragraph!block} instead of the default hanging 
paragraph\index{paragraph!hanging} style. Note the use of
the double \verb?##? mark for denoting the position of the argument to 
\cmd{\@hangfrom}.

\section{Specifying the pagestyle}

    The pagestyle is perhaps the most interesting aspect of the exercise.
Instead of the chapter and section titles being put at the top of the
pages they are put in the margin\index{margin} starting about seven lines 
below the top of the typeblock\index{typeblock}. The folios\index{folio} 
are put at the bottom of the page aligned with the outside of the 
typeblock\index{typeblock}.

    As the folios\index{folio} are easy, we'll deal with those first.
\begin{lcode}
%% Bringhurst page style
\makepagestyle{bringhurst}
\makeevenfoot{bringhurst}{\thepage}{}{}
\makeoddfoot{bringhurst}{}{}{\thepage}
\end{lcode}

    Putting text at a fixed point on a page is typically done by
first putting the text into a zero width picture (which as far as LaTeX
is concerned takes up zero space) and then placing the picture at the
required point on the page. This can be done by hanging it from the
header\index{header}.

    We might as well treat the titles so that they will align with any 
marginal\index{marginalia} notes, which are \lnc{\marginparsep} (17pt) 
into the margin\index{margin} 
and \lnc{\marginparwidth} (51pt) wide. Earlier in the manual I defined
two lengths called \lnc{\pwlayi} and \lnc{\pwlayii} which are no longer used.
I will use these as scratch lengths in 
performing some of the necessary calculations.

    For the recto page headers\index{header} the picture will be 
the \meta{right} part of the header\index{header} and for the verso pages 
the picture will be the \meta{left}
part of the header\index{header}, all other parts being empty. 

    For the picture on the \meta{right} the text must be 17pt to
the right of the origin, and some distance below the origin.
From some experiments, this distance turns out to be the \lnc{\headsep}
plus the \cmd{\topskip} plus 7.3 lines, which is calculated as follows:
\begin{lcode}
\setlength{\pwlayi}{\headsep}
\addtolength{\pwlayi}{\topskip}
\addtolength{\pwlayi}{7.3\onelineskip}
\end{lcode}

    There is a nifty internal LaTeX macro called \cmd{\strip@pt} which you
probably haven't heard about, and I have only recently come across. What it
does is strip the `pt' from a following length, reducing it to a plain 
real number. Remembering that the default \lnc{\unitlength} is 1pt we can
do the following, while making sure that the current \lnc{\unitlength}
\emph{is} 1pt:
\begin{lcode}
\makeatletter
\newcommand{\bringpicr}[1]{%
  \setlength{\unitlength}{1pt}
  \begin{picture}(0,0)
    \put(\strip@pt\marginparsep, -\strip@pt\pwlayi){%
      \begin{minipage}[t]{\marginparwidth}
        \raggedright\itshape #1
      \end{minipage}}
  \end{picture}
}
\makeatother
\end{lcode}
The new macro \cmd{\bringpicr}\marg{text} puts \meta{text} 
into a \Ie{minipage} of width \lnc{\marginparwidth}, 
typeset raggedright in an italic font, and puts the top
left of the minipage at the position (\lnc{\marginparsep}, -\lnc{\pwlayi}) 
in a zero width picture.

    We need a different picture for the \meta{left} as the text needs to be
typeset raggedleft with the right of the text 17pt from the left of the
typeblock\index{typeblock}. I will use the length \lnc{\pwlayii} 
to calculate the sum of \lnc{\marginparsep}
and \lnc{\marginparwidth}. Hence:
\begin{lcode}
\makeatletter
\setlength{\pwlayii}{\marginparsep}
\addtolength{\pwlayii}{\marginparwidth}
\newcommand{\bringpicl}[1]{%
  \setlength{\unitlength}{1pt}
  \begin{picture}(0,0)
    \put(-\strip@pt\pwlayii, -\strip@pt\pwlayi){%
      \begin{minipage}[t]{\marginparwidth}
        \raggedleft\itshape #1
      \end{minipage}}
  \end{picture}
}
\makeatother
\end{lcode}
The new macro \cmd{\bringpicl}\marg{text} puts \meta{text} 
into a \Ie{minipage} of width \lnc{\marginparwidth}, 
typeset raggedleft in an italic font, and puts the top
left of the minipage at the position 
(-(\lnc{\marginparsep} + \lnc{\marginparwidth}), -\lnc{\pwlayi}) 
in a zero width picture.


    Now we can proceed with the remainder of the pagestyle specification.
The next bit puts the chapter and section titles into the \verb?\...mark? 
macros.
\begin{lcode}
\makeatletter
\makepsmarks{bringhurst}{%
  \def\chaptermark##1{\markboth{##1}{##1}}
  \def\sectionmark##1{\markright{##1}}
}
\makeatother
\end{lcode}

    Finally, specify the evenhead using \cmd{\bringpicl} with the section
title as its argument, and the oddhead using \cmd{\bringpicr} with the
chapter title as its argument.
\begin{lcode}
\makeevenhead{bringhurst}{\bringpicl{\rightmark}}{}{}
\makeoddhead{bringhurst}{}{}{\bringpicr{\leftmark}}
\end{lcode}


\section{Captions and the \prtoc}

    The captions to figures\index{figure} and tables\index{table} are set 
in a small sans font and are neither named nor numbered, and there is no 
\lof\ or \lot. Setting the caption titles in the desired font is simple:
\begin{lcode}
\captiontitlefont{\small\sffamily}
\end{lcode}

    There are two options regarding table\index{table} and figure\index{figure}
captioning: either
use the \cmd{\legend} command (which produces an anonymous unnumbered title)
instead of the \cmd{\caption} command, or
use the \cmd{\caption} command with a modified definition. Just in case
the design might change at a later date to required numbered captions, 
it's probably best to use 
a modified version of \cmd{\caption}. In this case this is simple, just give
the \cmd{\caption} command the same definition as the \cmd{\legend} command.
\begin{lcode}
\let\caption\legend
\end{lcode}

    An aside: I initially used the default caption style (block paragraph) for
the diagram on \pref{fig:bplayout}, but this looked unbalanced so now it
has the last line centered. As a float\index{float} environment, 
like any other environment, forms a group, you can make local changes within 
the float\index{float}. I actually did it like this:
\begin{lcode}
\begin{figure}
\captiontitlefont{\small\sffamily}
\captionstyle{\centerlastline}
...
\legend{...} \label{...}
\end{figure} 
\end{lcode}
For fine typesetting you may wish to change the style of particular captions.
The default style for a single line caption works well, but for a caption with
two or three lines either the \texttt{centering} or \texttt{centerlastline}
style might look better. A very long caption is again probably best done in 
a block paragraph style.

    Only chapter titles are included in the \toc. To specify this we
use the \cmd{\settocdepth} command.
\begin{lcode}
\settocdepth{chapter}
\end{lcode}

    The \toc\ is typeset raggedright with no leaders and the page numbers 
coming immediately after the chapter title. This is specified via:
\begin{lcode}
\renewcommand{\cftchapterfont}{\normalfont}
\renewcommand{\cftchapterpagefont}{\normalfont}
\renewcommand{\cftchapterpresnum}{\bfseries}
\renewcommand{\cftchapterleader}{}
\renewcommand{\cftchapterafterpnum}{\cftparfillskip}
\end{lcode}

\section{Preamble or package?}


    When making changes to the document style, or just defining a new
macro or two, there is the question of where to put the changes --- in
the preamble\index{preamble} of the particular document or into a separate 
package\index{package}?

    If the same changes/macros are likely to be used in more than one
document then I suggest that they be put into a package. 
If just for the single document then the choice remains open.

    I have presented the code in this chapter as though it would be  put
into the preamble\index{preamble}, hence the use of \cmd{\makeatletter} and 
\cmd{\makeatother} to surround macros that include the 
\texttt{@}\idxatincode\ character (\seeatincode). The
code could just as easily be put into a package\index{package} called, say, 
\Lpack{bringhurst}. That is, by putting all the code, except for the
\cmd{\makeatletter} and \cmd{\makeatother} commands, into a file called
\file{bringhurst.sty}. It is a good idea also to end the code in the file
with \cmd{\endinput}; LaTeX stops reading the file at that point and 
will ignore any possible garbage after \cmd{\endinput}.

    You then use the \Lpack{bringhurst} package just like any other by
putting
\begin{lcode}
\usepackage{bringhurst}
\end{lcode}
in your document's preamble\index{preamble}.

%#% extend
