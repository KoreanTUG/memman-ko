\chapter{Poetry} \label{chap:verse}


    The typesetting of a poem should ideally be dependent on the
particular poem. Individual problems do not usually admit of a
general solution, so this manual and code should be used more
as a guide towards some solutions rather than providing a ready
made solution for any particular piece of verse.

    The doggerel used as illustrative material has been taken 
from~\cite{RUMOUR}.

    Note that for the examples in this section I have made no attempt
to do other than use the minimal \alltx\ capabilities; in particular
I have made no attempt to do any special page breaking so some stanzas
may cross onto the next page --- most undesireable for publication.

\index{verse|(}
\index{verse!typesetting environments|(}

    The standard \ltx\ classes provide the \Ie{verse} environment which 
is defined as a particular kind of list. Within the environment you 
use \cmd{\\}\index{verse!end a line} to end a line, 
and a blank line will end\index{verse!end a stanza}\index{stanza!end} a 
stanza. For example, here is the text of a single stanza poem:
\begin{lcode}
\newcommand{\garden}{
I used to love my garden \\
But now my love is dead \\
For I found a bachelor's button \\
In black-eyed Susan's bed.
}
\end{lcode}
\newcommand{\garden}{
I used to love my garden \\
But now my love is dead \\
For I found a bachelor's button \\
In black-eyed Susan's bed.
}
When this is typeset as a normal \ltx\ paragraph\index{paragraph} 
(with no paragraph indentation), i.e.,
\begin{lcode}
\noident\garden
\end{lcode}
it looks like: \\[\baselineskip]
\garden{}

\vspace*{\onelineskip}

   Typesetting it within the \Ie{verse} environment produces:% \\[\baselineskip]

\begin{verse}\index[lines]{I used to love my garden}  
\garden
\end{verse}

%\ablankline

The stanza could also be typeset within the \Ie{alltt} environment, defined
in the standard \Lpack{alltt} package~\cite{ALLTT}, 
using a normal font and no \cmd{\\} line endings.
\begin{lcode}
\begin{alltt}\normalfont
I used to love my garden 
But now my love is dead 
For I found a bachelor's button 
In black-eyed Susan's bed.
\end{alltt}
\end{lcode}
which produces:

\begin{alltt}\normalfont
I used to love my garden 
But now my love is dead 
For I found a bachelor's button 
In black-eyed Susan's bed.
\end{alltt}

The \Ie{alltt} environment is like the \Ie{verbatim} environment except that
you can use LaTeX macros inside it. 
   In the \Ie{verse} environment long lines\index{verse!long lines}  
will be wrapped and indented
but in the \Ie{alltt} environment there is no indentation. 

Some stanzas have certain lines\index{verse!indent line} indented, 
often alternate ones. To
typeset stanzas like this you have to add your own spacing. For
instance:
\begin{lcode}
\begin{verse}
There was an old party of Lyme \\
Who married three wives at one time. \\
\hspace{2em} When asked: `Why the third?' \\
\hspace{2em} He replied: `One's absurd, \\
And bigamy, sir, is a crime.'
\end{verse}
\end{lcode}
is typeset as: 
\begin{verse}\index[lines]{There was an old party of Lyme}
There was an old party of Lyme \\
Who married three wives at one time. \\
\hspace{2em} When asked: `Why the third?' \\
\hspace{2em} He replied: `One's absurd, \\
And bigamy, sir, is a crime.'
\end{verse}
\vspace{\onelineskip}

Using the \Ie{alltt} environment you can put in the spacing via ordinary
spaces. That is, this:
\begin{lcode}
\begin{alltt}\normalfont
There was an old party of Lyme
Who married three wives at one time.
      When asked: `Why the third?' 
      He replied: `One's absurd, 
And bigamy, sir, is a crime.'
\end{alltt}
\end{lcode}
is typeset as

\begin{alltt}
\normalfont
There was an old party of Lyme
Who married three wives at one time.
      When asked: `Why the third?' 
      He replied: `One's absurd, 
And bigamy, sir, is a crime.'
\end{alltt}

More exotically you could use the TeX \cmd{\parshape} 
command\footnote{See the \btitle{\tx book} for how to use this.}:
\begin{lcode}
\parshape = 5 0pt \linewidth 0pt \linewidth 
              2em \linewidth 2em \linewidth 0pt \linewidth
\noindent There was an old party of Lyme \\
Who married three wives at one time. \\
When asked: `Why the third?' \\
He replied: `One's absurd, \\
And bigamy, sir, is a crime.' \par
\end{lcode}
which will be typeset as:

\vspace*{\onelineskip}

\parshape = 5 0pt \linewidth 0pt \linewidth 
              2em \linewidth 2em \linewidth 0pt \linewidth
\noindent There was an old party of Lyme \\
Who married three wives at one time. \\
When asked: `Why the third?' \\
He replied: `One's absurd, \\
And bigamy, sir, is a crime.' \par


\vspace*{\onelineskip}

   This is about as much assistance as standard \alltx\ provides, except
to note that in the \Ie{verse} environment the \cmd{\\*} version of \cmd{\\}
will prevent\index{verse!prevent page break} a following page break. 
You can also make judicious use
of the \cmd{\needspace} macro to keep things together.

\index{verse!typesetting environments|)}

   Some books of poetry, and especially anthologies, have two or more
indexes\index{index}\index{verse!multiple indexes}, one, say for the 
poem titles and another for the 
first lines, and maybe even a third for the poets' names. 
If you are not using \Lclass{memoir} then 
the \Lpack{index}~\cite{INDEX} and \Lpack{multind}~\cite{MULTIND}
packages provide support for multiple indexes\index{index!multiple} 
in one document.

%\clearpage
\section{Classy verse} 

    The code provided by the \Lclass{memoir} class is meant to help
with some aspects of typesetting poetry but does not, and cannot,
provide a comprehensive solution to all the requirements that
will arise.

    The main aspects of typesetting poetry that differ from typesetting
plain text are:
\begin{itemize}
\item Poems are usually visually centered\index{verse!centering} on the page.
\item Some lines\index{verse!indent line} are indented, and often there 
      is a pattern\index{verse!indent pattern} to the indentation.
\item When a line is too wide\index{verse!long line} for the page it is 
      broken and the
      remaining portion indented with respect to the original start
      of the line.
\end{itemize}
These are the ones that the class attempts to deal with.

\begin{syntax}
\senv{verse}\oarg{length} ... \eenv{verse} \\
\lnc{\versewidth} \\
\end{syntax}
\glossary(verse)%
  {\senv{verse}\oarg{length}}%
  {Environment for typesetting verse; if given the midpoint of \meta{length}
   is placed at the center of the typeblock measure.}
\glossary(versewidth)%
  {\cs{versewidth}}%
  {Scratch length, typically for use in verse typesetting.}
The \Ie{verse} environment provided by the class is an extension
of the usual LaTeX environment. The environment takes one optional
parameter, which is a length; for example \verb?\begin{verse}[4em]?.
You may have noticed that the earlier verse examples are all
near the left margin\index{margin!left}, whereas verses usually look better 
if they
are typeset about the center of the page. The length parameter,
if given, should be about the length of an average line, and then
the entire contents\index{verse!centering} will be typeset with the mid 
point of the length centered horizontally on the page.

The length \lnc{\versewidth} is provided as a convenience. It may be used,
for example, to calculate the length of a line of text for use
as the optional argument to the \Ie{verse} environment: 
\begin{lcode}
\settowidth{\versewidth}{This is the average line,}
\begin{verse}[\versewidth]
\end{lcode}

\begin{syntax}
\lnc{\vleftmargin} \\
\end{syntax}
\glossary(vleftmargin)%
  {\cs{vleftmargin}}%
  {General verse indent from the left of the typeblock.}
In the basic \ltx\ \Ie{verse} environment the body of the verse is indented
from the left of the typeblock by an amount \lnc{\leftmargini}, as is the
text in many other environments based on the basic \ltx\ \Ie{list}
environment. For the class's version of \Ie{verse} the default indent
is set by the length \lnc{\vleftmargin} (which is initially set to
\lnc{leftmargini}). For poems with particularly long 
lines\index{verse!long line} it could perhaps be
advantageous to eliminate any general indentation by:
\begin{lcode}
\setlength{\vleftmargin}{0em}
\end{lcode}
If necessary the poem could even be moved into the left margin by giving
\lnc{\vleftmargin} a negative length value, such as -1.5em.

\begin{syntax}
\lnc{\stanzaskip} \\
\end{syntax}
\glossary(stanzaskip)%
  {\cs{stanzaskip}}%
  {Vertical space between verse stanzas.}
    The vertical space\index{stanza!space} between stanzas is the 
length \lnc{\stanzaskip}. It can be changed by the usual methods.

\begin{syntax}
\cmd{\vin} \\
\lnc{\vgap} \\
\lnc{\vindent} \\
\end{syntax}
The command \cmd{\vin} is shorthand for \verb?\hspace*{\vgap}? for use
at the start of an indented\index{verse!indent space} line of verse. 
The length \lnc{\vgap}
(initially 1.5em) can be changed by \cmd{\setlength} or \cmd{\addtolength}.
When a verse line\index{verse!long line} is too long to fit within the 
typeblock\index{typeblock} it is wrapped\index{verse!wrapped line indent} 
to the next line with an initial indent given by the value of the length
\lnc{vindent}. Its initial value is twice the default value of \lnc{\vgap}.

\begin{syntax}
\cmd{\\}\oarg{length} \\
\cmd{\\*}\oarg{length} \\
\cmdprint{\\!}\oarg{length} \\
\end{syntax}
\glossary(\\)%
  {\Vprint{\\}\oarg{length}}%
  {Ends a verse line, and adds \meta{length} vertical space.}
\glossary(\\*)%
  {\Vprint{\\*}\oarg{length}}%
  {Ends a line while preventing a pagebreak, and adds \meta{length} vertical space.}
\glossary(\\!)
  {\Vprint{\\!}\oarg{length}}%
  {Ends a verse stanza, and adds \meta{length} vertical space.}

Each line in the \Ie{verse} environment, except possibly for the last line
in a stanza\index{stanza!last line}, 
must be ended by \cmd{\\}, which comes in several variants. In each variant
the optional \meta{length} is the vertical space to be left before the next
line. The \cmd{\\*} form prohibits a page 
break\index{stanza!prevent page break} after the line. 
The \pixslashbang\ form is to be used only for the last 
line\index{stanza!last line!numbering} in a stanza
when the lines are being numbered; this is because the line 
numbers\index{line number} are incremented by the \cmd{\\} macro. 
It would normally be followed by a blank line.

\begin{syntax}
\cmd{\verselinebreak}\oarg{length} \\
\cmdprint{\\>}\oarg{length} \\
\end{syntax}
\glossary(verselinebreak)
  {\cs{verselinebreak}\oarg{length}}%
  {Makes a break in a verse line, indenting the next part by twice \cs{vgap},
   or by \meta{length} if it is given.}
\glossary(\\>)
  {\Vprint{\\>}\oarg{length}}%
  {Shorthand for \cs{verselinebreak}.}
Using \cmd{\verselinebreak} will cause later\index{stanza!line break}
text in the line to be typeset 
indented\index{stanza!line break!indent} on the following line. 
If the optional \meta{length} is not given
the indentation is twice \lnc{\vgap}, otherwise it is \meta{length}.
The broken line will count as a single line as far as the \Ie{altverse}
and \Ie{patverse} environments are concerned. The macro \cmd{\\>} is
shorthand for \cmd{\verselinebreak}, and unlike other members of the \cmd{\\}
family the optional \meta{length} is the indentation of the following 
partial\index{stanza!line break!indent} line, not a vertical skip. 
Also, the \cmd{\\>} macro does not increment any 
line number\index{line number}.

\begin{syntax}
\cmd{\vinphantom}\marg{text} \\
\end{syntax}
\glossary(vinphantom)%
  {\cs{vinphantom}\marg{text}}%
  {Leaves a space as wide as \meta{text}.}
Verse lines are sometimes indented according to the space taken by
the text on the previous line. The macro \cmd{\vinphantom} can be used
at the start of a line\index{verse!indent space} to give an indentation 
as though the
line started with \meta{text}. For example here are a few lines from
the portion of \textit{Fridthjof's Saga} where Fridthjof and Ingeborg part:
\begin{egsource}{eg:fridthjof}
\settowidth{\versewidth}{Nay, nay, I leave thee not, 
                         thou goest too}
\begin{verse}[\versewidth]
\ldots \\* 
His judgement rendered, he dissolved the Thing. \\*
\flagverse{Ingeborg} And your decision? \\*
\flagverse{Fridthjof} \vinphantom{And your decision?} 
                      Have I ought to choose? \\*
Is not mine honour bound by his decree? \\*
And that I will redeem through Angantyr \\*
His paltry gold doth hide in Nastrand's flood. \\*
Today will I depart. \\*
\flagverse{Ingeborg} \vinphantom{Today will I depart.} 
                     And Ingeborg leave? \\*
\flagverse{Fridthjof} Nay, nay, I leave thee not, 
                      thou goest too. \\*
\flagverse{Ingeborg} Impossible! \\*
\flagverse{Fridthjof} \vinphantom{Impossible!} 
                      O! hear me, ere thou answerest.
\end{verse}
\end{egsource}

\begin{egresult}[Phantom text in verse]{eg:fridthjof}%
\settowidth{\versewidth}{Nay, nay, I leave thee not, 
                         thou goest too}%
\begin{verse}[\versewidth]
\ldots\index[lines]{His judgement rendered, he dissolved the Thing} \\*
His judgement rendered, he dissolved the Thing. \\*
\flagverse{Ingeborg} And your decision? \\*
\flagverse{Fridthjof} \vinphantom{And your decision?} 
                      Have I ought to choose? \\*
Is not mine honour bound by his decree? \\*
And that I will redeem through Angantyr \\*
His paltry gold doth hide in Nastrand's flood. \\*
Today will I depart. \\*
\flagverse{Ingeborg} \vinphantom{Today will I depart.} 
                      And Ingeborg leave? \\*
\flagverse{Fridthjof} Nay, nay, I leave thee not, 
                      thou goest too. \\*
\flagverse{Ingeborg} Impossible! \\*
\flagverse{Fridthjof} \vinphantom{Impossible!} 
                       O! hear me, ere thou answerest.
\end{verse}
\end{egresult}

    Use of \cmd{\vinphantom} is not restricted to the start of verse lines ---
it may be used anywhere in text to leave some some 
blank\index{blank space} space.
For example, compare the two lines below, which are produced by this code:
\begin{lcode}
   \noindent Come away with me and be my love --- Impossible. \\
Come away with me \vinphantom{and be my love} --- Impossible.
\end{lcode}
\noindent Come away with me and be my love --- Impossible. \\
          Come away with me \vinphantom{and be my love} --- Impossible.

\begin{syntax}
\cmd{\vleftofline}\marg{text} \\
\end{syntax}
\glossary(vleftofline)%
  {\cs{vleftofline}\marg{text}}%
  {`Hanging left' \meta{text} at the start of a verse line.}
A verse line may start with something, for example open quote marks, 
where it is desireable that it is ignored as far as the alignment of the
rest of the line is concerned\footnote{Requested by Matthew 
Ford\index{Ford, Matthew} who also provided the example text.} --- a sort
of `hanging left punctuation'. When it is put at the start of a line
in the \Ie{verse} environment the \meta{text} is typeset but ignored as 
far as horizontal indentation is concerned.
Compare the two examples.

\begin{egsource}{eg.joel1}
\noindent ``No, this is what was spoken by the prophet Joel:
\begin{verse}
``\,`\,``In the last days,'' God says, \\
``I will pour out my spirit on all people. \\
Your sons and daughters will prophesy, \\
\ldots \\
And everyone who calls \ldots ''\,'
\end{verse}
\end{egsource}

\begin{egresult}[Verse with regular quote marks]{eg.joel1}
\noindent ``No, this is what was spoken by the prophet Joel:
\begin{verse}
``\,`\,``In the last days,'' God says, \\
``I will pour out my spirit on all people. \\
Your sons and daughters will prophesy, \\
\ldots \\
And everyone who calls \ldots ''\,'
\end{verse}
\end{egresult}

\begin{egsource}{eg.joel2}
\noindent ``No, this is what was spoken by the prophet Joel:
\begin{verse}
\vleftofline{``\,`\,``}In the last days,'' God says, \\
\vleftofline{``}I will pour out my spirit on all people. \\
Your sons and daughters will prophesy, \\
\ldots \\
And everyone who calls \ldots ''\,'
\end{verse}
\end{egsource}

\begin{egresult}[Verse with hanging left quote marks]{eg.joel2}
\noindent ``No, this is what was spoken by the prophet Joel:
\begin{verse}
\vleftofline{``\,`\,``}In the last days,'' God says, \\
\vleftofline{``}I will pour out my spirit on all people. \\
Your sons and daughters will prophesy, \\
\ldots \\
And everyone who calls \ldots ''\,'
\end{verse}
\end{egresult}



\subsection{Indented lines}

Within the \Ie{verse} environment stanzas are normally separated by a 
blank line in the input. 

\begin{syntax}
\senv{altverse} ... \eenv{altverse} \\
\end{syntax}
\glossary(altverse)%
  {\senv{altverse}}%
  {Alternate lines in the stanza are indented.}
Individual stanzas within \Ie{verse} may, however, 
be enclosed in the \Ie{altverse}\index{stanza!indent alternate lines} 
environment. This has the effect of
indenting the 2nd, 4th, etc., lines of the stanza by the length \lnc{\vgap}.

\begin{syntax}
\senv{patverse} ... \eenv{patverse} \\
\senv{patverse*} ... \eenv{patverse*} \\
\cmd{\indentpattern}\marg{digits} \\
\end{syntax}
\glossary(patverse)%
  {\senv{patverse}}%
  {Stanza lines are indented according to the \cs{indentpattern};
   lines after the pattern is completed are not indented.}
\glossary(patverse*)%
  {\senv{patverse*}}%
  {Like \Pe{patverse} except that the pattern will keep repeating.}
\glossary(indentpattern)%
  {\cs{indentpattern}\marg{digits}}%
  {Stanza lines in \Pe{patverse} environment are indented according to
   the \meta{digits} pattern.}
As an alternative to the \Ie{altverse} environment, 
individual stanzas within the \Ie{verse} environment may be enclosed
in the \Ie{patverse} environment. Within this environment the indentation
of each line is specified by an indentation\index{stanza!indent pattern} 
pattern, which consists
of an array of digits, \(d_{1}\) to \(d_{n}\), and the \(n\)th line is
indented by \(d_{n}\) times \lnc{\vgap}. However, the first line is
not indented, irrespective of the value of \(d_{1}\).

    The indentation pattern for a \Ie{patverse} or \Ie{patverse*}
environment is specified
via the \cmd{\indentpattern} command, where \meta{digits} is a string
of digits (e.g., \texttt{3213245281}). With the \Ie{patverse}
environment, if the pattern is
shorter than the number of lines in the stanza, the trailing lines will
not be indented. However, in the \Ie{patverse*} environment the pattern
keeps repeating until the end of the stanza.

\subsection{Numbering}

\begin{syntax}
\cmd{\flagverse}\marg{flag} \\
\lnc{\vleftskip} \\
\end{syntax}
\glossary(flagverse)%
  {\cs{flagverse}\marg{flag}}%
  {Used at the start of a verse line to put \meta{flag} distance
  \cs{vleftskip} before the start of the line.}
\glossary(vleftskip)%
  {\cs{vleftskip}}%
  {Space between the argument to \cs{flageverse} and \cs{flagverse}.}
Putting \cmd{\flagverse} at the start of a line will typeset \meta{flag},
for example the stanza\index{stanza!number} number,
ending at a distance \lnc{\vleftskip} before the line. The default for
\lnc{\vleftskip} is 3em.

The lines in a poem may be numbered.

\begin{syntax}
\cmd{\linenumberfrequency}\marg{nth} \\
\cmd{\setverselinenums}\marg{first}\marg{startat} \\
\end{syntax}
\glossary(setverselinenums)%
  {\cs{setverselinenums}\marg{first}\marg{startat}}%
  {The first line of the following verse is number \marg{first} and the
   first printed line number should be \meta{startat}.}
The declaration \cmd{\linenumberfrequency}\marg{nth} will cause every 
\meta{nth} line\index{line number!frequency} 
of succeeding verses to be numbered. 
For example, 
\verb?\linenumberfrequency{5}?
will number every fifth line. The default is \verb?\linenumberfrequency{0}? 
which prevents any numbering. The \cmd{\setverselinenums} macro can be
used to specify that the number of the first line of the following \Ie{verse}
shall be \meta{first} and the first printed number shall be \meta{startat}.
For example, perhaps you are quoting part of a numbered poem. The original
numbers every tenth line but if your extract starts with line 7, then
\begin{lcode}
\linenumberfrequency{10}
\setverselinenums{7}{10}
\end{lcode}
is what you will need.

\begin{syntax}
\cmd{\thepoemline} \\
\cmd{\linenumberfont}\marg{font-decl} \\
\end{syntax}
\glossary(thepoemline)%
  {\cs{thepoemline}}%
  {The numeric representation of verse line numbers (default arabic).}
    The \Icn{poemline} counter is used in numbering the lines, so the
number representation is \cmd{\thepoemline}, which defaults to 
arabic numerals, and they are typeset using the font 
specified\index{line number!font}
via \cmd{\linenumberfont}; the default is 
\begin{lcode}
\linenumberfont{\small\rmfamily}
\end{lcode}
for small numbers in the roman font. 

\begin{syntax}
\cmd{\verselinenumbersright} \\
\cmd{\verselinenumbersleft} \\
\lnc{\vrightskip} \\
\end{syntax}
\glossary(verselinenumbersright)%
  {\cs{verselinenumbersright}}%
  {Following this declaration verse line numbers are set at the right of the
   verse lines.}
\glossary(verselinenumbersleft)%
  {\cs{verselinenumbersleft}}%
  {Following this declaration verse line numbers are set at the left of the
   verse lines.}
\glossary(vrightskip)
  {\cs{vrightskip}}%
  {Verse line numbers are set distance \cs{vrightskip} into the margin.}
Following the declaration
\cmd{\verselinenumbersright}, which is the default, any verse line numbers
will be set\index{line number!position} in the righthand margin. 
The \cmd{\verselinenumbersleft}
declaration will set any subsequent line numbers to the left of the lines.
The numbers are set at a distance
\lnc{\vrightskip} (default 1em) into the margin. 


\section{Titles}

    The \cmd{\PoemTitle} command is provided for typesetting titles
of poems.

\begin{syntax}
\cmd{\PoemTitle}\oarg{fortoc}\oarg{forhead}\marg{title} \\
\cmd{\NumberPoemTitle} \\
\cmd{\PlainPoemTitle} \\
\cmd{\thepoem} \\
\end{syntax}
\glossary(PoemTitle)%
  {\cs{PoemTitle}\oarg{fortoc}\oarg{forhead}\marg{title}}%
  {Typesets the title for a poem and puts it into the ToC.}
\glossary(NumberPoemTitle)%
  {\cs{NumberPoemTitle}}%
  {Declaration for \cs{PoemTitle} to be numbered.}
\glossary(PlainPoemTitle)%
  {\cs{PlainPoemTitle}}%
  {Declaration for \cs{PoemTitle} to be unnumbered.}
\glossary(thepoem)%
  {\cs{thepoem}}%
  {Typeset the current Poem Title number}%
The \cmd{\PoemTitle} command takes the same arguments as the 
\cmd{\chapter} command; it typesets the title for a poem\index{poem title} 
and adds it to the ToC\index{poem title!in ToC}. 
Following the declaration \cmd{\NumberPoemTitle}
the title is numbered but there is no numbering after the
\cmd{\PlainPoemTitle} declaration. 

\begin{syntax}
\cmd{\poemtoc}\marg{sec} \\
\end{syntax}
\glossary(poemtoc)%
  {\cs{poemtoc}}%
  {Kind of entry for a \cs{PoemTitle} in the ToC.}
The kind of entry made in the \toc\ by\index{poem title!in ToC} 
\cmd{\PoemTitle} is defined by \cmd{\poemtoc}. The initial definition is: 
\begin{lcode}
\newcommand{\poemtoc}{section}
\end{lcode}
for a section-like \toc\ entry. This can be changed to, say, \texttt{chapter}
or \texttt{subsection} or \ldots.

\begin{syntax}
\cmd{\poemtitlemark}\marg{forhead} \\
\cmd{\poemtitlepstyle} \\
\end{syntax}
\glossary(poemtitlemark)%
  {\cs{poemtitlemark}\marg{forhead}}%
  {Used to set marks for a \cs{PoemTitle}.}
\glossary(poemtitlepstyle)%
  {\cs{poemtitlepstyle}}%
  {Page style for a \cs{PoemTitle}.}
    The macro \cmd{\poemtitlemark}
is called with the argument \meta{forhead} so that it may be used
to set marks for use in a page header via the normal mark process. 
The \cmd{\poemtitlepstyle}
macro, which by default does nothing, is provided as a hook so that,
for example, it can be redefined to specify a particular pagestyle that should
be used. For example:
\begin{lcode}
\renewcommand*{\poemtitlemark}[1]{\markboth{#1}{#1}}
\renewcommand*{\poemtitlepstyle}{%
  \pagestyle{headings}%
  \thispagestyle{empty}}
\end{lcode}

\begin{syntax}
\cmd{\PoemTitle*}\oarg{forhead}\marg{title} \\
\cmd{\poemtitlestarmark}\marg{forhead} \\
\cmd{\poemtitlestarpstyle} \\
\end{syntax}
\glossary(PoemTitle*)%
  {\cs{PoemTitle*}\oarg{fortoc}\oarg{forhead}\marg{title}}%
  {Typesets an unnumbered title for a poem but does not add it to the ToC.}
\glossary(poemtitlestarmark)%
  {\cs{poemtitlestarmark}\marg{forhead}}%
  {Used to set marks for a \cs{PoemTitle*}.}
\glossary(poemtitlestarpstyle)%
  {\cs{poemtitlestarpstyle}}%
  {Page style for a \cs{PoemTitle*}.}

    The \cmd{\PoemTitle*} command produces an unnumbered title that is
not added to the ToC. Apart from that it operates in the same manner
as the unstarred version. The \cmd{\poemtitlestarmark} and 
\cmd{\poemtitlestarpstyle} can be redefined to set marks and pagestyles.

\subsection{Main Poem Title layout parameters}

\begin{syntax}
\cmd{\PoemTitleheadstart} \\
\cmd{\printPoemTitlenonum} \\
\cmd{\printPoemTitlenum} \\
\cmd{\afterPoemTitlenum} \\
\cmd{\printPoemTitletitle}\marg{title} \\
\cmd{\afterPoemTitle} \\
\end{syntax}
\glossary(PoemTitleheadstart)%
  {\cs{PoemTitleheadstart}}%
  {Called at the start of typesetting a \cs{PoemTitle}.}
\glossary(printPoemTitlenum)%
  {\cs{printPoemTitlenum}}%
  {Typesets the number for a \cs{PoemTitle}.}
\glossary(printPoemTitlenonum)%
  {\cs{printPoemTitlenonum}}%
  {Used instead of \cs{printPoemTitlenum} for an unnumbered \cs{PoemTitle}.}
\glossary(afterPoemTitlenum)%
  {\cs{afterPoemTitlenum}}%
  {Called after printing the number of a \cs{PoemTitle}.}
\glossary(printPoemTitletitle)%
  {\cs{printPoemTitletitle}\marg{title}}%
  {Typesets the title of a \cs{PoemTitle}.}
\glossary(afterPoemTitle)%
  {\cs{afterPoemTitle}}%
  {Called after printing the title of a \cs{PoemTitle}.}


    The essence of the code used to typeset a numbered \meta{title} from
a \cmd{\PoemTitle} is:
\begin{lcode}
\PoemTitleheadstart
\printPoemTitlenum
\afterPoemTitlenum
\printPoemTitletitle{title}
\afterPoemTitle
\end{lcode}
If the title is unnumbered then \cmd{\printPoemTitlenonum} is used
instead of the \cmd{\printPoemTitlenum} and 
\cmd{\afterPoemTitlenum} pair of macros.

    The various elements of this can be modified to change the layout.
By default the number is centered above the title, which is also typeset 
centered, and all in a \cmd{\large} font.

    The elements are detailed in the next section.

\subsection{Detailed Poem Title layout parameters}

\begin{syntax}
\lnc{\beforePoemTitleskip} \\
\cmd{\PoemTitlenumfont} \\
\lnc{\midPoemTitleskip} \\
\cmd{\PoemTitlefont} \\
\lnc{\afterPoemTitleskip} \\
\end{syntax}
\glossary(beforePoemTitleskip)%
  {\cs{beforePoemTitleskip}}%
  {Vertical space before a poem title.}
\glossary(midPoemTitleskip)%
  {\cs{midPoemTitleskip}}%
  {Vertical space between the number and text of a poem title.}
\glossary(afterPoemTitleskip)%
  {\cs{afterPoemTitleskip}}%
  {Vertical space after a poem title}
\glossary(PoemTitlenumfont)%
  {\cs{PoemTitlenumfont}}%
  {Font for the number of a poem title}
\glossary(PoemTitlefont)%
  {\cs{PoemTitlefont}}%
  {Font for the text of a poem title}


As defined, \cmd{\PoemTitleheadstart} inserts vertical space 
before a poem title. The default definition is:
\begin{lcode}
\newcommand*{\PoemTitleheadstart}{\vspace{\beforePoemTitleskip}}
\newlength{\beforePoemTitleskip}
  \setlength{\beforePoemTitleskip}{1\onelineskip}
\end{lcode}

\cmd{\printPoemTitlenum} typesets the number for a poem title.
The default definition, below, prints the number centered and in
a large font.
\begin{lcode}
\newcommand*{\printPoemTitlenum}{\PoemTitlenumfont \thepoem}
\newcommand*{\PoemTitlenumfont}{\normalfont\large\centering}
\end{lcode}

The definition of \cmd{\printPoemTitlenonum}, which is used
when there is no number, is simply
\begin{lcode}
\newcommand*{\printPoemTitlenonum}{}
\end{lcode}

\cmd{\afterPoemTitlenum} is called between setting the number 
and the title. It ends a paragraph (thus making sure any previous
\cmd{\centering} is used) and then may add some vertical
space. The default definition is:
\begin{lcode}
\newcommand*{\afterPoemTitlenum}{\par\nobreak\vskip \midPoemTitleskip}
\newlength{\midPoemTitleskip}
  \setlength{\midPoemTitleskip}{0pt}
\end{lcode}

The default definition of \cmd{\printPoemTitletitle} is below.
It typesets the title centered and in a large font.
\begin{lcode}
\newcommand*{\printPoemTitletitle}[1]{\PoemTitlefont #1}
\newcommand*{\PoemTitlefont}{\normalfont\large\centering}
\end{lcode}

The macro \cmd{\afterPoemTitle} finishes off the title 
typesetting. The default definition is:
\begin{lcode}
\newcommand*{\afterPoemTitle}{\par\nobreak\vskip \afterPoemTitleskip}
\newlength{\afterPoemTitleskip}
  \setlength{\afterPoemTitleskip}{1\onelineskip}
\end{lcode}



%\clearpage
\section{Examples}

   Here are some sample verses using the class facilities.

First a Limerick, but titled\index{poem title} and centered:
\begin{lcode}
\renewcommand{\poemtoc}{subsection}
\PlainPoemTitle
\PoemTitle{A Limerick}
\settowidth{\versewidth}{There was a young man of Quebec}
\begin{verse}[\versewidth]
There was a young man of Quebec \\
Who was frozen in snow to his neck. \\
\vin When asked: `Are you friz?' \\
\vin He replied: `Yes, I is, \\
But we don't call this cold in Quebec.'
\end{verse}
\end{lcode}
which gets typeset as below. The \cmd{\poemtoc} is redefined
to \texttt{subsection} so that the \cmd{\poemtitle} titles
are entered\index{poem title!in ToC} into the \toc\ as 
subsections. The titles will not be numbered because of the
\cmd{\PlainPoemTitle} declaration.

\renewcommand{\poemtoc}{subsection}
\PlainPoemTitle
\PoemTitle{A Limerick}
\settowidth{\versewidth}{There was a young man of Quebec}
\begin{verse}[\versewidth] \index[lines]{There was a young man of Quebec}
There was a young man of Quebec \\
Who was frozen in snow to his neck. \\
\vin When asked: `Are you friz?' \\
\vin He replied: `Yes, I is, \\
But we don't call this cold in Quebec.'
\end{verse}

\vspace{\onelineskip}
%\ablankline

    Next is the Garden verse within the \Ie{altverse} environment. Unlike
earlier renditions this one is titled\index{poem title} and centered. 
\begin{lcode}
\settowidth{\versewidth}{But now my love is dead}
\PoemTitle{Love's lost}
\begin{verse}[\versewidth]
\begin{altverse}
\garden
\end{altverse}
\end{verse}
\end{lcode}
Note how the alternate lines\index{stanza!indent alternate lines} 
are automatically indented in the typeset result below.

\settowidth{\versewidth}{But now my love is dead}
\PoemTitle{Love's lost}
\begin{verse}[\versewidth]\index[lines]{I used to love my garden}
\begin{altverse}
\garden
\end{altverse}
\end{verse}

\vspace{\onelineskip}
% \ablankline

It is left up to you how you might want to add information about
the author\index{poem!author} of a poem. Here is one example of a 
macro for this:
\begin{lcode}
\newcommand{\attrib}[1]{%
   \nopagebreak{\raggedleft\footnotesize #1\par}}
\end{lcode}
\providecommand{\attrib}[1]{%
   \nopagebreak{\raggedleft\footnotesize #1\par}}

   This can be used as in the next bit of doggerel.
\begin{lcode}
\PoemTitle{Fleas}
\settowidth{\versewidth}{What a funny thing is a flea}
\begin{verse}[\versewidth]
What a funny thing is a flea. \\
You can't tell a he from a she. \\
But he can. And she can. \\
Whoopee!
\end{verse}
\attrib{Anonymous}
\end{lcode}

\PoemTitle{Fleas}
\settowidth{\versewidth}{What a funny thing is a flea}
\begin{verse}[\versewidth]\index[lines]{What a funny thing is a flea}
What a funny thing is a flea. \\
You can't tell a he from a she. \\
But he can. And she can. \\
Whoopee!
\end{verse}
\attrib{Anonymous}

\vspace{\onelineskip}
%\ablankline

The next example demonstrates the automatic line wrapping for 
overlong\index{stanza!long line} lines.
\begin{lcode}
\PoemTitle{In the beginning}
\settowidth{\versewidth}{And objects at rest tended to 
                         remain at rest}
\begin{verse}[\versewidth]
Then God created Newton, \\
And objects at rest tended to remain at rest, \\
And objects in motion tended to remain in motion, \\
And energy was conserved
   and momentum was conserved
   and matter was conserved \\
And God saw that it was conservative.
\end{verse}
\attrib{Possibly from \textit{Analog}, circa 1950}
\end{lcode}

%%\enlargethispage{\baselineskip}
\PoemTitle{In the beginning}
\settowidth{\versewidth}{And objects at rest tended to remain at rest}
\begin{verse}[\versewidth]\index[lines]{Then God created Newton}
Then God created Newton, \\
And objects at rest tended to remain at rest, \\
And objects in motion tended to remain in motion, \\
And energy was conserved
   and momentum was conserved
   and matter was conserved \\
And God saw that it was conservative.
\end{verse}
\attrib{Possibly from \textit{Analog}, circa 1950}

\vspace{\onelineskip}
%\ablankline

The following verse demonstrates the use of a forced\index{stanza!line break} 
linebreak; I have
used the \cmd{\\>} command instead of the more descriptive,
but discursive, \cmd{\verselinebreak}. It also
has a slightly different title\index{poem title!styling} style.
\begin{lcode}
\renewcommand{\PoemTitlefont}{%
              \normalfont\large\itshape\centering}
\poemtitle{Mathematics}
\settowidth{\versewidth}{Than Tycho Brahe, or Erra Pater:}
\begin{verse}[\versewidth]
In mathematics he was greater \\
Than Tycho Brahe, or Erra Pater: \\
For he, by geometric scale, \\
Could take the size of pots of ale;\\ 
\settowidth{\versewidth}{Resolve by}%
Resolve, by sines \\>[\versewidth] and tangents straight, \\
If bread or butter wanted weight; \\
And wisely tell what hour o' the day \\
The clock does strike, by Algebra.
\end{verse}
\attrib{Samuel Butler (1612--1680)}
\end{lcode}

%%\clearpage

\renewcommand{\PoemTitlefont}{\normalfont\large\itshape\centering}
\PoemTitle{Mathematics}
\settowidth{\versewidth}{Than Tycho Brahe, or Erra Pater:}
\begin{verse}[\versewidth]\index[lines]{In mathematics he was greater}
In mathematics he was greater \\
Than Tycho Brahe, or Erra Pater: \\
For he, by geometric scale, \\
Could take the size of pots of ale;\\ 
\settowidth{\versewidth}{Resolve by}%
Resolve, by sines \\>[\versewidth] and tangents straight, \\
If bread or butter wanted weight; \\
And wisely tell what hour o' the day \\
The clock does strike, by Algebra.
\end{verse}
\attrib{Samuel Butler (1612--1680)}

\vspace{\onelineskip}
%\ablankline
%\clearpage

Another limerick, but this time taking advantage of 
the \Ie{patverse}\index{verse!indent pattern}
environment. If you are typesetting a series of limericks 
a single \cmd{\indentpattern} will do for all of them.
\begin{lcode}
\settowidth{\versewidth}{There was a young lady of Ryde}
\indentpattern{00110}
\needspace{7\onelineskip}
\PoemTitle{The Young Lady of Ryde}
\begin{verse}[\versewidth]
\begin{patverse}
There was a young lady of Ryde \\
Who ate some apples and died. \\
The apples fermented \\
Inside the lamented \\
And made cider inside her inside.
\end{patverse}
\end{verse}
\end{lcode}
Note that I used the \cmd{\needspace} command to ensure that 
the limerick will not get broken across a page.

\settowidth{\versewidth}{There was a young lady of Ryde}
\indentpattern{00110}
\needspace{7\onelineskip}
\PoemTitle{The Young Lady of Ryde}
\begin{verse}[\versewidth]\index[lines]{There was a young lady of Ryde}
\begin{patverse}
There was a young lady of Ryde \\
Who ate some apples and died. \\
The apples fermented \\
Inside the lamented \\
And made cider inside her inside.
\end{patverse}
\end{verse}

\vspace{\onelineskip}


    The next example is a song you may have heard of. This uses 
\cmd{\flagverse} for labelling\index{stanza!number} the stanzas, 
and because the lines are numbered\index{line number} they can be referred to. 

\begin{lcode}
\settowidth{\versewidth}{In a cavern, in a canyon,}
\PoemTitle{Clementine}
\begin{verse}[\versewidth]
\linenumberfrequency{2}
\begin{altverse}
\flagverse{1.} In a cavern, in a canyon, \\
Excavating for a mine, \\
Lived a miner, forty-niner, \label{vs:49} \\
And his daughter, Clementine. \\!
\end{altverse}

\begin{altverse}
\flagverse{\textsc{chorus}} Oh my darling, Oh my darling, \\
Oh my darling Clementine. \\
Thou art lost and gone forever, \\
Oh my darling Clementine.
\end{altverse}
\linenumberfrequency{0}
\end{verse}
The `forty-niner' in line~\ref{vs:49} of the song
refers to the gold rush of 1849.
\end{lcode}

\settowidth{\versewidth}{In a cavern, in a canyon,}
\PoemTitle{Clementine}
\begin{verse}[\versewidth]\index[lines]{In a cavern, in a canyon}
\linenumberfrequency{2}
\begin{altverse}
\flagverse{1.} In a cavern, in a canyon, \\
Excavating for a mine, \\
Lived a miner, forty-niner, \label{vs:49} \\
And his daughter, Clementine. \\!
\end{altverse}

\begin{altverse}
\flagverse{\textsc{chorus}} Oh my darling, Oh my darling, \\
Oh my darling Clementine. \\
Thou art lost and gone forever, \\
Oh my darling Clementine.
\end{altverse}
\linenumberfrequency{0}
\end{verse}

\vspace{\onelineskip}

The `forty-niner' in line~\ref{vs:49} of the song
refers to the gold rush of 1849.

%\ablankline

 The last example is a much more ambitious use\index{stanza!indent pattern} 
of \cmd{\indentpattern}. In
this case it is defined as: 
\begin{lcode}
\indentpattern{0135554322112346898779775545653222345544456688778899}
\end{lcode}
and the result is shown on the next page.


\clearpage
\PoemTitle{Mouse's Tale}
\settowidth{\versewidth}{a mouse that morning}
\indentpattern{0135554322112346898779775545653222345544456688778899}
\begin{verse}[\versewidth]\index[lines]{Fury said to}
\setlength{\vgap}{1em}
\begin{patverse}
\large Fury said to \\
  a mouse, That \\
  he met \\
  in the \\
  house, \\
\normalsize `Let us \\
  both go \\
  to law: \\
  \emph{I} will \\
  prosecute \\
  \textit{you.} --- \\
  Come, I'll \\
\small take no \\
  denial; \\
  We must \\
  have a \\
  trial: \\
  For \\
\footnotesize really \\
  this \\
  morning \\
  I've \\
  nothing \\
  to do.' \\
  Said the \\
  mouse to \\
\scriptsize the cur, \\
  Such a \\
  trial, \\
  dear sir, \\
  With no \\
  jury or \\
  judge, \\
  would be \\
  wasting \\
  our breath.' \\
\tiny  `I'll be \\
  judge, \\
  I'll be \\
  jury.' \\
  Said \\
  cunning \\
  old Fury; \\
  `I'll try \\
  the whole \\
  cause \\
  and \\
  condemn \\
  you \\
  to \\
  death.'  \par
\end{patverse}
\end{verse}
\attrib{Lewis Carrol, \textit{Alice's Adventures in Wonderland}, 1865}


\index{verse|)}

%#% extend
%#% extstart include boxes-verbatims-and-files.tex

\svnidlong
{$Ignore: $}
{$LastChangedDate: 2018-03-06 15:58:01 +0100 (Tue, 06 Mar 2018) $}
{$LastChangedRevision: 584 $}
{$LastChangedBy: daleif@math.au.dk $}

%%%%%%%%%%%%%%%%%%%%%%%%%%%%%%%%%%%%%%%%%%%%%%%%%%%%%%
%%%%%% membook
