
%%%%%%%%%%%%%%%%%%%%%%%%%%%%%%%%%%%%%%%%%%%%%%%%%
%%%%%%%%%%%%%%%%%%%%%%%%%%%%%%%%
%%\chapterstyle{hangnum}
%%%%%%%%%%%%%%%%%%%%%%%%%%%%%%%
 \chapter{Floats and captions} \label{chap:captions}


    A float\index{float} environment is a particular kind of 
box\index{box}  --- one that \ltx\ decides where it should go although
you can provide hints as to where it should be placed; 
all other boxes are put at the point where they are defined. 
Within reason you can put what you like within a float but it is 
unreasonable, for example, to put a float inside 
another float. The standard classes provide two kinds of float 
environments, namely \Ie{figure} and \Ie{table}. The only difference
between these is the naming and numbering of any caption\index{caption} 
within the environments --- a \cmd{\caption} in a \Ie{figure} 
environment uses \cmd{\figurename} while a \cmd{\caption} in a 
\Ie{table} environment uses \cmd{\tablename}. Figures and tables are 
numbered sequentially but the two numbering schemes are independent 
of each other.

    The class provides means of defining new kinds of floats. It also
provides additional forms of captions for use both within and outside 
float environments together with handles for changing the style 
of captions.

\section{New float environments} \label{sec:newfloat}

    It is often forgotten that the \ltx\ float environments come
in both starred and unstarred forms. The unstarred form typesets the 
float contents\index{float!single column} in one column, which is 
the most usual form for a book. The starred form typesets the contents 
of the float across the top of both columns\index{float!double column} 
in a \Lopt{twocolumn} document. In a \Lopt{onecolumn} document there 
is no difference between the starred and unstarred forms.

\index{float!new|(} %)| emacs

\begin{syntax}
 \cmd{\newfloat}\oarg{within}\marg{fenv}\marg{ext}\marg{capname} \\
\end{syntax}
\glossary(newfloat)%
  {\cs{newfloat}\oarg{within}\marg{fenv}\marg{ext}\marg{capname}}%
  {Creates new float environments, \texttt{fenv} and \texttt{fenv*},
   using counter \texttt{fenv}, which may be restarted by the 
   \meta{within} counter, putting captioning information into the
   file with extension \meta{ext}, and using \meta{capname} as the
   name for a caption.}
 The \cmd{\newfloat} command creates two new floating environments 
called \meta{fenv} and \meta{fenv*}. If there
is not already a counter\index{counter} defined for \meta{fenv} a new 
one will be created to be restarted by the counter \meta{within}, 
if that is specified.  A caption\index{caption} within the environment 
will be written out to a file with extension \meta{ext}.
The caption, if present, will start with \meta{capname}. For example, 
the \texttt{figure} float\index{figure!float definition} for the class 
is defined as:
\begin{lcode}
\newfloat[chapter]{figure}{lof}{\figurename}
\renewcommand{\thefigure}{%
  \ifnum\c@chapter>\z@ \thechapter.\fi \@arabic\c@figure}
\end{lcode}
The last bit of the definition is internal code to make sure that if a
figure\index{figure} is in the document before chapter numbering starts, 
then the figure\index{figure} number will not be preceeded by a 
non-existent chapter number.

 The captioning style\index{caption!style} for floats defined 
with \cmd{\newfloat} is the same as
for the figures\index{figure} and tables\index{table}.

    The \cmd{\newfloat} command generates several new commands, some of
which are internal \ltx\ commands. For convenience, assume that the 
command was called as 
\begin{lcode}
\newfloat{F}{X}{capname}
\end{lcode}
so \texttt{F} is the name of the float environment and also the name of 
the counter for the caption, and \texttt{X} is the file extension.
The following float environment and related commands are then created.

\begin{syntax}
\senv{F} float material \eenv{F} \\
\senv{F*} float material \eenv{F*} \\
\end{syntax}
 The new float environment is called \texttt{F}, and can be used as 
either \senv{F} or \senv{F*}, with the matching \eenv{F} or \eenv{F*}.
It is given the standard default position\index{float!position} 
specification of 
[\textsf{t}\ixposarg{t}\textsf{b}\ixposarg{b}\textsf{p}\ixposarg{p}].

\begin{syntax}
 \Icn{Xdepth} \\
\end{syntax}
 The \Icn{Xdepth} counter is analogous to the standard \Icn{tocdepth} 
counter in that it specifies that entries in a listing should not be
typeset\index{ToC!controlling entries} if their numbering level is 
greater than \Icn{Xdepth}. The default definition is 
\begin{lcode}
\setcounter{Xdepth}{1}
\end{lcode}
To have a subfloat of \texttt{X} appear in the listing do
\begin{lcode}
\setcounter{Xdepth}{2}
\end{lcode}


    As an example, suppose you wanted both figures\index{figure} 
(which come with the class), and diagrams\index{float!new diagram}. 
You could then do something 
like the following.
\begin{lcode}
\newcommand{\diagramname}{Diagram}
\newcommand{\listdiagramname}{List of Diagrams}
\newlistof{listofdiagrams}{dgm}{\listdiagramname}
\newfloat{diagram}{dgm}{\diagramname}
\newlistentry{diagram}{dgm}{0}
\begin{document}
 ...
\listoffigures
\listfofdiagrams
 ...
\begin{diagram}
\caption{A diagram} \label{diag1}
 ...
\end{diagram}
As diagram~\ref{diag1} shows ...
\end{lcode}


\begin{syntax}
  \cmd{\setfloatadjustment}\marg{floatname}\marg{code}
\end{syntax}
\glossary(setfloatadjustment)%
  {\cs{setfloatadjustment}\marg{floatname}\marg{code}}%
  {Add global internal adjustment to a given type of float. Empty by
    default, but can easily be used to make all floats centered or all
  tables in a smaller font size.}
Often it is useful to add some global configuration to a given type of
float such that one will not have to add this to each and every
float. For example to have all (floating) figures and tables automatically
centered plus have all (floating) tables typeset in \cmd{\small} use
\begin{lcode}
  \setfloatadjustment{figure}{\centering}
  \setfloatadjustment{table}{\small\centering}
\end{lcode}
\LMnote{2011/02/18}{I've removed the figure*adjustment feature, so
  there is no differenting between starred and non starred floats when
it comes to inner adjustment}


\subsection{Margin floats}
\label{sec:margin-floats}

We also provide two environments to insert an image or table into the
margin (using \cmd{\marginpar}). The construction is inspired by the
Tufte \LaTeX\ collection.
\begin{syntax}
  \senv{marginfigure}\oarg{len} float material \eenv{marginfigure} \\
  \senv{margintable}\oarg{len} float material \eenv{margintable} \\
\end{syntax}
\glossary(marginfigure)%
  {\senv{marginfigure}\oarg{length}}%
  {Environment which inserts its contents into the margin, and enables
  figure captions. The optional argument should be a length and is used
to perform manual up/down adjustments to the placement.}
\glossary(margintable)%
  {\senv{margintable}\oarg{length}}%
  {Environment which inserts its contents into the margin, and enables
  figure captions.The optional argument should be a length and is used
to perform manual up/down adjustments to the placement.}
Because this is inserted differently than the ordinary \Ie{figure} or
\Ie{table} floats, one might get into the situation where a figure
float inserted before a margin float, might float \emph{past} the
margin float and thus have different caption numbering. For this
reason the margin float contain a float blocking device such that any
unplaced floats are forced to be placed before we start typesetting a
margin figure.

\fancybreak{}

The \Ie{marginfigure} and \Ie{margintable} environments can of course
be adjusted using \cmd{\setfloatadjustment}, default
\begin{lcode}
  \setfloatadjustment{marginfigure}{\centering}
  \setfloatadjustment{margintable}{\centering}
\end{lcode}
It may be useful to adjust the captioning separately, for this we have
added
\begin{syntax}
  \cmd{\setmarginfloatcaptionadjustment}\marg{float}\marg{code}
\end{syntax}
where \meta{float} is \verb?figure? or \verb?table?. The intent is to
enable the user to choose a different captioning style (or similar)
within a margin float, for example typesetting the caption ragged
left/right depending on the page.

This \emph{left/right depending on the page} is a little hard to do,
so for the \cmd{\marginpar} (which the margin float use internally) we
provide the following two macros
\begin{syntax}
  \cmd{\setmpjustification}\marg{at left of textblock}\marg{at right
    of textblock}\\
  \cmd{\mpjustification}
\end{syntax}
\glossary(setmpjustification)%
  {\cs{setmpjustification}\marg{at left of textblock}\marg{at right
    of textblock}}%
  {Loads the \cs{mpjustification} command to execute the left part
    when placed at the left of the text block and vice versa.}%
\glossary(mpjustification)%
  {\cs{mpjustification}}%
  {Specialized macro to be used within \cs{marginpar}s}
Basically \cmd{\mpjustification} execute \meta{at left of textblock}
when it is executed at the left of the text block and vice versa. For
it to work the margin into which the \cmd{marginpar} should do,
\emph{has} to be specified using \cmd{marginparmargin}. The default is
\begin{lcode}
  \setmpjustification{\raggedleft}{\raggedright}
\end{lcode}
To have both a margin figure and its caption typeset ragged against
the text block, use
\begin{lcode}
  \setfloatadjustment{marginfigure}{\mpjustification}
  \setmarginfloatcaptionadjustment{figure}{\captionstyle{\mpjustification}}
\end{lcode}
It may be useful to allow hyphenation within the raggedness, which can
be done using the \Lpack{ragged2e} package and
\begin{lcode}
  \setmpjustification{\RaggedLeft}{\RaggedRight}
\end{lcode}







\index{float!new|)} %|

\section{Setting off a float} \label{sec:floatsetoff}

\index{float!set off|(} %|

    Sometimes it is desireable to set off a float, more probably
an illustration than a tabular, from its surroundings. The \Ie{framed}
environment, described later in \Cref{chap:bvf}, might come in handy 
for this.

    The following code produces the example\indextwo{framed}{float} 
\figurerefname s~\ref{fig:framef} and~\ref{fig:framefcap}.

\begin{lcode}
\begin{figure}
\centering
\begin{framed}\centering
FRAMED FIGURE
\end{framed}
\caption{Example framed figure}\label{fig:framef}
\end{figure}

\begin{figure}
\begin{framed}\centering
FRAMED FIGURE AND CAPTION
\caption{Example framed figure and caption}\label{fig:framefcap}
\end{framed}
\end{figure}
\end{lcode}

\begin{figure}
%\begin{shadefigure}
\centering
\begin{framed}\centering
FRAMED FIGURE
\end{framed}
\caption{Example framed figure}\label{fig:framef}
%\end{shadefigure}
\end{figure}

\begin{figure}
\begin{framed}\centering
FRAMED FIGURE AND CAPTION
\caption{Example framed figure and caption}\label{fig:framefcap}
\end{framed}
\end{figure}

    If framing seems overkill then you can use 
rules\indextwo{ruled}{float} instead, as in the example code below
which produces 
\figurerefname s~\ref{fig:rulef} and~\ref{fig:rulefcap}.

\begin{lcode}
\begin{figure}
\centering
\hrule\vspace{\onelineskip}
RULED FIGURE
\vspace{\onelineskip}\hrule
\vspace{\onelineskip}
\caption{Example ruled figure}\label{fig:rulef}
\end{figure}

\begin{figure}
\centering
\hrule\vspace{\onelineskip}
RULED FIGURE AND CAPTION
\vspace{\onelineskip}\hrule
\vspace{0.2pt}\hrule
\vspace{\onelineskip}
\caption{Example ruled figure and caption}\label{fig:rulefcap}
\hrule
\end{figure}
\end{lcode}

\begin{shadefigure}
%\centering
%\definecolor{shadecolor}{gray}{0.75}
%\begin{shaded}
\hrule\vspace{\onelineskip}
RULED FIGURE
\vspace{\onelineskip}\hrule
\vspace{\onelineskip}
\caption{Example ruled figure}\label{fig:rulef}
%\end{shaded}
\end{shadefigure}

\begin{shadefigure}
%\centering
%\definecolor{shadecolor}{gray}{0.75}
%\begin{shaded}
\hrule\vspace{\onelineskip}
RULED FIGURE AND CAPTION
\vspace{\onelineskip}\hrule
\vspace{0.2pt}\hrule
\vspace{\onelineskip}
\caption{Example ruled figure and caption}\label{fig:rulefcap}
\hrule
%\end{shaded}
\end{shadefigure}

\index{float!set off|)}%|

 \section{Multiple floats} \label{sec:multfloats}

\index{float!multiple|(}%|

You can effectively put what you like inside a float box. Normally there 
is just a single picture or tabular in a float but you can include 
as many of these as will fit inside the box.

% \begin{figure}
 \begin{shadefigure}
 \centering
 \hspace*{\fill} 
   {ILLUSTRATION 1} \hfill {ILLUSTRATION 2} 
 \hspace*{\fill}
 \caption{Example float with two illustrations} \label{fig:mult1}
 \end{shadefigure}
% \end{figure}

    Three typical cases of multiple figures\index{figure}/tables\index{table} 
in a single float come to mind:
 \begin{itemize}
 \item Multiple illustrations\index{illustration}/tabulars with a single 
caption.
 \item Multiple illustrations\index{illustration}/tabulars each 
individually captioned.
 \item Multiple illustrations\index{illustration}/tabulars with one 
main caption and individual subcaptions.
 \end{itemize}

    Figure~\ref{fig:mult1} is an example of multiple 
illustrations\index{illustration!multiple} in a single float 
with a single caption. The figure was produced by the following code.
 \begin{lcode}
 \begin{figure}
 \centering
 \hspace*{\fill} 
   {ILLUSTRATION 1} \hfill {ILLUSTRATION 2} 
 \hspace*{\fill}
 \caption{Example float with two illustrations} \label{fig:mult1}
 \end{figure}
 \end{lcode}
 The \verb?\hspace*{\fill}? and \cmd{\hfill} commands were used to 
space the two illustrations\index{illustration} equally. Of course 
\cmd{\includegraphics} 
or \Ie{tabular} environments could just as well be used instead of the 
\verb?{ILLUSTRATION N}? text.

    The following code produces \figurerefname s~\ref{fig:mult2} 
and~\ref{fig:mult3} which are examples of two separately 
captioned\index{caption!multiple}
illustrations\index{illustration} in one float.
 \begin{lcode}
 \begin{figure}
 \centering
 \begin{minipage}{0.4\textwidth}
   \centering
   GRAPHIC 1
   \caption{Graphic 1 in a float} \label{fig:mult2}
 \end{minipage} 
 \hfill
 \begin{minipage}{0.4\textwidth}
   \centering
   GRAPHIC 2
   \caption{Graphic 2 in same float} \label{fig:mult3}
 \end{minipage} 
 \end{figure}
 \end{lcode}
 In this case the illustrations\index{illustration} (or graphics or 
tabulars) are put into separate \Ie{minipage} 
environments\index{minipage!in float} within the float, 
and the captions are also put within the \Ie{minipage}s. Note that 
any required \cmd{\label} must also be inside the \Ie{minipage}. 
If you wished, you could add yet another (main) caption after the end 
of the two \Ie{minipage}s.

%\begin{figure}
 \begin{shadefigure}
 \centering
 \begin{minipage}{0.4\textwidth}
   \centering
   GRAPHIC 1
   \caption{Graphic 1 in a float} \label{fig:mult2}
 \end{minipage} 
 \hfill
 \begin{minipage}{0.4\textwidth}
   \centering
   GRAPHIC 2
   \caption{Graphic 2 in same float} \label{fig:mult3}
 \end{minipage} 
 \end{shadefigure}
% \end{figure}

    It is slightly more complex if you want to put, say, both a 
tabulation captioned as a table and a graph, captioned as a figure, 
which illustrates the tabulation, as a 
float\index{float!multiple!table and figure} 
only permits one kind of caption. The class solves this problem by 
letting you define `fixed' captions\index{caption!fixed} which are
independent of the particular kind of the float. These are described in 
detail later.

    Things do get a little trickier, though, if the bodies and/or
the captions in a float are different heights 
(as in \figurerefname s~\ref{fig:mult2} and~\ref{fig:mult3}) 
and you want to align them horizontally.
Here are some examples.


    This code produces \figurerefname s~\ref{fig:left1} 
and~\ref{fig:right1}. The new \cmd{\hhrule} macro produces a rule
twice as thick as \cmd{\hrule} does.
\begin{lcode}
\newcommand*{\hhrule}{\hrule height 0.8pt}% double thickness

\begin{figure}
\hhrule \vspace{\onelineskip}
\null\hfill\parbox{0.45\linewidth}{%
  \centering
  Aligned to the center of the right figure
}\hfill
\parbox{0.45\linewidth}{%
  \centering
   This is the right figure which is taller
   than the first one (the one at the left)
}\hfill\null
\vspace{\onelineskip}\hrule
\null\hfill\parbox[t]{0.4\linewidth}{%
  \caption{Left figure}\label{fig:left1}%
}\hfill
\parbox[t]{0.4\linewidth}{%
  \caption{Right figure. This has more text than the adjacent
           caption (\ref{fig:left1}) so the heights are unequal}%
           \label{fig:right1}%
}\hfill\null
\hhrule
\end{figure}
\end{lcode}

\newcommand*{\hhrule}{\hrule height 0.8pt}% double thickness

%\begin{figure}
\begin{shadefigure}
\hhrule \vspace{\onelineskip}
\null\hfill\parbox{0.45\linewidth}{%
  \centering
  Aligned to the center of the right figure
}\hfill
\parbox{0.45\linewidth}{%
  \centering
   This is the right figure which is taller
   than the first one (the one at the left)
}\hfill\null
\vspace{\onelineskip}\hrule
\null\hfill\parbox[t]{0.4\linewidth}{%
  \caption{Left center aligned}\label{fig:left1}%
}\hfill
\parbox[t]{0.4\linewidth}{%
  \caption{Right figure. This has more text than the adjacent
           caption (\ref{fig:left1}) so the heights are unequal}%
           \label{fig:right1}%
}\hfill\null
\hhrule
\end{shadefigure}
%\end{figure}


    The following code produces \figurerefname s~\ref{fig:left2} 
and~\ref{fig:right2}.
\begin{lcode}
\begin{figure}
\hhrule \vspace{0.5\onelineskip}
\null\hfill\parbox[t]{0.45\linewidth}{%
  \centering
  Aligned to the top of the right figure
}\hfill
\parbox[t]{0.45\linewidth}{%
  \centering
   This is the right figure which is taller
   than the first one (the one at the left)
}\hfill\null
\vspace{0.5\onelineskip}\hrule
\null\hfill\parbox[t]{0.4\linewidth}{%
  \caption{Left top aligned}\label{fig:left2}%
}\hfill
\parbox[t]{0.4\linewidth}{%
  \caption{Right figure. This has more text than the adjacent
           caption (\ref{fig:left2}) so the heights are unequal}%
           \label{fig:right2}%
}\hfill\null
\hhrule
\end{figure}
\end{lcode}

%\begin{figure}
\begin{shadefigure}
\hhrule \vspace{0.5\onelineskip}
\null\hfill\parbox[t]{0.45\linewidth}{%
  \centering
  Aligned to the top of the right figure
}\hfill
\parbox[t]{0.45\linewidth}{%
  \centering
   This is the right figure which is taller
   than the first one (the one at the left)
}\hfill\null
\vspace{0.5\onelineskip}\hrule
\null\hfill\parbox[t]{0.4\linewidth}{%
  \caption{Left top aligned}\label{fig:left2}%
}\hfill
\parbox[t]{0.4\linewidth}{%
  \caption{Right figure. This has more text than the adjacent
           caption (\ref{fig:left2}) so the heights are unequal}%
           \label{fig:right2}%
}\hfill\null
\hhrule
\end{shadefigure}
%\end{figure}

    The next code produces \figurerefname s~\ref{fig:left3} 
and~\ref{fig:right3}.
\begin{lcode}
\begin{figure}
\hhrule \vspace{0.5\onelineskip}
\null\hfill\parbox[b]{0.45\linewidth}{%
  \centering
  Aligned to the bottom of the right figure
}\hfill
\parbox[b]{0.45\linewidth}{%
  \centering
   This is the right figure which is taller
   than the first one (the one at the left)
}\hfill\null
\vspace{0.5\onelineskip}\hrule
\null\hfill\parbox[t]{0.4\linewidth}{%
  \caption{Left bottom aligned}\label{fig:left3}%
}\hfill
\parbox[t]{0.4\linewidth}{%
  \caption{Right figure. This has more text than the adjacent
           caption (\ref{fig:left3}) so the heights are unequal}%
           \label{fig:right3}%
}\hfill\null
\hhrule
\end{figure}
\end{lcode}

%\begin{figure}
\begin{shadefigure}
\hhrule \vspace{0.5\onelineskip}
\null\hfill\parbox[b]{0.45\linewidth}{%
  \centering
  Aligned to the bottom of the right figure
}\hfill
\parbox[b]{0.45\linewidth}{%
  \centering
   This is the right figure which is taller
   than the first one (the one at the left)
}\hfill\null
\vspace{0.5\onelineskip}\hrule
\null\hfill\parbox[t]{0.4\linewidth}{%
  \caption{Left bottom aligned}\label{fig:left3}%
}\hfill
\parbox[t]{0.4\linewidth}{%
  \caption{Right figure. This has more text than the adjacent
           caption (\ref{fig:left3}) so the heights are unequal}%
           \label{fig:right3}%
}\hfill\null
\hhrule
\end{shadefigure}
%\end{figure}


\begin{syntax}
 \cmd{\newsubfloat}\marg{float} \\
\end{syntax}
\glossary(newsubfloat)%
  {\cs{newsubfloat}\marg{float}}%
  {Creates subcaptions for use in the \meta{float} float.}
 The \cmd{\newsubfloat} command\index{float!new subfloat}  
creates subcaptions\index{caption!new subcaption}
(\cmd{\subcaption}, \cmd{\subtop} and \cmd{\subbottom})
for use within the float environment \meta{fenv} previously
defined via \cmd{\newfloat}\oarg{...}\marg{fenv}\marg{...}. 
Subcaptions are discussed below in \Sref{sec:subcaps}. 
\PWnote{2009/08/23}{Improved cross references between \cs{newsubfloat} 
and \cs{subcaption}.}

\index{float!multiple|)}%|

 \section{Where \ltx\ puts floats} \label{sec:floatplace}

\index{float!placement|(}%|

 The general format for a float environment is: \\
 \senv{float}\oarg{loc} ... \eenv{float} 
or for double column\index{float!double column} floats: \\
 \senv{float*}\oarg{loc} ... \eenv{float*} \\
where the optional argument \meta{loc}, consisting of one or more characters,
specifies a location where the float may be placed. Note that the 
\Lpack{multicol}\index{column!multiple} package only supports the 
starred floats and it will not let you have a single 
column\index{float!single column} float. The possible \meta{loc} values 
are one or more of the following:
\begin{itemize}
\item[\textsf{b}\ixposarg{b}] \emph{bottom}: at the bottom\index{float!bottom}
    of a page. 
    This does not apply to double column\index{float!double column} floats 
    as they may only be placed at the top of a page.
\item[\textsf{h}\ixposarg{h}] \emph{here}: if possible exactly where 
  the float\index{float!here} environment is defined. 
  It does not apply to double 
  column\index{float!double column} floats.
\item[\textsf{p}\ixposarg{p}] \emph{page}: on a separate page
  containing only floats\index{float!page} (no text); this is called
  a \emph{float page}.
\item[\textsf{t}\ixposarg{t}] \emph{top}: at the top\index{float!top} 
    of a page. 
\item[\textsf{!}] make an extra effort to place the float at the 
  earliest place specified by the rest of the argument.
\end{itemize}
The default for \meta{loc} is \textsf{t}\textsf{b}\textsf{p}, 
so the float may be placed at the top, or bottom, or on a 
float\index{float!page} page; the default works well 95\% of the time.
Floats of the same kind are output in
definition order, except that a double column\index{float!double column} float 
may be output before
a later single column\index{float!single column} float of the same kind, or 
\textit{vice-versa}\footnote{As of 2015 this has been fixed in the
  \LaTeX{} kernel.}. 
A float is never put on an earlier page than its definition but may be 
put on the same or later page of its definition. If a float cannot be 
placed, all suceeding floats will be held up, and \ltx\ can store no 
more than 16 held up floats. A float cannot be placed if it would cause 
an overfull page, or it otherwise cannot be fitted according the  
float placement parameters.
A \cmd{\clearpage} or \cmd{\cleardoublepage} or \eenv{document} 
flushes\index{float!flush}
out all unprocessed floats, irrespective of the \meta{loc} and float
parameters, putting them on float-only\index{float!page} pages. 

\begin{syntax}
\cmd{\setfloatlocations}\marg{float}\marg{locs} \\
\end{syntax}
\glossary(setfloatlocations)%
  {\cs{setfloatlocations}\marg{float}\marg{locs}}%
  {Sets the default location for the \meta{float} (e.g., \Pe{table})
   to \meta{locs} (default \texttt{tbp}).}
You can set the location for all floats of type \meta{float} to
\meta{locs} with the \cs{setfloatlocations} declaration. The class
initialises these using:
\begin{lcode}
\setfloatlocations{figure}{tbp}
\setfloatlocations{table}{tbp}
\end{lcode}

\begin{syntax}
 \cmd{\suppressfloats}\oarg{pos} \\
\end{syntax}
\glossary(suppressfloats)%
  {\cs{suppressfloats}\oarg{pos}}%
  {Suppresses any floats on the current page at the given \meta{pos} placement.}
    You can use the command \cmd{\suppressfloats} to 
suppress\index{float!suppress} floats
at a given \meta{pos} on the current page. 
\cmd{\suppressfloats}\verb?[t]? 
prevents any floats at the top\index{float!suppress top} of the page and 
\cmd{\suppressfloats}\verb?[b]? 
prevents any floats at the bottom\index{float!suppress bottom} of the page. 
The simple \cmd{\suppressfloats} prevents both top and bottom floats.

\begin{syntax}
  \cmd{\FloatBlock}\\
  \cmd{\FloatBlockAllowAbove}\\
  \cmd{\FloatBlockAllowBelow}
\end{syntax}
\glossary(FloatBlock)%
  {\cs{FloatBlock}}%
  {Force \LaTeX\ to place all unplaced floats before proceeding this point.}
\glossary(FloatBlockAllowAbove)%
  {\cs{FloatBlockAllowAbove}}%
  {Lessens the restriction by \cs{FloatBlock} such that a float
    inserted \emph{after} a \cs{FloatBlock} can appear at the top of
    the same page as \cs{FloatBlock}.}
\glossary(FloatBlockAllowBelow)%
  {\cs{FloatBlockAllowBelow}}%
  {Lessens the restriction by \cs{FloatBlock} such that a float
    inserted \emph{before} a \cs{FloatBlock} can appear at the bottom of
    the same page as \cs{FloatBlock}.}
Contrary to \cmd{\suppressfloats} \cmd{\FloatBlock}\footnote{Yes, it
  \emph{is} the same as \cs{FloatBarrier} from the
  \Lpack{placeins} package, kudos to Donald Arseneau. For various
  reasons we cannot emulate the \Lpack{placeins} package and its
  options, thus we have verbatimly copied and renamed it instead.} will block
floats from passing this point, i.e.\ it demands \LaTeX\ to place any
unprocessed floats before proceeding. It is similar to
\cmd{\clearpage} but it does not necessarily introduce a page break
before proceeding.

\cmd{\FloatBlockAllowAbove} lessens the restriction a little, in a
situation like this
\begin{lcode}
  \FloatBlock
  some float here
\end{lcode}
\cmd{\FloatBlockAllowAbove} will allow the float to be placed at the
top of the same page as \cmd{\FloatBlock}. \cmd{\FloatBlockAllowBelow}
is the reverse situation.

It may be beneficial to be able to add \cmd{\FloatBlock} to sectional
commands. This can be done via
\begin{syntax}
  \cmd{\setFloatBlockFor}\marg{sectional name}
\end{syntax}
\glossary(setFloatBlockFor)%
{\cs{setFloatBLockFor}\marg{sectional name}}%
{Adds \cs{FloatBlock} within the \cs{\meta{section name}} macro.}
where \meta{sectional name} is \emph{withput} the \cs{}, i.e.\
\begin{lcode}
  \setFloatBlockFor{section}
\end{lcode}



 
\fancybreak{}

    The \Lpack{flafter} package, which should have come with your \ltx\
distribution, provides a means of preventing floats from moving
backwards from their definition position in the text. This can be useful to
ensure, for example, that a float early in a \verb?\section{...}? is not 
typeset before the section heading\index{heading}.

\begin{figure}
\centering
\drawparameterstrue
\drawfloatpage
\caption{Float and text page parameters}\label{fig:fpp}
\end{figure}

\begin{figure}
\centering
\drawparameterstrue
\setlayoutscale{0.9}
\drawfloat
\caption{Float parameters}\label{fig:flp}
\end{figure}

 \begin{table}
\begin{adjustwidth}{-3cm}{-3cm}
 \centering
%% \captionnamefont{\small\sffamily}
%% \captiontitlefont{\small\sffamily}
 \setlength{\belowcaptionskip}{10pt}
 \caption{Float placement parameters}\label{tab:fpp}
 \begin{tabular}{lp{0.5\textwidth}r} \toprule
 Parameter & Controls & Default \\ \midrule
 \multicolumn{3}{c}{Counters --- change with \cs{setcounter} } \\ \midrule
 \Icn{topnumber}  & max number of floats at top of a page & 2 \\
 \Icn{bottomnumber} & max number of floats at bottom of a page & 1 \\
 \Icn{totalnumber} & max number of floats on a text page & 3 \\
 \Icn{dbltopnumber} & like \Icn{topnumber} for double column 
                      floats\index{float!double column} & 2 \\ \midrule
 \multicolumn{3}{c}{Commands --- change with \cs{renewcommand} } \\ \midrule
 \cmd{\topfraction} & max fraction of page reserved for top 
                      floats\index{float!top} & 0.7 \\
 \cmd{\bottomfraction} & max fraction of page reserved for bottom 
                         floats\index{float!bottom} & 0.3 \\
 \cmd{\textfraction} & min fraction of page that must have text & 0.2 \\
 \cmd{\dbltopfraction} & like \cmd{\topfraction} for double column
                         floats\index{float!double column} floats & 0.7 \\
 \cmd{\floatpagefraction} & min fraction of a float page that must have float(s) & 0.5 \\
 \cmd{\dblfloatpagefraction} & like \cmd{\floatpagefraction} for double column
                              floats\index{float!double column} & 0.5 \\ \bottomrule
\end{tabular}
\end{adjustwidth}
\end{table}


 \begin{table}
\begin{adjustwidth}{-3cm}{-3cm}
 \centering
%% \captionnamefont{\small\sffamily}
%% \captiontitlefont{\small\sffamily}
 \setlength{\belowcaptionskip}{10pt}
 \caption{Float spacing parameters}\label{tab:fsp}
 \begin{tabular}{lp{0.5\textwidth}r} \toprule
 Parameter & Controls & Default \\ \midrule
 \multicolumn{3}{c}{Text page lengths --- change with \cs{setlength} } \\ \midrule
 \lnc{\floatsep} & vertical space between floats & 12pt \\
 \lnc{\textfloatsep} & vertical space between a top (bottom) float and 
                       suceeding (preceeding) text & 20pt  \\
 \lnc{\intextsep} & vertical space above and below an \texttt{h} 
                    float\index{float!here} & 12pt \\
 \lnc{\dblfloatsep} & like \lnc{\floatsep} for double column
                      floats\index{float!double column} & 12pt \\
 \lnc{\dbltextfloatsep} & like  \lnc{\textfloatsep} for double column
                          floats\index{float!double column} & 20pt \\ \midrule
 \multicolumn{3}{c}{Float page lengths --- change with \cs{setlength} } \\ \midrule
 \lnc{\@fptop} & space at the top of the page & \verb?0pt plus 1fil? \\
 \lnc{\@fpsep} & space between floats & \verb?8pt plus 2fil? \\
 \lnc{\@fpbot} & space at the bottom of the page & \verb?0pt plus 1fil? \\
 \lnc{\@dblfptop} & like \lnc{\@fptop} for double column
                    floats\index{float!double column} & \verb?0pt plus 1fil? \\
 \lnc{\@dblfpsep} & like \lnc{\@fpsep} for double column 
                    floats\index{float!double column} & \verb?8pt plus 2fil? \\
 \lnc{\@dblfpbot} & like \lnc{\@fpbot} for double column 
                    floats\index{float!double column} & \verb?0pt plus 1fil? \\ 
\bottomrule
 \end{tabular}
\end{adjustwidth}
 \end{table}

 Figures~\ref{fig:fpp} and~\ref{fig:flp} illustrate the many float 
parameters\index{float!parameters}
and \tref{tab:fpp} lists the float parameters and the typical 
standard default values. The lengths controlling the spaces surroundind
floats are listed
in \tref{tab:fsp}; typical values are shown as they depend on both
the class and the size option.

    Given the displayed defaults, the height of a top float must be 
less than 70\% of the textheight and there can be no more than 2 top 
floats\index{float!top}
on a text page. Similarly, the height of a bottom float must not
exceed 30\% of the textheight and there can be no more than 1 bottom
float\index{float!bottom} on a text page. There can be no more than 
3 floats (top, bottom and here\index{float!here})
on the page. At least 20\% of a text page with floats must be text.
On a float page\index{float!page} (one that has no text, only floats) 
the sum of the heights
of the floats must be at least 50\% of the textheight. The floats on a float
page should be vertically centered.

    Under certain extreme and unlikely conditions and with the defaults
\ltx\ might have trouble finding a place for a float. 
Consider what will happen if a float is specified as a bottom float and
its height is 40\% of the textheight and this is followed by a float whose
height is 90\% of the textheight. The first is too large to actually 
go at the bottom of a text page but too small to go on a float page by 
itself. The second has to go on a float page but it is too large to share 
the float page with the first float. \ltx\ is stuck!

    At this point it is worthwhile to be precise about the effect of a
 one character \meta{loc} argument:
\begin{itemize}
\item[\textsf{b}\ixposarg{b}] means: 
      `put the float at the bottom of a page with some
      text above it, and nowhere else'. 
      The float must fit into the \cmd{\bottomfraction} space 
      otherwise it and subsequent floats will be held up.
\item[\textsf{h}\ixposarg{h}] means: 
      `put the float at this point and nowhere else'. 
      The float must fit into the space left on the page 
      otherwise it and subsequent floats will be held up.
\item[\textsf{p}\ixposarg{p}] means: 
      `put the float on a page that has no text but may
      have other floats on it'. 
      There must be at least `\cmd{\floatpagefraction}' worth of 
      floats to go on a float only page before the float will be output.
\item[\textsf{t}\ixposarg{t}] means: 
      `put the float at the top of a page with some
      text below it, and nowhere else'. 
      The float must fit into the \cmd{\topfraction} space 
      otherwise it and subsequent floats will be held up.
\item[\textsf{!...}] means: 
      `ignore the \cs{...fraction} values for this float'.
\end{itemize}

 You must try and pick a combination from these that will let \ltx\ find
a place to put your floats. However, you can 
also change the float parameters to make it easier to find places
to put floats. Some examples are:
\begin{itemize}
\item Decrease \cmd{\textfraction} to get more `float' on a text page, 
  but the sum of \cmd{\textfraction} and \cmd{\topfraction} and the sum 
  of \cmd{\textfraction} and \cmd{\bottomfraction} should not exceed 1.0, 
  otherwise the placement algorithm falls apart. A minimum value for 
  \cmd{\textfraction} is about 0.10 --- a page with less than 10\% text 
  looks better with no text at all, just floats.

\item Both \cmd{\topfraction} and \cmd{\bottomfraction} can be increased, 
  and it does not matter if their sum exceeds 1.0. A good typographic 
  style is that floats are encouraged to go at the top of a page, and 
  a better balance is achieved if the float space on a page is larger
  at the top than the bottom.

\item Making \cmd{\floatpagefraction} too small might have the effect of a
 float page just having one small float. However, to make sure that a float
 page never has more than one float on it, do: 
\begin{lcode}
\renewcommand{\floatpagefraction}{0.01}
\setlength{\@fpsep}{\textheight}
\end{lcode}

\item Setting \lnc{\@fptop} and \lnc{\@dblftop} to \texttt{0pt}, 
      \lnc{\@fpsep} to \texttt{8pt}, 
       and \lnc{\@fpbot} and \lnc{\@dblfpbot} to \texttt{0pt plus 1fil} 
       will force floats on 
       a float page to start at the top of the page.


\item Setting \lnc{\@fpbot} and \lnc{\@dblfpbot} to \texttt{0pt}, 
      \lnc{\@fpsep} to \texttt{8pt},
       and \lnc{\@fptop} and \lnc{\@dblfptop} to \texttt{0pt plus 1fil} 
       will force floats on 
       a float page to the bottom of the page.
\end{itemize}

     If you are experimenting, a reasonable starting position is:
\begin{lcode}
\setcounter{topnumber}{3}
\setcounter{bottomnumber}{2}
\setcounter{totalnumber}{4}
\renewcommand{\topfraction}{0.85}
\renewcommand{\bottomfraction}{0.5}
\renewcommand{\textfraction}{0.15}
\renewcommand{\floatpagefraction}{0.7}
\end{lcode}
and similarly for double column\index{float!double column} floats if you will 
have any. Actually, there is no need
to try these settings as they are the default for this class.

    One of \ltx's little quirks is that on a text page, the `height' of 
a float is its actual height plus \lnc{\textfloatsep} or \lnc{\floatsep}, 
while on a float page the `height' is the actual height. This means that 
when using the default \meta{loc} of \verb?[tbp]? at least one of the text 
page float fractions (\cmd{\topfraction} and/or \cmd{\bottomfraction}) 
must be larger than the \cmd{\floatpagefraction} by an amount sufficient 
to take account of the maximum text page separation value.
 

\index{float!placement|)} % |

\section{Captions}

\index{caption|(} %)|

 Some publishers require, and some authors prefer, captioning styles
other than the one style provided by standard \ltx. 
Further, some demand that documents that include multi-part
tables\index{table} use a \textit{continuation caption} on all but the first
part of the multi-part table\index{table}. For the times where such 
a table\index{table} is specified by the author as a set of 
tables\index{table}, the class provides a simple `continuation' 
caption\index{caption!continuation} command to meet this 
requirement. It also provides a facility for an `anonymous' 
caption\index{caption!anonymous}
which can be used in any float\index{float} environment. 
Captions can be defined that are suitable for use in non-float
environments, such as placing a picture in a minipage and captioning
it just as though it had been put into a normal 
figure\index{figure} environment.

    The commands described below are very similar to
those supplied by the \Lpack{ccaption} package~\cite{CCAPTION}.

\section{Caption styling} 

\index{caption!style|(} %)|

    Just as a reminder, the default appearance of a caption for, say,
a table looks like this:
\begin{center}
Table 11.7: Title for the table
\end{center}
That is, it is typeset in the normal body font, with a colon after
the number.

    The class uses the following to specify the standard \ltx\ caption
style:
\begin{lcode}
\captionnamefont{}
\captiontitlefont{}
\captionstyle{}
\captionwidth{\linewidth}
\normalcaptionwidth
\normalcaption
\captiondelim{: }
\end{lcode}
These macros are explained in detail below.

\begin{syntax}
\cmd{\captiondelim}\marg{delim} \\
\end{syntax}
\glossary(captiondelim)%
  {\cs{captiondelim}\marg{delim}}%
  {Specifies \meta{delim} to be the delimeter between the number and title in a caption.}
 The default captioning style is to put a delimeter\index{caption!delimeter} 
in the form of a colon between the caption
number and the caption title. The command \cmd{\captiondelim}
can be used to change the delimeter. For example, to have an en-dash instead
of the colon, \verb?\captiondelim{-- }? will do the trick. 
Notice that no space is put between the delimeter and the title unless 
it is specified in the \meta{delim} parameter. 
The class initially specifies \verb?\captiondelim{: }? to give the normal 
delimeter.
% \begin{syntax}
%   \cmd{\captiondelimnocap}\marg{code}
% \end{syntax}
% \glossary(captiondelimnocap)%
%   {\cs{captiondelimnocap}\marg{code}}%
%   {Used as the delimiter after the caption number when the caption is
%     empty. Empty by default.}
% Whenever the caption text is empty, it looks a little strange to have
% \texttt{Figure 2.3:} and nothing else. Instead, in this case we replace
% \cmd{\captiondelim} by \cmd{\captiondelimnocap}, which is empty by
% default, i.e., we end up with \texttt{Figure 2.3} instead.



\begin{syntax}
\cmd{\captionnamefont}\marg{fontspec} \\
\end{syntax}
\glossary(captionnamefont)%
  {\cs{captionnamefont}\marg{fontspec}}%
  {Set the font for the first part (name and number) of a caption, upto and including
   the delimeter.}
 The \meta{fontspec} specified by \cmd{\captionnamefont} is used
for typesetting the caption\index{caption!font} name; 
that is, the first part of the caption
up to and including the delimeter (e.g., the portion `Table 3:').
\meta{fontspec} can be any kind of font specification and/or command and/or 
text. This first part of the caption is treated like: 
\begin{lcode}
{\captionnamefont Table 3: }
\end{lcode}
so font declarations, not font text-style commands, are needed for 
\meta{fontspec}. For instance, 
\begin{lcode}
\captionnamefont{\Large\sffamily}
\end{lcode} 
to specify a large sans-serif font. The class initially specifies 
\verb?\captionnamefont{}? to give the normal font.
 

\begin{syntax}
\cmd{\captiontitlefont}\marg{fontspec} \\
\end{syntax}
\glossary(captiontitlefont)%
  {\cs{captiontitlefont}\marg{fontspec}}%
  {Set the font for the caption title text.}
 Similarly, the \meta{fontspec} specified by \cmd{\captiontitlefont}
is used for typesetting the title text\index{caption!font} of a caption. 
For example,
\verb?\captiontitlefont{\itshape}? for an italic title text.
The class initially specifies \verb?\captiontitlefont{}?
to give the normal font.

\begin{syntax}
\cmd{\captionstyle}\oarg{short}\marg{style} \\
\cmd{\raggedleft} \cmd{\centering} \cmd{\raggedright} \cmd{\centerlastline} \\
\end{syntax}
\glossary(captionstyle)%
  {\cs{captionstyle}\oarg{short}\marg{style}}%
  {Set the paragraph style for the caption. The optional \meta{short} is
   the style for captions shorter than a full line.}
   By default the name and title of a caption are typeset as a block 
(non-indented) paragraph\index{paragraph!block}. 
\cmd{\captionstyle} can be used to alter this.
Sensible values for \meta{style} are: \cmd{\centering}, \cmd{\raggedleft} or
\cmd{\raggedright} for styles\index{caption!paragraph style} 
corresponding to these declarations. 
The \cmd{\centerlastline} style gives a block paragraph\index{paragraph!block}
but with the last line centered.
The class initially specifies \verb?\captionstyle{}?
to give the normal block paragraph style.

    If a caption is less than one line in length it may look odd if the
\meta{style} is \cmd{\raggedright}, say, as it will be left justified. 
The optional \meta{short} argument to \cmd{\captionstyle} can be used to
specify the style\index{caption!short style} for such short captions 
if it should differ from that for multiline\index{caption!multiline} 
captions. For example, I think that short captions look better 
centered:
\begin{lcode}
\captionstyle[\centering]{\raggedright}
\end{lcode} 

\begin{syntax}
\cmd{\hangcaption} \\
\cmd{\indentcaption}\marg{length} \\
\cmd{\normalcaption} \\
\end{syntax}
\glossary(hangcaption)%
  {\cs{hangcaption}}%
  {Multiline captions will be typeset as a hanging paragraph.}
\glossary(indentcaption)%
  {\cs{indentcaption}\marg{length}}%
  {Multiline captions will be typeset as a hanging paragraph hung by \meta{length}.}
\glossary(normalcaption)%
  {\cs{normalcaption}}%
  {Multiline captions will be typeset as a block paragraph.}
\PWnote{2009/06/25}{Hang/indent no longer affects short captions}
 The declaration \cmd{\hangcaption} causes captions to be typeset with 
the second and later lines of a multiline\index{caption!multiline} 
caption title indented by 
the width of the caption name. 
The declaration \cmd{\indentcaption} will indent title lines after 
the first by \meta{length}. These commands are independent of 
the \cmd{\captionstyle}\verb?{...}? and have no effect on short captions.
Note that a caption will not 
be simultaneously hung and indented. The \cmd{\normalcaption} declaration 
undoes any previous \cmd{\hangcaption} or \cmd{\indentcaption} declaration.
The class initially specifies \cmd{\normalcaption} to give the normal 
 non-indented paragraph\index{paragraph!indentation} style.

\begin{syntax}
\cmd{\changecaptionwidth} \\
\cmd{\captionwidth}\marg{length} \\
\cmd{\normalcaptionwidth} \\
\end{syntax}
\glossary(changecaptionwidth)%
  {\cs{changecaptionwidth}}%
  {Captions will be set within the width specified by \cs{captionwidth}.}
\glossary(captionwidth)%
  {\cs{captionwidth}\marg{length}}%
  {Set the caption width to \meta{length}.}
\glossary(normalcaptionwidth)%
  {\cs{normalcaptionwidth}}%
  {Captions will be set to the full width.}
   Issuing the declaration \cmd{\changecaptionwidth} causes the captions to
be typeset within a total width\index{caption!width} \meta{length} 
as specified by \cmd{\captionwidth}. 
Issuing the declaration \cmd{\normalcaptionwidth}
causes captions to be typeset as normal full width captions.
The class initially specifies
\begin{lcode}
\normalcaptionwidth
\captionwidth{\linewidth}
\end{lcode}
to give the normal width. If a caption is being set within the 
side captioned\index{caption!side caption} environments from 
the \Lpack{sidecap} package~\cite{SIDECAP}
then it must be a \cmd{\normalcaptionwidth} caption.

\begin{syntax}
\cmd{\precaption}\marg{pretext} \\
\cmd{\captiontitlefinal}\marg{text} \\
\cmd{\postcaption}\marg{posttext} \\
\end{syntax}
\glossary(precaption)%
  {\cs{precaption}\marg{pretext}}%
  {\meta{pretext} will be processed at the start of a caption.}
\glossary(captiontitlefinal)%
  {\cs{captiontitlefinal}\marg{text}}%
  {\meta{text} will be put immediately at the end of a caption title,
   but will not appear in a \listofx.}
\glossary(postcaption)%
  {\cs{postcaption}\marg{posttext}}%
  {\meta{posttext} will be processed at the end of a caption.}

  The commands \cmd{\precaption} and  \cmd{\postcaption}
specify \meta{pretext} and \meta{posttext} that will be processed at the
start and end of a caption. For example 
\begin{lcode}
\precaption{\rule{\linewidth}{0.4pt}\par}
\postcaption{\rule{\linewidth}{0.4pt}}
\end{lcode}
  will draw a horizontal line\index{caption!ruled} above and below 
the captions.
The class initially specifies
\begin{lcode}
\precaption{}
\postcaption{}
\end{lcode}
to give the normal appearance.

    The argument to \cmd{\captiontitlefinal} is put immediately after the 
title text but will not appear in the LoF or LoT. The default is
\begin{lcode}
\captiontitlefinal{}
\end{lcode}
but it could be used instead as, say
\begin{lcode}
\captiontitlefinal{.}
\end{lcode}
to put a period (full stop) after the title.
 
    If any of the above commands are used in a float\index{float}, 
or other, environment their effect is limited to the environment. 
If they are used in the preamble\index{preamble}
or the main text, their effect persists until replaced by a similar
command with a different parameter value. The commands do not affect the
appearance of the title in any \listofx.

\begin{syntax}
\cmd{\\}\oarg{length} \\
\cmd{\\*}\oarg{length} \\
\end{syntax}
 The normal \ltx\  command \cmd{\\} can be used within the
caption text to start a new line. Remember that \cmd{\\} is a fragile 
command, so if it is used within text that will be added to a \listofx\
it must be protected.
 As examples: 
\begin{lcode}
\caption{Title with a \protect\\ new line in 
         both the body and List of}
\caption[List of entry with no new line]%
        {Title with a \\ new line}
\caption[List of entry with a \protect\\ new line]%
        {Title text}
\end{lcode}

 Effectively, a caption is typeset as though it were:
 \begin{lcode}
 \precaption
 {\captionnamefont NAME NUMBER\captiondelim}
 {\captionstyle\captiontitlefont THE TITLE\captiontitlefinal}
 \postcaption
 \end{lcode}
 Replacing the above commands by their defaults leads to the simple
 format: \\
 \verb?{NAME NUMBER: }{THE TITLE}?

 As well as using the styling commands to make simple changes to the
captioning style, more noticeable modifications can also be made.
To change the captioning style so that the name and title are typeset in
a sans font\index{caption!font} it is sufficient to do:
 \begin{lcode}
 \captionnamefont{\sffamily}
 \captiontitlefont{\sffamily}
 \end{lcode}

 \begin{shadetable}
% \centering
 \captionnamefont{\sffamily}
 \captiondelim{}
 \captionstyle{\\}
 \captiontitlefont{\scshape}
 \setlength{\belowcaptionskip}{10pt}
 \caption{Redesigned table caption style} \label{tab:style}
 \begin{tabular}{lr} \toprule
  three & III \\
  five  & V \\
  eight & VIII \\ \bottomrule
  \end{tabular}
 \end{shadetable}

 A more obvious change in styling is shown in \tref{tab:style},
 which was coded as:
 \begin{lcode}
 \begin{table}
 \centering
 \captionnamefont{\sffamily}
 \captiondelim{}
 \captionstyle{\\}
 \captiontitlefont{\scshape}
 \setlength{\belowcaptionskip}{10pt}
 \caption{Redesigned table caption style} \label{tab:style}
 \begin{tabular}{lr} \toprule
  ...
 \end{table}
 \end{lcode}
This leads to the approximate caption format 
(processed within \cmd{\centering}): 
\begin{lcode}
{\sffamily NAME NUMBER}{\\ \scshape THE TITLE}
\end{lcode}
 Note that the newline command (\cmd{\\}) cannot be put in the first part
 of the format (i.e., the \verb?{\sffamily NAME NUMBER}?); it has to go into
 the second part, which is why it is specified via \verb?\captionstyle{\\}?
 and not \verb?\captiondelim{\\}?.

    If a mixture of captioning styles will be used you may want to
define a special caption command for each non-standard style. For
example for the style of the caption in \tref{tab:style}:
\begin{lcode}
\newcommand{\mycaption}[2][\@empty]{
  \captionnamefont{\sffamily\hfill}
  \captiondelim{\hfill}
  \captionstyle{\centerlastline\\}
  \captiontitlefont{\scshape}
  \setlength{\belowcaptionskip}{10pt}
  \ifx \@empty#1 \caption{#2}\else \caption[#1]{#2}\fi}
\end{lcode}
Remember that any code that involves the \idxatincode\texttt{@} sign must 
be either in
a package (\file{sty}) file or enclosed between a \cmd{\makeatletter} \ldots
\cmd{\makeatother} pairing (\seeatincode).

 The code for the \tref{tab:style} example can now be written as:
\begin{lcode}
\begin{table}
\centering
\mycaption{Redesigned table caption style} \label{tab:style}
\begin{tabular}{lr} \toprule
 ...
\end{table}
\end{lcode}
 Note that in the code for \cs{mycaption} I have added two
\cmd{\hfill} commands and \cmd{\centerlastline} compared with the original
specification.
It turned out that the original definitions
worked for a single line caption but not for a 
multiline\index{caption!multiline} caption.
The additional commands makes it work in both cases, forcing the
name to be centered as well as the last line of a multiline title,
thus giving a balanced appearence.

 \index{caption!style|)}%|


 \section{Continuation captions and legends}

\index{caption!continued|(}%|

\begin{syntax}
\cmd{\contcaption}\marg{text} \\
\end{syntax}
\glossary(contcaption)%
  {\cs{contcaption}\marg{text}}%
  {A continued caption, replacing the original title with \meta{text}.}
    The \cmd{\contcaption} command can be used to put 
 a `continued' or `concluded'
 caption into a float\index{float} environment. It neither increments the
 float number nor makes any entry into a float listing, but it
 does repeat the numbering of the previous \cmd{\caption} command.
 

   Table~\ref{tab:m} illustrates the use of the \cmd{\contcaption}
 command. The table\index{table} was produced from the following code.
 \begin{lcode}
   \begin{table}
   \centering
   \caption{A multi-part table} \label{tab:m}
   \begin{tabular}{lc} \toprule
    just a single line & 1 \\ \bottomrule
   \end{tabular}
   \end{table}

   \begin{table}
   \centering
   \contcaption{Continued}
   \begin{tabular}{lc} \toprule
    just a single line & 2 \\ \bottomrule
   \end{tabular}
   \end{table}

   \begin{table}
   \centering
   \contcaption{Concluded}
   \begin{tabular}{lc} \toprule
    just a single line & 3 \\ \bottomrule
   \end{tabular}
   \end{table}
 \end{lcode}

   \begin{shadetable}
%   \centering
   \caption{A multi-part table} \label{tab:m}
   \begin{tabular}{lc} \toprule
    just a single line & 1 \\ \bottomrule
   \end{tabular}
   \end{shadetable}

   \begin{shadetable}
%   \centering
   \contcaption{Continued}
   \begin{tabular}{lc} \toprule
    just a single line & 2 \\ \bottomrule
   \end{tabular}
   \end{shadetable}

   \begin{shadetable}
%   \centering
   \contcaption{Concluded}
   \begin{tabular}{lc} \toprule
    just a single line & 3 \\ \bottomrule
   \end{tabular}
   \end{shadetable}

\index{caption!continued|)}%|

\index{legend}
\index{caption!anonymous|(}%|

\begin{syntax}
\cmd{\legend}\marg{text} \\
\end{syntax}
\glossary(legend)%
  {\cs{legend}\marg{text}}%
  {A legend (an anonymous caption).}
  The \cmd{\legend} command is intended to be used to put an 
anonymous caption, or legend\index{legend} into a float\index{float} 
environment, but may be used anywhere.

   \begin{shadetable}
%   \centering
   \caption{Another table} \label{tab:legend}
   \begin{tabular}{lc} \toprule
    A legendary table & 5 \\
    with two lines    & 6 \\ \bottomrule
   \end{tabular}
   \legend{The legend}
   \end{shadetable}

    For example, the following code was used to produce the two-line
\tref{tab:legend}. The \cmd{\legend} command can be used within a 
float\index{float}
independently of any \cmd{\caption} command.
\begin{lcode}
\begin{table}
  \centering
  \caption{Another table} \label{tab:legend}
  \begin{tabular}{lc} \toprule
  A legendary table & 5 \\
  with two lines    & 6 \\ \bottomrule
  \end{tabular}
  \legend{The legend}
\end{table}
\end{lcode}

     \marginpar{\definecolor{shadecolor}{gray}{0.75}\begin{shaded}\legend{LEGEND}
                This is a marginal note with a legend.\end{shaded}}

  Captioned floats\index{float} are usually thought of in terms of the 
\Ie{table} and \Ie{figure} environments. There can be other kinds of 
float\index{float}.
As perhaps a more interesting example, the following code produces
the titled marginal\index{marginalia} note which should be displayed near here.
\begin{lcode}
     \marginpar{\legend{LEGEND}
                This is a marginal note with a legend.}
\end{lcode}

%You can even \legend{Legend in running text} use the \cmd{\legend}
%command in running text, as has been done in this sentence, 
%but I'm not sure why one might want to do that as \ltx\ already
%provides the \Ie{center} environment.

 If you want the legend text to be included\index{legend!in list of} 
in the \listofx{}
you can do it like this with the \cmd{\addcontentsline} macro.
\begin{lcode}
\legend{Legend title}
% left justified
\addcontentsline{lot}{table}{Legend title}    % or
% indented
\addcontentsline{lot}{table}{\protect\numberline{}Legend title}
\end{lcode}
The first of these forms will align the first line of the legend text
under the normal table\index{table} numbers. The second form will align 
the first line of the legend text under the normal \Ie{table} titles. 
In either case, second and later lines of a multi-line text will be 
aligned under the normal title lines.

   \begin{shadetable}
%   \centering
   \captiontitlefont{\sffamily}
   \legend{Legendary table}
   \addcontentsline{lot}{table}{Legendary table (toc 1)}
   \addcontentsline{lot}{table}{\protect\numberline{}
                                Legendary table (toc 2)}
   \begin{tabular}{lc} \toprule
    An anonymous table & 5 \\
    with two lines     & 6 \\ \bottomrule
   \end{tabular}
   \end{shadetable}

 As an example, the \textsf{Legendary table} is produced by the following code:
\begin{lcode}
\begin{table}
\centering
\captiontitlefont{\sffamily}
\legend{Legendary table}
\addcontentsline{lot}{table}{Legendary table (toc 1)}
\addcontentsline{lot}{table}{\protect\numberline{}
                             Legendary table (toc 2)}
\begin{tabular}{lc} \toprule
   An anonymous table & 5 \\
   with two lines     & 6 \\ \bottomrule
\end{tabular}
\end{table}
\end{lcode}
 Look at the List of Tables to see how the two forms of \cmd{\addcontentsline}
are typeset.


\begin{syntax}
\cmd{\namedlegend}\oarg{short-title}\marg{long-title} \\
\end{syntax}
\glossary(namedlegend)%
  {\cs{namedlegend}\oarg{short}\marg{long}}%
  {Like \cs{caption} but no number and no \listofx\ entry.}
 As a convenience, the \cmd{\namedlegend}\index{legend!named}
command is like the \cmd{\caption} command except that it does not number
the caption and, by default, puts no entry into a \listofx{} file. Like
the \cmd{\caption} command, it picks up the name to be prepended to the
title text from the float\index{float} environment in which it is called (e.g.,
it will use \cmd{\tablename} if called within a \Ie{table} environment). The
following code is the source of the \textit{Named legendary table}.
 \begin{lcode}
 \begin{table}
 \centering
 \captionnamefont{\sffamily}
 \captiontitlefont{\itshape}
 \namedlegend{Named legendary table}
 \begin{tabular}{lr} \toprule
 seven & VII \\
 eight & VIII \\ \bottomrule
 \end{tabular}
 \end{table}
 \end{lcode}

 \begin{shadetable}
% \centering
 \captionnamefont{\sffamily}
 \captiontitlefont{\itshape}
 \namedlegend{Named legendary table}
 \begin{tabular}{lr} \toprule
 seven & VII \\
 eight & VIII \\ \bottomrule
 \end{tabular}
 \end{shadetable}

\begin{syntax}
\cmd{\flegfloat}\marg{name} \\
\cmd{\flegtocfloat}\marg{title} \\
\end{syntax}
\glossary(flegfloat)%
  {\cs{flegfloat}\marg{name}}%
  {Where \texttt{float} is a float type (e.g. \texttt{table}), defines the \meta{name} used by \cs{namedlegend}.}
\glossary(flegtocfloat)%
  {\cs{flegtocfloat}\marg{title}}%
  {Where \texttt{float} is a float type (e.g., \texttt{figure}), called by 
   \cs{namedlegend} to add \meta{title} to a \listofx.}
 The macro \cmd{\flegfloat}, where \texttt{float} is the name 
of a float\index{float} environment
(e.g., \texttt{figure}) is called by the \cmd{\namedlegend} macro. 
It is provided as a hook that defines the \meta{name} to be used as 
the name in \cmd{\namedlegend}. Two defaults are provided, \cmd{\flegtable}
and \cmd{\flegfigure} defined as:
 \begin{lcode}
 \newcommand{\flegtable}{\tablename}
 \newcommand{\flegfigure}{\figurename}
 \end{lcode}
\glossary(flegtable)%
  {\cs{flegtable}}%
  {The name for a \cs{legend} in a \texttt{table}.}
\glossary(flegfigure)%
  {\cs{flegfigure}}%
  {The name for a \cs{legend} in a \texttt{figure}.}
which may be altered via \cmd{\renewcommand} if desired. 

The macro \cmd{\flegtocfloat}, where again \texttt{float} is the name 
of a float\index{float} environment
(e.g., \texttt{table}) is also called by the \cmd{\namedlegend} macro. 
It is provided as a hook that can be used to add \meta{title} to the \listofx.
Two examplars are provided, \cmd{\flegtocfigure} and \cmd{\flegtoctable}.
By default they are defined to do nothing, and can be changed via
\cmd{\renewcommand}. For instance, one could be changed for 
tables\index{table} as:
 \begin{lcode}
 \renewcommand{\flegtoctable}[1]{
   \addcontentsline{lot}{table}{#1}}
 \end{lcode}

\index{caption!anonymous|)}%|
\index{caption!outside a float|(}%|

  The \cmd{\legend} command produces a plain, unnumbered heading. It can also
be useful sometimes to have named and numbered captions outside
a floating\index{float} environment, perhaps in a \Ie{minipage},
if you want the table\index{table} or picture\index{illustration} 
to appear at a precise location in your document.


\index{caption!fixed|(}%|

\begin{syntax}
\cmd{\newfixedcaption}\oarg{capcommand}\marg{command}\marg{float} \\
\cmd{\renewfixedcaption}\oarg{capcommand}\marg{command}\marg{float} \\
\cmd{\providefixedcaption}\oarg{capcommand}\marg{command}\marg{float} \\
\end{syntax}
\glossary(newfixedcaption)%
  {\cs{newfixedcaption}\oarg{capcommand}\marg{command}\marg{float}}%
  {Defines a captioning command \cs{command} that may used outside the
   \meta{float} float as though it was inside it. The \cs{capcommand}
  must have been previously defined as a captioning command for \meta{float}.}
\glossary(renewfixedcaption)%
  {\cs{renewfixedcaption}\oarg{capcommand}\marg{command}\marg{float}}%
  {A `renew' version of \cs{newfixedcaption}.}
\glossary(providefixedcaption)%
  {\cs{providefixedcaption}\oarg{capcommand}\marg{command}\marg{float}}%
  {A `provide' version of \cs{newfixedcaption}.}
 The \cmd{\newfixedcaption} command, and its friends, can be used 
to create or modify a captioning \meta{command} that may be used 
outside the float\index{float} environment \meta{float}.
Both the environment \meta{float} and a captioning command, 
\meta{capcommand}, for that environment must have been defined before
calling \cmd{\newfixedcaption}. Note that \cmd{\namedlegend} can be used
as \meta{capcommand}.
% The \cmd{\renewfixedcaption} and \cmd{\providefixedcaption} commands 
% take the same arguments as \cmd{\newfixedcaption}; the three commands 
% are analagous to those in the \cmd{\newcommand} family.

 For example, to define a new \cs{figcaption} command for captioning pictures
 outside the \Ie{figure} environment, do
\begin{lcode}
 \newfixedcaption{\figcaption}{figure}
\end{lcode}
 The optional \meta{capcommand} argument is the name of the float\index{float}
captioning command that is being aliased. It defaults to \cmd{\caption}.
As an example of where the optional argument is required, if you
want to create a new continuation\index{caption!continued!outside a float} 
caption command for non-floating
tables\index{table}, say \cs{ctabcaption}, then do
\begin{lcode}
 \newfixedcaption[\contcaption]{\ctabcaption}{table}
\end{lcode}

 Captioning commands created by \cmd{\newfixedcaption} will be named and
 numbered in the same style as the original \meta{capcommand}, can
 be given a \cmd{\label}, and will appear in the appropriate 
 \listofx. They can also be used within floating\index{float}
 environments, but will not use the environment name as a guide to
 the caption name or entry into the \listofx. For
 example, using \cs{ctabcaption} in a \Ie{figure} environment will still
 produce a \textbf{Table\ldots} named caption.

   Sometimes captions are required on the 
opposite\index{caption!on opposite page} page to a 
figure\index{figure}, and a fixed caption can be useful in this context. 
For example, if figure\index{figure} captions should be placed on an 
otherwise empty page immediately before the actual figure\index{figure}, 
then this can be accomplished by the following hack:
\begin{lcode}
\newfixedcaption{\figcaption}{figure}
 ...
\afterpage{ % fill current page then flush pending floats
  \clearpage
  \begin{midpage}  %  vertically center the caption
  \figcaption{The caption} %  the caption
  \end{midpage}
  \clearpage
  \begin{figure}THE FIGURE, NO CAPTION HERE\end{figure}
  \clearpage
 }  % end of \afterpage
 \end{lcode}
 Note that the \Lpack{afterpage} package~\cite{AFTERPAGE} is needed, 
which is part of the required tools bundle. 
The \Lpack{midpage} package supplies the \verb?midpage?
environment, which can be simply defined as:
\begin{lcode}
\newenvironment{midpage}{\vspace*{\fill}}{\vspace*{\fill}}
\end{lcode}
The code, in particular the use of \cmd{\clearpage}, might need 
adjusting\index{start new page} to meet your particular requirements.
\begin{itemize}
\item \cmd{\clearpage} gets you to the next page, which may be odd or even.
\item \cmd{\cleardoublepage} gets you to the next odd-numbered page.
\item \cmd{\cleartoevenpage} ensures that you get to the next
even-numbered page.
\end{itemize}

    As a word of warning, if you mix both floats and fixed environments 
with the same kind of caption you have to ensure that they get printed 
in the correct order in the final document. If you do not do this, then 
the \listofx\ captions will come out in the wrong order (the lists are 
ordered according the
page number in the typeset document, \emph{not} your source input order).


\index{caption!fixed|)}%|

\index{caption!outside a float|)}%|

 \section{Bilingual captions}

\index{caption!bilingual|(}%|
\index{bilingual captions|(}%|

    Some documents require bilingual (or more) captions. The class 
 provides a set of commands for bilingual captions. Extensions to the
 set, perhaps to support trilingual captioning, are left as an exercise
 for the document author.
Essentially, the bilingual commands call the \cmd{\caption}
command twice, once for each language.

    Several commands for bilingual captions are provided. They all produce
the same appearance in the text but differ in what they put into 
the \listofx. 

\begin{syntax}
\cmd{\bitwonumcaption}\oarg{label}\marg{short1}\marg{long1}\% \\
                      \marg{NAME}\marg{short2}\marg{long2} \\
\cmd{\bionenumcaption}\oarg{label}\marg{short1}\marg{long1}\% \\
                      \marg{NAME}\marg{short2}\marg{long2} \\
\end{syntax}
\glossary(bitwonumcaption)%
  {\cs{bitwonumcaption}\oarg{label}\marg{short1}\marg{long1}\marg{NAME}\marg{short2}\marg{long2}}%
  {A bilingual caption with both captions numbered in the float and in the \listofx.}
\glossary(bionenumcaption)%
  {\cs{bionenumcaption}\oarg{label}\marg{short1}\marg{long1}\marg{NAME}\marg{short2}\marg{long2}}%
  {A bilingual caption with both captions numbered in the float but only the first in the \listofx.}

  Bilingual captions can be typeset by the \cmd{\bitwonumcaption} 
 command which has six arguments. 
 The first, optional argument \meta{label}, is the name of a label, if
 required.
 \meta{short1} and \meta{long1} are the short (i.e., equivalent
 to the optional argument
 to the \cmd{\caption} command) and long caption texts for
 the main language of the document. The value of the \meta{NAME} argument
 is used as the caption name for the second language caption, while
 \meta{short2} and \meta{long2} are the short and long caption texts
 for the second language. For example, if the main and secondary languages
 are English and German and a figure\index{figure} is being captioned:
\begin{lcode}
\bitwonumcaption{Short}{Long}{Bild}{Kurz}{Lang}
\end{lcode}
If the short title text(s) is not required, then leave the appropriate
argument(s) either empty or as one or more spaces, like:
\begin{lcode}
\bitwonumcaption[fig:bi1]{}{Long}{Bild}{  }{Lang}
\end{lcode}
Both language texts are entered into the 
appropriate\index{caption!bilingual!in list of} \listofx,
and both texts are numbered.

 Figure~\ref{fig:bi1}, typeset from the following code, is an example.
\begin{lcode}
\begin{figure}
\centering
EXAMPLE FIGURE WITH BITWONUMCAPTION
\bitwonumcaption[fig:bi1]%
    {}{Long \cs{bitwonumcaption}}%
    {Bild}{  }{Lang \cs{bitwonumcaption}} 
\end{figure}
\end{lcode}

 \begin{shadefigure}
% \centering
 EXAMPLE FIGURE WITH BITWONUMCAPTION
 \bitwonumcaption[fig:bi1]{}{Long \cs{bitwonumcaption}}{Bild}{  }{Lang \cs{bitwonumcaption}} 
 \end{shadefigure}

  Both \cmd{\bionenumcaption} and \cmd{\bitwonumcaption} take the same 
arguments.
The difference between the two commands is that \cmd{\bionenumcaption} does
not number the second language text in the \listofx.
Figure~\ref{fig:bi3}, typeset from the following, is an example of this.
\begin{lcode}
\begin{figure}
\centering
EXAMPLE FIGURE WITH BIONENUMCAPTION
\bionenumcaption[fig:bi3]%
    {}{Long English \cs{bionenumcaption}}%
    {Bild}{  }{Lang Deutsch \cs{bionenumcaption}} 
 \end{figure}
\end{lcode}

 \begin{shadefigure}
% \centering
 EXAMPLE FIGURE WITH BIONENUMCAPTION
 \bionenumcaption[fig:bi3]%
    {}{Long English \cs{bionenumcaption}}%
    {Bild}{  }{Lang Deutsch \cs{bionenumcaption}} 
 \end{shadefigure}

\begin{syntax}
\cmd{\bicaption}\oarg{label}\marg{short1}\marg{long1}\% \\
                \marg{NAME}\marg{long2} \\
\end{syntax}
\glossary(bicaption)%
  {\cs{bicaption}\oarg{label}\marg{short1}\marg{long1}\marg{NAME}\marg{long2}}%
  {A bilingual caption in a float but only the first added to the \listofx.}

When bilingual captions are typeset via the \cmd{\bicaption} 
command the second language text is not put into
the \listofx. 
The command takes 5 arguments. 
The optional \meta{label} is for a label if required.
\meta{short1} and \meta{long1} are the short and long caption texts for
the main language of the document. The value of the \meta{NAME} argument
is used as the caption name for the second language caption. The last
argument, \meta{long2}, is the caption text
for the second language (which is not put into the \listofx). 

 For example, if the main and secondary languages are English and German: 
\begin{lcode}
\bicaption{Short}{Long}{Bild}{Langlauf}
\end{lcode}
If the short title text is not required, then leave the appropriate
argument either empty or as one or more spaces.

 Figure~\ref{fig:bi2} is an example of using \cmd{\bicaption} and was 
 produced by the following code:
\begin{lcode}
\begin{figure}
\centering
   EXAMPLE FIGURE WITH A RULED BICAPTION
\precaption{\rule{\linewidth}{0.4pt}\par}
\midbicaption{\precaption{}%
              \postcaption{\rule{\linewidth}{0.4pt}}}
\bicaption[fig:bi2]%
    {Short English \cs{bicaption}}{Longingly}%
    {Bild}{Langlauf}
\end{figure}
\end{lcode}

 \begin{shadefigure}
% \centering
 EXAMPLE FIGURE WITH A RULED BICAPTION
 \precaption{\rule{\linewidth}{0.4pt}\par}
 \midbicaption{\precaption{}%
               \postcaption{\rule{\linewidth}{0.4pt}}}
 \bicaption[fig:bi2]{Short English \cs{bicaption}}{Longingly}%
    {Bild}{Langlauf}
 \end{shadefigure}

\begin{syntax}
\cmd{\bicontcaption}\marg{long1}\% \\
                    \marg{NAME}\marg{long2} \\
\end{syntax}
\glossary(bicontcaption)%
  {\cs{bicontcaption}\marg{long1}\marg{NAME}\marg{long2}}%
  {A continued bilingual caption.}
 Bilingual continuation captions can be typeset via the \cmd{\bicontcaption} 
command. In this case, neither language text is put into the \listofx. 
The command takes 3 arguments.
\meta{long1} is the caption text for
the main language of the document. The value of the \meta{NAME} argument
is used as the caption name for the second language caption. The last
argument, \meta{long2}, is the caption text
for the second language.
For example, if the main and secondary languages
are again English and German:
\begin{lcode}
\bicontcaption{Continued}{Bild}{Fortgefahren}
\end{lcode}

\begin{syntax}
 \cmd{\midbicaption}\marg{text} \\
\end{syntax}
\glossary(midbicaption)%
  {\cs{midbicaption}\marg{text}}%
  {In bilingual captions, \meta{text} is inserted after the first \cs{caption}
   and immediately before the second \cs{caption}.}
 The bilingual captions are implemented by calling \cmd{\caption} twice,
 once for each language. The command \cmd{\midbicaption}, 
 which is similar to the \cmd{\precaption} and \cmd{\postcaption} commands,
 is executed 
 just before calling the second \cmd{\caption}. Among other things,
 this can be used to
 modify the style\index{caption!bilingual!styling} of the second 
caption with respect to the first.
 For example, if there is a line above and below normal
 captions, it is probably undesirable to have a double line in the
 middle of a bilingual caption. So, for bilingual captions the
 following may be done within the float\index{float} before the caption:
\begin{lcode}
\precaption{\rule{\linewidth}{0.4pt}\par}
\postcaption{}
\midbicaption{\precaption{}%
              \postcaption{\rule{\linewidth}{0.4pt}}}
\end{lcode}
 This sets a line before the first of the two captions, then the
\cmd{\midbicaption}\verb?{...}? nulls the pre-caption line and adds 
a post-caption line for the second caption. The class initially specifies
\verb?\midbicaption{}?.

\index{bilingual captions|)}%|
\index{caption!bilingual|)}%|

 \section{Subcaptions}  \label{sec:subcaps}

\index{caption!subcaption|(}%|
\index{figure!subfigure|(}%|
\index{table!subtable|(}%|
\index{float!subfloat|(}%|

     The \Lpack{subfigure} package enables the captioning of 
sub-figures within a larger figure\index{figure}, 
and similarly for tables\index{table}.
The \Lpack{subfigure} package may be used with the class, or you
can use the class commands described below; these commands can only
be used inside a float environment for which a 
subfloat\footnote{See \Sref{sec:multfloats}.} has been specified
via \cmd{\newsubfloat}.

\begin{syntax}
\cmd{\subcaption}\oarg{list-entry}\marg{subtitle} \\
\end{syntax}
\glossary(subcaption)%
  {\cs{subcaption}\oarg{list-entry}\marg{subtitle}}%
  {Analagous to \cs{caption} but for an identified subcaption within a float.}
The \cmd{\subcaption} command is similar to the \cmd{\caption} command
and can only be used inside a float environment.
It typesets an identified \meta{subtitle}, where the identification
is an alphabetic character enclosed in parentheses. If the optional
\meta{list-entry} argument is present, \meta{list-entry} is added to
the caption listings for the float. If it is not present, then
\meta{subtitle} is added to the listing.

    The \meta{subtitle} is typeset within a box which is the width of
the surrounding environment, so \cmd{\subcaption} should only be used
within a fixed width box of some kind, for example a \Ie{minipage} as shown
below.
\begin{lcode}
\begin{figure}
\centering
\begin{minipage}{0.3\textwidth}
  \verb?Some verbatim text?
  \subcaption{First text}
\end{minipage}
\hfill
\begin{minipage}{0.3\textwidth}
  \verb?More verbatim text?
  \subcaption{Second text}
\end{minipage}
\caption{Verbatim texts}
\end{figure}
\end{lcode}
As the example code shows, the \cmd{\subcaption} command provides a 
means of putting verbatim elements into subfigures.

\begin{syntax}
\cmd{\subtop}\oarg{list-entry}\oarg{subtitle}\marg{text} \\
\cmd{\subbottom}\oarg{list-entry}\oarg{subtitle}\marg{text} \\
\end{syntax}
\glossary(subtop)%
  {\cs{subtop}\oarg{list-entry}\oarg{subtitle}\marg{text}}%
  {Puts a subcaption identifier, and optionally \meta{subtitle}, on top of \meta{text}.}
\glossary(subbottom)%
  {\cs{subbottom}\oarg{list-entry}\oarg{subtitle}\marg{text}}%
  {Puts a subcaption identifier, and optionally \meta{subtitle}, below \meta{text}.}
The command \cmd{\subtop} puts a subcaption identifier on top of
\meta{text}. If both optional arguments are present, \meta{list-entry}
will be added to the appropriate\index{caption!subcaption!in list of} 
listing, and \meta{subtitle} is
placed above the \meta{text} with the identifier. If only one optional
argument is present this is treated as being \meta{subtitle}; the
identifier and \meta{subtitle} are placed above the \meta{text}
and \meta{subtitle} is added to the listing. Regardless of the optional
arguments the identifier is always added to the listing and placed above
the \meta{text}.

    The \cmd{\subbottom} command is identical to \cmd{\subtop} except
that the identifier, and potentially the \meta{subtitle}, is placed
below the \meta{text}. Note that verbatim text cannot be used
in the \meta{text} argument to \cmd{\subbottom} or \cmd{\subtop}.

    The main caption can be at either the top or the bottom of the float.
The positioning of the main and subcaptions are independent.
For example
\begin{lcode}
\begin{figure}
  \subbottom{...}   % captioned as (a) below
  \subtop{...}      % captioned as (b) above
  \caption{...}
\end{figure}
\end{lcode}


    If a figure that\index{figure} includes subfigures
is itself continued then it may be desirable to
continue the captioning of the subfigures. For example, if Figure~3
has three subfigures, say A, B and C, and Figure~3 is continued then
the subfigures in the continuation should be D, E, etc.
\begin{syntax}
\cmd{\contsubcaption}\oarg{list-entry}\marg{subtitle} \\
\cmd{\contsubtop}\oarg{list-entry}\oarg{subtitle}\marg{text} \\
\cmd{\contsubbottom}\oarg{list-entry}\oarg{subtitle}\marg{text} \\
\cmd{\subconcluded} \\
\end{syntax}
\glossary(contsubcaption)%
  {\cs{contsubcaption}\oarg{list-entry}\marg{subtitle}}%
  { A continued \cs{subcaption}.}
\glossary(contsubtop)%
  {\cs{contsubtop}\oarg{list-entry}\oarg{subtitle}\marg{text}}%
  { A continued \cs{subtop}.}
\glossary(contsubbottom)%
  {\cs{contsubbottom}\oarg{list-entry}\oarg{subtitle}\marg{text}}%
  { A continued \cs{subbottom}.}
\glossary(subconcluded)%
  {\cs{subconcluded}}
  {Indicates (to \ltx) that a continued subfloat is finished.}

The \cmd{\contsubcaption}, \cmd{\contsuptop} and \cmd{\contsubbottom}
commands are the continued\index{caption!subcaption!continued} 
versions of the respective subcaptioning
commands. These continue the subcaption numbering scheme across
(continued) floats. In any event, the main caption can 
 be at the top or bottom of the float\index{float}.
 The \cmd{\subconcluded} command is used to indicate that the continued 
 (sub) float has been concluded and the numbering
 scheme is reinitialized. The command should be placed immediately
 before the end of the last continued environment.
    For example:
 \begin{lcode}
 \begin{figure}
 \subbottom{...}  captioned as (a) below
 \subbottom{...}  captioned as (b) below
 \caption{...}
 \end{figure}
 \begin{figure}
 \contsubtop{...}  captioned as (c) above
 \contsubtop{...}  captioned as (d) above
 \contcaption{Concluded}
 \subconcluded
 \end{figure}
 ...
 \begin{table}
 \caption{...}
 \subtop{...}     captioned as (a) above
 \subbottom{...}  captioned as (b) below
 \end{table}
 \end{lcode}


\begin{syntax}
\cmd{\label}\parg{bookmark}\marg{labstr} \\
\cmd{\subcaptionref}\marg{labstr} \\
\end{syntax}
\glossary(label)%
  {\cs{label}\parg{bookmark}\marg{labstr}}%
  {Associates \meta{labstr} with the current (section, caption, etc.)
   number and page number. If used inside a subfloat and with the 
  \Ppack{hyperref} package the optional \meta{bookmark} (note the parentheses
  not square brackets) is available to specify a hyperref bookmark.}
\glossary(subcaptionref)%
  {\cs{subcaptionref}\marg{labstr}}%
  {Print the subcaption identifer for a \meta{labstr} labelled subcaption.}
A \cmd{\label} command may be included in the \meta{subtitle} argument
of the subcaptioning\index{caption!subcaption!referencing} commands. 
Using the normal \cmd{\ref} macro to
refer to the label will typeset the number of the float (obtained
from a \cmd{\label}ed main \cmd{\caption}) and the subcaption
identifier. If the \cmd{\subcaptionref} macro is used instead of 
\cmd{\ref} then only the subcaption identifier is printed.

    In cases where the \Lpack{hyperref} package is used, the \cmd{\label}
command when used inside the \meta{subtitle} argument can take an optional
\meta{bookmark} argument, 
\emph{enclosed in parenthese \emph{not} square brackets},
which will create a bookmark field of the form `Subfigure 4.7(d)'.

    As an example to show the difference between \cmd{\subcaptionref} and
\cmd{\ref}, \fref{fig:twosubfig} and the paragraph immediately
following this one were produced by the code shown below.

    Figure \ref{fig:twosubfig} has two subfigures,
namely \ref{sf:1} and \subcaptionref{sf:2}.
\begin{shadefigure}
%\centering
\subbottom[Subfigure 1]{\fbox{SUBFIGURE ONE}\label{sf:1}}
\hfill
\subbottom[Subfigure 2]{\fbox{SUBFIGURE TWO}\label{sf:2}}
\caption{Figure with two subfigures} \label{fig:twosubfig}
\end{shadefigure}
\begin{lcode}
    Figure \ref{fig:twosubfig} has two subfigures,
namely \ref{sf:1} and \subcaptionref{sf:2}.
\begin{figure}
\centering
\subbottom[Subfigure 1]{\fbox{SUBFIGURE ONE}\label{sf:1}}
\hfill
\subbottom[Subfigure 2]{\fbox{SUBFIGURE TWO}\label{sf:2}}
\caption{Figure with two subfigures} \label{fig:twosubfig}
\end{figure}
\end{lcode}

\begin{syntax}
\cmd{\tightsubcaptions} \cmd{\loosesubcaptions} \\
\end{syntax}
\glossary(tightsubcaptions)%
  {\cs{tightsubcaptions}}%
  {Specifies the default vertical space around subcaptions.}
\glossary(loosesubcaptions)%
  {\cs{loosesubcaptions}}%
  {Specifies extra vertical space around subcaptions.}
As with many other aspects of typesetting the style of 
subcaptions\index{caption!subcaption!styling} may 
be specified. There is a small amount of vertical space 
surrounding a subcaption. More space is used after the \cmd{\loosesubcaptions}
declaration compared to that produced after the default
\cmd{\tightsubcaptions} declaration.

\begin{syntax}
\cmd{\subcaptionsize}\marg{size} \\
\cmd{\subcaptionlabelfont}\marg{fontspec} \\
\cmd{\subcaptionfont}\marg{fontspec} \\
\end{syntax}
\glossary(subcaptionsize)%
  {\cs{subcaptionsize}\marg{size}}%
  {Font size for subcaptions.}
\glossary(subcaptionlabelfont)%
  {\cs{subcaptionlabelfont}\marg{fontspec}}%
  {Font for subcaption identifiers.}
\glossary(subcaptionfont)%
  {\cs{subcaptionfont}\marg{fontspec}}%
  {Font for subcaption titles.}

The size of the font used for subcaptions is specified by 
\cmd{\subcaptionsize}, and the fonts for the identifier and text
are specified by \cmd{\subcaptionlabelfont} for the identifier and by
\cmd{\subcaptionfont} for the title text. The defaults are:
\begin{lcode}
\subcaptionsize{\footnotesize}
\subcaptionlabelfont{\normalfont}
\subcaptionfont{\normalfont}
\end{lcode}

\begin{syntax}
\cmd{\subcaptionstyle}\marg{style} \\
\cmd{\raggedleft} \cmd{\centering} \cmd{\raggedright} \cmd{\centerlastline} \\
\end{syntax}
\glossary(subcaptionstyle)%
  {\cs{subcaptionstyle}\marg{style}}%
  {Paragraph \meta{style} for subcaptions.}
The identifier and title of a subcaption is typeset as a block (i.e.,
non-indented) paragraph by specifying
\begin{lcode}
\subcaptionstyle{}
\end{lcode}
Other styles are available by calling \cmd{\subcaptionstyle}
with a styling \meta{cmd}. Values that you might use are:
\cmd{\centering} for a centered paragraph, \cmd{\raggedleft} or
\cmd{\raggedright} for ragged left or right paragraphs, or
\cmd{\centerlastline} which calls for a block paragraph with the last
line centered.

\begin{syntax}
\cmd{\hangsubcaption} \\
\cmd{\shortsubcaption} \\
\cmd{\normalsubcaption} \\
\end{syntax}
\glossary(hangsubcaption)%
  {\cs{hangsubcaption}}%
  {The subcaption version of \cs{hangcaption}.}
\glossary(shortsubcaption)%
  {\cs{shortsubcaption}}%
  {The subcaption version of \cs{shortcaption}.}
\glossary(normalsubcaption)%
  {\cs{normalsubcaption}}%
  {The subcaption version of \cs{normalcaption}.}
The \cmd{\hangsubcaption} declaration causes subcaptions to be typeset
with the identifier above the title. Following the \cmd{\shortsubcaption}
declaration subcaptions that are less than a full line in length are
typeset left justified instead of centered. The \cmd{\normalsubcaption}
declaration, which is the default, undoes any previous \cmd{\hangsubcaption}
or \cmd{\shortsubcaption} declaration, so that subcaptions are normally
centered.

\index{float!subfloat|)}%|
\index{table!subtable|)}%|
\index{figure!subfigure|)}%|
\index{caption!subcaption|)}%|

\section{Side captions}

\index{caption!side|(}%|

    Typically captions are put either above or below the element they
are describing. Sometimes it is desireable to put a caption at the
side of the element instead.


\begin{syntax}
\senv{sidecaption}\oarg{fortoc}\marg{title}\oarg{label} \\
 the body of the float \\
\eenv{sidecaption} \\
\end{syntax}
\glossary(sidecaption)%
  {\senv{sidecaption}\oarg{fortoc}\marg{title}\oarg{label}}%
  {Environment for setting a sidecaption.}%
    The \Ie{sidecaption} environment is used for a sidecaption rather than a
macro. The body of the float is put inside the environment. For example:
\begin{lcode}
\begin{figure}
  \begin{sidecaption}{An illustration}[fig:ill]
    \centering
    \includegraphics{...}
  \end{sidecaption}
\end{figure}
\end{lcode}
whereby the caption, `Figure N: An illustration', will be placed in the 
margin alongside the graphic, and for reference purposes will be given 
given the \cs{label} \texttt{fig:ill}.

\begin{syntax}
\lnc{\sidecapwidth} \lnc{\sidecapsep} \\
\cmd{\setsidecaps}\marg{sep}\marg{width} \\
\end{syntax}
\glossary(sidecapwidth)%
  {\cs{sidecapwidth}}%
  {Length specifying the maximum width of a sidecaption.}%
\glossary(sidecapsep)%
  {\cs{sidecapsep}}%
  {Length specifying the horizontal separation between a sidecaption and the float.}%
\glossary(setsidecaps)%
  {\cs{setsidecaps}\marg{sep}\marg{width}}%
  {Sets the lengths \cs{sidecapsep} and \cs{sidcapwidth} to the given
   values.}%

The caption is set in a box \lnc{\sidecapwidth} wide (the default is
\lnc{\marginparwidth}) offset \lnc{\sidecapsep} (default \lnc{\marginparsep})
into the margin.
The command \cmd{\setsidcaps} sets the \lnc{\sidecapsep} and 
\lnc{\sidecapwidth} to the given values. Changing the marginpar parameters,
for example with \cmd{\setmarginnotes}, will not change the side caption
settings. Note also that \cmd{\checkandfixthelayout}
neither checks nor fixes the side caption parameters.

\begin{syntax}
\cmd{\sidecapmargin}\marg{margin} \\
\piif{ifscapmargleft} \piif{scapmarglefttrue} \piif{scapmargleftfalse} \\
\end{syntax}
\glossary(sidecapmargin)%
  {\cs{sidecapmargin}\marg{margin}}%
  {Sets the margin for sidecaptions.}%
\glossary(ifsidecapleft)%
  {\cs{ifsidecapleft}}%
  {\ptrue\ if sidecaptions will be set in the left margin,
    otherwise they will be set in the right margin.}%

    If the float is a single column float in a twocolumn document then
the caption is 
always\footnote{Well, nearly always. See the \cs{overridescapmargin} command 
later.}
placed in the adjacent margin, otherwise the \cmd{\sidecapmargin} command
controls the margin where the sidecaption will be placed. 
The possible values for \meta{margin} are one of: 
\texttt{left}, 
\texttt{right},
\texttt{inner}, or
\texttt{outer}.
If \texttt{left} or \texttt{right} is specified the caption will go into
the left or right margin. If \texttt{inner} or \texttt{outer} is specified
then in a two sided document the caption will be on different sides of the
typeblock according to whether it is a recto or verso page; in a one sided
document the caption margin is fixed. The left margin is the default.

    When the caption is to be set in the left margin, \piif{ifscapmargleft} is
set \ptrue, and for a right margin it is set \pfalse.

\begin{syntax}
\cmd{\setsidecappos}\marg{pos} \\
\end{syntax}
\glossary(setsidcappos)%
  {\cs{setsidcappos}\marg{pos}}%
  {Declaration of the vertical position of a sidecaption with respect
   to the float.}%
By default a sidecaption is vertically centered with respect to the float
it is captioning. This can be altered by using the \cmd{\setsidecappos}
declaration. The allowed values for \meta{pos} are:
\begin{description}
\item[\texttt{t}] --- the top of the caption is aligned 
                      with the top of the float
\item[\texttt{c}] --- (the default) the center of the caption is aligned 
                      with the center of the float
\item[\texttt{b}] --- the bottom of the caption is aligned 
                      with the bottom of the float
\end{description}

    The other kinds of simple captions can also be put at the side
of a float. The positioning and styling commands for these are exactly
those for \Ie{sidecaption}. 
Bilingual captions, which are not considered to be simple, can only be 
placed above or below a float; 
no facilities are provided for setting them at the side..

\begin{syntax}
\senv{sidecontcaption}\marg{title}\oarg{label} \\
 the body of the float \\
\eenv{sidecontcaption} \\
\end{syntax}
\glossary(sidecontcaption)%
  {\senv{sidecontcaption}\marg{title}\oarg{label}}%
  {Environment for setting a continued sidecaption.}%
Sidecaptions may be continued with the \Ie{sidecontcaption} environment.

\begin{syntax}
\senv{sidenamedlegend}\oarg{fortoc}\marg{title} \\
 the body of the float \\
\eenv{sidenamedlegend} \\
\end{syntax}
\glossary(sidenamedlegend)%
  {\senv{sidenamedlegend}\marg{title}\oarg{label}}%
  {Environment for setting a named legend kind of sidecaption.}%
Named legends may be set at the side with the \Ie{sidenamedlegend} environment.

\begin{syntax}
\senv{sidelegend}\marg{title} \\
 the body of the float \\
\eenv{sidelegend} \\
\end{syntax}
\glossary(sidelegend)%
  {\senv{sidelegend}\marg{title}\oarg{label}}%
  {Environment for setting a legend kind of sidecaption.}%
Legends may be set at the side with the \Ie{sidelegend} environment.

\fancybreak{}

\textbf{Caveat}: Note that the \texttt{side...} envs expect the body
of the float to be \emph{taller} than the typeset caption/legend. In
case you write a long caption/legend for a short float, you may want
to visit this answer:
\url{http://tex.stackexchange.com/a/228412/3929}.


\subsection{Tweaks}

\begin{syntax}
\cmd{\sidecapstyle} \\
\end{syntax}
\glossary(sidecapstyle)%
  {\cs{sidecapstyle}}%
  {Style settings for a sidecaption.}%
Just before the caption is set, the \cmd{\sidecapstyle} command is called.
This may be used to set the styling for the particular caption. By default
it sets captions that are in the left margin raggedleft, and those
that are in the right margin are set raggedright. The default definition
is:
%\begin{lcode}
\begin{shadecode}
\newcommand*{\sidecapstyle}{%
%%  \captionnamefont{\bfseries}
  \ifscapmargleft
    \captionstyle{\raggedleft}%
  \else
    \captionstyle{\raggedright}%
  \fi}
\end{shadecode}
%\end{lcode}
 You can change the command to suit your purposes; for example, uncommenting
the \cmd{\captionnamefont} line would result in the caption's float name being
set in a bold font. 

\begin{syntax}
\cmd{\overridescapmargin}\marg{margin} \\
\lnc{\sidecapraise} \\
\end{syntax}
\glossary(overridescapmargin)%
  {\cs{overridescapmargin}\marg{margin}}%
  {A one-time override of \cs{sidecapmargin}.}%
\glossary(sidecapraise)%
  {\cs{sidecapraise}}%
  {Vertical distance added to the default vertical placement of a sidecaption.}%
Sometimes the caption may not be placed exactly where you want it --- it
may be in the wrong margin or at the wrong height.

The command \cmd{\overridescapmargin} will force the following caption into
the \meta{margin} you specify which can only be \texttt{left} or
\texttt{right}. In a twosided document where \cmd{\sidecapmargin}
is \texttt{inner} or \texttt{outer} and the caption goes in the wrong margin,
it is likely that the declaration \piif{strictpagecheck} will solve
the problem. The wrong margin might be chosen in a twocolumn document
where the float is in the second column; use 
\begin{lcode}
\overridescapmargin{right}
\end{lcode}
to fix this.

    The caption may not be at quite the height you want with respect to the
float. The caption will be raised by the length \lnc{\sidecapraise} 
in addition to the calculated movement (or lowered if \lnc{\sidecapraise}
is negative). 

\begin{syntax}
\cmd{\sidecapfloatwidth}\marg{length} \\
\end{syntax}
\glossary(sidecapfloatwidth)%
  {\cs{sidecapfloatwidth}\marg{length}}%
  {Macro holding the width of a float with a sidecaption.}%

    The float is set in a \Ie{minipage} with width \cmd{sidecapfloatwidth},
whose default definition is
\begin{lcode}
\newcommand*{\sidecapfloatwidth}{\linewidth}
\end{lcode}
That is, the normal width is the same as the current \lnc{\linewidth}.
For a narrow table, say, you may want to reduce this, for example to
half by
\begin{lcode}
\renewcommand*{\sidecapfloatwidth}{0.5\linewidth}
\end{lcode}
    Note that \cmd{\sidecapfloatwidth} is a macro, not a length, 
so it must be altered by using a \cmd{\renewcommand*}, 
\emph{not} by \cmd{\setlength}.

    If you do reduce the \cmd{\sidecapfloatwidth} you may notice that the
sidecaption is actually placed a distance \lnc{\sidecapsep} with respect
to the float's \Ie{minipage}, not with respect to the text block.

\newlength{\mylength}
\setlength{\mylength}{\linewidth}
\addtolength{\mylength}{-\sidecapsep}
\addtolength{\mylength}{-\sidecapwidth}
\begin{shadetable}
  \sidecapmargin{left}%
  \renewcommand*{\sidecapfloatwidth}{\mylength}%
  \raggedleft
  \begin{sidecaption}{%
    Permitted arguments for some sidecaption related commands}[scap:one]
  \centering
%  \begin{tabular}{ccc} \toprule
%  \cs{setsidecappos} & \cs{sidecapmargin} & \cs{overridescapmargin} \\ \midrule
%  \texttt{t}         & \texttt{left}      & \texttt{left}       \\
%  \texttt{c}         & \texttt{right}     & \texttt{right}       \\
%  \texttt{b}         & \texttt{inner}      &  \\
%                     & \texttt{outer}      &  \\ \bottomrule
%  \end{tabular}
  \begin{tabular}{cc} \toprule
  \cs{sidecapmargin} & \cs{overridescapmargin} \\ \midrule
  \texttt{left}      & \texttt{left}       \\
  \texttt{right}     & \texttt{right}       \\
  \texttt{inner}      &  \\
  \texttt{outer}      &  \\ \bottomrule
  \end{tabular}
\end{sidecaption}
\end{shadetable}

Table~\ref{scap:one} was created by the following code.
%\begin{lcode}
\begin{shadecode}
\newlength{\mylength}
\setlength{\mylength}{\linewidth}
\addtolength{\mylength}{-\sidecapsep}
\addtolength{\mylength}{-\sidecapwidth}
\begin{table}
  \sidecapmargin{left}%
  \renewcommand*{\sidecapfloatwidth}{\mylength}%
  \raggedleft
  \begin{sidecaption}{%
    Permitted arguments for some sidecaption related commands}[scap:one]
  \centering
  \begin{tabular}{cc} \toprule
  \cs{sidecapmargin} & \cs{overridescapmargin} \\ \midrule
  \texttt{left}      & \texttt{left}       \\
  \texttt{right}     & \texttt{right}       \\
  \texttt{inner}     &  \\
  \texttt{outer}     &  \\ \bottomrule
  \end{tabular}
\end{sidecaption}
\end{table}
\end{shadecode}
%\end{lcode} 

    The calculations on the \cs{mylength} length are so that the sidecaption
and float will just fit inside the typeblock.

Note that the \cmd{\raggedleft} command before the \Ie{sidecaption} environment
makes the float's \Ie{minipage} be placed raggedleft (i.e., moved across to
the right hand edge of the typeblock) while the \cmd{\centering} centers
the \Ie{tabular} within the minipage. You can get a variety of horizontal
placements by judicious use of \cmd{\raggedright}, \cmd{\centering}
and \cmd{\raggedleft} commands. If you do move the float sideways to leave
space for the caption make sure that the caption will go to the side you
want. In the example code I `moved' the float to the right so I made
sure that the caption would go on the left by explicitly setting
\begin{lcode}
\sidecapmargin{left}
\end{lcode}

    As far as \tx\ is concerned a sidecaption takes no horizontal space. If
you use a sidecaption in a wrapped float from, say, the \Lpack{wrapfig} 
package, make sure that the sidecaption gets placed where it won't be 
overlaid by the main text.

\index{caption!side|)}%|

 \section{How \ltx\ makes captions} \label{sec:ltx}%|

\index{caption!\ltx\ methods|(}%|

 This section provides an overview of how \ltx\ creates captions and
 gives some examples of how to change the captioning style directly.
 The section need not be looked at more than once unless you like 
 reading \ltx\ code
 or you want to make changes to \ltx's style of captioning not supported
by the class.

 The \ltx\ kernel provides tools to help in the definition of captions,
 but it is the particular class that decides on their format.

\begin{syntax}
 \cmd{\caption}\oarg{short}\marg{long} \\
\end{syntax}
\glossary(caption)%
  {\cs{caption}\oarg{short}\marg{long}}%
  {Typeset a caption with title \meta{long}, and add it, or \meta{short}
   instead if given, to a \listofx.} 
 The kernel (in \file{ltfloat.dtx}) defines the caption command via 
\begin{lcode}
\def\caption{%
    \refstepcounter\@captype \@dblarg{\@caption\@captype}}
\end{lcode}

\begin{syntax}
 \cmd{\@captype} \\
\end{syntax}
 \cmd{\@captype} is defined by the code that creates a new float\index{float} 
environment and is set to the environment's name (see the code for 
\cmd{\@xfloat} in \file{ltfloat.dtx}). For a \Ie{figure} environment,
 there is an equivalent to 
\begin{lcode}
\def\@captype{figure}
\end{lcode}

\begin{syntax}
 \cmd{\@caption}\marg{type}\oarg{short}\marg{long} \\
\end{syntax}
 The kernel also provides the \cmd{\@caption} macro as:
\begin{lcode}
\long\def\@caption#1[#2]#3{\par
  \addcontentsline{\csname ext@#1\endcsname}{#1}%     <-
    {\protect\numberline{\csname the#1\endcsname}%
     {\ignorespaces #2}}
  \begingroup
    \@parboxrestore
    \if@minipage
      \@setminipage
    \fi
    \normalsize
    \@makecaption{\csname fnum@#1\endcsname}%         <-
                 {\ignorespaces #3}\par
  \endgroup}
 \end{lcode}
 where \meta{type} is the name of the environment in which the caption
 will be used. Putting these three commands together results in the user's 
view of the caption command as \cmd{\caption}\oarg{short}\marg{long}.

 It is the responsibilty of the class (or package) which defines 
floats\index{float} to provide definitions for \cmd{\ext@type}, 
\cmd{\fnum@type} and \cmd{\@makecaption} which appear in the 
definition of \cmd{\@caption} (in the lines marked \verb?<-? above).

\begin{syntax}
 \cmd{\ext@type} \\
\end{syntax}
 This macro holds the name of the extension for a \listofx{} file.
 For example for the \Ie{figure} float\index{float} environment there is the
 definition equivalent to 
\begin{lcode}
 \newcommand{\ext@figure}{lof}
\end{lcode}

\begin{syntax}
 \cmd{\fnum@type} \\
\end{syntax}
 This macro is responsible for typesetting the caption number. For example,
 for the \Ie{figure} environment there is the definition equivalent to
\begin{lcode}
 \newcommand{\fnum@figure}{\figurename~\thefigure}
\end{lcode}

\begin{syntax}
 \cmd{\@makecaption}\marg{number}\marg{text} \\
\end{syntax}
 The \cmd{\@makecaption} macro, where \meta{number}
 is a string such as `Table~5.3' and \meta{text} is the caption text,
 performs the typesetting of the caption, and
 is defined in the standard classes (in \file{classes.dtx}) as the
 equivalent of:
\begin{lcode}
\newcommand{\@makecaption}[2]{
  \vskip\abovecaptionskip         <- 1
  \sbox\@tempboxa{#1: #2}         <- 2
  \ifdim \wd\@tempboxa >\hsize
    #1: #2\par                    <- 3
  \else
    \global \@minipagefalse
    \hb@xt@\hsize{\hfil\box\@tempboxa\hfil}
  \fi
  \vskip\belowcaptionskip}        <- 4
\end{lcode}

\begin{syntax}
 \lnc{\abovecaptionskip}
 \lnc{\belowcaptionskip} \\
\end{syntax}
\glossary(abovecaptionskip)%
  {\cs{abovecaptionskip}}%
  {Vertical space above a caption.}
\glossary(belowcaptionskip)%
  {\cs{belowcaptionskip}}%
  {Vertical space below a caption.}
  Vertical space is added before and after a caption (lines marked 1 and 4
in the code for \cmd{\@makecaption} above) and the amount of space is given
by the lengths \lnc{\abovecaptionskip}\index{caption!space above} 
and \lnc{\belowcaptionskip}. The
standard classes set these to 10pt and 0pt respectively. If you want
to change the space before or after a caption, use \cmd{\setlength} to change
the values. In figures\index{figure}, the caption is usually placed below the
illustration\index{illustration}. The actual space between the bottom 
of the illustration\index{illustration} and the baseline of the first 
line of the caption\index{caption!space above} is 
the \lnc{\abovecaptionskip} plus the 
\lnc{\parskip} plus the \lnc{\baselineskip}.
If the illustration\index{illustration} is in a \Ie{center} environment 
then additional space will be added by the \eenv{center}; it is usually 
better to use the \cmd{\centering} command rather than the 
\Ie{center} environment.

 The actual typesetting of a caption is effectively performed by the code
in lines marked 2 and 3 in the code for \cmd{\@makecaption}; note that
these are where the colon that is typeset after the number is specified. 
If you want to
make complex changes to the default captioning style you may have to
create your own version of \cmd{\@caption} using 
\cmd{\renewcommand}. On the other hand, many such changes can be achieved
by changing the definition of the 
the appropriate \cmd{\fnum@type} command(s). For example, to make the 
figure\index{figure} name and number\index{caption!font} bold: 
\begin{lcode}
\renewcommand{\fnum@figure}{\textbf{\figurename~\thefigure}}
\end{lcode}

 REMEMBER: If you are doing anything involving commands that include
the \idxatincode\texttt{@} character, and it's not in a class or package 
file, you have
to do it within a \cmd{\makeatletter} and \cmd{\makeatother} pairing
(\seeatincode). So,
if you modify the \cmd{\fnum@figure} command anywhere in your document
it has to be done as:
\begin{lcode}
 \makeatletter
 \renewcommand{\fnum@figure}{......}
 \makeatother
\end{lcode}

 \makeatletter
\let\oldfnum@figure\fnum@figure
 \renewcommand{\fnum@figure}{\textsc{\figurename~\thefigure}}
 \makeatother
 \begin{shadefigure}
 \centering
 A THOUSAND WORDS\ldots
 \caption{A picture is worth a thousand words}\label{fig:sc}
 \end{shadefigure}

 As an example, \fref{fig:sc} was created by the following code:
\begin{lcode}
 \makeatletter
 \renewcommand{\fnum@figure}{\textsc{\figurename~\thefigure}}
 \makeatother
 \begin{figure}
 \centering
 A THOUSAND WORDS\ldots
 \caption{A picture is worth a thousand words}\label{fig:sc}
 \end{figure}
\end{lcode}

 As another example, suppose that you needed to typeset the \cmd{\figurename}
and its number in a bold font, replace the 
colon\index{caption!delimeter} that normally appears
after the number by a long dash, and typeset the actual title 
text\index{caption!font} in
a sans-serif font, as is illustrated by the caption for 
\fref{fig:sf}. The following code does this.

 \makeatletter
 \renewcommand{\fnum@figure}[1]{\textbf{\figurename~\thefigure} --- \sffamily}
 \makeatother
 \begin{shadefigure}
  \centering
  ANOTHER THOUSAND WORDS\ldots
 \caption{A different kind of figure caption}\label{fig:sf}
 \end{shadefigure}

\begin{lcode}
 \makeatletter
 \renewcommand{\fnum@figure}[1]{\textbf{\figurename~\thefigure} 
                                --- \sffamily}
 \makeatother
 \begin{figure}
  \centering
  ANOTHER THOUSAND WORDS\ldots
 \caption{A different kind of figure caption}\label{fig:sf}
 \end{figure}
\end{lcode}
 Perhaps a little description of how this works is in order.
 Doing a little bit of \tx's macro processing by hand, the typesetting
 lines in \cmd{\@makecaption} (lines 2 and 3) get instantiated like:
\begin{lcode}
\fnum@figure{\figurename~\thefigure}: text
\end{lcode}
 Redefining \cmd{\fnum@figure} to take one argument and then not using the
 value of the argument essentially gobbles up the colon. Using
\begin{lcode}
\textbf{\figurename~\thefigure}
\end{lcode}
in the definition causes \cmd{\figurename} and the number to be typeset in
a bold font. After this comes the long dash. Finally, putting \cmd{\sffamily}
at the end of the redefinition causes any following text (i.e., the actual 
title) to be typeset using the sans-serif font.

 If you do modify \cmd{\@makecaption}, then spaces in the definition may be
important; also you must use the comment (\%) character in the same
places as I have done above. Hopefully, though, the class provides the
tools that you need to make most, if not all, of any likely caption styles.

\makeatletter
\let\fnum@figure\oldfnum@figure
\makeatother

\index{caption!\ltx\ methods|)}%|


 \section{Footnotes in captions}

\index{caption!footnote|(}%|
\index{footnote!in caption|(}%|

    If you want to have a caption with a footnote, think long and hard
 as to whether this is really essential. It is not normally considered
 to be good typographic practice, and to rub the point in \ltx\ does not
 make it necessarily easy to do. However, if you (or your publisher)
 insists, read on.

    If it is present, the optional argument to \cmd{\caption} is put into
the \listofx\ as appropriate. If the argument is not present, then the
text of the required argument is put into the \listofx. In the first case,
the optional argument is moving, and in the second case the required
argument is moving. The \cmd{\footnote} command\index{footnote!fragile} 
is fragile and must be
\cmd{\protect}ed (i.e., \verb?\protect\footnote{}?) if it is used in a moving 
argument. If you don't want the footnote to appear in 
the \listofx, use a
footnoteless optional argument and a footnoted required argument.

   You will probably be surprised if you just do, for example:
 \begin{lcode}
 \begin{figure}
 ...
 \caption[For LoF]{For figure\footnote{The footnote}}
 \end{figure}
 \end{lcode}
because (a) the footnote number may be greater than you thought, and (b)
the footnote text has vanished. This latter is because \ltx\ won't typeset
footnotes\index{footnote!in float} from a float\index{float}. 
To get an actual footnote within the float you have to use a minipage, 
like:
 \begin{lcode}
 \begin{figure}
 \begin{minipage}{\linewidth}
 ...
 \caption[For LoF]{For figure\footnote{The footnote}}
 \end{minipage}
 \end{figure}
 \end{lcode}
 If you are using the standard classes you may now find that you get two 
footnotes for the price of one. Fortunately this will not occur with this 
class, nor will an unexpected increase of the footnote number.

    When using a minipage as above, the footnote text is typeset at the
bottom of the minipage (i.e., within the float\index{float}). 
If you want the footnote text typeset at the bottom of the page, 
then you have to use the
\cmd{\footnotemark} and \cmd{\footnotetext} commands like:
 \begin{lcode}
 \begin{figure}
 ...
 \caption[For LoF]{For figure\footnotemark}
 \end{figure}
 \footnotetext{The footnote}
 \end{lcode}
 This will typeset the argument of the \cmd{\footnotetext} command at the
bottom of the page where you called the command. 
Of course, the figure\index{figure} might have floated\index{float} to a 
later page, and then it's a matter of some manual fiddling to get everything 
on the same page, and possibly to get the footnote 
marks\index{footnote!mark} to match correctly with the 
footnote\index{footnote!text} text.

 At this point, you are on your own.
\index{footnote!in caption|)}%|
\index{caption!footnote|)}%|


\section{The class versus the caption package (and its friends)}
\label{sec:class-versus-caption}

For some, the configurations for captions provided by the class, are
either a bit too complicated or not complicated enough.

The \Lpack{caption} package by Alex Sommerfeldt may provide a simpler
and much more extensive configuration interface for captions. The
package can be used with the class, but there are a few caveats:
\begin{enumerate}[(a)]
\item All of the special configuration macros provided by the class
  will no longer have any effect (\Lpack{caption} overwrites the core,
  and thus our interfaces will have no effect),
\item \cmd{\abovecaptionskip} will be reset to 10\,pt, and
  \cmd{\belowcaptionskip} to zero. (The class would set both to
  0.5\cmd{\onelineskip}, so if you need to change these, move the
  change until after \Lpack{caption} has been loaded) 
\end{enumerate}








\index{caption|)}%|


%#% extend
%#% extstart include rows-and-columns.tex

\svnidlong
{$Ignore: $}
{$LastChangedDate: 2018-09-06 15:05:23 +0200 (Thu, 06 Sep 2018) $}
{$LastChangedRevision: 612 $}
{$LastChangedBy: daleif@math.au.dk $}


%%%%%%%%%%%%%%%%%%%%%%%%%%%%%%%%%%%%%%%%%%%%%%%%%%%%%%%%%%%%%%%%%
%%%%%%%%%%%% Original memman

%%%%%%%%%%%%%%% end original memman
%%%%%%%%%%%%%%%%%%%%%%%%%%%%%%%%%%%%%%%%%%%%%%%%%%%%%%%%%%%%%%%%%

%%%%%%%%%%%%%%%%%%%%%%%%%%%%%%%%%%%%%%
%%%%\DeleteShortVerb{\|}
%%%%\MakeShortVerb{\=}
%%%%\input{tabmanbody}  %% rows & columns
%%%%  tabmanbody.tex
