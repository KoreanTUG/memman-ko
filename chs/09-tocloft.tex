%%%%%%%%%%%%%%%%%%%%%%%%%%
%%\chapterstyle{section}
%%%%%%%%%%%%%%%%%%%%%%%%%%
%\chapter{Tops and tails} \label{chap:topsandtails}
%\chapter{Contents lists} \label{chap:toc}
\chapter{목차 만들기} \label{chap:toc}

%This chapter describes how to change the appearance of the Table of Contents
%(\toc) and similar lists like the List of Figures (\lof). In the standard
%classes the typographic design of these is virtually fixed as it is buried
%within the class definitions.
이 장에서는 목차(Table of Contents, \toc)나 그림 목차(List of Figures, \lof)의
생김새를 바꾸는 방법을 다룬다.
보통의 클래스에서는 이것들의 디자인은 클래스 내부에 정의되어있기 때문에
고정되어 있으며 거의 바뀌지 않는다.

    As well as allowing these lists to appear multiple times in a
document, the \Mname\ class gives handles to easily manipulate the design 
elements. The class also provides means for you to define your own new kinds of
`\listofx'. 

    The functionality described is equivalent to the combination
of the \Lpack{tocloft} and \Lpack{tocbibind} 
packages~\cite{TOCLOFT,TOCBIBIND}.

\begin{syntax}
\cmd{\tableofcontents} \cmd{\tableofcontents*} \\
\cmd{\listoffigures} \cmd{\listoffigures*} \\
\cmd{\listoftables} \cmd{\listoftables*} \\
\end{syntax}
The commands \cmd{\tableofcontents}, \cmd{\listoffigures} and 
\cmd{\listoftables} typeset, repectively, the Table of Contents (\toc),
List of Figures (\lof) and List of Tables (\lot). In \Mname, unlike the 
standard classes, the unstarred versions add their respective titles to 
the \toc. The starred versions act like the standard classes' unstarred 
versions as they don't add their titles to the \toc.

    This chapter explains the inner workings behind the \toc\ and friends,
how to change their appearance and the apperance of the entries, and how to
create new \listofx. If you don't need any of these then you can
skip the remainder of the chapter.

 \section{General \prtoc\ methods}

 In \S\ref{sec:class-prtoc-methods} we will provide the class
 configuration interface for the various parts of the ToC. 

 In order to understand how these macros are used, we start by
 providing some background information this is a general description
 of how the standard \ltx\ classes process a Table of Contents (\toc).
 As the processing of List of Figures (\lof) and List of Tables (\lot)
 is similar I will just discuss the \toc. You may wish to skip this
 section on your first reading.

    The basic process is that each sectioning command writes out information
about itself --- its number, title, and page --- to the \pixfile{toc} file.
The \cmd{\tableofcontents} command reads this file and typesets the contents.

    First of all, remember that each sectional division has an associated
level as listed in in \tref{tab:seclevels} on 
\pref{tab:seclevels}. \ltx\ will not typeset an entry in the \toc{}
unless the value of the \Icn{tocdepth} counter is equal to or greater
than the level of the entry. The value of the \Icn{tocdepth} counter
can be changed by using \cmd{\setcounter} or \cmd{\settocdepth}.

\begin{syntax}
\cmd{\addcontentsline}\marg{file}\marg{kind}\marg{text} \\
\end{syntax}
\glossary(addcontentsline)%
  {\cs{addcontentsline}\marg{file}\marg{kind}\marg{text}}%
  {Writes heading/caption data to the \meta{file} in the form of
   the \cs{contentsline} macro.}
    \ltx\ generates a \pixfile{toc} file if the document contains a
 \cmd{\tableofcontents} command. The sectioning 
 commands\footnote{For figures and tables it is the 
 \cmd{\caption} command
 that populates the \pixfile{lof} and \pixfile{lot} files.}
 put entries into the \pixfile{toc} file by calling the 
 \cmd{\addcontentsline} 
 command, where \meta{file} is the file extension (e.g., \texttt{toc}),
 \meta{kind} is the kind of entry (e.g., \texttt{section} or \texttt{subsection}),
 and \meta{text} is the (numbered) title text. In the cases where
 there is a number, the \meta{text} argument is given in the
 form \verb?{\numberline{number}title text}?.

\begin{syntax}
\cmd{\contentsline}\marg{kind}\marg{text}\marg{page} \\
\end{syntax}
\glossary(contentsline)%
  {\cs{contentsline}\marg{kind}\marg{text}\marg{page}}%
  {An entry in a `\listofx' file for a \meta{kind} entry, with
  \meta{text} being the title which was on \meta{page}.}
     The \cmd{\addcontentsline} command writes an entry to the given file
 in the form: \\
 \cmd{\contentsline}\marg{kind}\marg{text}\marg{page} \\
 where \meta{page} is the page number.

    For example, if \verb?\section{Head text}? was typeset as 
`\textbf{3.4 Head text}' on page 27, then there would be the following
entry in the \pixfile{toc} file:
\begin{lcode}
\contentsline{section}{\numberline{3.4} Head text}{27}
\end{lcode}
Extracts from \pixfile{toc}, \pixfile{lof} and \pixfile{lot} files are shown in
\fref{fig:tocloflotfiles}.

\begin{figure}
\centering
Parts of a \file{toc} file:
\begin{lcode}
...
\contentsline{section}{\numberline{10.1}The spread}{77}
\contentsline{section}{\numberline{10.2}Typeblock}{89}
\contentsline{subsection}{\numberline{10.2.1}Color}{77}
...
\contentsline{chapter}{Index}{226}

\end{lcode}

Part of a \file{lof} file:

\begin{lcode}
...
\contentsline{figure}{\numberline{8.6}Measuring scales}{56}
\addvspace{10pt}
\addvspace{10pt}
\contentsline{figure}{\numberline{10.1}Two subfigures}{62}
\contentsline{subfigure}{\numberline{(a)}Subfigure 1}{62}
\contentsline{subfigure}{\numberline{(b)}Subfigure 2}{62}
...
\end{lcode}

Part of a \file{lot} file:

\begin{lcode}
...
\contentsline{table}{\numberline{1.7}Font declarations}{11}
\contentsline{table}{\numberline{1.8}Font sizes}{13}
\addvspace
\contentsline{table}{\numberline{3.1}Division levels}{21}
...
\end{lcode}
\caption{Example extracts from \file{toc}, \file{lof} and \file{lot} files}
\label{fig:tocloflotfiles}
\end{figure}

     For each \meta{kind} that might appear in a \file{toc} 
(\file{lof}, \file{lot}) file, \ltx\ provides a command: \\
 \cmd{\l@kind}\marg{title}\marg{page} \\
which performs the actual typesetting of the \cmd{\contentsline} entry. 


\begin{figure}
\setlayoutscale{0.8}
\drawtoc
\caption{Layout of a \prtoc{} (\prlof, \prlot) entry} \label{fig:ltoc}
\end{figure}

 
\begin{syntax}
\cmd{\@pnumwidth}\marg{length} \\
\cmd{\@tocrmarg}\marg{length} \\
\cmd{\@dotsep}\marg{number} \\
\end{syntax}
\glossary(@pnumwidth)%
  {\cs{@pnumwidth}\marg{length}}%
  {Space for a page number in the \prtoc\ etc.}
\glossary(@tocrmarg)%
  {\cs{@tocrmarg}\marg{length}}%
  {Right hand margin for titles in the \prtoc\ etc.}
\glossary(@dotsep)%
  {\cs{@docsep}\marg{num}}%
  {Distance, as \meta{num} math units, bewteen dots in the dotted lines 
   in the \prtoc\ etc.}
The general layout of a typeset entry is illustrated in \fref{fig:ltoc}. 
There are three
internal \ltx\ commands that are used in the typesetting. The page
number is typeset flushright in a box of width \cmd{\@pnumwidth}, and the box
is at the righthand margin\index{margin!right}. If the page number is too long 
to fit into  the box it will stick out into the righthand 
margin\index{margin!right}.
The title text is indented from the righthand margin\index{margin!right} by an 
amount given by \cmd{\@tocrmarg}.
Note that \cmd{\@tocrmarg} should be greater than \cmd{\@pnumwidth}. Some
entries are typeset with a dotted leader between the end of the title
title text and the righthand margin\index{margin!right} indentation. 
The distance, in
\index{math unit}\emph{math units}\footnote{There are 18mu to 1em.} 
between the dots in the leader is given by the value of \cmd{\@dotsep}. 
In the standard classes the same values are used for the \toc, \lof{} and 
the \lot.

    The standard values for these internal commands are:
 \begin{itemize}
 \item \cmd{\@pnumwidth} = 1.55em
 \item \cmd{\@tocrmarg} = 2.55em 
 \item \cmd{\@dotsep} = 4.5
 \end{itemize}
 The values can be changed by using \cmd{\renewcommand}, in spite of the
 fact that the first two appear to be lengths.

    Dotted leaders are not available for Part\index{part} and 
Chapter\index{chapter} \toc{} entries.

\begin{syntax}
\cmd{\numberline}\marg{number} \\
\end{syntax}
    Each \cmd{\l@kind} macro is responsible for setting the general 
 \textit{indent} from the lefthand margin\index{margin!left}, and the 
\textit{numwidth}.
The \cmd{\numberline} macro is responsible for typesetting the number 
flushleft in a box of width \textit{numwidth}. If the number is too long 
for the box then it will protrude into the title text. The title text is 
indented by (\textit{indent + numwidth}) from the lefthand 
margin\index{margin!left}. That is, the title text is typeset in a 
block of width \\
(\lnc{\linewidth} - \textit{indent} - \textit{numwidth} - \cmd{\@tocrmarg}). 

 \begin{table}
 \centering
 \caption[Indents and Numwidths]{Indents and Numwidths (in ems)} \label{tab:indents}
 \begin{tabular}{lcrrrr} \toprule
 Entry & Level & \multicolumn{2}{c}{Standard} & \multicolumn{2}{c}{\Lclass{memoir} class} \\
       &       & indent & numwidth & indent & numwidth \\ \midrule
 book          & -2 & ---  & --- & 0    & --- \\
 part          & -1 & 0    & --- & 0    & 1.5 \\
 chapter       & 0  & 0    & 1.5 & 0    & 1.5 \\
 section       & 1  & 1.5  & 2.3 & 1.5  & 2.3 \\
 subsection    & 2  & 3.8  & 3.2 & 3.8  & 3.2 \\
 subsubsection & 3  & 7.0  & 4.1 & 7.0  & 4.1 \\
 paragraph     & 4  & 10.0 & 5.0 & 10.0 & 5.0 \\
 subparagraph  & 5  & 12.0 & 6.0 & 12.0 & 6.0 \\
 figure/table  & (1) & 1.5 & 2.3 & 0    & 1.5 \\ 
 subfigure/table & (2) & ---& ---& 1.5  & 2.3 \\ 
\bottomrule
 \end{tabular}
 \end{table}

 Table~\ref{tab:indents} lists the standard values for the \textit{indent}
and \textit{numwidth}. There is no explicit \textit{numwidth} for a
part\index{part}; instead a gap of 1em is put between the number and the 
title text. 
Note that for a sectioning command the values
depend on whether or not the document class provides the \cmd{\chapter}
command; the listed values are for the \Lclass{book} and \Lclass{report} 
classes --- in the \Lclass{article} class a \cmd{\section} is treated
like a \cmd{\chapter}, and so on. Also, which somewhat surprises me, the 
table\index{table} and figure\index{figure} entries are all indented.

\begin{syntax}
\cmd{\@dottedtocline}\marg{level}\marg{indent}\marg{numwidth} \\
\end{syntax}
\glossary(@dottedtocline)%
  {\cs{@dottedtocline}\marg{level}\marg{indent}\marg{numwidth}}%
  {For a \prtoc, (\prlof, \prlot) entry at \meta{level} specifies the
   \meta{indent} and \meta{numwidth} and draws a dotted line between
   the title and page number.}
    Most of the \verb?\l@kind? commands are defined in terms of the
 \cmd{\@dottedtocline} command. This command takes three arguments: 
the \meta{level} argument is the level as shown in \tref{tab:indents},
and \meta{indent} and \meta{numwidth} are the \textit{indent} and 
\textit{numwidth} as illustrated in \fref{fig:ltoc}.
 For example, one definition of the \cmd{\l@section} command is:
\begin{lcode}
\newcommand*{\l@section}{\@dottedtocline{1}{1.5em}{2.3em}}
\end{lcode}
If it is necessary to change the default typesetting of the entries,
then it is usually necessary to change these definitions, but \Mname\ 
gives you handles to easily alter things without having to know the \ltx\
internals.

    You can use the \cmd{\addcontentsline} command to add 
\cmd{\contentsline} commands to a file. 
\index{add to contents}\index{insert in contents}

\begin{syntax}
\cmd{\addtocontents}\marg{file}\marg{text} \\
\end{syntax}
\glossary(addtocontents)%
  {\cs{addtocontents}\marg{file}\marg{text}}%
  {Inserts \meta{text} into \meta{file} (\prtoc, etc).}
    \ltx\ also provides the \cmd{\addtocontents}
 command that will insert \meta{text} into \meta{file}. You can use
 this for adding extra text and/or macros into the file, for processing
 when the file is typeset by \cmd{\tableofcontents} (or whatever other
 command is used for \meta{file} processing, such as \cmd{\listoftables}
 for a \file{lot} file).

 As \cmd{\addcontentsline} and \cmd{\addtocontents} write their arguments to a
 file, any fragile commands used in their arguments must be \cmd{\protect}ed.
 
    You can make certain adjustments to the \toc, etc., layout by modifying
some of the above macros. Some examples are:
 \begin{itemize}
 \item If your page numbers stick out into the righthand margin\index{margin!right}
  \begin{lcode}
  \renewcommand{\@pnumwidth}{3em} 
  \renewcommand{\@tocrmarg}{4em}
  \end{lcode}
 but using lengths appropriate to your document.

 \item To have the (sectional) titles in the \toc, etc., typeset ragged 
right with no  hyphenation
 \begin{lcode}
 \renewcommand{\@tocrmarg}{2.55em plus1fil}
 \end{lcode}
 where the value \texttt{2.55em} can be changed for whatever 
margin\index{margin} space you want.

 \item The dots in the leaders can be eliminated by increasing \cmd{\@dotsep}
 to a large value:
  \begin{lcode}
  \renewcommand{\@dotsep}{10000}
  \end{lcode}

 \item To have dotted leaders in your \toc\ and \lof\ but not in your \lot:
 \begin{lcode}
 ...
 \tableofcontents
 \makeatletter \renewcommand{\@dotsep}{10000} \makeatother
 \listoftables
 \makeatletter \renewcommand{\@dotsep}{4.5} \makeatother
 \listoffigures
 ...
 \end{lcode}

 \item To add a horizontal line across the whole width of the \toc\ below 
 an entry for a Part\index{part}:
 \begin{lcode}
 \part{Part title}
 \addtocontents{toc}{\protect\mbox{}\protect\hrulefill\par}
 \end{lcode}
 As  said earlier any fragile commands in the arguments to 
\cmd{\addtocontents} and \cmd{\addcontentsline} must be protected
 by preceding each fragile command with \cmd{\protect}. 
 The result of the example above
 would be the following two lines in the \file{.toc} file (assuming that it
 is the second Part and is on page 34):
 \begin{lcode}
 \contentsline {part}{II\hspace {1em}Part title}{34}
 \mbox {}\hrulefill \par
 \end{lcode}
 If the \cmd{\protect}s were not used, then the second line would 
instead be:
 \begin{lcode}
 \unhbox \voidb@x \hbox {}\unhbox \voidb@x \leaders \hrule \hfill 
         \kern \z@ \par
 \end{lcode}
which would cause \ltx\ to stop and complain because of the commands
that included the \texttt{@}\idxatincode\ (\seeatincode).
If you are modifying any command that includes an 
\texttt{@}\idxatincode\
sign then this must be done in either a \file{.sty} file or if in the 
document itself it must be surrounded by \cmd{\makeatletter} and 
\cmd{\makeatother}. For example, if you
 want to modify \cmd{\@dotsep} in the preamble\index{preamble} to your 
document you have to do it like this:
 \begin{lcode}
 \makeatletter
 \renewcommand{\@dotsep}{9.0}
 \makeatother
 \end{lcode}

\item To change the level of entries printed in the \toc\ (for example
      when normally subsections are listed in the \toc\ but for
      appendices\index{appendix} only the main title is required)
  \begin{lcode}
  \appendix
  \addtocontents{toc}{\protect\setcounter{tocdepth}{0}}
  \chapter{First appendix}
  ...
  \end{lcode}

 \end{itemize}
 
 \section{The class \prtoc{} methods} 
 \label{sec:class-prtoc-methods}

  The class provides various means of changing the look of the \toc, etc.,
without having to go through some of the above. 

\begin{syntax}
\cmd{\tableofcontents} \cmd{\tableofcontents*} \\
\cmd{\listoffigures} \cmd{\listoffigures*} \\
\cmd{\listoftables} \cmd{\listoftables*} \\
\end{syntax}
\glossary(tableofcontents)%
  {\cs{tableofcontents}}%
  {Typeset the \prtoc, adding its title to the \prtoc\ itself.}
\glossary(tableofcontents*)%
  {\cs{tableofcontents*}}%
  {Typeset the \prtoc.}
\glossary(listoffigures)%
  {\cs{listoffigures}}%
  {Typeset the \prlof, adding its title to the \prtoc.}
\glossary(listoffigures*)%
  {\cs{listoffigures*}}%
  {Typeset the \prlof.}
\glossary(listoftables)%
  {\cs{listoftables}}%
  {Typeset the \prlot, adding its title to the \prtoc.}
\glossary(listoftables*)%
  {\cs{listoftables*}}%
  {Typeset the \prlot.}
 The \toc, \lof, and \lot\ are printed at the point in the document where
 these commands are called, as per normal \ltx.
    You can use \cmd{\tableofcontents}, \cmd{\listoffigures}, etc., more
than once in a \Lclass{memoir} class document.


However, there are
 two differences between the standard \ltx\ behaviour and the behaviour
 with this class. In the standard \ltx\ classes
 that have \cmd{\chapter} headings\index{heading}, the \toc, \lof\ and \lot\ 
each appear on a new page. With this class they do not necessarily
 start new pages; if you want them to be on new pages you may have to
 specifically issue an appropriate command beforehand. For example:
 \begin{lcode}
  ...
 \clearpage
 \tableofcontents
 \clearpage
 \listoftables
 ...
 \end{lcode}
Also, the unstarred versions of the commands put their headings\index{heading} 
into the \toc, while the starred versions do not.

% \PWnote{2009/08/09}{Added description of KeepFromToc}
\begin{syntax}
\senv{KeepFromToc} \cs{listof...} \eenv{KeepFromToc} \\
\end{syntax}
\glossary(KeepFromToc)%
  {\senv{KeepFromToc}}%
  {Stop the titles of the enclosed \cs{listof...} commands from
   being added to the ToC.}

There is at least one package that uses \cs{tableofcontents} for its
own \listofx. When used with the class this will put the package's 
\listofx\ 
title into the \toc, and the package doesn't seem to know about 
\cs{tableofcontents*}.
The heading of any \cs{listof...} command that is in the \Ie{KeepFromToc} 
environment will not be added to the \toc. For example:
\begin{lcode}
\begin{KeepFromToc}
\listoffigures
\end{KeepFromToc}
\end{lcode}
is equivalent to \cs{listoffigures*}.


\PWnote{2009/03/17}{Added \cs{(one|two|doc)coltocetc} descriptions.}
\begin{syntax}
\cmd{\onecoltocetc} \\
\cmd{\twocoltocetc} \\
\cmd{\doccoltocetc} \\
\end{syntax}
\glossary(onecoltocetc)%
  {\cs{onecoltocetc}}%
  {Set the ToC, etc., in one column.}
\glossary(twocoltocetc)%
  {\cs{twocoltocetc}}%
  {Set the ToC, etc., in two columns.}
\glossary(doccoltocetc)%
  {\cs{doccoltocetc}}%
  {Set the ToC, etc., in one or two columns according to the class option.}


In the standard classes the \toc, etc., are set in one column even if the
document as a whole is set in two columns. This limitation is removed.
Following the \cmd{\onecoltocetc} declaration, which is the default, the
\toc\ and friends will be set in one column but after the \cmd{\twocoltocetc}
declaration they will be set in two columns. Following the \cmd{\doccoltocetc}
declaration they will be set in either one or two columns to match the 
document class \Lopt{onecolumn} or \Lopt{twocolumn} option. 

\begin{syntax}
\cmd{\maxtocdepth}\marg{secname} \\
\cmd{\settocdepth}\marg{secname} \\
\end{syntax}
\glossary(maxtocdepth)%
  {\cs{maxtocdepth}\marg{secname}}%
  {Sets the maximum value of the \Pcn{tocdepth} counter.}
\glossary(settocdepth)%
  {\cs{settocdepth}\marg{secname}}%
  {Sets the value of the \Pcn{tocdepth} counter in the \file{toc} file.}
% The class \cmd{\maxtocdepth} command sets the maximum 
% allowable value
% for the \Icn{tocdepth} counter. If used, the command must appear
% before the \cmd{\tableofcontents} command. By default, the class
% sets \verb?\maxtocdepth{section}?.
\LMnote{2013/04/24}{As far as I can see \cs{maxtocdepth} is not used
  anywhere. Manual clarified, see http://tex.stackexchange.com/a/97018/3929}
The class \cmd{\maxtocdepth} command sets the \Icn{tocdepth}
counter. It is currently not used in the \Lclass{memoir} class.


The \Lclass{memoir} class command \cmd{\settocdepth} is somewhat
analagous to the \cmd{\setsecnumdepth} command described in
\S\ref{sec:secnumbers}.  It sets the value of the \Icn{tocdepth}
counter and puts it into the \toc{} to (temporarily) modify what will
appear.  The \cmd{\settocdepth} and \cmd{\maxtocdepth} macros are from
the \Lpack{tocvsec2} package~\cite{TOCVSEC2}.


\begin{syntax}
\cmd{\phantomsection} \\
\end{syntax}
\glossary(phantomsection)
  {\cs{phantomsection}}%
  {A macro to be put before \cs{addcontentsline} when the \Ppack{hyperref} 
   package is used.}
\Note{} The \Lpack{hyperref} package~\cite{HYPERREF} appears to dislike 
authors using 
 \cmd{\addcontentsline}. To get it to work properly with \Lpack{hyperref}
 you normally have to put \cmd{\phantomsection} (a macro defined within
this class and the \Lpack{hyperref} package) immediately 
 before \cmd{\addcontentsline}.

 \subsection{Changing the titles} \label{sec:titles}

    Commands are provided for controlling the appearance of the
\toc, \lof\ and \lot\ titles. 

\begin{syntax}
\cmd{\contentsname} \cmd{\listfigurename} \cmd{\listtablename} \\
\end{syntax}
\glossary(contentsname)%
  {\cs{contentsname}}%
  {The title for the Table of Contents.}
\glossary(listfigurename)%
  {\cs{listfigurename}}%
  {The title for the List of Figures.}
\glossary(listtablename)%
  {\cs{listtablename}}%
  {The title for the List of Tables.}
Following \ltx\ custom, the title texts are the values
of the \cmd{\contentsname}, \cmd{\listfigurename} and \cmd{\listtablename}
commands.

 The commands for controlling the typesetting of the \toc, \lof\ and \lot\ 
titles all follow a similar pattern, so for convenience (certainly mine, 
and hopefully yours) in the following descriptions I will use \texttt{X},
as listed in \tref{tab:Xlistofxtitles},
to stand for the file extension for the appropriate \listofx. That is, any
of the following:
\begin{itemize}
\item \texttt{toc} or
\item \texttt{lof} or
\item \texttt{lot}. 
\end{itemize}
For example, \verb?\Xmark? stands for \cmd{\tocmark} or \cmd{\lofmark} or 
\cmd{\lotmark}.

\begin{table}
\centering
\caption{Values for \texttt{X} in macros for styling the titles of \listofx}
\label{tab:Xlistofxtitles}
\begin{tabular}{cccc}\toprule
\texttt{toc} & \texttt{lof} & \texttt{lot} & \texttt{\ldots} \\
\bottomrule
\end{tabular}
\end{table}

    The code for typesetting the \toc\ title looks like:
\begin{lcode}
\tocheadstart
\printtoctitle{\contentsname}
\tocmark
\thispagestyle{chapter}
\aftertoctitle
\end{lcode}
where the macros are described below.

\begin{syntax}
\cmd{\Xheadstart} \\
\end{syntax}
\glossary(Xheadstart)%
  {\cs{Xheadstart}}%
  {Generic macro called before printing a 'X List of' title.}
\begin{comment}
\glossary(tocheadstart)%
  {\cs{tocheadstart}}%
  {Called before printing the \prtoc\ title.}
\glossary(lofheadstart)%
  {\cs{lofheadstart}}%
  {Called before printing the \prlof\ title.}
\glossary(lotheadstart)%
  {\cs{lotheadstart}}%
  {Called before printing the \prlot\ title.}
\end{comment}
This macro is called before the title is actually printed.
Its default definition is
\begin{lcode}
\newcommand{\Xheadstart}{\chapterheadstart}
\end{lcode}

\begin{syntax}
\cmd{\printXtitle}\marg{title} \\
\end{syntax}
\glossary(printXtitle)%
  {\cs{printXtitle}\marg{title}}%
  {Generic macro printing \meta{title} as the title for the `X List of'.}
\begin{comment}
\glossary(printtoctitle)%
  {\cs{printtoctitle}\marg{title}}%
  {Prints \meta{title} as the title for the \prtoc.}
\glossary(printloftitle)%
  {\cs{printloftitle}\marg{title}}%
  {Prints \meta{title} as the title for the \prlof.}
\glossary(printlottitle)%
  {\cs{printlottitle}\marg{title}}%
  {Prints \meta{title} as the title for the \prlot.}
\end{comment}
The title is typeset via \cmd{\printXtitle}, which defaults to
using \cmd{\printchaptertitle} for the actual typesetting. 

\begin{syntax}
\cmd{\Xmark} \\
\end{syntax}
\glossary(Xmark)%
  {\cs{Xmark}}%
  {Generic macro setting the marks for the 'X List of'.}
\begin{comment}
\glossary(tocmark)%
  {\cs{tocmark}}%
  {Macro setting the marks for the \prtoc.}
\glossary(lofmark)%
  {\cs{lofmark}}%
  {Macro setting the marks for the \prlof.}
\glossary(lotmark)%
  {\cs{lotmark}}%
  {Macro setting the marks for the \prlot.}
\end{comment}
 These macros sets the marks for use by the running heads on the \toc, \lof, and
 \lot\ pages. The default definition is equivalent to:
\begin{lcode}
\newcommand{\Xmark}{\markboth{\...name}{\...name}}
\end{lcode}
where \verb?\...name? is \cmd{\contentsname} or \cmd{\listfigurename} or
\cmd{\listtablename} as appropriate. You probably don't need to change these, and
in any case they may well be changed by the particular \cmd{\pagestyle} in
use.

\begin{syntax}
\cmd{\afterXtitle} \\
\end{syntax}
\glossary(afterXtitle)%
  {\cs{afterXtitle}}%
  {Generic macro called after typesetting the title of the `X List of'.}
\begin{comment}
\glossary(aftertoctitle)%
  {\cs{aftertoctitle}}%
  {Macro called after typesetting the title of the \prtoc.}
\glossary(afterloftitle)%
  {\cs{afterloftitle}}%
  {Macro called after typesetting the title of the \prlof.}
\glossary(afterlottitle)%
  {\cs{afterlottitle}}%
  {Macro called after typesetting the title of the \prlot.}
\end{comment}
 This macro is called after the title is typeset and by
default it is defined to be \cmd{\afterchaptertitle}.

    Essentially, the \toc, \lof\ and \lot\ titles use the same format
as the chapter titles, and will be typeset according to the current
chapterstyle. You can modify their appearance by either using a
different chapterstyle for them than for the actual chapters, or
by changing some of the macros. As examples:
\begin{itemize}
\item Doing
      \begin{lcode}
      \renewcommand{\printXtitle}[1]{\hfill\Large\itshape #1}
      \end{lcode}
      will print the title right justified in a Large italic font.
\item For a Large bold centered title you can do
      \begin{lcode}
      \renewcommand{\printXtitle}[1]{\centering\Large\bfseries #1}
      \end{lcode}
\item Writing
      \begin{lcode}
      \renewcommand{\afterXtitle}{%
                        \thispagestyle{empty}\afterchaptertitle}
      \end{lcode}
      will result in the first page of the listing using the \pstyle{empty}
      pagestyle instead of the default \pstyle{chapter} pagestyle.
\item Doing
      \begin{lcode}
      \renewcommand{\afterXtitle}{%
        \par\nobreak \mbox{}\hfill{\normalfont Page}\par\nobreak}
      \end{lcode}
      will put the word `Page' flushright on the line following
      the title.
\end{itemize}


 \subsection{Typesetting the entries} \label{sec:entries}

 Commands are also provided to enable finer control over the typesetting
 of the different kinds of entries. The parameters defining the default
 layout of the entries are illustrated as part of the \Lpack{layouts}
 package~\cite{LAYOUTS} or in~\cite[p. 51]{COMPANION}, and are repeated in
 \fref{fig:ltoc}.

    Most of the commands in this section start as \cs{cft...}, where
\texttt{cft} is intended as a mnemonic for \textit{Table of \textbf{C}ontents, 
List of \textbf{F}igures, List of \textbf{T}ables}.

\begin{syntax}
 \cmd{\cftdot} \\
\end{syntax}
\glossary(cftdot)%
  {\cs{cftdot}}%
 {The `dot' in the dotted leaders in \listofx.}
  In the default \toc{} typesetting only the more minor entries have dotted
 leader lines between the sectioning title and the page number. The
 class provides for general leaders for all entries.
 The `dot' in a leader is given by the value of \cmd{\cftdot}. Its default
 definition is \verb?\newcommand{\cftdot}{.}? which gives the default
 dotted leader. By changing \cmd{\cftdot} you can use symbols other than
 a period in the leader. For example 
 \begin{lcode}
 \renewcommand{\cftdot}{\ensuremath{\ast}}
 \end{lcode}
 will result in a dotted leader using asterisks as the symbol.

\begin{syntax}
 \cmd{\cftdotsep} \\
 \cmd{\cftnodots} \\
\end{syntax}
\glossary(cftdotsep)%
  {\cs{cftdotsep}}%
  {The separation between dots in a dotted leader in a \listofx.}
\glossary(cftnodots)%
  {\cs{cftnodots}}%
  {A separation between dots in a dotted leader in a \listofx\
   that is too large for any dots to occur.}
    Each kind of entry can control the separation between the dots
 in its leader (see below). For consistency though, all dotted leaders
 should use the same spacing. The macro \cmd{\cftdotsep} specifies the
 default spacing. 
 However, if the separation is too large
 then no dots will be actually typeset. The macro \cmd{\cftnodots} is
 a separation value that is `too large'. 

\begin{syntax}
 \cmd{\setpnumwidth}\marg{length} \\
 \cmd{\setrmarg}\marg{length} \\
\end{syntax}
\glossary(setpnumwidth)%
  {\cs{setpnumwidth}\marg{length}}%
  {Sets the width of the page number box (\cs{@pnumwidth}) in a \listofx\ to 
   \meta{length}.}
\glossary(setrmarg)%
  {\cs{setrmarg}\marg{length}}%
  {Sets the right hand title margin (\cs{@tocrmarg}) in a \listofx\ to
   \meta{length}.}
 The page numbers are typeset in a fixed width box. The command
 \cmd{\setpnumwidth} can be used to change the width
 of the box (LaTeX 's internal \cmd{\@pnumwidth}). 
 The title texts will end before reaching the righthand 
margin\index{margin!right}.
 \cmd{\setrmarg} can be used to set this distance 
 (LaTeX 's internal \cmd{\@tocrmarg}).
 Note that the length used in \cmd{\setrmarg} should be greater
 than the length set in \cmd{\setpnumwidth}. These values should remain
 constant in any given document.

    This manual requires more space for the page numbers than the default,
so the following was set in the preamble\index{preamble}:
\begin{lcode}
\setpnumwidth{2.55em}
\setrmarg{3.55em}
\end{lcode}


\begin{syntax}
\lnc{\cftparskip} \\
\end{syntax}
\glossary(cftparskip)%
  {\cs{cftparskip}}%
  {The \cs{parskip} to be used in a \listofx.}
 Normally the \lnc{\parskip} in the \toc, etc., is zero. This may be changed
 by changing the length \lnc{\cftparskip}. Note that the current value
 of \lnc{\cftparskip} is used for the \toc, \lof\ and \lot, but you can change
 the value before calling \cmd{\tableofcontents} or \cmd{\listoffigures} or
 \cmd{\listoftables} if one or other of these should have different values
 (which is not a good idea).


    Again for convenience, in the following I will use \texttt{K} to stand 
for the \emph{kind} of entry, as listed in \tref{tab:Klistofxtitles}; that is, 
any of the following:
 \begin{itemize}
 \item \texttt{book} for \cmd{\book} titles.
 \item \texttt{part} for \cmd{\part} titles
 \item \texttt{chapter} for \cmd{\chapter} titles
 \item \texttt{section} for \cmd{\section} titles
 \item \texttt{subsection} for \cmd{\subsection} titles
 \item \texttt{subsubsection} for \cmd{\subsubsection} titles
 \item \texttt{paragraph} for \cmd{\paragraph} titles
 \item \texttt{subparagraph} for \cmd{\subparagraph} titles
 \item \texttt{figure} for figure \cmd{\caption} titles
 \item \texttt{subfigure} for subfigure \cmd{\caption} titles
 \item \texttt{table} for table \cmd{\caption} titles
 \item \texttt{subtable} for subtable \cmd{\caption} titles
 \end{itemize}

\begin{table}
\centering
\caption{Value of \texttt{K} in macros for styling entries in a \listofx}
\label{tab:Klistofxtitles}
\begin{tabular}{llcll} \toprule
\multicolumn{1}{c}{\texttt{K}} & \multicolumn{1}{c}{Kind of entry} & & 
\multicolumn{1}{c}{\texttt{K}} & \multicolumn{1}{c}{Kind of entry} \\ \midrule
\texttt{book} & \cmd{\book} title & & \texttt{subparagraph} & \cmd{\subparagraph} title \\
\texttt{part} & \cmd{\part} title & & \texttt{figure} & figure caption  \\
\texttt{chapter} & \cmd{\chapter} title & & \texttt{subfigure} & subfigure caption \\
\texttt{section} & \cmd{\section} title & & \texttt{table} & table caption \\
\texttt{subsection} & \cmd{\subsection} title & & \texttt{subtable} & subtable caption \\
\texttt{subsubsection} & \cmd{\subsubsection} title & & \texttt{\ldots} & \ldots  \\
\bottomrule
\end{tabular}
\end{table}


\begin{syntax}
\cmd{\cftbookbreak} \\
\cmd{\cftpartbreak} \\
\cmd{\cftchapterbreak} \\
\end{syntax}
\glossary(cftbookbreak)%
  {\cs{cftbookbreak}}%
  {Starts a \cs{book} entry in the \prtoc.}
\glossary(cftpartbreak)%
  {\cs{cftpartbreak}}%
  {Starts a \cs{part} entry in the \prtoc.}
\glossary(cftchapterbreak)%
  {\cs{cftchapterbreak}}%
  {Starts a \cs{chapter} entry in the \prtoc.}
When \cmd{\l@book} starts to typeset a \cmd{\book} entry in the
\toc{} the first thing it does is to call the macro \cmd{\cftbookbreak}.
This is defined as:
\begin{lcode}
\newcommand{\cftbookbreak}{\addpenalty{-\@highpenalty}}
\end{lcode}
which encourages a page break before rather than after the entry. As usual,
you can change \cmd{\cftbookbreak} to do other things that you feel might
be useful. The macros \cmd{\cftpartbreak} and \cmd{\cftchapterbreak} apply
to \cmd{\part} and \cmd{\chapter} entries, respectively, and have the same
default definitions as \cmd{\cftbookbreak}.

\begin{syntax}
\lnc{\cftbeforeKskip} \\
\end{syntax}
\glossary(cftbeforeKskip)%
  {\cs{cftbeforeKskip}}%
  {Generic vertical space before a `K' entry in a \listofx.}
 This length controls the vertical space before an entry. It can be changed
 by using \cmd{\setlength}. 

\begin{syntax}
\lnc{\cftKindent} \\
\end{syntax}
\glossary(cftKindent)%
  {\cs{cftKindent}}%
  {Generic indent of an `K' entry from the left margin in a \listofx.}
 This length controls the indentation of an entry from the left 
margin\index{margin!left} (\textit{indent} in \fref{fig:ltoc}). It
 can be changed using \cmd{\setlength}. 

\begin{syntax}
\lnc{\cftKnumwidth} \\
\end{syntax}
\glossary(cftKnumwidth)%
  {\cs{cftKnumwidth}}%
  {Generic space for the number of a `K' entry in a \listofx.}
 This length controls the space allowed for typesetting title numbers 
 (\textit{numwidth} in \fref{fig:ltoc}). It can
 be changed using \cmd{\setlength}. Second and subsequent lines of a multiline
 title will be indented by this amount.

 The remaining commands are related to the specifics of typesetting
 an entry.
 This is a simplified pseudo-code version for the typesetting of numbered 
 and unnumbered entries.
\LMnote{2012/07/29}{Added extra brace pair}
 \begin{lcode}
 {\cftKfont {{\cftKname \cftKpresnum SNUM\cftKaftersnum\hfil} \cftKaftersnumb TITLE}}
         {\cftKleader}{\cftKformatpnum{PAGE}}\cftKafterpnum\par

 {\cftKfont TITLE}{\cftKleader}{\cftKformatpnum{PAGE}}\cftKafterpnum\par
 \end{lcode}
 where \texttt{SNUM} is the section number, \texttt{TITLE} is the title text 
and \texttt{PAGE} 
 is the page number. In the numbered entry the pseudo-code
\begin{lcode}
{\cftKpresnum SNUM\cftKaftersnum\hfil}
\end{lcode}
 is typeset within a box of width \lnc{\cftKnumwidth}, see the
 \cs{...numberlinebox} macros later on.

\begin{syntax}
\cmd{\cftKfont} \\
\end{syntax}
\glossary(cftKfont)%
{\cs{cftKfont}}%
{Controls the appearance of the number and title of a `K' entry in a \listofx.}
This controls the appearance of the title (and its preceding number, 
if any). It may be changed using \cmd{\renewcommand}. 

\cmd{\cftKfont} takes no arguments as such, but the the number and
title is presented to it as an argument. Thus one may end
\cmd{\cftKfont} with a macro taking one argument, say
\cmd{\MakeUppercase}, and which then readjust the text as needed.

\begin{caveat}
  Please read the section entitled
  \emph{\titleref{sec:about-upper-or}} on
  page~\pageref{sec:about-upper-or} if you consider using upper/lower
  cased TOC entries and especially if you are also using the
  \Lpack{hyperref} package.
\end{caveat}


\begin{syntax}
\cmd{\cftKname} \\
\end{syntax}
\glossary(cftKname)
  {\cs{cftKname}}%
  {Called as the first element of the  `K' entry line in a \listofx.}
    The first element typeset in an entry is 
\cmd{\cftKname}.\footnote{Suggested by Danie\index{Els, Danie} Els.}
Its default definition is 
\begin{lcode}
\newcommand*{\cftKname}{}
\end{lcode}
so it does nothing. However, to put the word `Chapter' before each chapter 
number in a \toc\ and `Fig.' before each figure number in a \lof\ do:
\begin{lcode}
\renewcommand*{\cftchaptername}{Chapter\space}
\renewcommand*{\cftfigurename}{Fig.\space}
\end{lcode}

\begin{syntax}
\cmd{\cftKpresnum} \cmd{\cftKaftersnum} \cmd{\cftKaftersnumb} \\
\end{syntax}
\glossary(cftKpresnum)%
  {\cs{cftKpresnum}}%
  {Called immediately before typesetting the number of a `K' entry in a \listofx.} 
\glossary(cftKaftersnum)%
  {\cs{cftKaftersnum}}%
  {Called immediately after typesetting the number of a `K' entry in a \listofx.} 
\glossary(cftKaftersnumb)%
  {\cs{cftKaftersnumb}}%
  {Called immediately after typesetting the number box of a `K' entry in a \listofx.} 
 The section number is typeset within a box of width \lnc{\cftKnumwidth}.
 Within the box the macro \cmd{\cftKpresnum} is first called, then the
 number is typeset, and the \cmd{\cftKaftersnum}
 macro is called after the number is typeset. The last command
 within the box is \cmd{\hfil} to make the box contents flushleft.
 After the box is
 typeset the \cmd{\cftKaftersnumb} macro is called before typesetting
 the title text. All three of these can be changed by \cmd{\renewcommand}.
 By default they are defined to do nothing.


\begin{syntax}
\cmd{\numberline}\marg{num} \\
\cmd{\partnumberline}\marg{num} \\
\cmd{\partnumberline}\marg{num} \\
\cmd{\chapternumberline}\marg{num} \\
\end{syntax}
\glossary(numberline)
  {\cs{numberline}\marg{num}}
  {Typeset sectional number for \cs{section} and siblings in ToC}
\glossary(partnumberline)
  {\cs{partnumberline}\marg{num}}
  {Typeset part number in ToC}
\glossary(booknumberline)
  {\cs{booknumberline}\marg{num}}
  {Typeset book number in ToC}
\glossary(chapternumberline)
  {\cs{chapternumberline}\marg{num}}
  {Typeset chapter number in ToC}
In the \toc, the macros \cmd{\booknumberline}, \cmd{\partnumberline} and 
\cmd{\chapternumberline}
are responsible respectively for typesetting the \cmd{\book}, \cmd{\part} 
and \cmd{\chapter}
numbers, whereas \cmd{\numberline} does the same for the sectional
siblings. Internally they use \cmd{\cftKpresnum}, \cmd{\cftKaftersnum}
and \cmd{\cftKaftersnumb} as above. If you do not want, say, 
the \cmd{\chapter} number to appear you
can do:
\begin{lcode}
\renewcommand{\chapternumberline}[1]{}
\end{lcode}


\begin{syntax}
  \cmd{\numberlinehook}\marg{num}\\
  \cmd{\cftwhatismyname}\\
  \cmd{\booknumberlinehook}\marg{num}\\
  \cmd{\partnumberlinehook}\marg{num}\\
  \cmd{\chapternumberlinehook}\marg{num}\\
  \cmd{\numberlinebox}\marg{length}\marg{code}\\
  \cmd{\booknumberlinebox}\marg{length}\marg{code}\\
  \cmd{\partnumberlinebox}\marg{length}\marg{code}\\
  \cmd{\chapternumberlinebox}\marg{length}\marg{code}\\
\end{syntax}
\glossary(numberlinehook)
 {\cs{numberlinehook}\marg{num}}
 {The first thing to be called within \cs{numberline}, does nothing by
 default.}
\glossary(cftwhatismyname)
 {\cs{cftwhatismyname}}
 {Since \cs{numberline} is shared by \cs{section} and siblings, 
   \cs{cftwhatismyname} can be used to tell which section type is
   calling \cs{numberlinehook}}
\glossary(partnumberlinehook)
 {\cs{partnumberlinehook}\marg{num}}
 {The first thing to be called within \cs{partnumberline}, does nothing by
 default.}
\glossary(booknumberlinehook)
 {\cs{booknumberlinehook}\marg{num}}
 {The first thing to be called within \cs{booknumberline}, does nothing by
 default.}
\glossary(chapternumberlinehook)
 {\cs{chapternumberlinehook}\marg{num}}
 {The first thing to be called within \cs{chapternumberline}, does nothing by
 default.}
\glossary(numberlinebox)
 {\cs{numberlinebox}\marg{length}\marg{code}}
 {The box command used to typeset the sectional number within the ToC,
   note that it will automatically align to the left}
\glossary(partnumberlinebox)
 {\cs{partnumberlinebox}\marg{length}\marg{code}}
 {The box command used to typeset the part number within the ToC, 
   note that it will automatically align to the left}
\glossary(booknumberlinebox)
 {\cs{booknumberlinebox}\marg{length}\marg{code}}
 {The box command used to typeset the book number within the ToC, 
   note that it will automatically align to the left}
\glossary(chapternumberlinebox)
 {\cs{chapternumberlinebox}\marg{length}\marg{code}}
 {The box command used to typeset the chapter number within the ToC, 
   note that it will automatically align to the left}
Inside the four \cs{...numberline} macros, the first thing we do is to
give the \cs{...numberline} argument to a hook. By default this hook
does nothing. But, with the right
tools,\footnote{Which we do not currently supply\dots, but have a look
at Sniplet~\ref{snip:autotoc} on page~\pageref{snip:autotoc}.}
they can be used to record the widths of the sectional number. Which then
can be used to automatically adjust the various \meta{numwidth} and
\meta{indent} within the \cs{cftsetindents} macro. In order to 
tell the section types apart (they all use \cmd{\numberline}), the
value of the \cmd{\cftwhatismyname} macro will locally reflect the
current type. 

As mentioned earlier, the \cmd{\book}, \cmd{\part} and \cmd{\chapter}
numbers are typeset inside a box of certain fixed widths. Sometimes it
can be handy \emph{not} having this box around. For this you can
redefine one of the four \cs{...numberlinebox} macros listed
above. For example via
\begin{lcode}
  \renewcommand\chapternumberlinebox[2]{#2}
\end{lcode}
The first argument is the width of the box to be made.  All four
macros are defined similar to this (where \texttt{\#1} is a length)
\begin{lcode}
\newcommand\chapternumberlinebox[2]{%
  \hb@xt@#1{#2\hfil}}
\end{lcode}


\begin{comment}
\Note{}  Because the \Lpack{hyperref} package~\cite{HYPERREF} 
does not understand
the \cmd{\partnumberline} and \cmd{\chapternumberline} commands,
if you use the \Lpack{hyperref} package you will also have to use
the \Lpack{memhfixc} package, which comes with memoir.
\end{comment}


\begin{syntax}
\cmd{\cftKleader} \\
\cmd{\cftKdotsep} \\
\end{syntax}
\glossary(cftKleader)
  {\cs{cftKleader}}%
  {Leader between the title and page number of a `K' entry in a \listofx.} 
\glossary(cftKdotsep)
  {\cs{cftKdotsep}}%
  {Separation between dots in a leader between the title and page number of a `K' entry in a \listofx.} 
 \cmd{\cftKleader} defines the leader between the title and the page number;
 it can be changed by \cmd{\renewcommand}.
 The spacing between any dots in the leader is controlled by \cmd{\cftKdotsep} 
 (\cmd{\@dotsep} in \fref{fig:ltoc}).
 It can be changed by \cmd{\renewcommand} and its value must be either a
 number (e.g., 6.6 or \cmd{\cftdotsep}) or \cmd{\cftnodots} (to
 disable the dots). The spacing is in terms of \emph{math units} where
 there are 18mu to 1em. 

The default leaders macro is similar to
\begin{lcode}
  \newcommand{\cftsectionleader}{\normalfont\cftdotfill{\cftsectiondotsep}}
\end{lcode}
Note that the spacing of the dots is affected by the font size (as the
math unit is affected by the font size). Also note that the
\cmd{\cftchapterleader} is bold by default.

\begin{syntax}
\cmd{\cftKformatpnum}\marg{pnum} \\
\cmd{\cftKformatpnumhook}\marg{num}\\
\cmd{\cftKpagefont} \\
\end{syntax}
\glossary(cftKformatpnum)%
  {\cs{cftKformatpnum}\marg{pnum}}%
  {Typesets the page number \meta{pnum} of a  `K' entry in a \listofx.} 
\glossary(cftKpagefont)%
  {\cs{cftKpagefont}}%
  {Font for the page number of a  `K' entry in a \listofx.} 
\glossary(cftKformatpnumhook)%
 {\cs{cftKformatpnumhook}\marg{num}}%
 {When formatting the page number in the ToC (via
   \cs{cftKformatpnum}) this hook is given the page value. Does nothing by default}%
The macro \cmd{\cftKformatpnum} typesets an entry's page number, using
the \cmd{\cftKpagefont}.\footnote{This addition to the class was suggested
by Dan\index{Luecking, Daniel} Luecking, \ctt\ \textit{Re: setting numbers in toc in their natural width box,} 2007/08/15.}
The default definition is essentially:
\begin{lcode}
\newcommand*{\cftKformatpnum}[1]{%
  \cftKformatpnumhook{#1}%
  \hbox to \@pnumwidth{\hfil{\cftKpagefont #1}}}
\end{lcode}
which sets the number right justified in a box \lnc{\@pnumwidth} wide.
To have, say, a \cmd{\part} page number left justified in its box, do:
\begin{lcode}
\renewcommand*{\cftpartformatpnum}[1]{%
  \cftpartformatpnumhook{#1}%
  \hbox to \@pnumwidth{{\cftpartpagefont #1}}}
\end{lcode}
The \cmd{\cftKformatpnumhook} does nothing by default (other than
eating the argument), but could be redefined to record the widest page
number and report it back, even reusing it to auto adjust on the next
run to set \cs{@pnumwidth} (see \cmd{\setpnumwidth}).


\begin{syntax}
\cmd{\cftKafterpnum} \\
\end{syntax}
\glossary(cftKafterpnum)
  {\cs{cftKafterpnum}}%
  {Called after typesetting the page number of a  `K' entry in a \listofx.} 
 This macro is called after the page number has been typeset. Its default
 is to do nothing. It can be changed by \cmd{\renewcommand}.

\begin{syntax}
\cmd{\cftsetindents}\marg{kind}\marg{indent}\marg{numwidth} \\
\end{syntax}
\glossary(cftsetindents)%
  {\cs{cftsetindents}\marg{kind}\marg{indent}\marg{numwidth}}%
  {Set the \meta{kind} entry \textit{indent} to \meta{indent} and
   its \textit{numwidth} to \meta{numwidth}.}
 The command 
 \cmd{\cftsetindents} sets the \meta{kind} entries \textit{indent} to the 
length \meta{indent} and its
 \textit{numwidth} to the length \meta{numwidth}. The \meta{kind} argument
 is the name of one of the standard entries (e.g., \texttt{subsection}) or the 
name of entry that has been defined within the document.
 For example 
\begin{lcode}
 \cftsetindents{figure}{0em}{1.5em}
\end{lcode}
 will make figure\index{figure} entries left justified.

    This manual requires more space for section numbers in the \toc{} than
the default (which allows for three digits). Consequently the preamble\index{preamble}
contains the following:
\begin{lcode}
\cftsetindents{section}{1.5em}{3.0em}
\cftsetindents{subsection}{4.5em}{3.9em}
\cftsetindents{subsubsection}{8.4em}{4.8em}
\cftsetindents{paragraph}{10.7em}{5.7em}
\cftsetindents{subparagraph}{12.7em}{6.7em}
\end{lcode}
Note that changing the indents at one level implies that any lower level
indents should be changed as well.


 Various effects can be achieved by changing the definitions of \cmd{\cftKfont},
 \cmd{\cftKaftersnum}, \cmd{\cftKaftersnumb}, \cmd{\cftKleader} and 
\cmd{\cftKafterpnum}, 
 either singly or in combination.
 For the sake of some examples, assume that we have the following initial
 definitions
 \begin{lcode}
 \newcommand*{\cftKfont}{}
 \newcommand*{\cftKaftersnum}{}
 \newcommand*{\cftKaftersnumb}{}
 \newcommand*{\cftKleader}{\cftdotfill{\cftKdotsep}}
 \newcommand*{\cftKdotsep}{\cftdotsep}
 \newcommand*{\cftKpagefont}{}
 \newcommand*{\cftKafterpnum}{}
 \end{lcode}
Note that the same font should be used for the title, leader and page 
 number to provide a coherent appearance.

 \begin{itemize}
 \item To eliminate the dots in the leader:
 \begin{lcode}
 \renewcommand*{\cftKdotsep}{\cftnodots}
 \end{lcode}

 \item To put something (e.g., a name) before the title (number):
 \begin{lcode}
 \renewcommand*{\cftKname}{SOMETHING }
 \end{lcode}

 \item To add a colon after the section number:
 \begin{lcode}
 \renewcommand*{\cftKaftersnum}{:}
 \end{lcode}

 \item To put something before the title number, add a double colon after 
    the title number, set everything in bold font,
 and start the title text on the following line:
 \begin{lcode}
 \renewcommand*{\cftKfont}{\bfseries}
 \renewcommand*{\cftKleader}{\bfseries\cftdotfill{\cftKdotsep}}
 \renewcommand*{\cftKpagefont}{\bfseries}
 \renewcommand*{\cftKname}{SOMETHING }
 \renewcommand{\cftKaftersnum}{::}
 \renewcommand{\cftKaftersnumb}{\\}
 \end{lcode}

    If you are adding text in the number box in addition to the number,
 then you will probably have to increase the width of the box so that
 multiline titles have a neat vertical alignment; changing box widths
 usually implies that the indents will require modification as 
 well. One possible method of adjusting the box width for the above example
 is:
 \begin{lcode}
 \newlength{\mylen}                  % a "scratch" length
 \settowidth{\mylen}{\bfseries\cftKaftersnum}
 \addtolength{\cftKnumwidth}{\mylen} % add the extra space
 \end{lcode} 

\LMnote{2012/07/29}{Added this example}
\item To set the chapter number and title as just
  `NUM\enspace\textperiodcentered\enspace TITLE', i.e. un-boxed number
  plus a symbolic separator, use
  \begin{lcode}
  \renewcommand\cftchapteraftersnumb{\enspace\textperiodcentered\enspace}
  \renewcommand\chapternumberlinebox[2]{#2}  
  \end{lcode}
  -- of couse, it works best, only if the TITLE is a single line.


\item Make chapter titles lower case small caps
  \begin{lcode}
    \renewcommand\cftchapterfont{\scshape\MakeTextLowercase}
  \end{lcode}
  -- here we do not touch the case of any math.

 \item To set the section numbers flushright:
 \begin{lcode}
 \setlength{\mylen}{0.5em}    % extra space at end of number
 \renewcommand{\cftKpresnum}{\hfill} % note the double `l'
 \renewcommand{\cftKaftersnum}{\hspace*{\mylen}}
 \addtolength{\cftKnumwidth}{\mylen}
 \end{lcode}
 In the above, the added initial \cmd{\hfill} in the box overrides the
 final \cmd{\hfil} in the box, thus shifting everything to the right hand
 end of the box. The extra space is so that the number is not typeset
 immediately at the left of the title text.

 \item To set the entry ragged left (but this only looks good for single
       line titles):
 \begin{lcode}
 \renewcommand{\cftKfont}{\hfill\bfseries}
 \renewcommand{\cftKleader}{}
 \end{lcode}

\item To set the titles ragged right instead of the usual flushright.
      Assuming that there are more than 100 pages in the document (otherwise
      adjust the length):
\begin{lcode}
\setrmarg{3.55em plus 1fil}
\end{lcode}
where the last four characters before the closing brace are: digit 1, 
lowercase F, lowercase I, and lowercase L.

 \item To set the page number immediately after the entry text instead of at
       the righthand margin\index{margin!right}:
 \begin{lcode}
 \renewcommand{\cftKleader}{}
 \renewcommand{\cftKafterpnum}{\cftparfillskip}
 \end{lcode}

\end{itemize}

\begin{syntax}
\cmd{\cftparfillskip} \\
\end{syntax}
\glossary(cftparfillskip)%
  {\cs{cftparfillskip}}%
  {Fills the last line in a paragraph in a \listofx.}
 By default the \cmd{\parfillskip} value is locally set to fill up the last
 line of a paragraph\index{paragraph}. Just changing \cmd{\cftKleader} as
in the above example puts horrible interword
 spaces into the last line of the title. The \cmd{\cftparfillskip} 
 command  is provided just so that the above effect can be achieved.

\begin{syntax}
\cmd{\cftpagenumbersoff}\marg{kind} \\
\cmd{\cftpagenumberson}\marg{kind} \\
\end{syntax}
\glossary(cftpagenumbersoff)%
  {\cs{cftpagenumbersoff}\marg{kind}}%
  {Eliminates page numbers for the \meta{kind} entries in a \listofx.}
\glossary(cftpagenumberson)%
  {\cs{cftpagenumberson}\marg{kind}}%
  {Reverses the effect of \cs{cftpagenumbersoff}.}
 The command \cmd{\cftpagenumbersoff} will
 eliminate the page numbers for \meta{kind} entries in the listing, where
 \meta{kind} is the name of one of the standard
 kinds of entries (e.g., \texttt{subsection}, or \texttt{figure}) or the 
 name of a new entry defined in the document.

    The command \cmd{\cftpagenumberson} reverses
 the effect of a corresponding \cmd{\cftpagenumbersoff} for \meta{kind}.
 
    For example, to eliminate page numbers for appendices\index{appendix} 
in the \toc:
 \begin{lcode}
 ...
 \appendix
 \addtocontents{toc}{\cftpagenumbersoff{chapter}}
 \chapter{First appendix}
 \end{lcode}
 If there are other chapter type headings\index{heading!chapter} to go 
into the \toc{} after the  appendices\index{appendix} (perhaps a 
bibliography\index{bibliography} or an index\index{index}), 
 then it will be necessary to do a similar 
 \begin{lcode}
 \addtocontents{toc}{\cftpagenumberson{chapter}}
 \end{lcode}
 after the appendices\index{appendix} to restore the page numbering in 
the \toc.

 Sometimes it may be desirable to make a change to the global parameters
 for an individual entry. For example, a figure\index{figure} might be 
placed on the end paper\index{paper!end} of a book (the inside of the front 
or back cover), and this needs to be placed in a \lof\ with the page number 
set as, say,  `inside front cover'. If `inside front cover' is typeset as 
an ordinary page number it will stick out into the margin\index{margin}. 
Therefore, the parameters for this particular entry need to be changed.

\begin{syntax}
\cmd{\cftlocalchange}\marg{ext}\marg{pnumwidth}\marg{tocrmarg} \\
\end{syntax}
\glossary(cftlocalchange)%
  {\cs{cftlocalchange}\marg{ext}\marg{pnumwidth}\marg{tocrmarg}}%
  {Writes commands to the \meta{ext} \listofx\ file resetting
   \cs{@pnumwidth} and \cs{@tocrmarg} to the specified values.}
 The command \cmd{\cftlocalchange} 
 will write an entry into the file with extension \meta{ext} to reset 
the global \cmd{\@pnumwidth} and \cmd{\@tocrmarg} parameter lengths. 
 The command should be called again after any special entry to reset
 the parameters back to their usual values. Any fragile commands used
 in the arguments must be protected.

\begin{syntax}
\cmd{\cftaddtitleline}\marg{ext}\marg{kind}\marg{title}\marg{page} \\
\cmd{\cftaddnumtitleline}\marg{ext}\marg{kind}\marg{num}\marg{title}\marg{page} \\
\end{syntax}
\glossary(cftaddtitleline)%
  {\cs{cftaddtitleline}\marg{ext}\marg{kind}\marg{title}\marg{page}}%
  {Writes a \cs{contentsline} to the \listofx\ \meta{ext} file for
   a \meta{kind} entry with \meta{title} and \meta{page} number.}
\glossary(cftaddnumtitleline)%
  {\cs{cftaddnumtitleline}\marg{ext}\marg{kind}\marg{num}\marg{title}\marg{page}}%
  {Writes a \cs{contentsline} to the \listofx\ \meta{ext} file for
   a \meta{kind} entry with number \meta{number} and \meta{title} and 
   \meta{page} number.}
 The command \cmd{\cftaddtitleline} 
 will write a \cmd{\contentsline} entry into \meta{ext} for a \meta{kind}
 entry with title \meta{title} and page number \meta{page}. 
 Any fragile commands used in the arguments must be protected.
That is,
 an entry is made of the form: 
\begin{lcode}
\contentsline{kind}{title}{page}
\end{lcode}
 The command \cmd{\cftaddnumtitleline}
 is similar to \cmd{\cftaddtitleline} except that it also includes 
\meta{num} as the argument to
 \cmd{\numberline}. That is, an entry is made of the form
\begin{lcode}
\contentsline{kind}{\numberline{num} title}{page}
\end{lcode}

 As an example of the use of these commands, 
 noting that the default LaTeX values for 
 \cmd{\@pnumwidth} and \cmd{\@tocrmarg} are 1.55em and 2.55em respectively, 
 one might do the following for a figure\index{figure} on the 
frontispiece\index{frontispiece} page.
 \begin{lcode}
 ...
  this is the frontispiece page with no number
  draw or import the picture (with no \caption)
 \cftlocalchange{lof}{4em}{5em} % make pnumwidth big enough for 
                                % frontispiece and change margin
 \cftaddtitleline{lof}{figure}{The title}{frontispiece}
 \cftlocalchange{lof}{1.55em}{2.55em} % return to normal settings
 \clearpage
 ...
 \end{lcode}
    Recall that a \cmd{\caption} command will put an entry in the \file{lof}
file, which is not wanted here. If a caption\index{caption} is required, 
then you can  either craft one youself or, assuming that your general 
captions\index{caption} are not too exotic, use the \cmd{\legend} command 
(see later). If the illustration\index{illustration} is numbered, use 
\cmd{\cftaddnumtitleline} instead of \cmd{\cftaddtitleline}.


\LMnote{2010/06/09}{Thought we needed a break here}
\subsubsection{Inserting stuff into the content lists}
\label{sec:inserting-stuff-into}


The next functions were suggested by Lars\index{Madsen, Lars} Madsen who
found them useful if, for example, you had two versions of the
\toc\ and you needed some aspects to be formatted differently.
\begin{syntax}
\cmd{\cftinsertcode}\marg{name}\marg{code} \\
\cmd{\cftinserthook}\marg{file}\marg{name} \\
\end{syntax}
\glossary(cftinsertcode)
  {\cs{cftinsertcode}\marg{name}\marg{code}}%
  {Defines Toc (LoF, LoT) \meta{name} insertion to be \meta{code}.}
\glossary(cftinserthook)
  {\cs{cftinserthook}\marg{file}\marg{name}}%
  {Inserts code \meta{name} into \meta{file} (e.g., \texttt{toc}, \texttt{lof}, etc.}
The \cmd{\cftinserthook} is somewhat like \cmd{\addtocontents} in that it 
enables you to insert a code hook into the \toc, etc., where \meta{file} is 
the (\texttt{toc}, \texttt{lof}, \ldots) file and \meta{name} is the `name'
of the hook. The \meta{code} for the hook is specified via \cmd{\cftinsertcode}
where \meta{name} is the name you give to the hook. These can be used to make
alterations to a \listofx\ on the fly. For example:
\begin{lcode}
\cftinsertcode{A}{%
  \renewcommand*{\cftchapterfont}{\normalfont\scshape}
  ... }% code for ToC
...
\frontmatter
\tableofcontents
\cftinsertcode{G}{...}% code for LoF
\cftinsertcode{F}{...}% code for LoF
\listoffigures
...
\cftinserthook{lof}{G}
...
\chapter{...}
...
\mainmatter
\cftinserthook{toc}{A}
\cftinserthook{lof}{F}
\chapter{...}
...
\end{lcode}
If you do not use \cmd{\cftinsertcode} \emph{before} calling the command to
type the \listofx\ that it is intended for then nothing will happen. No
harm will come if a matching \cmd{\cftinserthook} is never used. No harm
occurs either if you call \cmd{\cftinserthook} and there is no prior
matching \cmd{\cftinsertcode}.

One use of these \toc{} hooks is reusing the \toc{} data to, say,
create chapter \toc's. The code for this is shown in
Sniplet~\ref{snip:chaptertoc} on page~\pageref{snip:chaptertoc}. In
the sniplet we use the following two hooks that are executed right
before and right after \cmd{\chapter}, \cmd{\part}, \cmd{\book},
\cmd{\appendixpage}  writes to the \toc{}. By default
they do nothing.\footnote{More hooks may be added in later releases.}
\begin{syntax}
  \cmd{\mempreaddchaptertotochook}\\
  \cmd{\mempostaddchaptertotochook}\\
  \cmd{\mempreaddparttotochook}\\
  \cmd{\mempostaddparttotochook}\\
  \cmd{\mempreaddbooktotochook}\\
  \cmd{\mempostaddbooktotochook}\\
  \cmd{\mempreaddapppagetotochook}\\
  \cmd{\mempostaddapppagetotochook}
\end{syntax}
\glossary(mempreaddchaptertotochook)%
  {\cs{mempreaddchaptertotochook}}%
  {Hook executed right \emph{before} \cs{chapter} writes to the \protect\toc{}}
\glossary(mempostaddchaptertotochook)%
  {\cs{mempostaddchaptertotochook}}%
  {Hook executed right \emph{after} \cs{chapter} writes to the \protect\toc{}}
\glossary(mempreaddparttotochook)%
  {\cs{mempreaddparttotochook}}%
  {Hook executed right \emph{before} \cs{part} writes to the \protect\toc{}}
\glossary(mempostaddparttotochook)%
  {\cs{mempostaddparttotochook}}%
  {Hook executed right \emph{after} \cs{part} writes to the \protect\toc{}}
\glossary(mempreaddbooktotochook)%
  {\cs{mempreaddbooktotochook}}%
  {Hook executed right \emph{before} \cs{book} writes to the \protect\toc{}}
\glossary(mempostaddbooktotochook)%
  {\cs{mempostaddbooktotochook}}%
  {Hook executed right \emph{after} \cs{book} writes to the \protect\toc{}}
\glossary(mempreaddapppagetotochook)%
  {\cs{mempreaddapppagetotochook}}%
  {Hook executed right \emph{before} \cs{appendixpage} writes to the \protect\toc{}}
\glossary(mempostaddapppagetotochook)%
  {\cs{mempostaddapppagetotochook}}%
  {Hook executed right \emph{after} \cs{appendixpage} writes to the \protect\toc{}}



\LMnote{2010/06/09}{Ineffective to have this seceral places in the manual.}
\subsubsection{Extra chapter material in the ToC}
\label{sec:extra-chapt-mater}

\begin{syntax}
\cmd{\precistoctext}\marg{text} \cmd{\precistocfont} \cmd{\precistocformat}\\
\end{syntax}
The \cmd{\chapterprecistoc} macro puts  \cmd{\precistoctext}\marg{text} into 
the \pixfile{toc} file. Further information as to the definition of
this macro can be found in section~\ref{sec:chapter-precis}.


\LMnote{2012/09/21}{added}
\subsubsection{About upper or lower casing TOC entries}
\label{sec:about-upper-or}

Some designs call for upper (or lower casing) TOC entries. This
\emph{is} possible but the solution depends on whether the
\Lpack{hyperref} package is used or not.

Without \Lpack{hyperref} one can simply end the \cs{cftKfont} with say
\cs{MakeTextUppercase} and the \texttt{K}-type entry will be upper cased.

With \Lpack{hyperref} the possibilities are limited. Explanation: The
upper/lower casing macros are not that robust, and need the content to
be simple.\footnote{For some definition of simple.} When
\Lpack{hyperref} is used, the hyperlink is wrapped around the entry
before \cs{cftKfont} gains access to it, and is thus generally too
complicated for, say, \cs{MakeTextUppercase} to handle. The follow
workaround draw inspiration from
\url{http://tex.stackexchange.com/q/11892/3929}.
\begin{syntax}
  \cmd{\settocpreprocessor}\marg{type}\marg{code}
\end{syntax}
\glossary(settocpreprocessor)%
  {\cs{settocpreprocessor}\marg{type}\marg{code}}
  {Provide a method for preprocessing certain TOC entries before they
    are written to the \texttt{.toc} file.}
Here \meta{type} is one of \texttt{chapter}, \texttt{part} or
\texttt{book}.\footnote{If needed we will attempt to add a similar
  feature to the rest of the sectional types, please write the
  maintainer.} And \meta{code} can be something like this example:
\begin{verbatim}
\makeatletter
\settocpreprocessor{chapter}{%
  \let\tempf@rtoc\f@rtoc%
  \def\f@rtoc{%
    \texorpdfstring{\MakeTextUppercase{\tempf@rtoc}}{\tempf@rtoc}}%
}
\makeatother
\end{verbatim}
Where \cs{f@rtoc} is a placeholder inside \cmd{\chapter}, \cmd{\part}
and \cmd{\book}, holding the material to be written to the actual TOC
file before \Lpack{hyperref} accesses it. This way the upper casing is
sneaked into the TOC file, and the bookmark part of \Lpack{hyperref}
will not complain about the \cmd{\MakeTextUppercase} in the data. Of
course, you will not have upper cased titles in the bookmark list.





%%%%%%%%%%%%%%%%%%%%%%%%%%%%%%%%%%%%%%%%%%%%%%%%%%%%%%%%%%%%%%%%%%%


\subsection{Example: No section number}

    There are at least two ways of listing section titles in the \toc\
without displaying their numbers and both involve the \cmd{\numberline}
command which typesets the number in a box. 


    The first method redefines \cmd{\numberline} so it throws away the
argument. We do this by modifying the \cmd{\cftKfont} macro which is called
before \cmd{\numberline} and the \cmd{\cftKafterpnum} which is called after
the page number has been typeset.
\begin{lcode}
\let\oldcftsf\cftsectionfont%  save definition of \cftsectionfont
\let\oldcftspn\cftsectionafterpnum% and of \cftsectionafterpnum
\renewcommand*{\cftsectionfont}{%
  \let\oldnl\numberline%       save definition of \numberline
  \renewcommand*{\numberline}[1]{}% change it
  \oldcftsf}                 % use original \cftsectionfont
\renewcommand*{\cftsectionafterpnum}{%
  \let\numberline\oldnl%     % restore orginal \numberline
  \oldcftspn}                % use original \cftsectionafterpnum
\end{lcode}

    Probing a little deeper, the \cmd{\numberline} macro is called to
typeset section numbers and is defined as:
\begin{lcode}
\renewcommand*{\numberline}[1]{%
  \hb@xt@\@tempdima{\@cftbsnum #1\@cftasnum\hfil}\@cftasnumb}
\end{lcode}
Each
kind of heading \cmd{\let}s the \cmd{\@cftbsnum} macro to \cmd{\cftKpresnum},
and the \cmd{\@cftasnum} macro to \cmd{\cftKaftersnum}, and the
\cmd{\@cftasnumb} macro to \cmd{\cftKaftersnumb} as appropriate for the
heading. The second method for killing the number uses a \tx\ method 
for defining a macro with
a delimited argument.
\begin{lcode}
\def\cftsectionpresnum #1\@cftasnum{}
\end{lcode}
The interpretation of this is left as an exercise for anyone who might 
be interested.

\subsection{Example: Multicolumn entries}

    If the subsection entries, say, in the \toc\ are going to be very 
short it might be worth setting them in multiple columns. Here is one way
of doing that which depends on using the \Lpack{multicol}
package~\cite{MULTICOL}. This assumes that subsections will be the lowest
heading in the \toc.
\begin{lcode}
\newcounter{toccols}
\setcounter{toccols}{3}
\newenvironment{mysection}[1]{%
  \section{#1}%
  \addtocontents{toc}{\protect\begin{multicols}{\value{toccols}}}}%
  {\addtocontents{toc}{\protect\end{multocols}}}
\end{lcode}

The counter \texttt{toccols} controls the number of columns to be used.
For each section where you want subsections to be typeset in multiple columns
in the \toc, use the \texttt{mysection} environment instead of \cmd{\section},
like:
\begin{lcode}
\begin{mysection}{Columns}
...
\subsection{Fat}
...
\subsection{Thin}
...
\end{mysection}
\end{lcode}

Any \toc\ entries generated from within the environment will be enclosed
in a \Ie{multicols} environment in the \toc. The \cmd{\protect}s have to be
used because environment \cmd{\begin} and \cmd{\end} commands are 
fragile\index{fragile}.
  

\subsection{Example: Multiple contents}

\PWnote{2009/03/17}{Added bit about simple short \& long ToC.}
    It is easy to have two \toc s, one short and one long, when they are
of the same style, like this:
\begin{lcode}
...
\renewcommand*{\contentsname}{Short contents}
\setcounter{tocdepth}{0}%  chapters and above
\tableofcontents
% \clearpage
\renewcommand*{\contentsname}{Contents}
\setcounter{tocdepth}{2}%  subsections and above
\tableofcontents
\end{lcode}
(Note that you can't use \cmd{\settocdepth} in this case as that writes the 
change into the \toc, so that the second use would override the first.)


    This book has both a short and a long \toc, neither of which look like
those typically associated with \ltx. This is how they were done.

    The general style for the \toc, etc., is specified in the \Lpack{memsty}
package file.

\begin{lcode}
%%% need more space for ToC page numbers
\setpnumwidth{2.55em}
\setrmarg{3.55em}
%%% need more space for ToC section numbers
\cftsetindents{section}{1.5em}{3.0em}
\cftsetindents{subsection}{4.5em}{3.9em}
\cftsetindents{subsubsection}{8.4em}{4.8em}
\cftsetindents{paragraph}{10.7em}{5.7em}
\cftsetindents{subparagraph}{12.7em}{6.7em}
%%% need more space for LoF & LoT numbers
\cftsetindents{figure}{0em}{3.0em}
\cftsetindents{table}{0em}{3.0em}
%%% remove the dotted leaders
\renewcommand{\cftsectiondotsep}{\cftnodots}
\renewcommand{\cftsubsectiondotsep}{\cftnodots}
\renewcommand{\cftsubsubsectiondotsep}{\cftnodots}
\renewcommand{\cftparagraphdotsep}{\cftnodots}
\renewcommand{\cftsubparagraphdotsep}{\cftnodots}
\renewcommand{\cftfiguredotsep}{\cftnodots}
\renewcommand{\cfttabledotsep}{\cftnodots}
\end{lcode}

Three macros are defined to control the appearance of the short and 
the long \toc. First, the macro \cmd{\setupshorttoc} for the short 
version.
The first few lines ensure that only chapter or part titles will be set,
and any chapter precis text or \Icn{tocdepth} changes will be ignored.
The rest of the code specifies how the chapter titles are to be typeset,
and finally the part and book titles.
\begin{lcode}
\newcommand*{\setupshorttoc}{%
  \renewcommand*{\contentsname}{Short contents}
  \let\oldchangetocdepth\changetocdepth
  \renewcommand*{\changetocdepth}[1]{}
  \let\oldprecistoctext\precistoctext
  \renewcommand{\precistoctext}[1]{}
  \let\oldcftchapterfillnum\cftchapterfillnum
  \setcounter{tocdepth}{0}% chapters and above
  \renewcommand*{\cftchapterfont}{\hfill\sffamily}
  \renewcommand*{\cftchapterleader}{ \textperiodcentered\space}
  \renewcommand*{\cftchapterafterpnum}{\cftparfillskip}
%%  \setpnumwidth{0em}
%%  \setpnumwidth{1.5em}
  \renewcommand*{\cftchapterfillnum}[1]{%
    {\cftchapterleader}\nobreak
    \hbox to 1.5em{\cftchapterpagefont ##1\hfil}\cftchapterafterpnum\par}
  \setrmarg{0.3\textwidth}
  \setlength{\unitlength}{\@tocrmarg}
  \addtolength{\unitlength}{1.5em}
  \let\oldcftpartformatpnum\cftpartformatpnum
  \renewcommand*{\cftpartformatpnum}[1]{%
    \hbox to\unitlength{{\cftpartpagefont ##1}}}}
  \let\oldcftbookformatpnum\cftbookformatpnum
  \renewcommand*{\cftbookformatpnum}[1]{%
    \hbox to\unitlength{{\cftbookpagefont ##1}}}}
\end{lcode}

    You can do many things using the \cs{cft...} macros to change the 
appearance of a \toc\ but they can't be entirely coerced into specifying
the paragraphing of the \cmd{\subsection} titles. The
\cmd{\setupparasubsecs} also went in the preamble.
\begin{lcode}
\newcommand*{\setupparasubsecs}{%
  \let\oldnumberline\numberline
  \renewcommand*{\cftsubsectionfont}{\itshape}
  \renewcommand*{\cftsubsectionpagefont}{\itshape}
  \renewcommand{\l@subsection}[2]{%
    \def\numberline####1{\textit{####1}~}%
    \leftskip=\cftsubsectionindent
    \rightskip=\@tocrmarg
%% \advance\rightskip 0pt plus \hsize % uncomment this for raggedright
%% \advance\rightskip 0pt plus 2em    % uncomment this for semi-raggedright
    \parfillskip=\fill
    \ifhmode ,\ \else\noindent\fi
    \ignorespaces 
    {\cftsubsectionfont ##1}~{\cftsubsectionpagefont##2}%
    \let\numberline\oldnumberline\ignorespaces}
}
\AtEndDocument{\addtocontents{toc}{\par}
\end{lcode}
The above code changes the appearance of subsection titles in the \toc, 
setting each group as a single paragraph (each is normally set with 
a paragraph to itself). By uncommenting or commenting the noted lines 
in the code you can change the layout a little. 


\begin{caveat}
  We have an interesting caveat regarding \cmd{\setupparasubsecs} if
  you are using \Lpack{hyperref} \emph{and} you have subsubsections,
  that are not shown in the ToC. You may see some inline subsection
  entries showing up as `\dots\ text~15 ,\dots', that is a strange
  space appears before the comma.

  This is an artifact  due to the way \Lpack{hyperref} wraps itself
  around the ToC entries, even the ones that are not typeset, and thus
  an end of line space survives. We fix it using \cmd{\endlinechar}:
  \begin{lcode}
    \begingroup
    \endlinechar=-1
    \tableofcontents
    \endgroup
  \end{lcode}
  Note again that it only happen if you have subsubsections with an
  inline subsection entry list, \emph{and} you are using
  \Lpack{hyperref}.
\end{caveat}




    Normally, section titles (and below) are set as individual 
paragraphs. Effectively the first thing that is done is to end any 
previous paragraph, and also the last thing is to end the current 
paragraph. Notice that the main code above neither starts nor finishes 
a paragraph. If the group of subsections is followed by a section title, 
that supplies the paragraph end. The last line above ensures that
the last entry in the \file{toc} file is \piif{par} as this might be
needed to finish off a group of subsections if these are the last 
entries.

And thirdly for the main \toc, the macro \cmd{\setupmaintoc} reverts 
everything back to normal.
\begin{lcode}
\newcommand*{\setupmaintoc}{%
  \renewcommand{\contentsname}{Contents}
  \let\changetocdepth\oldchangetocdepth
  \let\precistoctext\oldprecistoctext
  \let\cftchapterfillnum\oldcftchapterfillnum
  \addtodef{\cftchapterbreak}{\par}{}
  \renewcommand*{\cftchapterfont}{\normalfont\sffamily}
  \renewcommand*{\cftchapterleader}{%
                 \sffamily\cftdotfill{\cftchapterdotsep}}
  \renewcommand*{\cftchapterafterpnum}{}
  \renewcommand{\cftchapterbreak}{\par\addpenalty{-\@highpenalty}}
  \setpnumwidth{2.55em}
  \setrmarg{3.55em}
  \setcounter{tocdepth}{2}}
  \let\cftpartformatpnum\oldcftpartformatpnum
    \addtodef{\cftpartbreak}{\par}{}
  \let\cftbookformatpnum\oldcftbookformatpnum
    \addtodef{\cftbookbreak}{\par}{}
\end{lcode}
The first few lines restore some macros to their original definitions.
\begin{lcode}
\addtodef{\cftchapterbreak}{\par}{}
\end{lcode}
ensures that a chapter entry starts off with a \piif{par}; this is needed
when the previous entry is a group of subsections and their paragraph
has to be ended. The remaining code lines simply set the appearance of 
the chapter titles and restore that for parts and books, as well as ensuring
that they start off new paragraphs.

In the document itself, \cmd{\tableofcontents} was called twice, 
after the appropriate setups:
\begin{lcode}
...
\setupshorttoc
\tableofcontents
\clearpage
\setupparasubsecs
\setupmaintoc
\tableofcontents
\setlength{\unitlength}{1pt}
...
\end{lcode}
After all this note that I ensured that \lnc{\unitlength} was set 
to its default value (it had been used as a scratch length in the 
code for \cmd{\setupparasubsecs}).


%%%%%%%%%%%%%%%%%%%%%%%%%%%%%%%%%%%%%%%%%%%%%%%%%%%%%%%%%%%%%%%%%%%



 \section{New \listofx\ and entries}

\index{list!new list of|(}

 \begin{syntax}
\cmd{\newlistof}\marg{listofcom}\marg{ext}\marg{listofname} \\
\end{syntax}
\glossary(newlistof)%
  {\cs{newlistof}\marg{listofcom}\marg{ext}\marg{listofname}}%
  {Creates two new List of \ldots commands, \cs{listofcom} 
   and \cs{listofcom*}, which use
   a file with extension \meta{ext} and \meta{listofname} for the
  title.}
 The command \cmd{\newlistof} 
 creates a new \listofx, and assorted commands to go along with it.
 The first argument, \meta{listofcom} is used to define a new
 command called \verb?\listofcom? which can then be used like 
\verb?\listoffigures?
to typeset the \listofx. The \meta{ext} argument is the file 
extension to
be used for the new listing. The last argument, \meta{listofname} is
the title for the \listofx. Unstarred and starred versions of
\verb?\listofcom? are created. The unstarred version, \verb?\listofcom?, 
will add \meta{listofname} to the \toc, while the starred version, 
\verb?\listofcom*?, makes no entry in the \toc.

 As an example:
 \begin{lcode}
 \newcommand{\listanswername}{List of Answers}
 \newlistof{listofanswers}{ans}{\listanswername}
 \end{lcode}
 will create a new \cmd{\listofanswers} command that can be used
to typeset a listing of answers under the
title \cmd{\listanswername}, where the answer titles are in an \file{ans}
file. 
   It is up to the author of the document to specify the `answer' code
for the answers in the document. For example:
 \begin{lcode}
 \newcounter{answer}[chapter]
 \renewcommand{\theanswer}{\arabic{answer}}
 \newcommand{\answer}[1]{
   \refstepcounter{answer}
   \par\noindent\textbf{Answer \theanswer. #1}
   \addcontentsline{ans}{answer}{\protect\numberline{\theanswer}#1}\par}
 \end{lcode}
 which, when used like:
\begin{lcode}
\answer{Hard} The \ldots
\end{lcode}
 will print as:
\begin{syntax}
 \textbf{Answer 1. Hard} \\
 \hspace*{2em} The \ldots \\
\end{syntax}

    As mentioned above, the \cmd{\newlistof} command creates several 
new commands in addition to \verb?\listofcom?, most of which you should 
now be familiar with. For convenience,
assume that \verb?\newlistof{...}{X}{...}? has been issued so that
\texttt{X} is the new file extension and corresponds to the \texttt{X} in
\S\ref{sec:titles}. Then in addition to \verb?\listofcom? the following 
new commands will be made available.

 The four commands, \verb?\Xmark?, 
 \verb?\Xheadstart?, 
\verb?\printXtitle?, and
\verb?\afterXtitle?, 
are analagous to the commands of the same names
described in \S\ref{sec:titles} (internally the class uses
the \cmd{\newlistof} macro to define the \toc, \lof\ and \lot). 
In particular the default definition of \verb?\Xmark? is equivalent to:
\begin{lcode}
\newcommand{\Xmark}{\markboth{listofname}{listofname}}
\end{lcode}
However, this may well be altered by the particular \cmd{\pagestyle} in
use.

\begin{syntax}
\verb?Xdepth? \\
\end{syntax}
 The counter \verb?Xdepth? is analagous to the standard
 \Icn{tocdepth} counter, in that it specifies that entries
 in the new listing should not be typeset if their numbering level 
 is greater
 than \verb?Xdepth?. The default definition is equivalent to
\begin{lcode}
\setcounter{Xdepth}{1}
\end{lcode}

\begin{syntax}
\cmd{\insertchapterspace} \\
\cmd{\addtodef}\marg{macro}\marg{prepend}\marg{append} \\
\end{syntax}
Remember that the \cmd{\chapter} command uses \cmd{\insertchapterspace}
to insert vertical spaces into the \lof{} and \lot. If you want similar
spaces added to your new listing then you have to modify
\cmd{\insertchapterspace}. The easiest way to do this is via
the \cmd{\addtodef} macro, like:
\begin{lcode}
\addtodef{\insertchapterspace}{}%
  {\addtocontents{ans}{\protect\addvspace{10pt}}}
\end{lcode}
The \cmd{\addtodef} macro is described later in \S\ref{sec:addtodef}.

    The other part of creating a new \listofx, is to specify the 
formatting of the entries, i.e., define an appropriate \verb?\l@kind? 
macro.

\begin{syntax}
\cmd{\newlistentry}\oarg{within}\marg{cntr}\marg{ext}\marg{level-1} \\
\end{syntax}
\glossary(newlistentry)%
  {\cs{newlistentry}\oarg{within}\marg{cntr}\marg{ext}\marg{level-1}}%
  {Creates the commands for typesetting an entry in a \listofx.
   \meta{cntr} is the new counter for the entry, which may be reset
   by the \meta{within} counter. \meta{ext} is the file extension
   and \meta{level-1} is one less than the entry's level.}

 The command \cmd{\newlistentry} creates the commands necessary for
typesetting an entry in a \listofx.
 The first required argument, \meta{cntr} is used to define a new
 counter called \texttt{cntr}, unless \texttt{cntr} is already defined. 
The optional \meta{within} argument can be used so that \texttt{cntr} 
gets reset to one every time the counter called \texttt{within} is changed. 
That is, the first two arguments when \texttt{cntr} is not
already defined, are equivalent to calling 
\cmd{\newcounter}\marg{cntr}\oarg{within}. If \texttt{cntr} is already
defined, \cmd{\newcounter} is not called. \texttt{cntr} is used for the 
number that goes along with the title of the entry.

The second required argument, \meta{ext}, is the file extension for
the entry listing.  The last argument, \meta{level-1}, is a number
specifying the numbering level minus one, of the entry in a listing.


Calling \cmd{\newlistentry} creates several new commands used to
configure the entry. So in order to configure the list look of  our
previous answer example we would add
\begin{lcode}
\newlistentry{answer}{ans}{0}  
\end{lcode}

Assuming that \verb?\newlistentry? is called as
 \verb?\newlistentry[within]{K}{X}{N}?, where \texttt{K} and
 \texttt{X} are similar to the previous uses of them (e.g., \texttt{K}
 is the kind of entry \texttt{X} is the file extension), and
 \texttt{N} is an integer number, then the following commands are made
 available.


  The set of commands \verb?\cftbeforeKskip?, 
 \verb?\cftKfont?, 
 \verb?\cftKpresnum?, 
 \verb?\cftKaftersnum?, 
 \verb?\cftKaftersnumb?, 
 \verb?\cftKleader?, 
 \verb?\cftKdotsep?, 
 \verb?\cftKpagefont?, and
 \verb?\cftKafterpnum?,
 are analagous to the commands of the same names
 described in \Sref{sec:entries}. Their default values are also
 as described earlier.

 The default values of \verb?\cftKindent? and \verb?\cftKnumwidth? are 
set according to the value of the \meta{level-1} argument 
(i.e., \texttt{N} in this example). For \verb?N=0? the settings 
correspond to those for figures\index{figure} and tables\index{table}, 
as listed in \tref{tab:indents} for the \Lclass{memoir} class.
For \verb?N=1? the settings correspond to subfigures\index{figure!sub-}, 
and so on. For values of \verb?N? less than zero or greater than four, 
or for non-default values, use the \cmd{\cftsetindents} command to 
set the values.

  \verb?\l@K? is an internal command that typesets an entry in the list, 
and is defined in terms of the above \verb?\cft*K*? commands. It will 
not typeset an entry if \verb?Xdepth? is \texttt{N} or less, where 
\texttt{X} is the listing's file extension.

 The command \verb?\theK? prints the value of the \texttt{K} counter. 
It is initially defined so that it prints arabic numerals. If the 
optional \meta{within} argument is used, \verb?\theK? is defined as 
\begin{lcode}
 \renewcommand{\theK}{\thewithin.\arabic{K}}
\end{lcode}
 otherwise as
\begin{lcode}
\renewcommand{\theK}{\arabic{K}}
\end{lcode}

 As an example of the independent use of \cmd{\newlistentry}, the 
following will set up for sub-answers.
 \begin{lcode}
 \newlistentry[answer]{subanswer}{ans}{1}
 \renewcommand{\thesubanswer}{\theanswer.\alph{subanswer}}
 \newcommand{\subanswer}[1]{
    \refstepcounter{subanswer}
    \par\textbf{\thesubanswer) #1}
    \addcontentsline{ans}{subanswer}{\protect\numberline{\thesubanswer}#1}
 \setcounter{ansdepth}{2}
 \end{lcode}
 And then:
 \begin{lcode}
 \answer{Harder} The \ldots
   \subanswer{Reformulate the problem} It assists \ldots
 \end{lcode}
 will be typeset as:
\begin{syntax}
\textbf{Answer 2. Harder} \\
\hspace*{2em} The \ldots \\
\hspace*{2em} \textbf{2.a) Reformulate the problem} It assists \ldots \\
\end{syntax}

     By default the answer entries will appear in the List of Answers 
listing (typeset by the \cs{listofanswers} command).
In order to get the subanswers to appear, 
the \verb?\setcounter{ansdepth}{2}? command was used above.

 To turn off page numbering for the subanswers, do
\begin{lcode}
\cftpagenumbersoff{subanswer}
\end{lcode}

    As another example of \cmd{\newlistentry}, suppose that an extra 
sectioning division below \texttt{subparagraph} is required, 
called \texttt{subsubpara}. The \verb?\subsubpara? command itself can 
be defined via the \ltx\ kernel \cmd{\@startsection} command. 
Also it is necessary to define a \verb?\subsubparamark? macro,
a new \texttt{subsubpara} counter, a \verb?\thesubsubpara? macro and a 
\verb?\l@subsubpara?  macro. Using \cmd{\newlistentry} takes care of 
most of these as shown below; remember the caveats about commands 
with \idxatincode\texttt{@} signs in them (\seeatincode).
\begin{lcode}
 \newcommand{\subsubpara}{\@startsection{subpara}
    {6}                              %                    level
    {\parindent}                     %  indent from left margin
    {3.25ex \@plus1ex \@minus .2ex}  %     skip above heading
    {-1em}       run-in heading with % 1em between title & text
    {\normalfont\normalsize\itshape} % italic number and title 
 }
 \newlistentry[subparagraph]{subsubpara}{toc}{5}
 \cftsetindents{subsubpara}{14.0em}{7.0em}
 \newcommand*{\subsubparamark}[1]{}  % gobble heading mark
 \end{lcode}

     Each \listofx\ uses a file to store the list entries, and these
 files must remain open for writing throughout the document processing.
\tx\ has only a limited number of files that it can keep open, and this
 puts a limit on the number of listings that can be used. For a document
 that includes a \toc\ but no other extra ancilliary files (e.g., no
 index\index{index} or bibliography\index{bibliography} output files) 
the maximum number of LoX's, including a \lof\ and \lot, is no more 
than about eleven. If you try and create too many new listings \ltx\ 
will respond with the error message: 
 \begin{center}
 \texttt{No room for a new write} 
 \end{center}
 If you get such a message the only recourse is to redesign your 
document.

\subsection{Example: plates}

    As has been mentioned earlier, some illustrations\index{illustration}
may be tipped in\index{tip in} to a book. Often, these are called 
\emph{plates}\index{plates} if they are on glossy paper\index{paper} 
and the rest of the book is on ordinary paper\index{paper}.
We can define a new kind of Listing for these.

\begin{lcode}
\newcommand{\listplatename}{Plates}
\newlistof{listofplates}{lop}{\listplatename}
\newlistentry{plate}{lop}{0}
\cftpagenumbersoff{plate}
\end{lcode}
This code defines the \cmd{\listofplates} command to start the listing 
which will be titled `Plates' from the \cmd{\listplatename} macro. 
The entry name is \texttt{plate} and the file extension is \texttt{lop}. 
As plate pages typically do not have printed folios\index{folio}, 
the \cmd{\cftpagenumbersoff} command has been used to prohibit page 
number printing in the listing.

    If pages are tipped in, then they are put between a verso and 
a recto page. The \Lpack{afterpage} package~\cite{AFTERPAGE} lets 
you specify something that should happen after the current page 
is finished. The next piece of code uses the package and its 
\cmd{\afterpage} macro to define two macros which let you specify 
something that is to be done after the next verso (\cmd{\afternextverso})
or recto (\cmd{\afternextrecto}) page has been completed.
\begin{lcode}
\newcommand{\afternextverso}[1]{%
  \afterpage{\ifodd\c@page #1\else\afterpage{#1}\fi}}
\newcommand{\afternextrecto}[1]{%
  \afterpage{\ifodd\c@page\afterpage{#1}\else #1\fi}}
\end{lcode}


    The \cmd{\pageref}\marg{labelid} command typesets the page number
corresponding to the location in the document where 
\cmd{\label}\marg{labelid} is specified. The following code defines
two macros\footnote{These only work for arabic page numbers.} 
that print the page number before (\cmd{\priorpageref}) or after 
(\cmd{\nextpageref}) that given by \cmd{\pageref}.
\begin{lcode}
\newcounter{mempref}
\newcommand{\priorpageref}[1]{%
  \setcounter{mempref}{\pageref{#1}}\addtocounter{mempref}{-1}\themempref}
\newcommand{\nextpageref}[1]{%
  \setcounter{mempref}{\pageref{#1}}\addtocounter{mempref}{1}\themempref}
\end{lcode}

    With these preliminaries out of the way, we can use code like the 
following for handling a set of physically tipped in plates.
\begin{lcode}
\afternextverso{\label{tip}
  \addtocontents{lop}{%
    Between pages \priorpageref{tip} and \pageref{tip}
    \par\vspace*{\baselineskip}}
  \addcontentsline{lop}{plate}{First plate}
  \addcontentsline{lop}{plate}{Second plate}
  ...
  \addcontentsline{lop}{plate}{Nth plate}
}
\end{lcode}
This starts off by waiting until the next recto page is started, which
will be the page immediately after the plates, and then inserts the 
label \texttt{tip}. The \cmd{\addtocontents} macro puts its argument
into the plate list \texttt{lop} file, indicating the page numbers 
before and after the set of plates. With the plates being physically 
added to the document it is not possible to use \cmd{\caption}, 
instead the \cmd{\addcontentsline} macros are used to add the plate 
titles to the \texttt{lop} file.

    With a few modifications the code above can also form the basis 
for listing plates that are electronically tipped in but do not have 
printed folios\index{folio} or \cmd{\caption}s.

\index{list!new list of|)}



\LMnote{2010/06/09}{Why do we have this exact section in two different
chapters? No this links to the version i document-divisions.tex }
\section{Chapter precis}

See section~\ref{sec:chapter-precis} on page~\pageref{sec:chapter-precis}.


% \index{chapter!precis|(}

%    Some old style novels, and even some modern text 
%  books,\footnote{For example, Robert Sedgewick, \textit{Algorithms},
%  Addison-Wesley, 1983.} include a short synopsis of the contents of 
%  the chapter either immediately
%  after the chapter heading\index{heading!chapter} or in the \toc, or in both places.

% \begin{syntax}
% \cmd{\chapterprecis}\marg{text} \\
% \end{syntax}
%      The command \cmd{\chapterprecis} prints its argument 
%  both at the
%  point in the document where it is called, and also adds it to the \file{.toc}
%  file. For example:
%  \begin{lcode}
%  ...
%  \chapter{}  first chapter
%  \chapterprecis{Our hero is introduced; family tree; early days.}
%  ...
%  \end{lcode}

% \begin{syntax}
% \cmd{\chapterprecishere}\marg{text} \\
% \cmd{\chapterprecistoc}\marg{text} \\
% \end{syntax}
%  The \cmd{\chapterprecis} command calls these two commands to print the
%  \meta{text} in the document (the \cmd{\chapterprecishere} command) 
%  and to put it into the \toc{} (the \cmd{\chapterprecistoc} command). 
%  These can be used individually if required.

% \begin{syntax}
% \cmd{\prechapterprecis} \cmd{\postchapterprecis} \\
% \end{syntax}
% The \cmd{\chapterprecishere} macro is intended for use immediately after 
% a \cmd{\chapter}. The \meta{text} argument is typeset in
% italics in a \Ie{quote} environment. The macro's definition is:
% \begin{lcode}
% \newcommand{\chapterprecishere}[1]{%
%   \prechapterprecis #1\postchapterprecis}
% \end{lcode}
% where \cmd{\prechapterprecis} and \cmd{\postchapterprecis} are defined
% as:
% \begin{lcode}
% \newcommand{\prechapterprecis}{%
%   \vspace*{\prechapterprecisshift}%
%   \begin{quote}\normalfont\itshape}
% \newcommand{\postchapterprecis}{\end{quote}}
% \end{lcode}
% The \cmd{\prechapterprecis} and \cmd{\postchapterprecis} macros can be 
% changed if another style of typesetting is required.

% \begin{syntax}
% \cmd{\precistoctext}\marg{text} \cmd{\precistocfont} \\
% \end{syntax}
% The \cmd{\chapterprecistoc} macro puts the macro \cmd{\precistoctext} into 
% the \pixfile{toc} file. The default definition is
% \begin{lcode}
% \DeclareRobustCommand{\precistoctext}[1]{%
%   {\leftskip \cftchapterindent\relax
%    \advance\leftskip \cftchapternumwidth\relax
%    \rightskip \@tocrmarg\relax
%    \precistocfont #1\par}}
% \end{lcode}
% Effectively, in the \toc{} \cmd{\precistoctext} typesets its argument like 
% a chapter title using the \cmd{\precistocfont} (default \cmd{\itshape}).

% \index{chapter!precis|)}


\section{Contents lists and bookmarks}
\label{sec:cont-lists-bookm}

\LMnote{2009/06/29}{Added section about contents lists and bookmarks}

With the \Lpack{hyperref} package, the table of contents is often
added as a list of bookmarks thus providing a nice navigation for the
user. There is one slight problem though: when using, say, parts in
the document, all chapters in that part ends up as a child of this
part bookmark---including the index and bibliography. A simple fix to
this is to add
\begin{lcode}
  \makeatletter
  \renewcommand*{\toclevel@chapter}{-1}
  \makeatother
\end{lcode}
just before the material you would like to pull out of the part tree.

\LMnote{2010/06/09}{Heikos own recommendation}
A better solution is the \Lpack{bookmark} package, add it to the
preamble, and add 
\begin{lcode}
  \bookmarksetup{startatroot}
\end{lcode}
before the stuff you want to have moved out of, say, a part.


%#% extend
%#% extstart include flosts-and-captions.tex
