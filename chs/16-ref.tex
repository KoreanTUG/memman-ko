


%%%%%%%%%%%%%%%%%%%%%%%%%%%%%%%%%%%%%%%
\chapter{Cross referencing} \label{chap:xref}
%%%%%%%%%%%%%%%%%%%%%%%%%%%%%%%%%%%%%

%\section{Introduction}

    A significant aspect of \ltx\ is that it provides a variety of
cross referencing\index{cross reference} methods, many of 
which are automatic. An example
of an automatic\index{cross reference!automatic} cross reference is the way in which a \cmd{\chapter}
command automatically adds its title and page number to the \toc,
or where a \cmd{\caption} adds itself to a \listofx.

    Some cross references have to be 
specifically\index{cross reference!specified} specified, such as
a reference in the text to a particular chapter number, and for
these \ltx\ provides a general mechanism that does not require you
to remember the particular number and more usefully does not require
you to remember to change the reference if the chapter number is later 
changed.

\section{Labels and references} \label{sec:lab&ref}

\index{reference!by number|(}

    The general \ltx\ cross reference method uses a pair of macros.
\begin{syntax}
\cmd{\label}\marg{labstr} \\
 \cmd{\ref}\marg{labstr} \cmd{\pageref}\marg{labstr} \\
\end{syntax}
\glossary(label)%
  {\cs{label}\marg{labstr}}%
  {Associates the current (section, caption, \ldots) number, and page number,
   to \meta{labstr}.}
\glossary(ref)%
  {\cs{ref}\marg{labstr}}%
  {Prints the (section, or other) number associated with \meta{labstr}
   from a \cs{label}.}
\glossary(pageref)%
  {\cs{pageref}\marg{labstr}}%
  {Prints the page number associated with \meta{labstr}
   from a \cs{label}.}
You can put a \cmd{\label} command where you want to label\index{label} 
some numbered
element in case you might want to refer to the number from elsewhere in
the document. The \meta{labstr} argument is a sequence of letters, digits, 
and punctuation symbols; upper and lower case letters are different.
The \cmd{\ref} macro prints the number\index{reference!to label} 
associated with \meta{labstr}. 
The \cmd{\pageref} macro prints the number of the 
page\index{reference!to page} where the
\cmd{\label} specifying the \meta{labstr} was placed.

    The \cmd{\label} and \cmd{\ref} mechanism is simple to use and
works well but
on occasions you might be surprised\index{reference!unexpected result}  
at what \cmd{\ref} prints.

    \ltx\ maintains a current \textit{ref} 
value\index{reference!current value} which is 
set\index{reference!set current value} by
the \cmd{\refstepcounter} command. This command is used by the sectioning
commands, by \cmd{\caption}, by numbered environments like
\Ie{equation}, by \cmd{\item} in an \Ie{enumerate} environment, and
so on. The \cmd{\label} command\index{label} writes an entry in the 
\pixfile{aux} file consisting of the \meta{labstr}, the current 
\textit{ref} value\index{reference!current value} and the curent 
page\index{page!current number} number. 
A \cmd{\ref} command picks up the \textit{ref}
value for \meta{labstr} and prints it. Similarly \cmd{\pageref} prints
the page number for \meta{labstr}.

    The critical point is that the \cmd{\label} command picks up the
value set by the \emph{most recent} visible\footnote{Remember, a
change within a group, such as an environment, is not visible
outside the group.}  \cmd{\refstepcounter}. 
\begin{itemize}
\item A \cmd{\label} after a \cmd{\section} picks up the \cmd{\section}
      number, not the \cmd{\chapter} number.
\item A \cmd{\label} after a \cmd{\caption} picks up the caption number.
\item A \cmd{\label} \emph{before} a \cmd{\caption} picks up the surrounding
      sectional number.
\end{itemize}
If you are defining your own macro that sets a counter, the counter value
will be invisible to any \cmd{\label} unless it is 
set\index{reference!set current value} using \cmd{\refstepcounter}. 



\begin{syntax}
\cmd{\fref}\marg{labstr} \cmd{\figurerefname} \\
\cmd{\tref}\marg{labstr} \cmd{\tablerefname} \\
\cmd{\pref}\marg{labstr} \cmd{\pagerefname} \\
\end{syntax}
\glossary(fref)%
  {\cs{fref}\marg{labstr}}%
  {Prints a named (\cs{figurerefname}) reference to a \cs{label}ed figure.}
\glossary(figurerefname)%
  {\cs{figurerefname}}%
  {Name for a figure used by \cs{fref}.}
\glossary(tref)%
  {\cs{tref}\marg{labstr}}%
  {Prints a named (\cs{tablerefname}) reference to a \cs{label}ed table.}
\glossary(tablerefname)%
  {\cs{tablerefname}}%
  {Name for a table used by \cs{tref}.}
\glossary(pref)%
  {\cs{pref}\marg{labstr}}%
  {Prints a named (\cs{pagerefname}) reference to a \cs{label} page reference.}
\glossary(pagerefname)%
  {\cs{pagerefname}}%
  {Name for a page used by \cs{pref}.}
    The class provides these more particular named\index{reference!to figure} 
references to a figure\index{figure!reference},
table\index{table!reference}\index{reference!to table} or 
page\index{page!reference}\index{reference!to page}. For example the 
default definitions of \cmd{\fref} and 
\cmd{\pref} are
\begin{lcode}
\newcommand{\fref}[1]{\figurerefname~\ref{#1}}
\newcommand{\pref}[1]{\pagerefname~\pageref{#1}}
\end{lcode}
and can be used as 
\begin{lcode}
\ldots footnote parameters are shown in~\fref{fig:fn} 
on~\pref{fig:fn}.
\end{lcode}
which in this document prints as: 
\begin{syntax}
\ldots footnote parameters are shown in~\fref{fig:fn} on~\pref{fig:fn}. \\
\end{syntax}

\begin{syntax}
\cmd{\Aref}\marg{labstr} \cmd{\appendixrefname} \\
\cmd{\Bref}\marg{labstr} \cmd{\bookrefname} \\
\cmd{\Pref}\marg{labstr} \cmd{\partrefname} \\
\cmd{\Cref}\marg{labstr} \cmd{\chapterrefname} \\
\cmd{\Sref}\marg{labstr} \cmd{\sectionrefname} \\
\end{syntax}
\glossary(Aref)%
  {\cs{Aref}\marg{labstr}}%
  {Prints a named (\cs{appendixrefname}) reference to a \cs{label}ed appendix.}
\glossary(appendixrefname)%
  {\cs{appendixrefname}}%
  {Name for an appendix used by \cs{Aref}.}
\glossary(Bref)%
  {\cs{Bref}\marg{labstr}}%
  {Prints a named (\cs{bookrefname}) reference to a \cs{label}ed book.}
\glossary(bookrefname)%
  {\cs{bookrefname}}%
  {Name for a book used by \cs{Bref}.}
\glossary(Pref)%
  {\cs{Pref}\marg{labstr}}%
  {Prints a named (\cs{partrefname}) reference to a \cs{label}ed part.}
\glossary(partrefname)%
  {\cs{partrefname}}%
  {Name for a part used by \cs{Pref}.}
\glossary(Cref)%
  {\cs{Cref}\marg{labstr}}%
  {Prints a named (\cs{chapterrefname}) reference to a \cs{label}ed chapter.}
\glossary(chapterrefname)%
  {\cs{chapterrefname}}%
  {Name for a chapter used by \cs{Cref}.}
\glossary(Sref)%
  {\cs{Sref}\marg{labstr}}%
  {Prints a named (\cs{sectionrefname}) reference to a \cs{label}ed section.}
\glossary(sectionrefname)%
  {\cs{sectionrefname}}%
  {Name for a section used by \cs{Sref}.}
Similarly, specific commands are supplied for referencing sectional 
divisions; \cmd{\Aref}\index{reference!to appendix} for \appendixrefname,
\cmd{\Bref}\index{reference!to book} for \bookrefname,
\cmd{\Pref}\index{reference!to part} for \partrefname, 
\cmd{\Cref}\index{reference!to chapter} for \chapterrefname,
and \cmd{\Sref}\index{reference!to section} for divisions 
below \chapterrefname. For example:
\begin{lcode}
This is \Sref{sec:lab&ref} in \Cref{chap:xref}.
\end{lcode}
This is \Sref{sec:lab&ref} in \Cref{chap:xref}.

\index{reference!by number|)}

\section{Reference by name} \label{sec:nxref}

\index{reference!by name|(}

    In technical works it is common to reference a chapter, say, by its
number. In non-technical works such cross references are likely to be
rare, and when they are given it is more likely that the chapter title
would be used instead of the number, as in:
\begin{lcode}
The chapter \textit{\titleref{chap:bringhurst}} describes \ldots
\end{lcode}
The chapter \textit{\titleref{chap:bringhurst}} describes \ldots

    There are two packages, \Lpack{nameref}~\cite{NAMEREF} and 
\Lpack{titleref}~\cite{TITLEREF},
 that let you refer to things by name instead of number.

    Name references were added to the class as a consequence of adding
a second optional argument to the sectioning commands. I found
that this broke the \Lpack{nameref} package, and hence the
\Lpack{hyperref} package as well, so they had to be fixed. The change 
also broke Donald Arseneau's \Lpack{titleref} package, and it turned out
that \Lpack{nameref} also clobbered \Lpack{titleref}. The class also
provides titles, like \cmd{\poemtitle}, that are not recognised by
either of the packages. From my viewpoint the most efficient
thing to do was to enable the class itself to provide name 
referencing.


\begin{syntax}
\cmd{\titleref}\marg{labstr} \\
\end{syntax}
\glossary(titleref)%
  {\cs{titleref}\marg{labstr}}%
  {Prints the (section, or other) title of the number associated 
   with \meta{labstr} from a \cs{label}.}
The macro \cmd{\titleref} is a class addition to the usual set of
cross referencing commands. Instead of printing a number it typesets
the title associated with the labelled number. This is really only useful
if there \emph{is} a title, such as from a \cmd{\caption} or
\cmd{\section} command. For example, look at this code 
and its result.

\begin{egsource}{eg:sec:nxref:1}
Labels may be applied to:
\begin{enumerate}
\item Chapters, sections, etc.            \label{sec:nxref:1}
...
\item Items in numbered lists, etc. \ldots \label{sec:nxref:5}
\end{enumerate}
Item \ref{sec:nxref:1} in section \textit{\titleref{sec:nxref}} 
mentions sections while item \titleref{sec:nxref:5}, on page 
\pageref{sec:nxref:5} in the same section, mentions things like
items in enumerated lists that should not be referenced 
by \verb?\titleref?.
\end{egsource}

\begin{egresult}[Named references should be to titled elements]{eg:sec:nxref:1}
Labels may be applied to:
\begin{enumerate}
\item Chapters, sections, etc.            \label{sec:nxref:1}
\item Captions
\item Legends
\item Poem titles
\item Items in numbered lists, etc. \ldots \label{sec:nxref:5}
\end{enumerate}
Item \ref{sec:nxref:1} in section \textit{\titleref{sec:nxref}} 
mentions sections while item\index{reference!unexpected result} 
\titleref{sec:nxref:5}, on page 
\pageref{sec:nxref:5} in the same section, mentions things like
items in enumerated lists 
that should not be referenced by \verb?\titleref?.
\end{egresult}

    As the above example shows, you have to be a little careful in using
\cmd{\titleref}.
Generally speaking, \cmd{\titleref}\marg{key} produces the last named 
thing before the \cmd{\label} that defines the \meta{key}. 

\begin{syntax}
\cmd{\headnameref} \cmd{\tocnameref} \\
\end{syntax}
\glossary(headnameref)%
  {\cs{headnameref}}%
  {Use header title for sectional title references.}
\glossary(tocnameref)%
  {\cs{tocnameref}}%
  {Use \prtoc{} title for sectional title references.}
There can be three possibilities for the name of a sectional division;
the full name, the name in the \toc, and the name in the page header.
As far as \cmd{\titleref} is concerned it does not use the fullname, 
and so the choice simplifies to the \toc{} or page header. Following
the declaration \cmd{\headnameref} it uses the page header name.
Following the opposite declaration \cmd{\tocnameref}, which is the
default, it uses the \toc\ name.

\Note{} Specifically with the \Lclass{memoir} class, 
do not put a \cmd{\label} command inside an
argument to a \cmd{\chapter} or \cmd{\section} or \cmd{\caption}, etc.,
command. Most likely it will either be ignored or referencing it will
produce incorrect values. This restriction does not apply to the standard
classes, but in any case I think it is good practice not to embed any 
\cmd{\label} commands.

\begin{syntax}
\cmd{\currenttitle} \\
\end{syntax}
\glossary(currenttitle)%
  {\cs{currenttitle}}%
  {Gives the title of the last sectional division command.}
    If you just want to refer to the current title you can do so with
\cmd{\currenttitle}. This acts as though there had been a label associated
with the title and then \cmd{\titleref} had been used to refer to that label.
For example:
\begin{egsource}{eg:sec:nxref:2}
This sentence in the section titled `\currenttitle' is an example of the
use of the command \verb?\currenttitle?.
\end{egsource}

\begin{egresult}[Current title]{eg:sec:nxref:2}
This sentence in the section titled `\currenttitle' is an example of the
use of the command \verb?\currenttitle?.
\end{egresult}

\begin{syntax}
\cmd{\theTitleReference}\marg{num}\marg{text} \\
\end{syntax}
\glossary(theTitleReference)%
  {\cs{theTitleReference}\marg{num}\marg{text}}
  {Called by \cs{titleref} and \cs{currenttitle} with the number and
   text of the title to print the values.}
Both \cmd{\titleref} and \cmd{\currenttitle} use the \cmd{\theTitleReference}
to typeset the title. This is called with two arguments --- 
the number, \meta{num}, and the text, \meta{text}, of the title. The
default definition is:
\begin{lcode}
\newcommand{\theTitleReference}[2]{#2}
\end{lcode}
so that only the \meta{text} argument is printed. You could, for example,
change the definition to
\begin{lcode}
\renewcommand{\theTitleReference}[2]{#1\space \textit{#2}}
\end{lcode}
to print the number followed by the title in italics. If you do this, only use
\cmd{\titleref} for numbered titles, as a printed number for an 
unnumbered title (a) makes no sense, and (b) will in any case be 
incorrect.

    The commands \cmd{\titleref}, \cmd{\theTitleReference} and 
\cmd{\currenttitle} are direct equivalents of those in the \Lpack{titleref}
package~\cite{TITLEREF}.

\begin{syntax}
\cmd{\namerefon} \cmd{\namerefoff} \\
\end{syntax}
\glossary(namerefon)%
  {\cs{namerefon}}%
  {Makes \cs{titleref} operative.}
\glossary(namerefoff)%
  {\cs{namerefoff}}%
  {Makes \cs{titleref} inoperative.}

    The capability for referencing by name has one potentially
unfortunate side effect --- it causes some arguments, such as that
for \cmd{\legend}, to be moving\index{moving argument} arguments 
and hence any fragile\index{fragile} command
in the argument will need \cmd{\protect}ing. However, not every document
will require the use of \cmd{\titleref} and so the declaration
\cmd{\namerefoff} is provided to switch it off (the argument to
\cmd{\legend} would then not be moving). The declaration
\cmd{\namerefon}, which is the default, enables name referencing.

\index{reference!by name|)}


%#% extend
%#% extstart include backmatter.tex
