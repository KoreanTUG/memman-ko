%\chapter{Text and fonts}
\chapter{텍스트와 글꼴}


%     Presumably you will be creating a document that contains at least some
% text. In this chapter I talk a little about the kinds of fonts that you might
% use and how text appears on a page.

새로운 문서를 작성할때, 아마 조금이더라도 텍스트를 포함할 것이다. 
이 챕터에서는 사용할 수 있는 글꼴의 종류와 페이지에 텍스트가 나타나는 방식에
대해 설명하겠다.

%%%%%%%%%%%%%%%%%%%%%%%%%%%%%%%%%%%%%%%%%%%%%%%%%%%%%%%%%


%\section{Fonts} \label{sec:fonts}
\section{글꼴} \label{sec:fonts}

%  \alltx{} comes with a standard set of fonts, designed by Donald Knuth,
% and known as the Computer Modern\index{font!Computer Modern} font family. 
\alltx{}에는 도널드 커누스(Donald Knuth)가 디자인한
Computer Modern\tidx{font!Computer Modern,글꼴!Computer Modern}
글꼴 제품군으로 알려진 표준 글꼴 집합이 있다.

% The Knuthian
% fonts\index{font!metafont?\metafont} were created via the \metafont{} 
% program~\cite{METAFONT,CM}\index{metafont?\metafont}
% and are in the form of bitmaps\index{bitmap}
% (i.e., each character is represented as a bunch of tiny dots). Fonts of
% this kind are called \emph{bitmap fonts}\indextwo{bitmap}{font}.
커누시안(Knuthian) 글꼴\tidx{font!metafont?\metafont,글꼴!메타폰트?\metafont}은
\metafont{}
프로그램~\cite{METAFONT,CM}\tidx{metafont?\metafont,메타폰트?\metafont}을
통해 만들어졌으며 비트맵\tidx{bitmaps,비트맵} 형식으로 되어 있다 (즉, 각
문자는 작은 점들로 표시된다).
이런 종류의 글꼴을
\emph{비트맵 글꼴}\indextwo{bitmap}{font}\indextwo{비트맵}{글꼴}이라고 한다.

% There is also a wide range of \metafont{} fonts available, created by 
% many others, in addition to the standard set.  More modern
% digital fonts, such as \pscript\indexsupsubmain{font}{PostScript} or 
% TrueType\indexsupsubmain{font}{TrueType} fonts are represented in terms
% of the curves outlining the character, and it is the job of the printing
% machine to fill in the outlines (with a bunch of tiny dots). Fonts of
% this type are called \emph{outline fonts}\indextwo{outline}{font}.
표준 모음에는, 다른 많은 이들에 의해 추가된 광범위한 메타폰트(\metafont)
글꼴들도 있다.
보다 현대적인 디지털 글꼴, 예를 들어
포스트스크립트\indexsupsubmain{font}{PostScript}\indexsupsubmain{글꼴}{포스트스크립트}
혹은 트루타입\indexsupsubmain{font}{TrueType}\indexsupsubmain{글꼴}{트루타입}
글꼴은 문자의 윤곽을 나타내는 곡선으로 표현되며, 인쇄 기계가 외곽선을
(수많은 작은 점들로) 채우는 역할을 한다.
이 유형의 글꼴을 윤곽선 글꼴(\emph{outline fonts}\indextwo{outline}{font})이라고
한다

% \metafont\ fonts
% are designed for a particular display resolution and cannot reasonably 
% be scaled to match an arbitrary display device, whereas outline fonts can be 
% scaled before they are physically displayed.

윤곽선 글꼴(\emph{outline fonts}\indextwo{outline}{font})은 표시되기 전에 크기를 조정할 수 있는 것과 달리,
\metafont\ 글꼴은 특정 디스플레이 화질 용도로 설계되어서, 임의의 디스플레이 장치와 일치하도록 합리적으로 조정하는 것이 불가능하다.

% \PWnote{2009/04/27}{Extended the font use/installation references}
%    There is an excessive number of \pscript\indexsupsubmain{font}{PostScript}
% and TrueType\indexsupsubmain{font}{TrueType}
% fonts available 
% and these can all, with some amount of effort, be used with \ltx. How to do
% that is outside the scope of this work; Alan Hoenig has written an excellent
% book on the subject~\cite{HOENIG98} and there is the invaluable 
% \btitle{The \ltx\ Companion} \cite[Chapter 7 Fonts and Encodings]{COMPANION}. 


\PWnote{2009/04/27}{폰트 사용 확대/설치 레퍼런스}
사용 가능한 \pscript\indexsupsubmain{font}{PostScript, 포스트스크립트} 및 트루타입\indexsupsubmain{font}{TrueType, 트루타입} 글꼴의 수가 굉장히 많고, 이들 모두 약간의 노력을 하면 \ltx에서 사용할 수 있다.
이를 수행하는 방법은 이 문서에서 다루지는 않고, 이에 대해 관심이 있다면, Alan Hoenig이 저술한 훌륭한 책~\cite{HOENIG98}을 보자.
그리고 여기에는 귀중한 \btitle{The \ltx\ Companion} \cite[Chapter 7 Fonts and Encodings(글꼴과 인코딩)]{COMPANION} 이 있다.


% The original \btitle{The \ltx\ Graphics Companion} had chapters on PostScript
% fonts and tools but these were dropped in the second edition to keep it below
% an overwhelming size. This material has been updated and is available
% free from \url{http://xml.cern.ch/lgc2}; you can also get a 
% `work in progress'\pagenote[`work in progress']{\pixxetx\ enables you to 
% use Opentype fonts with \ltx, and supports both left-to-right and 
% right-to-left typesetting. It has become very popular with those involved
% in linguistics and non-Latin scripts.}
% about \pixxetx, Unicode\index{Unicode}, and 
% Opentype\indexsupsubmain{font}{Opentype}
% fonts from the same source. 
% There is less detailed, but also free, 
% information available via \pixctan, for example Philipp Lehman's 
% \textit{Font Installation Guide}~\cite{FONTINST}; even if you are not 
% interested in
% installing \pscript\ fonts this is well worth looking at just as an
% example of the kind of elegant document that can be achieved with \ltx.
% If you choose one of the popular
% \pscript\indexsupsubmain{font}{PostScript} fonts, such as those built 
% into \ixpscript\ printers, you may 
% well find that the work has been done for you and it's just a question
% of using the appropriate 
% package.%

\btitle{The \ltx\ Graphics Companion}의 초판에는 PostScript 글꼴과 도구에 대한 챕터가 있었지만, 두 번째 판에서는 초판에서의 방대한 분량이 삭제되고 굉장히 조금만 실려있다.
이 자료는 업데이트 되었으며, \url{http://xml.cern.ch/lgc2}에서 무료로 이용할 수 있다. 또한, \pixxetx, Unicode\index{Unicode, 유니코드}, 그리고
Opentype\indexsupsubmain{font}{Opentype}에 대한 
`진행중인 작업'\pagenote[`work in progress']{\pixxetx\ enables you to 
use Opentype fonts with \ltx, and supports both left-to-right and 
right-to-left typesetting. It has become very popular with those involved
in linguistics and non-Latin scripts.}을 얻을 수 있다.
그닥 디테일하진 않지만, 무료이며, \pixctan을 통해 가능하고, Philipp Lehman의 \textit{Font Installation Guide,폰트 설치 가이드}~\cite{FONTINST}; 같은 것이 있다.
심지어 너가 만약 \pscript\ 글꼴을 설치하는 것에 대해 관심이 없더라도, 이것은 \ltx로 얻을 수 있는 우아한 문서의 예시로 관상하기에 훌륭하다.

\ixpscript\ 프린터에 내장된 것과 같이, \pscript\indexsupsubmain{font}{PostScript} 글꼴과 같이 널리 사용되는 글꼴 중 하나를 선택해야 한다면, 
너는 너의 문서가 굉장히 잘 완성된다(아름답다)는 것을 느낄 수 있을 것이며, 그저 적당한 package를 사용하기만 하면 된다.

\pagenote[using the appropriate package]{I have found Christopher
League's\index{League, Christopher} \textit{\tx\ support for the FontSite
500 CD}, obtainable from \url{http:contrapunctus.net/fs500tex}, 
extremely useful in providing packages for a wide range of 
\pscript\indexsupsubmain{font}{PostScript} fonts for me to use. You do have 
to buy a CD containing the sources of the fonts from FontSite
(\url{http://www.fontsite.com}); it cost me
a total of \$37.12, including taxes and shipping, in 2002 for 512 
\pscript\ and TrueType\indexsupsubmain{font}{TrueType} professional
quality fonts that are legal and very reasonably priced.


%  Many of the fonts fall into the Decorative/Display category but the book
% fonts include:
많은 글꼴이 장식(Decorative) / 디스플레이(Display) 범주에 속하지만 책 글꼴에는 다음이 포함됩니다.

\begin{description}

\item[Blackletter]\typesubidx{Blackletter}
Alte Schwabacher\facesubseeidx{Alte Schwabacher},
Engravers Old English\facesubseeidx{Engravers Old English},
Fette Fraktur\facesubseeidx{Fette Fraktur},
Fette Gotisch\facesubseeidx{Fette Gotisch}, and
Olde English\facesubseeidx{Olde English}.

\item[Uncial/Mediaeval]\typesubidx{Uncial}\typesubidx{Mediaeval}
American UncialXX{American Uncial},
Linden\facesubseeidx{Linden}, and
Rosslaire\facesubseeidx{Rosslaire}.

\item[Geralde/Venetian]\typesubidx{Geralde}\typesubidx{Venetian}
Bergamo\facesubseeidx{Bergamo} (also known as Bembo\facesubseeidx{Bembo}),
Caslon\facesubseeidx{Caslon}, 
Garamond\facesubseeidx{Garamond}, 
Goudy Old Style\facesubseeidx{Goudy Old Style},
Jenson Recut\facesubseeidx{Jenson Recut} (also known as Centaur\facesubseeidx{Centaur}),
URW Palladio\facesubseeidx{URW Palladio} (also known as Palatino\facesubseeidx{Palatino}),
Savoy\facesubseeidx{Savoy} (also known as Sabon\facesubseeidx{Sabon}),
Schnittger\facesubseeidx{Schnittger},
University Old Style\facesubseeidx{University Old Style}, and
Vendome\facesubseeidx{Vendome}.

\item[Transitional]\typesubidx{Transitional}
URW Antiqua\facesubseeidx{URW Antiqua}, 
Baskerville\facesubseeidx{Baskerville},
Century Old Style\facesubseeidx{Century Old Style},
ATF Clearface\facesubseeidx{ATF Clearface},
English Serif\facesubseeidx{English Serif},
Jessica\facesubseeidx{Jessica} (also known as Joanna\facesubseeidx{Joanna}),
Lanston Bell\facesubseeidx{Lanston Bell},
New Baskerville\facesubseeidx{New Baskerville}, and
Nicholas Cochin\facesubseeidx{Nicholas Cochin}.

\item[Modern/Didone]\typesubidx{Modern}\typesubidx{Didone}
Basel\facesubseeidx{Basel} (also known as Basilia\facesubseeidx{Basilia}),
Bodoni\facesubseeidx{Bodoni},
Modern\facesubseeidx{Modern}, and
Walbaum\facesubseeidx{Walbaum}.

\item[Free Form]\typesubidx{Free Form}
Barbedour\facesubseeidx{Barbedour},
Bernhard Modern\facesubseeidx{Bernhard Modern},
Della Robbia\facesubseeidx{Della Robbia},
Engravers Litho\facesubseeidx{Engravers Litho},
Flanders\facesubseeidx{Flanders}, and
Lydian\facesubseeidx{Lydian}.

\item[Sans Serif]\typesubidx{Sans Serif}
There are over 20 in this category but some of the ones I am most familiar
with are: 
Chantilly\facesubseeidx{Chantilly} (also known as Gill Sans\facesubseeidx{Gill Sans}),
Franklin Gothic\facesubseeidx{Franklin Gothic},
Function\facesubseeidx{Function} (also known as  Futura\facesubseeidx{Futura}),
Lanston Koch\facesubseeidx{Lanston Koch},
News Gothic\facesubseeidx{News Gothic},
Opus\facesubseeidx{Opus} (also known as Optima\facesubseeidx{Optima}),
Struktor\facesubseeidx{Struktor} (also known as Syntax\facesubseeidx{Syntax}), and
Unitus\facesubseeidx{Unitus} (also known as Univers\facesubseeidx{Univers}).

\item[Slab Serif]\typesubidx{Slab Serif}
Cheltenham\facesubseeidx{Cheltenham},
Clarendon\facesubseeidx{Clarendon},
Egyptian\facesubseeidx{Egyptian},
Glytus\facesubseeidx{Glytus} (also known as Glypha\facesubseeidx{Glypha}),
URW Latino\facesubseeidx{URW Latino} (also known as Melior\facesubseeidx{Melior}),
Litho Antique\facesubseeidx{Litho Antique},
Serific\facesubseeidx{Serific} (also known as Serifa\facesubseeidx{Serifa}).

\item[Script]\typesubidx{Script}
There are some sixteen Script fonts.

\item[Decorative]\typesubidx{Decorative}
There are over fifty Decorative fonts.

\item[Symbol]\typesubidx{Symbol}
There are a dozen miscellaneous symbol fonts which include, among others,
arrows, borders, fleurons\index{fleuron} and various icons.
\end{description}
}  % end pagenote

%     A standard \ltx\ distribution includes some 
% \pscript\indexsupsubmain{font}{PostScript} fonts and the packages to support
% them are in the \Ppack{psnfss}\index{psnfss?\Ppack{psnfss}}
% bundle. Most of the fonts are for normal text work but two supply
% symbols rather than characters. Table~\ref{tab:palatinoglyphs}, although it
% is specifically for Palatino\facesubseeidx{Palatino}, shows the glyphs 
% typically available. Tables~\ref{tab:symbolglyphs} and~\ref{tab:dingglyphs}
% show the glyphs in the two symbol fonts.

표준 \ltx\ 배포판에는 일부 \pscript\indexsupsubmain{font}{PostScript} 글꼴이 포함되어 있으며, 이를 서포트하는 패키지들이 \Ppack{psnfss}\index{psnfss?\Ppack{psnfss}} 안에 있다.
대부분의 글꼴은 일반 텍스트 작업용이지만, but two supply symbols rather than characters.
표~\ref{tab:palatinoglyphs} 는, 특히 Palatino\facesubseeidx{Palatino} 용도이지만, 일반적으로 사용 가능한 글리프이다.
표 s~\ref{tab:symbolglyphs}와 ~\ref{tab:dingglyphs}는 두 symbol fonts에서의 글리프를 보여준다.

\begin{table}
\centering
% \caption{Glyphs in the \ltx\ supplied Palatino roman font}\label{tab:palatinoglyphs}
\caption{\ltx\ 에 Palatino roman font의 글리프 }\label{tab:palatinoglyphs}
\nohexoct
\fonttable{pplr}
\end{table}

\begin{table}
\centering
% \caption{Glyphs in the \ltx\ distributed Symbol font}\label{tab:symbolglyphs}
\caption{\ltx\ 에 Symbol font가 배부된 글리프 }\label{tab:symbolglyphs}
\nohexoct
\fonttable{psyr}
\end{table}

\begin{table}
\centering
% \caption{Glyphs in the \ltx\ distributed Zapf Dingbat font}\label{tab:dingglyphs}
\caption{\ltx\ 에 Zapf Dingbat font가 배부된 글리프 }\label{tab:dingglyphs}
\nohexoct
\fonttable{pzdr}
\end{table}


% These supplied \pscript\ fonts, their respective \ltx\ fontfamily\index{fontfamily} names,
% and running text examples of each, are:
이들은 \pscript\ 글꼴을 제공하고, 그들 각각 \ltx\ 에서 fontfamily\index{fontfamily} 이름 및 각각의 텍스트 실행 예제는 다음과 같다 : 

\begin{description}
% \item[ITC Avant Garde Gothic\facesubseeidx{Avant Garde Gothic}] {\LXfont{pag} 
%   is a geometric sans type designed 
%   by Herb Lubalin\index{Lubalin, Herb} and Tom Carnase\index{Carnase, Tom} and 
%   based on the logo of the \textit{Avant Garde} magazine. 
% The fontfamily name is \pfontfam{pag}.

\item[ITC Avant Garde Gothic\facesubseeidx{Avant Garde Gothic}] {\LXfont{pag} 은 Herb Lubalin\index{Lubalin, Herb}와 
Tom Carnase\index{Carnase, Tom}가 디자인 한 geometric sans type이며 \textit{Avant Garde} 잡지의 로고를 기반으로 한다. fontfamily 이름은 \pfontfam{pag} 이다.

\vspace{0.5\onelineskip}
\fox\par\Kafka\par\namesAZ
\vspace{0.5\onelineskip}
}

% \item[ITC Bookman\facesubseeidx{Bookman}] {\LXfont{pbk} was originally 
% sold in 1860 by the Miller \&  Richard\index{Miller \& Richard} foundry 
% in Scotland; it was designed by 
% Alexander Phemister\index{Phemister, Alexander}. The ITC revival is by
% Ed Benguiat\index{Benguiat, Ed}. 
% The fontfamily name is \pfontfam{pbk}.

\item[ITC Bookman\facesubseeidx{Bookman}] {\LXfont{pbk} 은 원래 스코틀랜드의  Miller \&  Richard\index{Miller \& Richard} 공장에서 1860 년에 판매되었다. 
그것은 Alexander Phemister\index{Phemister, Alexander}에 의해 설계되었다. ITC 부흥은 Ed Benguiat\index{Benguiat, Ed}에 의한 것이다. fontfamily 이름은 \pfontfam{pbk} 이다.


\vspace{0.5\onelineskip}
\fox\par\Kafka\par\namesAZ
\vspace{0.5\onelineskip}
}

% \item[Bitstream Charter\facesubseeidx{Charter}] {\LXfont{bch} was designed
% by Matthew Carter\index{Carter, Matthew} for display on low resolution
% devices, and is useful for many applications, including bookwork.
% The fontfamily name is \pfontfam{bch}.

\item[Bitstream Charter\facesubseeidx{Charter}] {\LXfont{bch} 는 Matthew Carter\index{Carter, Matthew}가 저해상도 장치 디스플레이 용으로 설계 한 것으로 
책 제작을 비롯한 많은 응용 분야에 유용하다. fontfamily 이름은 \pfontfam{bch}이다.

\vspace{0.5\onelineskip}
\fox\par\Kafka\par\namesAZ
\vspace{0.5\onelineskip}
}

% \item[Courier\facesubseeidx{Courier}] {\LXfont{pcr} is a monospaced font
% that was originally
% designed by Howard Kettler\index{Kettler, Howard} at IBM and then later
% redrawn by Adrian Frutiger\index{Frutiger, Adrian}.
% The fontfamily name is \pfontfam{pcr}.

\item[Courier\facesubseeidx{Courier}] {\LXfont{pcr} 는 원래 하워드가 설계 한 고정 폭(monospaced) 글꼴이다.
IBM의 Kettler\index{Kettler, Howard}는 나중에 Adrian Frutiger\index{Frutiger, Adrian}가 다시 그려 냈다. fontfamily 이름은 \pfontfam{pcr}이다.

\vspace{0.5\onelineskip}
\fox\par\Kafka\par\namesAZ
\vspace{0.5\onelineskip}
}

% \item[Helvetica\facesubseeidx{Helvetica}] {\LXfont{phv} was originally 
% designed for the Haas
% foundry in Switzerland by Max Miedinger\index{Miedinger, Max}; it was later
% extended by the Stempel foundry and further refined by 
% Linotype\index{Linotype}.
% The fontfamily name is \pfontfam{phv}.

\item[Helvetica\facesubseeidx{Helvetica}] {\LXfont{phv}는 원래 Max Miedinger\index{Miedinger, Max}에 의해 스위스의 Haas 파운드리를 위해 설계되었다. 
그것은 나중에 Stempel 파운드리에 의해 확장되었고 Linotype\index{Linotype}에 의해 더 세련되어졌다. fontfamily 이름은 \pfontfam{phv}이다.

\vspace{0.5\onelineskip}
\fox\par\Kafka\par\namesAZ
\vspace{0.5\onelineskip}
}

% \item[New Century Schoolbook\facesubseeidx{New Century Schoolbook}] 
%   {\LXfont{pnc}
% was designed by Morris Benton\index{Benton, Morris} for ATF (American
% Type Founders) in the early 20th century. As its name implies it was designed
% for maximum legibility in schoolbooks.
% The fontfamily name is \pfontfam{pnc}.

\item[New Century Schoolbook\facesubseeidx{New Century Schoolbook}] {\LXfont{pnc}은 Morris Benton\index{Benton, Morris}이 20 세기 초 ATF (American Type Founders)를 위해 디자인했다. 이름에서 알 수 있듯이 교과서의 가독성을 극대화 할 수 있도록 설계되었다. fontfamily 이름은 \pfontfam{pnc}이다.

\vspace{0.5\onelineskip}
\fox\par\Kafka\par\namesAZ
\vspace{0.5\onelineskip}
}

% \item[Palatino\facesubseeidx{Palatino}] 
%   {\LXfont{ppl} was designed by Hermann 
% Zapf\index{Zapf, Hermann} and is one of the most popular typefaces today.
% The fontfamily name is \pfontfam{ppl}.

\item[Palatino\facesubseeidx{Palatino}] {\LXfont{ppl} 는 Hermann Zapf\index{Zapf, Hermann}가 디자인했으며 오늘날 가장 인기있는 서체 중 하나다. fontfamily 이름은 \pfontfam{ppl}이다.

\vspace{0.5\onelineskip}
\fox\par\Kafka\par\namesAZ
\vspace{0.5\onelineskip}
}

% \item[Times Roman\facesubseeidx{Times Roman}] 
%   {\LXfont{ptm} is Linotype's version
% of the Times New Roman\facesubseeidx{Times New Roman} face designed by 
% Stanley Morison\index{Morison, Stanley} for the
% Monotype Corporation for printing \emph{The Times} newspaper.
% The fontfamily name is \pfontfam{ptm}.

\item[Times Roman\facesubseeidx{Times Roman}] {\LXfont{ptm} 은 \emph{The Times} 신문을 인쇄 할 때 Monotype Corporation에서 Stanley Morison\index{Morison, Stanley}이 디자인 한 Times New Roman\facesubseeidx{Times New Roman} 표지의 Linotype 버전이다. 
fontfamily 이름은 \pfontfam{ptm}이다.

\vspace{0.5\onelineskip}
\fox\par\Kafka\par\namesAZ
\vspace{0.5\onelineskip}
}

\item[Utopia\facesubseeidx{Utopia}]
%  {\LXfont{put} was designed by Robert 
% Slimbach\index{Slimbach, Robert} and combines 
% Transitional\indextwo{type}{Transitional} features and contemporary details.
% The fontfamily name is \pfontfam{put}.
{Utopia는 Robert Slimbach\index{Slimbach, Robert}가 디자인했으며 과도기\indextwo{type}{Transitional}적 기능과 최신 세부 사항을 결합한다. fontfamily 이름은 \pfontfam{put}이다.

\vspace{0.5\onelineskip}
\fox\par\Kafka\par\namesAZ
\vspace{0.5\onelineskip}
}

% \item[ITC Zapf Chancery\facesubseeidx{Zapf Chancery}] 
%  {\LXfont{pzc} is a 
% Script\indextwo{type}{Script} type fashioned after the chancery
% handwriting styles of the Italian Renaissance. It was created by
% Hermann Zapf\index{Zapf, Hermann}.
% The fontfamily name is \pfontfam{pzc}.

\item[ITC Zapf Chancery\facesubseeidx{Zapf Chancery}] {\LXfont{pzc} 는 이탈리아 르네상스의 찬사 스타일에 필적하는 스크립트\indextwo{type}{Script} 유형이다. 
그것은 Hermann Zapf{Zapf, Hermann}에 의해 만들어졌다. fontfamily 이름은 \pfontfam{pzc}이다.

\vspace{0.5\onelineskip}
\fox\par\Kafka\par\namesAZ
\vspace{0.5\onelineskip}
}

% \item[Symbol\facesubseeidx{Symbol}]
%   {\LXfont{psy} contains various symbols and Greek letters for 
%   mathematical work; these are most easily accessible via
%   the \Lpack{pifont} package.
% The fontfamily name is \pfontfam{psy}.
\item[Symbol\facesubseeidx{Symbol}] {\LXfont{psy} 는 다양한 기호와 수학적 작업을위한 그리스 문자를 포함한다. 
이들은 \Lpack{pifont} 패키지를 통해 가장 쉽게 접근 할 수 있다. fontfamily 이름은 \pfontfam{psy}이다. 
%     The available glyphs are shown in \tref{tab:symbolglyphs}.
사용 가능한 글리프는 \tref{tab:symbolglyphs}에 나와 있다.
}

% \item[Zapf Dingbats]\facesubseeidx{Zapf Dingbats}
%   {\LXfont{pzd} contains a variety of dingbats which, like the Symbol
%    characters, are most easily 
%    accessible via the \Lpack{pifont} package.
% The fontfamily name is \pfontfam{pzd}.
\item[Zapf Dingbats]\facesubseeidx{Zapf Dingbats} {\LXfont{pzd} 에는 Symbol 문자와 마찬가지로 \Lpack{pifont} 패키지를 통해 가장 쉽게 액세스 할 수있는 다양한 dingbats가 포함되어 있다. 
fontfamily 이름은 \pfontfam{pzd}이다. 
%  The available glyphs are shown in \tref{tab:dingglyphs}.
  사용 가능한 글리프는 \tref{tab:dingglyphs}에 나와 있다.
}
\end{description}

%    In \ltx\ there are three characteristics that apply to a font. These are:
% (a)~the shape\index{font characteristic!shape}, 
% (b)~the series\index{font characteristic!series} 
% (or weight), %\index{font characteristic!weight|see{series}}, 
% and (c)~the family\index{font characteristic!family}. 
% Table \ref{tab:fontcat} illustrates these and lists the relevant 
% commands\index{font commands} to access the different font categories.

\ltx\ 에는 글꼴에 적용되는 세 가지 특성이 있다. 이것들은 : 
(a)~모양(shape)\index{font characteristic!shape}, 
(b)~시리즈(series)(또는 무게(weight))\index{font characteristic!series},  %\index{font characteristic!weight|see{series}}, 
(c)~가족(family)\index{font characteristic!family}
 표 \ref{tab:fontcat} 는 이러한 내용을 보여 주며 다른 글꼴 카테고리에 액세스하는 데 필요한 명령 \index{font commands}을 나열한다.

\begin{table}
\centering
\caption{Font categorisation and commands} \label{tab:fontcat}
\begin{tabular}{ll} \toprule
\multicolumn{2}{c}{Shape}\index{font characteristic!shape} \\ \addlinespace
\textup{Upright shape}     & \cmd{\textup}\verb?{Upright shape}? \\
\textit{Italic shape}      & \cmd{\textit}\verb?{Italic shape}? \\
\textsl{Slanted shape}     & \cmd{\textsl}\verb?{Slanted shape}? \\
\textsc{Small Caps shape}  & \cmd{\textsc}\verb?{Small Caps shape}? \\ \addlinespace
\multicolumn{2}{c}{Series or weight}\index{font characteristic!series} \\ \addlinespace
\textmd{Medium series}     & \cmd{\textmd}\verb?{Medium series}? \\
\textbf{Bold series}       & \cmd{\textbf}\verb?{Bold series}? \\ \addlinespace
\multicolumn{2}{c}{Family}\index{font characteristic!family} \\ \addlinespace
\textrm{Roman family}      & \cmd{\textrm}\verb?{Roman family}? \\ 
\textsf{Sans serif family} & \cmd{\textsf}\verb?{Sans serif family}? \\ 
\texttt{Typewriter family} & \cmd{\texttt}\verb?{Typewriter family}? \\ 
\bottomrule
\end{tabular}
\glossary(textup)%
  {\cs{textup}\marg{text}}%
  {Typeset \meta{text} with an upright font.}
\glossary(textit)%
  {\cs{textit}\marg{text}}%
  {Typeset \meta{text} with an italic font.}
\glossary(textsl)%
  {\cs{textsl}\marg{text}}%
  {Typeset \meta{text} with a slanted (oblique) font.}
\glossary(textsc)%
  {\cs{textsc}\marg{text}}%
  {Typeset \meta{text} with a small caps font.}
\glossary(textmd)%
  {\cs{textmd}\marg{text}}%
  {Typeset \meta{text} with a medium font.}
\glossary(textbf)%
  {\cs{textbf}\marg{text}}%
  {Typeset \meta{text} with a bold font.}
\glossary(textrm)%
  {\cs{textrm}\marg{text}}%
  {Typeset \meta{text} with a Roman font.}
\glossary(textsf)%
  {\cs{textsf}\marg{text}}%
  {Typeset \meta{text} with a Sans serif font.}
\glossary(texttt)%
  {\cs{texttt}\marg{text}}%
  {Typeset \meta{text} with a Typewriter (monospaced) font.}
\end{table}

%     The normal body font\index{body font} --- the font used for the bulk 
% of the text ---
% is an upright, medium, roman font of a size specified by the font size
% option for the \cmd{\documentclass}.

일반 본문 글꼴\index{body font}(텍스트 본문에 사용되는 글꼴) --- 대부분의 텍스트에 사용되는 글꼴 --- 은 \cmd{\documentclass}에 대한 글꼴 크기 옵션으로 지정된 크기의 직립, 중간, 로마자 글꼴이다.

\begin{syntax}
\cmd{\normalfont} \\
\end{syntax}
\glossary(normalfont)%
  {\cs{normalfont}}%
  {Declaration setting the font to the normal body font (upright, Roman,
and medium weight).} 
% The declaration \cmd{\normalfont} sets the font to be the normal body font.
\cmd{\normalfont}선언은 글꼴을 일반 본문 글꼴로 설정한다.

%    There is a set of font declarations\index{font declarations}, as shown in \tref{tab:fontdecl},
% that correspond to the commands listed in \tref{tab:fontcat}. The commands
% are most useful when changing the font for a word or two, while the 
% declarations are more convenient when you want to typeset longer passages
% in a different font.

\tref{tab:fontdecl} 에 표시된대로 \tref{tab:fontcat} 에 나열된 명령에 해당하는 글꼴 선언 세트\index{font declarations}가 있다.
명령은 한 두 단어의 글꼴을 변경할 때 가장 유용하지만, 다른 글꼴로 긴 구절을 입력하려고 할 때 선언이 더 편리하다.

\begin{table}
\centering
\caption{Font declarations} \label{tab:fontdecl}
\begin{tabular}{ll} \toprule
\multicolumn{2}{c}{Shape}\index{font characteristic!shape} \\ \addlinespace
\textup{Upright shape}     & \verb?{?\cmd{\upshape} \verb?Upright shape}? \\
\textit{Italic shape}      & \verb?{?\cmd{\itshape} \verb?Italic shape}? \\
\textsl{Slanted shape}     & \verb?{?\cmd{\slshape} \verb?Slanted shape}? \\
\textsc{Small Caps shape}  & \verb?{?\cmd{\scshape} \verb?Small Caps shape}? \\ \addlinespace
\multicolumn{2}{c}{Series or weight}\index{font characteristic!series} \\ \addlinespace
\textmd{Medium series}     & \verb?{?\cmd{\mdseries} \verb?Medium series}? \\
\textbf{Bold series}       & \verb?{?\cmd{\bfseries} \verb?Bold series}? \\ \addlinespace
\multicolumn{2}{c}{Family}\index{font characteristic!family} \\ \addlinespace
\textrm{Roman family}      & \verb?{?\cmd{\rmfamily} \verb?Roman family}? \\ 
\textsf{Sans serif family} & \verb?{?\cmd{\sffamily} \verb?Sans serif family}? \\ 
\texttt{Typewriter family} & \verb?{?\cmd{\ttfamily} \verb?Typewriter family}? \\ 
\bottomrule
\end{tabular}
\glossary(upshape)%
  {\cs{upshape}}%
  {Declaration for using an upright font.}
\glossary(itshape)%
  {\cs{itshape}}%
  {Declaration for using an italic font.}
\glossary(slshape)%
  {\cs{slshape}}%
  {Declaration for using a slanted font.}
\glossary(scshape)%
  {\cs{scshape}}%
  {Declaration for using a small caps font.}
\glossary(mdseries)%
  {\cs{mdseries}}%
  {Declaration for using a medium font.}
\glossary(bfseries)%
  {\cs{bfseries}}%
  {Declaration for using a bold font.}
\glossary(rmfamily)%
  {\cs{rmfamily}}%
  {Declaration for using a Roman font.}
\glossary(sffamily)%
  {\cs{sffamily}}%
  {Declaration for using a Sans serif font.}
\glossary(ttfamily)%
  {\cs{ttfamily}}%
  {Declaration for using a Typewriter (monospaced) font.}
\end{table}

%  Do not go wild seeing how many different kinds of fonts you can cram into
% your work as in example~\ref{egbadmf}.
~\ref{egbadmf} 에서와 같이 자신의 작업에 굳이 굉장히 많은 글꼴을 넣으려고 하지는 말자.

\begin{egsource}{egbadmf}
Mixing \textbf{different series, \textsf{families}} and
\textsl{\texttt{shapes,}} \textsc{especially in one sentence,} 
is usually \emph{highly inadvisable!}
\end{egsource}
\index{mixing fonts}
\begin{egresult}[Badly mixed fonts]{egbadmf}
Mixing \textbf{different series, \textsf{families}} and
\textsl{\texttt{shapes,}} \textsc{especially in one sentence,} 
is usually \emph{highly inadvisable!}
\end{egresult}

%    On the other hand there are occasions when several fonts may be used
% for a reasonable effect, as in example~\ref{eg16}.
반면에 예제 ~\ref{eg16}에서와 같이 효과적인 효과를 내기 위해 여러 글꼴을 사용할 수있는 경우가 있다.

\begin{egsource}{eg16}
\begin{center}
\textsc{Des Dames du Temps Jadis}
\end{center}%
\settowidth{\versewidth}{Or yet in a year where they are}
\begin{verse}[\versewidth] \begin{itshape}
Prince, n'enquerez de sepmaine \\*
Ou elles sont, ne de cest an, \\*
Qu'a ce reffrain ne vous remaine: \\*
Mais ou sont les neiges d'antan?
\end{itshape}

Prince, do not ask in a week \\*
Or yet in a year where they are, \\*
I could only give you this refrain: \\*
But where are the snows of yesteryear?
\end{verse}
\begin{flushright}
{\bfseries Fran\c{c}ois Villon} [1431--1463?]
\end{flushright}
\end{egsource}

\begin{egresult}[Sometimes mixed fonts work]{eg16}
\begin{center}
\textsc{Des Dames du Temps Jadis}
\end{center}%
\settowidth{\versewidth}{Or yet in a year where they are}
\begin{verse}[\versewidth] \begin{itshape}
Prince, n'enquerez de sepmaine \\* \index[lines]{Prince, n'enquerez de sepmaine}
Ou elles sont, ne de cest an, \\*
Qu'a ce reffrain ne vous remaine: \\*
Mais ou sont les neiges d'antan?
\end{itshape}

Prince, do not ask in a week \\* \index[lines]{Prince, do not ask in a week}
Or yet in a year where they are, \\*
I could only give you this refrain: \\*
But where are the snows of yesteryear?
\end{verse}
\begin{flushright}
{\bfseries Fran\c{c}ois Villon} [1431--1463?]
\end{flushright}
\end{egresult}

\begin{syntax}
\cmd{\emph}\marg{text} \\
\end{syntax}
\glossary(emph)%
  {\cs{emph}\marg{text}}%
  {Use a change in font to emphasise \meta{text}.}
% The \cmd{\emph}\index{emphasis} command is a font changing command that 
% does not fit
% into the above scheme of things. What it does is to typeset its \meta{text}
% argument using a different font than the surrounding text. By default,
% \cmd{\emph} switches between an upright shape and an italic shape. The
% commands can be nested to produce effects like those in the next example.

\cmd{\emph}\index{emphasis} 명령은 위의 계획에 맞지 않는 글꼴 변경 명령이다. 
그것은 주변 텍스트와 다른 글꼴을 사용하여 텍스트 인수를 조판하는 것이다. 기본적으로 \cmd{\emph}는 직립 모양과 기울임 꼴 사이를 전환한다.
다음 예제와 같은 효과를 내기 위해 명령을 중첩 할 수 있다.

\begin{egsource}{eg:emph}
The \verb?\emph? command is used to produce some text that 
should be \emph{emphasised for some reason and can be
\emph{infrequently interspersed} with some further emphasis} 
just like in this sentence.
\end{egsource}
 
\begin{egresult}[Emphasis upon emphasis]{eg:emph}
The \cmd{\emph} command is used to produce some text that 
should be \emph{emphasised for some reason and can be
\emph{infrequently interspersed} with some further emphasis} 
just like in this sentence.
\end{egresult}


\begin{syntax}
\cmd{\eminnershape}\marg{shape} \\
\end{syntax}
\glossary(eminnershape)%
  {\cs{eminnershape}\marg{shape}}%
  {Font shape for emphasized text within emphasized text.}%
% If the \cmd{\emph} command is used within italic text then the
% newly emphasized text will be typeset using the 
% \cmd{\eminnershape} font shape. The default definition is:
기울임 꼴 텍스트 내에서 \cmd{\emph} 명령을 사용하면 새롭게 강조된 텍스트는 \cmd{\eminnershape} 글꼴 모양을 사용하여 조판된다.
기본 정의는 다음과 같다 : 
\begin{lcode}
\newcommand*{\eminnershape}{\upshape}
\end{lcode}
% which you can change if you wish.
원하는 경우 변경할 수 있다.



\section{Font sizes}

    The Computer Modern \metafont{} fonts come in a fixed number of 
sizes\index{font size}, with
each size being subtly different in shape so that they blend harmoniously.
Traditionally, characters were designed for each size to be cut, and
Computer Modern follows the traditional type design. For example, the smaller
the size the more likely that the characters will have a relatively larger
width.
Outline fonts\indextwo{outline}{font} can be scaled to any size, 
but as the scaling is typically 
linear, different sizes do not visually match quite as well.

    Computer Modern fonts come in twelve sizes which, rounded to a point,
are: 5, 6, 7, 8, 9, 10, 11, 12, 14, 17, 20 and 25pt.
   In \ltx\ the size for a particular font is specifed by a macro name like
\cs{scriptsize} and not by points; for example \cs{scriptsize}, not 
7pt.\footnote{It is possible to use points but that is outside the scope
of this manual.} 
The actual size of, say, \cs{scriptsize} characters, is not fixed but depends
on the type size option given for the document. 

\begin{table}
\centering
\caption{Standard font size declarations} \label{tab:fontsize}
\begin{tabular}{llcll} \toprule
\cmd{\tiny} & {\tiny tiny} & & \cmd{\scriptsize} & {\scriptsize scriptsize} \\[5pt]
\cmd{\footnotesize} & {\footnotesize footnotesize} & & \cmd{\small} & {\small small} \\[5pt]
\cmd{\normalsize} & {\normalsize normalsize} & & \cmd{\large} & {\large large} \\[5pt]
\cmd{\Large} & {\Large Large} & & \cmd{\LARGE} & {\LARGE LARGE} \\[5pt]
\cmd{\huge} & {\huge huge} & & \cmd{\Huge} & {\Huge Huge} \\[5pt]
\bottomrule
\end{tabular}
\end{table}

 Standard
\ltx\ provides ten
declarations, illustrated in \tref{tab:fontsize}, for setting the type size, 
which means that two of the sizes are
not easily accessible. Which two depend on the class and the 
selected point size option. However, for normal typesetting four different
sizes should cover the majority of needs, so there is plenty of scope with
a mere ten to choose from.

\begin{table}
\centering
\caption{Standard font sizes} \label{tab:standardclassfontsize}
\begin{tabular}{lrrr} \toprule
Class option        & 10pt & 11pt & 12pt \\ \midrule
\cmd{\tiny}         &  5pt &  6pt &  6pt \\
\cmd{\scriptsize}   &  7pt &  8pt &  8pt \\
\cmd{\footnotesize} &  8pt &  9pt & 10pt \\
\cmd{\small}        &  9pt & 10pt & 11pt \\
\cmd{\normalsize}   & 10pt & 11pt & 12pt \\ 
\cmd{\large}        & 12pt & 12pt & 14pt \\
\cmd{\Large}        & 14pt & 14pt & 17pt \\
\cmd{\LARGE}        & 17pt & 17pt & 20pt \\
\cmd{\huge}         & 20pt & 20pt & 25pt \\
\cmd{\Huge}         & 25pt & 25pt & 25pt \\ 
\bottomrule
\end{tabular}
\end{table}

    The \cmd{\normalsize} is the size that is set as the class 
option\index{class option} and is the size used for the body\index{body font} 
text. The \cmd{\footnotesize} is the size normally
used for typesetting footnotes\index{footnote}. The standard classes 
use the other sizes, usually the larger ones, for typesetting certain 
aspects of a document, for example sectional headings. 

With respect to the standard classes, the \Mname\ class 
provides a wider range of the document class type size options and
adds two extra font size declarations, namely \cmd{\miniscule}
and \cmd{\HUGE}, one at each end of the range.

    The \Mname\ class font size declarations names are given in 
\tref{tab:fsizenames} together with the name set in the specified size 
relative to the manual's \cs{normalsize} font.
font. The corresponding actual sizes are given in \tref{tab:fsizepoints}.

\begin{table}
\centering 
\caption{The \Mname\ class font size declarations} \label{tab:fsizenames}
\begin{tabular}{llll} \toprule
\cmd{\miniscule} & {\miniscule miniscule} & \cmd{\tiny} & {\tiny tiny} \\
\cmd{\scriptsize} & {\scriptsize scriptsize} & \cmd{\footnotesize} & {\footnotesize footnotesize} \\
\cmd{\small} & {\small small} & \cmd{\normalsize} & {\normalsize normalsize} \\
\cmd{\large} & {\large large} & \cmd{\Large} & {\Large\strut Large} \\ 
\cmd{\LARGE} & {\LARGE LARGE} & \cmd{\huge} & {\huge\strut huge} \\ 
\cmd{\Huge} & {\Huge Huge}    & \cmd{\HUGE} & {\HUGE\strut HUGE} \\ \bottomrule
\end{tabular}
\end{table}

\begin{table}
\begin{adjustwidth}{-1in}{-1in}
\centering
\caption{The \Mname\ class font sizes} \label{tab:fsizepoints}
\begin{tabular}{lrrrrrrrrrrrr}\toprule
Class option & \Lopt{9pt} & \Lopt{10pt} & \Lopt{11pt} & \Lopt{12pt} & \Lopt{14pt} & \Lopt{17pt} & \Lopt{20pt} & \Lopt{25pt} & \Lopt{30pt} & \Lopt{36pt} & \Lopt{48pt} & \Lopt{60pt} \\ \midrule
\cmd{\miniscule}    &  4pt &  5pt &  6pt &  7pt &  8pt &  9pt & 
                      10pt & 11pt & 12pt & 14pt & 17pt & 20pt \\
\cmd{\tiny}         &  5pt &  6pt &  7pt &  8pt &  9pt & 10pt &
                      11pt & 12pt & 14pt & 17pt & 20pt & 25pt \\
\cmd{\scriptsize}   &  6pt &  7pt &  8pt &  9pt & 10pt & 11pt &
                      12pt & 14pt & 17pt & 20pt & 25pt & 30pt \\
\cmd{\footnotesize} &  7pt &  8pt &  9pt & 10pt & 11pt & 12pt &
                      14pt & 17pt & 20pt & 25pt & 30pt & 36pt \\
\cmd{\small}        &  8pt &  9pt & 10pt & 11pt & 12pt & 14pt &
                      17pt & 20pt & 25pt & 30pt & 36pt & 48pt \\
\cmd{\normalsize}   & 9pt & 10pt & 11pt & 12pt & 14pt & 17pt &
                      20pt & 25pt & 30pt & 36pt & 48pt & 60pt \\
\cmd{\large}        & 10pt & 11pt & 12pt & 14pt & 17pt & 20pt &
                      25pt & 30pt & 36pt & 48pt & 60pt & 72pt \\
\cmd{\Large}        & 11pt & 12pt & 14pt & 17pt & 20pt & 25pt &
                      30pt & 36pt & 48pt & 60pt & 72pt & 84pt \\
\cmd{\LARGE}        & 12pt & 14pt & 17pt & 20pt & 25pt & 30pt &
                      36pt & 48pt & 60pt & 72pt & 84pt & 96pt \\
\cmd{\huge}         & 14pt & 17pt & 20pt & 25pt & 30pt & 36pt &
                      48pt & 60pt & 72pt & 84pt & 96pt & 108pt \\
\cmd{\Huge}         & 17pt & 20pt & 25pt & 30pt & 36pt & 48pt &
                      60pt & 72pt & 84pt & 96pt & 108pt & 120pt \\
\cmd{\HUGE}         & 20pt & 25pt & 30pt & 36pt & 48pt & 60pt & 
                      72pt & 84pt & 96pt & 108pt & 120pt & 132pt \\ \bottomrule
\end{tabular}
\end{adjustwidth}
\end{table}

    Whereas the standard font sizes range from 5pt to 25pt, \Mname\  
provides for fonts ranging from 4pt to 132pt. That is: \par
{\fontsize{4}{5}\selectfont from the 4pt size (the 9pt \cs{miniscule} size)} \par
{\fontsize{9}{10}\selectfont through the 9pt normal size\raggedright\par}% \par
{\fontsize{60}{72}\selectfont through the 60pt normal size\raggedright\par}% \par
{\fontsize{132}{160}\selectfont \raggedright to the 132pt size (the 60pt
\cs{HUGE} size).\raggedright\par}


This extended range, though,
is only accessible if you are using outline\indextwo{outline}{font} fonts
and the \Lopt{extrafontsizes} class option.
If you are using bitmap fonts\indextwo{bitmap}{font} then, for example,
the \cmd{\HUGE} font will be automatically limited to 25pt, and the minimum
size of a \cmd{\miniscule} font is 5pt.


\section{Spaces}

\subsection{Paragraphs}

    In traditional typography the first line of a paragraph, unless it comes
immediately after a chapter or section heading, is indented. Also, there
is no extra space between paragraphs. Font designers go to great pains
to ensure that they look good when set with the normal leading. Sometimes,
such as when trying to meet a University's requirements for the layout
of your thesis, you may be forced to ignore the experience of centuries.

\index{space!inter-paragraph|(}
   If you like the idea of eliminating paragraph indentation and using extra
inter-paragraph space to indicate where paragraphs start and end, consider how
confused your reader will be if the last paragraph on the page ends with a full
line; how will the reader know that a new paragraph starts at the top of
the following page?

\begin{syntax}
\cs{par} \\
\lnc{\parskip} \\
\cmd{\abnormalparskip}\marg{length} \\
\cmd{\nonzeroparskip} \\
\cmd{\traditionalparskip} \\
\end{syntax}
\glossary(par)%
  {\cs{par}}%
  {Ends a paragraph.}
\glossary(parksip)%
  {\cs{parskip}}%
  {Space between paragraphs.}
\glossary(abnormalparskip)%
  {\cs{abnormalparskip}\marg{length}}%
  {Sets the inter-paragraph spacing to \meta{length}.}
\glossary(nonzeroparskip)%
  {\cs{nonzeroparskip}}%
  {Sets the inter-paragraph spacing to a `perhaps not too unreasonable' non-zero value.}
\glossary(traditionalparskip)%
  {\cs{traditionalparskip}}%
  {Sets the interparagraph spacing to its traditional value.}
In the input text the end of a paragraph is indicated either by leaving
a blank line, or by the \cs{par} command. 
The length \lnc{\parskip} is the inter-paragraph spacing, and is normally 0pt.
You can change this by saying, for example:
\begin{lcode}
\setlength{\parskip}{2\baselineskip}
\end{lcode}
but you are likely to find that many things have changed that you
did not expect, because \ltx\ uses the \cs{par} command in many
places that are not obvious.

If, in any event, you wish to do a disservice to your readers you can use 
\cmd{\abnormalparskip} to set the inter-paragraph spacing to
a value of your own choosing. Using the \cmd{\nonzeroparskip} will set
the spacing to  what might be a reasonable non-zero value.\footnote{Except
that all values except zero are unreasonable.} 
Both these macros try and eliminate the worst of the side effects that occur
if you just simply change \lnc{\parskip} directly.

Following the \cmd{\traditionalparskip} declaration all will be returned 
to their traditional values. 

    I based the code for these functions upon the NTG classes~\cite{NTG}
which indicated some of the pitfalls in increasing the spacing. The difficulty
is that \cs{par}, and hence \lnc{\parskip}, occurs in many places, some
unexpected and others deeply buried in the overall code.

\index{space!inter-paragraph|)}

\index{space!at start of paragraph|see{paragraph indentation}}

\begin{syntax}
\lnc{\parindent} \\
\end{syntax}
\glossary(parindent)%
  {\cs{parindent}}
  {Indentation at the start of the first line of a paragraph.}
\index{paragraph indentation}
The length \lnc{\parindent} is the indentation at the start of a paragraph's
first line. This is usually of the order of 1 to 1\slashfrac{1}{2} em.
To make the first line of a paragraph flushleft set this to zero:
\begin{lcode}
\setlength{\parindent}{0pt}
\end{lcode}

\subsection{Double spacing}

\index{double spacing|(}

    Some of those that have control over the visual appearance of academic
theses like them to be `double spaced'. This, of course, will make the thesis
harder to read\footnote{I certainly found them so when I was having to read
them before examining the candidates for their degrees. The writers of the
regulations, which were invariably single spaced, seemed immune to any
suggestions.} but perhaps that is the purpose, or maybe they have stock
(shares) in papermills and lumber companies, as the theses were usually 
required to be printed single sided as well.

\begin{syntax}
\lnc{\baselineskip} \lnc{\onelineskip} \\
\end{syntax}
The length \lnc{\baselineskip} is the space, or 
leading\index{leading}, between the baselines of adjacent text lines, 
and is constant
throughout a paragraph. The value  may change
depending on the size of the current font. More precisely, the 
\cmd{\baselineskip} depends on the font being used at the \emph{end} of 
the paragraph. The class also provides the length \lnc{\onelineskip}
which is the default leading for the normal body font.\footnote{That
  is \cmd{\onelineskip} is set in the \text{memX.clo} file
  corresponding to the font size class option. For \texttt{10pt}, the
  size is set via \texttt{mem10.clo}.} As far as the class
is concerned this is a constant value; that is, unlike \lnc{\baselineskip},
it never alters \lnc{\onelineskip}. You can use (fractions) of 
\lnc{\onelineskip} to specify vertical spaces in terms of normal text 
lines.

    The following is heavily based on the \Lpack{setspace} 
package~\cite{SETSPACE}, but
the names have been changed to avoid any clashes. Like the nonzero 
\lnc{\parskip}, the \lnc{\baselineskip} rears its head in many places, and 
it is hard for a package to get at the internals of the overlying class
and kernel code. This is not to say that all is well with trying to deal
with it at the class level.

\LMnote{2010/09/19}{Added description about the starred macros}
\begin{syntax}
\cmd{\OnehalfSpacing} \cmd{\OnehalfSpacing*} \\
\cmd{\DoubleSpacing} \cmd{\DoubleSpacing*}\\
\end{syntax}
\glossary(OnehalfSpacing)%
  {\cs{OnehalfSpacing}}%
  {Declaration increasing the baseline to create the illusion of double spacing.}
\glossary(OnehalfSpacing*)%
  {\cs{OnehalfSpacing*}}%
  {Same as \cs{OnehalfSpacing} but also effects page note and floats.}
\glossary(DoubleSpacing)%
  {\cs{DoubleSpacing}}%
  {Declaration doubling the baselineskip.}
\glossary(DoubleSpacing*)%
  {\cs{DoubleSpacing*}}%
  {Same as \cs{DoubleSpacing} but also effects page notes and floats.}

  The declaration \cmd{\OnehalfSpacing} increases the spacing between
  lines so that they appear to be double spaced (especially to the
  thesis layout arbiters), while the declaration \cmd{\DoubleSpacing}
  really doubles the spacing between lines which really looks bad; but
  if you have to use it, it is there. The spacing in footnotes and
  floats (e.g., captions) is unaltered, which is usually required once
  the controllers see what a blanket double spacing brings. Sometimes
  it is also required to make page notes and floats (including
  captions) in `double' spacing. The starred version of the two macros
  above takes care of this. Alternatively the spacing in page notes
  (i.e. footnotes and friends) or floats an be set explicitly using
\begin{syntax}
  \cmd{\setPagenoteSpacing}\marg{factor}\\
  \cmd{\setFloatSpacing}\marg{factor}
\end{syntax}
\glossary(setFloatSpacing)%
  {\cs{setFloatSpacing}}%
  {Explicitly set the spacing used inside floats.}
\glossary(setPagenoteSpacing)%
  {\cs{setPagenoteSpacing}}%
  {Explicitly set the spacing used inside page notes such including footnotes.}
\cs{setFloatSpacing} should go after say \cs{OnehalfSpacing*} if used.



\PWnote{2009/06/24}{Corrected \cs{SetSingleSpace} to \cs{setSingleSpace}}
\begin{syntax}
  \cmd{\SingleSpacing}\\
  \cmd{\SingleSpacing*}\\
  \cmd{\setSingleSpace}\marg{factor} \\
\end{syntax}
\glossary(SingleSpacing)%
  {\cs{SingleSpacing}}%
  {Declaration restoring normal single spacing (or that set by
    \cs{setSingleSpace}). Note, this added a skip at the end.}
\glossary(SingleSpacing*)%
  {\cs{SingleSpacing*}}%
  {Same as \cs{SingleSpacing}, but does not add a skip at the end.}
\glossary(setSingleSpace)%
  {\cs{setSingleSpace}\marg{factor}}%
  {Change the baselineskip by \meta{factor}.}
The \cmd{\setSingleSpace} command is meant to be used to
adjust \emph{slightly} the normal spacing betwen lines, perhaps because
the font being used looks too crampled or loose. The effect is that the 
normal \lnc{\baselineskip} spacing will be multiplied by \meta{factor}, which
should be close to 1.0. Using \cmd{\setSingleSpace} will also reset
the float and page note spacings.

The declaration \cmd{\SingleSpacing} returns everthing to normal, or at
least the setting from \cmd{\setSingleSpace} if it has been used. It
will also reset float and page note spacings to the same value.

\begin{note}
  \sloppy
  \cmd{\SingleSpacing} will also issue a
  \cmd{\vskip}\cmd{\baselineskip} at the end (which is ignored if
  \cmd{\SingleSpacing} is used in the preamble). This skip makes sure
  that comming from \cmd{\DoubleSpacing} to \cmd{\SingleSpacing} still
  looks ok.

  But in certain cases, this skip is unwanted. Therefore as of 2018 we
  added a \cmd{\SingleSpacing*} that is equal to \cmd{\SingleSpacing}
  but \emph{does not} add this skip.
\end{note}


\begin{syntax}
\senv{SingleSpace} ...\eenv{SingleSpace} \\
\senv{Spacing}\marg{factor} ... \eenv{Spacing} \\
\senv{OnehalfSpace} ... \eenv{OnehalfSpace} \\
\senv{OnehalfSpace*} ... \eenv{OnehalfSpace*} \\
\senv{DoubleSpace} ... \eenv{DoubleSpace} \\
\senv{DoubleSpace*} ... \eenv{DoubleSpace*} \\
\end{syntax}
\glossary(SingleSpace)%
  {\senv{SingleSpace}}%
  {Environment form of \cs{SingleSpacing}}
\glossary(Spacing)%
  {\senv{SingleSpace}\marg{factor}}%
  {Environment form of \cs{setSingleSpace}}
\glossary(OnehalfSpace)%
  {\senv{OnehalfSpace}}%
  {Environment form of \cs{OnehalfSpacing}}
\glossary(OnehalfSpace*)%
  {\senv{OnehalfSpace*}}%
  {Environment form of \cs{OnehalfSpacing*}}
\glossary(DoubleSpace)%
  {\senv{DoubleSpace}}%
  {Environment form of \cs{DoubleSpacing}}
\glossary(DoubleSpace*)%
  {\senv{DoubleSpace*}}%
  {Environment form of \cs{DoubleSpacing*}}

These are the environments corresponding to the declarations presented
earlier, for when you want to change the spacing locally.

\begin{syntax}
\cmd{\setDisplayskipStretch}\marg{fraction} \\
\cmd{\memdskipstretch} \\
\cmd{\noDisplayskipStretch} \\
\cmd{\memdskips} \\
\end{syntax}
\glossary(setDisplayskipStretch)%
  {\cs{setDisplayskipStretch}\marg{factor}}%
  {Increase the display skips by gmeta{factor}.}%
\glossary(noDisplayskipStretch)%
  {\cs{noDisplayskipStretch}}%
  {No increased display skips.}%
\glossary(memdskipstretch)%
  {\cs{memdskipstretch}}%
  {The current factor for increasing display skips.}%
\glossary(memdskips)%
  {\cs{memdskips}}%
  {Adjusts the display skips according to \cs{memdskipstretch}.}%


If you have increased the interlinear space in the text you may wish, or be
required, to increase it around displays (of maths). 
The declaration \cmd{\setDisplayskipStretch} will increase the before 
and after displayskips by \meta{fraction}, which must be at least 0.0. 
More precisely, it defines \cmd{\memdskipstretch} to be \meta{fraction}.
The \cmd{\noDisplayskipStretch} declaration
sets the skips back to their normal values. It is equivalent to
\begin{lcode}
\setDisplayskipStretch{0.0}
\end{lcode}
The skips are changed within the macro \cmd{\memdskips} which, in turn, is
called by \cmd{\everydisplay}. If you find odd spacing around displays then
redefine \cmd{\memdskips} to do nothing. Its orginal specification is:
\begin{lcode}
\newcommand*{\memdskips}{%
  \advance\abovedisplayskip \memdskipstretch\abovedisplayskip
  \advance\belowdisplayskip \memdskipstretch\belowdisplayskip
  \advance\abovedisplayshortskip \memdskipstretch\abovedisplayshortskip
  \advance\belowdisplayshortskip \memdskipstretch\belowdisplayshortskip}
\end{lcode}

    If you need to use a \Ie{minipage} as a stand-alone item in a widely 
spaced text then you
may need to use the \Ie{vminipage} environment instead to get the before and
after spacing correct.


\section{Overfull lines}

\index{overfull lines|(}
\index{space!interword|(}


    \tx\ tries very hard to keep text lines justified while keeping the 
interword spacing as constant as possible, but sometimes fails and complains
about an overfull hbox.

\begin{syntax}
\cmd{\fussy} \cmd{\sloppy} \\
\senv{sloppypar} ... \eenv{sloppypar} \\
\cmd{\midsloppy} \\
\senv{midsloppypar} ... \eenv{midsloppypar} \\
\end{syntax}
\glossary(fussy)%
  {\cs{fussy}}%
  {Declaration for \tx\ to minimise interword space variations in justified text lines.}
\glossary(sloppy)%
  {\cs{sloppy}}%
  {Declaration for \tx\ to allow large interword space variations in justified 
   text lines.}%
\glossary(sloppypar)%
  {\senv{sloppypar}}%
  {Typeset contents of the enclosed paragraph(s) using \cs{sloppy}.}%
\glossary(midsloppy)%
  {\cs{midsloppy}}%
  {Declaration for \tx\ to allow moderate interword space variations in justified 
   text lines.}%
\glossary(midsloppypar)%
  {\senv{midsloppypar}}%
  {Typeset contents of the enclosed paragraph(s) using \cs{midsloppy}.}%

    The default mode for \ltx\ typesetting is \cmd{\fussy} where 
the (variation of) interword spacing in justified text is kept to a 
minimum. Following the \cmd{\sloppy}
declaration there may be a much looser setting of justified text. 
The \Ie{sloppypar} environment is equivalent to:
\begin{lcode}
{\par \sloppy ... \par}
\end{lcode}

    Additionally the class provides the \cmd{\midsloppy} declaration (and the
\Ie{midsloppypar} environment) which allows a setting somewhere between 
\cmd{\fussy} and \cmd{\sloppy}. Using \cmd{\midsloppy} you will get fewer 
overfull lines compared with \cmd{\fussy} and fewer obvious large 
interword spaces than with \cmd{\sloppy}.
I have used \cmd{\midsloppy} for this manual; it hasn't prevented 
overfull lines or noticeably different interword spaces, but has markedly 
reduced them compared with \cmd{\fussy} and \cmd{\sloppy} respectively.

\index{overfull lines|)}
\index{space!interword|)}

\section{Sloppybottom}

\indextwo{widow}{line}\indextwo{orphan}{line}
    \tx\ does its best to avoid widow and orphan lines --- a widow is where 
the last line of a paragraph ends up at the top of a page, and an 
orphan\footnote{Knuth uses the term `club' instead of the normal typographers'
terminology.} 
is when the first line of a paragraph is at the bottom of a page.

    The following is the generally suggested method of eliminating widows 
and orphans, but it may well result in some odd looking pages, especially
if \cmd{\raggedbottom} is not used.
\begin{lcode}
\clubpenalty=10000
\widowpenalty=10000
\raggedbottom
\end{lcode}

\begin{syntax}
\cmd{\enlargethispage}\marg{length} \\
\end{syntax}
\glossary(enlargethispage)%
  {\cs{enlargethispage}\marg{length}}%
  {Increase (or decrease) the text height of the current page by \meta{length}.}
    You can use \cmd{\enlargethispage} to add or subtract to the text height
on a particular page to move a line forwards or backwards between two pages.


 Here is one person's view on the matter:
\begin{quote}
\ldots in experimenting with raggedbottom, widowpenalty, and clubpenalty,
I think that I have not found a solution that strikes me as particularly
desirable. I think what I would really like is that widows (i.e., left-over
single lines that begin on the following page) are resolved not by pushing
one extra line from the same paragraph also onto the next page, but by
stretching the textheight to allow this one extra at the bottom of the
same page. \\
\hfill /iaw (from \ctt, \textit{widow handling?}, May 2006)
\end{quote}

As so often happens, Donald Arseneau\index{Arseneau, Donald} 
came up with a solution.

\begin{syntax}
\cmd{\sloppybottom} \\
\end{syntax}
\glossary(sloppybottom)%
  {\cs{sloppybottom}}%
  {Declaration for \tx\ to allow an extra line at the bottom of a page.
   The \cs{topskip} must have been increased beforehand.}%
The declaration \cmd{\sloppybottom} lets \tx\ put an extra line at
the bottom of a page to avoid a widow on the following page. 

    The \lnc{\topskip} must have been increased beforehand for this to
work (a 60\% increase is reasonable) and this will push the text
lower on the page. Run \cmd{\checkandfixthelayout} after the change (which
may reduce the number of lines per page). For example, in the preamble:
\begin{lcode}
\setlength{\topskip}{1.6\topskip}
\checkandfixthelayout
\sloppybottom
\end{lcode}

\indextwo{widow}{line}\indextwo{orphan}{line}
    The late Michael Downes\index{Downes, Michael} provided the following 
(from \ctt\ \textit{widow/orphan control package (for 2e)?}, 1998/08/31):
\begin{quote}
    For what it's worth here are the penalty values that I use when I don't
[want] to \emph{absolutely} prohibit widow/orphan break, but come
about as close as \tx\ permits otherwise. This is copied straight out
of some code that I had lying around. I guess I could wrap it into
package form and post it to CTAN. \\
\hfill Michael Downes
\end{quote}

\begin{lcode}
% set \clubpenalty, etc. to distinctive values for use
% in tracing page breaks. These values are chosen so that
% no single penalty will absolutely prohibit a page break, but 
% certain combinations of two or more will.
\clubpenalt=9996
\widowpenalty=9999
\brokenpenalty=4991
% Reiterate the default value of \redisplaypenalty, for
% completeness.
% Set postdisplaypenalty to a fairly high value to discourage a
% page break between a display and a widow line at the end of a
% paragraph.
\predisplaypenalty=10000
\postdisplaypenalty=1549
% And then \displaywidowpenalty should be at least as high as
% \postdisplaypenalty, otherwise in a situation where two displays
% are separated by two lines, TeX will prefer to break between the
% two lines, rather than before the first line.
\displaywidowpenalty=1602
\end{lcode}

\indextwo{widow}{line}\indextwo{orphan}{line}
    As you can see, perfect automatic widow/orphan control is problematic 
though typographers are typically more concerned about widows than orphans ---
a single line of a paragraph somehow looks worse at the top of a page than
at the bottom. If all else fails, the solution is either to live with the 
odd line or to reword the text.

\section{Text case}
\label{sec:text-case}

The standard kernel \cmd{\MakeUppercase}\marg{text} and
\cmd{\MakeLowercase}\marg{text} 
basically upper or lower case everything it can get its hands on. This
is not particularly nice if the \meta{text} contain, say, math.

In order to help with this we provide the \cmd{\MakeTextUppercase} and
\cmd{\MakeTextLowercase} macros from the \Lpack{textcase} package
(\cite{textcase}) by David Carlisle. The following is DCs own
documentation of the provided code changed to match the typography we
use.

\fancybreak{}

\begin{syntax}
  \cmd{\MakeTextUppercase}\marg{text}\\
  \cmd{\MakeTextLowercase}\marg{text}
\end{syntax}
\glossary(MakeTextUppercase)
  {\cs{MakeTextUppercase}\marg{text}}
  {Upper case \meta{text} by leave math, references, citations and
    \cs{NoCaseChange}\marg{text} alone.}
\glossary(MakeTextLowercase)
  {\cs{MakeTextLowercase}\marg{text}}
  {Lower case \meta{text} by leave math, references, citations and
    \cs{NoCaseChange}\marg{text} alone.}
\cmd{\MakeTextUppercase} and \cmd{\MakeTextLowercase} are versions of
the standard \cmd{\MakeUppercase} and \cmd{\MakeLowercase} that do not
change the case of any math sections in their arguments.
\begin{verbatim}
\MakeTextUppercase{abc\ae\ \( a = b \)  and $\alpha \neq a$ 
  or even \ensuremath{x=y} and $\ensuremath{x=y}$}
\end{verbatim}
Should produce:
\begin{quotation}
 ABC\AE\ \( a = b \)  AND $\alpha \neq a$ 
  OR EVEN \ensuremath{x=y} AND $\ensuremath{x=y}$
\end{quotation}

We incorporates some changes suggested by Donald Arseneau so that as
well as math mode, the arguments of \cmd{\cite}, \cmd{\label} and
\cmd{\ref} are also prevented from being uppercased.  So you can now
go
\begin{verbatim}
\MakeTextUppercase{%
   Text in section~\ref{sec:text-case}, about \cite[pp 2--4]{textcase}}
\end{verbatim}
which produces
\begin{quotation}
\MakeTextUppercase{%
   Text in section~\ref{sec:text-case}, about \cite[pp 2--4]{textcase}}
\end{quotation}
If, instead, the standard \cmd{\MakeUppercase} were used here, the keys
`sec:text-case' and `textcase' would be uppercased and generate errors about
undefined references to SEC:TEXT-CASE  and TEXTCASE.

\begin{syntax}
  \cmd{\NoCaseChange}\marg{text}
\end{syntax}
\glossary(NoCaseChange)
 {\cs{NoCaseChange}\marg{text}}
 {The argument of this macro is not touched by \cs{MakeTextUppercase}
   or \cs{MakeTextLowercase}.}
Sometimes there may be a special section of text that should not be
uppercased. This can be marked with \cmd{\NoCaseChange}, as follows.
\begin{verbatim}
\MakeTextUppercase{% 
   Text \NoCaseChange{More Text} yet more text}
\end{verbatim}
which produces
\begin{quotation}
\MakeTextUppercase{%
   Text \NoCaseChange{More Text} yet more text}
\end{quotation}

\cmd{\NoCaseChange} has other uses. If for some reason you need a
tabular environment within an uppercased section, then you need
to ensure that the name `tabular' and the preamble (eg `ll')
does not get uppercased:
\begin{verbatim}
\MakeTextUppercase{%
   Text \NoCaseChange{\begin{tabular}{ll}}%
                       table&stuff\\goes&here
        \NoCaseChange{\end{tabular}}
   More text}
\end{verbatim}
which produces
\begin{quotation}
\MakeTextUppercase{%
   Text \NoCaseChange{\begin{tabular}{ll}}%^^A
                       table&stuff\\goes&here
        \NoCaseChange{\end{tabular}}
   More text}
\end{quotation}

\subsection{Nested text}
The commands defined here only skip math sections and \cmd{\ref} arguments
if they are not `hidden' inside a \verb|{ }| brace group. All text inside
such a group will be made uppercase just as with the standard
\cmd{\MakeUppercase}.
\begin{verbatim}
\MakeTextUppercase{a b {c $d$} $e$}
\end{verbatim}
produces
\begin{quotation}
 \MakeTextUppercase{a b {c $d$} $e$}
\end{quotation}
Of course, this restriction does not apply to the arguments of the
supported commands \verb|\ensuremath|, \verb|\label|, \verb|\ref|, and
\verb|\cite|.

If you cannot arrange for your mathematics to be at the outer level of
brace grouping, you should use the following basic technique (which
works even with the standard \cmd{\MakeUppercase} command). Define a
new command that expands to your math expression, and then use that
command, with \cmd{\protect}, in the text to be uppercased. Note that
if the text being uppercased is in a section title or other moving
argument you may need to make the definition in the document preamble,
rather than just before the section command, so that the command is
defined when the table of contents file is read.
\begin{verbatim}
\MakeTextUppercase{%
       Text \fbox{$a=b$ and $x=y$}}%

\newcommand{\mathexprone}{$a=b$}
\newcommand{\mathexprtwo}{$x=y$}
\MakeTextUppercase{%
       Text \fbox{\protect\mathexprone\ and \protect\mathexprtwo}}%
\end{verbatim}
which produces
\begin{quotation}
\MakeTextUppercase{%
       Text \fbox{$a=b$ and $x=y$}}%

\newcommand{\mathexprone}{$a=b$}
\newcommand{\mathexprtwo}{$x=y$}
\MakeTextUppercase{%
       Text \fbox{\protect\mathexprone\ and \protect\mathexprtwo}}%
\end{quotation}


\fancybreak{}

See also \cite{textcase} for some information about upper casing
citations using a non-nummeric style.


% \subsubsection{Citations}
% As documented above, \cmd{\cite} and \cmd{\ref} commands are not
% uppercased by \cmd{\MakeTextUppercase}. If you are using a non-numeric
% citation scheme you may want the replacement text for \cmd{\cite} to
% be uppercased.

% It is difficult to arrange that \cmd{\MakeTextUppercase} uppercases
% such text, not least because this would lead to interaction with the
% many bibliography packages which redefine \cmd{\cite} one way or
% another. One possibility to achieve this is to use Donald Arseneau's
% cite package and to locally redefine \cmd{\citeform} to add
% \cmd{\MakeUppercase} around the final text string produced by \cmd{\cite}.
% \begin{verbatim}
% \MakeTextUppercase{%
%        Text \cite{bbb} and \cite{ccc}}

% {\renewcommand\citeform{\MakeUppercase}\MakeTextUppercase{%
%        Text \cite{bbb} and \cite{ccc}}}
% \end{verbatim}
% which produces\footnote{This is faked, so this document does not
%  rely on \texttt{cite.sty} being installed}
% \begin{quotation}
%  TEXT [1] AND [David Carlisle 1997]

%  TEXT [1] AND [DAVID CARLISLE 1997]
% \end{quotation}




%#% extend
%#% extstart include titles.tex
