%\chapter{Preface}
\chapter{프레훠스}
%
%    From personal experience and also from lurking on the \url{comp.text.tex}
%newsgroup the major problems with using \ltx\ are related to document
%design. Some years ago most questions on \ctt\ were answered by
%someone providing a piece of code that solved a particular problem, and
%again and again. More recently these questions are answered along the
%lines of `Use the ---------{} package', and again and again.
개인적 경험으로나 \url{comp.tex.tex} 뉴스그룹을 보거나 \ltx 을 사용함에 있어 주요 문제는 
문서 디자인에 관련된 것이다.
몇 년 전만 하더라도 \ctt 에 질문이 올라오면 누군가 주어진 특정 문제를 해결하는 
짧은 코드를 제공하여 답변했고 이 일이 반복되었다.
더 최근에는 답변이 ‘----------{} 패키지를 사용하라’는 것으로 바뀌었고 역시
이런 상황이 반복되고 있다.

%    I have used many of the more common of these packages but my filing system
%is not always well ordered and I tend to mislay the various user manuals,
%even for the packages I have written. The \Pclass{memoir} class is an attempt
%to integrate some of the more design-related packages with the LaTeX
%\Pclass{book} class. I chose the \Pclass{book} class as the \Pclass{report} class
%is virtually identical to \Pclass{book}, except that \Pclass{book} does
%not have an \Ie{abstract} environment while \Pclass{report} does; however it is 
%easy to fake an \Ie{abstract} if it is needed. With a little bit of tweaking,
%\Pclass{book} class documents can be made to look just like \Pclass{article}
%class documents, and the \Pclass{memoir} class is designed with tweaking very
%much in mind.
너무 많은 패키지를 사용하다 보니 순서도 뒤죽박죽이 되고 각 패키지의 사용설명서를 어디에 두었는지 잊어버리게 되었는데, 심지어 내가 작성한 패키지마저 그런 상태가 되었다.
\Pclass{memoir}는 \Pclass{book} 클래스에서의 디자인 관련 패키지를 통합하려는 시도이다.
필자는 \Pclass{report}가 아니라 \Pclass{book} 클래스를 문제삼았다. 이 둘은 사실상 거의 동일한 클래스이고 유일한 차이는 \Pclass{book}에 \Ie{abstract} 환경이 없다는 점이다. 
사실 \Ie{abstract}를 필요하다면 가져다 쓰는 것은 어려운 일도 아니다. 
심지어 조금 손을 보면 \Pclass{book} 클래스 문서를 \Pclass{article} 문서처럼 보이게 하는 것도 가능하다. 이런 관점에서 여러 상황에 맞도록 손보는 일을 유념하여 \Pclass{memoir} 클래스를 작성하였다.

%
%    The \Pclass{memoir} class effectively incorporates the facilties that
%are usually accessed by using external packages. In most cases the class
%code is new code reimplementing package functionalities. The exceptions
%tend to be where I have cut and pasted code from some of my packages.
%I could not have written the \Pclass{memoir} class without the excellent 
%work presented by the implementors of \ltx\ and its many packages.
\Pclass{memoir} 클래스는 외부 패키지와도 대부분 잘 동작한다. 이 클래스로 들여온 각 패키지의 코드는 대부분 새롭게 손을 본 것이다.
다만 필자 자신이 작성한 패키지는 단순히 복사--붙여넣기하였다. 
\ltx 과 여러 패키지의 도움이 없었다면 이 클래스는 작성될 수 없었을 것이다.


%
%    Apart from packages that I happen to have written I have gained many
%ideas from the other packages listed in the \bibname. One way or another
%their authors have all contributed, albeit unknowingly. 
%The participants in the
%\url{comp.text.tex} newsgroup have also provided valuable input, partly
%by questioning how to do something in \ltx, and partly by providing
%answers. It is a friendly and educational forum.
%
직접 활용한 패키지 말고도 \bibname 에 열거된 다양한 패키지의 아이디어를 빌어온 것도 많다.
이 패키지의 저자들은 스스로 인식하지 못하더라도 어떤 식으로든 이 클래스에 기여한 바가 있다.
\url{comp.text.tex} 뉴스그룹의 참여자들 또한 귀중한 기여를 하였는데,
한편으로는 \ltx 에 관한 뭔가를 질문함으로써, 다른 한편으로 답변을 제공함으로써 그리하였다.
매우 분위기좋고 공부가 되는 포럼이었다.


%{\raggedleft{\scshape Peter Wilson} \\ Seattle, WA \\ June 2001\par}
{\raggedleft{\scshape Peter Wilson} \\ Seattle, WA \\ June 2001\par}


%
%%%%%%%%%%%%%%%%%%%%%%%%%%%%%%%%%%%%%%%%%%%%%%%%%%%%%%%%%%%%%%%%%%%%%%%%%%
%\begin{comment}
%\chapter{Introduction to the first edition}
%
%    This is not a guide to the general use of LaTeX but rather concentrates
%on where the \index{class}\Lclass{memoir} class differs from the standard LaTeX
%\Lclass{book} and \Lclass{report} classes. There are other sources that deal with LaTeX in 
%general, some of which are listed in the \bibname. Lamport~\cite{LAMPORT94}
%is of course the original user manual for LaTeX, while the Companion
%series, for example~\cite{COMPANION}, go into further details and auxiliary
%programs. The Comprehensive TeX Archive Network (CTAN) is a valuable source
%of free information and the LaTeX system itself. For general questions see
%the FAQ~\cite{FAQ} which also has pointers to information sources. Among
%these are \btitle{The Not So Short Introduction to LaTeX2e}~\cite{LSHORT},
%Keith Reckdahl's \btitle{Using imported graphics in LaTeX2e}~\cite{EPSLATEX}
%and Piet van Oostrum's \btitle{Page layout in LaTeX}~\cite{FANCYHDR}.
%The question of how to use different fonts with LaTeX is left strictly alone;
%Alan Hoenig's book~\cite{HOENIG98} is the best guide to this that I know of
%although Philipp Lehman's \btitle{The Font Installation Guide}~\cite{FONTINST} 
%has much information.
%
%
%    The first part of the manual discusses briefly some aspects of book
%design and typography, independently of the means of typesetting. Among
%the several books on the subject listed in the \bibname{} I prefer
%Bringhurst's \btitle{The Elements of Typographic Style}~\cite{BRINGHURST99}.
%I have used the LaTeX \Lclass{book} design, which is the default
%\Lclass{memoir} class style, in typesetting Part~\ref{part:art}.
%
%    The second part then goes into some detail on how the \Lclass{memoir}
%class can be used to implement a particular design.
%
%    With two exceptions, the \Lclass{memoir} class has all the capabilities
%of the standard LaTeX \Lclass{book} class. These exceptions are:
%\begin{itemize}
%\item The old LaTeX v2.09 font commands --- 
%      \cmd{\rm} (roman), 
%      \cmd{\tt} (\texttt{typewriter}), 
%      \cmd{\sf} (\textsf{sans}),
%      \cmd{\bf} (\textbf{bold}), 
%      \cmd{\sl} (\textsl{slanted}), 
%      \cmd{\it} (\textit{italic}),
%      and \cmd{\sc} (\textsc{small caps}) ---
%      are not supported and will give error messages if used.
% 
%      The \cmd{\em} (\emph{emphasis}) command is supported but gives 
%a warning message if used.
%
%\item There are no commands for making titles. This is because title pages are
%      usually designed individually for each book. The 
%     \index{package}\Lpack{titling}
%      package~\cite{TITLING} can be used if you want the titling commands.
%      The package
%      provides extended titling facilities when compared to those in the
%      standard LaTeX classes.
%
%\end{itemize}
%I hope that, apart from these, the class supports everything that the 
%\Lclass{book} class provides.
%
%    The major extra functions provided by the class include:
%\begin{itemize}
%\item Font sizes for the main text of 9, 10, 11, 12 and 14pt.
%\item A reasonably intuitive means of setting the page, text and margin sizes.
%\item Page trimming marks.
%\item If really required, typesetting as though in the olden typewriter days
%      (double spaced, raggedright, no hyphenation, typewriter font).
%\item Configurable sectional headings, with several predefined styles for
%      chapter headings and simple methods for defining new ones.
%\item `Anonymous' section breaks (e.g., a blank or decorated line or two).
%\item Configurable page headers and footers, with several predefined styles
%      and simple methods for defining new ones.
%\item Configurable captions, with several predefined captioning styles and
%      methods for defining new ones.
%\item Ability to define new `\listofx' and new floats with their accompanying
%      captions and `\listofx'.
%\item Control over whether the `\listofx', bibliography, index, etc., appear
%      in the Table of Contents.
%\item Support for epigraphs.
%\end{itemize}
%Also, along the way you will find other more minor but still useful things.
%
%    As Part~\ref{part:practice} progresses I demonstrate some of the changes
%that the \Lclass{memoir} class lets you easily make to the normal LaTeX
%page and titling style.
%
%\section{Type conventions}
%
%    The following conventions are used in the manual:
%\begin{itemize}
%\item \Pclass{The names of LaTeX classes\index{class} and 
%              packages\index{package} are typeset in this font,
%             as are class and package options\index{option}.}
%\item \Ppstyle{The names of chapterstyles\index{chapterstyle} and 
%               pagestyles\index{pagestyle} are typeset in this font.}
%\item \texttt{LaTeX code is typeset in this font.}
%\end{itemize}
%
%\section{Caveats}
%
%    At the moment both this manual and the class code are in a beta state.
%That is, they hopefully serve the purposes they are intended for, but 
%it is probable that there are errors, poor explanations and missing
%elements. So, be warned that I'm sure that there will be further releases
%and these may not be entirely compatible with the current release.
%
%    That said, I will be grateful for any constructive comments that 
%anyone\footnote{I have received valuable comments from
%Javier Bezos (\url{jbezos@wanadoo.es}), 
%Sven Bovin (\url{sven.bovin@chem.kuleuven.ac.be}),
%Scott Pakin (\url{pakin@uiuc.edu}),
%and
%Paul Stanley (\url{pstanley@essexcourt-chambers.co.uk}).}
%may have, especially regarding errors, mispeaking, and desireable 
%enhancements. I can be reached by posting to \url{comp.text.tex}.
%
%    TeX was designed principally for typesetting documents containing a 
%lot of mathematics. In such works the mathematics breaks up the flow
%of the text on the page, and the vertical space required for displayed
%mathematics is arbitrary. Most non-technical books are typeset on a fixed
%grid as they do not have arbitrary insertions into the text.
%
%    TeX is designed to handle arbitrary sized inserts in an elegant manner,
%and does this by allowing vertical spaces to stretch and shrink a little
%so that the actual height of the typeblock is constant. Therefore LaTeX, being
%built on top of TeX, does not typeset on a fixed grid, and it would be a 
%very major task to try and make it do so; this has been tried but as far as 
%I know nobody has been successful. Experimental work, though, is still ongoing.
%
%    The manual includes many unbreakable names of LaTeX commands,
%some of which stick out into the margin. The way of getting rid of these
%is to rewrite the text so that they don't come at the end of a typeset
%line. This is tedious and I haven't done it because I expect the manual
%to be revised and that would throw off any hand tweaking done now.
%
%
%\chapter{Introduction to the second edition}
%
%    Since the first edition of the manual was published the \Lclass{memoir}
%class has undergone some changes and extensions. The changes, to remove
%some unfortunate errors, are upwards compatible. The extensions, by 
%their nature, are not upward compatible.
%
%    The main extensions and changes include:
%\begin{itemize}
%\item A \index{option}\Lopt{subfigure} option to counteract an unfortunate 
%      interplay\footnote{Discovered by Ignasi Furi\'{o} Caldentey 
%      (\url{ignasi@ipc4.uib.es}).}
%      if the \Lpack{subfigure} package is used with the class.
%
%\item An \Lopt{article} option so that \Lclass{article} class typesetting
%      may be \emph{simulated}.
%
%\item Incorporation of the essential code from the \Lpack{titling}
%      package~\cite{TITLING} (to support the \Lopt{article} option).
%
%\item Incorporation of the essential code from the \Lpack{abstract}
%      package~\cite{ABSTRACT} (to support the \Lopt{article} option).
%
%\end{itemize}
%
%    The description of how to modify the \prtoc, \prlof{} and \prlot{} headings
%in the first edition was completely wrong, as was the corresponding
%description of the \cmd{\newlistof} macro. No noticeable
%changes have been made to the class code but the descriptions now
%reflect reality. I must have been a few bricks short of a full load when
%I wrote the original.
%
%    There are other more minor changes and extensions\footnote{With thanks
%to, among others, Kevin Lin (\url{kevinlin@runtop.com.tw}) and
%Adriano Pascoletti (\url{pascolet@dimi.uniud.it}).} 
%which you may find if you recall the first edition.
%
%    There was no mention of typesetting verse in the first edition,
%although the class does provide the normal LaTeX \Ie{verse}
%environment. A poem
%is usually individually typeset as the appearance often has an
%effect on the emotional response when reading it. The \Lpack{verse}
%package~\cite{VERSE} may be useful when typesetting poetry.
%
%\chapter{Introduction to the third edition}
%
%    Since the second  edition of the manual was published the \Lclass{memoir}
%class has been upgraded from beta code to a production release. Extensions
%have been made to both the class and this manual.
%
%    The main extensions and changes include:
%\begin{itemize}
%\item An \Lopt{openleft} option to enable chapters to start
%      on verso pages.
%
%\item Incorporation of the essential code from the \Lpack{verse}
%      package~\cite{VERSE} for more flexibility when typesetting
%      poetry.
%
%\item Replacement of the macro called \cmd{\makepshook} by
%      \cmd{\makepsmarks}. Note that this is a non-upward compatible
%      change.
%
%\item The first part of the manual has been reorganised and
%      extended, principally
%      by providing more typesetting examples.
%
%\item As usual, minor glitches have been removed from both the code
%      and the manual; each revision, of course, eliminates the gap between
%      advertisement and reality.
%
%\end{itemize}
%
%
%    There are other more minor changes and extensions\footnote{With thanks
%to, among others, Ignasi Furi\'{o} Caldentey (\url{ignasi@ipc4.uib.es}),
%Daniel Richard G. (\url{skunk@mit.edu}) and
%Vladimir G. Ivanovi\'c (\url{vladimir@acm.org}).} 
%which you may find if you recall the second edition.
%
%\chapter{Introduction to the fourth edition}
%
%    Since the third edition of the manual was published the \Lclass{memoir}
%class has been upgraded from version~1.0 to version~1.1. Modifications 
%have been made to both the class and this manual.
%
%    The main extensions and changes include:
%\begin{itemize}
%\item The \Lopt{subfigure} option is no longer required.
%
%\item Subfloats have been added to the class. Steve Cochran kindly gave
%      permission for me to use some of his \Lpack{subfigure} package
%      code for this.
%
%\item Some packages still use the old, deprecated LaTeX version~2.09
%      font commands. I have reluctantly introduced an option
%      to enable the old, deprecated font commands to be used.
%
%\item The class now works harmoniously with the \Lpack{natbib}
%      package~\cite{NATBIB}.
%
%\item As usual, minor glitches have been removed from both the code
%      and the manual; each revision hopefully eliminates the gap between
%      advertisement and reality.
%
%\end{itemize}
%
%
%    There are other more minor changes and extensions\footnote{With thanks
%to, among others, 
%William Adams (\url{WillAdams@aol.com})
%Ignasi Furi\'{o} Caldentey (\url{ignasi@ipc4.uib.es}),
%Steven Douglas Cochran (\url{sdc+@cs.cmu.edu}),
%Henrik Holm (\url{henrik@tele.ntnu.no}),
%and Rolf Niepraschk (\url{niepraschk@ptb.de}).
%}
%which you may find if you have used earlier editions.
%
%\chapter{Introduction to the fifth edition}
%
%    Since the fourth edition of the manual was published the \Lclass{memoir}
%class has been upgraded from version~1.1 to version~1.2. Modifications 
%have been made to both the class and this manual.
%
%    The main extensions and changes include:
%\begin{itemize}
%\item The size options have been extended\footnote{At the
%      request of Vittorio De Martino whose children use LaTeX
%      for their school projects.}
%      to include a \Lopt{17pt} option.
%
%\item A few font sizes corresponding to the font size commands (e.g., \verb?\Large?)
%      have been changed to give regular steps between sizes.
%
%\item Boxes that can break over pages and/or contain verbatim text.
%
%\item Several ways of dealing with verbatim text, including footnotes
%      that can contain verbatim text.
%
%\item Some control over the typesetting of footnotes. Unfortunately
%      this necessitated some changes in the methods for styling
%      thanks notes.
%
%\item Comment environments.
%
%\item Convenient methods for file input and output.
%
%\item Additional \verb?\provide...? commands.
%
%\item The \Ie{description} environment has been modified to match
%      the appearance of the standard environment. The original
%      \Lclass{memoir} form is still available as the \Ie{blockdescription}
%      environment.
%
%\item A new optional argument has been added to the \cmd{\chapter} and 
%      \cmd{\chapter*} commands for setting header texts.
%
%\end{itemize}
%
%     As usual, minor glitches have been removed from both the code
%and the manual. Hopefully, new ones have not been introduced.
%
%\chapter{Introduction to the sixth edition}
%
%    Since the fifth edition of the manual was published the \Lclass{memoir}
%class has been upgraded from version~1.2 to version~1.6. 
%Many new functions have
%been added to the class and this manual has been updated to reflect all
%the additions.
%
%    The main extensions and changes include:
%\begin{itemize}
%
%\item Major extensions for typesetting arrays and tabulars, including
%      continuous tabulars and automatic tabulation.
%
%\item Major extensions to footnote styles and the ability to have
%      multiple series of footnotes.
%
%\item Major extensions for indexing, including one column and multiple indexes.
%
%\item Major extensions to crop marks. 
%
%\item \verb?\tableofcontents? and friends can be used multiple times.
%
%\item Section titles (as well as numbers) can be referenced.
%
%\item Sheet numbers and access to the numbers of the last sheet and last page.
%
%\item Various methods for formatting numbers.
%
%\item Better cooperation with the \Lpack{chapterbib} and \Lpack{natbib} packages.
%
%\end{itemize}
%
%     There are many more minor additions.
%     As usual, glitches have been removed from both the code
%and the manual. Hopefully, new ones have not been introduced.
%
%
%     Many people have contributed to the \Lclass{memoir} class and this manual
%in the forms of code, solutions to problems, suggestions for new functions, 
%bringing my attention to errors and infelicities in the code 
%and manual, and last but not least in simply being encouraging. 
%I am very grateful to the following for all they have done, whether they
%knew it or not:
%Paul Abrahams,
%William Adams,
%Donald Arseneau,
%Jens Berger,
%Karl Berry,
%Javier Bezos,
%Sven Bovin,
%Alan Budden,
%Ignasi Furi\'{o} Caldentey,
%Ezequiel Mart\'{\i}n C\'{a}mara,
%David Carlisle,
%Steven Douglas Cochran,
%Michael Downes,
%Victor Eijkhout,
%Danie Els,
%Robin Fairbairns,
%Simon Fear,
%Kai von Fintel,
%Daniel Richard G,
%Romano Giannetti,
%Kathryn Hargreaves,
%Sven Hartrumpf,
%Florence Henry,
%Cartsten Heinz,
%Peter Heslin,
%Morton H\o{}gholm,
%Henrik Holm,
%Vladimir Ivanovich,
%Stefan Kahrs,
%J\o{}gen Larsen,
%Kevin Lin,
%Matthew Lovell,
%Daniel Luecking,
%Lars Madsen,
%Vittorio De Martino,
%Frank Mittelbach,
%Rolf Niepraschk,
%Patrik Nyman,
%Heiko Oberdiek, 
%Scott Pakin,
%Adriano Pascoletti,
%Bernd Raichle,
%Robert,
%Chris Rowley,
%Doug Schenck,
%Rainer Sch\"{o}pf,
%Paul Stanley,
%Reuben Thomas,
%Bastiaan Niels Veelo,
%Emanuele Vicentini,
%J\"{u}rgen Vollmer,
%and others.
%
%If I have inadvertently left anyone off the list I apologise, 
%and please let me know so that I can correct the omisssion.
%
%    Of course, none of this would have been possible without Donald Knuth's
%TeX system and the subsequent development of LaTeX by Leslie Lamport.
%
%\end{comment}
%%%%%%%%%%%%%%%%%%%%%%%%%%%%%%%%%%%%%%%%%%%%%%%%%%%%%%%%%%%
%\begin{comment}
%\chapter{Introduction to the seventh edition}
%
%    The \Lclass{memoir} class and this manual have seen many changes since
%they first saw the light of day. The major functions, and extensions to
%them, were listed in the various
%introductions to the previous editions of this manual and it would now be 
%tedious to read them.
%
%The \Mname\ class was first released in 2001 and since then has proven
%to be reasonably popular. The class can be used as a replacement for
%the \Lclass{book} and \Lclass{report} classes, by default generating
%documents virtually indistinguisable from ones produced by those
%classes.  The class includes some options to produce documents with
%other appearances; for example an \Lclass{article} class look or one
%that looks as though the document was produced on a typewriter with a
%single font, double spacing, no hyphenation, and so on. In the
%following I use the term `standard class'\index{standard class} to
%denote the \Lclass{book} and \Lclass{report} classes and, when
%appropriate, the \Lclass{article} class as well.
%
%    The \Mname\ class includes the functionality of many packages, for
%instance the \Lpack{tocloft} package for controlling the table of contents or
%methods similar to the \Lpack{fancyhdr} package for designing your own
%headers. The built-in package functions are mainly related to document
%design and layout; \Mname\ does not touch upon areas like those that are 
%covered by the \Lpack{babel} or \Lpack{hyperref} packages or any related to 
%typesetting mathematics. On the other hand it is easy to configure a work
%produced with \Mname\ 
%to meet a university's thesis layout requirements.
%
%    \Mname\ has improved substantially since it was first released ---
%over 50 \ltx ers have provided code or suggestions for improvements.
%The class is included in the \TeXUG\ \tx\ distributions and the latest 
%version of the class and its supporting documentation is always
%available from %
%\ctan\ at \url{latex/contrib/memoir}.
%
%    This is not a guide to the general use of \ltx\ but rather concentrates
%on where the \index{class}\Lclass{memoir} class differs from the standard \ltx\
%\Lclass{book} and \Lclass{report} classes. There are other sources that deal 
%with \ltx\ in general, some of which are noted later.
%I assume that you have already used \ltx\ and therefore know how to prepare
%a \ltx\ manuscript, how to run \ltx\ and print the resulting document,
%and that you can also use auxiliary programs like \Lmakeindex\ 
%and \Lbibtex.
%
%
%\section{General considerations}
%
%    The class is a large one consisting of about 10,000 lines of \ltx\ code
%documented in a 400 page report; there is no need for most users to look at 
%this~\cite{MEMCODE}. However if you want to see exactly how some part, 
%or all of, \Mname\ is defined it is there for you to peruse.
%The document you are now reading is the separate comprehensive 
%User Manual~\cite{MEMMAN} which runs to about 500 pages, and from time to 
%time an Addendum~\cite{MEMADD} is released noting extensions to the class.
%Again, if you want to see how something was done in this Manual, which
%of course was prepared using \Mname\ itself, the source
%is available for you to read.
%There is also the \Lpack{memexsupp} package by Lars Madsen~\cite{MEMEXSUPP} 
%which provides some extra facilities for the class.
%
%
%The first part of this Manual discusses some aspects of book design 
%and typography in
%general, something that I haven't come across in the usual \ltx\ books
%and manuals. This is intended to provide a little background for when
%you design your own printed documents.
%
%    The second, and by far the longer part, describes the capabilities
%of \Mname\ and how to use them. This manual is not a \ltx\ tutorial; I assume
%that you already know the basics. If you don't then there are several 
%free tutorials available. In some instances I show you the internal code
%for the class which may involve \ltx\ commands that you won't come
%across in the tutorials and also sometimes basic \tx\ commands. Information
%on these, if you want it, is obtained from reading the \ltx\ source itself
%and the \txbook, and perhaps one of the free \tx\ manuals such as
%\btitle{TeX for the Impatient}~\cite{IMPATIENT} or 
%\btitle{TeX by Topic}~\cite{TEXBYTOPIC}.
%
%\section{Class options}
%
%    The standard classes provide point options of 10, 11, or 12 points for the
%main body font. \Mname\ extends this by also providing a 9 point option, and 
%options ranging from 14 to 60 points.
%The width of the text block is automatically adjusted according to 
%the selected point size to try and keep within generally accepted 
%typographical limits for line lengths; you can override this if you wish. 
%The class also provides easy methods for specifying the 
%page layout parameters such as the margins --- both the side margins and 
%those at 
%the top and bottom of the page; the methods are similar to those of the
%\Lpack{geometry} package.
%
%    The page layout facilities also include methods, like those provided
%by the \Lpack{fancyhdr} package, for defining your own
%header and footer styles, and you can have as many different ones as you wish.
%In fact the class provides seven styles to choose from before having to
%create your own if none of the built-in styles suit you. 
%
%   Sometimes it is useful, or even required, to place trimming marks on
%each page showing the desired size of the final page with respect to the sheet
%of paper that is used in the printer. This is provided by the \Lopt{showtrims}
%option. A variety of trim marks are provided and you can define your own 
%if you need some other kind.
%
%\section{Sectioning styles}
%
%    Handles are provided for designing and using your own styles for chapter
%titles and such. The class comes with over 20 predefined chapter styles ranging
%from the default look to a style that mimics that used in the 
%\emph{Companion} series of \ltx\ books. There are even a couple which use
%words instead of numerals for chapter numbers.
%% The Manual shows 
%%examples of these styles and about 30 are shown in Lars 
%%Madsen's collection~\cite{MEMCHAPS}.
%
%   For those who like putting quotations near chapter titles the 
%\Ie{epigraph} environment can be used.
%
%    The options for changing \cs{section} and lower level titles
%are more constrained, but generally speaking document design, unless for
%advertisements or other eye-catching ephemera, should be constrained.
%The class does provide 9 integrated sets of sectional heading styles instead
%of the usual single set.
%
%    Sometimes, but particularly in novels, a sectional division is indicated
%by just leaving a blank line or two between a pair of paragraphs, or there 
%might be some decorative item like three or four asterisks, or a fleuron
%or two. (A \emph{fleuron}\index{fleuron} is a printers ornament looking 
%like a leaf, such as \ding{166} or \ding{167}.) Commands
%are available for typesetting such anonymous divisions.
%
%   In the standard classes the sectioning commands have an optional argument
%which can be used to put a short version of the section title into the 
%table of contents and the page header. \Mname\ extends this with a second 
%optional argument so you can specify one short version for the contents and 
%an even shorter one for page headers where space is at a premium.
%
%\section{Captions}
%
%    \Mname\ incorporates the code from my \Lpack{ccaption} package which
%lets you easily modify the appearance of figure and table captions; bilingual
%captions are available if required, as are captions placed at the side of
%a figure or table or continuation captions from, say, one illustration to
%another. Captioning can also be applied
%to `non-floating' illustrations or as legends (i.e., unnumbered captions) to
%the regular floats. The captioning system
%also supports subfigures and subtables along the lines of the \Lpack{subfig}
%package, plus letting you define your own new kinds of floats together
%with the corresponding `\listofx'. 
%
%\section{Tables}
%
%    Code from the \Lpack{array}, \Lpack{dcolumn}, \Lpack{delarray} and
%\Lpack{tabularx} packges is integrated within the class. To improve
%the appearance of rules in tabular material the \Lpack{booktabs}
%package is also included.
%
%    Multipage tabulations are often set with the \Lpack{longtable} or
%\Lpack{xtab} packages, which can of course be used with the class. For
%simple tabulations that may continue from one page to the next, \Mname\
%offers a `continuous tabular' environment. This doesn't have all the 
%flexibility provided by the packages but can often serve instead
%of using them.
%
%    More interestingly, but more limited, the class provides `automatic 
%tabulars'. For these you provide a list of simple entries, like a set of names,
% and a number of
%columns and the entries are automatically put into the appropriate column.
%You choose whether the entries should be added row-by-row, like this with
%the \cs{autorows} command:
%
%\begin{lcode}
%\autorows{c}{5}{l}{one, two, three, four,
%    five, six, seven, eight, nine, ten,
%    eleven, twelve, thirteen }
%\end{lcode}
%
%\showit{
%{\centering
%\begin{tabular}{lllll}
%one & two & three & four & five \\
%six & seven & eight & nine & ten \\
%eleven & twelve & thirteen \\
%\end{tabular} 
%\par}
%}
%
% Or if you use the \cs{autocols} command the entries are listed 
%column-by-column, like this :
%
%\begin{lcode}
%\autocols{c}{5}{l}{one, two, three, four,
%    five, six, seven, eight, nine, ten,
%    eleven, twelve, thirteen }
%\end{lcode}
%
%\showit{
%{\centering
%\begin{tabular}{lllll}
%one &    four & seven & ten & thirteen \\
%two &    five & eight & eleven &  \\
%three &  six  & nine  & twelve &  \\
%\end{tabular} 
%\par}
%}
%
%\section{Verse}
%
%    The standard classes provide a very simple \Ie{verse} environment for
%typesetting poetry. This is greatly extended in \Mname. For example in the
%standard classes the verse stanzas are at a fixed indentation from the 
%left margin whereas \Mname\ lets you control the amount of indentation so 
%that you can make a poem appear optically centered within the textwidth.
%
%    Stanzas may be numbered, as can individual lines within a poem. There is
%a special environment for stanzas where lines are alternately indented. Also
%you can define an indentation pattern for stanzas when this is not regular 
%as, for example, in a limerick where the 3rd and 4th of the five lines are 
%indented with respect to the other three as shown below. 
%
%\begin{lcode}
%\indentpattern{00110}
%\begin{verse}
%\begin{patverse}
%There was a young man of Quebec \\
%Who was frozen in snow to his neck. \\
%When asked: `Are you friz?' \\
%He replied: `Yes, I is, \\
%But we don't call this cold in Quebec.'
%\end{patverse}
%\end{verse}
%\end{lcode}
%
%\showit{
%\begin{verse}
%There was a young man of Quebec \\
%Who was frozen in snow to his neck. \\
%\hspace*{2em}When asked: `Are you friz?' \\
%\hspace*{2em}He replied: `Yes, I is, \\
%But we don't call this cold in Quebec.'
%\end{verse}
%}
%
%    It is not always possible to fit
%a line into the available space and you can specify the particular indentation
%to be used when a `logical' verse line spills over the available textwidth, 
%thus forming two or more typeset `physical' lines. On other occasions
%where there are two half lines the poet might want the second half line
%to start where the first one finished, like this:
%
%\begin{lcode}
%\begin{verse}
%Come away with me. \\
%\vinphantom{Come away with me.} Impossible!
%\end{verse}
%\end{lcode}
%
%\showit{
%\begin{verse}
%Come away with me. \\
%\leavevmode\phantom{Come away with me.} Impossible!
%\end{verse}
%}
%
%
%\section{End matter}
%
%    Normally appendices come after the main body of a book. The class provides
%various methods for introducing appendices at the end, or you can place one or 
%more appendices at the end of selected chapters if that suits you better.
%
%    \Mname\ also lets you have more than one index and an index can be set in 
%either the normal double column style or as a single column which would be more
%appropriate, say, for an index of first lines in a book of poetry. The titles
%of any bibliography or indexes are added to the table of contents, but you
%can prevent this if you wish.
%
%    The class provides a set of tools for making glossaries or lists of 
%symbols, the appearance of which can, of course, be easily altered. The 
%\Lmakeindex\ program is used to sort the entries. 
%%An example is
%%shown in the current version of the Addendum. A recent addition
%Also, the class provides configurable end notes which can be used as well as, 
%or instead of, footnotes. 
%
%
%\section{Miscellaneous}
%
%%    As already noted, the Manual for \Mname\ runs to some 300 pages and it
%%is impossible to cover everything in a short article. 
%%Suffice it to say that 
%Hooks and macros are provided for most aspects of document layout; 
%for instance,
%footnotes can be as normal, typeset in two or three columns, or all run 
%into a single paragraph. There is a \cs{sidepar} macro which
%is a non-floating \cs{marginpar} as well as the \cs{sidebar} macro for
%typesetting sidebars in the margin, starting at the top of the text block. 
%You can create new verbatim-like environments, read 
%and write information in external files, design your own style of 
%\cs{maketitle}, convert numbers to words, reserve space at the bottom of a 
%page, and so on and so forth.
%
%
%%% \appendix
%\section{Packages}
%
%    Most packages work with the \Mname\ class, the main exception being
%the \Lpack{hyperref} package. This package modifies
%many of the internals of the standard classes but does not cater for all of
%the differences between \Mname\ and the standard ones. If you wish to use
%\Lpack{hyperref} with \Mname\ then you must use the \Lpack{memhfixc}
%package\footnote{\Lpack{memhfixc} is supplied as part of the \Mname\
%distribution.} after using \Lpack{hyperref}. For example like:
%\begin{lcode}
%\documentclass[...]{memoir}
%...
%\usepackage[...]{hyperref}
%\usepackage{memhfixc}
%...
%\begin{document}
%\end{lcode}
%However, if you have a version of \Lpack{hyperref} dated 2006/11/15 or after, 
%\Lpack{hyperref}
%will automatically call in \Lpack{memhfixc} so that you don't have to do 
%anything.
%
%The \Mname\ class includes code either equivalent to, or extensions of, the 
%following packages; that is, the set of commands and environments is at least
%the same as those in the packages: 
%%\begin{itemize}%\item 
%\begin{lineitems}
%      \Lpack{abstract}
%\item \Lpack{appendix}
%\item \Lpack{array}
%\item \Lpack{booktabs}
%\item \Lpack{ccaption}
%\item \Lpack{chngcntr}
%\item \Lpack{chngpage}
%\item \Lpack{dcolumn}
%\item \Lpack{delarray}
%\item \Lpack{enumerate}
%\item \Lpack{epigraph}
%\item \Lpack{framed}
%\item \Lpack{ifmtarg}
%\item \Lpack{ifpdf}
%\item \Lpack{index}
%\item \Lpack{makeidx}
%\item \Lpack{moreverb}
%\item \Lpack{needspace}
%\item \Lpack{newfile}
%\item \Lpack{nextpage}
%\item \Lpack{parskip}
%\item \Lpack{patchcmd}
%\item \Lpack{setspace}
%\item \Lpack{shortvrb}
%\item \Lpack{showidx}
%\item \Lpack{tabularx}
%\item \Lpack{titleref}
%\item \Lpack{titling}
%\item \Lpack{tocbibind}
%\item \Lpack{tocloft}
%\item \Lpack{verbatim}
%\item \Lpack{verse}.
%\end{lineitems}
%%\end{itemize}
%The class automatically ignores any 
%\verb?\usepackage? or \verb?\RequirePackage? related to these. However, if
%you want to specifically use one of these packages rather than the integrated
%version then you can do so. For arguments sake, assuming you really want 
%to use the \Lpack{titling} the package you can do this:
%\begin{lcode}
%\documentclass[...]{memoir}
%\DisemulatePackage{titling}
%\usepackage{titling}
%\end{lcode}
%
%    The \Mname\ class incorporates a version of the \Lpack{setspace} package, 
%albeit using different names for the macros. The package enables documents
%to be set double spaced but leaves some document elements, 
%like captions for example, single spaced. To do this it has to make some 
%assumptions about how the document class works. I felt that this kind
%of capability should be part of the class and not depend on assumptions.
%In the particular case of the \Lpack{setspace} package, even with the
%standard classes, there can be some unexpected spacing around displayed
%material; this has not occured with \Mname's implementation. 
%
%The class also provides functions similar to those provided by the following 
%packages, although the commands are different: 
%%\begin{itemize}%\item 
%\begin{lineitems}%\item 
%\Lpack{crop}
%\item \Lpack{fancyhdr}
%\item \Lpack{geometry}
%\item \Lpack{sidecap}
%\item \Lpack{subfigure}
%\item \Lpack{titlesec}.
%\end{lineitems}
%%\end{itemize}
%You can use these packages 
%if you wish, or just use the capabilities of the \Mname\ class.
%
%
%\section{Resources} \label{sec:resources}
%
%
%    Scattered throughout
%%, but mainly in Part~\ref{part:art},
%are comments about aspects of book design and typography, in some cases
%accompanied by examples of better and poorer practice. If you want more
%authorative remarks there are several books on the subject listed in 
%the \bibname;  I prefer
%Bringhurst's \textit{The Elements of Typographic Style}~\cite{BRINGHURST99}.
%
%
%    \ltx\ is based on the \tx\ program which was designed principally 
%for typesetting documents containing a 
%lot of mathematics. In such works the mathematics breaks up the flow
%of the text on the page, and the vertical space required for displayed
%mathematics is highly dependent on the mathematical particularities. 
%Most non-technical books are typeset on a fixed
%grid as they do not have arbitrary insertions into the text; it is these
%kinds of publications that typographers are most comfortable talking about.
%
%    There are other sources that deal with \ltx\ in 
%general, some of which are listed in the \bibname. Lamport~\cite{LAMPORT94}
%is of course the original user manual for \ltx, while the Companion
%series~\cite{COMPANION,GCOMPANION,WCOMPANION} go into further 
%details and auxiliary
%programs. George Gr\"{a}tzer's \textit{Math into \ltx} is valuable if you
%typeset a lot of mathematics with excellent coverage of the American
%Mathematical Society's packages.
%
%   The \cTeXan{} (\pixctan) is an invaluable source
%of free information and of the \ltx\ system itself. For general questions see
%the FAQ (Frequently Asked Questions, and answers) maintained by
%Robin Fairbairns~\cite{FAQ}, which also has pointers to many information 
%sources. Among
%these are \textit{The Not So Short Introduction to \ltxe}~\cite{LSHORT},
%Keith Reckdahl's \textit{Using imported graphics in \ltxe}~\cite{EPSLATEX}
%and Piet van Oostrum's \textit{Page layout in \ltx}~\cite{FANCYHDR}.
%Peter Flynn's \textit{Formatting information}~\cite{FLYNN02} is unique 
%in that it describes how to install a \ltx\ system and editors for 
%writing your documents as well as how to use \ltx. There are a myriad
%of packages and software tools freely available to enhance any \ltx\ system;
%the great majority of these are listed in Graham Williams' magnificent 
%on line searchable catalogue~\cite{CATALOGUE}, which also links directly
%to \pixctan. This is just one of the services offered by the \TeXUG{} (\pixtug)
%and information on how to access it is available 
%at \url{http://www.tug.org}
%which is the homepage for the \TeXUG.
%
%    The most recent crops of messages on the \url{comp.text.tex}
%newsgroup (\pixctt) show an increasing interest in using a wider range
%of fonts with \ltx. This is a question that I have left alone.
%Alan Hoenig's book~\cite{HOENIG98} is the best guide to this that I know of.
%\pixctan\ hosts Philipp Lehman's font installation guide~\cite{FONTINST}; 
%this is well worth looking at just as an example of fine typesetting.
%
%    The source code for the \Lclass{memoir} class is, of course,
%freely available from \pixctan\ if you wish to see exactly what it does
%and how it does it.
%
%    For a more interactive resource you can ask questions on the
%\url{comp.text.tex} newsgroup. If you are a newcomer to \pixctt\
%please read the FAQ~\cite{FAQ} before asking a question, and also read
%a few day's worth of messages to check that your question hasn't just
%been answered.
%
%\end{comment}
%
%%#% extend
%
%%#% extstart include intro-8.tex
%
%\svnidlong
%{$Ignore: $}
%{$LastChangedDate: 2015-04-22 17:17:51 +0200 (Wed, 22 Apr 2015) $}
%{$LastChangedRevision: 527 $}
%{$LastChangedBy: daleif $}
%
%\chapter{Introduction to the eighth edition}
%
%The \Mname\ class and this manual have seen many changes since they
%first saw the light of day. The major functions, and extensions to
%them, were listed in the various introductions to the previous
%editions of this manual and it would now be tedious to read them.
%
%The \Mname\ class was first released in 2001 and since then has proven
%to be reasonably popular. The class can be used as a replacement for
%the \Lclass{book} and \Lclass{report} classes, by default generating
%documents virtually indistinguisable from ones produced by those
%classes.  The class includes some options to produce documents with
%other appearances; for example an \Lclass{article} class look or one
%that looks as though the document was produced on a typewriter with a
%single font, double spacing, no hyphenation, and so on. In the
%following I use the term `standard class'\index{standard class} to
%denote the \Lclass{book} and \Lclass{report} classes and, when
%appropriate, the \Lclass{article} class as well.
%
%    The \Mname\ class includes the functionality of many packages, for
%instance the \Lpack{tocloft} package for controlling the table of contents or
%methods similar to the \Lpack{fancyhdr} package for designing your own
%headers. The built-in package functions are mainly related to document
%design and layout; \Mname\ does not touch upon areas like those that are 
%covered by the \Lpack{babel} or \Lpack{hyperref} packages or any related to 
%typesetting mathematics. On the other hand it is easy to configure a work
%produced with \Mname\ 
%to meet a university's thesis layout requirements.
%
%    \Mname\ has improved substantially since it was first released ---
%over 50 \ltx ers have provided code or suggestions for improvements.
%The class is included in the \TeXUG\ \tx\ distributions and the latest 
%version of the class and its supporting documentation is always
%available from %
%\ctan\ at \url{latex/contrib/memoir}.
%
%    This is not a guide to the general use of \ltx\ but rather concentrates
%on where the \index{class}\Lclass{memoir} class differs from the standard \ltx\
%\Lclass{book} and \Lclass{report} classes. There are other sources that deal 
%with \ltx\ in general, some of which are noted later.
%I assume that you have already used \ltx\ and therefore know how to prepare
%a \ltx\ manuscript, how to run \ltx\ and print the resulting document,
%and that you can also use auxiliary programs like \Lmakeindex\ 
%and \Lbibtex.
%
%
%\section{General considerations}
%
%    The class is a large one consisting of about 10,000 lines of \ltx\ code
%documented in a 400 page report; there is no need for most users to look at 
%this~\cite{MEMCODE}. However if you want to see exactly how some part, 
%or all of, \Mname\ is defined it is there for you to peruse.
%The document you are now reading is the separate comprehensive 
%User Manual~\cite{MEMMAN} which runs to about 500 pages, and from time to 
%time an Addendum %\cite{MEMADD} 
%is released noting extensions to the class.\footnote{Currently not in use.}
%Again, if you want to see how something was done in this Manual, which
%of course was prepared using \Mname\ itself, the source
%is available for you to read.
%% There is also the \Lpack{memexsupp} package by Lars Madsen~\cite{MEMEXSUPP} 
%% which provides some extra facilities for the class.
%
%    The previous editions of the Manual consisted of two parts. The first
%discussing some aspects of book design and typography in
%general, something that I hadn't come across in the usual \ltx\ books
%and manuals. That was intended to provide a little background for when
%you design your own printed documents. The second, and by far the longest
%part, described the capabilities of \Mname\ and how to use them. 
%
%    The Manual kept growing until it was approaching 700 pages and I decided
%that it was better to put the original discussion on typography into a
%separate document~\cite{MEMDESIGN}, which is independent of \Mname, and in this
%edition concentrate on how to use \Mname. This has reduced the size of
%this document, but it is still large.
%
%This manual is not a \ltx\ tutorial; I assume
%that you already know the basics. If you don't then there are several 
%free tutorials available. In some instances I show you the internal code
%for the class which may involve \ltx\ commands that you won't come
%across in the tutorials and also sometimes basic \tx\ commands. Information
%on these, if you want it, is obtained from reading the \ltx\ source itself
%and the \txbook, and perhaps one of the free \tx\ manuals such as
%\btitle{TeX for the Impatient}~\cite{IMPATIENT} or 
%\btitle{TeX by Topic}~\cite{TEXBYTOPIC}.
%
%\section{Class options}
%
%    The standard classes provide point options of 10, 11, or 12 points for the
%main body font. \Mname\ extends this by also providing a 9 point option, and 
%options ranging from 14 to 60 points.
%The width of the text block is automatically adjusted according to 
%the selected point size to try and keep within generally accepted 
%typographical limits for line lengths; you can override this if you wish. 
%The class also provides easy methods for specifying the 
%page layout parameters such as the margins --- both the side margins and 
%those at 
%the top and bottom of the page; the methods are similar to those of the
%\Lpack{geometry} package.
%
%    The page layout facilities also include methods, like those provided
%by the \Lpack{fancyhdr} package, for defining your own
%header and footer styles, and you can have as many different ones as you wish.
%In fact the class provides seven styles to choose from before having to
%create your own if none of the built-in styles suit you. 
%
%   Sometimes it is useful, or even required, to place trimming marks on
%each page showing the desired size of the final page with respect to the sheet
%of paper that is used in the printer. This is provided by the \Lopt{showtrims}
%option. A variety of trim marks are provided and you can define your own 
%if you need some other kind.
%
%\section{Sectioning styles}
%
%    Handles are provided for designing and using your own styles for chapter
%titles and such. The class comes with over 20 predefined chapter styles ranging
%from the default look to a style that mimics that used in the 
%\emph{Companion} series of \ltx\ books. There are even a couple which use
%words instead of numerals for chapter numbers.
%% The Manual shows 
%%examples of these styles and about 30 are shown in Lars 
%%Madsen's collection~\cite{MEMCHAPS}.
%
%   For those who like putting quotations near chapter titles the 
%\Ie{epigraph} environment can be used.
%
%    The options for changing \cs{section} and lower level titles
%are more constrained, but generally speaking document design, unless for
%advertisements or other eye-catching ephemera, should be constrained.
%The class does provide 9 integrated sets of sectional heading styles instead
%of the usual single set.
%
%    Sometimes, but particularly in novels, a sectional division is indicated
%by just leaving a blank line or two between a pair of paragraphs, or there 
%might be some decorative item like three or four asterisks, or a fleuron
%or two. (A \emph{fleuron}\index{fleuron} is a printers ornament looking 
%like a leaf, such as \ding{166} or \ding{167}.) Commands
%are available for typesetting such anonymous divisions.
%
%   In the standard classes the sectioning commands have an optional argument
%which can be used to put a short version of the section title into the 
%table of contents and the page header. \Mname\ extends this with a second 
%optional argument so you can specify one short version for the contents and 
%an even shorter one for page headers where space is at a premium.
%
%\section{Captions}
%
%    \Mname\ incorporates the code from my \Lpack{ccaption} package which
%lets you easily modify the appearance of figure and table captions; bilingual
%captions are available if required, as are captions placed at the side of
%a figure or table or continuation captions from, say, one illustration to
%another. Captioning can also be applied
%to `non-floating' illustrations or as legends (i.e., unnumbered captions) to
%the regular floats. The captioning system
%also supports subfigures and subtables along the lines of the \Lpack{subfig}
%package, plus letting you define your own new kinds of floats together
%with the corresponding `\listofx'. 
%
%\section{Tables}
%
%    Code from the \Lpack{array}, \Lpack{dcolumn}, \Lpack{delarray} and
%\Lpack{tabularx} packges is integrated within the class. To improve
%the appearance of rules in tabular material the \Lpack{booktabs}
%package is also included.
%
%    Multipage tabulations are often set with the \Lpack{longtable} or
%\Lpack{xtab} packages, which can of course be used with the class. For
%simple tabulations that may continue from one page to the next, \Mname\
%offers a `continuous tabular' environment. This doesn't have all the 
%flexibility provided by the packages but can often serve instead
%of using them.
%
%    More interestingly, but more limited, the class provides `automatic 
%tabulars'. For these you provide a list of simple entries, like a set of names,
% and a number of
%columns and the entries are automatically put into the appropriate column.
%You choose whether the entries should be added row-by-row, like this with
%the \cs{autorows} command:
%
%\begin{lcode}
%\autorows{c}{5}{l}{one, two, three, four,
%    five, six, seven, eight, nine, ten,
%    eleven, twelve, thirteen }
%\end{lcode}
%
%\showit{
%{\centering
%\begin{tabular}{lllll}
%one & two & three & four & five \\
%six & seven & eight & nine & ten \\
%eleven & twelve & thirteen \\
%\end{tabular} 
%\par}
%}
%
% Or if you use the \cs{autocols} command the entries are listed 
%column-by-column, like this :
%
%\begin{lcode}
%\autocols{c}{5}{l}{one, two, three, four,
%    five, six, seven, eight, nine, ten,
%    eleven, twelve, thirteen }
%\end{lcode}
%
%\showit{
%{\centering
%\begin{tabular}{lllll}
%one &    four & seven & ten & thirteen \\
%two &    five & eight & eleven &  \\
%three &  six  & nine  & twelve &  \\
%\end{tabular} 
%\par}
%}
%
%\section{Verse}
%
%    The standard classes provide a very simple \Ie{verse} environment for
%typesetting poetry. This is greatly extended in \Mname. For example in the
%standard classes the verse stanzas are at a fixed indentation from the 
%left margin whereas \Mname\ lets you control the amount of indentation so 
%that you can make a poem appear optically centered within the textwidth.
%
%    Stanzas may be numbered, as can individual lines within a poem. There is
%a special environment for stanzas where lines are alternately indented. Also
%you can define an indentation pattern for stanzas when this is not regular 
%as, for example, in a limerick where the 3rd and 4th of the five lines are 
%indented with respect to the other three as shown below. 
%
%\begin{lcode}
%\indentpattern{00110}
%\begin{verse}
%\begin{patverse}
%There was a young man of Quebec \\
%Who was frozen in snow to his neck. \\
%When asked: `Are you friz?' \\
%He replied: `Yes, I is, \\
%But we don't call this cold in Quebec.'
%\end{patverse}
%\end{verse}
%\end{lcode}
%
%\showit{
%\begin{verse}
%There was a young man of Quebec \\
%Who was frozen in snow to his neck. \\
%\hspace*{2em}When asked: `Are you friz?' \\
%\hspace*{2em}He replied: `Yes, I is, \\
%But we don't call this cold in Quebec.'
%\end{verse}
%}
%
%    It is not always possible to fit
%a line into the available space and you can specify the particular indentation
%to be used when a `logical' verse line spills over the available textwidth, 
%thus forming two or more typeset `physical' lines. On other occasions
%where there are two half lines the poet might want the second half line
%to start where the first one finished, like this:
%
%\begin{lcode}
%\begin{verse}
%Come away with me. \\
%\vinphantom{Come away with me.} Impossible!
%\end{verse}
%\end{lcode}
%
%\showit{
%\begin{verse}
%Come away with me. \\
%\leavevmode\phantom{Come away with me.} Impossible!
%\end{verse}
%}
%
%
%\section{End matter}
%
%    Normally appendices come after the main body of a book. The class provides
%various methods for introducing appendices at the end, or you can place one or 
%more appendices at the end of selected chapters if that suits you better.
%
%    \Mname\ also lets you have more than one index and an index can be set in 
%either the normal double column style or as a single column which would be more
%appropriate, say, for an index of first lines in a book of poetry. The titles
%of any bibliography or indexes are added to the table of contents, but you
%can prevent this if you wish.
%
%    The class provides a set of tools for making glossaries or lists of 
%symbols, the appearance of which can, of course, be easily altered. The 
%\Lmakeindex\ program is used to sort the entries. 
%%An example is
%%shown in the current version of the Addendum. A recent addition
%Also, the class provides configurable end notes which can be used as well as, 
%or instead of, footnotes. 
%
%
%\section{Miscellaneous}
%
%%    As already noted, the Manual for \Mname\ runs to some 300 pages and it
%%is impossible to cover everything in a short article. 
%%Suffice it to say that 
%Hooks and macros are provided for most aspects of document layout; 
%for instance,
%footnotes can be as normal, typeset in two or three columns, or all run 
%into a single paragraph. There is a \cs{sidepar} macro which
%is a non-floating \cs{marginpar} as well as the \cs{sidebar} macro for
%typesetting sidebars in the margin, starting at the top of the text block. 
%You can create new verbatim-like environments, read 
%and write information in external files, design your own style of 
%\cs{maketitle}, convert numbers to words, reserve space at the bottom of a 
%page, and so on and so forth.
%
%
%%% \appendix
%\section{Packages}
%
%    Most packages work with the \Mname\ class, the main exception being
%the \Lpack{hyperref} package. This package modifies
%many of the internals of the standard classes but does not cater for all of
%the differences between \Mname\ and the standard ones. If you wish to use
%\Lpack{hyperref} with \Mname\ then you must use the \Lpack{memhfixc}
%package\footnote{\Lpack{memhfixc} is supplied as part of the \Mname\
%distribution.} after using \Lpack{hyperref}. For example like:
%\begin{lcode}
%\documentclass[...]{memoir}
%...
%\usepackage[...]{hyperref}
%\usepackage{memhfixc}
%...
%\begin{document}
%\end{lcode}
%However, if you have a version of \Lpack{hyperref} dated 2006/11/15 or after, 
%\Lpack{hyperref}
%will automatically call in \Lpack{memhfixc} so that you don't have to do 
%anything.
%
%The \Mname\ class includes code either equivalent to, or extensions of, the 
%following packages; that is, the set of commands and environments is at least
%the same as those in the packages: 
%%\begin{itemize}%\item 
%\begin{lineitems}
%      \Lpack{abstract}
%\item \Lpack{appendix}
%\item \Lpack{array}
%\item \Lpack{booktabs}
%\item \Lpack{ccaption}
%\item \Lpack{chngcntr}
%\item \Lpack{chngpage}
%\item \Lpack{dcolumn}
%\item \Lpack{delarray}
%\item \Lpack{enumerate}
%\item \Lpack{epigraph}
%\item \Lpack{framed}
%\item \Lpack{ifmtarg}
%\item \Lpack{ifpdf}
%\item \Lpack{index}
%\item \Lpack{makeidx}
%\item \Lpack{moreverb}
%\item \Lpack{needspace}
%\item \Lpack{newfile}
%\item \Lpack{nextpage}
%\item \Lpack{parskip}
%\item \Lpack{patchcmd}
%\item \Lpack{setspace}
%\item \Lpack{shortvrb}
%\item \Lpack{showidx}
%\item \Lpack{tabularx}
%\item \Lpack{titleref}
%\item \Lpack{titling}
%\item \Lpack{tocbibind}
%\item \Lpack{tocloft}
%\item \Lpack{verbatim}
%\item \Lpack{verse}.
%\end{lineitems}
%%\end{itemize}
%The class automatically ignores any 
%\verb?\usepackage? or \verb?\RequirePackage? related to these. However, if
%you want to specifically use one of these packages rather than the integrated
%version then you can do so. For arguments sake, assuming you really want 
%to use the \Lpack{titling} package you can do this:
%\begin{lcode}
%\documentclass[...]{memoir}
%\DisemulatePackage{titling}
%\usepackage{titling}
%\end{lcode}
%
%    The \Mname\ class incorporates a version of the \Lpack{setspace} package, 
%albeit using different names for the macros. The package enables documents
%to be set double spaced but leaves some document elements, 
%like captions for example, single spaced. To do this it has to make some 
%assumptions about how the document class works. I felt that this kind
%of capability should be part of the class and not depend on assumptions.
%In the particular case of the \Lpack{setspace} package, even with the
%standard classes, there can be some unexpected spacing around displayed
%material; this has not occured with \Mname's implementation. 
%
%The class also provides functions similar to those provided by the following 
%packages, although the commands are different: 
%%\begin{itemize}%\item 
%\begin{lineitems}%\item 
%\Lpack{crop}
%\item \Lpack{fancyhdr}
%\item \Lpack{geometry}
%\item \Lpack{sidecap}
%\item \Lpack{subfigure}
%\item \Lpack{titlesec}.
%\end{lineitems}
%%\end{itemize}
%You can use these packages 
%if you wish, or just use the capabilities of the \Mname\ class.
%
%    The class has built in support for the \Lpack{bidi} package for 
%bidirectional typesetting~\cite{BIDI}.
%
%
%\section{Resources} \label{sec:resources}
%
%
%    Scattered throughout
%%, but mainly in Part~\ref{part:art},
%are comments about aspects of book design and typography, in some cases
%accompanied by examples of better and poorer practice. If you want more
%comments and examples there are some notes on the topic~\cite{MEMDESIGN},
%and for 
%authorative remarks there are several books on the subject listed in 
%the \bibname;  I prefer
%Bringhurst's \textit{The Elements of Typographic Style}~\cite{BRINGHURST99},
%while Derek Birdsall's \textit{notes on book design}~\cite{BIRDSALL04}
%is much more oriented to illustrated works, like museum
%catalogues and art books.
%
%
%    \ltx\ is based on the \tx\ program which was designed principally 
%for typesetting documents containing a 
%lot of mathematics. In such works the mathematics breaks up the flow
%of the text on the page, and the vertical space required for displayed
%mathematics is highly dependent on the mathematical particularities. 
%Most non-technical books are typeset on a fixed
%grid as they do not have arbitrary insertions into the text; it is these
%kinds of publications that typographers are most comfortable talking about.
%
%    There are other sources that deal with \ltx\ in 
%general, some of which are listed in the \bibname. Lamport~\cite{LAMPORT94}
%is of course the original user manual for \ltx, while the Companion
%series~\cite{COMPANION,GCOMPANION,WCOMPANION} go into further 
%details and auxiliary
%programs. George Gr\"{a}tzer's \textit{Math into \ltx} is valuable if you
%typeset a lot of mathematics with excellent coverage of the American
%Mathematical Society's packages.
%
%   The \cTeXan{} (\pixctan) is an invaluable source
%of free information and of the \ltx\ system itself. For general questions see
%the FAQ (Frequently Asked Questions, and answers) maintained by
%Robin Fairbairns~\cite{FAQ}, which also has pointers to many information 
%sources. Among
%these are \textit{The Not So Short Introduction to \ltxe}~\cite{LSHORT},
%Keith Reckdahl's \textit{Using imported graphics in \ltxe}~\cite{EPSLATEX}
%and Piet van Oostrum's \textit{Page layout in \ltx}~\cite{FANCYHDR}.
%Peter Flynn's \textit{Formatting information}~\cite{FLYNN02} is unique 
%in that it describes how to install a \ltx\ system and editors for 
%writing your documents as well as how to use \ltx. There are a myriad
%of packages and software tools freely available to enhance any \ltx\ system;
%the great majority of these are listed in Graham Williams' magnificent 
%on line searchable catalogue~\cite{CATALOGUE}, which also links directly
%to \pixctan. This is just one of the services offered by the \TeXUG{} (\pixtug)
%and information on how to access it is available 
%at \url{http://www.tug.org}
%which is the homepage for the \TeXUG.
%
%    The most recent crops of messages on the \url{comp.text.tex}
%newsgroup (\pixctt) show an increasing interest in using a wider range
%of fonts with \ltx. This is a question that I have left alone.
%Alan Hoenig's book~\cite{HOENIG98} is the best guide to this that I know of.
%\pixctan\ hosts Philipp Lehman's font installation guide~\cite{FONTINST}; 
%this is well worth looking at just as an example of fine typesetting.
%
%    The source code for the \Lclass{memoir} class is, of course,
%freely available from \pixctan\ if you wish to see exactly what it does
%and how it does it.
%
%    For a more interactive resource you can ask questions on the
%\url{comp.text.tex} newsgroup. If you are a newcomer to \pixctt\
%please read the FAQ~\cite{FAQ} before asking a question, and also read
%a few day's worth of messages to check that your question hasn't just
%been answered.
%
%\section{Type conventions}
%
%    The following conventions are used:
%\begin{itemize}
%\item \Pclass{The names of \ltx\ classes\index{class} and 
%              packages\index{package} are typeset in this font.}
%\item \Popt{Class options\index{option} are typeset in this font.}
%\item \Ppstyle{The names of chapterstyles\index{chapterstyle} and 
%               pagestyles\index{pagestyle} are typeset in this font.}
%\item \texttt{\ltx\ code is typeset in this font.}
%\item \Pprog{The names of programs are in this font.}
%\end{itemize}
%\begin{syntax}
%Macro command syntax is enclosed in a rectangular box.\\
%For referential purposes, arguments are denoted by \meta{arg} \\
%\end{syntax}
%
%
%\section{Acknowledgements}
%
%     Many people have contributed to the \Lclass{memoir} class and this manual
%in the forms of code, solutions to problems, suggestions for new functions, 
%bringing my attention to errors and infelicities in the code 
%and manual, and last but not least in simply being encouraging. 
%I am very grateful to the following for all they have done, whether they
%knew it or not:
%Paul Abrahams,      % code
%William Adams,      % typography
%Tim Arnold,         % among other things, \leavespergathering in general
%Donald Arseneau,    % code
%Stephan von Bechtolsheim,
%Jens Berger,
%Karl Berry,         % code
%Ingo Beyritz,       % bug report (tabularx in subtable)
%Javier Bezos,
%Stefano Bianchi,    % chaptersytyle
%Sven Bovin,
%Alan Budden,
%Ignasi Furi\'{o} Caldentey,
%Ezequiel Mart\'{\i}n C\'{a}mara,
%David Carlisle,     % code
%Gustafo Cevolani, 
%Jean-C{\^o}me Charpentier,   % memmanadd typo fix
%Michael A. Cleverly,       % code
%Steven Douglas Cochran,    % code
%Frederic Connes,           % code
%\v{Z}arko F. \v{C}u\v{c}ej, % bug report (contcaption & hyperref)
%Christopher Culver,       % chapterstyle
%Iain Dalton,              % typos
%Michael W. Daniels,       % code
%Michael Downes,           % code
%Christopher Dutchyn,
%Thomas Dye,               % code, typos
%Victor Eijkhout,          % code
%Roman Eisele,             % fix of \parnopar (2008/09/13)
%Danie Els,                % code
%Robin Fairbairns,         % code
%Simon Fear,               % code
%Ant\'{o}nio Ferreira,     % many typos (2008/08/29)
%Kai von Fintel,
%Ivars Finvers,            % bug report
%Ulrike Fischer,           % general code ideas
%Matthew Ford,
%Musa Furber,
%Daniel Richard G,
%Ignacio Fern{\'a}ndez Galv{\'a}n,
%Gerardo Garcia,          % chapterstyle
%Romano Giannetti,        % code
%Kheng-Swee Goh,          % manual typo `docotoral' should be `doctoral'
%Donald Goodman,          % manual typo (1/2 title page should be in pagination)
%Gabriel Guernik,         % bug report & suggested fix
%Matthias Haldiman,       % bug report, fixed by Heiko
%Kathryn Hargreaves,      % code
%Sven Hartrumpf,
%hazydirk,                % code
%Carsten Heinz,           % code
%Florence Henry,
%Peter Heslin,
%Timo Hoenig,             % thank you letter
%Morten H{\o}gholm,       % code
%Henrik Holm,
%Vladimir G. Ivanovi\'c,
%Martin J{\o}rgensen,     % bug report
%Stefan Kahrs,
%Christian Keil,          % typos
%Marcus Kohm,             % algorithm
%Flavian Lambert,         % float type bug
%J\o{}gen Larsen,         % bug reports and fix
%Kevin Lin,
%Matthew Lovell,
%Daniel Luecking,         % codef
%Anders Lyhne,            % chapterstyle
%Lars Hendrik Gam Madsen, % extra space in Part title in ToC
%Lars~Madsen,             % code
%Vittorio De Martino,
%Ben McKay,               % errors in pagenote instructions
%Frank~Mittelbach,        % code
%Wilhelm M\"{u}ller,      % bugs & suggested extensions
%Vilar~Camara~Neto,
%Rolf Niepraschk,
%Patrik Nyman,    
%Heiko~Oberdiek,          % code
%Scott~Pakin,
%Adriano~Pascoletti,
%Paul,                    % bug report
%Ted Pavlic,              % typo report
%Troels Pedersen,         % chapterstyle
%Steve Peter,
%Fran\c{c}ois Poulain,    % typo in Magellan's voyage title
%Erik Quaeghebeur,        % bug report
%Bernd Raichle,           % code
%Martin Reinders,         % requested titleref extensions
%Aaron Rendahl,           % bug report and fix
%Ren{\'e},                % correction (paper folding)
%Alan Ristow,             % request for \leavespergathering
%Robert,
%Chris Rowley,
%Gary Ruben,              % bug report
%Robert Schlicht,         % code
%Doug Schenck,
%Dirk Schlimm,
%Arnaud Schmittbuhl,
%Rainer Sch\"{o}pf,       % code
%Paul Stanley,
%Per Starb{\"a}ck,        % boxedverbatim in narrow text bug, documentation
%James Szinger,           % code
%Jens Taprogge,
%Ajit Thakkar,            % reference to an appendix, typo
%Scott Thatcher,          % chapterstyle
%Reuben Thomas,
%Bastiaan Niels Veelo,    % code
%Guy Verville,            % chapterstyle
%Emanuele Vicentini,
%J{\"o}rg Vogt,           % suggestion re verse
%J\"{u}rgen Vollmer,
%M J Williams,            % \input in tabular bug
%and 
%David Wilson.
%
%
%
%
%If I have inadvertently left anyone off the list I apologise, 
%and please let me know so that I can correct the 
%omisssion.% \footnote{Peter is currently occasionably reachable via email
%% at \texttt{herries dot press (at) earthlink dot net}, otherwise write
%% the maintainer at \texttt{daleif at math dot au dot dk}}
%\footnote{Please write the maintainer at \texttt{daleif at math dot au
%    dot dk}} Along those lines, if you have any questions you may
%direct them to the \url{comp.text.tex} newsgroup or post them on
%\url{http://tex.stackexchange.com} as you are likely to get a
%satisfactory and timely response.
%
%    Of course, none of this would have been possible without Donald Knuth's
%\tx\ system and the subsequent development of \ltx\ by Leslie Lamport.
%
%\PWnote{2009/07/26}{Added `Remarks to the user' chapter}
%%%%%%%%%%%%%%%%%%%%%%%%%%%%%%%%%%%
%\chapter{Remarks to the user}
%%%%%%%%%%%%%%%%%%%%%%%%%%%%%%%%%%%
%
%    The \Lclass{memoir} class gives you many ways to change the appearance of your
%document, and also provides some ready-made styles that might be
%appropriate for your purposes.
%
%    As you can see, this manual is not slim and attempts to describe in
%some detail how the various aspects of \Lclass{memoir} work and gives
%examples of how you can change these to better match your needs. However,
%there are a myriad of different things that users might wish to do and
%it is not possible either for the class to provide ready made simple, 
%or even complex, methods to directly support these, or for this manual
%to give examples of how everything might be accomplished.
%
%    If many want a particular facility that is not available, then it 
%may be possible to add that. If it is only one who wishes it then, unless
%the one is the author, it is unlikely to be provided. But don't let this stop
%you from asking, especially if you can provide the necessary code.
%
%    The complete documented code for the class is available in the file
%\file{memoir.dtx}. If you want to know how something is done then you
%can read the code for \emph{all} the details. If you want to do something
%different, then the code is there for you to look at and experiment with.
%You should, though, not change any of the code in the class. If you need
%to do so, then copy the code you wish to change into the document's preamble
%or a package of your own or a class of your own (with a different name)
%and make the changes there. Do not expect any help if you change the
%\Lclass{memoir} class code directly.
%
%\fancybreak{}
%
%As the years go by support for \Lclass{memoir} will devolve from one
%person to another.\footnote{This is currently (July 2009) happening as
%  Lars Madsen is taking over from Peter Wilson.}  Therefore it is
%probably safer to ask questions, complain, make suggestions, etc., on
%a Q\&A site like \url{http://tex.stackexchange.com} or on the the
%newsgroup \url{comp.text.tex}, which is archived and read by many,
%than correspond directly with the maintainer, who might well be away
%for some considerable time and perhaps not notice your email after
%having returned to base.
%
%In either case please include the word \texttt{\theclass} in the
%subject, and if possible, please \emph{tag} the question with the
%\texttt{\theclass} tag.  That will help the memoir maintainer keep
%track of memoir related matters.
%
%\fancybreak{}
%
%\textit{From the maintainer:} It seems that traffic on
%\url{comp.text.tex} is less frequent. So most \theclass\ related
%questions should go to \url{http://tex.stackexchange.com}, please
%remember to tag them properly, that really helps locating the
%\theclass\ related questions. If no-one comes up with an answer, you
%can also write me directly via \texttt{daleif (at) math dot au dot dk}.
%
%
%
%%#% extend
%
%
%%#% extstart include terminology.tex
%
%\svnidlong
%{$Ignore: $}
%{$LastChangedDate: 2013-04-24 17:14:15 +0200 (Wed, 24 Apr 2013) $}
%{$LastChangedRevision: 442 $}
%{$LastChangedBy: daleif $}
%
%%%%%%%%%%%%%%%%%%%%%%%%%%%%%%%%%%
%\chapter{Terminology}
%%%%%%%%%%%%%%%%%%%%%%%%%%%%%%%%%%%
%
%    Like all professions and trades, typographers and printers have their
%specialised vocabulary.
%
%    First there is the question of pages, leaves and sheets. 
%The trimmed sheets of paper\index{paper} that make up a book are called 
%\emph{leaves}\index{leaf},
%and I will call the untrimmed sheets the \emph{stock}\index{stock} material. 
%A leaf
%has two sides, and a \emph{page}\index{page} is one side of a leaf. 
%If you think of a book
%being opened flat, then you can see two leaves. The front of the righthand
%leaf, is called the \emph{recto}\index{recto} page of that leaf, 
%and the side of the
%lefthand leaf that you see is called the \emph{verso}\index{verso} page 
%of that leaf. 
%So, a leaf has a recto and a verso page. Recto pages are the odd-numbered 
%pages and verso pages are even-numbered.
%
%   Then there is the question of folios. The typographical term for
%the number of a page is \emph{folio}\index{folio}.
%This is not to be confused with
%the same term as used in `Shakespeare's First Folio' where the reference is
%to the height and width of the book, nor to its use in the phrase
%`\emph{folio} signature'\index{signature} where the term refers to the 
%number of times a printed sheet is folded. 
%Not every page in a book has a printed
%folio, and there may be pages that do not have a folio at all. Pages with
%folios, whether printed or not, form the \emph{pagination}\index{pagination} 
%of the book. Pages
%that are not counted in the pagination have no folios.
%
% I have not been able to find what I think is a good
%definition for `type' as it seems to be used in different contexts with
%different meanings. It appears to be a kind of generic word; for instance
%there are type designers, type cutters, type setters, type foundries,...
%For my purposes I propose that \emph{type}\index{type|seealso{typeface}} is 
%one or more printable characters (or variations or extensions to this idea).  
%Printers use the term \emph{sort}\index{sort} to refer to one piece of lead
%type.
%
%   A \emph{typeface}\index{typeface} is a set of one or more fonts, in one
%or more sizes, designed as a stylistic whole. 
%
%   A \emph{font}\index{font} is a set of characters. In the days of 
%metal type and hot lead a font meant a complete alphabet and auxiliary
%characters in a given size. More recently it is taken to mean a complete
%set of characters regardless of size. A font of roman type normally
%consists of CAPITAL LETTERS, \textsc{small capitals}, lowercase letters,
%numbers, punctuation marks, ligatures (such as `fi' and `ffi'), and a
%few special symbols like \&.
%
%   A \emph{font family}\index{font!family} is a set of fonts designed to
%work harmoniously together, such as a pair of roman and italic fonts.
%
%   The size of a font\index{font} is expressed in points\index{point} 
%(72.27 points equals 1 inch
%equals 25.4 millimeters). The size is a rough indication of the height
%of the tallest character, but different fonts with the same size may have
%very different actual heights. Traditionally font sizes were referred to
%by names (see \tref{tab:fontsizes}) but nowadays just the number of points 
%is used.
%
%
%\begin{table}
%\centering
%\caption{Traditional font size designations} \label{tab:fontsizes}
%\begin{tabular}{cl@{\hspace{2em}}cl} \toprule
%Points & Name & Points & Name \\ \midrule
%%%3      & Excelsior \\
%\phantom{0}3      & Excelsior &
%11     &  Small Pica \\
%\phantom{0}3\rlap{\slashfrac{1}{2}} & Brilliant &
%12     & Pica \\
%\phantom{0}4      & Diamond &
%14     & English \\
%\phantom{0}5      & Pearl &
%18     & Great Primer \\
%\phantom{0}5\rlap{\slashfrac{1}{2}} & Agate &
%24     & Double (or Two Line) Pica \\
%\phantom{0}6      & Nonpareil &
%28     & Double (or Two Line) English \\
%\phantom{0}6\rlap{\slashfrac{1}{2}} & Mignonette &
%36     & Double (or Two Line) Great Primer \\
%\phantom{0}7      & Minion &
%48     & French Canon (or Four Line Pica) \\
%\phantom{0}8      & Brevier &
%60     & Five Line Pica \\
%\phantom{0}9      & Bourgeois &
%72     & Six line Pica \\
%10     & Long Primer &
%%%16     & Columbian \\
%%%20     & Paragon \\
%%%22     & Double Small Pica \\
%%%32     & Four Line Brevier \\
%%%40     & Double Paragon \\
%%%44     & Meridian \\
%96     & Eight Line Pica \\ \bottomrule
%\end{tabular}
%\end{table}
%
%
%
%    The typographers' and printers' term for the vertical space between
%the lines of normal text is \emph{leading}\index{leading}, which is also
%usually expressed in points and is usually larger than the font size.
%A convention for describing the font and leading is to give the font size 
%and leading separated by a slash; for instance $10/12$ for a
%10pt font set with a 12pt leading, or $12/14$ for a 12pt font set with a
%14pt leading.
%
%    The normal length of a line of text is often called the 
%\emph{measure}\index{measure} and is normally specified in terms of
%picas\index{pica} where 1 pica equals 12 points (1pc = 12pt).
%
%    Documents may be described as being typeset with a particular font
%with a particular size and a particular leading on a particular measure;
%this is normally given in a shorthand form. 
%A 10pt font with 11pt leading on a 20pc measure is described as
%\abyb{10/11}{20}, and \abyb{14/16}{22} describes a 14pt font
%with 16pt leading set on a a 22pc measure.
%
%\section{Units of measurement}
%
%    Typographers and printers use a mixed system of units, some of which
%we met above. The fundamental unit is the point; \tref{tab:units} lists 
%the most common units employed.
%
%\begin{table}
%\centering
%\caption{Printers units} \label{tab:units}
%\begin{tabular}{ll} \toprule
%Name (abbreviation) & Value \\ \midrule
%point (pt)\index{point}\index{pt}          &            \\
%pica (pc)\index{pica}\index{pc}           & 1pc = 12pt \\
%inch (in)\index{inch}\index{in}           & 1in = 72.27pt \\
%centimetre (cm)\index{centimetre}\index{cm}     & 2.54cm = 1in \\
%millimetre (mm)\index{millimetre}\index{mm}     & 10mm = 1cm \\ 
%big point (bp)\index{big point}\index{bp}      & 72bp = 72.27pt \\
%didot point (dd)\index{didot point}\index{dd}    & 1157dd = 1238pt \\
%cicero (cc)\index{cicero}\index{cc}         & 1cc = 12dd \\
%\bottomrule
%\end{tabular}
%\end{table}
%
%    Points\index{point} and picas\index{pica} 
%are the traditional printers units used in English-speaking countries. 
%The didot point\index{didot point} and cicero\index{cicero} are the 
%corresponding units used in continental Europe. In Japan `kyus'\index{kyus}
%(a quarter of a millimetre) may be used as the unit of measurement.
%Inches\index{inch} and centimetres\index{centimetre} are the units that we
%are all, or should be, familiar with.
%
%    The point system was invented by Pierre Fournier le jeune in 1737 with
%a length of 0.349mm. Later in the same century Fran\c{c}ois-Ambroise Didot
%introduced his point system with a length of 0.3759mm. This is the value
%still used in Europe. Much later, in 1886, the American Type Founders
%Association settled on 0.013837in as the standard size for the point, and
%the British followed in 1898. Conveniently for those who are not entirely
%metric in their thinking this means that 
%six picas are approximately equal to one inch.
%
%    The big point\index{big point} 
%is somewhat of an anomaly in that it is a recent
%invention. It tends to be used
%in page markup languages, like \pscript\footnote{\pscript{} is a 
%registered trademark of Adobe Systems Incorporated.\label{fn:ps}},
%in order to make calculations quicker and easier.
%
%    The above units are all constant in value. There are also some units
%whose value depends on the particular font\index{font} being used. 
%The \textit{em}\index{em}
%is the nominal height of the current font; it is used as a width measure.
%An \textit{en}\index{en} is half an em.
%The \textit{ex}\index{ex} is
%nominally the height of the letter `x' in the current font. You may also
%come across the term \textit{quad}\index{quad}, often as in a phrase
%like `starts with a quad space'. It is a length defined in terms of
%an em, often a quad is 1em.
%
%
%%#% extend
%
