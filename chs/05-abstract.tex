\chapter{Abstracts}


    Abstracts\index{abstract} do not normally appear in books but 
they are often an essential
part of an article in a technical journal. Reports may or may not
include an abstract, but if so it will often be called a `Summary'.
There may be an even shorter abstract as well, often called an 
`Executive Summary', for those who feel that details are irrelevant.

    In the standard classes appearance of the \Ie{abstract} environment
is fixed. The class provides a set of controls adjusting the appearance of
an \Ie{abstract} environment.

    Questions about how to have a 
one-column\index{column!single} abstract\index{abstract} in a 
two-column\index{column!double} paper
seem to pop up fairly regularly on the
\texttt{comp.text.tex} newsgroup. While an answer based on responses
on \ctt\ is provided in the FAQ,
the class provides a more author-friendly means
of accomplishing this. 

\section{Styling}

\index{abstract!styling|(}

    Much of this part of the class is a reimplementation of the 
\Lpack{abstract} package~\cite{ABSTRACT}.

   The typeset format of an \Ie{abstract} in a \Lclass{report} or 
\Lclass{article} class document depends on the class
options.\footnote{The \texttt{abstract} environment is not 
available for the \Lclass{book} class.} The formats are:
\begin{itemize}
\item \Lopt{titlepage} class option: The abstract 
    heading\indextwo{heading}{abstract} (i.e., value of 
   \cmd{\abstractname}) is typeset centered in a bold font; the text is set in
   the normal font and to the normal width.
\item \Lopt{twocolumn} class option: The abstract 
   heading\indextwo{heading}{abstract} is typeset like
   an unnumbered section; the text is set in the normal font and to the
   normal width (of a single column\index{column!single}). 
\item Default (neither of the above class options): The abstract 
   heading\indextwo{heading}{abstract}
   is typeset centered in a small bold font; the text is set in a small
   font and indented like the \Ie{quotation} environment.
\end{itemize}

   This class provides an \Ie{abstract} environment and
handles to modify the typesetting of an \Ie{abstract}.

\begin{syntax}
\senv{abstract} text \eenv{abstract} \\
\end{syntax}
\glossary(abstract)%
  {\senv{abstract}}%
  {Environment for typesetting an abstract.}
There is nothing special about using the \Ie{abstract} environment. 
Formatting is controlled by the macros described below.

\begin{syntax}
\cmd{\abstractcol} \\
\cmd{\abstractintoc} \\
\cmd{\abstractnum} \\
\cmd{\abstractrunin} \\
\end{syntax}
\glossary(abstractcol)%
  {\cs{abstractcol}}%
  {Declares that an abstract in a \Popt{twocolumn} document should be
   typeset like an unnumbered chapter.}
\glossary(abstractintoc)%
  {\cs{abstractintoc}}%
  {Specifies that the abstact's title is to be added to the ToC.} 
\glossary(abstractnum)%
  {\cs{abstractnum}}%
  {Specifies that an abstract is to be typeset like a numbered chapter.}
\glossary(abstractrunin)%
  {\cs{abstractrunin}}%
  {Specifies that the title of an abstract should be set as a run-in heading.}

    The normal format for an abstract is with a centered, bold title
and the text in a small font, inset from the margins\index{margin}.

The \cmd{\abstractcol} declaration specifies that an abstract in a
\Lopt{twocolumn} class option document should be typeset like a
normal, unnumbered chapter.
The \cmd{\abstractintoc} specifies that the abstract title should
be added to the \toc. The declaration \cmd{\abstractnum} specifies that the 
abstract should be typeset like a numbered chapter and 
\cmd{\abstractrunin} specifies that
the title of the abstract should look like a run-in heading; these two
declarations are mutually exclusive. Note that the \cmd{\abstractnum}
declaration has no effect if the abstract is in the \cmd{\frontmatter}.

\begin{syntax}
\cmd{\abstractname} \\
\cmd{\abstractnamefont} \\
\cmd{\abstracttextfont} \\
\end{syntax}
\glossary(abstractname)%
  {\cs{abstractname}}%
  {An abstract's title.}
\glossary(abstractnamefont)%
  {\cs{abstractnamefont}}%
  {Font for typesetting and abstract's title (\cs{abstractname}).}
\glossary(abstracttextfont)%
  {\cs{abstracttextfont}}%
  {Font for typesetting the body text of an abstract.}
\cmd{\abstractname} (default `\abstractname') is used as the title for
the \Ie{abstract} environment and is set using the \cmd{\abstractnamefont}.
The body of the abstract is typeset using the \cmd{\abstracttextfont}.
These two commands can be redefined to change the fonts if you wish.
The default definitions are 
\begin{lcode}
\newcommand{\abstractnamefont}{\normalfont\small\bfseries}
\newcommand{\abstracttextfont}{\normalfont\small}
\end{lcode}

\begin{syntax}
\lnc{\absleftindent} \lnc{\absrightindent} \\
\lnc{\absparindent} \lnc{\absparsep} \\
\end{syntax}
\glossary(absleftindent)%
  {\cs{absleftindent}}%
  {Indentation of the left of the \Pe{abstract} text.}
\glossary(absrightindent)%
  {\cs{absrightindent}}%
  {Indentation of the right of the \Pe{abstract} text.}
\glossary(absparindent)%
  {\cs{absparindent}}%
  {Paragraph indent in the \Pe{abstract} environment.}
\glossary(absparsep)%
  {\cs{absparsep}}%
  {Paragraph separation in the \Pe{abstract} environment.}
   This version of \Ie{abstract} uses a \Ie{list} environment for typesetting
the text. These four lengths can be changed (via \cmd{\setlength}
 or \cmd{\addtolength}) to adjust
the left and right margins\index{margin}, the paragraph indentation\index{paragraph!indentation}, and the vertical skip
between paragraphs in this environment. 
The default values depend on whether or not the \Lopt{twocolumn}
class option is used. The general layout parameters for lists are illustrated
in \fref{fig:listlay}.

\begin{syntax}
\cmd{\abslabeldelim}\marg{text} \\
\end{syntax}
\glossary(abslabeldelim)%
  {\cs{abslabeldelim}\marg{text}}%
  {\meta{text} is typeset immediately after \cs{abstractname} in a run-in
   heading.}
If the \cmd{\abstractrunin} declaration has been given, 
the heading is typeset as a run-in heading. That is, it is the 
first piece of text on the first line of the abstract text.
The \meta{text} argument of \cmd{\abslabeldelim} is typeset
immediately after the heading. By default it is defined to do nothing, but
if you wanted, for example, the \cmd{\abstractname} to be followed by 
a colon and some extra space you could specify
\begin{lcode}
\abslabeldelim{:\quad}
\end{lcode}

\begin{syntax}
\cmd{\absnamepos} \\
\end{syntax}
\glossary(absnamepos)%
  {\cs{absnamepos}}%
  {Position of a non run-in title for an \Pe{abstract} (\texttt{flushleft}, 
   \texttt{center}, or \texttt{flushright}).}
   If the \cmd{\abstractrunin} declaration is not used then the heading 
is typeset in its own environment, specified by 
\cmd{\absnamepos}. The default definition is
\begin{lcode}
\newcommand{\absnamepos}{center}
\end{lcode}
It can be defined to be one of \Ie{flushleft}, \Ie{center},
or \Ie{flushright} to give a left, centered or right aligned heading; 
or to any
other appropriate environment which is supported by a used package\index{package}.

\begin{syntax}
\lnc{\abstitleskip} \\
\end{syntax}
\glossary(abstitleskip)%
  {\cs{abstitleskip}}
  {Space around the title of an \Pe{abstract}.}
   With the \cmd{\abstractrunin} declaration a horizontal space of length 
\lnc{\abstitleskip} is typeset
before the heading. For example, if \lnc{\absparindent} is non-zero, then:
\begin{lcode}
\setlength{\abstitleskip}{-\absparindent}
\end{lcode}
 will typeset the heading flush left.

Without the \cmd{\abstractrunin} declaration, \lnc{\abstitleskip} is 
aditional vertical 
space (either positive
or negative) that is inserted between the abstract name and the text of
the abstract.

\index{abstract!styling|)}

\section{One column abstracts}

\index{abstract!one column|(}

   The usual advice~\cite{FAQ} about creating a 
one-column\index{column!double} 
\Ie{abstract} in a 
\Lopt{twocolumn} document is to write code like this:
\begin{lcode}
\documentclass[twocolumn...]{...}
...
\twocolumn[
   \begin{@twocolumnfalse}
     \maketitle               need full-width title
     \begin{abstract}
        abstract text...
     \end{abstract}
   \end{@twocolumnfalse}
]
... hand make footnotes for any \thanks commands
...
\end{lcode}

\begin{syntax}
\senv{onecolabstract} text \eenv{onecolabstract} \\
\cmd{\saythanks} \\
\end{syntax}
\glossary(onecolabstract)%
  {\senv{onecolabstract}}%
  {Environment for typesetting a one column abstract in a two column document.}
\glossary(saythanks)%
  {\cs{saythanks}}%
  {Following a \Pe{onecolabstract} it ensures that \cs{thanks} are printed.}
The class provides a \Ie{onecolabstract} environment that you can use
for a one column\index{column!single} abstract in a \Lopt{twocolumn} document, 
and it is used like this:
\begin{lcode}
\documentclass[twocolumn...]{memoir}
...
\twocolumn[
   \maketitle               need full-width title
   \begin{onecolabstract}
     abstract text...
   \end{onecolabstract}
]
\saythanks % typesets any \thanks commands
...
\end{lcode}
The command \cmd{\saythanks} ensures that any \cmd{\thanks} texts from
an earlier \cmd{\maketitle} are printed out as normal.

    If you want, you can use the \Ie{onecolabstract} environment in place
of the \Ie{abstract} environment --- it doesn't have to be within the 
optional argument of the \cmd{\twocolumn} command. In fact, 
\Ie{onecolabstract} internally uses \Ie{abstract} for the typesetting.


\index{abstract!one column|)}


%#% extend
%#% extstart include document-divisions.tex
