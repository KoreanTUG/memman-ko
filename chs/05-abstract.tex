%\chapter{Abstracts}
\chapter{요약문}

%    Abstracts\index{abstract} do not normally appear in books but 
%they are often an essential
%part of an article in a technical journal. Reports may or may not
%include an abstract, but if so it will often be called a `Summary'.
%There may be an even shorter abstract as well, often called an 
%`Executive Summary', for those who feel that details are irrelevant.
단행본에는 요약문(abstract)\tidx{abstract,요약문}이 없다. 그러나 과학기술 저널의 논문에는 필수적 요소이다. 보고서에는 요약문이 있을 수도 없을 수도 있지만 만약 붙는다면
흔히 ‘개관(Summary)’이라는 명칭을 더 자주 사용한다. 이보다 훨씬 짧은 요약문으로 ‘개요(Executive Summary)’라 불리는 것이 있는데 이것은 세부적인 사항에
대해서는 신경쓸 필요가 없는 사람을 위한 것이다.

%    In the standard classes appearance of the \Ie{abstract} environment
%is fixed. The class provides a set of controls adjusting the appearance of
%an \Ie{abstract} environment.
표준 클래스의 \Ie{abstract} 환경의 모양은 고정되어 있다. 그러나 이 클래스에서는
\Ie{abstract} 환경의 모양을 조절할 수 있는 방법을 제공한다.

%    Questions about how to have a 
%one-column\index{column!single} abstract\index{abstract} in a 
%two-column\index{column!double} paper
%seem to pop up fairly regularly on the
%\texttt{comp.text.tex} newsgroup. While an answer based on responses
%on \ctt\ is provided in the FAQ,
%the class provides a more author-friendly means
%of accomplishing this. 
이단(two-column)\tidx{column!double,컬럼!이단} 문서에
단단(one-column)\tidx{column!single,컬럼!단단} 요약문을 붙이는 방법에
대한 질문은 \texttt{comp.text.tex} 뉴스그룹에 꽤 자주 주기적으로 올라온다.
\ctt 의 답변을 모아놓은 FAQ에 해답이 실려 있지만 이 클래스는 좀더 문서작성자 친화적인
방법을 제공한다.

%\section{Styling}
\section{요약문의 양식}

\index{abstract!styling|(}
\index{요약문!스타일짜기}

%    Much of this part of the class is a reimplementation of the 
%\Lpack{abstract} package~\cite{ABSTRACT}.
이 부분은 \Lpack{abstract} 패키지~\cite{ABSTRACT}를 재구현한 것이다.

%
%   The typeset format of an \Ie{abstract} in a \Lclass{report} or 
%\Lclass{article} class document depends on the class
%options.\footnote{The \texttt{abstract} environment is not 
%available for the \Lclass{book} class.} The formats are:
\Lclass{report}와 \Lclass{article} 클래스의 \Ie{abstract}가 조판되는
형식은 클래스 옵션에 따라 달라진다.\footnote{\Lclass{book} 클래스에는 \texttt{abstract} 환경이 없다.}

%\begin{itemize}
%\item \Lopt{titlepage} class option: The abstract 
%    heading\indextwo{heading}{abstract} (i.e., value of 
%   \cmd{\abstractname}) is typeset centered in a bold font; the text is set in
%   the normal font and to the normal width.
\begin{itemize}
\item 클래스 옵션 \Lopt{titlepage}: 요약문 표제\indextwo{heading}{abstract}\indextwo{요약문}{표제}(즉 \cmd{\abstractname} 명령의 값)가 
굵은 글씨로 가운데 정렬되고 텍스트는 정상 폰트와 정상 길이로 식자된다.
%\item \Lopt{twocolumn} class option: The abstract 
%   heading\indextwo{heading}{abstract} is typeset like
%   an unnumbered section; the text is set in the normal font and to the
%   normal width (of a single column\index{column!single}). 
\item 클래스 옵션 \Lopt{twocolumn}: 요약문 표제는 번호없는 절 제목처럼
식자되고 텍스트는 정상 폰트로, 그리고 (단단 문서) 정상 길이로 식자된다.
%\item Default (neither of the above class options): The abstract 
%   heading\indextwo{heading}{abstract}
%   is typeset centered in a small bold font; the text is set in a small
%   font and indented like the \Ie{quotation} environment.
\item 디폴트 (위의 어떤 옵션도 주어지지 않았을 때): 요약문 표제는 작은 볼드 폰트로 가운데 정렬되어 식자되고 텍스트는 작은(small) 폰트로 \Ie{quotation} 환경과
같이 들여쓰기하여 식자된다.
%\end{itemize}
\end{itemize}

%
%   This class provides an \Ie{abstract} environment and
%handles to modify the typesetting of an \Ie{abstract}.
이 클래스는 \Ie{abstract} 환경을 제공하며
\Ie{abstract} 환경의 조판 형태를 수정할 수 있다. 

%
\begin{syntax}
\senv{abstract} text \eenv{abstract} \\
\end{syntax}
\glossary(abstract)%
  {\senv{abstract}}%
  {요약문(abstract)을 조판하는 환경}
%  {Environment for typesetting an abstract.}
%There is nothing special about using the \Ie{abstract} environment. 
%Formatting is controlled by the macros described below.
\Ie{abstract} 환경의 사용법에 특별할 것은 없다. 다음 명령을 이용하여
그 형태를 제어한다.

\begin{syntax}
\cmd{\abstractcol} \\
\cmd{\abstractintoc} \\
\cmd{\abstractnum} \\
\cmd{\abstractrunin} \\
\end{syntax}
\glossary(abstractcol)%
  {\cs{abstractcol}}%
  {\Popt{twocolumn} 문서에서 번호붙지 않는 chapter와 같은 모양으로 조판하도록
  선언한다.}
%  {Declares that an abstract in a \Popt{twocolumn} document should be
%   typeset like an unnumbered chapter.}
\glossary(abstractintoc)%
  {\cs{abstractintoc}}%
  {abstract의 표제를 ToC에 추가한다.}
%  {Specifies that the abstact's title is to be added to the ToC.} 
\glossary(abstractnum)%
  {\cs{abstractnum}}%
  {abstract가 번호붙는 chapter와 같은 모양으로 식자되도록 한다.}
%  {Specifies that an abstract is to be typeset like a numbered chapter.}
\glossary(abstractrunin)%
  {\cs{abstractrunin}}%
  {abstract 표제가 문단 첫머리에 표시되도록 한다.}
%  {Specifies that the title of an abstract should be set as a run-in heading.}

%    The normal format for an abstract is with a centered, bold title
%and the text in a small font, inset from the margins\index{margin}.
요약문의 표준 형식은 볼드체 표제를 가운데정렬하고 텍스트를 작은(small) 폰트로
찍으면서 양쪽 여백을 조금 두는 것이다\index{margin}\tidx{여백,마진}

%
%The \cmd{\abstractcol} declaration specifies that an abstract in a
%\Lopt{twocolumn} class option document should be typeset like a
%normal, unnumbered chapter.
\cmd{\abstractcol} 선언은 \Lopt{twocolumn} 클래스 옵션이 주어진 문서에서 
요약문은 정상적인 번호붙지 않은 chapter처럼 식자되도록 한다.
%The \cmd{\abstractintoc} specifies that the abstract title should
%be added to the \toc. The declaration \cmd{\abstractnum} specifies that the 
%abstract should be typeset like a numbered chapter and 
%\cmd{\abstractrunin} specifies that
%the title of the abstract should look like a run-in heading; these two
%declarations are mutually exclusive. Note that the \cmd{\abstractnum}
%declaration has no effect if the abstract is in the \cmd{\frontmatter}.
\cmd{\abstractintoc} 명령은 요약문 표제가 \toc 에 추가되도록 한다. \cmd{\abstractnum}이라고 선언하면 보통의 번호붙는 chapter와 같은 모양으로 식자되고
\cmd{\abstractrunin}이라 하면 요약문의 표제가 문단을 파고드는 헤딩으로 나타나게 한다.
이 두 가지 선언은 상호배타적이다. \cmd{\abstractnum} 선언은 요약문이 \cmd{\frontmatter} 부분 안에 놓이면 효과가 없다.

\begin{syntax}
\cmd{\abstractname} \\
\cmd{\abstractnamefont} \\
\cmd{\abstracttextfont} \\
\end{syntax}
\glossary(abstractname)%
  {\cs{abstractname}}%
  {abstract의 표제}
%  {An abstract's title.}
\glossary(abstractnamefont)%
  {\cs{abstractnamefont}}%
  {abstract 표제를 식자할 폰트}
%  {Font for typesetting and abstract's title (\cs{abstractname}).}
\glossary(abstracttextfont)%
  {\cs{abstracttextfont}}%
  {abstract의 본문 텍스트를 식자할 폰트}
%  {Font for typesetting the body text of an abstract.}
%\cmd{\abstractname} (default `\abstractname') is used as the title for
%the \Ie{abstract} environment and is set using the \cmd{\abstractnamefont}.
\cmd{\abstractname} (기본값은 “\abstractname”) 명령은 \Ie{abstract} 환경의
표제로서 \cmd{\abstractnamefont} 폰트로 식자한다.
%The body of the abstract is typeset using the \cmd{\abstracttextfont}.
%These two commands can be redefined to change the fonts if you wish.
%The default definitions are 
인용문의 본문은 \cmd{\abstracttextfont} 폰트를 써서 식자하며 이 두 명령은 원한다면 재정의하여 폰트를 바꿀 수 있다.
디폴트 정의는 다음과 같다.
\begin{lcode}
\newcommand{\abstractnamefont}{\normalfont\small\bfseries}
\newcommand{\abstracttextfont}{\normalfont\small}
\end{lcode}

\begin{syntax}
\lnc{\absleftindent} \lnc{\absrightindent} \\
\lnc{\absparindent} \lnc{\absparsep} \\
\end{syntax}
\glossary(absleftindent)%
  {\cs{absleftindent}}%
  {\Pe{abstract} 텍스트의 왼쪽에 인덴트를 둔다.}
%  {Indentation of the left of the \Pe{abstract} text.}
\glossary(absrightindent)%
  {\cs{absrightindent}}%
%  {Indentation of the right of the \Pe{abstract} text.}
  {\Pe{abstract} 텍스트의 오른쪽에 인덴트를 둔다.}
\glossary(absparindent)%
  {\cs{absparindent}}%
  {\Pe{abstract} 본문의 문단 들여쓰기를 행한다.}
%  {Paragraph indent in the \Pe{abstract} environment.}
\glossary(absparsep)%
  {\cs{absparsep}}%
  {\Pe{abstract} 환경의 문단 사이에 간격을 준다.}
%  {Paragraph separation in the \Pe{abstract} environment.}
%   This version of \Ie{abstract} uses a \Ie{list} environment for typesetting
%the text. These four lengths can be changed (via \cmd{\setlength}
% or \cmd{\addtolength}) to adjust
%the left and right margins\index{margin}, the paragraph indentation\index{paragraph!indentation}, and the vertical skip
%between paragraphs in this environment. 
이 버전의 \Ie{abstract}는 \Ie{list} 환경을 사용하여 텍스트를 식자한다.
이 네 길이값들은 (\cmd{\setlength} 또는 \cmd{\addtolength}를 통하여)
왼쪽 오른쪽 여백, 문단 들여쓰기, 또는 문단 사이의 간격을 재설정할 수 있다.\index{margin}\index{paragraph!indentation}\tidx{여백,마진,문단!들여쓰기}
%The default values depend on whether or not the \Lopt{twocolumn}
%class option is used. The general layout parameters for lists are illustrated
%in \fref{fig:listlay}.
기본값은 \Lopt{twocolumn} 옵션이 클래스에 주어졌는가 여부에 따라 달라진다.
리스트 환경에 사용되는 레이아웃 파라미터들은 \fref{fig:listlay}에 
소개되어 있다.

\begin{syntax}
\cmd{\abslabeldelim}\marg{text} \\
\end{syntax}
\glossary(abslabeldelim)%
  {\cs{abslabeldelim}\marg{text}}%
  {\meta{text}는 문단 첫머리 헤딩에서 \cs{abstractname} 직후에 식자되는 텍스트이다.}
%  {\meta{text} is typeset immediately after \cs{abstractname} in a run-in
%   heading.}
%If the \cmd{\abstractrunin} declaration has been given, 
%the heading is typeset as a run-in heading. That is, it is the 
%first piece of text on the first line of the abstract text.
%The \meta{text} argument of \cmd{\abslabeldelim} is typeset
%immediately after the heading. By default it is defined to do nothing, but
%if you wanted, for example, the \cmd{\abstractname} to be followed by 
%a colon and some extra space you could specify
만약 \cmd{\abstractrunin} 선언이 주어지면 표제가 문단 첫머리 헤딩으로 식자된다.
즉 표제기 문단 첫 줄의 맨 처음에 나타난다. \cmd{\abslabeldelim} 명령의 \meta{text} 옵션은
이 파고드는 표제 직후에 놓인다. 기본적으로 아무 것도 오지 않지만 원한다면
예컨대 콜론이나 추가 간격을 줄 수 있다.
\begin{lcode}
\abslabeldelim{:\quad}
\end{lcode}

\begin{syntax}
\cmd{\absnamepos} \\
\end{syntax}
\glossary(absnamepos)%
  {\cs{absnamepos}}%
  {\Pe{abstract} 표제가 문단 첫머리가 아닐 적에 그 정렬 위치를 지정한다.
  즉 \texttt{flushleft}, \texttt{center}, \texttt{flushright} 가운데 하나이다.}
%  {Position of a non run-in title for an \Pe{abstract} (\texttt{flushleft}, 
%   \texttt{center}, or \texttt{flushright}).}
%   If the \cmd{\abstractrunin} declaration is not used then the heading 
%is typeset in its own environment, specified by 
%\cmd{\absnamepos}. The default definition is
만약 \cmd{\abstractrunin} 선언이 사용되지 않으면 표제는 
\cmd{\absnamepos}에 의하여 지정되는 정렬방식으로 식자한다.
기본값은
\begin{lcode}
\newcommand{\absnamepos}{center}
\end{lcode}
으로 되어 있다.
%It can be defined to be one of \Ie{flushleft}, \Ie{center},
%or \Ie{flushright} to give a left, centered or right aligned heading; 
%or to any
%other appropriate environment which is supported by a used package\index{package}.
\Ie{flushleft}, \Ie{flushright}, \Ie{center} 가운데 하나를 정할 수 
있으며 각각 왼쪽 정렬, 오른쪽 정렬, 가운데 정렬한다.
만약 사용중인 패키지가 제공하는 정렬 환경이 있다면\index{package}\index{패키지}
그 환경을 사용할 수 있다.

\begin{syntax}
\lnc{\abstitleskip} \\
\end{syntax}
\glossary(abstitleskip)%
  {\cs{abstitleskip}}
  {\Pe{abstract} 표제의 상하좌우 간격}
%  {Space around the title of an \Pe{abstract}.}
%   With the \cmd{\abstractrunin} declaration a horizontal space of length 
%\lnc{\abstitleskip} is typeset
%before the heading. For example, if \lnc{\absparindent} is non-zero, then:
\cmd{\abstractrunin} 선언이 있으면 \lnc{\abstitleskip}은 표제의
전후에 오는 수평간격을 설정한다. 예를 들면 \lnc{\apsparindent}가 $0$이 아니면
\begin{lcode}
\setlength{\abstitleskip}{-\absparindent}
\end{lcode}
% will typeset the heading flush left.
이 결과는 표제를 왼쪽 정렬할 것이다.

%Without the \cmd{\abstractrunin} declaration, \lnc{\abstitleskip} is 
%aditional vertical 
%space (either positive
%or negative) that is inserted between the abstract name and the text of
%the abstract.
\cmd{\abstractrunin} 선언이 없으면 \lnc{\abstitleskip}은
요약문 표제와 본문 사이의 추가 수직 간격(음수일 수도 있고 양수일 수도 있다)을
설정한다.\index{abstract!styling|)}
%\index{abstract!styling|)}
%
%\section{One column abstracts}
\section{단단 요약문}
%
\index{abstract!one column|(}\index{요약문!단단}
%
%   The usual advice~\cite{FAQ} about creating a 
%one-column\index{column!double} 
%\Ie{abstract} in a 
%\Lopt{twocolumn} document is to write code like this:
\Lopt{twocolumn} 문서에서 단단 요약문을 만드는 데 대한
일반적 조언~\cite{FAQ}은 다음과 같이 하라는 것이다.
\begin{lcode}
\documentclass[twocolumn...]{...}
...
\twocolumn[
   \begin{@twocolumnfalse}
     \maketitle               need full-width title
     \begin{abstract}
        abstract text...
     \end{abstract}
   \end{@twocolumnfalse}
]
... hand make footnotes for any \thanks commands
...
\end{lcode}

\begin{syntax}
\senv{onecolabstract} text \eenv{onecolabstract} \\
\cmd{\saythanks} \\
\end{syntax}
\glossary(onecolabstract)%
  {\senv{onecolabstract}}%
  {이단 문서에서 단단 요약문을 조판하는 환경}
%  {Environment for typesetting a one column abstract in a two column document.}
\glossary(saythanks)%
  {\cs{saythanks}}%
  {\Pe{onecolabstract}에 이어서 \cs{thanks} 명령이 인쇄되도록 한다.}
%  {Following a \Pe{onecolabstract} it ensures that \cs{thanks} are printed.}
%The class provides a \Ie{onecolabstract} environment that you can use
%for a one column\index{column!single} abstract in a \Lopt{twocolumn} document, 
%and it is used like this:
이 클래스는 \Ie{onecolabstract} 환경을 제공하는데 이것은 \Lopt{twocolumn}
문서에서 단단 요약문\index{column!single}\tidx{다단!단단}을
식자하는 데 사용한다.
\begin{lcode}
\documentclass[twocolumn...]{memoir}
...
\twocolumn[
   \maketitle               need full-width title
   \begin{onecolabstract}
     abstract text...
   \end{onecolabstract}
]
\saythanks % typesets any \thanks commands
...
\end{lcode}
%The command \cmd{\saythanks} ensures that any \cmd{\thanks} texts from
%an earlier \cmd{\maketitle} are printed out as normal.
\cmd{\saythanks} 명령은 직전의 \cmd{\maketitle} 텍스트에서
사용된 \cmd{\thanks} 명령들이 제대로 인쇄되도록 한다.

%
%    If you want, you can use the \Ie{onecolabstract} environment in place
%of the \Ie{abstract} environment --- it doesn't have to be within the 
%optional argument of the \cmd{\twocolumn} command. In fact, 
%\Ie{onecolabstract} internally uses \Ie{abstract} for the typesetting.
\Ie{onecolabstract} 환경은 \Ie{abstract}를 써야 할 곳에 마음대로 사용할 수 있다.
\cmd{\twocolumn} 명령의 옵션 인자 안에 있어야 할 필요도 없다. 실제로
\Ie{onecolabstract}는 내부적으로 \Ie{abstract}를 사용하여 식자한다.

%
%
\index{abstract!one column|)}
%
%
%%#% extend
%%#% extstart include document-divisions.tex
