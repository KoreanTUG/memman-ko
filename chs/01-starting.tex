
\chapter{Starting off} \label{chap:starting}


    As usual, the \Lclass{memoir} class is called by 
\cmd{\documentclass}\oarg{options}\texttt{\{memoir\}}. The \meta{options}\index{class options}
include being able to select a paper\indextwo{paper}{size} size from among a range of sizes, 
selecting a type size, selecting the kind of manuscript, and some related
specifically to the typesetting of mathematics.

\section{Stock paper size options}

    The stock\indextwo{stock}{size} size is the size of a single sheet of the 
paper\index{paper} you expect to put through the printer. There is a range of 
stock\indextwo{stock}{size} paper sizes from which to make a selection. 
These are listed in\index{class options!stock size} \tref{tab:sizeoptsmetric} 
through \tref{tab:sizeoptsbrit}. Also 
included in the tables are commands that will set the stock size or paper size to 
the same dimensions.

\begin{table}
\centering
\caption{Class stock metric paper size options, and commands}\label{tab:sizeoptsmetric}
\begin{tabular}{llll} \toprule
Option & Size & stock size command & page size command \\ \midrule
\Lopt{a6paper}\index{paper!size!A6}\index{stock!size!A6} 
   & \abybm{148}{105}{mm} & \cmd{\stockavi} & \cmd{\pageavi} \\
\Lopt{a5paper}\index{paper!size!A5}\index{stock!size!A5}
   & \abybm{210}{148}{mm} & \cmd{\stockav} & \cmd{\pageav} \\
\Lopt{a4paper}\index{paper!size!A4}\index{stock!size!A4} 
    & \abybm{297}{210}{mm} & \cmd{\stockaiv} & \cmd{\pageaiv} \\
\Lopt{a3paper}\index{paper!size!A3}\index{stock!size!A3}
    & \abybm{420}{297}{mm} & \cmd{\stockaiii} & \cmd{\pageaiii} \\
\Lopt{b6paper}\index{paper!size!B6}\index{stock!size!B6} 
   & \abybm{176}{125}{mm} & \cmd{\stockbvi} & \cmd{\pagebvi} \\
\Lopt{b5paper}\index{paper!size!B5}\index{stock!size!B5} 
   &  \abybm{250}{176}{mm} & \cmd{\stockbv} & \cmd{\pagebv} \\
\Lopt{b4paper}\index{paper!size!B4}\index{stock!size!B4} 
   & \abybm{353}{250}{mm} & \cmd{\stockbiv} & \cmd{\pagebiv} \\
\Lopt{b3paper}\index{paper!size!B3}\index{stock!size!B3} 
   & \abybm{500}{353}{mm} & \cmd{\stockbiii} & \cmd{\pagebiii} \\
\Lopt{mcrownvopaper}\index{paper!size!metric crown octavo}\index{stock!size!metric crown octavo}
   & \abybm{186}{123}{mm} & \cmd{\stockmetriccrownvo} & \cmd{\pagemetriccrownvo} \\
\Lopt{mlargecrownvopaper}\index{paper!size!metric large crown octavo}\index{stock!size!metric large crown octavo}
   & \abybm{198}{129}{mm} & \cmd{\stockmlargecrownvo} & \cmd{\pagemlargecrownvo} \\
\Lopt{mdemyvopaper}\index{paper!size!metric demy octavo}\index{stock!size!metric demy octavo}
   & \abybm{216}{138}{mm} & \cmd{\stockmdemyvo} & \cmd{\pagemdemyvo} \\
\Lopt{msmallroyalvopaper}\index{paper!size!metric small royal octavo}\index{stock!size!metric small royal octavo}
   & \abybm{234}{156}{mm} & \cmd{\stockmsmallroyalvo} & \cmd{\pagemsmallroyalvo} \\
\bottomrule
\end{tabular}
\glossary(a6paperco)%
  {\Popt{a6paper}}%
  {Class option for A6 stock paper size.}%
\glossary(a5paperco)%
  {\Popt{a5paper}}%
  {Class option for A5 stock paper size.}%
\glossary(a4paperco)%
  {\Popt{a4paper}}%
  {Class option for A4 stock paper size.}%
\glossary(a3paperco)%
  {\Popt{a3paper}}%
  {Class option for A3 stock paper size.}%
\glossary(b6paperco)%
  {\Popt{b6paper}}%
  {Class option for B6 stock paper size.}%
\glossary(b5paperco)%
  {\Popt{b5paper}}%
  {Class option for B5 stock paper size.}%
\glossary(b4paperco)%
  {\Popt{b4paper}}%
  {Class option for B4 stock paper size.}%
\glossary(b3paperco)%
  {\Popt{b3paper}}%
  {Class option for B3 stock paper size.}%
\glossary(mcrownvopaperco)%
  {\Popt{mcrownvopaper}}%
  {Class option for metric crown octavo stock paper size.}%
\glossary(mlargecrownvopaperco)%
  {\Popt{mlargecrownvopaper}}%
  {Class option for metric large crown octavo stock paper size.}%
\glossary(mdemyvopaperco)%
  {\Popt{mdemyvopaper}}%
  {Class option for metric demy octavo stock paper size.}%
\glossary(msmallroyalvopaperco)%
  {\Popt{msmallroyalvopaper}}%
  {Class option for metric small royal octavo stock paper size.}%
\end{table}





\begin{table}
\centering
\caption{Class stock US paper size options, and commands}\label{tab:sizeoptsus}
\begin{tabular}{llll} \toprule
Option & Size & stock size command & page size command \\ \midrule
\Lopt{dbillpaper}\index{paper!size!dollar bill}\index{stock!size!dollar bill} 
   & \abybm{7}{3}{in} & \cmd{\stockdbill} & \cmd{\pagedbill} \\
\Lopt{statementpaper}\index{paper!size!statement}\index{stock!size!statement} 
   & \abybm{8.5}{5.5}{in} & \cmd{\stockstatement} & \cmd{\pagestatement} \\
\Lopt{executivepaper}\index{paper!size!executive}\index{stock!size!executive} 
   & \abybm{10.5}{7.25}{in} & \cmd{\stockexecutive} & \cmd{\pageexecutive} \\
\Lopt{letterpaper}\index{paper!size!letterpaper}\index{stock!size!letterpaper} 
   & \abybm{11}{8.5}{in} & \cmd{\stockletter} & \cmd{\pageletter} \\
\Lopt{oldpaper}\index{paper!size!old}\index{stock!size!old} 
   & \abybm{12}{9}{in} & \cmd{\stockold} & \cmd{\pageold} \\
\Lopt{legalpaper}\index{paper!size!legal}\index{stock!size!legal} 
   & \abybm{14}{8.5}{in} & \cmd{\stocklegal} & \cmd{\pagelegal} \\
\Lopt{ledgerpaper}\index{paper!size!ledger}\index{stock!size!ledger} 
   & \abybm{17}{11}{in} & \cmd{\stockledger} & \cmd{\pageledger} \\
\Lopt{broadsheetpaper}\index{paper!size!broadsheet}\index{stock!size!broadsheet} 
   & \abybm{22}{17}{in} & \cmd{\stockbroadsheet} & \cmd{\pagebroadsheet} \\
\bottomrule
\end{tabular}
\glossary(dbillpaperco)%
  {\Popt{dbillpaper}}%
  {Class option for dollar bill stock paper size.}%
\glossary(statementpaperco)%
  {\Popt{statementpaper}}%
  {Class option for statement stock paper size.}%
\glossary(executivepaperco)%
  {\Popt{executivepaper}}%
  {Class option for executive-paper stock paper size.}%
\glossary(letterpaperco)%
  {\Popt{letterpaper}}%
  {Class option for letterpaper stock paper size.}%
\glossary(oldpaperco)%
  {\Popt{oldpaper}}%
  {Class option for old stock paper size.}%
\glossary(legalpaperco)%
  {\Popt{legalpaper}}%
  {Class option for legal-paper stock paper size.}%
\glossary(ledgerpaperco)%
  {\Popt{ledgerpaper}}%
  {Class option for ledger stock paper size.}%
\glossary(broadsheetpaperco)%
  {\Popt{broadsheetpaper}}%
  {Class option for broadsheet stock paper size.}%
\end{table}


\begin{table}
\centering
\caption{Class stock British paper size options, and commands}\label{tab:sizeoptsbrit}
\begin{tabular}{llll} \toprule
Option & Size & stock size command & page size command \\ \midrule
\Lopt{pottvopaper}\index{paper!size!pott octavo}\index{stock!size!pott octavo} 
   & \abybm{6.25}{4}{in} & \cmd{\stockpottvo} & \cmd{\pagepottvo} \\
\Lopt{foolscapvopaper}\index{paper!size!foolscap octavo}\index{stock!size!foolscap octavo} 
   & \abybm{6.75}{4.25}{in} & \cmd{\stockfoolscapvo} & \cmd{\pagefoolscapvo} \\
\Lopt{crownvopaper}\index{paper!size!crown octavo}\index{stock!size!crown octavo} 
   & \abybm{7.5}{5}{in} & \cmd{\stockcrownvo} & \cmd{\pagecrownvo} \\
\Lopt{postvopaper}\index{paper!size!post octavo}\index{stock!size!post octavo} 
   & \abybm{8}{5}{in} & \cmd{\stockpostvo} & \cmd{\pagepostvo} \\
\Lopt{largecrownvopaper}\index{paper!size!large crown octavo}\index{stock!size!large crown octavo} 
   & \abybm{8}{5.25}{in} & \cmd{\stocklargecrownvo} & \cmd{\pagelargecrown
vo} \\
\Lopt{largepostvopaper}\index{paper!size!large post octavo}\index{stock!size!large post octavo} 
   & \abybm{8.25}{5.25}{in} & \cmd{\stocklargepostvo} & \cmd{\pagelargepostvo} \\
\Lopt{smalldemyvopaper}\index{paper!size!small demy octavo}\index{stock!size!small demy octavo} 
   & \abybm{8.5}{5.675}{in} & \cmd{\stocksmalldemyvo} & \cmd{\pagesmalldemyvo} \\
\Lopt{demyvopaper}\index{paper!size!demy octavo}\index{stock!size!demy octavo} 
   & \abybm{8.75}{5.675}{in} & \cmd{\stockdemyvo} & \cmd{\pagedemyvo} \\
\Lopt{mediumvopaper}\index{paper!size!medium octavo}\index{stock!size!medium octavo} 
   & \abybm{9}{5.75}{in} & \cmd{\stockmediumvo} & \cmd{\pagemediumvo} \\
\Lopt{smallroyalvopaper}\index{paper!size!small royal octavo}\index{stock!size!small royal octavo} 
   & \abybm{9.25}{6.175}{in} & \cmd{\stocksmallroyalvo} & \cmd{\pagesmallroyalvo} \\
\Lopt{royalvopaper}\index{paper!size!royal octavo}\index{stock!size!royal octavo} 
   & \abybm{10}{6.25}{in} & \cmd{\stockroyalvo} & \cmd{\pageroyalvo} \\
\Lopt{superroyalvopaper}\index{paper!size!super royal octavo}\index{stock!size!super royal octavo} 
   & \abybm{10.25}{6.75}{in} & \cmd{\stocksuperroyalvo} & \cmd{\pagesuperroyalvo} \\
\Lopt{imperialvopaper}\index{paper!size!imperial octavo}\index{stock!size!imperial octavo} 
   & \abybm{11}{7.5}{in} & \cmd{\stockimperialvo} & \cmd{\pageimperialvo} \\
\bottomrule
\end{tabular}
\glossary(pottvopaperco)%
  {\Popt{pottvopaper}}%
  {Class option for pott octavo stock paper size.}%
\glossary(foolscapvopaperco)%
  {\Popt{foolscapvopaper}}%
  {Class option for foolscap octavo stock paper size.}%
\glossary(crownvopaperco)%
  {\Popt{crownvopaper}}%
  {Class option for crown octavo stock paper size.}%
\glossary(postvopaperco)%
  {\Popt{postvopaper}}%
  {Class option for post octavo stock paper size.}%
\glossary(largecrownvopaperco)%
  {\Popt{largecrownvopaper}}%
  {Class option for large crown octavo stock paper size.}%
\glossary(largepostvopaperco)%
  {\Popt{largepostvopaper}}%
  {Class option for large post octavo stock paper size.}%
\glossary(smalldemyvopaperco)%
  {\Popt{smalldemyvopaper}}%
  {Class option for small demy octavo stock paper size.}%
\glossary(demyvopaperco)%
  {\Popt{demyvopaper}}%
  {Class option for demy octavo stock paper size.}%
\glossary(mediumvopaperco)%
  {\Popt{mediumvopaper}}%
  {Class option for medium octavo stock paper size.}%
\glossary(smallroyalvopaperco)%
  {\Popt{smallroyalvopaper}}%
  {Class option for small royal octavo stock paper size.}%
\glossary(royalvopaperco)%
  {\Popt{royalvopaper}}%
  {Class option for royal octavo stock paper size.}%
\glossary(superroyalvopaperco)%
  {\Popt{superroyalvopaper}}%
  {Class option for super royal octavo stock paper size.}%
\glossary(imperialvopaperco)%
  {\Popt{imperialvopaper}}%
  {Class option for imperial octavo stock paper size.}%
\end{table}


There are two options that don't really fit into the tables.

\begin{itemize}
\item[\Lopt{ebook}]\index{stock!size!ebook}  
     for a stock size of \abybm{6}{9}{inches}, principally
                    for `electronic books' intended to be displayed
                    on a computer monitor
\glossary(ebookco)%
  {\Popt{ebook}}%
  {Class option for elecronic book stock size.}
\item[\Lopt{landscape}] to interchange the height and width of the stock.
\glossary(landscapeco)%
  {\Popt{landscape}}%
  {Class option to interchange height and width of stock paper size.}
\end{itemize}

    All the options, except for \Lopt{landscape}, are mutually exclusive.
The default stock\indextwo{stock}{default} paper\indextwo{paper}{size} size is 
\Lopt{letterpaper}\index{paper!size!letterpaper}\index{stock!size!letterpaper}.

   If you want to use a stock size that is not listed there are methods for doing this,
which will be described later.

\section{Type size options}

    The type size option sets the default font size throughout the document. The class 
offers a wider range of type sizes\index{type size} than usual. These are:\index{class options!type size}
\begin{itemize}
\item[\Lopt{9pt}] for 9pt as the normal type size
\glossary(9ptco)%
  {\Popt{9pt}}%
  {Class option for a 9pt body font.}
\item[\Lopt{10pt}] for 10pt as the normal type size
\glossary(10ptco)%
  {\Popt{10pt}}%
  {Class option for a 10pt body font.}
\item[\Lopt{11pt}] for 11pt as the normal type size
\glossary(11ptco)%
  {\Popt{11pt}}%
  {Class option for a 11pt body font.}
\item[\Lopt{12pt}] for 12pt as the normal type size
\glossary(12ptco)%
  {\Popt{12pt}}%
  {Class option for a 12pt body font.}
\item[\Lopt{14pt}] for 14pt as the normal type size\footnote{Note that
    for \Lopt{14pt}, \cs{huge}, \cs{Huge} and \cs{HUGE} will be the
    same as \cs{LARGE}, unless the \Lopt{extrafontsizes} option is
    also is activated.}  \glossary(14ptco)%
  {\Popt{14pt}}%
  {Class option for a 14pt body font.}
\item[\Lopt{17pt}] for 17pt as the normal type size
\glossary(17ptco)%
  {\Popt{17pt}}%
  {Class option for a 17pt body font.}
\item[\Lopt{20pt}] for 20pt as the normal type size
\glossary(20ptco)%
  {\Popt{20pt}}%
  {Class option for a 20pt body font.}
\item[\Lopt{25pt}] for 25pt as the normal type size
\glossary(25ptco)%
  {\Popt{25pt}}%
  {Class option for a 25pt body font.}
\item[\Lopt{30pt}] for 30pt as the normal type size
\glossary(30ptco)%
  {\Popt{30pt}}%
  {Class option for a 30pt body font.}
\item[\Lopt{36pt}] for 36pt as the normal type size
\glossary(36ptco)%
  {\Popt{36pt}}%
  {Class option for a 36pt body font.}
\item[\Lopt{48pt}] for 48pt as the normal type size
\glossary(48ptco)%
  {\Popt{48pt}}%
  {Class option for a 48pt body font.}
\item[\Lopt{60pt}] for 60pt as the normal type size
\glossary(60ptco)%
  {\Popt{60pt}}%
  {Class option for a 60pt body font.}
\item[\Lopt{*pt}] for an author-defined size as the normal type size
\glossary(*ptco)%
  {\Popt{*pt}}%
  {Class option for an author-defined size for the body font.}
\item[\Lopt{extrafontsizes}] Using scalable fonts that can exceed
  25pt.

  \emph{Note that this includes \cs{huge}, \cs{Huge} and \cs{HUGE}
  under \Lopt{14pt}. For \Lopt{17pt} and up, an error is thrown if
  used withput \Lopt{extrafontsizes}, no error is given for
  \Lopt{14pt}, there sizes above \cs{LARGE} will just be unavailable
  unless \Lopt{extrafontsizes} is used.}
\glossary(extrafontsizes)%
  {\Popt{extrafontsizes}}%
  {Class option for using scalable fonts that can exceed 25pt.}
\end{itemize}

    These options, except for \Lopt{extrafontsizes}, are mutually exclusive.
The default type size\indextwo{default}{type size} is \Lopt{10pt}.

    Options greater than \Lopt{17pt} or \Lopt{20pt} are of little use unless
you are using scalable fonts --- the regular Computer 
Modern\facesubseeidx{Computer Modern} bitmap fonts only go up
to 25pt. The option \Lopt{extrafontsizes} indicates that you will be using
scalable fonts that can exceed 25pt. By default this option makes 
Latin Modern in the \texttt{T1} encoding as the default font (normally
Computer Modern in the \texttt{OT1} encoding is the default).

\subsection{Extended font sizes}

    By default, if you use the \Lopt{extrafontsizes} option the default
font for the document is Latin Modern\facesubseeidx{Latin Modern} 
in the \texttt{T1} font encoding.
This is like putting 
\begin{lcode}
\usepackage{lmodern}\usepackage[T1]{fontenc}
\end{lcode}
in the documents's preamble (but with the \Lopt{extrafontsizes} option
you need not do this). 

\begin{syntax}
\verb?\newcommand*{\memfontfamily}?\marg{fontfamily} \\
\verb?\newcommand*{\memfontenc}?\marg{fontencoding} \\
\verb?\newcommand*{\memfontpack}?\marg{package} \\
\end{syntax}
\glossary(memfontfamily)%
  {\cs{memfontfamily}}%
  {Font family for the \Popt{extrafontsizes} class option (default \texttt{lmr})}
\glossary(memfontenc)%
  {\cs{memfontenc}}%
  {Font encoding for the \Popt{extrafontsizes} class option (default \texttt{T1})}
\glossary(memfontpack)%
  {\cs{memfontpack}}%
  {Font package for the \Popt{extrafontsizes} class option (default \texttt{lmodern})}
Internally the class uses \cmd{\memfontfamily} and \cmd{\memfontenc} as 
specifying
the new font and encoding, and uses \cmd{\memfontpack} as the name of the
package to be used to implement the font. The internal definitions are:
\begin{lcode}
\providecommand*{\memfontfamily}{lmr}
\providecommand*{\memfontenc}{T1}
\providecommand*{\memfontpack}{lmodern}
\end{lcode}
which result in the \texttt{lmr} font 
(Latin Modern)\facesubseeidx{Latin Modern}  in the \texttt{T1} 
encoding as the default font, which is implemented by the \Lpack{lmodern}
package. If you want a different default, say 
New Century Schoolbook\facesubseeidx{New Century Schoolbook}
(which comes in the \texttt{T1} encoding), then 
\begin{lcode}
\newcommand*{\memfontfamily}{pnc}
\newcommand*{\memfontpack}{newcent}
\documentclass[...]{memoir}
\end{lcode}
will do the trick, where the \cs{newcommand*}s are put \emph{before} the 
\cs{documentclass} declaration (they will then override the \cs{provide...}
definitions within the class code).

    If you use the \Lopt{*pt} option then you have to supply a \file{clo}
file containing all the size and space specifications for your chosen font 
size, and also tell \Mname\ the name of the file. \emph{Before} the 
\cmd{\documentclass} command define two macros, \cmd{\anyptfilebase} and
\cmd{\anyptsize} like: 
\begin{syntax}
  \verb|\newcommand*{\anyptfilebase}|\marg{chars}\\
  \verb|\newcommand*{\anyptsize}|\marg{num} 
\end{syntax}
\glossary(anyptsize)%
  {\cs{anyptsize}}%
  {Second part (the pointsize) of the name the \texttt{clo} file for the
   \Popt{*pt} class option (default \texttt{10}).}
\glossary(anyptfilebase)%
  {\cs{anyptfilebase}}%
  {First part of the name of the \texttt{clo} file  for the 
   \Popt{*pt} class option (default \texttt{mem}).}
  
When it comes time to get the font size and spacing information \Mname\
will try and input a file called \verb?\anyptfilebase\anyptsize.clo? which
you should have made available; the \cmd{\anyptsize} \meta{num} must be an
integer.\footnote{If it is not an integer then \tx\ could get confused
as to the name of the file --- it normally expects there to be only one
period (.) in the name of a file.} Internally, the class specifies
\begin{lcode}
\providecommand*{\anyptfilebase}{mem}
\providecommand*{\anyptsize}{10}
\end{lcode}
which names the default as \file{mem10.clo}, which is for a 10pt font. 
If, for example, you have an 18pt font you want to use, then
\begin{lcode}
\newcommand*{\anyptfilebase}{myfont}
\newcommand*{\anyptsize}{18}
\documentclass[...*pt...]{memoir}
\end{lcode}
will cause \ltx\  to try and input the \texttt{myfont18.clo} file that
you should have provided. Use one
of the supplied \file{clo} files, such as \file{mem10.clo} or \file{mem60.clo}
as an example of what must be specified in your \file{clo} file.

\section{Printing options}

    This group of options\index{class options!printing} includes:

\begin{itemize}
\item[\Lopt{twoside}] for when the document will be published with printing
                        on both sides of the paper.
\glossary(twosideco)%
  {\Popt{twoside}}%
  {Class option for text on both sides of the paper.}
\item[\Lopt{oneside}] for when the document will be published with only
                        one side of each sheet being printed on.
\glossary(onesideco)%
  {\Popt{oneside}}%
  {Class option for text on only one side of the paper.}

                        The \Lopt{twoside} and \Lopt{oneside} options
                        are mutually exclusive.

\item[\Lopt{onecolumn}] only one column\index{column!single} of text on a page.
\glossary(onecolumnco)%
  {\Popt{onecolumn}}%
  {Class option for a single column.}
\item[\Lopt{twocolumn}] two equal width columns\index{column!double} of text on a page.
\glossary(twocolumnco)%
  {\Popt{twocolumn}}%
  {Class option for two columns.}

                        The \Lopt{onecolumn} and \Lopt{twocolumn} options
                        are mutually exclusive.

\item[\Lopt{openright}] each chapter\index{chapter} will start on a recto page.
\glossary(openrightco)%
  {\Popt{openright}}%
  {Class option for chapters to start on recto pages.}
\item[\Lopt{openleft}] each chapter\index{chapter} will start on a verso page.
\glossary(openleftco)%
  {\Popt{openleft}}%
  {Class option for chapters to start on verso pages.}
\item[\Lopt{openany}] a chapter\index{chapter} may start on either a recto or verso page.
\glossary(openanyco)%
  {\Popt{openany}}%
  {Class option for chapters to start on a recto or a verso page.}

                        The \Lopt{openright}, \Lopt{openleft} and 
                        \Lopt{openany} options
                        are mutually exclusive.

\item[\Lopt{final}] for camera-ready copy of your labours.
\glossary(finalco)%
  {\Popt{final}}%
  {Class option for final document.}
\item[\Lopt{draft}] this marks overfull lines with black bars and enables
                      some change marking to be shown. There may be other 
                      effects as well, particularly if some packages are used.
\glossary(draftco)%
  {\Popt{draft}}%
  {Class option for draft document.}
\item[\Lopt{ms}] this tries to make the document look as though it was 
                   prepared on a typewriter. Some publishers prefer to receive
                   poor looking submissions.
\glossary(msco)%
  {\Popt{ms}}%
  {Class option for `typewritten manuscript'.}

                   The \Lopt{final}, \Lopt{draft} and \Lopt{ms} options
                   are mutually exclusive.

\item[\Lopt{showtrims}] this option prints marks at the corners of the 
                   sheet so that you can see where the stock\index{stock} 
                   must be trimmed to produce the final page size.
\glossary(showtrimsco)%
  {\Popt{showtrims}}%
  {Class option for printing trimming marks.}

\end{itemize}
%
    The defaults among the printing options\index{default!printing options} 
are \Lopt{twoside}, \Lopt{onecolumn}, \Lopt{openright}, and \Lopt{final}.

\section{Other options}

    The remaining options are:
\begin{itemize}

\item[\Lopt{leqno}]\index{class options!math} 
     equations will be numbered at the left (the default is
     to number them at the right).
\glossary(leqnoco)%
  {\Popt{leqno}}%
  {Class option for numbering equations at the left.}

\item[\Lopt{fleqn}]\index{class options!math} 
     displayed math environments will be indented an amount
     \cmd{\mathindent} from the left margin\index{margin} (the default is to
     center the environments).
\glossary(fleqnco)%
  {\Popt{fleqn}}%
  {Class option for fixed indentation of displayed math.}

\item[\Lopt{openbib}]\index{class options!bibliography} 
     each part of a bibliography\index{bibliography} entry will start on a
                        new line, with second and succeding lines indented
                        by \cmd{\bibindent} (the default is for an entry
                        to run continuously with no indentations).
\glossary(openbibco)%
  {\Popt{openbib}}%
  {Class option for indenting continuation lines in a bibliography.}

\item[\Lopt{article}]\index{class options!article} 
  typesetting \emph{simulates} the \Lclass{article} class,
  but the \cmd{\chapter} command is not disabled, basically
  \cmd{\chapter} will behave as if it was \cmd{\section}.
  Chapters\index{chapter} do not start a new page and chapter
  headings\index{heading!chapter} are typeset 
  like a section heading\index{heading!sections}. The numbering of 
  figures\index{figure}, etc., is continuous
  and not per chapter. However, a \cmd{\part} command still puts
  its heading\index{heading!part} on a page by itself.
\glossary(articleco)%
  {\Popt{article}}%
  {Class option for simulating the \Pclass{article} class.}

\item[\Lopt{oldfontcommands}]\index{class options!fonts} 
  makes the old, deprecated LaTeX version~2.09
  font commands available. Warning messages will be produced whenever
  an old font command is encountered.
\glossary(oldfontcommandsco)%
  {\Popt{oldfontcommands}}%
  {Class option for permitting obsolete, deprecated font commands.}


\item[\Lopt{fullptlayout}]\index{class options!layout truncation} 
  disable point trunction of certain layout lengths, for example
  \cmd{\textwidth}. The default is to round these of to a whole number
  of points, this option disables this feature.
  \glossary(fullptlayout)%
  {\Popt{fullptlayout}}%
  {Class option to disable point truncation of certain layout lengths.}
\end{itemize}
%
None of these options are defaulted.

\section{Remarks}

   Calling the class with no options is equivalent to:
\begin{lcode}
\documentclass[letterpaper,10pt,twoside,onecolumn,openright,final]{memoir}
\end{lcode}
   The source file for this manual starts
\begin{lcode}
\documentclass[letterpaper,10pt,extrafontsizes]{memoir}
\end{lcode}
which is overkill as both \Lopt{letterpaper} and \Lopt{10pt} are among
the default options.

    Actual typesetting only occurs within the \Ie{document} environment. The
region of the file between the \cmd{\documentclass} command and the start
of the \Ie{document} environment is called the 
\emph{preamble}\index{preamble}. This is where you ask for external packages
and define your own macros if you feel so inclined.

\begin{syntax}
\cmd{\flushbottom} \cmd{\raggedbottom} \\
\end{syntax}
\glossary(flushbottom)%
  {\cs{flushbottom}}%
  {Declaration for last line on a page to be at a constant height.}
\glossary(raggedbottom)%
  {\cs{raggedbottom}}%
  {Declaration allowing the last line on a page to be at a variable height.}
When the \Lopt{twoside} or \Lopt{twocolumn} option is selected then
typesetting is done with \cmd{\flushbottom}, otherwise it is done
with \cmd{\raggedbottom}.

    When \cmd{\raggedbottom} is in effect LaTeX makes little attempt to
keep a constant height for the typeblock\index{typeblock}; pages may run short.

    When \cmd{\flushbottom} is in effect LaTeX ensures that the typeblock\index{typeblock}
on each page is a constant height, except when a page break is deliberately
introduced when the page might run short. In order to maintain a constant
height it may stretch or shrink some vertical spaces 
(e.g., between paragraphs\index{paragraph}, around headings\index{heading} or 
 around floats\index{float} or other inserts like displayed maths).
This may have a deleterious effect on the color\index{page color} 
of some pages. 
% Serendipitously this has happened on \pref{chap:lpage} where
% there is additional space between the paragraphs\index{paragraph} (caused by the next sectional
% division having to be put at the top of the next page). You may wish to
% compare that page with the following one to see the difference in the 
% colors. 

%    I could have made the page run short by inserting \cmd{\raggedbottom}
% at an appropriate place, followed later by a \cmd{\flushbottom}.

    If you get too many strung out pages with \cmd{\flushbottom} you may
want to put \cmd{\raggedbottom} in the preamble\index{preamble}.

    If you use the \Lopt{ebook} option you may well also want to use the
\Lopt{12pt} and \Lopt{oneside} options.

%#% extend

\clearpage
\pagestyle{ruled}

%#% extstart include laying-out-page.tex

\svnidlong
{$Ignore: $}
{$LastChangedDate: 2018-03-12 11:00:16 +0100 (Mon, 12 Mar 2018) $}
{$LastChangedRevision: 589 $}
{$LastChangedBy: daleif@math.au.dk $}

