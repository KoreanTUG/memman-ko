% !TEX root = ../memman-ko.tex
%\chapter{Starting off} \label{chap:starting}
\chapter{시작하기} \label{chap:starting}


%    As usual, the \Lclass{memoir} class is called by 
%\cmd{\documentclass}\oarg{options}\texttt{\{memoir\}}. The \meta{options}\index{class options}
%include being able to select a paper\indextwo{paper}{size} size from among a range of sizes, 
%selecting a type size, selecting the kind of manuscript, and some related
%specifically to the typesetting of mathematics.

다른 클래스와 마찬가지로, \Lclass{memoir} 클래스는 
\cmd{\documentclass}\oarg{options}\texttt{\{memoir\}}로 불러온다.
옵션(option)\tidx{class options,클래스 옵션}으로는 
종이 크기,\indextwo{paper}{size}\indextwo{종이}{크기} 본문의 활자 크기, 원고의 종류, 수식 조판상의 지시 등을 지정할 수 있다.

%\section{Stock paper size options}
\section{용지 크기 옵션}

%    The stock\indextwo{stock}{size} size is the size of a single sheet of the 
%paper\index{paper} you expect to put through the printer. There is a range of 
%stock\indextwo{stock}{size} paper sizes from which to make a selection. 
%These are listed in\index{class options!stock size} \tref{tab:sizeoptsmetric} 
%through \tref{tab:sizeoptsbrit}. Also 
%included in the tables are commands that will set the stock size or paper size to 
%the same dimensions.
기본 용지 크기(stock size)\indextwo{stock}{size}\indextwo{용지}{크기}는 인쇄기에 공급되는 종이(paper)\tidx{paper,종이} 한 장의 크기를 말한다. 
옵션으로 지정할 수 있는 용지 크기\indextwo{stock}{size}\indextwo{용지}{크기}가 \tref{tab:sizeoptsmetric}부터 \tref{tab:sizeoptsbrit}까지 나열되어\tidx{class options!stock size,클래스 옵션!크기} 있으며, 여기에는 용지 크기나 종이 크기를 같은 단위(dimension)로 설정하기 위한 명령도 들어 있다.


\begin{table}
\centering
%\caption{Class stock metric paper size options, and commands}\label{tab:sizeoptsmetric}
\caption{미터법 표시 클래스 용지 옵션과 그 명령어}\label{tab:sizeoptsmetric}
\begin{tabular}{llll} \toprule
%Option & Size & stock size command & page size command \\ \midrule
옵션 & 크기 & 용지 크기 명령 & 페이지 크기 명령 \\ \midrule
%\Lopt{a6paper}\index{paper!size!A6}\index{stock!size!A6} 
\Lopt{a6paper}\tidx{paper!size!A6,종이!크기!A6}\tidx{stock!size!A6,용지!크기!A6} 
   & \abybm{148}{105}{mm} & \cmd{\stockavi} & \cmd{\pageavi} \\
%\Lopt{a5paper}\index{paper!size!A5}\index{stock!size!A5}
\Lopt{a5paper}\tidx{paper!size!A5,종이!크기!A5}\tidx{stock!size!A5,용지!크기!A5}
   & \abybm{210}{148}{mm} & \cmd{\stockav} & \cmd{\pageav} \\
%\Lopt{a4paper}\index{paper!size!A4}\index{stock!size!A4} 
\Lopt{a4paper}\tidx{paper!size!A4,종이!크기!A4}\tidx{stock!size!A4,용지!크기!A4}
    & \abybm{297}{210}{mm} & \cmd{\stockaiv} & \cmd{\pageaiv} \\
%\Lopt{a3paper}\index{paper!size!A3}\index{stock!size!A3}
\Lopt{a3paper}\tidx{paper!size!A3,종이!크기!A3}\tidx{stock!size!A3,용지!크기!A3}
    & \abybm{420}{297}{mm} & \cmd{\stockaiii} & \cmd{\pageaiii} \\
%\Lopt{b6paper}\index{paper!size!B6}\index{stock!size!B6} 
\Lopt{b6paper}\tidx{paper!size!B6,종이!크기!B6}\tidx{stock!size!B6,용지!크기!B6} 
   & \abybm{176}{125}{mm} & \cmd{\stockbvi} & \cmd{\pagebvi} \\
%\Lopt{b5paper}\index{paper!size!B5}\index{stock!size!B5} 
\Lopt{b5paper}\tidx{paper!size!B5,종이!크기!B5}\tidx{stock!size!B5용지!크기!B5} 
   &  \abybm{250}{176}{mm} & \cmd{\stockbv} & \cmd{\pagebv} \\
%\Lopt{b4paper}\index{paper!size!B4}\index{stock!size!B4} 
\Lopt{b4paper}\tidx{paper!size!B4,종이!크기!B4}\tidx{stock!size!B4용지!크기!B4} 
   & \abybm{353}{250}{mm} & \cmd{\stockbiv} & \cmd{\pagebiv} \\
%\Lopt{b3paper}\index{paper!size!B3}\index{stock!size!B3} 
\Lopt{b3paper}\tidx{paper!size!B3,종이!크기!B3}\tidx{stock!size!B3,용지!크기!B3} 
   & \abybm{500}{353}{mm} & \cmd{\stockbiii} & \cmd{\pagebiii} \\
%\Lopt{mcrownvopaper}\index{paper!size!metric crown octavo}\index{stock!size!metric crown octavo}
\Lopt{mcrownvopaper}\tidx{paper!size!metric crown octavo,종이!크기!metric crown octavo}\tidx{stock!size!metric crown octavo,용지!크기!metric crown octavo}
   & \abybm{186}{123}{mm} & \cmd{\stockmetriccrownvo} & \cmd{\pagemetriccrownvo} \\
%\Lopt{mlargecrownvopaper}\index{paper!size!metric large crown octavo}\index{stock!size!metric large crown octavo}
\Lopt{mlargecrownvopaper}\tidx{paper!size!metric large crown octavo,종이!크기!metric large crown octavo}\tidx{stock!size!metric large crown octavo,용지!크기!metric large crown octavo}
   & \abybm{198}{129}{mm} & \cmd{\stockmlargecrownvo} & \cmd{\pagemlargecrownvo} \\
%\Lopt{mdemyvopaper}\index{paper!size!metric demy octavo}\index{stock!size!metric demy octavo}
\Lopt{mdemyvopaper}\tidx{paper!size!metric demy octavo,종이!크기!metric demy octavo}\tidx{stock!size!metric demy octavo,용지!크기!metric demy octavo}
   & \abybm{216}{138}{mm} & \cmd{\stockmdemyvo} & \cmd{\pagemdemyvo} \\
%\Lopt{msmallroyalvopaper}\index{paper!size!metric small royal octavo}\index{stock!size!metric small royal octavo}
\Lopt{msmallroyalvopaper}\tidx{paper!size!metric small royal octavo,종이!크기!metric small royal octavo}\tidx{stock!size!metric small royal octavo,용지!크기!metric small royal octavo}
   & \abybm{234}{156}{mm} & \cmd{\stockmsmallroyalvo} & \cmd{\pagemsmallroyalvo} \\
\bottomrule
\end{tabular}
\glossary(a6paperco)%
  {\Popt{a6paper}}%
  {Class option for A6 stock paper size.}%
\glossary(a5paperco)%
  {\Popt{a5paper}}%
  {Class option for A5 stock paper size.}%
\glossary(a4paperco)%
  {\Popt{a4paper}}%
  {Class option for A4 stock paper size.}%
\glossary(a3paperco)%
  {\Popt{a3paper}}%
  {Class option for A3 stock paper size.}%
\glossary(b6paperco)%
  {\Popt{b6paper}}%
  {Class option for B6 stock paper size.}%
\glossary(b5paperco)%
  {\Popt{b5paper}}%
  {Class option for B5 stock paper size.}%
\glossary(b4paperco)%
  {\Popt{b4paper}}%
  {Class option for B4 stock paper size.}%
\glossary(b3paperco)%
  {\Popt{b3paper}}%
  {Class option for B3 stock paper size.}%
\glossary(mcrownvopaperco)%
  {\Popt{mcrownvopaper}}%
  {Class option for metric crown octavo stock paper size.}%
\glossary(mlargecrownvopaperco)%
  {\Popt{mlargecrownvopaper}}%
  {Class option for metric large crown octavo stock paper size.}%
\glossary(mdemyvopaperco)%
  {\Popt{mdemyvopaper}}%
  {Class option for metric demy octavo stock paper size.}%
\glossary(msmallroyalvopaperco)%
  {\Popt{msmallroyalvopaper}}%
  {Class option for metric small royal octavo stock paper size.}%
\end{table}





\begin{table}
\centering
\caption{미국 용지의 클래스 옵션과 그 명령}\label{tab:sizeoptsus}
\begin{tabular}{llll} \toprule
%Option & Size & stock size command & page size command \\ \midrule
옵션 & 크기 & 용지 크기 명령 & 페이지 크기 명령 \\ \midrule
\Lopt{dbillpaper}\tidx{paper!size!dollar bill,종이!크기!dollar bill}\tidx{stock!size!dollar bill,용지!크기!dollar bill} 
   & \abybm{7}{3}{in} & \cmd{\stockdbill} & \cmd{\pagedbill} \\
\Lopt{statementpaper}\tidx{paper!size!statement,종이!크기!statement}\tidx{stock!size!statement,용지!크기!statement} 
   & \abybm{8.5}{5.5}{in} & \cmd{\stockstatement} & \cmd{\pagestatement} \\
\Lopt{executivepaper}\tidx{paper!size!executive,종이!크기!executive}\tidx{stock!size!executive,용지!크기!executive} 
   & \abybm{10.5}{7.25}{in} & \cmd{\stockexecutive} & \cmd{\pageexecutive} \\
\Lopt{letterpaper}\tidx{paper!size!letterpaper,종이!크기!letterpaper}\tidx{stock!size!letterpaper,용지!크기!letterpaper} 
   & \abybm{11}{8.5}{in} & \cmd{\stockletter} & \cmd{\pageletter} \\
\Lopt{oldpaper}\tidx{paper!size!old,종이!크기!old}\tidx{stock!size!old,용지!크기!old} 
   & \abybm{12}{9}{in} & \cmd{\stockold} & \cmd{\pageold} \\
\Lopt{legalpaper}\tidx{paper!size!legal,종이!크기!legal}\tidx{stock!size!legal,용지!크기!legal} 
   & \abybm{14}{8.5}{in} & \cmd{\stocklegal} & \cmd{\pagelegal} \\
\Lopt{ledgerpaper}\tidx{paper!size!ledger,종이!크기!ledger}\tidx{stock!size!ledger,용지!크기!ledger} 
   & \abybm{17}{11}{in} & \cmd{\stockledger} & \cmd{\pageledger} \\
\Lopt{broadsheetpaper}\tidx{paper!size!broadsheet,종이!크기!broadsheet}\tidx{stock!size!broadsheet,용지!크기!broadsheet} 
   & \abybm{22}{17}{in} & \cmd{\stockbroadsheet} & \cmd{\pagebroadsheet} \\
\bottomrule
\end{tabular}
\glossary(dbillpaperco)%
  {\Popt{dbillpaper}}%
  {Class option for dollar bill stock paper size.}%
\glossary(statementpaperco)%
  {\Popt{statementpaper}}%
  {Class option for statement stock paper size.}%
\glossary(executivepaperco)%
  {\Popt{executivepaper}}%
  {Class option for executive-paper stock paper size.}%
\glossary(letterpaperco)%
  {\Popt{letterpaper}}%
  {Class option for letterpaper stock paper size.}%
\glossary(oldpaperco)%
  {\Popt{oldpaper}}%
  {Class option for old stock paper size.}%
\glossary(legalpaperco)%
  {\Popt{legalpaper}}%
  {Class option for legal-paper stock paper size.}%
\glossary(ledgerpaperco)%
  {\Popt{ledgerpaper}}%
  {Class option for ledger stock paper size.}%
\glossary(broadsheetpaperco)%
  {\Popt{broadsheetpaper}}%
  {Class option for broadsheet stock paper size.}%
\end{table}


\begin{table}
\centering
\caption{영국 용지의 클래스 옵션과 그 명령}\label{tab:sizeoptsbrit}
\begin{tabular}{llll} \toprule
%Option & Size & stock size command & page size command \\ \midrule
옵션 & 크기 & 용지 크기 명령 & 페이지 크기 명령 \\ \midrule
\Lopt{pottvopaper}\tidx{paper!size!pott octavo,종이!크기!pott octavo}\tidx{stock!size!pott octavo,용지!크기!pott octavo} 
   & \abybm{6.25}{4}{in} & \cmd{\stockpottvo} & \cmd{\pagepottvo} \\
\Lopt{foolscapvopaper}\tidx{paper!size!foolscap octavo,종이!크기!foolscap octavo}\tidx{stock!size!foolscap octavo,용지!크기!foolscap octavo} 
   & \abybm{6.75}{4.25}{in} & \cmd{\stockfoolscapvo} & \cmd{\pagefoolscapvo} \\
\Lopt{crownvopaper}\tidx{paper!size!crown octavo,종이!크기!crown octavo}\tidx{stock!size!crown octavo,용지!크기!crown octavo} 
   & \abybm{7.5}{5}{in} & \cmd{\stockcrownvo} & \cmd{\pagecrownvo} \\
\Lopt{postvopaper}\tidx{paper!size!post octavo,종이!크기!post octavo}\tidx{stock!size!post octavo,용지!크기!post octavo} 
   & \abybm{8}{5}{in} & \cmd{\stockpostvo} & \cmd{\pagepostvo} \\
\Lopt{largecrownvopaper}\tidx{paper!size!large crown octavo,종이!크기!large crown octavo}\tidx{stock!size!large crown octavo,용지!크기!large crown octavo} 
   & \abybm{8}{5.25}{in} & \cmd{\stocklargecrownvo} & \cmd{\pagelargecrown
vo} \\
\Lopt{largepostvopaper}\tidx{paper!size!large post octavo,종이!크기!large post octavo}\tidx{stock!size!large post octavo,용지!크기!large post octavo} 
   & \abybm{8.25}{5.25}{in} & \cmd{\stocklargepostvo} & \cmd{\pagelargepostvo} \\
\Lopt{smalldemyvopaper}\tidx{paper!size!small demy octavo,종이!크기!small demy octavo}\tidx{stock!size!small demy octavo,용지!크기!small demy octavo} 
   & \abybm{8.5}{5.675}{in} & \cmd{\stocksmalldemyvo} & \cmd{\pagesmalldemyvo} \\
\Lopt{demyvopaper}\tidx{paper!size!demy octavo,종이!크기!demy octavo}\tidx{stock!size!demy octavo,용지!크기!demy octavo} 
   & \abybm{8.75}{5.675}{in} & \cmd{\stockdemyvo} & \cmd{\pagedemyvo} \\
\Lopt{mediumvopaper}\tidx{paper!size!medium octavo,종이!크기!medium octavo}\tidx{stock!size!medium octavo,용지!크기!medium octavo} 
   & \abybm{9}{5.75}{in} & \cmd{\stockmediumvo} & \cmd{\pagemediumvo} \\
\Lopt{smallroyalvopaper}\tidx{paper!size!small royal octavo,종이!크기!small royal octavo}\tidx{stock!size!small royal octavo,용지!크기!small royal octavo} 
   & \abybm{9.25}{6.175}{in} & \cmd{\stocksmallroyalvo} & \cmd{\pagesmallroyalvo} \\
\Lopt{royalvopaper}\tidx{paper!size!royal octavo,종이!크기!royal octavo}\tidx{stock!size!royal octavo,용지!크기!royal octavo} 
   & \abybm{10}{6.25}{in} & \cmd{\stockroyalvo} & \cmd{\pageroyalvo} \\
\Lopt{superroyalvopaper}\tidx{paper!size!super royal octavo,종이!크기!super royal octavo}\tidx{stock!size!super royal octavo,용지!크기!super royal octavo} 
   & \abybm{10.25}{6.75}{in} & \cmd{\stocksuperroyalvo} & \cmd{\pagesuperroyalvo} \\
\Lopt{imperialvopaper}\tidx{paper!size!imperial octavo,종이!크기!imperial octavo}\tidx{stock!size!imperial octavo,용지!크기!imperial octavo} 
   & \abybm{11}{7.5}{in} & \cmd{\stockimperialvo} & \cmd{\pageimperialvo} \\
\bottomrule
\end{tabular}
\glossary(pottvopaperco)%
  {\Popt{pottvopaper}}%
  {Class option for pott octavo stock paper size.}%
\glossary(foolscapvopaperco)%
  {\Popt{foolscapvopaper}}%
  {Class option for foolscap octavo stock paper size.}%
\glossary(crownvopaperco)%
  {\Popt{crownvopaper}}%
  {Class option for crown octavo stock paper size.}%
\glossary(postvopaperco)%
  {\Popt{postvopaper}}%
  {Class option for post octavo stock paper size.}%
\glossary(largecrownvopaperco)%
  {\Popt{largecrownvopaper}}%
  {Class option for large crown octavo stock paper size.}%
\glossary(largepostvopaperco)%
  {\Popt{largepostvopaper}}%
  {Class option for large post octavo stock paper size.}%
\glossary(smalldemyvopaperco)%
  {\Popt{smalldemyvopaper}}%
  {Class option for small demy octavo stock paper size.}%
\glossary(demyvopaperco)%
  {\Popt{demyvopaper}}%
  {Class option for demy octavo stock paper size.}%
\glossary(mediumvopaperco)%
  {\Popt{mediumvopaper}}%
  {Class option for medium octavo stock paper size.}%
\glossary(smallroyalvopaperco)%
  {\Popt{smallroyalvopaper}}%
  {Class option for small royal octavo stock paper size.}%
\glossary(royalvopaperco)%
  {\Popt{royalvopaper}}%
  {Class option for royal octavo stock paper size.}%
\glossary(superroyalvopaperco)%
  {\Popt{superroyalvopaper}}%
  {Class option for super royal octavo stock paper size.}%
\glossary(imperialvopaperco)%
  {\Popt{imperialvopaper}}%
  {Class option for imperial octavo stock paper size.}%
\end{table}


%There are two options that don't really fit into the tables.
표에 잘 어울리지 않는 다음 두 가지 옵션을 따로 적는다.

\begin{itemize}
%%\item[\Lopt{ebook}]\index{stock!size!ebook}  
%     for a stock size of \abybm{6}{9}{inches}, principally
%                    for `electronic books' intended to be displayed
%                    on a computer monitor
\item[\Lopt{ebook}]\tidx{stock!size!ebook,용지!크기!ebook}  
     \abybm{6}{9}{inches}의 용지 크기로, 주로 컴퓨터 모니터용 `전자책'에 적합함.
\glossary(ebookco)%
  {\Popt{ebook}}%
  {Class option for elecronic book stock size.}%
%\item[\Lopt{landscape}] to interchange the height and width of the stock.%
\item[\Lopt{landscape}] 가로와 세로의 길이를 맞바꾸어 (옆으로 긴 종이로) 조판하기 위함.
\glossary(landscapeco)%
  {\Popt{landscape}}%
  {Class option to interchange height and width of stock paper size.}
\end{itemize}

%    All the options, except for \Lopt{landscape}, are mutually exclusive.
%The default stock\indextwo{stock}{default} paper\indextwo{paper}{size} size is 
%\Lopt{letterpaper}\index{paper!size!letterpaper}\index{stock!size!letterpaper}.
%The default stock\indextwo{stock}{default} paper\indextwo{paper}{size} size is 
%\Lopt{letterpaper}\index{paper!size!letterpaper}\index{stock!size!letterpaper}.
 \Lopt{landscape}를 제외한 모든 옵션은 상호배타적이므로 같이 쓰지 않는다.
용지\indextwo{stock}{default}\indextwo{용지}{기본값}와 종이 크기\indextwo{paper}{size}\indextwo{종이}{크기}의 기본값(default)은 
\Lopt{letterpaper}\tidx{paper!size!letterpaper,종이!크기!letterpaper}\tidx{stock!size!letterpaper,용지!크기!letterpaper}이다.
 
%   If you want to use a stock size that is not listed there are methods for doing this,
%which will be described later.
   만약 리스트에 없는 용지 크기를 사용하고 싶다면 그에 맞는 방법들이 있겠지만,
   나중에 설명하기로 한다.
   
%\section{Type size options}
\section{활자 크기 옵션}

%    The type size option sets the default font size throughout the document. The class 
%offers a wider range of type sizes\index{type size} than usual. These are:\index{class options!type size}
    활자 크기 옵션은 문서내 글꼴(font) 크기의 기본값을 설정한다. 이 클래스는 다음과 같이 상당히 많은 활자 크기\tidx{type size,활자 크기}를 제공한다.\tidx{class options!type size,클래스 옵션!활자 크기}
\begin{itemize}
%\item[\Lopt{9pt}] for 9pt as the normal type size
\item[\Lopt{9pt}] 기본 글꼴 크기 9pt
\glossary(9ptco)%
  {\Popt{9pt}}%
  {Class option for a 9pt body font.}
%\item[\Lopt{10pt}] for 10pt as the normal type size
\item[\Lopt{10pt}] 기본 글꼴 크기 10pt
\glossary(10ptco)%
  {\Popt{10pt}}%
  {Class option for a 10pt body font.}
%\item[\Lopt{11pt}] for 11pt as the normal type size
\item[\Lopt{11pt}] 기본 글꼴 크기 11pt
\glossary(11ptco)%
  {\Popt{11pt}}%
  {Class option for a 11pt body font.}
%\item[\Lopt{12pt}] for 12pt as the normal type size
\item[\Lopt{12pt}] 기본 글꼴 크기 12pt
\glossary(12ptco)%
  {\Popt{12pt}}%
  {Class option for a 12pt body font.}
%\item[\Lopt{14pt}] for 14pt as the normal type size\footnote{Note that
%    for \Lopt{14pt}, \cs{huge}, \cs{Huge} and \cs{HUGE} will be the
%    same as \cs{LARGE}, unless the \Lopt{extrafontsizes} option is
%    also is activated.}  \glossary(14ptco)%
\item[\Lopt{14pt}] 기본 글꼴 크기 14pt\footnote{
    \Lopt{14pt} 옵션의 경우, \Lopt{extrafontsizes} 옵션이 없으면,  
    \cs{huge}, \cs{Huge}, \cs{HUGE}가 
    \cs{LARGE}와 같은 크기로 나타난다.}  \glossary(14ptco)%
  {\Popt{14pt}}%
  {Class option for a 14pt body font.}
%\item[\Lopt{17pt}] for 17pt as the normal type size
\item[\Lopt{17pt}] 기본 글꼴 크기 17pt
\glossary(17ptco)%
  {\Popt{17pt}}%
  {Class option for a 17pt body font.}
%\item[\Lopt{20pt}] for 20pt as the normal type size
\item[\Lopt{20pt}] 기본 글꼴 크기 20pt
\glossary(20ptco)%
  {\Popt{20pt}}%
  {Class option for a 20pt body font.}
%\item[\Lopt{25pt}] for 25pt as the normal type size
\item[\Lopt{25pt}] 기본 글꼴 크기 25pt
\glossary(25ptco)%
  {\Popt{25pt}}%
  {Class option for a 25pt body font.}
%\item[\Lopt{30pt}] for 30pt as the normal type size
\item[\Lopt{30pt}] 기본 글꼴 크기 30pt
\glossary(30ptco)%
  {\Popt{30pt}}%
  {Class option for a 30pt body font.}
%\item[\Lopt{36pt}] for 36pt as the normal type size
\item[\Lopt{36pt}] 기본 글꼴 크기 36pt
\glossary(36ptco)%
  {\Popt{36pt}}%
  {Class option for a 36pt body font.}
%\item[\Lopt{48pt}] for 48pt as the normal type size
\item[\Lopt{48pt}] 기본 글꼴 크기 48pt
\glossary(48ptco)%
  {\Popt{48pt}}%
  {Class option for a 48pt body font.}
%\item[\Lopt{60pt}] for 60pt as the normal type size
\item[\Lopt{60pt}] 기본 글꼴 크기 60pt
\glossary(60ptco)%
  {\Popt{60pt}}%
  {Class option for a 60pt body font.}
%\item[\Lopt{*pt}] for an author-defined size as the normal type size
\item[\Lopt{*pt}] 저자가 정하는 기본 글꼴 크기
\glossary(*ptco)%
  {\Popt{*pt}}%
  {Class option for an author-defined size for the body font.}
%\item[\Lopt{extrafontsizes}] Using scalable fonts that can exceed 25pt.
\item[\Lopt{extrafontsizes}] 25pt보다 큰 scalable 글꼴을 사용할 수 있음

%  \emph{Note that this includes \cs{huge}, \cs{Huge} and \cs{HUGE}
%  under \Lopt{14pt}. For \Lopt{17pt} and up, an error is thrown if
%  used withput \Lopt{extrafontsizes}, no error is given for
%  \Lopt{14pt}, there sizes above \cs{LARGE} will just be unavailable
%  unless \Lopt{extrafontsizes} is used.}
  \emph{이는, \Lopt{14pt} 옵션의 경우, \cs{huge}, \cs{Huge}와 \cs{HUGE}를 포함한다.
  \Lopt{17pt}이상의 경우에는 \Lopt{extrafontsizes}를 함께 사용하지 않으면, 
  에러가 발생한다.
  \Lopt{14pt}에서는 에러를 내지 않지만, \Lopt{extrafontsizes} 옵션이 없으면, 
  \cs{LARGE}보다 큰 크기를 나타낼 수 없다.}
\glossary(extrafontsizes)%
  {\Popt{extrafontsizes}}%
  {Class option for using scalable fonts that can exceed 25pt.}
\end{itemize}

%    These options, except for \Lopt{extrafontsizes}, are mutually exclusive.
%The default type size\indextwo{default}{type size} is \Lopt{10pt}.
    \Lopt{extrafontsizes}를 제외하면, 이 옵션들은 서로 배타적이다.
기본 활자 크기\indextwo{default}{type size}\indextwo{기본값}{활자 크기}는 \Lopt{10pt}이다.

%    Options greater than \Lopt{17pt} or \Lopt{20pt} are of little use unless
%you are using scalable fonts --- the regular Computer 
%Modern\facesubseeidx{Computer Modern} bitmap fonts only go up
%to 25pt. The option \Lopt{extrafontsizes} indicates that you will be using
%scalable fonts that can exceed 25pt. By default this option makes 
%Latin Modern in the \texttt{T1} encoding as the default font (normally
%Computer Modern in the \texttt{OT1} encoding is the default).
    \Lopt{17pt} 혹은 \Lopt{20pt}보다 큰 옵션들은 scalable fonts를 사용하지 않는 이상 
    그 사용은 제한적이다 --- the regular Computer 
Modern\facesubseeidx{Computer Modern} bitmap fonts는 25pt까지만 지원된다.
\Lopt{extrafontsizes} 옵션을 지정함으로써 25pt를 초과하는 scalable fonts를 사용할 것이라고 선언하는 것이다.
이 옵션은 \texttt{T1} 인코딩 Latin Modern을 기본 글꼴로 설정한다. (일반적으로는 
\texttt{OT1} 인코딩 Computer Modern이 기본 글꼴이다.)

%\subsection{Extended font sizes}
\subsection{확장된 글꼴 크기}

%    By default, if you use the \Lopt{extrafontsizes} option the default
%font for the document is Latin Modern\facesubseeidx{Latin Modern} 
%in the \texttt{T1} font encoding.
%This is like putting 
%\begin{lcode}
%\usepackage{lmodern}\usepackage[T1]{fontenc}
%\end{lcode}
%in the documents's preamble (but with the \Lopt{extrafontsizes} option
%you need not do this). 
    기본값을 변경하지 않는 한, \Lopt{extrafontsizes} 옵션을 쓰면, 문서의 기본 글꼴은 \texttt{T1} 인코딩 Latin Modern\facesubseeidx{Latin Modern}이다.
    이는 문서 전처리부(preamble)에 다음을 사용한 것과 같다(\Lopt{extrafontsizes} 옵션을 쓰면 불필요한 작업임). \begin{lcode}
\usepackage{lmodern}\usepackage[T1]{fontenc}
\end{lcode}

\begin{syntax}
\verb?\newcommand*{\memfontfamily}?\marg{fontfamily} \\
\verb?\newcommand*{\memfontenc}?\marg{fontencoding} \\
\verb?\newcommand*{\memfontpack}?\marg{package} \\
\end{syntax}
\glossary(memfontfamily)%
  {\cs{memfontfamily}}%
  {Font family for the \Popt{extrafontsizes} class option (default \texttt{lmr})}
\glossary(memfontenc)%
  {\cs{memfontenc}}%
  {Font encoding for the \Popt{extrafontsizes} class option (default \texttt{T1})}
\glossary(memfontpack)%
  {\cs{memfontpack}}%
  {Font package for the \Popt{extrafontsizes} class option (default \texttt{lmodern})}
%Internally the class uses \cmd{\memfontfamily} and \cmd{\memfontenc} as 
%specifying
%the new font and encoding, and uses \cmd{\memfontpack} as the name of the
%package to be used to implement the font. The internal definitions are:
내부적으로 이 클래스는 새로운 글꼴과 인코딩을 특정하기 위해 
\cmd{\memfontfamily}와 \cmd{\memfontenc}를 사용하며, 
\cmd{\memfontpack}을 글꼴 적용을 위한 패키지의 이름으로 사용한다. 
내부적으로는 다음과 같이 정의된다.
\begin{lcode}
\providecommand*{\memfontfamily}{lmr}
\providecommand*{\memfontenc}{T1}
\providecommand*{\memfontpack}{lmodern}
\end{lcode}
%which result in the \texttt{lmr} font 
%(Latin Modern)\facesubseeidx{Latin Modern}  in the \texttt{T1} 
%encoding as the default font, which is implemented by the \Lpack{lmodern}
%package. If you want a different default, say 
%New Century Schoolbook\facesubseeidx{New Century Schoolbook}
%(which comes in the \texttt{T1} encoding), then 
이것은 결과적으로 \texttt{T1} 인코딩 \texttt{lmr}(Latin Modern)\facesubseeidx{Latin Modern} 글꼴을 기본 글꼴로 설정하며, 이는 \Lpack{lmodern} 패키지로 실행된다. 다른 기본 글꼴을 사용하고 싶은 경우, 예를 들어 \texttt{T1} 인코딩 New Century Schoolbook\facesubseeidx{New Century Schoolbook}을 사용하고 싶다면 다음과 같은 방법으로 해결할 수 있다.
\begin{lcode}
\newcommand*{\memfontfamily}{pnc}
\newcommand*{\memfontpack}{newcent}
\documentclass[...]{memoir}
\end{lcode}
%will do the trick, where the \cs{newcommand*}s are put \emph{before} the 
%\cs{documentclass} declaration (they will then override the \cs{provide...}
%definitions within the class code).
위 코드에서 \cs{newcommand*}들이 \cs{documentclass} \emph{이전}에 위치해 있는데, 이는 이 클래스 코드 내부의 \cs{provide...} 정의를 덮어씌울 수 있도록 하기 위함이다.

%    If you use the \Lopt{*pt} option then you have to supply a \file{clo}
%file containing all the size and space specifications for your chosen font 
%size, and also tell \Mname\ the name of the file. \emph{Before} the 
%\cmd{\documentclass} command define two macros, \cmd{\anyptfilebase} and
%\cmd{\anyptsize} like: 
만일 \Lopt{*pt} 옵션을 사용하는 경우에는 선택한 글꼴의 크기와 스페이스(space)에 관련된 모든 값들을 포함하는 \file{clo} 파일을 연결해주어야 한다. 또한 \Mname 에 그 파일명을 알려주어야 한다. 이를 위해 \cmd{\documentclass} \emph{이전}에 다음과 같이 \cmd{\anyptfilebase}와 \cmd{\anyptsize} 두 매크로를 정의해야 한다.

\begin{syntax}
  \verb|\newcommand*{\anyptfilebase}|\marg{chars}\\
  \verb|\newcommand*{\anyptsize}|\marg{num} 
\end{syntax}
\glossary(anyptsize)%
  {\cs{anyptsize}}%
  {Second part (the pointsize) of the name the \texttt{clo} file for the
   \Popt{*pt} class option (default \texttt{10}).}
\glossary(anyptfilebase)%
  {\cs{anyptfilebase}}%
  {First part of the name of the \texttt{clo} file  for the 
   \Popt{*pt} class option (default \texttt{mem}).}
  
%When it comes time to get the font size and spacing information \Mname\
%will try and input a file called \verb?\anyptfilebase\anyptsize.clo? which
%you should have made available; the \cmd{\anyptsize} \meta{num} must be an
%integer.\footnote{If it is not an integer then \tx\ could get confused
%as to the name of the file --- it normally expects there to be only one
%period (.) in the name of a file.} 
%Internally, the class specifies
\Mname\는 글꼴의 크기와 스페이싱(spacing)에 관한 정보가 필요할 때 
사용자가 제공한 \verb?\anyptfilebase\anyptsize.clo?라는 파일에서 찾아 넣는다. 
\cmd{\anyptsize} \meta{num}은 정수여야 한다.\footnote{정수가 아니라면 \tx 이 파일명을 헷갈릴 수 있는데 --- 이는 파일명에 단 하나의 온점 (.)을 기대하기 때문이다.}
내부적으로 이 클래스는 다음과 같은 동작을 한다.
\begin{lcode}
\providecommand*{\anyptfilebase}{mem}
\providecommand*{\anyptsize}{10}
\end{lcode}
%which names the default as \file{mem10.clo}, which is for a 10pt font. 
%If, for example, you have an 18pt font you want to use, then
이로써 10pt 글꼴에 해당하는 \file{mem10.clo}를 기본값으로 지정하는 것이다. 만약, 예를 들어, 사용자가 18pt 크기의 글꼴 파일을 가지고 있어서 이를 쓰고 싶다면 다음과 같은 방법을 사용할 수 있다.
\begin{lcode}
\newcommand*{\anyptfilebase}{myfont}
\newcommand*{\anyptsize}{18}
\documentclass[...*pt...]{memoir}
\end{lcode}
%will cause \ltx\  to try and input the \texttt{myfont18.clo} file that
%you should have provided. Use one
%of the supplied \file{clo} files, such as \file{mem10.clo} or \file{mem60.clo}
%as an example of what must be specified in your \file{clo} file.
이렇게 하면 \ltx 은 사용자가 제공하는 \texttt{myfont18.clo} 파일을 찾아 필요 정보를 입력한다. 
사용자의 \file{clo} 파일에 어떤 값들이 들어있어야 하는지 알아보고 싶다면, 제공되는 \file{mem10.clo}나 \file{mem60.clo} 등을 그 예로 참고하자.

%\section{Printing options}
\section{인쇄 옵션}

%    This group of options\index{class options!printing} includes:
    인쇄 옵션은 다음과 같다.

\begin{itemize}
%\item[\Lopt{twoside}] for when the document will be published with printing
%                        on both sides of the paper.
%\glossary(twosideco)%
%  {\Popt{twoside}}%
%  {Class option for text on both sides of the paper.}
\item[\Lopt{twoside}] 종이 양면을 이용하여 인쇄할 문서를 작성한다.
\glossary(twosideco)%
  {\Popt{twoside}}%
  {Class option for text on both sides of the paper.}
  
%\item[\Lopt{oneside}] for when the document will be published with only
%                        one side of each sheet being printed on.
%\glossary(onesideco)%
%  {\Popt{oneside}}%
%  {Class option for text on only one side of the paper.}
%
%                        The \Lopt{twoside} and \Lopt{oneside} options
%                        are mutually exclusive.
\item[\Lopt{oneside}] 단면 인쇄용 문서를 작성한다.
\glossary(onesideco)%
  {\Popt{oneside}}%
  {Class option for text on only one side of the paper.}

                        \Lopt{twoside}와 \Lopt{oneside} 옵션은 서로 배타적이다.

%\item[\Lopt{onecolumn}] only one column\index{column!single} of text on a page.
%\glossary(onecolumnco)%
%  {\Popt{onecolumn}}%
%  {Class option for a single column.}
\item[\Lopt{onecolumn}] 한 페이지에 텍스트를 1단으로 출력한다.
\glossary(onecolumnco)%
  {\Popt{onecolumn}}%
  {Class option for a single column.}

%\item[\Lopt{twocolumn}] two equal width columns\index{column!double} of text on a page.
%\glossary(twocolumnco)%
%  {\Popt{twocolumn}}%
%  {Class option for two columns.}
%
%                        The \Lopt{onecolumn} and \Lopt{twocolumn} options
%                        are mutually exclusive.
\item[\Lopt{twocolumn}] 한 페이지에 너비(width)가 같은 2단\tidx{column!double,컬럼!double}의 텍스트가 출력된다.
\glossary(twocolumnco)%
  {\Popt{twocolumn}}%
  {Class option for two columns.}

                        \Lopt{onecolumn}과 \Lopt{twocolumn} 옵션은 서로 배타적이다.

%\item[\Lopt{openright}] each chapter\index{chapter} will start on a recto page.
%\glossary(openrightco)%
%  {\Popt{openright}}%
%  {Class option for chapters to start on recto pages.}
\item[\Lopt{openright}] 각 장(chapter)\tidx{chapter,장}이 펼침면의 오른쪽(홀수쪽)에서 시작된다.
\glossary(openrightco)%
  {\Popt{openright}}%
  {Class option for chapters to start on recto pages.}

%\item[\Lopt{openleft}] each chapter\index{chapter} will start on a verso page.
%\glossary(openleftco)%
%  {\Popt{openleft}}%
%  {Class option for chapters to start on verso pages.}
\item[\Lopt{openleft}] 각 장\tidx{chapter,장}이 펼침면의 왼쪽(짝수쪽)에서 시작된다.
\glossary(openleftco)%
  {\Popt{openleft}}%
  {Class option for chapters to start on verso pages.}

%\item[\Lopt{openany}] a chapter\index{chapter} may start on either a recto or verso page.
%\glossary(openanyco)%
%  {\Popt{openany}}%
%  {Class option for chapters to start on a recto or a verso page.}
%
%                        The \Lopt{openright}, \Lopt{openleft} and 
%                        \Lopt{openany} options
%                        are mutually exclusive.
%
\item[\Lopt{openany}] 각 장이 짝수^^b7홀수쪽 어디에서나 시작하도록 한다.
\glossary(openanyco)%
  {\Popt{openany}}%
  {Class option for chapters to start on a recto or a verso page.}

                        \Lopt{openright}, \Lopt{openleft}, 
                        \Lopt{openany} 옵션은 서로 배타적이다.

%\item[\Lopt{final}] for camera-ready copy of your labours.
%\glossary(finalco)%
%  {\Popt{final}}%
%  {Class option for final document.}
\item[\Lopt{final}] 문서의 인쇄용 최종본을 만든다.
\glossary(finalco)%
  {\Popt{final}}%
  {Class option for final document.}
%\item[\Lopt{draft}] this marks overfull lines with black bars and enables
%                      some change marking to be shown. There may be other 
%                      effects as well, particularly if some packages are used.
%\glossary(draftco)%
%  {\Popt{draft}}%
%  {Class option for draft document.}
\item[\Lopt{draft}] 행넘침(overfull)이 있을 때 검은 선을 그어주고, 몇 가지 변화점을 표시해준다.
                    어떤 패키지를 사용하면 다른 효과들을 나타낼 수도 있다.
\glossary(draftco)%
  {\Popt{draft}}%
  {Class option for draft document.}

%\item[\Lopt{ms}] this tries to make the document look as though it was 
%                   prepared on a typewriter. Some publishers prefer to receive
%                   poor looking submissions.
%\glossary(msco)%
%  {\Popt{ms}}%
%  {Class option for `typewritten manuscript'.}
%
%                   The \Lopt{final}, \Lopt{draft} and \Lopt{ms} options
%                   are mutually exclusive.
%
\item[\Lopt{ms}] 문서가 타자기로 친 것처럼 보이도록 한다.
                 일부 출판편집자는 정제되지 않은 듯 보이는 원고 상태를 선호하기도 한다.
\glossary(msco)%
  {\Popt{ms}}%
  {Class option for `typewritten manuscript'.}

                   \Lopt{final}, \Lopt{draft}, \Lopt{ms} 옵션은 서로 배타적이다.

%\item[\Lopt{showtrims}] this option prints marks at the corners of the 
%                   sheet so that you can see where the stock\index{stock} 
%                   must be trimmed to produce the final page size.
%\glossary(showtrimsco)%
%  {\Popt{showtrims}}%
%  {Class option for printing trimming marks.}
\item[\Lopt{showtrims}] 인쇄기를 거친 최종본에서 잘라내야할 용지\tidx{stock,용지}의 
                    여백을 표시하는 트림 마크를 찍어 준다.
\glossary(showtrimsco)%
  {\Popt{showtrims}}%
  {Class option for printing trimming marks.}

\end{itemize}
%
%    The defaults among the printing options\index{default!printing options} 
%are \Lopt{twoside}, \Lopt{onecolumn}, \Lopt{openright}, and \Lopt{final}.
인쇄 옵션\tidx{default!printing options,기본값!인쇄 옵션}의 기본값은 \Lopt{twoside}, \Lopt{onecolumn}, \Lopt{openright}, \Lopt{final}이다.

%\section{Other options}
\section{기타 옵션}

    그 밖에 남아 있는 옵션은 다음과 같다.
\begin{itemize}

%\item[\Lopt{leqno}]\index{class options!math} 
%     equations will be numbered at the left (the default is
%     to number them at the right).
%\glossary(leqnoco)%
%  {\Popt{leqno}}%
%  {Class option for numbering equations at the left.}
\item[\Lopt{leqno}]\tidx{class options!math,클래스 옵션!math} 
     수식(equation) 번호가 왼쪽에 붙는다(기본값은 오른쪽).
\glossary(leqnoco)%
  {\Popt{leqno}}%
  {Class option for numbering equations at the left.}

%\item[\Lopt{fleqn}]\index{class options!math} 
%     displayed math environments will be indented an amount
%     \cmd{\mathindent} from the left margin\index{margin} (the default is to
%     center the environments).
%\glossary(fleqnco)%
%  {\Popt{fleqn}}%
%  {Class option for fixed indentation of displayed math.}
\item[\Lopt{fleqn}]\tidx{class options!math,클래스 옵션!math} 
     별행수식(displaystyle math)이 왼쪽 마진에서 
     \cmd{\mathindent} 만큼 들여쓰기 된다(기본값은 중앙정렬).
\glossary(fleqnco)%
  {\Popt{fleqn}}%
  {Class option for fixed indentation of displayed math.}

%\item[\Lopt{openbib}]\index{class options!bibliography} 
%     each part of a bibliography\index{bibliography} entry will start on a
%                        new line, with second and succeding lines indented
%                        by \cmd{\bibindent} (the default is for an entry
%                        to run continuously with no indentations).
%\glossary(openbibco)%
%  {\Popt{openbib}}%
%  {Class option for indenting continuation lines in a bibliography.}
\item[\Lopt{openbib}]\tidx{class options!bibliography,클래스 옵션!bibliography} 
                        참고문헌(bibliography) 항목이 새로운 줄에서 시작하며, 
                        두 번째 줄부터 \cmd{\bibindent}만큼 
                        들여쓰기 된다(기본값은 들여쓰기 없이 계속 이어지는 것).
\glossary(openbibco)%
  {\Popt{openbib}}%
  {Class option for indenting continuation lines in a bibliography.}

%\item[\Lopt{article}]\index{class options!article} 
%  typesetting \emph{simulates} the \Lclass{article} class,
%  but the \cmd{\chapter} command is not disabled, basically
%  \cmd{\chapter} will behave as if it was \cmd{\section}.
%  Chapters\index{chapter} do not start a new page and chapter
%  headings\index{heading!chapter} are typeset 
%  like a section heading\index{heading!sections}. The numbering of 
%  figures\index{figure}, etc., is continuous
%  and not per chapter. However, a \cmd{\part} command still puts
%  its heading\index{heading!part} on a page by itself.
%\glossary(articleco)%
%  {\Popt{article}}%
%  {Class option for simulating the \Pclass{article} class.}
\item[\Lopt{article}]\tidx{class options!article,클래스 옵션!article} 
    article 클래스를 훙내내지만 \cmd{\chapter} 명령을 쓸 수 있다. 이 경우, 기본적으로, \cmd{\chapter}는 \cmd{\section}처럼 동작한다. 장(chapter)\tidx{chapter,장}이 새 페이지에서 시작되지 않으며 chapter headings\tidx{heading!chapter,heading!장}는 section headings\tidx{heading!sections,heading!절}처럼 식자된다. 그림 따위의 번호는 장이 바뀌어도 재설정되지 않고 연속적으로 매겨진다. 그러나 \cmd{\part} 명령은 여전히 그 자체로 한 페이지에 heading\tidx{heading!part,heading!편}을 표시한다.
\glossary(articleco)%
  {\Popt{article}}%
  {Class option for simulating the \Pclass{article} class.}

%\item[\Lopt{oldfontcommands}]\index{class options!fonts} 
%  makes the old, deprecated LaTeX version~2.09
%  font commands available. Warning messages will be produced whenever
%  an old font command is encountered.
%\glossary(oldfontcommandsco)%
%  {\Popt{oldfontcommands}}%
%  {Class option for permitting obsolete, deprecated font commands.}
\item[\Lopt{oldfontcommands}]\tidx{class options!fonts,클래스 옵션!fonts} 
오래되어 더 이상 쓰지 않는 LaTeX version~2.09 폰트 명령을 사용할 수 있게 한다. old font 명령을 사용할 때마다 경고 메지시를 받게 된다.
\glossary(oldfontcommandsco)%
  {\Popt{oldfontcommands}}%
  {Class option for permitting obsolete, deprecated font commands.}

%\item[\Lopt{fullptlayout}]\index{class options!layout truncation} 
%  disable point trunction of certain layout lengths, for example
%  \cmd{\textwidth}. The default is to round these of to a whole number
%  of points, this option disables this feature.
%  \glossary(fullptlayout)%
%  {\Popt{fullptlayout}}%
%  {Class option to disable point truncation of certain layout lengths.}
\item[\Lopt{fullptlayout}]\tidx{class options!layout truncation,클래스 옵션!layout truncation} 
어떤 layout 값(예로, \cmd{\textwidth})에서 소수점자리 버림을 허용하지 않는다. 기본값으로는 소수점 반올림을 하는데, 이 옵션은 이러한 기능을 무력화한다.
  \glossary(fullptlayout)%
  {\Popt{fullptlayout}}%
  {Class option to disable point truncation of certain layout lengths.}
\end{itemize}
%
%None of these options are defaulted.
위 옵션은 모두 기본값이 아니다.

%\section{Remarks}
\section{첨언}

%   Calling the class with no options is equivalent to:
   어떤 옵션도 없이 이 클래스를 부르는 것은 다음과 동일하다.
\begin{lcode}
\documentclass[letterpaper,10pt,twoside,onecolumn,openright,final]{memoir}
\end{lcode}
%   The source file for this manual starts
이 매뉴얼의 소스파일은 다음과 같이 시작한다.
\begin{lcode}
\documentclass[letterpaper,10pt,extrafontsizes]{memoir}
\end{lcode}
%which is overkill as both \Lopt{letterpaper} and \Lopt{10pt} are among
%the default options.
이는 다소 과한 것인데,  \Lopt{letterpaper}와 \Lopt{10pt}는 이미 기본값이기 때문이다.

%    Actual typesetting only occurs within the \Ie{document} environment. The
%region of the file between the \cmd{\documentclass} command and the start
%of the \Ie{document} environment is called the 
%\emph{preamble}\index{preamble}. This is where you ask for external packages
%and define your own macros if you feel so inclined.
문서는 \Ie{document} 환경내의 내용만 조판된다. \cmd{\documentclass} 명령에서 \Ie{document} 환경 이전까지의 영역을 \emph{전처리부}(\emph{preamble})\tidx{preamble,전처리부}라 한다. 필요하다면 이 곳에서 외부 패키지를 불러들이거나 자신의 매크로를 정의한다.

\begin{syntax}
\cmd{\flushbottom} \cmd{\raggedbottom} \\
\end{syntax}
\glossary(flushbottom)%
  {\cs{flushbottom}}%
  {Declaration for last line on a page to be at a constant height.}
\glossary(raggedbottom)%
  {\cs{raggedbottom}}%
  {Declaration allowing the last line on a page to be at a variable height.}
%When the \Lopt{twoside} or \Lopt{twocolumn} option is selected then
%typesetting is done with \cmd{\flushbottom}, otherwise it is done
%with \cmd{\raggedbottom}.
\Lopt{twoside}나 \Lopt{twocolumn} 옵션이 주어지면 \cmd{\flushbottom}, 그 외의 경우에는 \cmd{\raggedbottom}으로 조판된다.

%    When \cmd{\raggedbottom} is in effect LaTeX makes little attempt to
%keep a constant height for the typeblock\index{typeblock}; pages may run short.
\cmd{\raggedbottom}이 선언되면 LaTeX이 조판 영역(type block)\tidx{typeblock,조판 영역}의 높이를 일정하게 유지하려는 시도를 하지 않기 때문에 페이지가 조금 부족하게 채워질 수도 있다.

%    When \cmd{\flushbottom} is in effect LaTeX ensures that the typeblock\index{typeblock}
%on each page is a constant height, except when a page break is deliberately
%introduced when the page might run short. In order to maintain a constant
%height it may stretch or shrink some vertical spaces 
%(e.g., between paragraphs\index{paragraph}, around headings\index{heading} or 
% around floats\index{float} or other inserts like displayed maths).
%This may have a deleterious effect on the color\index{page color} 
%of some pages. 
\cmd{\flushbottom}이 선언되면, 사용자가 페이지 나눔을 일부러 사용하지 않는 한, LaTeX은 매 페이지마다 조판 영역\tidx{typeblock, 조판 영역}의 높이를 일정하게 유지하려고 한다. LaTeX이 페이지 높이를 일정하게 유지하기 위해 사용하는 방법은 수직 간격(예를 들면, 문단\tidx{paragraph,문단}과 문단의 사이의 간격이나 headings\tidx{heading,heading}, 떠다니는 개체(float)\tidx{float,떠다니는 개체}, 수식과 같이 삽입된 내용의 전후 공간 등)을 늘리거나 줄이는 것이다.
이는 일부의 페이지 컬러\tidx{page color,페이지 컬러}에 악영향을 줄 수도 있다.

%% === 원저자의 commenting out ======
%% Serendipitously this has happened on \pref{chap:lpage} where
%% there is additional space between the paragraphs\index{paragraph} (caused by the next sectional
%% division having to be put at the top of the next page). You may wish to
%% compare that page with the following one to see the difference in the 
%% colors. 
%
%%    I could have made the page run short by inserting \cmd{\raggedbottom}
%% at an appropriate place, followed later by a \cmd{\flushbottom}.
%% ==================================

%    If you get too many strung out pages with \cmd{\flushbottom} you may
%want to put \cmd{\raggedbottom} in the preamble\index{preamble}.
만약 너무 많은 페이지가 \cmd{\flushbottom} 때문에 늘어난 상태라면 전처리부에\cmd{\raggedbottom}을 사용하는 것이 좋다.

%    If you use the \Lopt{ebook} option you may well also want to use the
%\Lopt{12pt} and \Lopt{oneside} options.
\Lopt{ebook} 옵션을 쓸 때는 \Lopt{12pt}와 \Lopt{oneside} 옵션을 함께 사용하는 편이 좋다.

%#% extend

\clearpage
\pagestyle{ruled}

%#% extstart include laying-out-page.tex

\svnidlong
{$Ignore: $}
{$LastChangedDate: 2018-03-12 11:00:16 +0100 (Mon, 12 Mar 2018) $}
{$LastChangedRevision: 589 $}
{$LastChangedBy: daleif@math.au.dk $}

