% !TEX root = memman-ko.tex
%%%%%%%%%%%%%%%%%%%%%%%%%%%%%%%%%%%%%%%%%%%%%%%%%%%%
% \chapter{Pagination and headers} \label{chap:pagination}
\chapter{페이지 매김과 머릿글} \label{chap:pagination}
%%%%%%%%%%%%%%%%%%%%%%%%%%%%%%%%%%%%%%%%%%%%%%%%%%%%%

%\section{Introduction}

%     The focus of this chapter is on marking the pages
% with signposts so that the reader can more readily navigate through
% the document.
본 단원은 페이지에 이정표를 표시함으로써 독자가 문서를 더 수월하게 찾을 수
있도록 하는데 집중한다.

% \section{Pagination and folios}
\section{페이지 매김과 폴리오}

%     Every page in a \ltx\ document is included in the
% pagination\index{pagination}. That is,
% there is a number associated with every page and this is the value of
% the \Icn{page} counter. This value\index{pagination!changing}
% can be changed at any time via either \cmd{\setcounter} or
% \cmd{\addtocounter}.
\ltx\ 문서의 모든 페이지는 페이지 매김\tidx{pagination,페이지 매김}된다.
즉, 각 페이지에는 번호가 할당되며 이것이 \Icn{page} 카운터의 값이다.
이 값\tidx{pagination!changing,페이지 매김!변경}은 \cmd{\setcounter}나
\cmd{\addtocounter}을 통해 언제든지 바꿀 수 있다.

\begin{syntax}
\cmd{\pagenumbering}\marg{rep} \\
\cmd{\pagenumbering*}\marg{rep} \\
\end{syntax}
\glossary(pagenumbering)%
  {\cs{pagenumbering}\marg{rep}}%
  {Resets the page number to 1, and causes the folios (page numbers) to be 
   printed using the \meta{rep}
   representation (e.g., \texttt{arabic}, \texttt{roman}, \ldots)}
\glossary(pagenumbering*)%
  {\cs{pagenumbering*}\marg{rep}}%
  {Like \cs{pagenumbering} except that the page number is not reset.}
% The macros \cmd{\pagenumbering} and \cmd{\pagenumbering*} cause
% the folios\index{folio}\index{folio!changing representation}
% to be printed using the
% counter representation\index{counter representation}
% \meta{rep} for the page number, where \meta{rep} can be one of:
% \pixcrep{Alph}, \pixcrep{alph}, \pixcrep{arabic}, \pixcrep{Roman} or
% \pixcrep{roman} for uppercase and lowercase letters, arabic numerals, and
% uppercase and lowercase Roman numerals,
% respectively.\index{alphabetic numbering}\index{roman numerals} As there
% are only 26~letters, \pixcrep{Alph} or \pixcrep{alph} can only be
% used for a limited number of pages. Effectively, the macros redefine
% \cmd{\thepage} to be \verb?\rep{page}?.
매크로 \cmd{\pagenumbering}과 \cmd{\pagenumbering*}은
폴리오(면 번호)\tidx{folio!changing representation,폴리오!표시 변경}가 카운터
표시\tidx{counter representation,카운터 표시} \meta{rep}를 통해 페이지 번호를
출력하도록 할 수 있는데, \meta{rep}는 \pixcrep{Alph}, \pixcrep{alph},
\pixcrep{arabic}, \pixcrep{Roman}, 혹은 \pixcrep{roman}을 사용해 각각 알파벳
대문자와 소문자, 아라비아 숫자, 대문자와 소문자 로마 숫자 중 하나가 될 수
있다.\tidx{alphabetic numbering,알파벳 번호}\tidx{roman numerals,로마 숫자}
알파벳에는 26개의 문자만 있기 때문에 \pixcrep{Alph} 혹은 \pixcrep{alph}는
한정된 수의 페이지에만 사용될 수 있다.
사실상 이 매크로들은 \cmd{\thepage}가 \verb|\rep{page}|가 되도록 재정의한다.

%     Additionally, the \cmd{\pagenumbering}\index{pagination!changing}
% command resets the \Icn{page} counter to one; the starred version does not
% change the counter.
%     It is usual to reset the page number back to one each time the style
% is changed, but sometimes it may be desirable to have a continuous sequence
% of numbers irrespective of their displayed form, which is where
% the starred version comes in handy.
추가적으로, \cmd{\pagenumbering}\tidx{pagination!changing,페이지 매김!변경}
명령은 \Icn{page} 카운터를 1로 초기화하며, 별표 붙은 것은 카운터를 변경하지
않는다.
양식이 바뀔 때마다 페이지 번호를 다시 1로 초기화하는 것이 일반적이지만, 간혹
보여지는 형식과는 상관없이 일련의 숫자를 가지도록 해야할 때에는 별표 붙은 버전이
유용하다.

\begin{syntax}
\cmd{\savepagenumber} \\
\cmd{\restorepagenumber} \\
\end{syntax}
\glossary(savepagenumber)%
  {\cs{savepagenumber}}%
  {Saves the current page number.}
\glossary(restorepagenumber)%
  {\cs{restorepagenumber}}%
  {Sets the page number to that saved by the most recent \cs{savepagenumber}.}
% The macro \cmd{\savepagenumber} saves the current page number, and the
% macro \cmd{\restorepagenumber} sets the page number to the saved value.
% This pair of commands may be used to apparently
% interrupt\index{pagination!interrupt} the pagination.
% For example, perhaps some full page illustrations\index{illustration}
% will be electronically tipped in\index{tip in} to the document and
% pagination is not required for these.
% This could be done along the lines of:
매크로 \cmd{\savepagenumber}는 현재 페이지 번호를 저장하며, 매크로
\cmd{\restorepagenumber}는 페이지 번호를 저장된 것으로 설정한다.
이 한 쌍의 명령은 페이지 매김을 외관상
중단\tidx{pagination!interrupt,페이지 매김!중단}할 때 사용될 수 있다.
예를 들어, 전면 삽화\tidx{illustration,삽화}가 전자 문서에 끼워 넣어질 수
있는데 이 때는 페이지 매김이 불필요하다.
이는 다음과 같이 할 수 있다.
\begin{lcode}
\clearpage          % get onto next page
\savepagenumber     % save the page number
\pagestyle{empty}   % no headers or footers
%% insert the illustrations
\clearpage
\pagestyle{...}
\restorepagenumber
...
\end{lcode}
% If you try this sort of thing, you may have to adjust the restored page
% number by one.
여러분이 이런 비슷한 작업을 한다면, 복원된 페이지 번호를 1 정도 조절해야 할
수도 있다.
\begin{lcode}
\restorepagenumber
% perhaps \addtocounter{page}{1} or \addtocounter{page}{-1}
\end{lcode}
% Depending on the timing of the \cs{...pagenumber} commands and \tx's
% decisions on page breaking, this may or may not be necessary.
\cs{...pagenumber} 명령의 실행 시점과 페이지 나눔에 대한 \tx의 결정에 따라 이런
조정이 필요할 수도, 그렇지 않을 수도 있다.



% \section{Page styles} \label{sec:pagestyles}
\section{페이지 양식} \label{sec:pagestyles}

% \index{pagestyle!specifying|(}
\tidx{페이지 양식!지정|(}

%     The class provides a selection of pagestyles that you can use and if
% they don't suit, then there are means to define your own.
본 클래스는 여러분이 쓸 수 있는 페이지 양식 모음을 제공하며, 여러분 마음에 들지
않는다면 직접 정의할 수도 있다.

    % These facilities were inspired by the \Lpack{fancyhdr}
% package~\cite{FANCYHDR}, although the command set is different.
이 기능들은 \Lpack{fancyhdr} 패키지~\cite{FANCYHDR}에 영향을 받았으나,
명령어들은 다르다.

%     The standard classes provide for a footer\index{footer} and
% header\index{header} for odd and even
% pages. Thus there are four elements to be specified for a pagestyle.
% This class partitions the headers\index{header} and footers\index{footer}
% into left, center and right
% portions, so that overall there is a total of 12 elements that have
% to be specified
% for a pagestyle. You may find, though, that one of the built in pagestyles
% meets your needs so you don't have to worry about all these specifications.
표준 클래스들은 홀수와 짝수 페이지에 대한 바닥글\tidx{footer,바닥글}과
머릿글\tidx{header,머릿글}을 제공한다.
따라서 페이지 양식에 대한 네 요소를 지정할 수 있다.
이 클래스는 머릿글\tidx{header,머릿글}과 바닥글\tidx{footer,바닥글}을 왼쪽,
가운데, 그리고 오른쪽 부분으로 나누어 페이지 양식을 위해 총 12개의 요소를
지정해야 한다.
그러나 내장된 페이지 양식이 충분할 경우에는 이 모든 설정에 대해서 걱정하지
않아도 된다.

\begin{syntax}
\cmd{\pagestyle}\marg{style} \\
\cmd{\thispagestyle}\marg{style} \\
\end{syntax}
\glossary(pagestyle)%
  {\cs{pagestyle}\marg{style}}%
  {Sets the current pagestyle to \meta{style}.}
\glossary(thispagestyle)%
  {\cs{thispagestyle}\marg{style}}%
  {Sets the pagestyle to \meta{style} for the current page only.}
%    \cmd{\pagestyle} sets the current pagestyle to \meta{style}, where
% \meta{style} is a word containing only letters. On a particular
% page \cmd{\thispagestyle} can be used to override the current pagestyle for
% the one page.
\cmd{\pagestyle}은 현재 페이지 양식을 \meta{style}로 정하는데, \meta{style}은
문자로만 구성된 단어이다.
특정 페이지에서는 \cmd{\thispagestyle}을 사용해 현재 페이지 양식을 해당 쪽만
덮어쓸 수 있다.

%     Some of the class' commands automatically call \cmd{\thispagestyle}.
% For example:
클래스의 일부 명령들은 자동으로 \cmd{\thispagestyle}을 부른다.
예를 들어,
% \begin{itemize}
% \item the \Ie{titlingpage} environment calls
%       \begin{lcode}
% \thispagestyle{titlingpagestyle}
%       \end{lcode}
% \item if \cmd{\cleardoublepage} will result in an empty verso page it calls
%       \begin{lcode}
% \thispagestyle{cleared}
%       \end{lcode}
%       for the empty page.
% \end{itemize}
\begin{itemize}
\item \Ie{titlingpage} 환경은
      \begin{lcode}
\thispagestyle{titlingpagestyle}
      \end{lcode} 
      을 부른다.
\item 만약 \cmd{\cleardoublepage} 이 빈 왼쪽 면을 만든다면
      \begin{lcode}
\thispagestyle{cleared}
      \end{lcode}
      를 해당 빈 페이지에서 부른다.
\end{itemize}
% For reference, the full list is given in \tref{tab:callthispagestyle}.
참고로, 전체 목록은 \tref{tab:callthispagestyle}에 주어져 있다.

\PWnote{2009/07/26}{Added the simple pagestyle}
    % The page styles provided by the class\index{pagestyle!class} are:
본 클래스\tidx{pagestyle!class,페이지 양식!클래스}에 의해 제공되는 페이지
양식은 다음과 같다.
\begin{plainlist}
% \item[\pstyle{empty}] The headers\index{header} and footers\index{footer} are empty.
% \item[\pstyle{plain}] The header\index{header} is empty and the folio\index{folio} (page number)
%      is centered at the bottom of the page.
% \item[\pstyle{headings}] The footer\index{footer} is empty. The header\index{header} contains
%      the folio\index{folio} at the outer side of the page; on verso
%      pages the chapter name, number and title, in slanted uppercase is
%      set at the spine margin and on recto pages the section number
%      and uppercase title is set by the spine margin.
% \item[\pstyle{myheadings}] Like the \pstyle{headings} style the footer
%      is empty. You have to specify what is to go in the headers\index{header}.
% \item[\pstyle{simple}] The footer\index{footer} is empty and the
%      header\index{header} contains the folio\index{folio} (page number)
%      at the outer side of the page. It is like the \pstyle{headings}
%      style but without any title texts.
% \item[\pstyle{ruled}] The footer\index{footer} contains the
%      folio\index{folio} at the outside. The header\index{header}
%      on verso pages contains the chapter number and title in small caps
%      at the outside; on recto pges the section title is typeset at the
%      outside using the normal font. A line is drawn
     % underneath the header\index{header}.
% \item[\pstyle{Ruled}] This is like the \pstyle{ruled} style except that
%      the headers\index{header} and footers\index{footer} extend into
%      the \foredge\ margin\index{margin!foredge?\foredge}.
% \item[\pstyle{companion}] This is a copy of the pagestyle in the
%      \textit{Companion} series (e.g., see~\cite{COMPANION}). It is
%      similar to the \pstyle{Ruled} style in that the header has a rule which
%      extends to the outer edge of the marginal notes. The folios are set
%      in bold at the outer ends of the header. The chapter title is set in
%      a bold font flushright in the verso headers, and the section number
%      and title, again in bold, flushleft in the recto headers\index{header}.
%      There are no footers\index{footer}.
% \item[\pstyle{book}] This is the same as the \pstyle{plain} pagestyle.
% \item[\pstyle{chapter}]  This is the same as the \pstyle{plain} pagestyle.
% \item[\pstyle{cleared}]  This is the same as the \pstyle{empty} pagestyle.
% \item[\pstyle{part}] This is the same as the \pstyle{plain} pagestyle.
% \item[\pstyle{title}]   This is the same as the \pstyle{plain} pagestyle.
% \item[\pstyle{titlingpage}] This is the same as the \pstyle{empty} pagestyle.
% \end{plainlist}
\begin{plainlist}
\item[\pstyle{empty}] 머릿글\tidx{header,머릿글}과 바닥글\tidx{footer,바닥글}이
  비어 있다.
\item[\pstyle{plain}] 머릿글\tidx{header,머릿글}이 비어 있고
     폴리오\tidx{folio,폴리오} (페이지 번호)는 페이지 하단 중앙에 놓인다.
\item[\pstyle{headings}] 바닥글\tidx{footer,바닥글}이 비어 있다.
     머릿글\tidx{header,머릿글}은 폴리오\tidx{folio,폴리오}를 페이지 바깥쪽에
     둔다. 왼쪽 페이지에는 장 이름, 번호와 제목이 기울어진 대문자로 책 등
     여백에 놓이고, 오른쪽 페이지에는 절 이름과 대문자 제목이 책 등 여백에
     놓인다.
\item[\pstyle{myheadings}] \pstyle{heading} 양식처럼 바닥글이 비어 있다.
     여러분이 머릿글\tidx{header,머릿글}에 들어갈 내용을 지정해야 한다.
\item[\pstyle{simple}] 바닥글\tidx{footer,바닥글}이 비어 있고
     머릿글\tidx{header,머릿글}은 폴리오\tidx{folio,폴리오} (페이지 번호)를
     페이지 바깥쪽에 놓는다.
     이는 \pstyle{headings} 양식과 비슷하지만 제목 문구가 없다.
\item[\pstyle{ruled}] 바닥글\tidx{footer,바닥글}이
     폴리오\tidx{folio,폴리오}를 바깥쪽에 놓는다.
     왼쪽 페이지의 머릿글\tidx{header,머릿글}과 제목은 바깥쪽에 small caps로,
     오른족 페이지에서는 절 제목이 normal 폰트로 바깥쪽에 식자된다.
     머릿글\tidx{header,머릿글} 밑에 선이 그려진다.
\item[\pstyle{Ruled}] 이는 머릿글\tidx{header,머릿글}과
     바닥글\tidx{footer,바닥글}이 재단
     여백\tidx{margin!foredge?\foredge,여백!재단 여백}으로 확장된다는 것을
     제외하고는 \pstyle{ruled} 양식과 같다.
 \item[\pstyle{companion}] 이는 \textit{Companion} 시리즈의 페이지 양식
     사본이다 (예시로는 \cite{COMPANION}을 보라).
     이는 머릿글이 여백 주석의 바깥쪽 모서리까지 이어지는 선을 가지고 있다는
     점에서 \pstyle{Ruled}와 비슷하다.
     폴리오는 왼쪽 페이지 머릿글에 볼트체 오른쪽 정렬되어 있으며, 절 이름과
     제목이 오른쪽 머릿글\tidx{header,머릿글}에 다시 볼드체로 왼쪽 정렬되어
     있다.
     바닥글\tidx{footer,바닥글}은 없다.
\item[\pstyle{book}] 이는 \pstyle{plain} 페이지 양식과 같다.
\item[\pstyle{chapter}]  이는 \pstyle{plain} 페이지 양식과 같다.
\item[\pstyle{cleared}]  이는 \pstyle{empty} 페이지 양식과 같다.
\item[\pstyle{part}] 이는 \pstyle{plain} 페이지 양식과 같다.
\item[\pstyle{title}] 이는 \pstyle{plain} 페이지 양식과 같다.
\item[\pstyle{titlingpage}] 이는 \pstyle{empty} 페이지 양식과 같다.
\end{plainlist}

\begin{table}
\centering
\caption{\protect\cs{thispagestyle}의 사용}\label{tab:callthispagestyle}
\begin{tabular}{l !{\qquad} l} \toprule
사용된 곳 & 양식 \\ \midrule
\cmd{\book}    & \pstyle{book} \\
\cmd{\chapter} & \pstyle{chapter} \\
\cmd{\cleardoublepage} & \pstyle{cleared} \\
\cmd{\cleartorecto} & \pstyle{cleared} \\
\cmd{\cleartoverso} & \pstyle{cleared} \\
\cmd{\epigraphhead} & \pstyle{epigraph} \\
\cmd{\listoffigures} & \pstyle{chapter} \\
\cmd{\listoftables} & \pstyle{chapter} \\
\cmd{\maketitle} & \pstyle{title} \\
\cmd{\part}      & \pstyle{part} \\
\cmd{\tableofcontents} & \pstyle{chapter} \\
\Ie{thebibliography} & \pstyle{chapter} \\
\Ie{theindex} & \pstyle{chapter} \\
\Ie{titlingpage} & \pstyle{titlingpage} \\
\bottomrule
\end{tabular}
\end{table}

\begin{syntax}
\cmd{\uppercaseheads} \cmd{\nouppercaseheads}  \\
\end{syntax}
\glossary(uppercaseheads)%
  {\cs{uppercaseheads}}%
  {Set the titles in the headings pagestyle in Uppercase.}
\glossary(nouppercaseheads)%
  {\cs{nouppercaseheads}}%
  {Do not uppercase the titles in the headings.}
%     Following the declaration \cmd{\nouppercaseheads} the titles in the
% \pstyle{headings} pagestyle will not be automatically uppercased. The default
% is \cmd{\uppercaseheads} which specifies that the titles are to be
% automatically uppercased.
\pstyle{headings}에서 \cmd{\nouppercaseheads} 정의에 뒤따르는 제목은 자동으로
대문자로 바뀌지 않을 것다.
기본은 \cmd{\uppercaseheads}로, 제목은 자동으로 대문자 처리되도록 지정하는
것이다.

% \textbf{Change 2012:} The upper casing macro used by \cmd{\uppercaseheads} has
% been changed into \cmd{\MakeTextUppercase} such that the upper casing
% does not touch math, references or citations.
\textbf{2012 수정:} \cmd{uppercaseheads}에 사용되는 대문자화 매크로는
\cmd{MakeTextUppercase}로 변경되어 수식, 참조나 인용을 건드리지 않게 되었다.

%     For the \pstyle{myheadings} pagestyle above, you have to define your own
% titles to go into the header\index{header}. Each sectioning command,
% say \cs{sec},
% calls a macro called \cs{secmark}. A pagestyle usually defines this command
% so that it picks up the title, and perhaps the number, of the \cs{sec}. The
% pagestyle can then use the information for its own purposes.
위의 \pstyle{myheadings}을 위해서, 여러분은 머릿글\tidx{haeder,머릿글}에 들어갈
제목을 직접 정의해야 한다.
\cs{sec}과 같은 장절 명령은 각각 \cs{secmark}라는 매크로를 부른다.
페이지 양식은 보통 이 명령어를 정의해 \cs{sec}의 제목이나 번호를 추출한다.
페이지 양식은 이후 이 정보를 용도에 맞게 사용할 수 있다.

\begin{syntax}
\cmd{\markboth}\marg{left}\marg{right} \\
\cmd{\markright}\marg{right} \\
\end{syntax}
\glossary(markboth)%
  {\cs{markboth}\marg{left}\marg{right}}%
  {Sets values of two markers to \meta{left} and \meta{right} respectively
   (see \cs{leftmark} and \cs{rightmark}).}
\glossary(markright)%
  {\cs{markright}\marg{right}}%
  {Sets value of one marker to \meta{right} (see \cs{rightmark}).}
%     \cmd{\markboth} sets the values of two \emph{markers}\index{markers}
% to \meta{left} and \meta{right} respectively, at the point in the text
% where it is called. Similarly, \cmd{\markright} sets the value of a
% marker to \meta{right}.
\cmd{\markboth}는 이 명령이 불리운 자리에서 두 \emph{표지}\tidx{markers,표지}의
값을 \meta{left}와 \meta{right}에 설정한다.

\begin{syntax}
\cmd{\leftmark} \cmd{\rightmark} \\
\end{syntax}
\glossary(leftmark)%
  {\cs{leftmark}}%
  {Contains the value of the \meta{left} argument of the last \cs{markboth}.}
\glossary(rightmark)%
  {\cs{rightmark}}%
  {Contains the value of the \meta{right} argument of the first \cs{markboth}
   or \cs{markright} on the page; if there is none then the value of the most
   recent \meta{right} argument.}
% The macro \cmd{\leftmark} contains the value of the \meta{left} argument
% of the \emph{last} \cmd{\markboth} on the page. The macro \cmd{\rightmark}
% contains the value of the \meta{right} argument of the \emph{first}
% \cmd{\markboth} or \cmd{\markright} on the page, or if there is not one it
% contains the value of the most recent \meta{right} argument.
매크로 \cmd{\leftmark}는 쪽에서 \emph{마지막} \cmd{\markboth} 명령의
\meta{left} 인자의 값을 담는다.
매크로 \cmd{\rightmark}는 페이지의 \emph{처음} \cmd{\markboth}나
\cmd{\markright}의 \meta{right} 인자의 값을 담거나, 해당 명령이 없다면 가장
최근 것의 \meta{right} 인자의 값을 담는다.

%     A pagestyle can define the \cs{secmark} commands in terms of
% \cmd{\markboth} or \cmd{\markright}, and then use \cmd{\leftmark} and/or
% \cmd{\rightmark} in the headers\index{header} or footers\index{footer}.
% I'll show examples of how this
% works later, and this is often how the \pstyle{myheadings} style gets
% implemented.
페이지 양식은 \cmd{\markboth}와 \cmd{\markright}으로 \cmd{secmark} 명령어를
정의할 수 있고, \cmd{\leftmark}와(혹은) \cmd{\rightmark}를
머릿글\tidx{header,머릿글}이나 바닥글\tidx{footer,바닥글}에서 쓴다.
이것이 어떻게 되는 것인지 나중에 예시를 보일 것이며, 이는 \pstyle{myheadings}
양식이 보통 구현되는 방식이다.

%     All the division commands include a macro that you can define to set
% marks related to that heading. Other commands also include macros that
% you can redefine for setting marks.
모든 장절 구획 명령은 해당 머릿글과 관련해 여러분이 정의할 수 있는 매크로를
포함하고 있다.
다른 명령어들도 여러분이 재정의할 수 있는 표지 설정을 포함한다.

\begin{table}
\centering
% \caption{Mark macros for page headers} \label{tab:markmacros}
\caption{페이지 머릿글에 대한 표지 매크로} \label{tab:markmacros}
\begin{tabular}{ll} \toprule
% Main macro & default mark definition \\ \midrule
주 매크로 & 기본 표지 정의 \\ \midrule
\cs{book(*)}            & \verb?\newcommand*{\bookpagemark}[1]{}? \\
\cs{part(*)}            & \verb?\newcommand*{\partmark}[1]{}? \\
\cs{chapter(*)}         & \verb?\newcommand*{\chaptermark}[1]{}? \\
\cs{section(*)}         & \verb?\newcommand*{\sectionmark}[1]{}? \\
\cs{subsection(*)}      & \verb?\newcommand*{\subsectionmark}[1]{}? \\
\cs{subsubsection(*)}   & \verb?\newcommand*{\subsubsectionmark}[1]{}? \\
\cs{paragraph(*)}       & \verb?\newcommand*{\paragraphmark}[1]{}? \\
\cs{subparagraph(*)}    & \verb?\newcommand*{\subparagraphmark}[1]{}? \\
\cs{tableofcontents(*)} & \verb?\newcommand*{\tocmark}[1]{}? \\
\cs{listoffigures(*)}   & \verb?\newcommand*{\lofmark}[1]{}? \\
\cs{listoftables(*)}    & \verb?\newcommand*{\lotmark}[1]{}? \\
\cs{thebibliography}    & \verb?\newcommand*{\bibmark}{}? \\
\cs{theindex}           & \verb?\newcommand*{\indexmark}{}? \\
\cs{theglossary}        & \verb?\newcommand*{\glossarymark}{}? \\
\cs{PoemTitle}          & \verb?\newcommand*{\poemtitlemark}[1]{}? \\
\cs{PoemTitle*}         & \verb?\newcommand*{\poemtitlestarmark}[1]{}? \\
\bottomrule
\end{tabular}
\end{table}

% The \cs{...mark} commands are listed in \tref{tab:markmacros}. When they are
% called by the relevant main macro, those that take an argument are called with
% the `title' as the argument's value. For example, the \cmd{\chapter} macro
% calls \cmd{\chaptermark} with the value of the title specified as being
% for the header.
\cs{...mark} 명령어들이 \tref{tab:markmacros}에 나열되어 있다. 이들이 관련된 주
매크로에 의해 불리우면, `title'을 인자의 값으로 취한다.
예를 들어, \cmd{\chapter} 매크로는 \cmd{\chaptermark}이 머릿글이 되도록 제목의
값을 지정하여 부른다.

% Please remember that the macros listed in \tref{tab:markmacros} are
% `provider' macros, i.e. they provide information for \cmd{\leftmark}
% and \cmd{\rightmark} for you to use later on. To gain access to the
% section title, you do \emph{not} use \cmd{\sectionmark} in the header
% or footer. It is a macro that provides information, but you need to
% use \cmd{\leftmark} or \cmd{\rightmark} to access depending on how you
% have defined \cmd{\sectionmark}.
부디 \tref{tab:markmacros}에 나열된 매크로들이 `제공자' 매크임을 잊지 말아달라.
즉, 이들은 여러분이 \cmd{\leftmark}와 \cmd{\rightmark}의 정보를 나중에 쓸 수
있도록 제공해 준다.
절 제목에 접근하기 위해서 \cmd{\sectionmark}를 머릿글이나 바닥글에 사용하면
\emph{안 된다}.
매크로가 정보를 제공해주지만, 여러분은 \cmd{\sectionmark}를 어떻게 정의했는지에
따라서 \cmd{\leftmark} 혹은 \cmd{\rightmark}를 사용해 접근해야 한다.


% \section{Making headers and footers}
\section{머릿글과 바닥글 만들기}

    As mentioned, the class provides for left, center, and right slots in
even and odd headers\index{header} and footers\index{footer}. 
This section describes how you can make 
your own pagestyle using these 12 slots. The 6 slots for a page 
are diagrammed in \fref{lay:header}.
언급하였듯이, 이 클래스는 짝수와 홀수 머릿글\tidx{header,머릿글}과
바닥글\tidx{footer,바닥글}의 왼쪽, 가운데, 그리고 오른쪽 영역을 제공한다.
본 절은 여러분이 이 12개의 영역을 사용해 직접 페이지 양식을 만드는 방법을
설명한다.
페이지의 6개 영역은 \fref{lay:header}에 도표로 나타나 있다.

\begin{figure}
\setlayoutscale{1}
\centering
\headerfooterdiagram
% \caption{Header and footer slots} \label{lay:header}
\caption{머릿글과 바닥글 영역}\label{lay:header}
\end{figure}

%     The class itself uses the commands from this section. For example,
% the \pstyle{plain} pagestyle is defined as
이 클래스 자체도 이 절의 명령들을 사용한다.
예를 들어, \pstyle{plain} 페이지 양식은 다음과 같이 정의되어 있다.
\begin{lcode}
\makepagestyle{plain}
  \makeevenfoot{plain}{}{\thepage}{}
  \makeoddfoot{plain}{}{\thepage}{}
\end{lcode}
% which centers the page number at the bottom of the page.
이는 페이지 번호를 페이지 하단에 중앙 정렬한다.


\begin{syntax}
\cmd{\makepagestyle}\marg{style} \\
\cmd{\aliaspagestyle}\marg{alias}\marg{original} \\
\cmd{\copypagestyle}\marg{copy}\marg{original}\\
\end{syntax}
\glossary(makepagestyle)%
  {\cs{makepagestyle}\marg{style}}%
  {Used to define a pagestyle \meta{style}.}
\glossary(aliaspagestyle)%
  {\cs{aliaspagestyle}\marg{alias}{original}}%
  {Defines the \meta{alias} pagestyle to be the same as the \meta{original}
  pagestyle.}
\glossary(copypagestyle)%
  {\cs{copypagestyle}\marg{copy}{original}}%
  {Creates a new pagestyle called \meta{copy} using the \meta{original}
   pagestyle specification.}

% The command \cmd{\makepagestyle} specifies a pagestyle \meta{style} which
% is initially equivalent to the \pstyle{empty} pagestyle. On the other hand,
% \cmd{\aliaspagestyle} defines the \meta{alias} pagestyle to be the same as
% the \meta{original} pagestyle. As an example of the latter, the class includes
% the code
\cmd{\makepagestyle} 명령은 \meta{style} 페이지 양식을 지정하는데, 이는 초기에
\pstyle{empty} 페이지 양식과 동일하다.
반면, \cmd{\aliaspagestyle}은 \meta{alias} 페이지 양식을 \meta{original} 페이지
양식과 같도록 정의한다.
후자의 예시로 본 클래스가 다음 코드를 포함한다.
\begin{lcode}
\aliaspagestyle{part}{plain}
\aliaspagestyle{chapter}{plain}
\aliaspagestyle{cleared}{empty}
\end{lcode}
% The \cmd{\copypagestyle} command creates a new pagestyle called \meta{copy}
% using the \meta{original} pagestyle specification.
\cmd{\copypagestyle} 명령은 \meta{original} 페이지 양식 설정으로 \meta{copy}라는
새로운 페이지 양식을 생성한다.

%     If an alias and a copy pagestyle are created based on the same
% \meta{original} and later the \meta{original} is modified,
% the alias and copy behave differently.
% The appearance of the alias pagestyle will continue to match the
% modified \meta{original} but the copy pagestyle is unaffected by any change
% to the \meta{original}. You cannot modify an alias pagestyle but you can
% modify a copy pagestyle.
만약 참조(alias)와 사본(copy) 페이지 양식이 \meta{original}을 토대로 생성된 후,
\meta{original}이 수정되었다면, 참조와 사본은 다르게 행동한다.
참조 페이지 양식은 수정된 \meta{original}과 여전히 일치하지만 사본 페이지
양식은 \meta{original}의 어떠한 변경에도 영향을 받지 않는다.
여러분은 참조 페이지 양식을 수정할 수 없지만 사본 페이지 양식은 수정할 수 있다.

\begin{syntax}
\cmd{\makeevenhead}\marg{style}\marg{left}\marg{center}\marg{right} \\
\cmd{\makeoddhead}\marg{style}\marg{left}\marg{center}\marg{right} \\
\cmd{\makeevenfoot}\marg{style}\marg{left}\marg{center}\marg{right} \\
\cmd{\makeoddfoot}\marg{style}\marg{left}\marg{center}\marg{right} \\
\end{syntax}
\glossary(makeevenhead)%
  {\cs{makeevenhead}\marg{style}\marg{left}\marg{center}\marg{right}}%
  {Defines the \meta{left}, \meta{center} and \meta{right} parts of the
   even (verso) page header of the \meta{style} pagetstyle.}
\glossary(makeoddhead)%
  {\cs{makeoddhead}\marg{style}\marg{left}\marg{center}\marg{right}}%
  {Defines the \meta{left}, \meta{center} and \meta{right} parts of the
   odd (recto) page header of the \meta{style} pagetstyle.}
\glossary(makeevenfoot)%
  {\cs{makeevenfoot}\marg{style}\marg{left}\marg{center}\marg{right}}%
  {Defines the \meta{left}, \meta{center} and \meta{right} parts of the
   even (verso) page footer of the \meta{style} pagetstyle.}
\glossary(makeoddfoot)%
  {\cs{makeoddfoot}\marg{style}\marg{left}\marg{center}\marg{right}}%
  {Defines the \meta{left}, \meta{center} and \meta{right} parts of the
   odd (recto) page footer of the \meta{style} pagetstyle.}
The macro \cmd{\makeevenhead} defines the \meta{left}, \meta{center}, and
\meta{right} portions of the \meta{style} pagestyle header\index{header} 
for even numbered (verso) pages. 
Similarly \cmd{\makeoddhead}, \cmd{\makeevenfoot}, and
\cmd{\makeoddfoot} define the \meta{left}, \meta{center} and \meta{right}
portions of the \meta{style} header\index{header} for odd numbered 
(recto) pages, and the footers\index{footer} for verso and recto pages. 
These commands for \meta{style}
should be used after the corresponding \cmd{\makepagestyle} for \meta{style}.
매크로 \cmd{\makeevenhead}는 짝수 페이지에 대해 \meta{left}, \met{center},
그리고 \met{right}

\begin{syntax}
\cmd{\makerunningwidth}\marg{style}\oarg{footwidth}\marg{headwidth} \\
\lnc{\headwidth} \\
\end{syntax}
\glossary(makerunningwidth)%
  {\cs{makerunningwidth}\marg{style}\oarg{footwidth}\marg{headwidth}}%
  {Sets the width of the \meta{style} pagestyle headers to \meta{headwidth}.
   The footers are set to \meta{headwidth}, or \meta{footwidth} if it
   is given.}
\glossary(headwidth)%
  {\cs{headwith}}%
  {A (scratch) length normally used in the definition of headers and footers.}
The macro \cmd{\makerunningwidth} sets the widths of the \meta{style}
pagestyle headers\index{header} and footers\index{footer}. The header
width is set to \meta{headwidth}. If the optional \meta{footwidth} is
present, then the footer width is set to that, otherwise to \meta{headwidth}.
The header width is stored as the length \cs{\meta{style}headrunwidth}
and the footer width as \cs{\meta{style}footrunwidth}.

The \cmd{\makepagestyle} initialises the widths to be the textwidth, 
so the macro need only be used if some
other width is desired. The length \lnc{\headwidth} is provided as a
(scratch) length that may be used for headers\index{header} or 
footers\index{footer}, or any other purpose.

\begin{syntax}
\cmd{\makeheadrule}\marg{style}\marg{width}\marg{thickness} \\
\cmd{\makefootrule}\marg{style}\marg{width}\marg{thickness}\marg{skip} \\
\cmd{\makeheadfootruleprefix}\marg{style}\marg{for headrule}\marg{for footrule}\\
\end{syntax}
\glossary(makeheadrule)%
  {\cs{makeheadrule}\marg{style}\marg{width}\marg{thickness}}%
  {Specifies the \meta{width} and \meta{thickness} of the rule drawn below the
   headers of the \meta{style} pagestyle.}%
\glossary(makefootrule)%
  {\cs{makefootrule}\marg{style}\marg{width}\marg{thickness}\marg{skip}}%
  {Specifies the \meta{width} and \meta{thickness} of the rule drawn 
  \meta{skip} (see \cs{footskip}) above the footers of the
  \meta{style} pagestyle.}% 
\glossary(makeheadfootruleprefix)
{\cs{makeheadfootruleprefix}\marg{style}\marg{for headrule}\marg{for
    footrule}}%
{Can be used to add alternative colors to the head/foot rule}%
A header\index{header} may have a rule drawn between it and the top of 
the typeblock\index{typeblock}, and similarly a rule may be drawn 
between the bottom of the typeblock\index{typeblock} and the 
footer\index{footer}. 
The \cmd{\makeheadrule} macro specifies the \meta{width}
and \meta{thickness} of the rule below the \meta{style} pagestyle 
header\index{header}, and the \cmd{\makefootrule} does the same for 
the rule above the footer\index{footer}; the
additional \meta{skip} argument is a distance that specifies the vertical
positioning of the foot rule (see \cmd{\footruleskip}).
The \cmd{\makepagestyle} macro initialises the \meta{width} to the 
\lnc{\textwidth} and the \meta{thickness} to 0pt, so by default no rules
are visible. The macro \cmd{\makeheadfootruleprefix} is intended for
adding alternative colors to the head/foot rules, e.g.
\begin{lcode}
  \makeheadfootruleprefix{mystyle}{\color{red}}{\color{blue}}
\end{lcode}


\begin{syntax}
\lnc{\normalrulethickness} \\
\end{syntax}
\glossary(normalrulethickness)%
  {\cs{normalrulethickness}}%
  {The normal thickness of a visible rule (default 0.4pt).}
\lnc{\normalrulethickness} is the normal\index{rule!thickness} 
thickness of a visible rule, by 
default 0.4pt. It can be changed using \cmd{\setlength}, although I suggest 
that you do not unless perhaps when using at least the \Lopt{14pt} class 
option. 

\begin{syntax}
\cmd{\footruleheight} \\
\cmd{\footruleskip} \\
\end{syntax}
\glossary(footruleheight)%
  {\cs{footruleheight}}%
  {Macro specifying the height of a normal rule above a footer.}
\glossary(footruleskip)%
  {\cs{footruleskip}}%
  {Macro specifying a distance sufficient to ensure that a rule above a footer
   will lie in the space between the footer and the typeblock.}
The macro \cmd{\footruleheight} is the height of a normal
rule above a footer\index{footer} (default zero). 
\cmd{\footruleskip} is a distance 
sufficient to ensure that a foot rule will be placed between the bottom
of the typeblock\index{typeblock} and the footer\index{footer}. 
Despite appearing to be lengths, if you really need to change the values 
use \cmd{\renewcommand}, not \cmd{\setlength}.

\begin{syntax}
\cmd{\makeheadposition}\marg{style}\\
    \marg{eheadpos}\marg{oheadpos}\marg{efootpos}\marg{ofootpos} \\
\end{syntax}
\glossary(makeheadposition)%
  {\cs{makeheadposition}\marg{style}\marg{eheadpos}\marg{oheadpos}\marg{efootpos}\marg{ofootpos}}%
  {Specifies the horizontal positioning of the even and odd headers and
   footers respectively for the \meta{style} pagestyle.}
The \cmd{\makeheadposition} macro specifies the horizontal positioning
of the even and odd headers\index{header} and footers\index{footer}, 
respectively, for the \meta{style} pagestyle. 
Each of the \meta{...pos} arguments may be \texttt{flushleft}, \texttt{center},
or \texttt{flushright}, with the obvious meanings. An empty, or unrecognised, 
argument is equivalent to \texttt{center}. This macro is really only of use 
if the header/footer\index{header}\index{footer} width is not the 
same as the \lnc{\textwidth}.

\begin{syntax}
\cmd{\makepsmarks}\marg{style}\marg{code} \\
\end{syntax}
\glossary(makepsmark)
  {\cs{makepsmarks}\marg{style}\marg{code}}%
  {Hook into the \meta{style} pagestyle, usually used for the \meta{code}
   setting any marks.}
The last thing that the \cmd{\pagestyle}\marg{style} does is call the
\meta{code} argument of the \cmd{\makepsmarks} macro for \meta{style}.
This is normally used for specifying non-default code 
(i.e., code not specifiable via any of the previous macros) for the 
particular pagestyle. The code normally defines the marks, if any, 
that will be used in
the headers\index{header} and footers\index{footer}.

\LMnote{2010/06/25}{Added a mentioning of \cs{makeheadfootstrut}}
\begin{syntax}
  \cmd{\makeheadfootstrut}\marg{style}\marg{head strut}\marg{foot strut}
\end{syntax}
The headers and footers are each made up of three separate
entities. At the front and end of these a special \meta{style} related
strut is inserted. By default \cmd{\makepagestyle} will initialize
them to \cmd{\strut} (except the \pstyle{empty} style where the struts
are empty). One can use the macro above to change these struts to
something different.


\subsection{Example pagestyles}

    Perhaps when preparing drafts you want to note on each page
that it is a draft\index{draft document} document. Assuming that 
you are using the 
\pstyle{headings} page style and that the default \pstyle{plain}
page style is used on chapter openings, then you could define
the following in the preamble (\piif{ifdraftdoc} is provided by
the class and is set \ptrue\ when the \Lopt{draft} option is used).
\label{ex:draft.pagestyle}
\begin{lcode}
\ifdraftdoc
  \makeevenfoot{plain}{}{\thepage}{\textit{Draft: \today}}
  \makeoddfoot{plain}{\textit{Draft: \today}}{\thepage}{}
  \makeevenfoot{headings}{}{}{\textit{Draft: \today}}
  \makeoddfoot{headings}{\textit{Draft: \today}}{}{}
\fi
\end{lcode}
Now when the \Lopt{draft} option is used the word `Draft:' and the current
date will be typeset in italics at the bottom of each page by the spine
margin. If any \pstyle{empty} pages should be marked as well, specify
similar footers for that style as well.

    Here is part of the standard definition of the \pstyle{headings}
pagestyle for the \Lclass{book} class which uses many internal \ltx\ commands;
but note that \Mname\ does not use this.
\begin{lcode}
\def\ps@headings{%
  \let\@oddfoot\@empty\let\@evenfoot\@empty
  \def\@evenhead{\thepage\hfil\slshape\leftmark}%
  \def\@oddhead{{\slshape\rightmark}\hfil\thepage}%
  \def\chaptermark##1{%
    \markboth{\MakeUppercase{%
      \ifnum\c@secnumdepth > \m@ne
        \if@mainmatter
          \@chapapp\ \thechapter. \ %
        \fi
      \fi
      ##1}}{}}%
  \def\sectionmark##1{%
    \markright{\MakeUppercase{%
      \ifnum\c@secnumdepth > \z@
        \thesection. \ %
      \fi
      ##1}}}}
\end{lcode}
You don't need to understand this but in outline the first three lines specify
the contents of the footers and headers, and the remainder of the code sets 
the marks that will be used in the headers. The \cmd{\leftmark} is specified to be
the word `chapter', 
followed by the number if it is in the \cmd{\mainmatter} and the \texttt{secnumdepth}
is such that chapters are numbered, followed by the chapter's title; all this 
is made to be in upper case (via the \cmd{\MakeUppercase} macro). Similarly
the other mark, \cmd{\rightmark}, is the section number, if there is one, and
the section's title, again all in upper case.

    A transliteration of this code into \Mname's original coding style is:
\begin{lcode}
\makepagestyle{headings}
\makeevenhead{headings}{\thepage}{}{\slshape\leftmark}
\makeoddhead{headings}{\slshape\rightmark}{}{\thepage}
\makepsmarks{headings}{%
  \def\chaptermark##1{%
    \markboth{\MakeUppercase{%
      \ifnum\c@secnumdepth > \m@ne
        \if@mainmatter
          \@chapapp\ \thechapter. \ %
        \fi
      \fi
      ##1}}{}}%
  \def\sectionmark##1{%
    \markright{\MakeUppercase{%
      \ifnum\c@secnumdepth > \z@
        \thesection. \ %
      \fi
      ##1}}}
  \def\tocmark{\markboth{\MakeUppercase{\contentsname}}{}}
  \def\lofmark{\markboth{\MakeUppercase{\listfigurename}}{}}
  \def\lotmark{\markboth{\MakeUppercase{\listtablename}}{}}
  \def\bibmark{\markboth{\MakeUppercase{\bibname}}{}}
  \def\indexmark{\markboth{\MakeUppercase{\indexname}}{}}
  \def\glossarymark{\markboth{\MakeUppercase{\glossaryname}}{}}}
\end{lcode}
As you can see, defining the marks for a pagestyle is not necessarily the
simplest thing in the world. However, courtesy of Lars\index{Madsen, Lars} Madsen,
help is at hand.

\begin{syntax}
\cmd{\createplainmark}\marg{type}\marg{marks}\marg{text} \\
\cmd{\memUChead}\marg{text} \\
\cmd{\uppercaseheads} \cmd{\nouppercaseheads} \\
\cmd{\createmark}\marg{sec}\marg{marks}\marg{show}\marg{prefix}\marg{postfix} \\
\end{syntax}
\glossary(createplainmark)%
  {\cs{createplainmark}\marg{type}\marg{marks}\marg{text}}%
  {Defines the \cs{typemark} macro using \meta{text} as the mark, where
  \meta{marks} is \texttt{left}, \texttt{both} or \texttt{right}.}
\glossary(createmark)%
  {\cs{createmark}\marg{sec}\marg{marks}\marg{show}\marg{prefix}\marg{postfix}}%
  {Defines the \cs{secmark} macro where \meta{show} (\texttt{shownumber} 
   or \texttt{nonumber}) controls whether the division number will be
   displayed within \cs{mainmatter}, \meta{marks} is \texttt{left}, 
   \texttt{both} or \texttt{right}, and \meta{prefix} and \meta{postfix}
   are affixed before and after the \meta{sec} (division) number.}
\glossary(memUChead)%
  {\cs{memUChead}\marg{text}}%
  {May uppercase \meta{text}, depending on \cs{uppercaseheads} and
  \cs{nouppercaseheads}.}
\glossary(uppercaseheads)%
  {\cs{uppercaseheads}}%
  {Defines \cs{memUChead} as equivalent to \cs{MakeUppercase}.}
\glossary(nouppercaseheads)%
  {\cs{nouppercaseheads}}%
  {Defines \cs{memUChead} as \cs{relax} (i.e., do nothing).}


The macro \cmd{\createplainmark} defines the \verb?\<type>mark?, where 
\meta{type} is an unnumbered division-like head, such as \texttt{toc},
\texttt{lof}, \texttt{index}, using \meta{text} as the mark value, and 
\meta{marks} is \texttt{left}, \texttt{both} or \texttt{right}. For example:
\begin{lcode}
\createplainmark{toc}{left}{\contentsname}
\createplainmark{lot}{right}{\listtablename}
\createplainmark{bib}{both}{\bibname}
\end{lcode}
is equivalent to
\begin{lcode}
\def\tocmark{\markboth{\memUChead{\contentsname}}{}}
\def\lotmark{\markright{\memUChead{\listtablename}}}
\def\lofmark{\markboth{\memUChead{\bibname}}{\memUChead{\bibname}}}
\end{lcode}

    Following the declaration \cmd{\uppercaseheads} the \cmd{\memUChead} 
command is equivalent to \cmd{\MakeUppercase} but after the 
\cmd{\nouppercaseheads} it is equivalent to \cmd{\relax} (which does nothing).
The \cmd{\createplainmark} macro wraps \cmd{\memUChead} around the \meta{text}
argument within the generated \cs{mark(both/right)} macro. By using the
\cs{(no)uppercaseheads} declarations you can control the uppercasing, or
otherwise, of the mark texts. The default is \cmd{\uppercaseheads}.

\LMnote{2010/02/08}{added the following paragraph}
Note that if you want to use a predefined page style, but would like
to not use automatic uppercasing, then issue \cs{nouppercaseheads} and
reload the page style, for example with the default page style in \theclass\
\begin{lcode}
  \nouppercaseheads
  \pagestyle{headings}
\end{lcode}


    The macro \cmd{\createmark}\marg{sec}\marg{marks}\marg{show}\marg{prefix}\marg{postfix}
defines the \verb?\<sec>mark? macro where \meta{sec} is a sectional division
such as \texttt{part}, \texttt{chapter}, \texttt{section}, etc., 
and \meta{show} (\texttt{shownumber} 
or \texttt{nonumber}) controls whether the division number will be
displayed within \cs{mainmatter}. The \meta{marks} argument is \texttt{left}, 
\texttt{both} or \texttt{right}, and \meta{prefix} and \meta{postfix}
are affixed before and after the division number. For example:
\begin{lcode}
\createmark{section}{left}{nonumber}{}{}
\createmark{section}{both}{nonumber}{}{}
\createmark{section}{right}{nonumber}{}{}
\end{lcode}
is equivalent to, respectively
\begin{lcode}
\def\sectionmark#1{\markboth{#1}{}}
\def\sectionmark#1{\markboth{#1}{#1}}
\def\sectionmark#1{\markight{#1}}
\end{lcode}

The difference between \cmd{\createmark} and \cmd{\createplainmark} is
that the former create a macro that takes an argument, whereas
\cmd{\createplainmark} does not.


    Using these macros \Mname's current definition of 
\verb?\makepsmarks{headings}? is much simpler (it also leads to a 
slightly different result as the \texttt{toc} etc., marks set both
the \cmd{\leftmark} and \cmd{\rightmark} instead of just the 
\cmd{\leftmark}):
\begin{lcode}
\makepsmarks{headings}{%
  \createmark{chapter}{left}{shownumber}{\@chapapp\ }{. \ }
  \createmark{section}{right}{shownumber}{}{. \ }
  \createplainmark{toc}{both}{\contentsname}
  \createplainmark{lof}{both}{\listfigurename}
  \createplainmark{lot}{both}{\listtablename}
  \createplainmark{bib}{both}{\bibname}
  \createplainmark{index}{both}{\indexname}
  \createplainmark{glossary}{both}{\glossaryname}}
\end{lcode}


\LMnote{2010/02/08}{fixed typo}
When \Mname{} runs the marks part of page style, it does not zero out
old marks, i.e.\ if an old \cmd{\sectionmark} exist, it still exist
even if we do not change it.  This is both a good and a bad thing. To help
users redefine these marks to doing nothing we provide
\begin{syntax}
\cmd{\clearplainmark}\marg{type}\\
\cmd{\clearmark}\marg{type}\\  
\end{syntax}
The used types are the same as for \cmd{\createplainmark} and \cmd{\createmark}.



\PWnote{2009/07/30}{Added sections on Document title and Part title in headers}
\subsubsection{Header with the document title}

    As mentioned before, some publishers like the title of the book 
to be in the header. A simple header is probably all that is needed 
as it is unlikely to be a technical publication. Here is a
use for \pstyle{myheadings}.
\begin{lcode}
\makevenhead{myheadings}{\thepage}{}{DOCUMENT TITLE}
\makeoddhead{myheadings}{Chapter~\thechapter}{}{\thepage}
\end{lcode}

\subsubsection{Part and chapter in the header}

    Some documents have both part and chapter divisions and in such
cases it may be useful for the reader to have the current part and chapter
titles in the header. The \pstyle{headings} pagestyle can be easily modified
to accomplish this by simply resetting the marks for part and chapter:
\begin{lcode}
\makepsmarks{headings}{%
  \createmark{part}{left}{shownumber}{\partname\ }{. \ }
  \createmark{chapter}{right}{shownumber}{\@chapapp\ }{. \ }
  \createplainmark{toc}{both}{\contentsname}
  \createplainmark{lof}{both}{\listfigurename}
  \createplainmark{lot}{both}{\listtablename}
  \createplainmark{bib}{both}{\bibname}
  \createplainmark{index}{both}{\indexname}
  \createplainmark{glossary}{both}{\glossaryname}}
\end{lcode}
  

\subsubsection{The Companion pagestyle}

    This example demonstrates most of the page styling commands.
In the \textit{\ltx\ Companion} series of 
books~\cite{COMPANION,GCOMPANION,WCOMPANION} the header\index{header} is wider 
than the typeblock\index{typeblock}, sticking out into the outer 
margin\index{margin!outer}, and has a rule underneath it. 
The page number is in 
bold and at the outer end of the header\index{header}.
Chapter titles are in verso headers\index{header} and section titles 
in recto headers\index{header}, both in bold font and at the inner 
margin\index{margin!inner}. The footers\index{footer} are empty.

    The first thing to do in implementing this style is to calculate 
the width of the headers\index{header}, which extend to cover any 
marginal\index{marginalia} notes.
\begin{lcode}
\setlength{\headwidth}{\textwidth}
  \addtolength{\headwidth}{\marginparsep}
  \addtolength{\headwidth}{\marginparwidth}
\end{lcode}
Now we can set up an empty \pstyle{companion} pagestyle and start to change
it by specifying the new header\index{header} and footer\index{footer} width:
\begin{lcode}
\makepagestyle{companion}
\makerunningwidth{companion}{\headwidth}
\end{lcode}
and specify the width and thickness for the header\index{header} rule, 
otherwise it will be invisible.
\begin{lcode}
\makeheadrule{companion}{\headwidth}{\normalrulethickness}
\end{lcode}

    In order to get the header\index{header} to stick out into the \foredge\ 
margin\index{margin!foredge?\foredge}, verso headers\index{header} 
have to be flushright 
(raggedleft) and recto headers\index{header} to be flushleft (raggedright). 
As the footers\index{footer} are empty, their position is immaterial.
\begin{lcode}
\makeheadposition{companion}{flushright}{flushleft}{}{}
\end{lcode}

    The current chapter and section titles are obtained from the 
\cmd{\leftmark} and \cmd{\rightmark} macros which are defined via the
\cmd{\chaptermark} and \cmd{\sectionmark} macros. Remember that
\cmd{\leftmark} is the last \meta{left} marker and \cmd{\rightmark}
is the first \meta{right} marker\index{markers} on the page.

Chapter numbers are not
put into the header\index{header} but the section number, 
if there is one, is put into
the header\index{header}. We have to make sure that
the correct definitions are used for these as well as for the 
\toc\footnote{The \toc\ and friends are described in detail 
in \protect\Cref{chap:toc}.} 
and other similar elements, and this is where the 
\cmd{\makepsmarks} macro comes into play. 
\begin{lcode}
\makepsmarks{companion}{%
  \nouppercaseheads
  \createmark{chapter}{both}{nonumber}{}{}
  \createmark{section}{right}{shownumber}{}{. \space}
  \createplainmark{toc}{both}{\contentsname}
  \createplainmark{lof}{both}{\listfigurename}
  \createplainmark{lot}{both}{\listtablename}
  \createplainmark{bib}{both}{\bibname}
  \createplainmark{index}{both}{\indexname}
  \createplainmark{glossary}{both}{\glossaryname}
\end{lcode}

    The preliminaries have all been completed, and it just remains to specify
what goes into each header\index{header} and footer\index{footer} slot 
(but the footers\index{footer} are empty).
\begin{lcode}
\makeevenhead{companion}%
  {\normalfont\bfseries\thepage}{}{%
   \normalfont\bfseries\leftmark}
\makeoddhead{companion}%
  {\normalfont\bfseries\rightmark}{}{%
   \normalfont\bfseries\thepage}
\end{lcode}

    Now issuing the command \verb?\pagestyle{companion}? will produce pages 
typeset with \pstyle{companion} pagestyle headers\index{header}. This pagestyle
is part of the class.

\begin{syntax}
\cmd{\addtopsmarks}\marg{pagestyle}\marg{prepend}\marg{append} \\
\end{syntax}
\glossary(addtopsmarks)%
  {\cs{addtopsmarks}\marg{pagestyle}\marg{prepend}\marg{append}}%
  {Inserts \meta{prepend} and \meta{append} before and after the current
   definition of \cs{makepsmarks} for \meta{pagestyle}.}
\cmd{\addtopsmarks}\marg{pagestyle}\marg{prepend}\marg{append} is the last
of this group of helper macros. It inserts \meta{prepend} and \meta{append} 
before and after the current definition of \cs{makepsmarks} 
for \meta{pagestyle}. For instance, if you wanted \cs{subsection} titles to appear
in the page headers of the \pstyle{companion} pagestyle then this would be a way
of doing it:
\begin{lcode}
\addtopsmarks{companion}{}{%
  \createmark{subsection}{right}{shownumber}{}{. \space}}
\end{lcode}


\subsubsection{The ruled pagestyle}

    For practical reasons I prefer a page style with headings
where the chapter title is at least in the center 
of the page, and for technical works is at the \foredge. I also prefer the
page number to be near the outside edge. When picking up a book and skimming
through it, either to get an idea of what is in it or to find something more
specific, I hold it in one hand at the spine and use the other for flicking
the pages. The book is half closed while doing this and it's much easier
to spot things at the \foredge\ than those nearer the spine. 
The \pstyle{ruled} page style is like this. The general plan is defined as:
\begin{lcode}
\makepagestyle{ruled}
\makeevenfoot {ruled}{\thepage}{}{} % page numbers at the outside
\makeoddfoot  {ruled}{}{}{\thepage}
\makeheadrule {ruled}{\textwidth}{\normalrulethickness}
\makeevenhead {ruled}{\scshape\leftmark}{}{} % small caps
\makeoddhead  {ruled}{}{}{\rightmark}
\end{lcode}
The other part of the specification has to ensure that the \cmd{\chapter}
and \cmd{\section} commands make the appropriate marks for the headers.
I wanted the numbers to appear in the headers, but not those for sections. The following
code sets these up, as well as the marks for the other document elements.
\begin{lcode}
\makepsmarks{ruled}{%
  \nouppercaseheads
  \createmark{chapter}{left}{shownumber}{}{. \space}
  \createmark{section}{right}{nonumber}{}{}
  \createplainmark{toc}{both}{\contentsname}
  \createplainmark{lof}{both}{\listfigurename}
  \createplainmark{lot}{both}{\listtablename}
  \createplainmark{bib}{both}{\bibname}
  \createplainmark{index}{both}{\indexname}
  \createplainmark{glossary}{both}{\glossaryname}
}
\end{lcode}

\index{pagestyle!specifying|)}



\subsection{Index headers}

\index{pagestyle!index pages|(}

    If you look at the Index\index{index} you will see that the header\index{header} 
shows the first and last entries on the page.
A main entry in the index\index{index} looks like:
\begin{lcode}
\item \idxmark{entry}, page number(s)
\end{lcode}
and in the preamble\index{preamble} to this book \cmd{\idxmark} is defined as
\begin{lcode}
\newcommand{\idxmark}[1]{#1\markboth{#1}{#1}}
\end{lcode}
This typesets the entry and also uses the entry as markers so that
the first entry on a page is held in \cmd{\rightmark} and the last
is in \cmd{\leftmark}.

    As index\index{index} entries are usually very short, the 
Index\index{index} is set in two columns\index{column!double}. 
Unfortunately \ltx's marking mechanism can be very
fragile on twocolumn\index{column!double} pages.\footnote{This was
  fixed in the \LaTeX{} kernel, but including the functionality from
  the \Lpack{fixltx2e} package.}

    The index\index{index} itself is called by\index{indexing}
\begin{lcode}
\clearpage
\pagestyle{index}
\renewcommand{\preindexhook}{%
The first page number is usually, but not always, 
the primary reference to
the indexed topic.\vskip\onelineskip}
\printindex
\end{lcode}

\makepagestyle{index}
  \makeheadrule{index}{\textwidth}{\normalrulethickness}
  \makeevenhead{index}{\rightmark}{}{\leftmark}
  \makeoddhead{index}{\rightmark}{}{\leftmark}
  \makeevenfoot{index}{\thepage}{}{}
  \makeoddfoot{index}{}{}{\thepage}

    The \pstyle{index} pagestyle, which is the crux of
this example, is defined here as:
\begin{lcode}
\makepagestyle{index}
  \makeheadrule{index}{\textwidth}{\normalrulethickness}
  \makeevenhead{index}{\rightmark}{}{\leftmark}
  \makeoddhead{index}{\rightmark}{}{\leftmark}
  \makeevenfoot{index}{\thepage}{}{}
  \makeoddfoot{index}{}{}{\thepage}
\end{lcode}
This, as you can hopefully see, puts the first and last index\index{index} 
entries on the page into the header\index{header} at the left and right, 
with the folios\index{folio} in the footers\index{footer} at the 
outer margin\index{margin!outer}.

\index{pagestyle!index pages|)}

\subsection{Float pages}

\index{pagestyle!float pages|(}
\index{float!page|(}

\begin{syntax}
\piif{ifonlyfloats}\marg{yes}\marg{no} \\
\end{syntax}
\glossary(ifonlyfloats)%
  {\cs{ifonlyfloats}\marg{yes}\marg{no}}%
  {Processes \meta{yes} on a page containing only floats, otherwise process
  \meta{no}.}
    There are occasions when it is desirable to have different 
headers\index{header} on pages that only contain figures\index{figure} 
or tables\index{table}. If the command \piif{ifonlyfloats}
is issued on a page that contains no text and only floats then the \meta{yes}
argument is processed, otherwise on a normal page the \meta{no} argument
is processed. The command is most useful when defining a pagestyle that 
should be different on a float-only page\index{page!of floats}.

    For example, assume that the \pstyle{companion} pagestyle is to be
generally used, but on float-only pages all that is required is a pagestyle
similar to \pstyle{plain}. Borrowing some code from the \pstyle{companion}
specification this can be accomplished like:
\begin{lcode}
\makepagestyle{floatcomp}
% \headwidth has already been defined for the companion style
\makeheadrule{floatcomp}{\headwidth}%
  {\ifonlyfloats{0pt}{\normalrulethickness}}
\makeheadposition{floatcomp}{flushright}{flushleft}{}{}
\makepsmarks{floatcomp}{\companionpshook}
\makeevenhead{floatcomp}%
             {\ifonlyfloats{}{\normalfont\bfseries\thepage}}%
             {}%
             {\ifonlyfloats{}{\normalfont\bfseries\leftmark}}
\makeoddhead{floatcomp}%
             {\ifonlyfloats{}{\normalfont\bfseries\rightmark}}%
             {}%
             {\ifonlyfloats{}{\normalfont\bfseries\thepage}}
\makeevenfoot{floatcomp}{}{\ifonlyfloats{\thepage}{}}{}
\makeoddfoot{floatcomp}{}{\ifonlyfloats{\thepage}{}}{}
\end{lcode}
The code above for the \pstyle{floatcomp} style should be compared with 
that for the earlier \pstyle{companion} style.

    The headrule is invisible\index{rule!thickness}\index{rule!invisible} 
on float pages by giving it zero thickness, 
otherwise it has the \cmd{\normalrulethickness}. The head position is 
identical for both pagestyles. However, the headers\index{header} are empty for
\pstyle{floatcomp} and the footers\index{footer} have centered page numbers 
on float pages; on ordinary pages the footers\index{footer} are empty 
while the headers\index{header}
are the same as the \pstyle{companion} headers\index{header}.

    The code includes one `trick'. The macro \cmd{\makepsmarks}\verb?{X}{code}?
is equivalent to
\begin{lcode}
\newcommand{\Xpshook}{code}
\end{lcode}
I have used this knowledge in the line:
\begin{lcode}
\makepsmarks{floatcomp}{\companionpshook}
\end{lcode}
which avoids retyping the code from 
\verb?\makepsmarks{companion}{...}?,
and ensures that the code is actually the same for the two pagestyles.

\begin{syntax}
\cmd{\mergepagefloatstyle}\marg{style}\marg{textstyle}\marg{floatstyle} \\
\end{syntax}
    If you have two pre-existing pagestyles, one that will be used for
text pages and the other that can be used for float pages, then the
\cmd{\mergepagefloatstyle} command provides a simpler means of 
combining\index{pagestyle!combining}
them than the above example code for \pstyle{floatcomp}. The argument
\meta{style} is the name of the pagestyle being defined. The
argument \meta{textstyle}
is the name of the pagestyle for text pages and \meta{floatstyle} is the
name of the pagestyle for float-only pages. Both of these must have been 
defined before calling \cmd{\mergepagefloatstyle}. So, instead of the long
winded, and possibly tricky, code I could have simply said:
\begin{lcode}
\mergepagefloatstyle{floatcomp}{companion}{plain}
\end{lcode}


    One author thought it would be nice to be able to have different 
page headings according
to whether the page was a floatpage, or there was a float at the top of 
the page, or a float at the bottom of a page or there was text at the 
top and bottom.

    This, I think, is not a common requirement and, further, that to provide 
this involves changing parts of the LaTeX output routine --- something only
to be tackled by the bravest of the brave. If it were to be done then were
best done in a package that could be easily ignored. The following is an
outline of what might be done; I do not recommend it and if you try this 
and all your work dissappears then on your own head be it.

\begin{lcode}
% notefloat.sty
\newif\iffloatattop
  \floatattopfalse
\newif\iffloatatbot
  \floatatbotfalse

\renewcommand*{\@addtotoporbot}{%
  \@getfpsbit \tw@
  \ifodd \@tempcnta
    \@flsetnum \@topnum
    \ifnum \@topnum>\z@
      \@tempswafalse
      \@flcheckspace \@toproom \@toplist
      \if@tempswa
        \@bitor\@currtype{\@midlist\@botlist}%
        \if@test
        \else
          \@flupdates \@topnum \@toproom \@toplist
          \@inserttrue
  \global\floatattoptrue
        \fi
      \fi
    \fi
  \fi
  \if@insert
  \else
    \@addtobot
  \fi}

\renewcommand*{\@addtobot}{%
  \@getfpsbit 4\relax
  \ifodd \@tempcnta
    \@flsetnum \@botnum
    \ifnum \@botnum>\z@
      \@tempswafalse
      \@flcheckspace \@botroom \@botlist
      \if@tempswa
        \global \maxdepth \z@
        \@flupdates \@botnum \@botroom \@botlist
        \@inserttrue
  \global\floatatbottrue
      \fi
    \fi
  \fi}

\let\p@wold@output\@outputpage
\renewcommand*{\@outputpage}{%
  \p@wold@output
  \global\floatattopfalse
  \global\floatatbotfalse}

\endinput
\end{lcode}
\cs{floatattop} is probably set \ptrue\ if there is a float at the 
top of the page and
\cs{floatatbot} is probably set \ptrue\ if there is a float at the bottom
of the page.


\index{float!page|)}
\index{pagestyle!float pages|)}

\section{The showlocs pagestyle}

    The \pstyle{showlocs} pagestyle is somewhat special as it is meant to be
used as an aid when designing a page layout. Lines are drawn showing the 
vertical positions of the headers and footers and a box is drawn around
the textblock. It is implemented using two 
zero-sized\index{zero-sized picture} pictures.\verbfootnote{A
zero-sized picture starts off with \verb?begin{picture}(0,0)...?.}
\begin{syntax}
\cmd{\framepichead} \\
\cmd{\framepictextfoot} \\
\cmd{\framepichook} \\
\cmd{\showheadfootlocoff} \\
\cmd{\showtextblockoff} \\
\end{syntax}
\glossary(framepichead)%
  {\cs{framepichead}}%
  {Used by the \Ppstyle{showlocs} pagestyle to draw a line at the header 
   location.}
\glossary(framepictextfoot)%
  {\cs{framepictextfoot}}%
  {Used by the \Ppstyle{showlocs} pagestyle to draw a box around the textblock
   and a line at the footer location.}
\glossary(framepichook)%
  {\cs{framepichook}}%
  {First thing called inside the zero width pictures provided by
    \cs{framepichead} and \cs{framepictextfoot}. Empty by default.}%
\glossary(showtextblockoff)%
  {\cs{showtextblockoff}}%
  {Prevents \cs{framepictextfoot} from drawing a box around the textblock.}
\glossary(showheadfootlocoff)%
  {\cs{showheadfootlocoff}}%
  {Prevents \cs{framepichead} and \cs{framepictextfoot} from drawing
   lines at the header and footer locations.}
%
The macro \cmd{\framepichead} creates a zero-sized\index{zero-sized picture} 
picture that draws a line at the header location, and the macro 
\cmd{\framepictextfoot} creates a zero-sized\index{zero-sized picture} 
picture that draws a line at the footer location
and also draws a box around the typeblock. Following the declaration
\cmd{\showheadfootlocoff} the macros \cmd{\framepichead} and 
\cmd{\framepictextfoot} do not draw lines showing the header and footer
locations. The declaration \cmd{\showtextblockoff} prevents
\cmd{\framepictextfoot} from drawing a box around the textblock.

In case you want to change the color of the \pstyle{showlocs}, simply do
\begin{lcode}
  \renewcommnand\framepichook{\color{red}}
\end{lcode}



    If you generally want a box around the textblock you may want to create
your own pagestyle using \cmd{\framepictextfoot} and the \pstyle{showlocs}
code as a starting point, see \path{memoir.cls} for details.


\section{Other things to do with page styles}
\label{sec:other-things-do}

Back on \pref{ex:draft.pagestyle} we presented a way of adding some
draft information. Here is a more advanced example of this.

One interesting use for page styles is to provide extra information
below the footer. This might be some kind of copyright information. Or
if your document is under version control with a system like
Subversion, and you have all your chapter laying in seperate files,
then why not add information at the start of very chapter, specifying
who did the last change to this chapter at which time. See the
\texttt{svn-multi} package (\cite{svn-multi}) and the Prac\TeX{}
Journal article \cite{practex-2007-3-ms} by the same author. Then this
information can be added to the start of every chapter using something
like:
\begin{lcode}
\usepackage[filehooks]{svn-multi}
\makeatletter
% remember to define a darkgray color
\newcommand\addRevisionData{%
  \begin{picture}(0,0)%
    \put(0,-10){%
      \tiny%
      \expandafter\@ifmtarg\expandafter{\svnfiledate}{}{%
        \textcolor{darkgray}{Chapter last updated 
          \svnfileyear/\svnfilemonth/\svnfileday
         \enspace \svnfilehour:\svnfileminute\ (revision \svnfilerev)}
     }%
    }%
  \end{picture}%
}
\makeatother
% chapter is normally an alias to the plain style, we want to change
% it, so make it a real pagestyle
\makepagestyle{chapter} 
\makeoddfoot{chapter}{\addRevisionData}{\thepage}{}
\makeevenfoot{chapter}{\addRevisionData}{\thepage}{}
\end{lcode}
