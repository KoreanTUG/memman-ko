\chapter{\prBackmatter} \label{chap:backmatter}

The \pixbackmatter\ consists of reference and supportive elements for 
the \pixmainmatter;
things like bibliographies, indexes, glossaries, endnotes, and other
material. The class provides additional elements and features of such 
matter that are not in the standard classes.

\section{Bibliography} \label{sec:xref:bibliography}

\index{bibliography|(}

Just as a reminder the bibliography is typeset within the 
\Ie{thebibliography} environment.
\begin{syntax}
\cmd{\bibname} \\
\senv{thebibliography}\marg{exlabel} \\
\cmd{\bibitem} ... \\
\eenv{thebibliography} \\
\end{syntax}
\glossary(bibname)%
  {\cs{bibname}}%
  {The title for a bibliography}
\glossary(thebibliography)%
  {\senv{thebibliography}\marg{exlabel}}%
  {Environment for typesetting a bibliography. \meta{exlabel} is an arbitrary
    piece of text as wide as the widest label for the bibliographic items.}
\glossary(bibitem)%
  {\cs{bibitem}}%
  {Starts a new bibliographic entry in a \Pe{thebibliography} listing.}
The environment takes one required argument, \meta{exlabel}, which is a 
piece of text
as wide as the widest label in the bibliography. The value of 
\cmd{\bibname} (default `Bibliography') is used
as the title. 

\begin{syntax}
\cmd{\bibintoc} \cmd{\nobibintoc} \\
\end{syntax}
\glossary(bibintoc)%
  {\cs{bibintoc}}%
  {Title of \Pe{thebibliography} will be added to the \prtoc.}
\glossary(nobibintoc)%
  {\cs{nobibintoc}}%
  {Title of \Pe{thebibliography} is not added to the \prtoc.}
The declaration \cmd{\bibintoc} will cause the \Ie{thebibliography}
environment to add the title\index{bibliography!title in ToC} to 
the \toc, while the declaration
\cmd{\nobibintoc} ensures that the title is not added to the \toc. 
The default is \cmd{\bibintoc}.

\begin{syntax}
\cmd{\cite}\oarg{detail}\marg{labstr-list} \\
\end{syntax}
\glossary(cite)%
  {\cs{cite}\oarg{detail}\marg{labstr-list}}%
  {Citation in the text to bibliographic items specified in the 
   \meta{labstr-list} of comma-separated bibliographic identifiers;
  optional information, e.g., page number, is supplied via \meta{detail}.}
Within the text you call out a bibliographic\index{cite bibliographic item} 
reference using the
\cmd{\cite} command, where \meta{labstr-list} is a comma-separated
list of identifiers for the cited works; there must be no spaces in this 
list. The optional \meta{detail} argument is for any additional
information regarding the citation such as a chapter or page number;
this is printed after the main reference. 

    Various aspects of a bibliography can be changed and at this point
it may be helpful to look at some of the internals of the \Ie{thebibliography}
environment, which is defined like this
\begin{lcode}
\newenvironment{thebibliography}[1]{%
  \bibsection
  \begin{bibitemlist}{#1}}%
  {\end{bibitemlist}\postbibhook}
\end{lcode}
The bibliographic entries are typeset as a list, the \Ie{bibitemlist}.

\begin{syntax}
\cmd{\bibsection} \\
\end{syntax}
\glossary(bibsection)%
  {\cs{bibsection}}%
  {Initialises the bibliography and typesets the title.}
The macro \cmd{\bibsection} defines the heading\index{bibliography!heading} 
for the \Ie{thebibliography}
environment; that is, everything before the actual list of items.
It is effectively defined as
\begin{lcode}
\newcommand{\bibsection}{%
  \chapter*{\bibname}
  \bibmark
  \ifnobibintoc\else
    \phantomsection
    \addcontentsline{toc}{chapter}{\bibname}
  \fi
  \prebibhook}
\end{lcode}
If you want to change the heading, redefine \cmd{\bibsection}. For example,
to have the bibliography as a numbered section instead of an unnumbered
chapter, redefine it like
\begin{lcode}
\renewcommand{\bibsection}{%
  \section{\bibname}
  \prebibhook}
\end{lcode}
If you use the \Lpack{natbib}~\cite{NATBIB} and/or the 
\Lpack{chapterbib}~\cite{CHAPTERBIB} packages with the \Lopt{sectionbib}
option, then they change \cmd{\bibsection} appropriately to typeset the
heading as a numbered section.

\begin{syntax}
\cmd{\bibmark} \\
\end{syntax}
\glossary(bibmark)%
  {\cs{bibmark}}%
  {Can be used in pagestyles for page headers in a bibliography.}
\cs{bibmark} may be used in pagestyles for page headers in a bibliography.
Its default definition is: 
\begin{lcode}
\newcommand*{\bibmark}{}
\end{lcode}
but could be redefined like, say,
\begin{lcode}
\renewcommand*{\bibmark}{\markboth{\bibname}{}}
\end{lcode}


\begin{syntax}
\cmd{\prebibhook} \cmd{\postbibhook} \\
\end{syntax}
\glossary(prebibhook)%
  {\cs{prebibhook}}%
  {Called between typesetting the title of a bibliography and starting
   the list of bibliographic entries.}
\glossary(postbibhook)%  
  {\cs{postbibhook}}%
  {Called after typesetting the list of of bibliographic entries.}
The commands \cmd{\prebibhook} and \cmd{postbibhook} are called after 
typesetting the title of the bibliography and after typesetting the list of
entries, respectively. By default, they are defined to do nothing. You may
wish to use one or other of these to provide some general 
information\index{bibliography!explanatory text} about
the bibliography. For example:
\begin{lcode}
\renewcommand{\prebibhook}{%
CTAN is the Comprehensive \tx\ Archive Network and URLS for the 
several CTAN mirrors can be found at \url{http://www.tug.org}.}
\end{lcode}

\index{bibliography!list styling|(}

\begin{syntax}
\cmd{\biblistextra} \\
\end{syntax}
\glossary(biblistextra)%
  {\cs{biblistextra}}%
  {Called immediately after the \Pe{bibitemlist} is set up.}
Just at the end of setting up the \Ie{bibitemlist} the \cmd{\biblistextra}
command is called. By default this does nothing but you may change it to
do something useful. For instance, it can be used to change
the list parameters so that the entries are 
typeset flushleft.\index{bibliography!flushleft entries}
\begin{lcode}
\renewcommand*{\biblistextra}{%
  \setlength{\leftmargin}{0pt}%
  \setlength{\itemindent}{\labelwidth}%
  \addtolength{\itemindent}{\labelsep}}
\end{lcode}

\begin{syntax}
\cmd{\setbiblabel}\marg{style} \\
\end{syntax}
\glossary(setbiblabel)%
  {\cs{setbiblabel}\marg{style}}%
  {Define the look of the bibliographic entry identifiers.}
The style of the labels\index{bibliography!label styling} marking the 
bibliographic entries can be set
via \cmd{\setbiblabel}. The default definition is
\begin{lcode}
\setbiblabel{[#1]\hfill}
\end{lcode}
where \verb?#1? is the citation mark position, which generates flushleft 
marks enclosed in square brackets. To have marks just
followed by a dot
\begin{lcode}
\setbiblabel{#1.\hfill}
\end{lcode}

\begin{syntax}
\cmd{\bibitem}\oarg{label}\marg{labstr} \\
\cmd{\newblock} \\
\end{syntax}
\glossary(bibitem)%
  {\cs{bibitem}\oarg{label}\marg{labstr}}%
  {Introduces an entry in the bibliography. The \meta{labstr} argument
   corresponds to a \cs{cite}'s \meta{labstr} argument. The optional
   \meta{label} overides the default numerical printed entry label.}
\glossary(newblock)%
  {\cs{newblock}}%
  {Used in a bibliography to indicate a convenient place in an entry to
   have a pagebreak.}
Within the \Ie{bibitemlist} environment the entries are introduced
by \cmd{\bibitem} instead of the more normal \cmd{\item} macro.
The required \meta{labstr} argument is the identifier for the citation and
corresponds to a \meta{labstr} for \cmd{\cite}. The items in the list
are normally labelled numerically but this can be overriden by using
the optional \meta{label} argument. The \cmd{\newblock} command can be used
at appropriate places in the entry for encouraging a linebreak (this is
used by the \Lopt{openbib} option).

\begin{syntax}
\lnc{\bibitemsep} \\
\end{syntax}
\glossary(bibitemsep)%
  {\cs{bibitemsep}}%
  {Vertical space between entries in a bibliography.}
In the listing the vertical space between entries is controlled by the
length \lnc{\bibitemsep}, which by default is set to the normal 
\lnc{\itemsep} value. The vertical space is 
\texttt{(\lnc{\bibitemsep} + \lnc{\parsep})}. If you wish to eliminate
the space between items do
\begin{lcode}
\setlength{\bibitemsep}{-\parsep}
\end{lcode}

\index{bibliography!list styling|)}

\subsection{BibTex}

     Often the \Lbibtex\ program~\cite{BIBTEX} is used to generate the 
bibliography list from database(s) of 
bibliographic\index{bibliographic database} data. For \Lbibtex\ 
a bibliographic data base is a \pixfile{bib} file containing information
necessary to produce entries in a bibliography. \Lbibtex\ 
extracts the raw data from the files for each citation in the text and 
formats it for typesetting according to a particular style.


\begin{syntax}
\cmd{\bibliography}\marg{bibfile-list} \\
\end{syntax}
\glossary(bibliography)%
  {\cs{bibliography}\marg{bibfile-list}}%
  {Print the bibliography having used \Pbibtex\ to extract entries from
   the \meta{bibfile-list} of comma-separated names of \file{bib} files.}
    The bibliography will be printed at the location of the \cmd{\bibliography}
command. Its argument is a comma-separated list of \Pbibtex\ \pixfile{bib} files 
which will be searched by \Lbibtex\ to generate the bibliography.
Only the file name(s) should be supplied, the extension must not be given.


\begin{syntax}
\cmd{\nocite}\marg{labstr} \verb?\nocite{*}? \\
\end{syntax}
\glossary(nocite)%
  {\cs{nocite}\marg{labstr}}%
  {Add entry \meta{labstr} to the bibliography, but with no in-text citation.}
The command \cmd{\nocite} causes \Lbibtex\ to make an entry 
in the bibliography but no citation will appear in the text. The special
case \verb?\nocite{*}? causes \emph{every} entry in the database to be
listed in the bibliography.

\begin{syntax}
\cmd{\bibliographystyle}\marg{style} \\
\end{syntax}
\glossary(bibliographystyle)%
  {\cs{bibliographystyle}\marg{style}}%
  {Typeset the bibliographic entries according to \meta{style}.}
Many different \Pbibtex\ styles\index{BibTeX style?\Pbibtex\ style}
are available and the particular one to be used is specified
by calling \cmd{\bibliographystyle} before the bibliography itself.
The `standard' bibliography \meta{style}s follow the general schemes
for mathematically oriented papers and are:
\begin{itemize}
\item[\texttt{plain}]\index{BibTeX style?\Pbibtex\ style!plain?\texttt{plain}} 
     The entry format is similar to one suggested by
     Mary-Claire van Leunen~\cite{LEUNEN92}, and entries are sorted
     alphabetically and labelled with numbers.
\item[\texttt{unsrt}]\index{BibTeX style?\Pbibtex\ style!unsrt?\texttt{unsrt}} 
     The format is the same as \texttt{plain} but
     that entries are ordered by the citation order in the text.
\item[\texttt{alpha}]\index{BibTeX style?\Pbibtex\ style!alpha?\texttt{alpha}} 
     The same as \texttt{plain} but entries are 
     labelled like `Wil89', formed from the author and publication date.
\item[\texttt{abbrv}]\index{BibTeX style?\Pbibtex\ style!abbrv?\texttt{abbrv}} 
     The same as \texttt{plain} except that some elements, like month 
     names, are abbreviated.
\end{itemize}
There are many other styles available, some of which are used
in collaboration with a package, one popular one being 
Patrick Daly's \Lpack{natbib}~\cite{NATBIB} package for the kinds of 
author-year citation styles used in the natural sciences. 
Another package is \Lpack{jurabib}~\cite{JURABIB} for citation styles
used in legal documents where the references are often given in footnotes
rather than listed at the end of the document.

    I assume you know how to 
generate\index{running BibTeX?running \Pbibtex} a bibliography using \Lbibtex,
so this is just a quick reminder. You first run \ltx\
on your document, having specified the bibliography style, cited
your reference material and listed the relevant \Lbibtex\ database(s). 
You then run \Lbibtex, and after running \ltx\ twice more the
bibliography should be complete. After a change to your citations you have to
run \ltx\ once, \Lbibtex\ once, and then \ltx\ twice more to get an 
updated set of references.

    The format and potential contents of a \Pbibtex\
database\index{BibTeX database?\Pbibtex\ database} file 
(a \pixfile{bib} file) are specified in detail in Lamport~\cite{LAMPORT94} 
and the first of the \btitle{Companions}~\cite{COMPANION}. 
Alternatively
there is the documentation by Oren Patashnik~\cite{BIBTEX} who wrote the
\Lbibtex\ program.

\index{BibTeX style?\Pbibtex\ style!changing|(}

    A \Pbibtex\ style, specified in a \pixfile{bst} file, is written 
using an anonymous stack based language
created specifically for this purpose. If you can't find a 
\Pbibtex\ style\index{BibTeX style?\Pbibtex\ style}
that provides what you want you can either use the 
\Lpack{makebst}~\cite{MAKEBST} package
which leads you through creating your own style via a question and answer
session, or you can directly write your own. If you choose the latter
approach then Patashnik's \textit{Designing BibTeX files}~\cite{BIBTEXHACK}
is essential reading. As he says, it is best to take an existing style and
modify it rather than starting from scratch. 

\begin{comment}

This is what I did for the
style for this book, as all I wanted was a slight change and extension
to the standard 
\texttt{alpha}\index{BibTeX style?\Pbibtex\ style!alpha?\texttt{alpha}}  
style, which is in \file{alpha.bst}\ixfile{bst}.

    There were three things that I wanted to do:
\begin{itemize}
\item Add an `isbn' field to the entries so an ISBN\index{ISBN} number 
      could be easily quoted;
\item Add an `annote' field so that I could perhaps provide an 
      annotated\index{bibliography!annotated}
      bibliography;
\item Use the modern \cmd{\emph} command instead of the deprecated \cmd{\em}
      command for titles;
\end{itemize}

    If you aren't interested in how I did it, skip the next part, 
but if you are you might find it easier to follow if you have a copy 
of the \pixfile{alpha.bst} file to hand. 

This is what I did, although not in this order as I kept flitting
back and forth in order to resolve the problems that arose.
\begin{itemize}
\item Copied the \pixfile{alpha.bst} file to a new file I called 
      \pixfile{typo.bst} as my new style was to be called \texttt{typo}.

\item Near the start of the file are some lines:
\begin{lcode}
ENTRY
  { address
    author
    ...
    year
    isbn
    annote
  }
\end{lcode}
at the end of which I added the \texttt{isbn} and \texttt{annote}. These are
the fields that may apear in an entry. Later I have to describe how these
fields are to be dealt with.

\item
    Shortly after the \texttt{ENTRY} list there is a set of \texttt{FUNCTION}
specifications, the 17th of which is called \texttt{emphasize}. This is the 
only place in the file where the \cmd{\em} macro appears, so this is what I 
have to modify so that the \texttt{typo} style will use \cmd{\emph} instead
of \cmd{\em}. My revised definition is:
\begin{lcode}
FUNCTION {emphasize}
{ duplicate$ empty$ 
    { pop$ "" }
%%    { "{\em " swap$ * "}" * }  % original, change to
    { "\emph{ " swap$ * "}" * }  % PW mod
  if$
}
\end{lcode}
I didn't (and still don't) know just how the function operated but my
modification worked.

\item
    After this were a lot of functions of the form \verb?{format.something}?
which I took to be the formatting instructions for the fields. Looking
at the various functions I added the following ones.
\begin{lcode}
%% PW added format.isbn
FUNCTION {format.isbn}
{ isbn empty$
  { "" }
  { new.block " ISBN " isbn * }
    if$
}

%% PW added format.annote
{ annote empty$
  { "" }
  { " \begin{quotation}\noindent " annote * 
    " \end{quotation} " * }
    if$
}

%% PW added fin.annote
{ annote empty$
  {  }
  { newline$ }
    if$
}
\end{lcode}
%$

\item
    The last thing that I had to do was to get the entries to write
out the new \texttt{annote} and \texttt{isbn} fields. As an example, here 
is the revised function for a \texttt{booklet} entry, which is one of the
shorter ones.
\begin{lcode}
FUNCTION {booklet}
{ output.bibitem
  format.authors output
  new.block
  format.title "title" output.check
  howpublished address new.block.checkb
  howpublished output
  address output
  format.date output
  format.isbn output    %% PW added
  new.block
  note output
  fin.entry
  format.annote write$  %% PW added
  fin.annote            %% PW added
}
\end{lcode}
%$

I added similar lines to all the other entry functions except, for example, 
the \texttt{article} function where I only added the \texttt{annote}
lines as I assumed that an article would not have an ISBN\index{ISBN}.

\end{itemize}

    It took me three or four attempts to make it all work as I didn't 
really know what I was doing. I basically copied something that looked
close to what I might need, changed some names, and tried it out. If it
didn't work then I tried something a bit different until it did.
For someone who knew what they were doing it would have been a trivial
task and they would probably have used a more elegant solution, but
it works and didn't take too long.
\end{comment}

\index{BibTeX style?\Pbibtex\ style!changing|)}


\index{bibliography|)}

\section{Index} \label{sec:xref:index}


   It is time to take a closer look at indexing. The class allows 
multiple indexes\index{index!multiple} and an index may be typeset as a 
single\indextwo{index}{single column} or a 
double\indextwo{index}{double column} column.

    The general process is to put indexing commands into your source text, 
and \ltx\ will write this raw indexing data to an \pixfile{idx} file. 
The raw index data is then processed, not by \ltx\ but by yourself if you 
have plenty of spare time on your hands, or more usually by a separate 
program, to create a sorted list of indexed items in a second file (usually
an \pixfile{ind} file). This can then be input to \ltx\ to print the sorted
index data.

\subsection{Printing an index}
\index{index!printing|(}

\begin{syntax}
\cmd{\makeindex}\oarg{file} \\
\cmd{\printindex}\oarg{file} \\
\end{syntax}
\glossary(makeindex)%
  {\cs{makeindex}\oarg{file}}%
  {Preamble command to collect raw index information. By default this
   is written to file \cs{jobname}\texttt{.idx}. If the optional argument
   is used it may be written to file \meta{file}\texttt{.idx}.}
\glossary(printindex)%
  {\cs{printindex}\oarg{file}}%
  {Print the sorted index. By default this is read from file 
   \cs{jobname}\texttt{.ind}. If the optional argument is given
   it will read the data from file \meta{file}\texttt{.ind}.}
In order to make \ltx\ collect indexing information the declaration 
\cmd{\makeindex} must be put in the preamble\index{preamble}. By default
the raw index data is put into the \file{jobname.idx}\ixfile{idx} file. If
the optional \meta{file} argument is given then index data can be
output to \file{file.idx}. Several \cmd{\makeindex} declarations
can be used provided they each call for a different file.

    The \cmd{\printindex} command will print\index{index!print} an index 
where by default the indexed items are assumed to be in a file called 
\pixfile{jobname.ind}\ixfile{ind}. If the optional \meta{file} argument 
is given
then the indexed items are read from the file called \file{file.ind}.

%    The typical method of generating an \pixfile{ind} file containing
%the sorted index entries from the raw index data in an
%\pixfile{idx} file is to use the \Lmakeindex\ program~\cite{CHEN88}.


\begin{syntax}
\senv{theindex} entries \eenv{theindex} \\
\cmd{\onecolindex} \cmd{\twocolindex} \\
\cmd{\indexname} \\
\end{syntax}
\glossary(theindex)%
  {\senv{theindex}}%
  {Environment for typesetting an index}
\glossary(onecolindex)%
  {\cs{onecolindex}}%
  {Typeset index in one column.}
\glossary(twocolindex)%
  {\cs{twocolindex}}%
  {Typeset index in two columns (the default).}
\glossary(indexname)%
  {\cs{indexname}}%
  {Name used for the theindex title.}
The index entries are typeset within the \Ie{theindex} 
environment. By default it is typeset with two 
columns\indextwo{double column}{index}
but following the \cmd{\onecolindex} declaration the environment
uses a single\indextwo{index}{single column} column. 
The default two column behaviour is restored
after the \cmd{\twocolindex} declaration.
The index\indextwo{index}{name} title is given by the current value of 
\cmd{\indexname} (default `Index').

\begin{syntax}
\cmd{\indexintoc} \cmd{\noindexintoc} \\
\end{syntax}
\glossary(indexintoc)%
  {\cs{indexintoc}}%
  {Add the index title to the \prtoc\ (the default).}
\glossary(noindexintoc)%
  {\cs{noindexintoc}}%
  {Do not add the index title to the \prtoc.}
The declaration \cmd{\indexintoc} will cause the \Ie{theindex}
environment to add the title\index{index!title in ToC} to the \toc, 
while the declaration
\cmd{\noindexintoc} ensures that the title is not added to the \toc. 
The default is \cmd{\indexintoc}.

\begin{syntax}
\lnc{\indexcolsep} \\
\lnc{\indexrule} \\
\end{syntax}
\glossary(indexcolsep)%
  {\ls{indexcolsep}}%
  {Width of the gutter in two column indexes.}
\glossary(indexrule)%
  {\ls{indexrule}}%
  {Width of the inter-column rule in two column indexes.}
The length \lnc{\indexcolsep} is the width of the gutter between the two
index columns\index{index!double column!gutter}
The length \lnc{\indexrule}, default
value 0pt, is the thickness of a vertical rule separating the two columns.


\begin{syntax}
\cmd{\preindexhook} \\
\end{syntax}
\glossary(preindexhook)%
  {\cs{preindexhook}}%
  {Called between typesetting an index's title and the start of the list.}
    The macro \cmd{\preindexhook}\index{index!explanatory text} 
is called after the title is typeset and
before the index listing starts. By default it does nothing but
can be changed. For example
\begin{lcode}
\renewcommand{\preindexhook}{Bold page numbers are used 
  to indicate the main reference for an entry.}
\end{lcode}

\begin{syntax}
\cmd{\indexmark} \\
\end{syntax}
\glossary(indexmark)%
  {\cs{indexmark}}%
  {Can be used in pagestyles for page headers in an index.}
\cs{indexmark} may be used in pagestyles for page headers in an index.
Its default definition is: 
\begin{lcode}
\newcommand*{\indexmark}{}
\end{lcode}
but could be redefined like, say,
\begin{lcode}
\renewcommand*{\indexmark}{\markboth{\indexname}{\indexname}}
\end{lcode}




\begin{syntax}
\cmd{\ignorenoidxfile} \\
\cmd{\reportnoidxfile} \\
\end{syntax}
\glossary(ignorenoidxfile)%
  {\cs{ignorenoidxfile}}%
  {Do not report attempts to use an \file{idx} file that has not been
   declared by \cs{makeindex}.}
\glossary(reportnoidxfile)%
  {\cs{reportnoidxfile}}%
  {Report attempts to use an \file{idx} file that has not been
   declared by \cs{makeindex}.}
Following the declaration \cmd{\ignorenoidxfile}, which is the default,
LaTeX will silently pass over attempts to use an \pixfile{idx} file which has
not been declared via \cmd{\makeindex}. After the declaration
\cmd{\reportnoidxfile} LaTeX will whine about any attempts to 
write to an unopened file.

\index{index!printing|)}

\subsection{Preparing an index}

\index{index!preparation|(}

    It it is easy for a computer to provide a list of all the words you
have used, and where they were used. This is called a 
concordance\index{concordance}. 
Preparing an index, though, is not merely a gathering of words but 
is an intellectual
process that involves recognising and naming concepts, constructing a
logical hierarchy of these and providing links between related concepts.
No computer can do that for you though it can help with some tasks, such as
sorting things into alphabetical order, eliminating duplicates, and so forth.

    Several iterations may be required before you have an acceptable index.
Generally you pick out the important words or phrases used on the first pass.
Part of the skill of indexing is finding appropriate words to describe things
that may not be obvious from the text. If there are several ways of describing
something they may all be included using a `see' 
reference\index{index!see reference} to the most
obvious of the terms, alternatively you could 
use `see also'\index{index!see also reference} references
between the items. Entries should be broken down into subcategories so
that any particular item will not have a long string of page numbers and
your reader is more likely to quickly find the relevant place. After having 
got the first index you will most probably have to go back and correct
all the sins of ommission and commission, and start the cycle again.

    I found that indexing this manual was the most difficult part of preparing
it. It was easy to index the names of all the macros, environments, and so on
as I had commands that would simultaneously print and index these. It was
the concepts that was difficult. I inserted \cmd{\index} commands as I went
along at what seemed to be appropriate places but turned out not to be.
I would use slightly different words for the same thing, and what was worse
the same word for different things. It took a long time to improve it to 
its present rather pitiful state.

\begin{syntax}
\cmd{\index}\oarg{file}\marg{stuff} \\
\end{syntax}
\glossary(index)%
  {\cs{index}\oarg{file}\marg{stuff}}%
  {Add \meta{stuff} and the current page number to the raw index data. 
   By default this is written to
   file \cs{jobname}\texttt{.idx}. If the optional argument
   is given it will be written to file \meta{file}\texttt{.idx}.}
The \cmd{\index} macro specifies that \meta{stuff} is to appear in
an index. By default the raw index data --- the \meta{stuff} and the 
page number --- will be output
to the \pixfile{jobname.idx}\ixfile{idx} file, but if the optional \meta{file}
argument is given then output will be to the \file{file.idx} file.

    This book has two\index{index!multiple} indexes. 
The main index uses the default indexing
commands, while the second index does not. They are set up like this:
\begin{lcode}
% in preamble
\makeindex
\makeindex[lines]
% in body
...\index{main} ...\index[lines]{First line} ...
...
% at the end
\clearpage
% main index
\pagestyle{Index}
\renewcommand{\preindexhook}{%
The first page number is usually, but not always,
the primary reference to the indexed topic.\vskip\onelineskip}
\printindex

% second index
\clearpage
\pagestyle{ruled}
\renewcommand{\preindexhook}{}
\renewcommand{\indexname}{Index of first lines}
\onecolindex
\printindex[lines]
\end{lcode}


\begin{syntax}
\cmd{\specialindex}\marg{file}\marg{counter}\marg{stuff} \\
\end{syntax}
\glossary(specialindex)%
  {\cs{specialindex}\marg{file}\marg{counter}\marg{stuff}}%
  {Add \meta{stuff} and the current value of \meta{counter}
   to the raw index data file \meta{file}\texttt{.idx}.}
The \cmd{\index} command uses the page number\index{reference!to page} 
as the reference for 
the indexed item. In contrast the \cmd{\specialindex} command uses
the \meta{counter} as the reference\index{reference!to counter} 
for the indexed \meta{stuff}.
It writes \meta{stuff} to the \file{file.idx} file, and also writes
the page number (in parentheses) and the value of the \meta{counter}.
This means that indexing can be with respect to something other than page 
numbers. However, if the \Lpack{hyperref} package is used the special
index links will be to pages even though they will appear to be with 
respect to the \meta{counter}; for example, if figure numbers are used
as the index reference the hyperref link will be to the page where the
figure caption appears and not to the figure itself.


\begin{syntax}
\cmd{\showindexmarks} \cmd{\hideindexmarks} \\
\cmd{\indexmarkstyle} \\
\end{syntax}
\glossary(showindexmarks)%
  {\cs{showindexmarks}}%
  {The \meta{stuff} argument to \cs{index} and \cs{specialindex} will
   be printed in the margin (for use in noting what has been indexed where).}
\glossary(hideindexmarks)%
  {\cs{hideindexmarks}}%
  {The \meta{stuff} argument to \cs{index} and \cs{specialindex} will
   not be printed in the margin (the default).}
\glossary(indexmarkstyle)%
  {\cs{indexmarkstyle}}%
  {Font style for printing index marks in the margin.}
The declaration \cmd{\showindexmarks} causes the argument to practically 
any \cmd{\index} and \cmd{\specialindex} to be 
printed\index{index!show indexed items} in the margin of the
page where the indexing command was issued. The argument is printed using
the \cmd{\indexmarkstyle} which is initially specified as
\begin{lcode}
\indexmarkstyle{\normalfont\footnotesize\ttfamily}
\end{lcode}
For reasons I don't fully understand, spaces in the argument are displayed
as though it was typeset using the starred version of \cmd{\verb}.
The \cmd{\hideindexmarks}, which is the default, turns off 
\cmd{\showindexmarks}.

    The standard classes just provide the plain \cmd{\index} command with
no optional \meta{file} argument. In those classes the contents of the
\file{jobname.idx} file is limited to the index entries actually seen in 
the document. In particular, if you are using \cmd{\include} for
some parts of the document and \cmd{\includeonly} to exclude one or more
files, then any \cmd{\index} entries in an excluded file will not appear
in the \file{jobname.idx} file. The new implementation of indexing eliminates 
that potential problem.

\begin{syntax}
\cmd{\item} \cmd{\subitem} \cmd{\subsubitem} \\
\end{syntax}
\glossary(item)%
  {\cs{item}}%
  {Introduces a main index entry.}
\glossary(subitem)%
  {\cs{subitem}}%
  {Introduces a subsidiary index entry.}
\glossary(subsubitem)%
  {\cs{subsubitem}}%
  {Introduces a third level index entry.}
The \Ie{theindex} environment\index{index!entry levels} supports 
three levels of entries.
A \cmd{\item} command
flags a main\index{index!main entry} entry; a subentry\index{index!subentry} 
of a main entry is indicated by
\cmd{\subitem} and a subentry\index{index!subentry!subsubentry} 
of a subentry is flagged by 
\cmd{\subsubitem}. For example a portion of an index might look like:
\egstart[-2em]
\begin{lcode}
\item bridge, 2,3,7
\subitem railway, 24
\subsubitem Tay, 37
\end{lcode}
\egmid
bridge, 2, 3, 7\\
\hspace*{2em} railway, 24 \\
\hspace*{4em} Tay, 37
\egend
\noindent if the Tay Bridge\footnote{A railway (railroad) bridge in Scotland
that collapsed in 1879 killing 90 people. The disaster lives for ever in
the poem \textit{The Tay Bridge Disaster} by William McGonagall (1830--?), 
the first verse of which goes:
\begin{verse}
Beautiful Railway Bridge of the Silv'ry Tay!\index[lines]{Beautiful 
  Railway Bridge of the Silv'ry Tay} \\
Alas! I am very sorry to say \\
That ninety lives have been taken away \\
On the last Sabbath day of 1879, \\
Which will be remember'd for a very long time.
\end{verse}}
was mentioned on page 37.


\subsection{MakeIndex}

    It is possible, but time consuming and error prone, to create your
index by hand from the output of the \cmd{\index} commands you have scattered
throughout the text. Most use the \Lmakeindex\ program to do this
for them; there is also the \Lprog{xindy} program~\cite{XINDY} 
but this is much less known. 


\begin{syntax}
\cmd{\xindyindex} \\
\end{syntax}
It turns out that \Lprog{xindy} cannot handle a \Mname\ 
hyperindex\index{hyperindex} (which
can be obtained with the aid of the \Lpack{hyperref} package), although
\Lmakeindex\ can do so.\footnote{This deficiency in \Pprog{xindy}
was discovered by Frederic Connes\index{Connes, Frederic}, who also provided
the \cs{xindyindex} command.}
If you are going to use \Lprog{xindy} to process
the raw index data put \cmd{\xindyindex} in the preamble, which will prevent
hyperindexing\index{hyperindex}.

%%\index{MakeIndex?\Pmakeindex!raw data|(}
\Iprogsub{MakeIndex}{raw data|(}%

   \Lmakeindex\ reads an \pixfile{idx} file containg the raw index
data (which may include some commands to \Lmakeindex\ itself), sorts
the data, and then outputs an \pixfile{ind} file containing the sorted data,
perhaps with some \ltx\ commands to control the printing.
\Lmakeindex\ was created as a general purpose index processing program
and its operation can be controlled by a 
`makeindex configuration file'%
%%\index{MakeIndex?\Pmakeindex!configuration file}%
\Iprogsub{MakeIndex}{configuration file}%
\index{configuration file!MakeIndex?\Pmakeindex} 
(by default this is an \pixfile{ist} file). Such a file consists of two 
parts. The first
part specifies \Pmakeindex\ commands that can be included in the
\meta{stuff} argument to \cmd{\index}. The second part controls how 
the sorted index data is to be output.

    I will only describe the most common elements of what you can put in 
an \pixfile{ist} file; consult the \Pmakeindex\
manual~\cite{CHEN88}, or the \btitle{Companion}~\cite{COMPANION}, for all 
the details.

You can embed commands, in the form of single characters,
in the argument to \cmd{\index} that guide 
\Lmakeindex\ in converting the raw \pixfile{idx} file into an 
\pixfile{ind} file for final typesetting. The complete set of these
is given in \tref{tab:configin}. They all have defaults and you can modify
these via a \Lmakeindex\ configuration file.

\newcommand*{\kwd}[1]{\texttt{#1}}
\newcommand*{\kty}[1]{\textit{(#1)}}

\begin{table}
\centering
\caption{\Pmakeindex\ configuration file input parameters} \label{tab:configin}
\begin{tabular}{llp{0.5\textwidth}}\toprule
\multicolumn{1}{c}{Keyword} & \multicolumn{1}{c}{Default} & \multicolumn{1}{c}{Description} \\ \midrule
\kwd{keyword} \kty{s} & \verb?"\\indexentry"? & 
  The argument to this command is a \Pmakeindex{} index entry \\
\kwd{arg\_open} \kty{c} & \verb?'{'? &
  Argument start delimeter \\
\kwd{arg\_close} \kty{c} & \verb?'}'? &
  Argument end delimeter \\
\kwd{range\_open} \kty{c} & \verb?'('? &
  Start of an explicit page range \\
\kwd{range\_close} \kty{c} & \verb?')'? &
  End of an explicit page range \\
\kwd{level} \kty{c} & \verb?'!'? &
  Character denoting a new subitem level \\
\kwd{actual} \kty{c} & \verb?'@'? &
  Character denoting that the following text is to appear in the
  actual index file \\
\kwd{encap} \kty{c} & \verb?'|'? &
  Character denoting that the rest of the argument is to be used as
  an encapsulating command for the page number \\
\kwd{quote} \kty{c} & \verb?'"'? &
  Character that escapes the following character \\
\kwd{escape} \kty{c} & \verb?'\\'? &
  Symbol with no special meaning unless followed by the \kwd{quote}
  character, when both characters will be printed. The \kwd{quote}
  and \kwd{escape} characters must be different. \\
\kwd{page\_compositor} \kty{s} & \verb?"-"? &
  Composite number separator \\
\bottomrule
\multicolumn{3}{c}{\kty{s} of type string, \kty{c} of type character} 
\end{tabular}
\end{table}

    In the simplest case you just use the name of the index item as the 
argument to the \cmd{\index} command. However, spaces are significant as far
as \Lmakeindex\ is concerned. The following three uses of \cmd{\index}
will result in three different entries in the final index \\
\verb*?\index{ entry}? \verb*?\index{entry}? \verb*?\index{entry }?

\begin{figure}
\centering
\begin{small}
\begin{tabular}{ll|l}
p. v: & \verb?\index{Alf}? & \verb?\indexentry{Alf}{v}? \\
p. 1:  & \verb?\index{Alf}? & \verb?\indexentry{Alf}{1}? \\
p. 2:  & \verb?\index{Alf}? & \verb?\indexentry{Alf}{2}? \\
p. 3:  & \verb?\index{Alf}? & \verb?\indexentry{Alf}{3}? \\
p. 5: & \verb?\index{Alfabet|see{Bet}}? & \verb?\indexentry{Alfabet|see{Bet}}{5}? \\
p. 7: & \verb?\index{Alf@\textit{Alf}}? & \verb?\indexentry{Alf@\textit{Alf}}{7}? \\
         & \verb?\index{Bet|textbf}? & \verb?\indexentry{Bet|textbf}{7}? \\
p. 8:  & \verb?\index{Alf!Bet!Con}? & \verb?\indexentry{Alf!Bet!Con}{8}? \\
p. 9: & \verb?\index{Alf!Dan}? & \verb?\indexentry{Alf!Dan}{9}? \\
\end{tabular}\par
\end{small}
\caption{Raw indexing: (left) index commands in the source text; (right)
         \file{idx} file entries}
\end{figure}

\begin{figure}
\centering
\egstart
\begin{lcode}
\begin{theindex}
\item Alf, v, 1-3
  \subitem Bet
    \subsubitem Con, 8
  \subitem Dan, 9
\item \textit{Alf}, 7
\item Alfabet, \see{Bet}{5}
\indexspace
\item Bet, \textbf{7}
\end{theindex}
\end{lcode}
\egmid
Alf, v, 1-3 \\
\hspace*{2em} Bet \\
\hspace*{4em} Con, 8 \\
\hspace*{2em} Dan, 9 \\
\textit{Alf}, 7 \\
Alfabet, \textit{see} Bet \\[0.5\onelineskip]
Bet, \textbf{7}
\egend
\caption{Processed index: (left) alphabeticized \file{ind} file;
         (right) typeset index}
\end{figure}

\subsubsection{The \texttt{!} character}

    The \texttt{level} specifier starts a new minor level, or subitem,
with a maximum of two sub-levels. The default \texttt{level} specifier
is the special character \texttt{!}\index{"! (ls)?\texttt{"!} (level specifier)}. For example:
\begin{lcode}
\index{item!sub item!sub sub item}
\end{lcode}

\subsubsection{The \texttt{@} character}

    An indexable item may be represented in two portions, separated
by the \texttt{actual} specifier, which by default is the
\texttt{@} character\index{@ (as)?\texttt{@} (actual specifier)}. 
The portion before the \texttt{@} is used
when \Lmakeindex\ sorts the raw index data, and the portion after
the \texttt{@} is used as the entry text. For example:
\begin{lcode}
\index{MakeIndex@\textit{MakeIndex}}
\end{lcode}
will result in the final index entry of \verb?\textit{MakeIndex}? in the 
alphabetic position accorded to \verb?MakeIndex?. 
The same treatment can be applied for subitems:
\begin{lcode}
\index{program!MakeIndex@\textit{MakeIndex}!commands}
\end{lcode}

\subsubsection{The \texttt{|} character}

    Anything after the \texttt{encap} specifier, which by default
is the \texttt{|} character\index{"| (es)?\texttt{"|} (encap specifier)}, 
is treated as applying to the page number. In general
\begin{lcode}
\index{...|str}
\end{lcode}
will produce a page number, say n, in the form
\begin{lcode}
\str{n}
\end{lcode}
For example, if you wanted the page number of one particular entry
to be in a bold font, say to indicate that this is where the entry
is defined, you would do
\begin{lcode}
\index{entry|textbf}
\end{lcode}

    As a special case, if you want an index item to have a page range 
put the two
characters \verb?|(? at the end of the argument on the first page, and 
the character pair \verb?|)? at the end of the argument on the last page.
For example:
\begin{lcode}
... \index{range|(} pages about range  \index{range|)} ...
\end{lcode}
The two arguments must match exactly except for the final \verb?(? 
and \verb?)?. You can also do
\begin{lcode}
\index{...|(str}
\end{lcode}
which will produce a page range of the form
\begin{lcode}
\str{n-m}
\end{lcode}
In this case, if the range is only a single page, the result is simply
\begin{lcode}
\str{n}
\end{lcode}

\begin{syntax}
\cmd{\see}\marg{text}\marg{page} \cmd{seename} \\
\cmd{\seealso}\marg{text}\marg{page} \cmd{alsoname} \\
\end{syntax}
\glossary(see)%
  {\cs{see}}%
  {\textit{see} entry in an index using \cs{seename} for the wording.} 
\glossary(seename)%
  {\cs{seename}}%
  {Wording for a \textit{see} index entry.}
\glossary(seealso)%
  {\cs{seealso}}%
  {\textit{see also} entry in an index using \cs{alsoname} for the wording.} 
\glossary(alsoname)%
  {\cs{alsoname}}%
  {Wording for a \textit{see also} index entry.}
The macros \cmd{\see}\index{index!see reference} and 
\cmd{\seealso}\index{index!see also reference} are specifically for use in
an \cmd{\index} command after the \texttt{|}. The \cmd{\see} command
replaces the page number by the phrase `\textit{see} \meta{text}', 
while the \cmd{\seealso} command adds `\textit{see also} \meta{text}' 
to the entry. 
For example, in the source for this manual I have
\begin{lcode}
\index{chapter!style|see{chapterstyle}}
\index{figure|seealso{float}}
\end{lcode}
A \cmd{\see} or \cmd{\seealso}
should be used once only for a particular entry. The `see' texts for 
\cmd{\see} and \cmd{\seealso} are stored in \cmd{\seename}
and \cmd{\alsoname}, whose default definitions are:
\begin{lcode}
\newcommand*{\seename}{see}
\newcommand*{alsoname}{see also}
\end{lcode}

\subsubsection{The \texttt{"} and \texttt{\bs} characters}

    If, for some reason, you want to index something that includes one
of the \texttt{!}, \texttt{@}, \texttt{|} or \texttt{"} characters there
is the difficulty of persuading \Lmakeindex\ to ignore the special
meaning. This is solved by the \texttt{quote} specifier, which is
normally the \texttt{"} character\index{\" (qs)?\texttt{"} (quote specifier)}. 
The character
immediately after \texttt{"} is treated as non-special. For example,
if you needed to index the \texttt{@} and \texttt{!} characters:
\begin{lcode}
\index{"@ (commercial at)}
\index{"! (exclamation)}
\end{lcode}
The leading \texttt{"} is stripped off before entries are alphabetized.

    The \texttt{escape} specifier is used to strip the special meaning
from the \texttt{quote} specifier. This is usually the \texttt{\bs}
character\index{"\ (es)?\texttt{\bs} (escape specifier)}. 
So, to index the double quote character itself:
\begin{lcode}
\index{\" (double quote)}
\end{lcode}

\subsubsection{Example of using the special characters}

    Here is a short example of indexing the special characters. Given an
input like this in the document
\begin{lcode}
\index{exclamation mark ("!)}
\index{vicious|see{circle}}
\index{atsign@\texttt{"@} sign|\textbf}
\index{quote!double ("")}
\index{circle|see{vicious}}
\end{lcode}
then an index could eventually be produced that looks like:
\begin{quote}
\texttt{@} sign, \textbf{30}\\
circle, \textit{see} vicious\\
exclamation mark (!), 21 \\
quote \\
\hspace*{1.5em} double ("), 47 \\
vicious, \textit{see} circle\\
\end{quote}


%%\index{MakeIndex?\Pmakeindex!raw data|)}
\Iprogsub{MakeIndex}{raw data|)}%

\subsection{Controlling MakeIndex output}

%%\index{MakeIndex?\Pmakeindex!output styling|(}
\Iprogsub{MakeIndex}{output styling|(}%

Table~\ref{tab:configout} lists the parameters that control \Pmakeindex's
output, except for the keywords that control the setting of page numbers. 
The special characters and strings are not fixed within the
\Lmakeindex\ program. The program will read an \pixfile{ist} file
in which you can redefine all of \Lmakeindex's defaults.

\begin{table}
\begin{adjustwidth}{-1.5cm}{-1.5cm}
\centering
\caption{\Pmakeindex\ configuration file output parameters} \label{tab:configout}
\begin{tabular}{llp{0.5\textwidth}}\toprule
\multicolumn{1}{c}{Keyword} & \multicolumn{1}{c}{Default} & \multicolumn{1}{c}{Description} \\ \midrule
\kwd{preamble} \kty{s} & \verb?"\\begin{theindex}\n"? &
  Text for the start of the output file \\
\kwd{postamble} \kty{s} & \verb?"\n\n\\end{theindex}\n"? &
  Text at the end of the output file \\          \midrule
%\kwd{setpage\_prefix} \kty{s} & \verb?"\n\\setcounter{page}{"? &
%  Prefix for the command setting the page number \\ 
%\kwd{setpage\_suffix} \kty{s} & \verb?"}\n"? &
%  Suffix for the command setting the page number \\  \midrule
\kwd{group\_skip} \kty{s} & \verb?"\n\n\\indexspace\n"? &
  Vertical space before a new letter group \\
\kwd{heading\_prefix} \kty{s} & \verb?""? &
  Prefix for heading for a new letter group \\
\kwd{heading\_suffix} \kty{s} & \verb?""? &
  Suffix for heading for a new letter group \\
\kwd{headings\_flag} \kty{n} & \verb?0? &
  A value $= 0$ inserts nothing between letter groups. 
  A value $>0$ includes an uppercase instance of the new symbol,
  while a value $<0$ includes a lowercase instance, all
  within \kwd{heading\_prefix} and \kwd{heading\_suffix} \\ \midrule
\kwd{item\_0} \kty{s} & \verb?"\n\item "? &
  Command inserted in front of a level 0 entry \\
\kwd{item\_1} \kty{s} & \verb?"\n \subitem "? &
  As above for a level 1 entry \\
\kwd{item\_2} \kty{s} & \verb?"\n  \subsubitem "? &
  As above for a level 2 entry \\
\kwd{item\_01} \kty{s} & \verb?"\n \subitem "? &
  Command inserted in front of a level 1 entry starting at level 0 \\
\kwd{item\_12} \kty{s} & \verb?"\n  \subsubitem "? &
  Command inserted in front of a level 2 entry starting at level 1 \\
\kwd{item\_x1} \kty{s} & \verb?"\n \subitem "? &
  Command inserted in front of a level 1 entry when the parent level
  has no page numbers \\
\kwd{item\_x2} \kty{s} & \verb?"\n \subitem "? &
  As above for a level 2 entry \\                  \midrule
\kwd{delim\_0} \kty{s} & \verb?", "? &
  Delimiter between level 0 entry and first page number \\
\kwd{delim\_1} \kty{s} & \verb?", "? &
  As above for level 1 entry \\
\kwd{delim\_2} \kty{s} & \verb?", "? &
  As above for level 2 entry \\
\kwd{delim\_n} \kty{s} & \verb?", "? &
  Delimiter between page numbers \\
\kwd{delim\_r} \kty{s} & \verb?"-"? &
  Designator for a page range \\                  \midrule
\kwd{encap\_prefix} \kty{s} & \verb?"\\"? &
  Prefix in front of a page encapsulator \\
\kwd{encap\_infix} \kty{s} & \verb?"{"? &
  Infix for a page encapsulator \\
\kwd{encap\_suffix} \kty{s} & \verb?"}"? &
  Suffix for a page encapsulator \\                \midrule
\kwd{page\_precedence} \kty{s} & \verb?"rnaRA"? &
  Page number precedence for sorting. 
  \texttt{r} and \texttt{R} are lower- and uppercase roman;
  \texttt{a} and \texttt{A} are lower- and uppercase alphabetic;
  \texttt{n} is numeric \\                         \midrule
\kwd{line\_max} \kty{n} & \verb?"72"? &
  Maximum length of an output line  \\
\kwd{indent\_space} \kty{s} & \verb?"\t\t"? &
  Indentation commands for wrapped lines  \\
\kwd{indent\_length} \kty{n} & \verb?"16"? &
  Indentation length for wrapped lines  \\
\bottomrule
\multicolumn{3}{c}{\kty{s} of type string, \kty{n} of type number,
   \texttt{"\bs n"} and \texttt{"\bs t"} are newline and tab.} 
\end{tabular}
\end{adjustwidth}
\end{table}

    I have used a file called \file{memman.ist} for configuring
\Lmakeindex\ for this manual. Here it is:
\begin{lcode}
% MakeIndex style file  memman.ist

% @ is a valid character in some entries, use ? instead
actual '?'

% output main entry <entry> as: \item \idxmark{<entry>}, 
item_0  "\n\\item \\idxmark{"
delim_0 "}, "
% not forgetting the subitem case
item_x1 "} \n \\subitem "

% Wrap and uppercase head letters
headings_flag 1
heading_prefix "\\doidxbookmark{"
heading_suffix "}"
\end{lcode}

    Many items that I need to index include \texttt{@} as part of their
names, which is one of the special characters.
The \texttt{actual} line says that the character \texttt{?} performs 
the same function as the default \texttt{@} (and by implication, \texttt{@}
is not a special character as far as \Lmakeindex\ is concerned).

    The \verb?item_0? line says that a main entry in the generated index
starts 
\begin{lcode}
\item \idxmark{
\end{lcode}
and the \verb?delim_0? and \verb?item_x1? lines say that the main entry ends
\begin{lcode}
}, % or
}
    \subitem
\end{lcode}
Thus, main entries will appear in an \pixfile{ind} file like
\begin{lcode}
\item \idxmark{a main entry}, <list of page numbers> 
\item \idxmark{a main entry with no page numbers}
    \subitem subitem, <list of page numbers>
\end{lcode}

%%\index{MakeIndex?\Pmakeindex!output styling|)}
\Iprogsub{MakeIndex}{output styling|)}%


 Read the \Lmakeindex\ manual~\cite{CHEN88} for full details
on how to get \Lmakeindex\ to do what you want.

\LMnote{2009/06/30}{described \doidxbookmark}
The \verb?\doidxbookmark? that is used to wrap around the letter group
headers, can be used to not only style the group header, but can also
be used to add the headers in the bookmarks list. This can be done using
\begin{lcode}
\newcommand{\doidxbookmark}[1]{{\def\@tempa{Symbols}\def\@tempb{#1}%
  \centering\bfseries \ifx\@tempa\@tempb %
  Analphabetics 
  \phantomsection%
  \pdfbookmark[0]{Analphabetics}{Analphabetics-idx}%
%  \label{AnalphabeticsAnalphabeticsAnalphabetics-idx}%
  \else 
  #1%
  \phantomsection%
  \pdfbookmark[0]{#1}{#1-idx}%
%  \label{#1#1#1-idx}%
  \fi%
  \vskip\onelineskip\par}}
\end{lcode}
The labels are generally not needed but can be used to add a visual
representation of the index bookmarks into the index itself.


\subsection{Indexing and the \Lpack{natbib} package}

    The \Lpack{natbib} package~\cite{NATBIB} will make an index 
of citations if
\cmd{\citeindextrue} is put in the preamble after the \Lpack{natbib}
package is called for.

\begin{syntax}
\cmd{\citeindexfile} \\
\end{syntax}
\glossary(citeindexfile)%
  {\cs{citeindexfile}}%
  {File name for the citation index.}
The name of the file for the citation index is stored in the
macro \cmd{\citeindexfile}. This is initially defined as:
\begin{lcode}
\newcommand{\citeindexfile}{\jobname}
\end{lcode}
That is, the citation entries will be written to the default 
\pixfile{idx} file.
This may be not what you want so you can change this, for example to:
\begin{lcode}
\renewcommand{\citeindexfile}{names}
\end{lcode}
If you do change \cmd{\citeindexfile} then you have to put
\begin{lcode}
\makeindex[\citeindexfile]
\end{lcode}
\emph{before}
\begin{lcode}
\usepackage[...]{natbib}
\end{lcode}

    So, there are effectively two choices, either along the lines of
\begin{lcode}
\renewcommand{\citeindexfile}{authors} % write to authors.idx
\makeindex[\citeindexfile]
\usepackage{natbib}
\citeindextrue
...
\renewcommand{\indexname}{Index of citations}
\printindex[\citeindexfile]
\end{lcode}
or along the lines of
\begin{lcode}
\usepackage{natbib}
\citeindextrue
\makeindex
...
\printindex
\end{lcode}

\section{Glossaries}

    Unlike indexes, \ltx\ provides less than minimal support for 
glossaries. It provides a \cmd{\makeglossary} command for initiating a glossary
and a \cmd{\glossary} command which puts its argument, plus the page number,
into a \file{glo} file, and that's it. \Mname, combined with the
\Lmakeindex\ program~\cite{CHEN88}, enables you to generate 
and print a glossary in 
your document. The commands for creating a glossary are similar to those
for indexes.

\begin{syntax}
\cmd{\makeglossary}\oarg{file} \\
\end{syntax}
\glossary(makeglossary)%
  {\cs{makeglossary}\oarg{file}}%
  {Opens file \cs{jobname.glo}, or \cs{file.glo}, for glossary entries}%

You have to put \cmd{\makeglossary} in your preamble if you want a glossary.
This opens a file called by default \verb?\jobname.glo?. If you use the 
optional \meta{file} argument the file \verb?file.glo? will be opened.
A glossary \file{glo} file is analagous to an index \file{idx} file.

\begin{syntax}
\cmd{\printglossary}\oarg{file} \\
\end{syntax}
\glossary(printglossary)%
  {\cs{printglossary}\oarg{file}}%
  {Prints the glossary from file \cs{jobname.gls}, or \cs{file.gls}}%
To print a glossary call \cmd{\printglossary} which will print the glossary
from file \verb?\jobname.gls?, or from \verb?file.gls? if the optional 
argument is used. A glossary \file{gls} file is analagous to an
index \file{ind} file.

\begin{syntax}
\cmd{\glossary}\oarg{file}\parg{key}\marg{term}\marg{desc} \\
\end{syntax}
\glossary(glossary)%
  {\cs{glossary}\oarg{file}\parg{key}\marg{term}\marg{description}}%
  {Adds \meta{term} and its description, \meta{desc}, to a glossary file ---
   \cs{jobname.glo} by default or to \cs{file.glo}. The optional argument
   \meta{key} can be used to provide a different sort key for \meta{term}.}

Use the \cmd{\glossary} command to add a \meta{term} and its description,
\meta{desc},
to a glossary file. By default this will be \verb?\jobname.glo? but if the
optional \meta{file} argument is given then the information will be written
to \verb?file.glo?. The \parg{key} argument is optional. If present then
\meta{key} will be added to the file to act as a sort key for the \meta{term},
otherwise \meta{term} will be used as the sort key.

    By using the optional \meta{file} arguments you can have several 
glossaries, subject to \tx's limitations on the number of files that can
be open at any one time.

   A simple glossary entry might be:
\begin{lcode}
\glossary{glossary}{A list of terms and their descriptions.}
\end{lcode}

    The glossary facilites are designed so that the \Lmakeindex\ program
can be used to convert the raw glossary data in a \file{glo} file into
the printable glossary in a \file{gls} file.

\begin{syntax}
\senv{theglossary} entry list \eenv{theglossary} \\
\end{syntax}
\glossary(theglossary)%
  {\senv{theglossary}}%
  {Environment for typesetting a glossary.}%

Glossary entries are typeset in a \Ie{theglossary} environment. It is assumed
that a \file{gls} file will contain a complete \Ie{theglossary} environment,
from \senv{theglossary} all the way through to \eenv{theglossary}.

\begin{syntax}
\cmd{\glossitem}\marg{term}\marg{desc}\marg{ref}\marg{num} \\
\end{syntax}
\glossary(glossitem)%
  {\cs{glossitem}\marg{term}\marg{desc}\marg{ref}\marg{num}}%
  {Glossary entry used in a \Pe{theglossary} environment}%

A \cmd{\glossitem} is a glossary entry within a \Ie{theglossary} environment
for a \meta{term} with \meta{description}. The \meta{num} argument is the
page or section where the corresponding \cmd{\glossary} was issued. The
\meta{ref} argument, if not empty, might be the section or page number 
corresponding to the \meta{num} page or section number. The default definition
is
\begin{lcode}
\newcommand{\glossitem}[4]{#1 #2 #3 #4}
\end{lcode}
which is not very exciting. You may well prefer to use your own definition.

\subsection{Controlling the glossary}

\subsubsection{Setting up makeindex}

    If you just run \Lmakeindex\ on a \file{glo} file you will get lots
of errors; \Lmakeindex\ has to be configured to read a \file{glo}
file and generate a useful \file{gls} file as by default it expects to read
an index \file{idx} file and produce an index \file{ind} file. A configuration
file like an index \file{ist} file will be needed. There is no recommended
extension for such a file but I have come to favour \file{gst}. The
command line for \Lmakeindex\ to create a sorted glossary from the raw
data in a \file{glo} file, say \texttt{fred.glo}, using a configuration 
file called, say \texttt{basic.gst}, is
\begin{lcode}
makeindex -s basic.gst -o fred.gls fred.glo
\end{lcode}
For other jobs just change the file names appropriately.

    So, what is in a \file{gst} file? The potential contents were described
earlier, but at a minimum you need this:
\begin{lcode}
%%% basic.gst basic makindex glossary style file 
%%% Output style parameters
preamble "\\begin{theglossary}"
postamble "\n\\end{theglossary}\n"
item_0    "\n\\glossitem"
delim_0   "{\\memglonum{"
encap_suffix "}}}"
headings_flag 1
heading_prefix "\\doglobookmark{"
heading_suffix "}"
%%% Input style parameters
keyword "\\glossaryentry"
\end{lcode}

The \verb?keyword? line says that each entry in an input (\file{glo}) file
will be of the form:
\begin{lcode}
\glossaryentry{entry text}{number}
\end{lcode}
and by a miracle of coding, this is what \Pclass{memoir} will put in a 
\file{glo} file for each \cmd{\glossary} command.

    The \verb?preamble? and \verb?postamble? lines tell the program to start
and end its output file with \senv{theglossary} and \eenv{theglossary},
respectively.
The \verb?item_0? tells the program to start each output entry with
\cmd{\glossitem}. The \verb?delim_0? says that \verb?{\memglonum{?
should be put between the end of the entry text and the (page) number. Finally
\verb?encap_suffix? requests \verb?}}}? to be put after any `encapsulated'
(page) number.

    A complete listing of the possible entries in a configuration file,
also called a style file, for \Lmakeindex{} is in 
\tablerefname~\ref{tab:configin} and~\ref{tab:configout} with the exception
of the output file page number setting keywords.

\LMnote{2009/06/30}{Added bookmarks for letter groups in the glossary}
The \verb?\doglobookmark? macro can be used to add bookmarks for the
letter groups. In the case of this manual we do not write anything,
just add the letters to the glossary entry in the bookmark list. In
\Lpack{memsty} \verb?\doglobookmark? is defined as
\begin{lcode}
\newcommand\doglobookmark[1]{%
  \def\@tempa{Symbols}\def\@tempb{#1}%
  \ifx\@tempa\@tempb %
  \phantomsection\pdfbookmark[0]{Analphabetics}{Analphabetics-glo}%
  \else%
  \phantomsection\pdfbookmark[0]{#1}{#1-glo}%
  \fi%
}
\end{lcode}
\Lmakeindex\ uses the word 'Symbols' to specify the group that does not
start with a letter.


\subsubsection{Raw input data}

\begin{syntax}
\cmd{\@@wrglom@m}\marg{file}\marg{key}\marg{term}\marg{desc}\marg{ref}\marg{num}\\
\end{syntax}
The \cmd{\glossary} macro writes its arguments to the \file{aux} file
in the form of arguments to the \cmd{\@@wrglom@m} internal macro. In turn 
this calls a series of other macros that eventually write the data
to the \meta{file} \file{glo} file 
in the format (where \verb+@+ is the actual flag):
\begin{lcode}
\glossaryentry{key@{\memgloterm{term}} {\memglodesc{desc}}{\memgloref{ref}}
               |memglonumf}{num}
\end{lcode}
which \Lmakeindex\ then effectively converts into
\begin{lcode}
\glossitem{\memgloterm{term}}{\memglodesc{desc}}{\memgloref{ref}}
           {\memglonum{\memglonumf{num}}}
\end{lcode}

\begin{syntax}
\cmd{\memgloterm}\marg{term} \\
\cmd{\memglodesc}\marg{desc} \\
\cmd{\memgloref}\marg{ref} \\
\cmd{\memglonum}\marg{num} \\
\end{syntax}
\glossary(memgloterm)%
  {\cs{memgloterm}\marg{term}}%
  {Wrapper round a glossary term.}%
\glossary(memglodesc)%
  {\cs{memglodesc}\marg{desc}}%
  {Wrapper round a glossary description.}%
\glossary(memgloref)%
  {\cs{memgloref}\marg{ref}}%
  {Wrapper round a glossary ref.}%
\glossary(memglonum)%
  {\cs{memglonum}\marg{num}}%
  {Wrapper round glossary numbers.}%
These macros can be redefined to format the various parts of a glossary entry.
Their default definitions are simply
\begin{lcode}
\newcommand{\memgloterm}[1]{#1}
\newcommand{\memglodesc}[1]{#1}
\newcommand{\memgloref}[1]{#1}
\newcommand{\memglonum}[1]{#1}
\end{lcode}
For example, if you wanted the term in bold, the description in italics,
 and no numbers:
\begin{lcode}
\renewcommand{\memgloterm}[1]{\textbf{#1}}
\renewcommand{\memglodesc}[1]{\textit{#1}}
\renewcommand{\memglonum}[1]{}
\end{lcode}

   There are several macros that effect a glossary entry 
but which must not be directly modified (the \cs{memglonumf} shown above
as part of the \cmd{\glossaryentry} is one of these).
Each of the following \cs{changegloss...} macros takes an optional \meta{file}
argument. The changes to the underlying macro apply only to the 
glossary of that particular \meta{file} (or the \cs{jobname} file
if the argument is not present.
\begin{syntax}
\cmd{\changeglossactual}\oarg{file}\marg{char} \\
\cmd{\changeglossref}\oarg{file}\marg{thecounter} \\
\cmd{\changeglossnum}\oarg{file}\marg{thecounter} \\
\cmd{\changeglossnumformat}\oarg{file}\marg{format} \\
\end{syntax}
\glossary(changeglossactual)%
  {\cs{changeglossactual}\oarg{file}\marg{char}}%
  {Specifies \meta{char} as the \texttt{actual} character for 
   glossary \meta{file}.}%
\glossary(changeglossref)%
  {\cs{changeglossref}\oarg{file}\marg{thecounter}}%
  {Specifies \meta{thecounter} as the \meta{ref} for 
   glossary \meta{file}.}%
\glossary(changeglossnum)%
  {\cs{changeglossnum}\oarg{file}\marg{thecounter}}%
  {Specifies \meta{thecounter} as the \meta{num} for 
   glossary \meta{file}.}%
\glossary(changeglossnumformat)%
  {\cs{changeglossnumformat}\oarg{file}\marg{format}}%
  {Specifies \meta{format} as the format for \meta{num} for 
   glossary \meta{file}.}%

\cmd{\changeglossactual} sets \meta{char} as the \texttt{actual} character
for the \meta{file} glossary. It is initially \verb+@+. This must match 
with the \texttt{actual} specified for the \file{gst} file that will 
be applied.

\cmd{\changeglossref} specifies that \meta{thecounter} should be used
to generate the \meta{ref} for the \meta{file} glossary. It is
initially nothing.

\cmd{\changeglossnum} specifies that \meta{thecounter} should be used
to generate the \meta{num} for the \meta{file} glossary. It is
initially \cs{thepage}.

\cmd{\changeglossnumformat} specifies that \meta{format} should be used
to format the \meta{num} for the \meta{file} glossary. The format
of \meta{format} is \verb?|form?, where \verb?|? is the \texttt{encap}
character specified in the \file{gst} file, and \verb?form? is a
formatting command, taking one argument (the number), without any backslash. 
For example
\begin{lcode}
\changeglossnumformat{|textbf}
\end{lcode}
 to get bold numbers. It is
initially set as \verb?|memjustarg?, where this is defined as:
\begin{lcode}
\newcommand{\memjustarg}[1]{#1}
\end{lcode}
There must be a format defined for the \meta{num} otherwise the arguments
to \cmd{\glossitem} will not be set correctly.


    The \cmd{\makeglossary} command uses the \cs{change...}
commands to define the initial versions, so only use the \cs{change...}
macros \emph{after} \cmd{\makeglossary}.
In this document an early version of the glossary was set up by
\begin{lcode}
\makeglossary
\changeglossactual{?}
\makeatletter 
\changeglossnum{\@currentlabel} 
\makeatother
\changeglossnum{\thepage}
\end{lcode}
The first call of \cmd{\changeglossnum} makes the number the current 
numbered chapter, or numbered section, or numbered \ldots. I didn't 
like that when I tried it, so the second call resets the number to 
the page number.

\subsubsection{The listing}

    The final glossary data in the \file{gls} file is typeset in the
\Ie{theglossary} environment, which is much like the \Ie{theindex}
and \Ie{thebibliography} environments.

    The environment starts off with a chapter-style unnumbered title.
There are several macros for specifying what happens after that.

\begin{syntax}
\cmd{\glossaryname} \\
\cmd{\glossarymark} \\
\cmd{\glossaryintoc} \cmd{\noglossaryintoc} \\
\end{syntax}
\glossary(glossaryname)%
  {\cs{glossaryname}}%
  {Name for a glossary.}%
\glossary(glossarymark)%
  {\cs{glossarymark}}%
  {Redefine to specify marks for headers.}%
\glossary(glossaryintoc)%
  {\cs{glossaryintoc}}%
  {Declaration to add glossary title to the ToC.}%
\glossary(noglossaryintoc)%
  {\cs{noglossaryintoc}}%
  {Declaration to prohibit adding glossary title to the ToC.}%

The title for the glossary is \cmd{\glossaryname} whose initial definition
is 
\begin{lcode}
\newcommand*{\glossaryname}{Glossary}
\end{lcode}
\cmd{\glossarymark}, which by default does nothing, can be redefined to
set marks for headers. The glossary title will be added to the ToC
if the \cmd{\glossaryintoc} declaration is in force, but will not be
added to the ToC following the \cmd{\noglossaryintoc}.

\begin{syntax}
\cmd{\preglossaryhook} \\
\end{syntax}
\glossary(preglossaryhook)%
  {\cs{preglossaryhook}}%
  {Vacuous macro called after a glossary title is typeset.}
The macro \cmd{\preglossaryhook} is called after the glossary title 
has been typeset. By default it does nothing, but you could redefine
it to, for example, add some explanatory material before the entries
start.

\begin{syntax}
\cmd{\onecolglossary} \cmd{\twocolglossary} \\
\lnc{\glossarycolsep} \lnc{\glossaryrule} \\
\end{syntax}
\glossary(onecolglossary)%
  {\cs{onecolglossary}}%
  {Declaration for a single column glossary.}%
\glossary(onecolglossaryfalse)%
  {\cs{twocolglossary}}%
  {Declaration for a two column glossary.}%
\glossary(glossarycolsep)%
  {\cs{glossarycolsep}}%
  {Columns separation in a two column glossary.}%
\glossary(glossaryrule)%
  {\cs{glossaryrule}}%
  {Width of inter-column rule in a two column glossary.}%

The glossary can be typeset in two columns (\cmd{\twocolglossary})
but by default (\cmd{\onecolglossary}) it is set in one column.
When two columns are used, the length \lnc{\glossarycolsep} is the
distance between the columns and the length \lnc{\glossaryrule} is
the width (default 0) of a vertical rule between the columns.

\begin{syntax}
\cmd{\begintheglossaryhook} \\
\cmd{\atendtheglossaryhook} \\
\end{syntax}
\glossary(begintheglossaryhook)%
  {\cs{begintheglossaryhook}}%
  {Vacuous macro called as the last thing by \senv{theglossary}.}
\glossary(atendtheglossaryhook)%
  {\cs{atendtheglossaryhook}}%
  {Vacuous macro called as the first thing by \eenv{theglossary}.}

The last thing that \senv{theglossary} does is call 
\cmd{\begintheglossaryhook}. Similarly, the first thing that is done at
the end of the environment is to call \cmd{\atendtheglossaryhook}.
By default these macros do nothing but you can redefine them.

    For example, if you wanted the glossary in the form of a description
list, the following will do that.
\begin{lcode}
\renewcommand*{\begintheglossaryhook}{\begin{description}}
\renewcommand*{\atendtheglossaryhook}{\end{description}}
\renewcommand{\glossitem}[4]{\item[#1:] #2 #3 #4}
\end{lcode}

\subsubsection{The glossary for this document}

    The following is the code I have used to produce the glossary
in this document.

This is the code in the \file{sty} file that I used.
\begin{lcode}
\makeglossary
\changeglossactual{?}
\changeglossnum{\thepage}
\changeglossnumformat{|hyperpage}%% for hyperlinks
\renewcommand*{\glossaryname}{Command summary}
\renewcommand*{\glossarymark}{\markboth{\glossaryname}{\glossaryname}}
\renewcommand{\glossitem}[4]{%
  \sbox\@tempboxa{#1 \space #2 #3 \makebox[2em]{#4}}%
  \par\hangindent 2em
  \ifdim\wd\@tempboxa<0.8\linewidth
    #1 \space #2 #3 \dotfill \makebox[2em][r]{#4}\relax
  \else
    #1 \dotfill \makebox[2em][r]{#4}\\
    #2 #3
  \fi}
\end{lcode}

    The redefinition of \cmd{\glossitem} works as follows (it is similar
to code used in the setting of a \cmd{\caption}):
\begin{enumerate}
\item Put the whole entry into a temporary box.
\item Set up a hanging paragraph with 2em indentation after the first line.
\item Check if the length of the entry is less than 80\% of the linewidth.
\item For a short entry set the name, description, and any reference then
      fill the remainder of the line with dots with the number at the right
      margin.
\item For a longer entry, set the title and number on a line, separated
      by a line of dots, then set the description (and reference) on
      the following lines.
\end{enumerate}

    The \file{gst} file I have used for this document has a few more items
than the basic one.
\begin{lcode}
%%% memman.gst makindex glossary style file for memman and friends
%%% Output style parameters
preamble "\\begin{theglossary}"
postamble "\n\\end{theglossary}\n"
group_skip "\n\\glossaryspace\n"
item_0    "\n\\glossitem"
delim_0   "{\\memglonum{"
encap_suffix "}}}"
indent_space "\t"
indent_length 2
%%% Input style parameters
keyword "\\glossaryentry"
actual '?'
page_compositor "."
\end{lcode}

The \verb?group_skip? line asks that \verb?\glossaryspace? be put between the 
last entry for one letter and the first for the next letter. 
The \verb?indent_space? and \verb?indent_length? give a smaller indent for
continuation lines in the output than the default.

    The \verb?actual? entry says that the input file will use \verb+?+ instead
of the default \verb+@+ as the flag for separating a key from the start of 
the real entry. The \verb?page_compositor? indicates that any compound numbers
will be like \verb?1.2.3? instead of the default \verb?1-2-3?.

In the document the raw data is collected by the \cmd{\glossary} commands 
in the body of the text. For instance, although I have not actually used
the first two:
\begin{lcode}
\glossary(cs)%
  {\cs{cs}\marg{name}}%
  {Typesets \texttt{name} as a macro name with preceding backslash,
   e.g., \cs{name}.}%
\glossary(gmarg)%
  {\cs{gmarg}\marg{arg}}%
  {Typesets \texttt{arg} as a required argument, e.g., \marg{arg}.}
\glossary(glossaryname)%
  {\cs{glossaryname}}%
  {Name for a glossary}%
\glossary(memgloterm)%
  {\cs{memgloterm}\marg{term}}%
  {Wrapper round a glossary term.}%
\end{lcode}

    Any change to the glossary entries will be reflected in the
\file{glo} produced from that LaTeX run. \Lmakeindex\ has to be run
the \file{glo} file using the appropriate \file{gst} configuration file, 
and then LaTeX run again to get the corrected, sorted and formatted result
printed by \cmd{\printglossary}.

    In particular, for this document, which also includes an index so that
can be processed when the glossary is processed.
\begin{lcode}
pdflatex memman
makeindex -s memman.gst -o memman.gls memman.glo
makeindex -s memman.ist memman     %%% for the index
makeindex lines                    %%% for the index of first lines
pdflatex memman
\end{lcode} 


%%%%%%%%%%%%%%%%%%%%%%%%%%%%%%%%%%%%%%%%%%%%%%%%%%%%

\LMnote{2013/05/02}{Section about endnotes moved page-notes.tex}



%#% extend
%#% extstart include miscellaneous.tex
