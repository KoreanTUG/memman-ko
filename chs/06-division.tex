\chapter{Document divisions}

For this chapter the \cstyle{pedersen} chapterstyle has been used in
order to demonstrate how it appears.


In this chapter I first discuss the various kinds of divisions within
a book and the commands for typesetting these.

After that I describe the class methods for modifying the appearance
of the chapter and other sectional titles (subheads)\index{subhead}.
The facilities described here provide roughly the same as you would
get if you used the \Lpack{titlesec}~\cite{TITLESEC} and
\Lpack{sectsty}~\cite{SECTSTY} packages together; the commands are
different, though.

\section{Logical divisions}

    As described earlier there are three main logical divisions to a book;
the \pixfrontmatter, \pixmainmatter\ and \pixbackmatter. There are three \ltx\
commands that correspond to these, namely \cmd{\frontmatter},
\cmd{\mainmatter} and \cmd{\backmatter}.

\begin{syntax}
\cmd{\frontmatter} \cmd{\frontmatter*} \\
\end{syntax}
\glossary(frontmatter)%
  {\cs{frontmatter}}%
  {Sets folios to be printed in lowercase roman and prohibits sectional
   number.}
\glossary(frontmatter*)%
  {\cs{frontmatter*}}%
  {Same as \cs{frontmatter} except that folios are unaltered.}
The \cmd{\frontmatter} declaration sets the folios\index{folio} to be 
printed in lowercase roman numerals, starts the page numbering from~i, 
and prohibits any numbering
of sectional divisions. Caption\index{caption}, equations, etc., will be 
numbered continuously.  The starred version of the command,
\cmd{\frontmatter*}, is similar to the unstarred version except that it
makes no changes to the page numbering or the print style for the 
folios\index{folio}. Even though \cmd{\chapter} and other divisions will not
be numbered their titles will be added to the ToC.

    If it is to be used at all, the \cmd{\frontmatter} declaration should
come before any text is set, otherwise the pagination scheme will be
in disarray (in books pagination starts on the first page).

\begin{syntax}
\cmd{\mainmatter} \cmd{\mainmatter*} \\
\end{syntax}
\glossary(mainmatter)%
  {\cs{mainmatter}}%
  {Sets folio numbers to arabic, starting with 1. Floats, etc., will be
   numbered per chapter and sectional divisions will be numbered.}
\glossary(mainmatter*)%
  {\cs{mainmatter*}}%
  {Same as \cs{mainmatter} except that folios are unaltered.}
The \cmd{\mainmatter} declaration, which is the default at the start of a 
document, sets the folios\index{folio} to be printed in arabic numerals, starts 
the page numbering from~1, and sections and above will be numbered. 
Float\index{float} captions\index{caption}, equations, etc., will be numbered 
per chapter\index{chapter}. The starred version of the command,
\cmd{\mainmatter*}, is similar to the unstarred version except that it
makes no changes to the page numbering or the print style for the 
folios\index{folio}.

\LMnote{2010/01/23}{Added the following paragraph, just stating what
  \cs{mainmatter} does, was not clear before.}
Please note that \cmd{\mainmatter} will not only change the folio
numbers to arabic and restart it at 1, it will also make sure it
starts at the next coming recto page. (Even when running under the
\texttt{openany} option).



\begin{syntax}
\cmd{\backmatter} \\
\end{syntax}
\glossary(backmatter)%
  {\cs{backmatter}}%
  {Prohibits sectional numbering and floats, etc., will be numbered continuously.}
The \cmd{\backmatter} declaration makes no change to the pagination or 
folios\index{folio} but does prohibit sectional division numbering, and 
captions\index{caption}, etc., will be numbered continuously.

\fancybreak{}

\LMnote{2009/06/29}{Added the following section}

If you have other types of floats that might be used in the front-
main- or backmatter, then you can change some internals to add these
to be numbered in the same manner as we do with figures and
tables. They are defined as 
\begin{lcode}
\newcommand\@memfront@floats{%
  \counterwithout{figure}{chapter}
  \counterwithout{table}{chapter}}
\newcommand\@memmain@floats{% 
   \counterwithin{figure}{chapter}
   \counterwithin{table}{chapter}}
\newcommand\@memback@floats{%
    \counterwithout{figure}{chapter}
    \counterwithout{table}{chapter}
    \setcounter{figure}{0}
    \setcounter{table}{0}}
\end{lcode}
%
The macros can also be changed in case you want to have consecutive
figure numbering throughout, i.e.,
\begin{lcode}
\makeatletter
\counterwithout{figure}{chapter}
\counterwithout{table}{chapter}
\renewcommand\@memfront@floats{}
\renewcommand\@memmain@floats{}
\newcommand\@memback@floats{}
\makeatother
\end{lcode}
in the preamble.




\section{Sectional divisions}

    The \theclass{} class lets you divide a document up into eight levels
of named divisions. They range from book\index{book}, part\index{part} 
through chapter\index{chapter} and down to 
sub-paragraph. A particular sectional division is specified by one of
the commands \cmd{\book}, \cmd{\part}, \cmd{\chapter}, \cmd{\section}, 
\cmd{\subsection},
which is probably as deep as you want to go. If you really need finer
divisions, they are
 \cmd{\subsubsection}, \cmd{\paragraph} and lastly \cmd{\subparagraph}.
The sectional commands, except for \cmd{\book} and \cmd{\part},
have the same form, so rather than describing 
each one in turn I will use \cmd{\section} as model for all but the
two exceptions.

\begin{syntax}
\cmd{\section}\oarg{toc-title}\oarg{head-title}\marg{title}\\
\cmd{\section*}\marg{title}\\
\end{syntax} 
\glossary(section)%
  {\cs{section}\oarg{toc-title}\oarg{head-title}\marg{title}}%
  {Typesets a section subhead \meta{title}, adding \meta{title} to the ToC and 
   possibly the running headers. If given \meta{toc-title}
   is used instead of \meta{title} for the ToC and running header. If given
   \meta{head-title} is used for a running header.}
\glossary(section*)%
  {\cs{section*}\marg{title}}%
  {Typesets an unnumbered section subhead \meta{title}. There are no ToC or
   running header entries.}

There are two forms of the command; 
the starred version is simpler, so I'll describe its 
effects first --- it just typesets \meta{title} in the document in the format
for that particular sectional division. Like the starred version, the plain
version also typesets \meta{title} in the document, but it may be numbered.
Diferent forms of the division title are available for the 
Table of Contents (\toc) and a running header\index{header}, as follows:
\begin{itemize}
\item No optional argument: \meta{title} is used for the division title,
      the ToC title and a page header title.
\item One optional argument: \meta{title} is used for the division title;
      \meta{toc-title} is used for the ToC title and a page header\index{header} title.
\item Two optional arguments: \meta{title} is used for the division title;
      \meta{toc-title} is used for the ToC title; \meta{head-title}
      is used for a page header\index{header} title.
\end{itemize}

A \cmd{\section} command restarts the numbering of any \cmd{\subsection}s
from one.
For most of the divisions the \meta{title} is put on the page where the command
was issued. The \cmd{\book}, \cmd{\part} and \cmd{\chapter} commands behave 
a little differently.

    The \cmd{\book} and \cmd{\part} commands are simpler and both behave 
in the same way.
\begin{syntax}
\cmd{\book}\marg{title} \\
\cmd{\part}\marg{title} \\
\end{syntax}
\glossary(book)%
  {\cs{book}\marg{title}}%
  {Typesets a numbered book \meta{title} on a page by itself, adding
   \meta{title} to the ToC.}
\glossary(part)%
  {\cs{part}\marg{title}}%
  {Typesets a numbered part \meta{title} on a page by itself, adding
   \meta{title} to the ToC.}
The \cmd{\book}\marg{title} command puts the book name (default 
\texttt{\bookname}), number and \meta{title} on a page by itself. The
numbering of books\index{book!number} has no effect on the numbering of 
\cmd{\part}s\index{part!number} or \cmd{\chapter}s\index{chapter!number}.
Similarly the \cmd{\part}\marg{title} command puts the part name 
(default \texttt{Part}), number and \meta{title} on a page by itself. 
The numbering of parts\index{part!number} has no effect on the numbering of 
\cmd{\chapter}s\index{chapter!number}.

    Later I'll give a list of LaTeX's default names, like \texttt{Part}.

\begin{syntax}
\cmd{\chapter}\oarg{toc-title}\oarg{head-title}\marg{title} \\
\cmd{\chapter*}\oarg{head-title}\marg{title} \\
\end{syntax}
\glossary(chapter)%
  {\cs{chapter}\oarg{toc-title}\oarg{head-title}\marg{title}}
  {Starts a new page and puts the chapter number and \meta{title} 
   at the top of the page, adding \meta{title} to the ToC and possibly
   the running headers. If given \meta{toc-title}
   is used instead of \meta{title} for the ToC and running header. If given
   \meta{head-title} is used for a running header.
   It restarts numbering of any subsidiary elements
   such as \cs{section} of floats.}
\glossary(chapter*)%
  {\cs{chapter*}\oarg{head-title}\marg{title}}
  {Starts a new page and puts an unnumbered chapter \meta{title} 
   at the top of the page. If given \meta{head-title} is used for a 
   running header.}

The \cmd{\chapter} command starts a new page and puts
the chapter name (default \texttt{Chapter}), number and \meta{title}
at the top of the page. It restarts the numbering of any \cmd{\section}s 
from one. If no optional arguments are specified, \meta{title}
is used as the \toc{} entry and for any page headings. If one optional
argument is specified this is \meta{toc-title} and is used for the
\toc{} entry and for page headings. If both optional arguments
are specified the \meta{head-title} is used for page headings.

The \cmd{\chapter*} command starts a new page and puts
\meta{title} at the top of the page. It makes no \toc{} entry, 
changes no numbers and by default changes no page headings.
If the optional \meta{head-title} argument is given, this is used
for page headings. Use of the optional argument has the side-effect
that the \Icn{secnumdepth} counter is set to \Icn{maxsecnumdepth} (see below
for an explanation of these).

    When the \Lopt{article} option is in effect, however, things are slightly
different. New chapters do not necessarily start on a new page.
The \cmd{\mainmatter} command
just turns on sectional numbering and starts arabic page numbering; the 
\cmd{\backmatter} command just turns off sectional numbering.
    The \cmd{\tableofcontents} command and friends, as well as any
other commands created via \cmd{\newlistof}, 
\emph{always}\footnote{This is a consequence of the internal
timing of macro calls.}
call \verb?thispagestyle{chapter}?. If you are using the 
\Lopt{article} option you
will probably want to ensure that the \pstyle{chapter} pagestyle is the
same as you normally use for the document.

    Unlike the standard classes the \meta{title} is typeset ragged right.
This means that if you need to force a linebreak in the \meta{title} you 
have to use \cmd{\newline} instead of the more usual \cmd{\\}. For instance
\begin{lcode}
\section{A broken\newline title}
\end{lcode}

    In the standard classes a \cmd{\section} or other 
subhead\index{subhead!near bottom of page}\index{subhead!moved}
that is too close to the bottom of a page is moved to the top of the
following page. If this happens and \cmd{\flushbottom} is in effect, the
contents of the short page are stretched to make the last line flush with
the bottom of the typeblock.
\begin{syntax}
\cmd{\raggedbottomsection} \\
\cmd{\normalbottomsection} \\
\lnc{\bottomsectionskip} \\
\cmd{\bottomsectionpenalty}
\end{syntax}
\glossary(raggedbottomsection)%
  {\cs{raggedbottomsection}}%
  {Pages will be typeset short because of a moved subhead as if
   \cs{raggedbottom} was in effect.}
\glossary(normalbottomsection)%
  {\cs{normalbottomsection}}%
  {Cancels any previous \cs{raggedbottomsection}.}
\glossary(bottomsectionskip)%
  {\cs{bottomsectionskip}}%
  {Amount of stretch on a \cs{raggedbottomsection} short page.}
\glossary(bottomsectionpenalty)%
  {\cs{bottomsectionpenalty}}%
  {Penalty on a \cs{raggedbottomsection} short page.}
The \cmd{\raggedbottomsection} declaration will typeset any pages that 
are short because of a moved subhead as though \cmd{\raggedbottom}
was in effect for the short page; other pages are not affected. The
length \lnc{\bottomsectionskip} controls the amount of stretch on the short
page. Setting it to zero allows the last line to be flush with the bottom
of the typeblock. The default setting of 10mm appears to remove any
stretch. \cmd{\bottomsectionpenalty} control the penalty it costs to
make a page break at this point. The detault is zero as the stretch is
usually enough, by setting it to
a negative integer one can be a bit more incouraging regarding a
possible page break.

The declaration \cmd{\normalbottomsection}, which is the default,
cancels any previous \cmd{\raggedbottomsection} declaration.

\subsection{Appendices} \label{sec:appendices}

Appendices\index{appendix} normally come after the main text and are 
often considered to be part
of the \cmd{\mainmatter} as they are normally numbered (the \cmd{\backmatter}
declaration turns off all sectional numbering).

\begin{syntax}
\cmd{\appendix} \\
\cmd{\appendixname} \\
\end{syntax}
\glossary(appendix)%
  {\cs{appendix}}%
  {Sets chapter numbering to alphabetic and sets the chapter name to 
   \cs{appendixname}.}
\glossary(appendixname)%
  {\cs{appendixname}}%
  {Name given to chapters in appendices (default Appendix).}
The \cmd{\appendix} declaration changes the numbering of
chapters\index{chapter!number} to an alphabetic form and also changes 
the names of chapters from \cmd{\chaptername} (default \texttt{Chapter})
to the value of \cmd{\appendixname} (default \texttt{Appendix})\index{appendix}. 
Thus, the first and any subsequent \cmd{\chapter}s after 
the \cmd{\appendix} command
will be `Appendix A \ldots', `Appendix B \ldots', and so on. 
That is as far as the standard classes go but this class provides more ways
of dealing with appendices\index{appendix}.

\begin{syntax}
\cmd{\appendixpage}\\
\cmd{\appendixpage*}\\
\cmd{\appendixpagename}\\
\end{syntax}
\glossary(appendixpage)
  {\cs{appendixpage}}%
  {Creates an unnumbered anonymous  part-like page with the title \cs{appendixpagename} and adds it to the ToC.}
\glossary(appendixpage*)
  {\cs{appendixpage*}}%
  {Like \cs{appendixpage} but makes no ToC entry.}
\glossary(appendixpagename)
  {\cs{appendixpagename}}%
  {Title used for an \cs{appendixpage} (default \texttt{Appendices}.}
The \cmd{\appendixpage} command generates a part-like page (but no name 
or number) with the title given by the value of \cmd{\appendixpagename} 
(default \texttt{Appendices})\index{appendix}. It also makes an entry in 
the \toc{} using 
\cmd{\addappheadtotoc} (see below). The starred version generates the 
appendix\index{appendix} page but makes no \toc{} entry.

\begin{syntax}
\cmd{\addappheadtotoc}\\
\cmd{\appendixtocname}\\
\end{syntax}
\glossary(addappheadtotoc)%
  {\cs{addappheadtotoc}}%
  {Adds ToC entry with the title \cs{appendixtocname}.}
\glossary(appendixtocname)%
  {\cs{appendixtocname}}%
  {Title of ToC entry added by \cs{addappheadtotoc} (default \texttt{Appendices}.}
The command \cmd{\addappheadtotoc} adds an entry to the \toc. The title is
given by the value of \cmd{\appendixtocname} (default \texttt{Appendices})\index{appendix}.

\begin{syntax}
\senv{appendices} text \eenv{appendices}\\
\end{syntax}
\glossary(appendices)%
  {\senv{appendices}}%
  {An environment form of \cs{appendix}; chapters are restored to their condition
   (including numbering) after the environment ends.}
The \Ie{appendices} environment acts like the \cmd{\appendix}\index{appendix} 
command in that it resets the numbering and naming of chapters\index{chapter}. 
However, at the end of the environment, chapters are restored to their original 
condition and any chapter numbers continue in sequence as though the 
\Ie{appendices} environment had never been there.

\LMnote{2010/11/18}{Wrong names for \cs{namedsubappendices} and
  \cs{unnamedsubappendices} fixed}
\begin{syntax}
\senv{subappendices} text \eenv{subappendices} \\
\cmd{\namedsubappendices} \cmd{\unnamedsubappendices} \\
\end{syntax}
\glossary(subappendices)%
  {\senv{subappendices}}%
  {Like the \Pe{appendices} environment but used at the end of a chapter
   for per-chapter sub-appendices.}
\glossary(namedsubappendices)%
  {\cs{namedsubappendices}}%
  {Precede sub-appendix numbers with the name \cs{appendixname}.}
\glossary(unnamedsubappendices)%
  {\cs{unnamedsubappendices}}%
  {Do not precede the sub-appendix number with any name (the default).}
The \Ie{subappendices}\index{appendix!subappendix} environment can be used to put appendices\index{appendix} at the end
of a chapter\index{chapter}. Within the environment \cmd{\section} starts a 
new sub-appendix\index{appendix}. You may put \cmd{\addappheadtotoc} at the start
of the environment if you want a heading entry in the \toc.
If you put the declaration \cmd{\namedsubappendices}
\emph{before} the \Ie{subappendices} environment, the sub-appendix number
in the body of the document will be preceded by the value of 
\cmd{\appendixname}. The \cmd{\unnamedsubappendices} declaration, which is the 
default, may be used to switch off this behaviour.

\LMnote{2010/11/19}{Added this caveat, following a question on ctt}
\textbf{Caveat:} The implementation of the named subappendices make
use of \cmd{\setsecnumformat}, thus if you have used this command to
change the formating of the section number you will need to re-do this
in a special manner inside the \Ie{subappendices}
environemt. Something like this (where a user wanted to use old style
numerials for sectioning numbers)
\begin{verbatim}
\begin{subappendices}
\setsecnumformat{\sectionname\ 
  \oldstylenums{\csname the#1\endcsname\quad}}
\end{verbatim}
The macro \cmd{sectionname} is a special macro that only lives inside
the \Ie{subappendices} environment and is only available when
\cmd{\namedsubappendices} is applied.



\section{Numbering} \label{sec:secnumbers}

    Each type of sectional division\index{division!sectional} has an 
associated \emph{level}\index{sectional division!level number} as shown 
in \tref{tab:seclevels}.
Divisions are numbered if the value of the \Icn{secnumdepth} counter
is equal to or greater than their level. For example, with
\begin{lcode}
\setcounter{secnumdepth}{2}
\end{lcode}
then subsections up to book\index{book!number} will be numbered.

\begin{table}
\centering
\caption{Division levels} \label{tab:seclevels}
\begin{tabular}{lr} \toprule
Division       & Level \\ \midrule
\cmd{\book}           & -2 \\
\cmd{\part}           & -1 \\
\cmd{\chapter}        & 0 \\
\cmd{\section}        & 1 \\
\cmd{\subsection}     & 2 \\
\cmd{\subsubsection}  & 3 \\
\cmd{\paragraph}      & 4 \\
\cmd{\subparagraph}   & 5 \\ \bottomrule
\end{tabular}
\end{table}

\begin{syntax}
\cmd{\setsecnumdepth}\marg{secname} \\
\cmd{\maxsecnumdepth}\marg{secname} \\
\end{syntax}
\glossary(setsecnumdepth)%
  {\cs{setsecnumdepth}\marg{secname}}%
  {Sets division numbering level to \meta{secname}.}
\glossary(maxscnumdepth)%
  {\cs{maxsecnumdepth}\marg{secname}}%
  {Sets division numbering level in the \cs{mainmatter} to \meta{secname}.}
Instead of having to remember the levels if you want to change what
gets numbered you can use the \cmd{\setsecnumdepth} command. It
sets \Icn{secnumdepth} so that divisions \meta{secname} and above
will be numbered. The argument \meta{secname} is the name of a sectional
division without the backslash. For example, to have subsections
and above numbered:
\begin{lcode}
\setsecnumdepth{subsection}
\end{lcode}
You can also use \texttt{all} or \texttt{none} for \meta{secname} which
will either turn on numbering for all levels, or turn off numbering
altogether.

When used in the preamble\index{preamble} \cmd{\setsecnumdepth} also
calls \cmd{\maxsecnumdepth}, which is the numbering level used once
\cmd{\mainmatter} is called. You can use \cmd{\setsecnumdepth} anywhere
in the \cmd{\mainmatter} to (temporarily) change the numbering level.

    By default, the class sets:
\begin{lcode}
\setsecnumdepth{section}
\maxsecnumdepth{section}
\end{lcode}
The \cmd{\frontmatter} commands sets the numbering level to \texttt{none}.
The commands \cmd{\mainmatter} and \cmd{\mainmatter*} set the
numbering level to the value specified by \cmd{\maxsecnumdepth}.

    The number setting commands come from the \Lpack{tocvsec2}
package~\cite{TOCVSEC2}.

\section{Book and part headings}

\index{heading!book|(}
\index{heading!part|(}

    Book and part headings \emph{always} start on a new page with the 
\pstyle{book} and \pstyle{part} pagestyles, respectively. The typical
book and part heading consists of the name (e.g., `Book' or `Part') 
followed by a number represented as an uppercase Roman numeral. There is 
a vertical space after which the title is printed. Finally a new page is
started.

\PWnote{2009/07/30}{Updated description, etc., for Part and Book pages}
    Several aspects of the typesetting of the \cmd{\book} and \cmd{\part} 
title are configurable. Ignoring details, such as the optional argument,
the code for printing \cmd{\part} headings looks like this:
\begin{lcode}
\newcommand{\part}[1]{%    % THIS IS A VERY SIMPLIFIED VERSION
  \cleardoublepage         % start a new recto page                    
  \thispagestyle{part}     % set the page style
  \beforepartskip          % space before Name and Number
  \printpartname\partnamenum\printpartnum
  \midpartskip             % space after Name and Number
  \printparttitle{#1}      % print the title
  \partpageend}            % finish off
\newcommand{\partpageend}{%  THIS IS SIMPLIFIED
  \afterpartskip
  % ifblankpage then blank next page and restore twocolumn if necessary
  }
\end{lcode}
The code for \cmd{\book} headings is similar.

    The general layout for \cmd{\book}, \cmd{\part} and \cmd{\chapter} headings
is similar and you may wish to refer to \fref{lay:chap} which, although
it shows the vertical layout for a \cmd{chapter} head, is also applicable
to \cmd{\book} and \cmd{\part} heads with appropriate changes in the
names of the commands.


\begin{syntax}
\cmd{\beforebookskip} \cmd{\afterbookskip} \\
\cmd{\beforepartskip} \cmd{\afterpartskip} \\
\end{syntax}
\glossary(beforebookskip)%
  {\cs{beforebookskip}}%
  {Spacing above a \cs{book} title.}
\glossary(afterbookskip)%
  {\cs{afterbookskip}}%
  {Spacing below a \cs{book} title.}
\glossary(beforepartskip)%
  {\cs{beforepartskip}}%
  {Spacing above a \cs{part} title.}
\glossary(afterpartskip)%
  {\cs{afterbookskip}}%
  {Spacing below a \cs{part} title.}
These commands effectively control the spacing before and after the book
and  part titles. Their default definitions are:
\begin{lcode}
\newcommand*{\beforebookskip}{\null\vfil}
\newcommand*{\afterbookskip}{\vfil\newpage}
\newcommand*{\beforepartskip}{\null\vfil}
\newcommand*{\afterpartskip}{\vfil\newpage}
\end{lcode}
Together, these vertically center any typesetting on the page, and then start
a new page. To move the \cs{part} title upwards on the page, for example, 
you could do:
\begin{lcode}
\renewcommand*{\beforepartskip}{\null\vskip 0pt plus 0.3fil}
\renewcommand*{\afterpartskip}{\vskip 0pt plus 0.7fil \newpage}
\end{lcode} 

\begin{syntax}
\cmd{\midbookskip} \\
\cmd{\midpartskip} \\
\end{syntax}
\glossary(midbookskip)%
  {\cs{midbookskip}}%
  {Spacing between a \cs{book}'s number line and the title.}
\glossary(midpartskip)%
  {\cs{midpartskip}}%
  {Spacing between a \cs{part}'s number line and the title.}
The macros \cmd{\midbookskip} and \cmd{\midpartskip} are the spacings
between the number lines and the titles. The default definitions are:
\begin{lcode}
\newcommand{\midbookskip}{\par\vspace 2\onelineskip}
\newcommand{\midpartskip}{\par\vspace 2\onelineskip}
\end{lcode}
and they can be changed.

\begin{syntax}
\cmd{\printbookname} \cmd{\booknamefont} \\
\cmd{\booknamenum} \\
\cmd{\printbooknum} \cmd{\booknumfont} \\
\cmd{\printpartname} \cmd{\partnamefont} \\
\cmd{\partnamenum} \\
\cmd{\printpartnum} \cmd{\partnumfont} \\
\end{syntax}
\glossary(printbookname)%
  {\cs{printbookname}}
  {Prints the book name (\cs{bookname}) using the \cs{booknamefont}.}
\glossary(booknamefont)%
  {\cs{booknamefont}}%
  {Font used by \cs{printbookname} for the book name.}
\glossary(booknamenum)%
  {\cs{booknamenum}}%
  {Called between printing a book name and number.}
\glossary(printbooknum)%
  {\cs{printbooknum}}%
  {Prints a book number using the \cs{booknumfont}.}
\glossary(booknumfont)%
  {\cs{booknumfont}}%
  {Font used by \cs{printbooknum} for the book number.}

\glossary(printpartname)%
  {\cs{printpartname}}
  {Prints the part name (\cs{partname}) using the \cs{partnamefont}.}
\glossary(partnamefont)%
  {\cs{partnamefont}}%
  {Font used by \cs{printpartname} for the part name.}
\glossary(partnamenum)%
  {\cs{partnamenum}}%
  {Called between printing a part name and number.}
\glossary(printpartnum)%
  {\cs{printpartnum}}%
  {Prints a part number using the \cs{partnumfont}.}
\glossary(partnumfont)%
  {\cs{partnumfont}}%
  {Font used by \cs{printpartnum} for the part number.}
The macro \cmd{\printbookname} typesets the book name (the value of 
\cmd{\bookname}) using the font specified by \cmd{\booknamefont}. 
The default is the \cmd{\bfseries} font in the \cmd{\huge} size. 
Likewise the book number is typeset by \cmd{\printbooknum}
using the font specified by \cmd{\booknumfont}, which has the same default as
\cmd{\booknamefont}. The macro \cmd{\booknamenum}, which is defined to be 
a space, is called between printing the book name and the number. All 
these can be changed to obtain different effects. 

    Similarly, the macro \cmd{\printpartname} typesets the part name 
(the value of \cmd{\partname}) using the font specified by 
\cmd{\partnamefont}. The default is the \cmd{\bfseries} font in
the \cmd{\huge} size. Likewise the part number is typeset by 
\cmd{\printpartnum} using the font specified by \cmd{\partnumfont}, 
which has the same default as \cmd{\partnamefont}. The macro 
\cmd{\partnamenum}, which is defined to be a space, is called between printing
the part name and the number. 

    For example, to set a \cmd{\part} in a large sans font with the part 
name flush left:
\begin{lcode}
\renewcommand{\partnamefont}{\normalfont\huge\sffamily\raggedright}
\renewcommand{\partnumfont}{\normalfont\huge\sffamily}
\end{lcode}
or to only print the part number in the default font:
\begin{lcode}
\renewcommand{\printpartname}{}
\renewcommand{\partnamenum}{}
\end{lcode}


\begin{syntax}
\cmd{\printbooktitle}\marg{title} \cmd{\booktitlefont} \\
\cmd{\printparttitle}\marg{title} \cmd{\parttitlefont} \\
\end{syntax}
\glossary(printbooktitle)%
  {\cs{printbooktitle}}%
  {Prints the book title using the \cs{booktitlefont}.}
\glossary(booktitlefont)%
  {\cs{booktitlefont}}%
  {Font used by \cs{printbooktitle} for the title.}
\glossary(printbooktitle)%
  {\cs{printbooktitle}}%
  {Prints the book title using the \cs{booktitlefont}.}
\glossary(booktitlefont)%
  {\cs{booktitlefont}}%
  {Font used by \cs{printbooktitle} for the title.}
\glossary(printparttitle)%
  {\cs{printparttitle}}%
  {Prints the part title using the \cs{parttitlefont}.}
\glossary(parttitlefont)%
  {\cs{parttitlefont}}%
  {Font used by \cs{printparttitle} for the title.}
A book's title is typeset by \cmd{\printbooktitle} using the font specified 
by \cmd{\booktitlefont}. 
By default this is a \cmd{\bfseries} font in the \cmd{\Huge} size. This can
be changed to have, say, the title set raggedleft in a small caps font by
\begin{lcode}
\renewcommand{\booktitlefont}{\normalfont\Huge\scshape\raggedleft}
\end{lcode}

    Similarly a part's title is typeset by \cmd{\printparttitle} using 
the font specified by \cmd{\parttitlefont}. 
By default this is a \cmd{\bfseries} font in the \cmd{\Huge} size. 

    The \cmd{\parttitlefont} font is also used by 
\cmd{\appendixpage}, or its starred version, when
typesetting an appendix\index{appendix} page.

\begin{syntax}
\cmd{\bookpagemark}\marg{title} \\
\cmd{\partmark}\marg{title} \\
\end{syntax}
\glossary(bookpagemark)%
  {\cs{bookpagemark}\marg{title}}%
  {For setting any marks with the title from a \cs{book} for a running header.}
\glossary(partmark)%
  {\cs{partmark}\marg{title}}%
  {For setting any marks with the title from a \cs{part} for a running header.}
The \cmd{\book} code includes \cmd{\bookpagemark}\marg{title} for capturing
the \meta{title} of the book division if it is going to be used, 
for example, in page headers. Its definition is simply:
\begin{lcode}
\newcommand*{\bookpagemark}[1]{}
\end{lcode}
There is the corresponding \cmd{\partmark} for the title of \cmd{\part} 
divisions.

\begin{syntax}
\cmd{\bookpageend} \cmd{\bookblankpage} \cmd{\nobookblankpage} \\
\cmd{\partpageend} \cmd{\partblankpage} \cmd{\nopartblankpage} \\
\end{syntax}
\glossary(bookpageend)%
  {\cs{bookpageend}}%
  {Code to finish off typesetting a book title page.}
\glossary(bookblankpage)%
  {\cs{bookblankpage}}%
  {Follow a book title page with a blank page (the default).}
\glossary(nobookblankpage)%
  {\cs{nobookblankpage}}%
  {Do not follow a book title page with a blank page.}
\glossary(partpageend)%
  {\cs{partpageend}}%
  {Code to finish off typesetting a part title page.}
\glossary(partlankpage)%
  {\cs{partblankpage}}%
  {Follow a part title page with a blank page (the default).}
\glossary(nopartblankpage)%
  {\cs{nopartblankpage}}%
  {Do not follow a part title page with a blank page.}

The macro \cmd{\bookpageend} finishes off a book title page. It first
calls \cmd{\afterbookskip}. If the \cmd{\nobookblankpage} is in effect
it does nothing more. If the declaration \cmd{\bookblankpage} (the default)
is in effect then it finishes the current page, outputs a blank page and then,
if twocolumn typesetting was in effect before \cmd{\book} then it restores
twocolumn typesetting. The macro \cmd{\partpageend} performs similar
functions for \cmd{\part} pages.

\LMnote{2011/02/06}{This now uncommented text is wrong!}
%     If, for example, you wish to put something on a \cmd{\part} page, and maybe
% something on the following normally blank page, then this is a possible way
% of doing it:\index{text on a part page}
So to add something on the back side of a \cmd{\part} page (assuming
twoside) use something similar to
\begin{lcode}
...
\nopartblankpage
\part{Title of the Part}
\thispagestyle{simple}
Text on the following (normally blank page)
\clearpage
...
\end{lcode}
    Alternatively you can redefine \cmd{\partpageend}.
 
   If you use the declaration \cmd{\nopartblankpage} (or \cmd{\nobookblankpage})
then you are responsible for setting everything correctly to end off the
\cmd{\part} (or \cmd{\book}) page. This is the default definition of 
\cmd{\partpageend} (that for \cmd{\bookpageend} is similar):
\begin{lcode}
\newcommand{\partpageend}{%
  \afterpartskip
  \ifm@mnopartnewpage%   set by \(no)partblankpage
  \else%                 default finish off
    \if@twoside
      \if@openright%     output blank page
        \null
        \thispagestyle{afterpart}%
        \newpage
      \fi
    \fi
  \fi
  \if@tempswa%   true if twocolumn was being used
    \twocolumn
  \fi}
\end{lcode}
Here with the default definitions, \cmd{\afterpartskip} ends off the
\cmd{\part} page, and then the rest of the code in \cmd{\partpageend}
takes care of typesetting the blank back side of the \cmd{\part} page
(or send us back to twocolumn mode).

\LMnote{2011/02/06}{Added the example below}
If on the other hand we actually want to write something below the
part title on the \cmd{\part} page, then we need a different
route. The `air' above and below the part title is by default defined as
\begin{lcode}
  \newcommand*{\beforepartskip}{\null\vfil}
  \newcommand*{\afterpartskip}{\vfil\newpage}
\end{lcode}
Thus we need to redefined this such that it does not change the page
and such that it add useful spacing above and below the part
titling. Something like this may do the trick
\begin{lcode}
\makeatletter
  \newcommand*{\beforepartskip}{\null\vskip4cm}
  \newcommand*{\afterpartskip}{\par\vskip1cm%
    \@afterindentfalse\@afterheading} 
\makeatother
\end{lcode}



\index{heading!part|)}
\index{heading!book|)}

\subsection{Leadpage}

\begin{syntax}
\cmd{\newleadpage}\oarg{page-style}\marg{cmdname}\marg{title} \\
%\cmd{\newleadpage*}\oarg{page-style}\marg{cmdname}\marg{title} \\
\cmd{\renewleadpage}\oarg{page-style}\marg{cmdname}\marg{title} \\
%\cmd{\renewleadpage*}\oarg{page-style}\marg{cmdname}\marg{title} \\
\end{syntax}
\glossary(newleadpage)%
{\cs{newleadpage}\oarg{page-style}\marg{cmdname}\marg{title}}%
{Creates a macro to that when used will generate a sort of title
page. Very similar to \cs{appendixpage}.}
\glossary(renewleadpage)%
{\cs{renewleadpage}\oarg{page-style}\marg{cmdname}\marg{title}}%
{Redefines an existing new lead page macro.}
The \cmd{\newleadpage}\footnote{The suggestions for this came from
Danie Els\index{Els, Danie} and Lars Madsen\index{Madsen, Lars}.}
command defines a macro \verb?\cmdname? that when called 
will typeset an Appendixpage-like page (see \Sref{sec:appendices}) with a 
title \meta{title} using the \meta{page-style} as the pagestyle for
the page. The default is the \pstyle{empty} pagestyle. The macro
\cmd{\renewleadpage} redefines an existing leadpage command. 
\cmd{\cmdname} will add an entry to the TOC, the similarly defined
\cmd{\cmdname*} will not.


As an example:
\begin{lcode}
\newleadpage{plates}{Picture Gallery}
\end{lcode}
creates the new command \cs{plates} which when called generates an unnumbered
part-like page with the title \textbf{Picture Gallery}.

\begin{syntax}
\cmd{\leadpagetoclevel} \\
\end{syntax}
\glossary(leadpagetoclevel)%
{\cs{leadpagetoclevel}}%
{Holds the default toc level for \cs{newleadpage} generated macros,
default is \texttt{chapter}.}
When \cs{(re)newleadpage} is used the resulting command adds \meta{title}
to the \toc\ as though it was an unnumbered \cmd{\leadpagetoclevel} entry,
whose definition is
\begin{lcode}
\newcommand*{\leadpagetoclevel}{chapter}
\end{lcode}
If you wished them to be entered like a \cmd{\part} header then simply:
\begin{lcode}
\renewcommand*{\leadpagetoclevel}{part}
\end{lcode}


    The layout of the page matches that for unnumbered \cmd{\part} pages, and 
internally the resulting commands use \cmd{\partmark} in case you 
wish to capture the \meta{title} to use in running headers.


% 여기부터 6절 끝까지 박승원 담당
%\section{Chapter headings}
\section{장 제목}
\label{sec:chapter-headings}
\index{heading!chapter|(} %| emacs
\index{제목!장}

%    The chapter headings are configurable in much the same way as book or part 
%headings, but in addition there are some built in chapter styles that you may
%wish to try, or define your own.

장 제목은 book 이나 part 제목과 유사하게 조정할 수 있을 뿐만 아니라,
미리 정의되어 있는 장 제목 스타일이나 자신이 원하는 스타일로 설정할 수 있다.

%    Chapters, except when the \Lopt{article} class option is used,
% \emph{always} start on a new page with the \pstyle{chapter}
%pagestyle. The particular page, recto or verso, that they start on is
%mainly controlled by the class options. If the \Lopt{oneside} option is used
%they start on the next new page, but if the \Lopt{twoside} option is
%used the starting page may differ, as follows.
새로운 장은 \Lopt{article} 클래스 옵션이 쓰이지 않는 한,
\emph{항상} 새로운 페이지에서 \pstyle{chapter} 페이지 스타일로 시작한다.
새로운 장이 시작되는 페이지를 좌우 페이지 중 어느 쪽에 둘지는 클래스 옵션에 따라 설정된다.
\Lopt{oneside} 옵션을 사용할 경우 새로운 장은 바로 다음의 새 페이지에서 시작되지만,
\Lopt{twoside} 옵션을 사용할 경우 새로운 장이 시작하는 페이지가 다음과 같이 달라질 수 있다.

%\begin{itemize}
%\item[\Lopt{openright}] The chapter heading is typeset on the next recto page,
%  which may leave a blank verso leaf.
%\item[\Lopt{openleft}] The chapter heading is typeset on the next verso page,
%  which may leave a blank recto leaf.
%\item[\Lopt{openany}] The chapter heading is typeset on the next page and there
%  will be no blank leaf.
%\end{itemize}
\begin{itemize}
	\item[\Lopt{openright}] 새로운 장이 다음의 오른쪽(홀수쪽) 페이지에서 시작하며,
	경우에 따라 빈 왼쪽 페이지가 생길 수 있다.
	\item[\Lopt{openleft}] 새로운 장이 다음의 왼쪽(짝수쪽) 페이지에서 시작하며,
	경우에 따라 빈 오른쪽 페이지가 생길 수 있다.
	\item[\Lopt{openany}] 새로운 장이 바로 다음 페이지에서 시작하며,
	빈 페이지가 생기지 않는다.
\end{itemize}

\begin{syntax}
\cmd{\openright} \cmd{\openleft} \cmd{\openany} \\
\end{syntax}
\glossary(openright)%
  {\cs{openright}}%
  {새로운 장이 항상 오른쪽(홀수쪽) 페이지에서 시작하도록 강제.}
%  {Force chapters to start on a new recto page.}
\glossary(openleft)%
  {\cs{openleft}}%
  {새로운 장이 항상 왼쪽(짝수쪽) 페이지에서 시작하도록 강제.}
%  {Force chapters to start on a new verso page.}
\glossary(openany)%
  {\cs{openany}}%
  {새로운 장이 바로 다음 페이지에서 시작.}
%  {Allow chapters to start on the next new page.}
  
%These three declarations have the same effect as the options of the same name.
%They can be used anywhere in the document to change the chapter opening page.
이와 같은 세 명령어는 같은 이름을 가진 옵션과 동일하게 기능한다.
문서의 어느 곳에서나 장의 시작 페이지를 변경하기 위해 사용할 수 있다.

%Ignoring many details, like the optional arguments, the code for 
%printing a \cmd{\chapter} heading is similar to that for \cmd{\book} 
%and \cmd{\part} (the \cs{chapterhead} command below is \emph{not}
%part of the class).
선택적 옵션과 같은 세부적인 사항들을 제외하고는,
\cmd{\chapter} 제목을 출력하는 코드는 \cmd{\book} 이나 \cmd{\part} 과 유사하다
(아래의 \cs{chapterhead} 명령은 클래스의 일부가 \emph{아니다}).
\begin{lcode}
\newcommand{\chapterhead}[1]{ % THIS IS A SIMPLIFIED VERSION
  \clearforchapter        % move to correct page
  \thispagestyle{chapter} % set the page style
  \insertchapterspace     % Inserts space into LoF and LoT
  \chapterheadstart       % \beforechapskip space before heading
  \printchaptername\chapternamenum\printchapternum
  \afterchapternum        % \midchapskip space between number and title
  \printchaptertitle{#1}  % title
  \afterchaptertitle}     % \afterchapskip space after title
\end{lcode}
일반적인 레이아웃은 \fref{lay:chap}에 나와 있다.
%The general layout is shown in \fref{lay:chap}.

\begin{figure}
\setlayoutscale{1}
\centering
\chapterdiagram
\caption{
	장 제목을 위한 클래스 레이아웃 매개변수들.
	\cs{beforechapskip}을 사용하려면 생각해볼 내용이 더 있는데, 본문의 설명을 보면 된다.
%	Class layout parameters for chapter titles. Working with
%  \cs{beforechapskip} need a little thought, see the text.
} 
\label{lay:chap}
\end{figure}

\begin{syntax}
\cmd{\clearforchapter} \\
\end{syntax}
\glossary(clearforchapter)%
  {\cs{clearforchapter}}%
  {새로운 장이 시작하는 쪽을 선택하는 명령어.}
%  {Selects a chapter's opening page.}
새로운 장이 시작하는 페이지를 설정하는 매크로는 \cmd{\clearforchapter}다.
클래스 옵션 \texttt{openright}, \texttt{openleft}, \texttt{openany}
(또는 그들과 동등한 매크로인 \cmd{\openright}, \cmd{\openleft}, \cmd{\openany})
는 각각 \cmd{\clearforchapter} 이
\cmd{\cleartorecto}, \cmd{\cleartoverso}, \cmd{\clearpage} 이 되게끔 만든다.
(\S\ref{sec:moving} 장을 참조)
보다시피 \cmd{\clearforchapter}가 새로운 페이지를 시작하는 것 외의 기능을 하도록 만들 수 있다.
%The actual macro that sets the opening page for a chapter is 
%\cmd{\clearforchapter}. The class options \texttt{openright},
%\texttt{openleft} and \texttt{openany} (or their macro equivalents
%\cmd{\openright}, \cmd{\openleft} and
%\cmd{\openany}) define \cmd{\clearforchapter} to be respectively
%(see \S\ref{sec:moving}) 
%\cmd{\cleartorecto}, \cmd{\cleartoverso} and \cmd{\clearpage}. You can
%obviously change \cmd{\clearforchapter} to do more than just start a
%new page.

\begin{syntax}
\cmd{\memendofchapterhook} 
\end{syntax}
\glossary(memendofchapterhook)%
  {\cs{memendofchapterhook}}%
  {\cs{chapter} 명령어의 가장 마지막 부분에서 실행되는 hook.}
%  {Hook executed at the very end of the \cs{chapter} command.}
\cmd{\chapter} 명령의 가장 앞부분에는 \cmd{\clearforchapter}이,
가장 뒷부분에는 \cmd{\memendofchapterhook}이 따른다.
보통의 경우 아무 작업을 하지 않지만, 무언가를 삽입하기 위해 사용될 수 있다.
다음은 \cmd{\clearpage}를 삽입하는 예시이다:
%Where \cmd{\clearforchapter} comes at the very beginning,
%\cmd{\memendofchapterhook} comes at the very end of the \cmd{\chapter}
%command. It does nothing by default, but could be redefined to insert,
%say, \cmd{\clearpage}:
\begin{verbatim}
\makeatletter
\renewcommand\memendofchapterhook{%
  \clearpage\m@mindentafterchapter\@afterheading}
\makeatother
\end{verbatim}


%    Some books have the chapter headings on a verso page, with perhaps
%an illustration\index{illustration} or epigraph\index{epigraph}, and 
%then the text starts on the opposite
%recto page. The effect can be achieved like:
몇몇 책들은 새로운 장이 시작할 때 왼쪽 페이지에 일러스트\index{illustration}\tidx{일러스트}나
인용문\index{epigraph}\tidx{인용문}을 둔 뒤 오른쪽 페이지에서 본문을 시작하기도 한다.
이러한 디자인은 다음과 같이 제작할 수 있다:
\begin{lcode}
\openleft                % chapter title on verso page
\chapter{The title}      % chapter title
\begin{centering}        % include a centered illustration
\includegraphics{...}
\end{centering}
\clearpage               % go to recto page
Start of the text        % chapter body
\end{lcode}


\begin{syntax}
\cmd{\chapterheadstart} \lnc{\beforechapskip} \\ 
\cmd{\afterchapternum} \lnc{\midchapskip} \\
\cmd{\afterchaptertitle} \lnc{\afterchapskip} \\
\end{syntax}
\glossary(chapterheadstart)%
  {\cs{chapterheadstart}}%
  {장 헤딩의 시작에서 호출되며, 기본값으로는 \cs{beforechapskip} 공백을 삽입.}
%  {Called at start of printing a chapter header; by default inserts \cs{beforechapskip} space.}
\glossary(beforechapskip)%
  {\cs{beforechapskip}}%
  {장 이름과 번호 위의 공백.}
%  {Space above chapter name and number.}
\glossary(afterchapternum)%
  {\cs{afterchapternum}}%
  {장 헤딩 후에 호출되며, 기본값으로는 \cs{midchapskip} 공백을 삽입.}
%  {Called after printing a chapter number; by default inserts \cs{midchapskip} space.}
\glossary(midchapskip)%
  {\cs{midchapskip}}%
  {장 이름과 번호, 제목 사이의 공백.}
%  {Space between chapter name and number and the title.}
\glossary(afterchaptertitle)%
  {\cs{afterchaptertitle}}%
  {장 헤딩 후에 호출되며, 기본값으로는 \cs{afterchapskip} 공백을 삽입.}
%  {Called after printing a chapter title; by default inserts \cs{afterchapskip} space.}
\glossary(afterchapskip)%
  {\cs{afterchapskip}}%
  {장 제목 이후의 공백.}
%  {Space after chapter title.}

\cmd{\chapterheadstart} 매크로는 장 이름과 번호를 식자하기 직전에 호출된다.
기본값으로는 \lnc{\beforechapskip} 만큼의 공백을 삽입한다. (기본값 50pt)
%The macro \cmd{\chapterheadstart} is called just before printing a chapter
%name and number. By default it inserts \lnc{\beforechapskip} space (default
%50pt).

\cmd{\afterchapternum} 매크로는 장 번호를 식자한 직후에 호출된다.
기본값으로는 \lnc{\midchapskip} 만큼의 공백을 삽입한다. (기본값 20pt)
%The macro \cmd{\afterchapternum} is called just after printing a chapter
%number. By default it inserts \lnc{\midchapskip} space (default
%20pt).

\cmd{\afterchaptertitle} 매크로는 장 제목을 식자한 직후에 호출된다.
기본값으로는 \lnc{\midchapskip} 만큼의 공백을 삽입한다. (기본값 40pt)
%The macro \cmd{\afterchapskip} is called just after printing a chapter
%title. By default it inserts \lnc{\afterchapskip} space (default
%40pt).

\lnc{\beforechapskip}, \lnc{\midchapskip}, \lnc{\afterchapskip} 들은
\cmd{\setlength} 이나 \cmd{\addtolength} 명령을 통해 수정될 수 있다.
물론 \fref{lay:chap} 에서 언급되었듯 그들은 약간의 설명을 필요로 한다:
%The lengths \lnc{\beforechapskip}, \lnc{\midchapskip} and \lnc{\afterchapskip}
%can all be changed by \cmd{\setlength} or \cmd{\addtolength}. Though
%as mentioned in \fref{lay:chap} they need some explanation:
\begingroup
\medskip
\setlength\overfullrule{5pt}
\setlength\unitlength{\textwidth}
\addtolength\unitlength{-\labelsep}
\renewcommand\descriptionlabel[1]{\hspace\labelsep\parbox{\unitlength}{\cs{#1}}}
\begin{description}
\item[beforechapskip] \fref{lay:chap}를 참조. 
장의 baseline 과 본문 텍스트 블록의 실제 거리는 
\cmd{\beforechapskip}\,+\,\cmd{\topskip}\,+\,\cmd{\baselineskip} 이다.
하지만 \cmd{\chapter}의 구현 방식이 
\cmd{\chapterheadstart}를 통해 \cmd{\vspace*}를 사용하기 때문에,
\cmd{\beforechapskip} 를 없애야 하는 헛수고를 들이게 된다.
만약 장 헤딩의 첫 텍스트 직전에 공백이 오는 것을 피하고 싶다면,
\begin{lcode}
	\setlength\beforechapskip{-\baselineskip}
\end{lcode}
을 사용하거나 \cmd{\chapterheadstart}가 아무것도 하지 않도록 재정의하라.

%\item[beforechapskip] See \fref{lay:chap}.  The actual distance
%  between the first baseline of the chapter stuff to the top of the
%  text block is
%  \cmd{\beforechapskip}\,+\,\cmd{\topskip}\,+\,\cmd{\baselineskip}.
%  But because the implementation of \cmd{\chapter} (via
%  \cmd{\chapterheadstart}) make use of \cmd{\vspace*}, getting rid of
%  \cmd{\beforechapskip} a strange endeavour. If you want to avoid any
%  space before the first text in the chapter heading, use
%  \begin{lcode}
%    \setlength\beforechapskip{-\baselineskip}
%  \end{lcode}
%  or redefine \cmd{\chapterheadstart} to do nothing.

\item[midchapskip] 
\item[afterchapskip] 
둘 모두 baseline to baseline 거리를 얻기 위해서는 \cmd{\baselineskip}를 더해야 한다.
%  for both, one has to add \cmd{\baselineskip} to
%  get the distance baseline to baseline.
\end{description}
\endgroup





\begin{syntax}
\cmd{\printchaptername} \cmd{\chapnamefont} \\
\cmd{\chapternamenum} \\
\cmd{\printchapternum} \cmd{\chapnumfont} \\
\end{syntax}
\glossary(printchaptername)%
  {\cs{printchaptername}}%
  {\cs{chapnamefont}를 사용하여 장 제목을 출력.}
%  {Prints the chapter name using the \cs{chapnamefont}.}
\glossary(chapnamefont)%
  {\cs{chapnamefont}}%
  {\cs{printchaptername}에서 사용된 폰트.}
%  {Font used by \cs{printchaptername}.}
\glossary(chapternamenum)%
  {\cs{chapternamenum}}%
  {장 이름과 번호 사이에 삽입되며, 보통의 경우 공백이다.}
%  {Inserted between printing the chapter name and number. 
%   It is usually a space.}
\glossary(printchapternum)%
  {\cs{printchapternum}}%
  {\cs{chapnumfont}를 사용하여 장 번호를 출력.}
%  {Prints the chapter number using the \cs{chapnumfont}.}
\glossary(chapnumfont)%
  {\cs{chapnumfont}}%
  {\cs{printchapternum}에서 사용된 폰트.}
%  {Font used by \cs{printchapternum}.}
\cmd{\printchaptername} 매크로는 \cmd{\chapnamefont}에서 지정된
폰트를 사용하여 장 이름을 식자한다.
장 이름의 기본값은 \texttt{Chapter} 또는 \texttt{Appendix}이며,
\cmd{\huge} 크기와 \cmd{\bfseries} 폰트가 사용된다.
비슷하게, 장 번호는 \cmd{\chapnumfont}에서 지정된 폰트로
\cmd{\printchapternum}로 식자되며,
\cmd{\chapnamefont}와 동일한 기본값을 갖는다.
\cmd{\chapternamenum} 매크로는 장 이름과 번호 사이에 호출되어
보통의 경우 공백을 삽입해준다.
%The macro \cmd{\printchaptername} typesets the chapter name 
%(default \texttt{Chapter} or \texttt{Appendix}) using the font 
%specified by \cmd{\chapnamefont}. The default is the \cmd{\bfseries} font in
%the \cmd{\huge} size. Likewise the chapter number is typeset by 
%\cmd{\printchapternum} using the font specified by \cmd{\chapnumfont}, 
%which has the same default as \cmd{\chapnamefont}. The macro 
%\cmd{\chapternamenum}, which is defined to be a space, is called between 
%printing the chapter name and the number. 

\begin{syntax}
\cmd{\printchaptertitle}\marg{title} \cmd{\chaptitlefont} \\
\end{syntax}
\glossary(printchaptertitle)%
  {\cs{printchaptertitle}\marg{title}}%
  {장 제목 \meta{title}을 폰트 \cs{chaptitlefont}를 사용하여 출력.}
%  {Prints the chapter \meta{title} using the \cs{chaptitlefont}.}
\glossary(chaptitlefont)%
  {\cs{chaptitlefont}}%
  {\cs{printchaptertitle}에서 사용된 폰트.}
%  {Font used by \cs{printchaptertitle}.}
장 제목은 \cmd{\chaptitlefont}에서 지정된 폰트로
\cmd{\printchaptertitle}에 의해 식자된다.
기본값으로는 \cmd{\Huge} 크기와 \cmd{\bfseries} 폰트가 사용된다.
%The title is typeset by \cmd{\printchaptertitle} using the font specified 
%by \cmd{\chaptitlefont}. 
%By default this is a \cmd{\bfseries} font in the \cmd{\Huge} size. 

\begin{syntax}
\cmd{\printchapternonum} \\
\end{syntax}
\glossary(printchapternonum)%
  {\cs{printchapternonum}}%
  {번호가 붙지 않은 장에서 장 이름과 번호를 대체한다.}
%  {Replaces printing the chapter name and number in unnumbered chapters.}
\cmd{\frontmatter} 안에 있거나 \cmd{\chapter*}의 사용 등으로 인해
장의 번호가 없을 경우,
식자 시 \cmd{\printchapternonum} 명령이 호출되어 장 이름과 번호가 식자되는 것을 대체한다.
아래의 예시를 보라:
%If a chapter is unnumbered, perhaps because it is in the \cmd{\frontmatter}
%or because \cmd{\chapter*} is used, then when printing the command
%\cmd{\printchapternonum} is called instead of printing the name and number,
%as illustrated below:
\begin{lcode}
\newcommand{\chapterhead}[1]{ % THIS IS A SIMPLIFIED VERSION
  \clearforchapter        % move to correct page
  \thispagestyle{chapter} % set the page style
  \insertchapterspace     % Inserts space into LoF and LoT
  \chapterheadstart       % \beforechapskip space before heading
  \printchaptername\chapternamenum\printchapternum
  \afterchapternum        % \midchapskip space between number and title
  \printchaptertitle{#1}  % title
  \afterchaptertitle}     % \afterchapskip space after title
\end{lcode}
%
%
%
\LMnote{2010/07/16}{added}
기본값으로는 \cs{chapter} 뒤에 바로 이어지는 첫 문단은 들여써지지 \emph{않는다}.
이는 다음에 의해 조정될 수 있다:
%By default the first paragraph following a \cs{chapter} is \emph{not}
%indented, this can be controlled by
\begin{syntax}
  \cmd{\indentafterchapter}\\
  \cmd{\noindentafterchapter}
\end{syntax}
\glossary(indentafterchapter)%
{\cs{indentafterchapter}}%
{\cs{chapter} 뒤에 바로 이어지는 첫 문단이 들여써지도록 지정.}
%{Specifies that the first paragraph after a \cs{chapter} should be indented}
\glossary(noindentafterchapter)%
{\cs{noindentafterchapter}}%
{\cs{chapter} 뒤에 바로 이어지는 첫 문단이 들여써지지 않도록 지정.}
%{Specifies that the first paragraph after a \cs{chapter} should \emph{not} be indented}

% 동어반복이라서 생략
%The default is not to indent the first paragraph following a
%\cs{chapter}.

\begin{syntax}
\cmd{\insertchapterspace} \\
\end{syntax}
\glossary(insertchapterspace)%
  {\cs{insertchapterspace}}%
  {\cs{chapter}에 의해 호출되어 LoF(그림 목차)와 LoT(표 목차)에 공백을 삽입.}
%  {Called by \cs{chapter} to insert space into LoF and LoT.}
기본값으로는 \cmd{\chapter}는 그림 목차(LoF, List of Figures)와
표 목차(LoT, List of Tables)에 작은 양의 수직 공백을 삽입한다.
이를 하기 위해 \cmd{\insertchapterspace}가 호출되며, 정의는 다음과 같다:
%By default a \cmd{\chapter} inserts a small amount of vertical space
%into the List of Figures and List of Tables. It calls \cmd{\insertchapterspace}
%to do this. The default definition is:
\begin{lcode}
\newcommand{\insertchapterspace}{%
  \addtocontents{lof}{\protect\addvspace{10pt}}%
  \addtocontents{lot}{\protect\addvspace{10pt}}%
}
\end{lcode}
%If you would prefer no inserted spaces then 
만약 공백이 없길 원한다면
\begin{lcode}
\renewcommand{\insertchapterspace}{}
\end{lcode}
가 적당할 것이다.
%will do the job. 
\cmd{\addvspace} 명령에 있는 길이 인자의 값을 수정하여
다른 양의 공백이 삽입될 수 있다.
%Different spacing can be inserted by
%changing the value of the length arguments to \cmd{\addvspace}.

위의 매크로들에 적당한 변화를 가하여 레이아웃에 약간의 변화를 줄 수 있을 것이다.
%By making suitable changes to the above macros you can make some
%simple modifications to the layout.

\index{헤딩!장|)}
\index{heading!chapter|)}

\subsection{장 스타일 정의하기} \label{sec:chapterstyle}
%\subsection{Defining a chapter style} \label{sec:chapterstyle}

\index{chapterstyle|(}
\index{장스타일}

이 클래스에서는 스스로 chapter 헤딩을 구현하고 정의할 수 있는 많은 방법을 제공한다.
%    The class provides many ways in which you can implement your designs
%for chapter headings.

\begin{syntax}
\cmd{\chapterstyle}\marg{style} \\
\end{syntax}
\glossary(chapterstyle)%
  {\cs{chapterstyle}\marg{style}}%
  {chapter의 헤딩 양식을 \meta{style}로 설정.}
%  {Sets chapter heading layout to \meta{style}.}
\cmd{\chapterstyle} 매크로는 \cmd{\pagestyle} 명령과 비슷하지만,
이후의 어떤 chapter 에 대해서도 헤딩을 \meta{style}이 되도록 설정한다.
%The macro \cmd{\chapterstyle} is rather like the \cmd{\pagestyle} command in 
%that it sets the style of any subsequent chapter headings to be \meta{style}.

%    The class provides some predefined chapter styles, including the 
%\cstyle{default} style which is the familiar LaTeX \Lclass{book} class chapter
%headings style. To use the chapterstyle \cstyle{fred} just issue the command
이 클래스는 미리 정의된 chapter 스타일을 몇 가지 제공한다.
\cstyle{default} 스타일의 경우 우리에게 익숙한 \LaTeX{}의 \Lclass{book} 클래스와
동일하게 나온다.
\cstyle{fred} 라는 이름의 chapter 스타일을 사용하고 싶은 경우 단순히
\begin{lcode}
\chapterstyle{fred}
\end{lcode}
와 같은 명령을 사용하면 된다. 한 문서 내에서 여러 스타일을 사용하는 것 또한 가능하다.
%Different styles can be used in the same document.

%The simpler predefined styles include:
%\LMnote{2011/05/25}{I've moved the chapter style images to the
%  showcase appendix to save some space in the main part of the manual}
미리 정의된 간단한 스타일들은 아래와 같다:
\begin{itemize}
\item[\cstyle{default}] 
\glossary(defaultcs)%
  {\Pcstyle{default}}%
  {기본 \Ppack{book} 클래스의 chapter 스타일.}
%  {The default \Ppack{book} class chapterstyle.}
%The normal \ltx\ \Lclass{book} class chapter styling; shown in \fref{dcdefault}.
기본 \ltx\ \Lclass{book} 클래스의 chapter 스타일. 그림 \fref{dcdefault}.
%\begin{demochap}[-3\onelineskip]{default}\end{demochap}

\item[\cstyle{section}] 
\glossary(sectioncs)%
  {\Pcstyle{section}}%
  {Section 헤딩 형식의 chapter 스타일.}
%  {Simple chapterstyle looking like a section head.}
이 헤딩은 section 처럼 식자된다.
즉, 번호만이 표시되고 제목과 번호표제가 같은 줄에 놓인다. 그림 \fref{dcsection}.
%The heading is typeset like a section; that is, 
%there is just the number and the title on one line. This is illustrated
%in \fref{dcsection}. 
%%    See \Cref{chap:topsandtails} on \pref{chap:topsandtails} for an example.
%\begin{demochap}[2\onelineskip]{section}\end{demochap}

\item[\cstyle{hangnum}] 
\glossary(hangnumcs)%
  {\Pcstyle{hangnum}}%
  {Section 헤딩 형식이지만 번호표제가 마진 영역에 나타나는 chapter 스타일.}
%  {Simple chapterstyle looking like a section head but with the number in the margin.}
\pstyle{section} 스타일과 비슷하지만 번호표제가 마진 영역에 온다. 그림 \fref{dchangnum}.
%Like the \pstyle{section} style except that the chapter
%number is put in the margin\index{margin!left}, as shown in \fref{dchangnum}. 
%%    See \Cref{chap:captions} on \pref{chap:captions} for an example.
%\begin{demochap}[2\onelineskip]{hangnum}\end{demochap}

\item[\cstyle{companion}]
\glossary(companioncs)%
  {\Pcstyle{companion}}%
  {\textit{LaTeX companion} 시리즈와 같은 느낌의 chapter 스타일.}
%  {Chapterstyle like those in the \textit{LaTeX companion} series.}
\textit{LaTeX companion} 시리즈와 같은 느낌의 chapter 스타일. 그림 \fref{dccompanion}.
% This produces chapter headings like those of the
%\textit{LaTeX Companion} series of books. An example is in \fref{dccompanion}.
%%    See \Cref{chap:signposts} on \pref{chap:signposts} for an example.
%\begin{demochap}[2\onelineskip]{companion}\end{demochap}

\item[\cstyle{article}]
\glossary(articlecs)%
  {\Pcstyle{article}}%
  {\Pclass{article} 클래스의 section 헤딩과 비슷하지만 다른 크기의 폰트가 사용되는 chapter 스타일.}
%  {Chapter style similar to a section head in the \Pclass{article} class but 
%   with different
%   sized fonts.}
헤딩이 \Lclass{article} 클래스의 \cmd{\section}처럼 식자된다.
\cstyle{section} 스타일과 비슷하지만 폰트와 간격주기가 다르다. 그림 \fref{dcarticle}.
% The heading is typeset like a \cmd{\section}
%heading in the \Lclass{article} class. This is similar to the 
%\cstyle{section} style but different fonts and spacings are used,
%as shown in \fref{dcarticle}.
%\begin{demochap}[-1\onelineskip]{article}\end{demochap}

\item[\cstyle{reparticle}]
\glossary(reparticlecs)%
  {\Pcstyle{reparticle}}%
  {\Popt{article} 옵션을 통해 \Popt{article} 클래스의 section 헤딩과 동일하게 해준다.
  }
%  {With the \Popt{article} class option it replicates a section head in 
%   the \Pclass{article} class.}
	\Popt{article} 옵션이 쓰일 경우 
	나타나는 기본 chapter/section 스타일은
	\Popt{article} 클래스의 것들과 비슷하지만
	완전히 동일하지는 않다. 
	\cstyle{reparticle} 스타일은 \cmd{\chapter} 명령이
	\Lclass{article}의 \cmd{\section} 명령과 완전히 동일한
	결과물을 내도록 한다.
%   When the \Lopt{article} class option is used
%    the default chapter and section styles are close, but not
%    identical, to the corresponding division heads in the \Lclass{article}
%    class. The \cstyle{reparticle} chapterstyle makes \cmd{\chapter}
%    replicate the appearance of \cmd{\section} in the \Lclass{article}
%    class.

%%\item[\cstyle{demo}] Try this one to see what it does. On the other
%%   hand you can look at \Cref{chap:verse} on \pref{chap:verse} to see
%%   what it looks like.
\end{itemize}

앞서 소개한 것과 같이 미리 정의된 chapter 스타일을 사용하고자 할 경우
이 절의 나머지 내용을 읽을 필요가 없다.
다만 그 스타일의 결과를 보여주는 그림이 곧 나오니 그것을 챙겨보도록 해라.
%    If you use only the predefined chapterstyles there is no need to
%plough through the rest of this section, except to look at the illustrations
%of the remaining predefined chapterstyles shown a little later.

%The various macros shown in the \cs{chapterhead} code above are initially 
%set up as (the code is known as the \cstyle{default} style): 
\cs{chapterhead} 코드의 다양한 매크로는 처음에는 다음과 같이 정의되어 있다:
(이는 \cstyle{default} 스타일이라고도 불린다.)
\begin{lcode}
\newcommand{\chapterheadstart}{\vspace*{\beforechapskip}}
\newcommand{\printchaptername}{\chapnamefont \@chapapp}
\newcommand{\chapternamenum}{\space}
\newcommand{\printchapternum}{\chapnumfont \thechapter}
\newcommand{\afterchapternum}{\par\nobreak\vskip \midchapskip}
\newcommand{\printchapternonum}{}
\newcommand{\printchaptertitle}[1]{\chaptitlefont #1}
\newcommand{\afterchaptertitle}{\par\nobreak\vskip \afterchapskip}
\newcommand{\chapnamefont}{\normalfont\huge\bfseries}
\newcommand{\chapnumfont}{\normalfont\huge\bfseries}
\newcommand{\chaptitlefont}{\normalfont\Huge\bfseries}
\setlength{\beforechapskip}{50pt}
\setlength{\midchapskip}{20pt}
\setlength{\afterchapskip}{40pt}
\end{lcode}
(이 신비로운 \cmd{\@chapapp}은 \ltx\ 내부에서 chapter 의 이름을 저장하기 위해
사용되는 매크로다. \footnote{
매크로 이름에 \texttt{@}이 있을 경우 \texttt{@}가 letter로 인식되는 곳에서
사용되어야만 한다는 것을 기억하라.
})
이것은 매번 다른 값을 가질 것이며, 조판하며 자동으로 설정될 것이다.
예를 들어, 본문에서는 \texttt{Chapter}, 부록\index{부록}\index{appendix}에서는 
\texttt{Appendix}로 보일 것이다.
물론 직접 이를 설정할 수도 있다.
%(The mysterious \cmd{\@chapapp} is the internal macro that \ltx\ uses
%to store normally the chapter name.\footnote{Remember, if you use
%a macro that has an \texttt{@} in its name it must be in a place
%where \texttt{@} is treated as a letter.} It will normally have 
%different values,
%set automatically, when typesetting a chapter in the main body 
%(e.g., \texttt{Chapter}) or in the appendices\index{appendix} where
%it would usually be set to \texttt{Appendix}, but you can specify
%these names yourself.)



%    A new style is specified by changing the definitions of this last set of 
%macros and/or the various font and skip specifications.

새로운 스타일은 다음에 나오는 마지막 매크로 모음의 정의나
여러 가지 폰트/줄간격 설정을 바꿈으로서 정의된다.

\begin{syntax}
\cmd{\makechapterstyle}\marg{style}\marg{text} \\
\end{syntax}
chapter 스타일은 \cmd{\makechapterstyle} 커맨드로 정의된다.
\meta{style}은 정의되는 스타일의 이름이며 \meta{text}는 스타일을 정의하는
LaTeX 코드다.
여기서 \meta{text}는 \emph{항상} 매크로와 위에 보여진 값들을 리셋하는 것으로
시작함에 주목하라.
당신이 실제로 얻게 될 것은 다음과 같다:
%Chapter styles are defined via the \cmd{\makechapterstyle} command, where
%\meta{style} is the style being defined and \meta{text} is the LaTeX code 
%defining the style. Please note that the \meta{text} \emph{always}
%start by resetting macros and lengths to the values shown above. That
%is what you actually get is is the same as
\begin{syntax}
\cmd{\makechapterstyle}\marg{style}\{\\
  \cmd{\chapterstyle}\{default\}\\
  \meta{text}\\
\} 
\end{syntax}


\cstyle{default} chapter 스타일은 \file{memoir.cls} 에서 보이듯
표준 \Lclass{book}, \Lclass{report} 클래스의 chapter 헤딩을 흉내낸다:
%    To start things off,  the \cstyle{default} chapter
%style, which mimics the chapter heads in the standard \Lclass{book} and 
%\Lclass{report} classes, as it appears in \file{memoir.cls}:
\begin{lcode}
\makechapterstyle{default}{}
\chapterstyle{default}
\end{lcode}
%-- since the \cstyle{default} style \emph{is} the initial value of
%\cmd{\makechapterstyle}. The actual code is seen above. 
왜냐하면 \cstyle{default} 스타일은 \cmd{\makechapterstyle} 명령의
초기값이기 때문이다. 실제 코드는 위와 같다.



%    As an example of setting up a simple chapterstyle, here is the code for 
%defining the \cstyle{section} chapterstyle. In this case it is principally
%a question of eliminating most of the printing and zeroing some spacing.
%간단한 chapter 스타일을 지정하는 예시로, 여기에 \cstyle{section} 스타일을 정의하는
%코드를 여기 보인다. 이 경우 
예로서, \cstyle{section} 장 스타일 정의 코드를 살펴보자.
\begin{lcode}
\makechapterstyle{section}{%
  \renewcommand*{\printchaptername}{}
  \renewcommand*{\chapternamenum}{}
  \renewcommand*{\chapnumfont}{\chaptitlefont}
  \renewcommand*{\printchapternum}{\chapnumfont \thechapter\space}
  \renewcommand*{\afterchapternum}{}
}
\end{lcode}
여기서 \cmd{\printchaptername} 는 비어있다. 그러므로 `Chapter' 라는 문자열은
시작되지 않는다. 번호행과 타이틀에 동일한 폰트가 쓰였고, 번호 뒤에 공백 하나가 따라붙는다.
\cmd{\afterchapternum} 매크로가 비어 있기 때문에 장 제목이 번호 뒤에 바로 식자되는 것이다.
%In this style, \cmd{\printchaptername} is vacuous, so the normal `Chapter'
%is never typeset. The same font is used for the number and the title, and
%the number is typeset with a space after it. The macro \cmd{\afterchapternum}
%is vacuous, so the chapter title will be typeset immediately after the number.


표준 클래스에서 번호가 붙지 않은 장의 제목은 기존에 `Chapter'라는 단어가 식자되는
곳과 같은 위치에 온다.
\cmd{\printchapternonum} 매크로는 번호가 붙지 않은 장 제목이 식자되기 직전에 호출된다.
기본값은 아무것도 하지 않지만 원한다면 \cmd{\renewcommand}로 이를 수정할 수 있다.
예를 들면, 번호가 붙은 장과 번호가 붙지 않은 장이 모두 페이지 상에서 같은 높이에 있기를 원한다면
\cmd{\printchapternonum} 를 재정의하여 `Chapter N' 줄이 차지했을 만큼의
수직 공간을 넣을 수 있다.
%    In the standard classes the title of an unnumbered chapter is typeset
%at the same position on the page as the word `Chapter' for numbered chapters.
%The macro \cmd{\printchapternonum} is called just before an unnumbered 
%chapter title text is typeset. By default this does nothing but you can use
%\cmd{\renewcommand} to change this. For example, if you wished the title
%text for both numbered and unnumbered chapters to be at the same height on
%the page then you could redefine \cmd{\printchapternonum} to insert
%the amount of vertical space taken by any `Chapter N' line. For example,
%as \cmd{\printchapternonum} is vaucuous in the \cstyle{default} chapterstyle
%the vertical position of a title depends on whether or not it is numbered.
% 마지막 문장 이상해서 번역에 안 넣음

\cstyle{hangnum} 스타일은 장 번호가 마진에 붙는다는 점을 제외하고
\cstyle{section} 과 비슷하다.
정의는 다음과 같다:
%The \cstyle{hangnum} style, which is like \cstyle{section} except that it
%puts the number in the margin, is defined as follows:
\begin{lcode}
\makechapterstyle{hangnum}{%
  \renewcommand*{\chapnumfont}{\chaptitlefont}
  % allow for 99 chapters!
  \settowidth{\chapindent}{\chapnumfont 999}
  \renewcommand*{\printchaptername}{}
  \renewcommand*{\chapternamenum}{}
  \renewcommand*{\chapnumfont}{\chaptitlefont}
  \renewcommand*{\printchapternum}{%
    \noindent\llap{\makebox[\chapindent][l]{%
    \chapnumfont \thechapter}}}
  \renewcommand*{\afterchapternum}{}
}
\end{lcode}
장 번호는 세 자리 수가 들어갈 만큼 넉넉한 박스 좌측의 공간에 들어간다.
박스는 조판을 위해 \cmd{\llap} 명령을 통해 마진에 놓인다.
그 후 장 제목이 좌측 마진에서 조판되기 시작한다.
%The chapter number is put at the left of a box wide enough for three digits.
%The box is put into the margin, via \cmd{\llap}, for typesetting. The
%chapter title is then typeset, starting at the left margin.

\begin{syntax}
\lnc{\chapindent} \\
\end{syntax}
\glossary(chapindent)%
  {\cs{chapindent}}%
  {장 스타일(혹은 다른 임의의 것)을 정의할 때 사용할 수 있는 길이}
%  {A length you can use in a defining chapter style (or anything else).}
\lnc{\chapindent} 명령이 장 스타일을 정의하는 용도로 제공되지만,
그 외에 어떠한 다른 곳에서도 쓸 수 있다.
%The length \lnc{\chapindent} is provided for use in specifying chapterstyles,
%but you could use it for any other purposes.

%    The definition of the \cstyle{companion} chapterstyle is more complicated.
\cstyle{companion} 장 스타일의 정의는 더 복잡하다.
\begin{lcode}
\makechapterstyle{companion}{%
  \renewcommand*{\chapnamefont}{\normalfont\LARGE\scshape}
  \renewcommand*{\chapnumfont}{\normalfont\Huge}
  \renewcommand*{\printchaptername}{%
    \raggedleft\chapnamefont \@chapapp}
  \setlength{\chapindent}{\marginparsep}
  \addtolength{\chapindent}{\marginparwidth}
  \renewcommand{\printchaptertitle}[1]{%
    \begin{adjustwidth}{}{-\chapindent}
      \raggedleft \chaptitlefont ##1\par\nobreak
    \end{adjustwidth}}
}
\end{lcode}
\PWnote{2009/09/07}{Fixed adjuswidth typo.}
\fref{dccompanion}에서 보이듯 장 제목은 소문자로 우측 정렬되어 보인다.
또한, 제목은 마진 노트 영역의 가장 바깥쪽에 놓인다.
이는 \Ie{adjustwidth} 환경을 사용하여 구현된다 \footnote{\S\ref{sec:adjustwidth} 장을 참조하라.}.
to make \ltx\ think that the typeblock is locally wider than it actually is.
%As shown in \fref{dccompanion} the chapter name is in small caps and set
%flushright. The title is also set flushright aligned with the outermost 
%part of the marginal notes. This is achieved by use of the \Ie{adjustwidth}
%environment\footnote{See \S\ref{sec:adjustwidth}.} to make \ltx\ think
%that the typeblock is locally wider than it actually is.

\subsection{더 많은 장 스타일}
%\subsection{Further chapterstyles}

이 클래스는 이 절에서 보이듯 더 많은 장 스타일을 제공한다.
몇개는 내가 직접 작성한 것이며 다른 몇개는 \Pclass{memoir} 사용자들에
의해 \ctt\ 에 포스팅된 것이다.
나는 그 포스팅된 것들의 일부를 수정했다.
부록, 여러 줄에 걸친 제목, 번호 없는 장이 고려되지 않은 경우가 있었기 때문이다.
그 후 어떻게 그 문제들을 해결했는지 보이기 위해 수정된 코드를 공개했다. 
그와 별개로 Lars Madsen\index{Madsen, Lars}는 
수많은 스타일~\cite{CHAPSTYLES}들을 수집했고 그것들이 어떻게 만들어졌는지 보였다.
%    The class provides more chapterstyles, which are listed here. 
%Some are mine and others are from postings to \ctt\ by \Pclass{memoir}
%users. I have modified some of the posted ones to cater for things like 
%appendices, multiline titles, and unnumbered chapters which were not 
%considered in the originals. The code
%for some of them is given later to help you see how they are done.
%Separately, Lars Madsen\index{Madsen, Lars} has collected a wide variety of 
%styles~\cite{CHAPSTYLES} and shows how they were created.

%    If you want to try several chapterstyles in one document, request the
%\cstyle{default} style before each of the others to ensure that a previous 
%style's changes are not passed on to a following one.

만약 하나의 문서에서 여러 개의 장 스타일을 사용하고 싶다면,
매 장마다 \cstyle{default} 스타일을 호출하여 이전 장에서 일어난 스타일 변화가
다음 장에까지 연쇄적으로 적용되지 않도록 하여라.

\LMnote{2011/05/25}{I've moved the chapter style images to the
  showcase appendix to save some space in the main part of the manual}


\newcommand\PScaveat[1]{
  \smallskip\noindent
  \textbf{Caveat:} The \Pcstyle{#1} style requires the
\Lpack{graphicx} package.
\smallskip
}


\begin{itemize}

\item[\cstyle{bianchi}] 
\glossary(bianchics)%
  {\Pcstyle{bianchi}}%
  {A two line, centered chapterstyle with rules above and below.}
  This style was created by 
  Stefano\index{Bianchi, Stefano} Bianchi\footnote{\ctt,
  \textit{New chapter style: chapter vs chapter*}, 2003/12/09} and is a two 
  line centered arrangement with rules above and below the large bold 
  sanserif title line. 
  The chapter number line is in a smaller italic font. An example
is in \fref{dcbianchi}.

%\begin{demochap}[2\onelineskip]{bianchi}\end{demochap}
%\begin{demochap}[-2\onelineskip]{bianchi}\end{demochap}

\item[\cstyle{bringhurst}] The \cstyle{bringhurst} chapterstyle described
  in the manual and illustrated in \fref{dcbringhurst}.
\glossary(bringhurstcs)%
  {\Pcstyle{bringhurst}}%
  {A raggedright, unnumbered, small caps chapterstyle with a textwidth
   rule below.}

%\begin{demochap}{bringhurst}\end{demochap}


\item[\cstyle{brotherton}]
\glossary(brothertoncs)%
  {\Pcstyle{brotherton}}%
  {A chapterstyle like the default but with the number spelled out.}%
 A very simple style designed by 
William\index{Adams, William} Adams\footnote{\ctt, 
\textit{An example of a novel?}, 2006/12/09} for the science fiction novel 
\textit{Star Dragon} by Mike Brotherton. The novel is freely downloadable 
from Brotherton's web site. The style is the same as the \cstyle{default} 
except that the number is spelt out in words. It is demonstrated in
\fref{dcbrotherton}. In the novel the chapters are actually 
untitled i.e., via \verb?\chapter{}?.

%\begin{demochap}[2\onelineskip]{brotherton}\end{demochap}
%\begin{demochap}[-2\onelineskip]{brotherton}\end{demochap}


\item[\cstyle{chappell}] 
\glossary(chappellcs)%
  {\Pcstyle{chappell}}%
  { A centered chapterstyle with a rule between
   the number line (in a roman font) and the title line in italics.}
The name and number are centered above a rule and the
title in italics is below, again centered. It is illustrated in 
\fref{dcchappell}.

%\begin{demochap}{chappell}\end{demochap}

\item[\cstyle{crosshead}] 
\glossary(crossheadcs)%
  {\Pcstyle{crosshead}}%
  { A centered chapterstyle in a bold font.}
The number and title are centered and set with a large bold font.
It is illustrated in 
\fref{dccrosshead}.

%\begin{demochap}[-2\onelineskip]{crosshead}\end{demochap}


\item[\cstyle{culver}] 
\glossary(culvercs)%
  {\Pcstyle{culver}}%
  {One line, centered, bold chapterstyle using Roman numerals.}
A chapter style I created for 
  Christopher\index{Culver, Christopher}
  Culver\footnote{\ctt, \textit{"Biblical" formatting, how?}, 2004/03/29}
  based on the format of `ancient' texts.
 It is one line, centered, bold and with the number printed as Roman numerals,
as shown in \fref{dcculver}.

\begin{comment}
\makechapterstyle{culver}{
  \chapterstyle{default}
  \chapterstyle{article}
  \renewcommand*{\thechapter}{\Roman{chapter}}
  \renewcommand*{\printchapternum}{% center number/title
    \centering\chapnumfont \thechapter\space\space}%
  \renewcommand*{\printchapternonum}{\centering}
  \renewcommand*{\clearforchapter}{}% no new page
  \aliaspagestyle{chapter}{headings}% no special pagestyle
}
\end{comment}

%\begin{demochap}[-\onelineskip]{culver}\end{demochap}

    He also wanted sections to just start with the number and the text to 
immediately follow on the same line. That can be accomplished like this:
\begin{lcode}
\renewcommand*{\thesection}{\arabic{section}}
\renewcommand*{\section}[1]{%
  \refstepcounter{section}%
  \par\noindent
  \textbf{\thesection.}%
  \space\nolinebreak}
\end{lcode}

\item[\cstyle{dash}]
\glossary(dashcs)%
  {\Pcstyle{dash}}%
  {Two line, centered, regular font, chapterstyle. The number has a dash on 
   either side.}
 A simple two line centered chapterstyle. There is a short
dash on either side of the number and a slightly larger
version of the regular font is used for both the number and the title.
This style is shown in \fref{dcdash}.

%\begin{demochap}[-2\onelineskip]{dash}\end{demochap}
%\begin{demochap}[-4\onelineskip]{dash}\end{demochap}

%\item[\cstyle{default}] This was already in the class but it has been revised
%     to re-initialize all the settings.
%\glossary(defaultcs)%
%  {\Pcstyle{default}}%
%  {The default \Ppack{book} class chapterstyle.}

\item[\cstyle{demo2}]
\glossary(demo2cs)%
  {\Pcstyle{demo2}}%
  {A two line chapterstyle with a large sanserif title; the number is above, 
   centered and written (e.g., Six instead of 6) in a bold font. 
   There are rules above and below the title.}
   A two line chapterstyle with a large sanserif title; the number is above, 
   centered and written (e.g., Six instead of 6) in a bold font. 
   There are rules above and below the title. An example is shown in 
  \fref{dcdemo2}.


%\begin{demochap}[-2\onelineskip]{demo2}\end{demochap}
%\begin{demochap}[-\onelineskip]{demo2}\end{demochap}

\item[\cstyle{demo3}]
\glossary(demo3cs)%
  {\Pcstyle{demo3}}%
  {A two line chapterstyle with a large sanserif title; the number is above, 
   centered and written (e.g., Six instead of 6) in an italic font. 
   There are rules above and below
   the title line. It is a modified version of the \Pcstyle{demo2} style.}
   The chapterstyle used in this document. It is a
  modified version of the \cstyle{demo2} chapterstyle, using an italic
  rather than bold font for the number.

\item[\cstyle{dowding}] 
\glossary(dowdingcs)%
  {\Pcstyle{dowding}}%
  { A centered two line chapterstyle in an italic font and the number
   is spelled out.}
A centered style where the name and number are set in a bold font, with the 
number spelled out. The title is below in a large italic font. The style
is based on the design used in Dowding's \textit{Finer Points}~\cite{DOWDING96}.
It is illustrated in 
\fref{dcdowding}.

%\begin{demochap}[-2\onelineskip]{dowding}\end{demochap}


\item[\cstyle{ell}] 
\glossary(ellcs)%
  {\Pcstyle{ell}}%
  {A raggedleft large sanserif chapterstyle with the number in the margin. An
   `L' shaped rule separates the number and title lines.}
  A raggedleft sanserif chapterstyle. 
  The number line is separated
  from the title by rules like an `L' on its side and the number is placed
  in the margin, as shown in \fref{dcell}. 
I will probably use this in my next book. 

%\begin{demochap}[6\onelineskip]{ell}\end{demochap}
%\begin{demochap}{ell}\end{demochap}

\item[\cstyle{ger}] 
\glossary(gercs)%
  {\Pcstyle{ger}}%
  {A raggedright, large, bold, two line chapterstyle with rules
   above and below.}
This style was created by 
Gerardo\index{Garcia, Gerardo} Garcia\footnote{\ctt,
  \textit{Fancy Headings, Chapter Headings}, 2002/04/12} and is a two line,
  raggedright, large bold style with rules above and below. It is
  demonstrated in \fref{dcger}.

%\begin{demochap}{ger}\end{demochap}
%\begin{demochap}[-4\onelineskip]{ger}\end{demochap}

\item[\cstyle{komalike}] 
\glossary(komalikecs)%
  {\Pcstyle{komalike}}%
  { A section-like chapterstyle in a sans serif font.}
A section-like style set with a sans serif type. It is like that in the
\Lclass{scrbook} class (part of the KOMA bundle). 
It is illustrated in 
\fref{dckomalike}.

%\begin{demochap}[-2\onelineskip]{komalike}\end{demochap}


\item[\cstyle{lyhne}] 
\glossary(lyhnecs)%
  {\Pcstyle{lyhne}}%
  {A raggedleft bold sanserif chapter title set between two rules, with the
   name and number above. It requires the \Ppack{graphicx} package.}
  A style created by Anders\index{Lyhne, Anders}
  Lyhne\footnote{\ctt, \textit{Glossary}, 2006/02/09} and shown in \fref{dclyhne}
  where the raggedleft
  sanserif title is between two rules, with the name and number above.
  I modified the original to cater for unnumbered chapters. 

  \PScaveat{lyhne}

%\begin{demochap}[-\onelineskip]{lyhne}\end{demochap}
%\begin{demochap}[-2\onelineskip]{lyhne}\end{demochap}

\item[\cstyle{madsen}] 
\glossary(madsencs)%
  {\Pcstyle{madsen}}%
  {A raggedleft large bold sanserif chapterstyle with the number in the
   margin and a rule between the number and title lines.
   It requires the \Ppack{graphicx} package.}
  This was created by 
  Lars\index{Madsen, Lars} Madsen\footnote{\ctt,
  \textit{New chapter style: chapter vs chapter*}, 2003/12/09}
  and is shown in \fref{dcmadsen}.
  It is a large sanserif raggedleft style with the number in the margin
  and a rule between the number and title lines. 

  \PScaveat{madsen}

%\begin{demochap}[2\onelineskip]{madsen}\end{demochap}
%\begin{demochap}[-2\onelineskip]{madsen}\end{demochap}

\item[\cstyle{ntglike}] 
\glossary(ntglikecs)%
  {\Pcstyle{ntglike}}%
  { A smaller version of the standard chapterstyle.}
A smaller version of the standard chapterstyle; it is like that in the NTG 
classes (\Lclass{boek} class) developed by the Dutch \tx\ User Group.
It is illustrated in 
\fref{dcntglike}.

%\begin{demochap}[-3\onelineskip]{ntglike}\end{demochap}



\item[\cstyle{pedersen}] 
\glossary(pedersencs)%
  {\Pcstyle{pedersen}}%
  {A single line chapterstyle in large italics with the number set in the
  righthand margin. The title and/or number may be colored. The 
  \Ppack{graphicx} package is required and the \Ppack{color} 
  (or \Ppack{xcolor}) package if you want to color.}
  This was created by
  Troels\index{Pedersen, Troels} Pedersen\footnote{\ctt,
  \textit{Chapter style}, 2006/01/31} and requires the \Lpack{graphicx}
  package, and, to get the full effect, the \Lpack{color} package as well.
  The title is raggedright in large italics while the number is much larger
  and placed in the righthand margin (I changed the means of placing the 
  number). The head of this chapter is set with the \cstyle{pedersen} style,
  because it cannot be adequately demonstrated in an illustration.

  \PScaveat{pedersen}

%%%%\begin{demochap}{pedersen}\end{demochap}

\item[\cstyle{southall}] 
\glossary(southallcs)%
  {\Pcstyle{southall}}%
  { A raggedright chapterstyle with the number and title on the same line
   and a rule below.}
  This style was created by Thomas\index{Dye, Thomas} Dye. It
  is a simple numbered heading with the title set as a block 
  paragraph, and with a horizontal rule underneath. It is illustrated
  in \fref{dcsouthall}.

%\begin{demochap}[-\onelineskip]{southall}\end{demochap}

\item[\cstyle{tandh}] 
\glossary(tandhcs)%
  {\Pcstyle{tandh}}%
  {A simple section-like chapterstyle in a bold font.}
A simple section-like style in a bold font. It is based on the design used in
the Thames \& Hudson \textit{Manual of Typography}~\cite{MCLEAN80} and 
is illustrated in 
\fref{dctandh}.

%\begin{demochap}[-1\onelineskip]{tandh}\end{demochap}


\item[\cstyle{thatcher}] 
\glossary(thatchercs)%
  {\Pcstyle{thatcher}}%
  {A centered small caps chapterstyle with the number line separated from 
   the title by a short rule.}
A style created by 
  Scott\index{Thatcher, Scott} Thatcher\footnote{\ctt,
  \textit{memoir: chapter headings capitalize math symbols}, 2006/01/18} 
  which has the chapter name and number centered with the
  title below, also centered, and all set in small caps. There is a short rule
  between the number line and the title, as shown in \fref{dcthatcher}. 
  I have modified the original to
  cater for multiline titles, unnumbered chapters, and appendices.

%\begin{demochap}{thatcher}\end{demochap}



\item[\cstyle{veelo}]
\glossary(veelocs)%
  {\Pcstyle{veelo}}%
  {A raggedleft large bold chapterstyle with a large black square 
   in the margin by the number line. It requires the \Ppack{graphicx} package.}
   This style created by Bastiaan\index{Veelo, Bastiaan} Veelo 
  is shown in \fref{dcveelo} and is raggedleft, large, bold and with a
  black square in the margin by the number line. 

  \PScaveat{veelo}

%\cleardoublepage

%%\savefigcnt=1
%\demochapcnt=8
%\begin{demochap}{veelo}\end{demochap}
%\begin{demochap}[-2\onelineskip]{veelo}\end{demochap}

\item[\cstyle{verville}]
\glossary(vervillecs)%
  {\Pcstyle{verville}}%
  {A single line, large, centered, chapterstyle with rules above and below.}
 A chapterstyle I created for 
  Guy\index{Verville, Guy} Verville\footnote{\ctt,
  \textit{Headers and special formatting of sections}, 2005/01/18}. 
  It is a single line, large centered style with rules above 
  and below, as in \fref{dcverville}. Unlike my posted version, 
  this one properly caters for unnumbered chapters.

%\begin{demochap}{verville}\end{demochap}

\item[\cstyle{wilsondob}] 
\glossary(wilsondobcs)%
  {\Pcstyle{wilsondob}}%
  {A one line flushright chapterstyle in a large italic font.}
A one line flushright (raggedleft) section-like style in a large italic font. 
It is based on the design used in
Adrian Wilson's \textit{The Design of Books}~\cite{ADRIANWILSON93} and 
is illustrated in 
\fref{dcwilsondob}.

%\begin{demochap}[-1\onelineskip]{wilsondob}\end{demochap}

\end{itemize}

여기 주어진 스타일 중 몇개의 코드는 \ref{sec:chapter-styles}절의
showcase 부록에 주어져 있다.
각각의 장 스타일이 어떻게 정의되었는지에 대한 자세한 사항이 궁금하다면,
설명되어 있는 클래스 코드를 들여다보라.
나중에 당신이 직접 자신만의 스타일을 정의할 때 쓸 아이디어를 줄 것이다.
%The code for some of these styles is given in
%section~\ref{sec:chapter-styles} within the Showcase Appendix.  For
%details of how the other chapter styles are defined, look at the
%documented class code. This should give you ideas if you want to
%define your own style.

장 헤딩을 변경하고 싶다고 해서 아예 새로운 장 스타일을 만들어야 하는 것은 아니다.
그냥 각각의 매크로를 직접 수정하는 것으로 충분할 것이다.
%Note that it is not necessary to define a new chapterstyle if you want
%to change the chapter headings --- you can just change the individual
%macros without putting them into a style.


\index{장 스타일|)}
\index{chapterstyle|)}

\subsection{장 요약}
%\subsection{Chapter precis}
\label{sec:chapter-precis}

\index{장!요약|(}
\index{chapter!precis|(}


몇몇 고풍스러운 소설이나 현대의 교과서들%
\footnote{예를 들어, Robert Sedgewick의 \textit{Algorithms},
	Addison-Wesley, 1983.}은
장 헤딩\index{헤딩!장}\index{heading!chapter} 직후나,
\toc 혹은 둘 모두에 장의 내용을 소개하는 짧은 요약문이 있다.

%   Some old style novels, and even some modern text 
% books,\footnote{For example, Robert Sedgewick, \textit{Algorithms},
% Addison-Wesley, 1983.} include a short synopsis of the contents of 
% the chapter either immediately
% after the chapter heading\index{heading!chapter} or in the \toc, or in both places.

\begin{syntax}
\cmd{\chapterprecis}\marg{text} \\
\end{syntax}
\glossary(chapterprecis)%
  {\cs{chapterprecis}\marg{text}}%
  {\meta{text}를 식자하며 목차(ToC)에도 이를 추가.}
%  {Prints \meta{text} and also adds it to the ToC.}
\cmd{\chapterprecis} 명령은 그의 인자를 문서 내에서 호출된 곳과
\file{.toc} 파일 모두에 추가한다. 예를 들어:
%     The command \cmd{\chapterprecis} prints its argument 
% both at the
% point in the document where it is called, and also adds it to the \file{.toc}
% file. For example:
 \begin{lcode}
 ...
 \chapter{}%  first chapter
 \chapterprecis{Our hero is introduced; family tree; early days.}
 ...
 \end{lcode}

%    Now for the details.
이제 구체적인 내용을 소개하겠다.

\begin{syntax}
\lnc{\prechapterprecisshift} \\
\end{syntax}
\glossary(prechapterprecisshift)%
  {\cs{prechapterprecisshift}}%
  {장 헤딩과 장 요약문 사이의 공백.}
%  {Spacing between a chapter head and a chapter precis.}
\lnc{\prechapterprecisshift} 길이는 \cmd{\chapterprecis} 이전의 수직 공백의 양을 조절한다.
만약 요약문이 \cmd{\chapter} 뒤에 바로 나온다면,
\Lopt{article} 클래스 옵션이 사용되었는지의 여부에 따라
다른 공백의 양이 사용되어야 할 것이다.
클래스는 그를 다음과 같이 설정한다:
%The length \lnc{\prechapterprecisshift} controls the vertical spacing before
%a \cmd{\chapterprecis}. If the precis immediately follows a \cmd{\chapter}
%then a different space is required depending on whether or not the
%\Lopt{article} class option is used. The class sets:
\begin{lcode}
\ifartopt
  \setlength{\prechapterprecisshift}{0pt}
\else
  \setlength{\prechapterprecisshift}{-2\baselineskip}
\fi
\end{lcode}


\begin{syntax}
\cmd{\chapterprecishere}\marg{text} \\
\cmd{\chapterprecistoc}\marg{text} \\
\end{syntax}
\glossary(chapterprecishere)%
  {\cs{chapterprecishere}\marg{text}}%
  {\meta{text}를 장 요약문에 식자한다.}
%  {Typesets \meta{text} for a chapter precis.}
\glossary(chapterprecistoc)%
  {\cs{chapterprecistoc}\marg{text}}%
  {\meta{text}를 장 요약문을 위해 목차(ToC)에 추가한다.}
%  {Adds \meta{text} for a chapter precis to the ToC.}
\cmd{\chapterprecis} 명령은 이 두 명령을 호출하여
문서 내에 \meta{text}를 출력하고(\cmd{\chapterprecishere} 명령),
\toc{} 에도 넣는다. (\cmd{\chapterprecistoc} 명령). 
필요할 경우 둘은 따로 사용될 수 있다.
% The \cmd{\chapterprecis} command calls these two commands to print the
% \meta{text} in the document (the \cmd{\chapterprecishere} command) 
% and to put it into the \toc{} (the \cmd{\chapterprecistoc} command). 
% These can be used individually if required.

\begin{syntax}
\cmd{\precisfont} \\
\cmd{\prechapterprecis} \cmd{\postchapterprecis} \\
\end{syntax}
\cmd{\chapterprecishere} 매크로는 \cmd{\chapter} 바로 뒤에 사용될 목적으로 있다.
\meta{text} 인자가 \cmd{\precisfont} 폰트로 \Ie{quote} 환경 안에서 식자된다.
이 매크로의 정의는 다음과 같다:
%The \cmd{\chapterprecishere} macro is intended for use immediately after 
%a \cmd{\chapter}. The \meta{text} argument is typeset in
%the \cmd{\precisfont} font in a \Ie{quote} environment. The macro's 
%definition is:
\begin{lcode}
\newcommand{\chapterprecishere}[1]{%
  \prechapterprecis #1\postchapterprecis}
\end{lcode}
여기서 \cmd{\prechapterprecis}, \cmd{\postchapterprecis}, \cmd{\precisfont}
의 정의는 다음과 같다:
%where \cmd{\prechapterprecis}, \cmd{\postchapterprecis} and \cmd{\precisfont}
%are defined as:
\begin{lcode}
\newcommand{\prechapterprecis}{%
  \vspace*{\prechapterprecisshift}%
  \begin{quote}\precisfont}
\newcommand{\postchapterprecis}{\end{quote}}
\newcommand*{\precisfont}{\normalfont\itshape}
\end{lcode}
이들의 전부 혹은 일부는 필요에 따라 수정될 수 있다.
%Any or all of these can be changed if another style of typesetting is required.

\LMnote{2010/06/09}{Changed a bit, and moved the glossary stuff here
  where it naturally belong}
다음의 매크로들은 목차(ToC) 내에서의 요약문의 형식을 설정한다.
%Next the following macros control the formatting of the precis text in
%the ToC.
\begin{syntax}
\cmd{\precistoctext}\marg{text} \cmd{\precistocfont} \cmd{\precistocformat} \\
\end{syntax}
\glossary(precistoctext)%
  {\cs{precistoctext}\marg{text}}%
  {ToC에 들어갈 장 요약문 \meta{text}}
%  {ToC entry for chapter precis \meta{text}}
\glossary(precistocfont)%
  {\cs{precistocfont}}%
  {\cs{precistoctext} 식자에 사용될 폰트.}
%  {Font for typesetting \cs{precistoctext}.}
\glossary(precistocformat)%
  {\cs{precistocformat}}%
  {\cs{precistoctext} 식자에 사용될 형식.}
%  {Format for typesetting \cs{precistoctext}.}
\cmd{\chapterprecistoc} 매크로는 \cmd{\precistoctext}\marg{text}를
\pixfile{toc} 파일에 넣는다.
기본값 정의는 다음과 유사하다:
(하지만 정확히 동일하지는 않다.%
\footnote{내부적으로는 \cs{leftskip}와 \cs{rightskip}에 대해 다른 이름을 사용하여
우에서 좌로 진행되는 문서들을 조판하거나 \texttt{bidi} 패키지를 사용하기 편하게 한다.}
)
%The \cmd{\chapterprecistoc} macro puts \cmd{\precistoctext}\marg{text} into 
%the \pixfile{toc} file. The default definition similar to (but not
%exactly\footnote{Internally we use a different name for \cs{leftskip}
%  and \cs{rightskip} to make it easier to do right to left documents
%with the class and the \texttt{bidi} package.})
\begin{lcode}
\DeclareRobustCommand{\precistoctext}[1]{%
  {\nopagebreak\leftskip \cftchapterindent\relax
    \advance\leftskip \cftchapternumwidth\relax
    \rightskip \@tocrmarg\relax
    \precistocformat\precistocfont #1\par}}
\end{lcode}
\toc{} \cmd{\precistoctext}는 그의 인자를 \cmd{\precistocfont} (기본값 \cmd{\itshape})와
\cmd{\precistocformat} (기본값 \cmd{\noindent})를 사용하여
장 제목처럼 식자하게 된다.
%Effectively, in the \toc{} \cmd{\precistoctext} typesets its argument like 
%a chapter title using the \cmd{\precistocfont} (default
%\cmd{\itshape}), and \cmd{\precistocformat} (default \cmd{\noindent}).

\index{장!요약문|)}
\index{chapter!precis|)}


\section{Lower level headings}

\index{heading!sections|(}

    The lower level heads --- sections down to subparagraphs --- are also
configurable, but there is nothing corresponding to chapter styles.

    There are essentially three things that may be adjusted for these heads:
(a) the vertical distance between the baseline of the text above the heading to
the baseline of the title text, (b) the indentation of the heading from the
left hand margin\index{margin!left}, and (c) the style (font) used for the 
heading title. 
Additionally, a heading may be run-in to the text or as a display before 
the following text;
in the latter case the vertical distance between the heading and the 
following text may also be adjusted. Figure~\ref{fig:displaysechead} shows the
parameters controlling a displayed sectional heading and \fref{fig:runsechead}
shows the parameters for a run-in heading. The default values of the
parameters for the different heads are in \tref{tab:defdisplaySvals} for
the display heads and \tref{tab:defruninSvals} for the run-in heads.


\begin{figure}
\centering
\setlayoutscale{1}
\drawparameterstrue
\drawheading{}
\caption{Displayed sectional headings} \label{fig:displaysechead}
\end{figure}

\begin{figure}
\centering
\setlayoutscale{1}
\drawparameterstrue
\runinheadtrue
\drawheading{}
\caption{Run-in sectional headings} \label{fig:runsechead}
\end{figure}

\begin{table}
\centering
\caption{Default display sectioning layout parameter values}\label{tab:defdisplaySvals}
\begin{tabular}{lccc} \toprule
  & section & subsection & subsubsection \\ \midrule
beforeskip (-ex) & 3.5+1-.2 & 3.25+1-.2 & 3.25+1-.2 \\
indent          &    0     &      0    &     0     \\
afterskip (ex)  & 2.3+.2   & 1.5+.2    & 1.5+.2 \\
font   & \cs{Large}\cs{bfseries} & \cs{large}\cs{bfseries} & \cs{bfseries} \\
\bottomrule
\end{tabular}
\end{table}


\begin{table}
\centering
\caption{Default run-in sectioning layout parameter values}\label{tab:defruninSvals}
\begin{tabular}{lcc} \toprule
  & paragraph & subparagraph \\ \midrule
beforeskip (+ex) & 3.25+1-.2 & 3.25+1-.2 \\
indent          &    0      &  \cs{parindent} \\
afterskip       & -1em       & -1em       \\
font            & \cs{bfseries} & \cs{bfseries} \\
\bottomrule
\end{tabular}
\end{table}


    In the following I will use \texttt{S} to stand for one of \texttt{sec},
\texttt{subsec}, \texttt{subsubsec}, \texttt{para} or \texttt{subpara}, 
which are in turn shorthand for \texttt{section} through to 
\texttt{subparagraph}, as summarised in \tref{tab:Sshort}.

\begin{table}
\centering
\caption{Values for \texttt{S} in section styling macro names.} \label{tab:Sshort}
\ttfamily
\begin{tabular}{llllll}\toprule
S & sec & subsec & subsubsec & para & subpara \\
  & section & subsection & subsubsection & paragraph & subparagraph \\ 
\bottomrule
\end{tabular}
\end{table}

\begin{syntax}
\cmd{\setbeforeSskip}\marg{skip} \\
\end{syntax}
\glossary(setbeforeSskip)%
  {\cs{setbeforeSskip}\marg{skip}}%
  {Sets the \cs{beforeskip} for an S head.}
\glossary(setbeforesecskip)%
  {\cs{setbeforesecskip}\marg{skip}}%
  {Sets the \cs{beforeskip} for a \cs{section} head.}
\glossary(setbeforesubsecskip)%
  {\cs{setbeforesubsecskip}\marg{skip}}%
  {Sets the \cs{beforeskip} for a \cs{subsection} head.}
\glossary(setbeforesubsubsecskip)%
  {\cs{setbeforesubsubsecskip}\marg{skip}}%
  {Sets the \cs{beforeskip} for a \cs{subsubsection} head.}
\glossary(setbeforeparaskip)%
  {\cs{setbeforeparaskip}\marg{skip}}%
  {Sets the \cs{beforeskip} for a \cs{paragraph} head.}
\glossary(setbeforesubparaskip)%
  {\cs{setbeforesubparaskip}\marg{skip}}%
  {Sets the \cs{beforeskip} for a \cs{subparagraph} head.}
The absolute value of the \meta{skip} length argument is the space to leave
above the heading. If the actual value is negative then the first line 
after the heading will not be indented. The default \meta{skip} depends on the
particular level of heading, but for a \cmd{\section} 
(i.e., when \verb?S = sec?) it is 
\begin{lcode}
-3.5ex plus -1ex minus -.2ex
\end{lcode}
where the plus and minus values are the
allowable stretch and shrink; note that all the values are negative so that 
there is no indentation of the following text. If you wanted indentation then
you could do
\begin{lcode}
\setbeforesecskip{3.5ex plus 1ex minus .2ex}
\end{lcode}


\begin{syntax}
\cmd{\setSindent}\marg{length} \\
\end{syntax}
\glossary(setSindent)
  {\cs{setSindent}\marg{length}}%
  {Sets the \cs{indent} for an S head.} 
\glossary(setsecindent)
  {\cs{setsecindent}\marg{length}}%
  {Sets the \cs{indent} for a \cs{section} head.} 
\glossary(setsubsecindent)
  {\cs{setsubsecindent}\marg{length}}%
  {Sets the \cs{indent} for a \cs{subsection} head.} 
\glossary(setsubsubsecindent)
  {\cs{setsubsubsecindent}\marg{length}}%
  {Sets the \cs{indent} for a \cs{subsubsection} head.} 
\glossary(setparaindent)
  {\cs{setparaindent}\marg{length}}%
  {Sets the \cs{indent} for a \cs{paragraph} head.} 
\glossary(setsubparaindent)
  {\cs{setsubparaindent}\marg{length}}%
  {Sets the \cs{indent} for a \cs{subparagraph} head.} 
The value of the \meta{length} length argument is the indentation of
the heading (number and title) from the lefthand margin\index{margin!left}. 
This is normally 0pt.

\begin{syntax}
\cmd{\setSheadstyle}\marg{font} \\
\end{syntax}
\glossary(setSheadstyle)%
  {\cs{setSheadstyle}\marg{font}}%
  {Sets the style (font) for an S head.}
\glossary(setsecheadstyle)%
  {\cs{setsecheadstyle}\marg{font}}%
  {Sets the style (font) for a section head.}
\glossary(setsubsecheadstyle)%
  {\cs{setsubsecheadstyle}\marg{font}}%
  {Sets the style (font) for a subsection head.}
\glossary(setsubsubsecheadstyle)%
  {\cs{setsubsubsecheadstyle}\marg{font}}%
  {Sets the style (font) for a subsubsection head.}
\glossary(setparaheadstyle)%
  {\cs{setparaheadstyle}\marg{font}}%
  {Sets the style (font) for a paragraph head.}
\glossary(setsubparaheadstyle)%
  {\cs{setsubparaheadstyle}\marg{font}}%
  {Sets the style (font) for a subparagraph head.}
This macro specifies the style (font) for the sectional number and title. 
As before, the default value of the \meta{font} argument depends on the
level of the heading. For a \cmd{\subsection} (i.e., \verb?S=subsec?) it is
\verb?\large\bfseries\raggedright?, to typeset in the \cmd{\bfseries} font
in the \cmd{\large} size; the title will also be set ragged right (i.e.,
there will be no hyphenation in a multiline title).

    Note that the very last element in the \meta{font} argument may be a 
macro that takes one argument (the number and title of the heading). So,
if for some reason you wanted \cmd{\subsubsection} titles to be all uppercase,
centered, and in the normal font, you can do
\begin{lcode}
\setsubsubsecheadstyle{\normalfont\centering\MakeUppercase}
\end{lcode}

    As another example, although I don't recommend this, you can draw a
horizontal line under section titles via:
\begin{lcode}
\newcommand{\ruledsec}[1]{%
  \Large\bfseries\raggedright #1 \rule{\textwidth}{0.4pt}}
\setsecheadstyle{\ruledsec}
\end{lcode}


\begin{syntax}
\cmd{\setafterSskip}\marg{skip} \\
\end{syntax}
\glossary(setafterSskip)%
  {\cs{setafterSskip}\marg{skip}}%
  {Sets the \cs{afterskip} for an S head.}
\glossary(setaftersecskip)%
  {\cs{setaftersecskip}\marg{skip}}%
  {Sets the \cs{afterskip} for a section head.}
\glossary(setaftersubsecskip)%
  {\cs{setaftersubsecskip}\marg{skip}}%
  {Sets the \cs{afterskip} for a subsection head.}
\glossary(setaftersubsubsecskip)%
  {\cs{setaftersubsubsecskip}\marg{skip}}%
  {Sets the \cs{afterskip} for a subsubsection head.}
\glossary(setafterparaskip)%
  {\cs{setafterparaskip}\marg{skip}}%
  {Sets the \cs{afterskip} for a paragraph head.}
\glossary(setaftersubparaskip)%
  {\cs{setaftersubparaskip}\marg{skip}}%
  {Sets the \cs{afterskip} for a subparagraph head.}
If the value of the \meta{skip} length argument is positive it is the space 
to leave between the display heading and the following text. If it is negative,
then the heading will be run-in and the value is the horizontal space
between the end of the heading and the following text.
The default \meta{skip} depends on the
particular level of heading, but for a \cmd{\section} (i.e., when 
\verb?S = sec?) it is \verb?2.3ex plus .2ex?, 
and for a \cmd{\subparagraph} (i.e., \verb?S = subpara?), which is a run-in 
heading, it is \verb?-1em?.

\fancybreak{$*$}
\fancybreak{}

\begin{syntax}
\cmd{\@hangfrom}\marg{code} \\
\cmd{\sethangfrom}\marg{code} \\
\end{syntax}
\glossary(hangfromat)%
  {\cs{@hangfrom}\marg{code}}%
  {Kernel macro aiding the setting hanging paragraphs.}
\glossary(sethangfrom)%
  {\cs{sethangfrom}\marg{code}}%
  {User macro redefining \cs{@hangfrom} to \meta{code}.}
Internally all the titling macros use a macro called \cmd{\@hangfrom} which
by default makes multiline titles look like a hanging paragraph\index{paragraph!hanging}. The
default definition of \cmd{\@hangfrom} (in file \file{ltsect.dtx}) is
effectively:
\begin{lcode}
\newcommand{\@hangfrom}[1]{\setbox\@tempboxa\hbox{{#1}}%
  \hangindent \wd\@tempboxa\noindent\box\@tempboxa}
\end{lcode}
The argument is put into a box and its width is measured, then a hanging
paragraph\index{paragraph!hanging} is started with the argument as the 
first thing and second and later lines indented by the argument's width.

The \cmd{\sethangfrom} macro redefines \cmd{\@hangfrom} to be \meta{code}.
For example, to have the titles set as block paragraphs\index{paragraph!block} instead of hanging
paragraphs\index{paragraph!hanging}, simply do:
\begin{lcode}
\sethangfrom{\noindent #1}
\end{lcode}
Note that you have to use \verb?#1? at the position in the replacement
code for \cmd{\@hangfrom} where the argument to \cmd{\@hangfrom}
is to be located.

\begin{syntax}
\cmd{\@seccntformat}\marg{code} \\
\cmd{\setsecnumformat}\marg{code} \\
\end{syntax}
\glossary(seccntformatat)%
  {\cs{@seccntformat}\marg{code}}%
  {Kernel macro that formats the number in a sectional head.}
\glossary(setsecnumformat)%
  {\cs{setsecnumformat}\marg{code}}%
  {Redefines \cs{@seccntformat} to \meta{code}.}
Internally all the titling macros use a kernel macro called 
\cmd{\@seccntformat} 
which defines the formatting of sectional numbers in a title. Its
default definition (in file \file{ltsect.dtx}) is effectively:
\begin{lcode}
\newcommand{\@seccntformat}[1]{\csname the#1\endcsname\quad}
\end{lcode}
which formats the sectional numbers as \verb?\thesec...? with a space 
afterwards.
The command \cmd{\setsecnumformat} redefines \cmd{\@seccntformat} 
to be \meta{code}.
For example, to put a colon and space after the number
\begin{lcode}
\setsecnumformat{\csname the#1\endcsname:\quad}
\end{lcode}
Note that you have to use \verb?#1? where you want the argument 
(sectional number) of \cmd{\@seccntformat} to go.

Note that \cmd{\setsecnumformat} applies to all \cmd{\section},
\cmd{\subsection}, etc. If you want to change it only for, say,
\cmd{\subsection}, use the \cmd{\setsubsechook} described below.


\fancybreak{$*$}
\fancybreak{}

\begin{syntax}
\cmd{\hangsecnum} \\
\cmd{\defaultsecnum} \\
\end{syntax}
\glossary(hangsecnum)%
  {\cs{hangsecnum}}%
  {Declaration making sectioning numbers hang in the margin.}
\glossary(defaultsecnum)%
  {\cs{defaultsecnum}}%
  {Declaration reversing the effect of \cs{hangsecnum}.}
The macro \cmd{\hangsecnum} is a declaration that makes sectional numbers hang
in the margin\index{margin!left}. The macro \cmd{\defaultsecnum} is a declaration that reverses the
effect of \cmd{\hangsecnum}, that is, sectional numbers will be typeset in 
their familiar places.


\begin{syntax}
\cmd{\Shook} \\
\cmd{\setShook}\marg{text} \\
\end{syntax}
\glossary(Shook)%
  {\cs{Shook}}%
  {Hook called immediately before typesetting the title of an S head.}
\glossary(setShook)%
  {\cs{setShook}\marg{text}}%
  {Redefines \cs{Shook} to be \meta{text}.}
\glossary(sechook)%
  {\cs{sechook}}%
  {Hook called immediately before typesetting the title of a section head.}
\glossary(setsechook)%
  {\cs{setsechook}\marg{text}}%
  {Redefines \cs{sechook} to be \meta{text}.}
\glossary(subsechook)%
  {\cs{subsechook}}%
  {Hook called immediately before typesetting the title of a subsection head.}
\glossary(setsubsechook)%
  {\cs{setsubsechook}\marg{text}}%
  {Redefines \cs{subsechook} to be \meta{text}.}
\glossary(subsubsechook)%
  {\cs{subsubsechook}}%
  {Hook called immediately before typesetting the title of a subsubsection head.}
\glossary(setsubsubsechook)%
  {\cs{setsubsubsechook}\marg{text}}%
  {Redefines \cs{subsubsechook} to be \meta{text}.}
\glossary(parahook)%
  {\cs{parahook}}%
  {Hook called immediately before typesetting the title of a paragraph head.}
\glossary(setparahook)%
  {\cs{setparahook}\marg{text}}%
  {Redefines \cs{parahook} to be \meta{text}.}
\glossary(subparahook)%
  {\cs{subparahook}}%
  {Hook called immediately before typesetting the title of a subparagraph head.}
\glossary(setsubparahook)%
  {\cs{setsubparahook}\marg{text}}%
  {Redefines \cs{subparahook} to be \meta{text}.}
The macro \cmd{\Shook} is called immediately before the typesetting of the
title; its default definition does nothing. The macro \cmd{\setShook} 
redefines \cmd{\Shook} to be \meta{text}. You can use this hook, for example,
to insert a \cmd{\sethangfrom} or \cmd{\setsecnumformat} command into the
definition of a particular section division command. In that case,
remember that if you want to refer to the \verb|#1| argument, in the
argument for \cmd{\setsecnumformat}, then you have to double the
\verb|#|, i.e. use \verb|##1|, see the example below.

Here are some example lower level heads and the code used to produce them.



\newcommand*{\shortcenter}[1]{%
  \sethangfrom{\noindent ##1}%
  \normalfont\boldmath\bfseries
  \centering
  \parbox{3in}{\centering #1}\par}

\begin{egsource}{egsubheads}
\setsubsubsecheadstyle{\bfseries\raggedright}
    \subsubsection*{Bold raggedright}
\setsubsubsecheadstyle{\scshape\raggedright}
    \subsubsection*{Small caps raggedright}
\setsubsubsecheadstyle{\itshape\raggedright}
    \subsubsection*{Italic raggedright}
\setsubsubsecheadstyle{\Large\centering}
    \subsubsection*{Large centered}
\setsubsubsecheadstyle{\large\centering\MakeUppercase}
    \subsubsection*{large centered uppercase}
\setsubsubsecheadstyle{\bfseries\centering}
    \subsubsection*{Bold centered}
\setsubsubsecheadstyle{\scshape\centering}
    \subsubsection*{Small caps centered}
\setsubsubsecindent{2\parindent}
\setsubsubsecheadstyle{\scshape\raggedright}
    \subsubsection*{Small caps indented}
\setsubsubsecindent{0pt}
\setsubsubsecheadstyle{\itshape\raggedleft}
    \subsubsection*{Italic flushright}
\newcommand*{\shortcenter}[1]{%
  \sethangfrom{\noindent ##1}%
  \normalfont\boldmath\bfseries
  \centering
  \parbox{3in}{\centering #1}\par}
\setsubsubsecheadstyle{\shortcenter}
\subsubsection*{Bold centered but taking up no more than 3 inches
                if a long title}
\end{egsource}

\FloatBlock


\begin{egresult}[A variety of subhead styles]{egsubheads}
\setsubsubsecheadstyle{\bfseries\raggedright}
\subsubsection*{Bold raggedright}

\setsubsubsecheadstyle{\scshape\raggedright}
\subsubsection*{Small caps raggedright}

\setsubsubsecheadstyle{\itshape\raggedright}
\subsubsection*{Italic raggedright}

\setsubsubsecheadstyle{\Large\centering}
\subsubsection*{Large centered}

\setsubsubsecheadstyle{\large\centering\MakeUppercase}
\subsubsection*{large centered uppercase}

\setsubsubsecheadstyle{\bfseries\centering}
\subsubsection*{Bold centered}

\setsubsubsecheadstyle{\scshape\centering}
\subsubsection*{Small caps centered}

\setsubsubsecindent{2\parindent}
\setsubsubsecheadstyle{\scshape\raggedright}
\subsubsection*{Small caps indented}

\setsubsubsecindent{0pt}
\setsubsubsecheadstyle{\itshape\raggedleft}
\subsubsection*{Italic flushright}

\setsubsubsecheadstyle{\shortcenter}
\subsubsection*{Bold centered but taking up no more than 3 inches
                if a long title}
\end{egresult}

%%%%%%%%%%%%%%%%%%%%%%%%%%%%%%%%%%%%%%%%%%%%%%%%%%%%%%%%
%%% back to default settings
%\setsubsubsecindent{0pt}
%\setsubsubsecheadstyle{\bfseries\raggedright}
%%%%%%%%%%%%%%%%%%%%%%%%%%%%%%%%%%%%%%%%%%%%%%%%%%%%%%%%



\LMnote{2015/02/23}{Replaced \cs{mbox} with \cs{leavevmode} in def of
\cs{marginbox}, suggested in \url{http://tex.stackexchange.com/a/222178/3929} }

\newcommand{\marginbox}[1]{%
  \parbox[t][0pt]{6em}{\itshape\raggedleft\leavevmode #1}}
\newcommand{\marginhead}[1]{%
  {\llap{\marginbox{#1}\kern0.5em}}}
\setparaindent{0em}
\setafterparaskip{0em}
\setparaheadstyle{\marginhead}
\setparahook{\setsecnumformat{\csname the##1\endcsname\ }}
%%%\setsecnumdepth{paragraph}
\paragraph{Hang the whole heading in the margin}%%%\setsecnumdepth{subsection}
A less traditional style is to put the whole heading into the margin.
I have done this here for a \cmd{\paragraph} heading (which is not otherwise
used in this manual). The code is:
\begin{lcode}
\newcommand{\marginbox}[1]{%
  \parbox[t][0pt]{6em}{\itshape\raggedleft\leavevmode #1}}
\newcommand{\marginhead}[1]{%
  {\llap{\marginbox{#1}\kern0.5em}}}
\setparaindent{0em}
\setafterparaskip{0em}
\setparaheadstyle{\marginhead}
\setparahook{\setsecnumformat{\csname the##1\endcsname\ }}
\paragraph{Hang the whole heading in the margin}%
\end{lcode}
The macro \cmd{\marginbox} puts its argument, raggedleft, into a zero height
\cmd{\parbox} of width 6em, aligned at the top. 
The \cmd{\marginhead} macro puts its argument into a 
\cmd{\marginbox} and puts the \cmd{\marginbox} 0.5em to the left. 
The \cmd{\paragraph} head style is then set to use \cmd{\marginhead} to 
typeset the heading. The format for the number is reset via 
\cmd{\setparahook} and \cmd{\setsecnumformat}.


\fancybreak{$*$}
\fancybreak{}


A different approach is to create new macros, each named by the type
of sectional macro it formats, and then make the number format call
these macros.
In this example we will provide separate formatting for \cmd{\section} and \cmd{\subsection}. 
\begin{lcode}
  \setsecnumformat{\csname #1secnumformat\endcsname}
  \newcommand\sectionsecnumformat{\thesection:\quad} 
  \newcommand\subsectionsecnumformat{\fbox{\enspace\thesubsection\enspace}\enspace}
\end{lcode}
Since the macro is only called in the proper context, we can use
\cmd{\thesection} directly in the code for \cmd{\section}.



\index{heading!sections|)}

%%%%%%%%%%%%%%%%%%%%%%%%%%%%%%%%%%%%%%%%%%%%%%%%%%%%%%%%%%%%%%%



%\section{Fancy anonymous breaks}
\section{번호 없는 장절구획 꾸미기}

%    Often, in novels, there is a need to break up the text to indicate that
%there is a major break in the story, but not enough to warrant starting a new
%chapter. I have called these 
%\emph{anonymous divisions}\index{anonymous division}\index{division!anonymous} 
%as there is neither number nor title associated with them. 
소설을 쓸 때 종종 본문의 텍스트 사이를 쪼개
이야기의 흐름이 끝났다는 것을 보여줘야 하지만
그렇다고 아예 새로운 장을 열기에는 애매한 경우가 있다.
이 구획은 번호나 제목이 붙지 않기 때문에,
이를 \emph{번호 없는 장절구획}\index{번호 없는 장절구획}\index{division!anonymous} \index{anonymous division}
으로 부르곤 한다.
    
\begin{syntax}
\cmd{\plainbreak}\marg{num} \cmd{\plainbreak*}\marg{num}  \\
\cmd{\fancybreak}\marg{text} \cmd{\fancybreak*}\marg{text}   \\
\end{syntax} 
\glossary(plainbreak)%
  {\cs{plainbreak}\marg{num}}%
  {번호 없이 \meta{num} 개의 빈 줄로 이루어진 구획. 이어지는 문단은 들여써지지 않는다.
  }
%  {An anonymous division of \meta{num} blank lines and the following
%   paragraph is not indented.}
\glossary(plainbreak*)%
  {\cs{plainbreak*}\marg{num}}%
  {이어지는 문단이 들여써지지 않는다는 것을 제외하고 \cs{plainbreak}와 동일.}
%  {Like \cs{plainbreak} except that the following paragraph is indented.}
\glossary(fancybreak)%
  {\cs{fancybreak}\marg{text}}%
  {중앙정렬된 \meta{text}로 이루어진 구획. 이어지는 문단은 들여써지지 않는다.}
%  {An anonymous division with \meta{text} centered and the following
%   paragraph is not indented.}
\glossary(fancybreak*)%
  {\cs{fancybreak*}\marg{text}}%
  {이어지는 문단이 들여써지지 않는다는 것을 제외하고 \cs{fancybreak}와 동일.}
%  {Like \cs{fancybreak} except that the following paragraph is indented.}

%    The \cmd{\plainbreak} is an anonymous division. It puts \meta{num}
%blank lines into the typescript and the first line of the following
%paragraph is not indented\index{paragraph!indentation}. Another anonymous 
%division is \cmd{\fancybreak} which puts \meta{text} centered into the 
%typescript and the initial line of the following paragraph is not 
%indented\index{paragraph!indentation}. For
%example:
\cmd{\plainbreak} 는 번호 없는 장절구획이다.
이는 typescript 내에 \meta{num}개의 빈 줄을 삽입하며,
이어지는 문단의 첫 줄이 들여써지지\index{paragraph!indentation}\index{들여쓰기} 않는다.
다른 번호 없는 장절구획 명령에는 \cmd{\fancybreak}가 있다.
이는 중앙정렬된 \meta{text}를 typescript 내에 삽입하며,
마찬가지로 이어지는 문단의 첫 줄이 들여써지지\index{paragraph!indentation}\index{들여쓰기} 않는다.
예를 들어:
\begin{lcode}
\fancybreak{{*}\\{* * *}\\{*}}
\end{lcode}
은 작은 별 기호(asterisk)들로 이루어진 작은 마름모 모양을 만든다.
%typesets a little diamond made of asterisks.

%    The starred versions of the commands indent the first line of the
%following paragraph\index{paragraph!indentation}.
스타가 붙은 버전의 명령어들은 첫 줄을 들여쓴다.\index{paragraph!indentation}\index{들여쓰기}

\begin{syntax}
\cmd{\plainfancybreak}\marg{space}\marg{num}\marg{text} \\
\cmd{\plainfancybreak*}\marg{space}\marg{num}\marg{text} \\
\end{syntax}
\glossary(plainfancybreak)%
  {\cs{plainfancybreak}\marg{space}\marg{num}\marg{text}}%
  {페이지의 윗부분에 있을 경우 \cs{fancybreak}처럼,
  아래부분에 있을 경우 페이지에 남은 공간이 \meta{space}보다 적을 경우
	\cs{plainbreak}처럼 작동하는 번호 없는 장절구획.}
%  {An anonymous division that acts like \cs{fancybreak} at the top
%   of the page, or at the bottom if there is less than \meta{space}
%   left on the page, otherwise it acts like \cs{plainbreak}.
%   The following paragraph is not indented.}
\glossary(plainfancybreak*)%
  {\cs{plainfancybreak*}\marg{space}\marg{num}\marg{text}}%
  {이어지는 문단이 들여써지지 않는다는 것을 제외하고 \cs{plainfancybreak}와 동일.}
%  {Like \cs{plainfancybreak} except that the following paragraph is indented.}
만약 구획이 페이지의 맨 위나 아래에 등장한다면,
읽는 이는 내용의 구획이 있음을 알아차리기 어렵게 된다.
만약 어떤 페이지에 본문 뒤에 구획을 넣고 나서도
다른 텍스트를 넣을 수 있을 정도로 충분한 공간이 있다면,
\cmd{\plainfancybreak} 명령어는 \meta{num}개의 줄로 이루어진 
\cmd{\plainbreak}를 사용하게 될 것이다.
그렇지 않다면, (구획이 페이지의 맨 위나 아래에 등장한다면)
이는 \meta{text}로 구성된 \cmd{\fancybreak}를 사용하게 될 것이다.
\meta{space} 인자는 \meta{num}개의 빈 줄이나
구획 뒤의 본문을 이루는 텍스트의 줄 수 만큼의 높이를 나타낸다.
스타가 붙은 버전의 명령어들은 스타가 붙은 버전의 \cmd{\plainbreak}와
\cmd{\fancybreak}를 사용한다.
%If a plain break comes at the top or bottom of a page then it is very 
%difficult for a reader to discern that there is a break at all.
%If there is text on the page and enough space left to put some text
%after a break the \cmd{\plainfancybreak} command will use a \cmd{\plainbreak}
%with \meta{num} lines, 
%otherwise (the break would come at the top or bottom of the page) it
%will use a \cmd{\fancybreak} with \meta{text}. The \meta{space} argument is a
%length specifying the space needed for the \meta{num} blank lines and some
%number of text lines for after the plain break. The starred version of
%the command uses the starred versions of the \cmd{\plainbreak} and
%\cmd{\fancybreak} commands.

% 김강수 선생님의 http://wiki.ktug.org/wiki/wiki.php/MemucsManual?action=download&value=memucs-manual.pdf 
% 에 나와있는 번역문을 사용
불행하게도 사용자가 요구한 plain, fancy break 공간 사이에는 상호간섭이 있다. $ P $가
plain break를 위해서 요청된 공간(행)이고 $ F $가 fancy break를 위해서 요구되는 공간(행)
이라고 하자. $ S $는 \meta{space} 공간(행)이다. 몇 번의 실험을 통해, plain break 페이지의 위쪽
과 아래쪽을 피할 조건은 $ S −P > 1 $이라는 사실을 알게 되었다. 또한, fancy break가 페이
지의 중간에 놓이지 않을 조건(즉, 상단과 하단이 아닌 위치에 올 조건)은 $ S − F < 3 $이다.
예를 들어서, plain break와 fancy break가 동일한 수직 공간을 취하는 경우가 $ S = P + 2 $
이 만족되면 일어난다. 일반적으로 $ F = P +n $이면 그 조건은 $ 1 < S −P < 3+n $이고, 이럴
경우에 \cmd{\plainfancybreak} 명령은 fancy break가 항상 plain break에 필요한 공간 만큼을
취하게 된다.
%    Unfortunately there is an interaction between the requested, plain,
%and fancy break spaces.
%    Let $P$ be the space (in lines) required for the plain break, 
%$F$ the space (in lines)
%required for the fancy break, and $S$ the \meta{space} argument (in lines). 
%From some experiments it appears that the condition for the plain break 
%to avoid the top and bottom of the page is that $S - P > 1$. 
%Also, the condition for the fancy break to avoid being put in the middle 
%of a page (i.e., not at the top or bottom) is that  $S - F < 3$.
%For example, if the plain and fancy breaks take the same vertical space
%then $S = P +2$ is the only value that matches the conditions. In general, 
%if $F = P + n$ then the condition is $1 < S-P < 3+n$, which means that 
%for the \cmd{\plainfancybreak} command the
%fancy break must always take at least as much space as the plain break.


\fancybreak{\pfbreakdisplay}

\cmd{\plainfancybreak} 매크로는 페이지의 중간에 plain break 을 넣거나,
구획이 페이지의 맨 위나 아래에 등장한다면 fancy break 를 넣는다.
%    The \cmd{\plainfancybreak} macro inserts a plain break in the middle of
%a page or if the break would come at the bottom or top of a page it
%inserts a fancy break instead.

\begin{syntax}
\cmd{\pfbreak} \cmd{\pfbreak*} \\
\lnc{\pfbreakskip} \\
\cmd{\pfbreakdisplay}\marg{text} \\
\end{syntax}
\glossary(pfbreak)
  {\cs{pfbreak}}%
  {An alternative for \cs{plainfancybreak} using \cs{pfbreakskip} 
   and \cs{pfbreakdisplay}.}
\glossary(pfbreak*)
  {\cs{pfbreak*}}%
  {An alternative for \cs{plainfancybreak*} using \cs{pfbreakskip} 
   and \cs{pfbreakdisplay}.}
\glossary(pfbreakskip)%
  {\cs{pfbreakskip}}%
  {Space for a \cs{pfbreak}'s \cs{plainbreak}.}
\glossary(pfbreakdisplay)%
  {\cs{pfbreakdisplay}\marg{text}}%
  {\meta{text} for a \cs{pfbreak}'s \cs{fancybreak}.}
  
%The \cmd{\pfbreak} macro is an alternate for \cmd{\plainfancybreak} that may
%be more convenient to use. The gap for the plain break is given by the
%length \lnc{\pfbreakskip} which is initialised to produce two blank lines.
%The fancy break, which takes the same vertical space, is given by the
%\meta{text} argument of \cmd{\pfbreakdisplay}. The default definition:
%\LMnote{2010/11/22}{typo, not spelled pfbt}
 \cmd{\pfbreak} 매크로는 \cmd{\plainfancybreak} 를 대신하여 더 편하게 사용할 수 있다.
 Plain break를 위한 공백은 \lnc{\pfbreakskip}로 나타내어지는 길이로 정해지는데,
 이는 두 개의 빈 줄을 만드는 것이 초기 설정이다.
 Fancy break는 동일한 높이만큼의 공간을 사용하지만,
\cmd{\pfbreakdisplay} 명령의  \meta{text} 인자를 사용한다.
정의된 기본값은 앞서 보인 것처럼 3개의 별표(asterisk)를 사용하는 것이다.
\begin{lcode}
\newcommand*{\pfbreakdisplay}{*\quad*\quad*}
\end{lcode}

\index{anonymous division!styling|(}

%%\renewcommand{\pfbreakdisplay}{\huge \ding{167}\quad\ding{167}\quad\ding{167}}
\renewcommand{\pfbreakdisplay}{%
 \ensuremath{\clubsuit\quad\diamondsuit\quad\clubsuit}}
\fancybreak{\pfbreakdisplay}

\cmd{\pfbreakdisplay}의 정의를 바꿔 다른 모양으로 보이게 할 수도 있다.
방금 보인 fancy break는 다음의 명령으로 만들어졌다:
%    You can change the definition of \cmd{\pfbreakdisplay} for a different
%style if you wish. The
%fancy break just before this was produced via:
\begin{lcode}
\renewcommand{\pfbreakdisplay}{%
  \ensuremath{\clubsuit\quad\diamondsuit\quad\clubsuit}}
\fancybreak{\pfbreakdisplay}
\end{lcode}
여기서는 \cmd{\fancybreak}를 사용했다.
구획이 페이지 나눔이 될지 어떨지 문서를 작성하는 중에는 알 수 없기 때문이다.
만약 여기서 \cmd{\pfbreak}를 사용했다면,
내가 보여주고자 했던 fancy display 대신 단순히 두 개의 빈 줄이 나타났을수도 있다.
%I used \cmd{\fancybreak} as I'm not sure where the break will come on the
%page and the simple \cmd{\pfbreak} macro might just have produced a couple
%of blank lines instead of the fancy display.

%  The paragraph following \cmd{\pfbreak} is not indented. If you want
%it indented use the \cmd{\pfbreak*} starred version.
\cmd{\pfbreak} 명령 뒤의 문단은 들여써지지 않으며, 만약 들여쓰지게 하고 싶다면
별표가 있는 버전 \cmd{\pfbreak*}를 사용하면 된다.

\renewcommand{\pfbreakdisplay}{\ding{167}\quad\ding{167}\quad\ding{167}}
\fancybreak{\pfbreakdisplay}

% The fancy break using 
%fleurons\index{fleuron} just before this paragraph was produced by:
방금 나타난 꽃문양의\index{꽃문양}\index{fleuron} fancy break는,
\Lpack{pifont} 패키지의 \cmd{\ding} 명령을 사용하여 아래의 코드로 만들어졌다:
\begin{lcode}
\renewcommand{\pfbreakdisplay}{%
  \ding{167}\quad\ding{167}\quad\ding{167}}
\fancybreak{\pfbreakdisplay}
\end{lcode}
%where the \cmd{\ding} command is from the \Lpack{pifont} package.

\makeatletter
\newcommand{\motif}[1]{\cleaders\hbox{#1}\hfil}
\newcommand{\chain}[2]{\leavevmode\hbox to #2{\motif{#1}}}
\newcommand{\diamondpattern}{\m@th$\mkern-.6mu \diamond \mkern-.6mu$}
\makeatother

\fancybreak{\chain{\diamondpattern}{0.25\textwidth}}

꽃문양을 이용한 fancy break는 지정하기 간단했다.
또한, \ltx\ 에서 사용할 수 있는 기호가 많기 때문에
잘 조합하여 위와 같이 예쁜 모양의 fancy break 를 만들어내는데에 사용할 수 있다.

%    The fancy break made with fleurons was simple to specify. There are 
%many other symbols that you can
%use in \ltx\ and these can be combined in potentially attractive ways to
%produce a fancy break like the one just above.

다음의 아이디어는 Christina Thiele~\cite{ORNAMENTAL} 에 의해 제안되었으며,
수학적 기호들을 묶는 데에 사용될 수 있다.
이는 목차(table of contents)에서의 dot leaders 와 같은 원리로 작동한다.
%    The following idea was originally suggested by Christina 
%Thiele~\cite{ORNAMENTAL}, and can
%be used to string together mathematical symbols. It works following the same
%principles as the dot leaders in the Table of Contents.

%    Define a macro called with the syntax \cs{motif}\marg{shape}, where
%\meta{shape} is a symbol or other shape to be repeated in a chain,
\cs{motif}\marg{shape} 와 같은 양식으로 매크로를 지정하라.
여기서 \meta{shape}는 체인 안에서 반복될 기호다.
\begin{lcode}
\newcommand{\motif}[1]{\cleaders\hbox{#1}\hfil}
\end{lcode}
\cs{motif}의 정의는 \tx 에서 따온 것이며, dot leaders 과 같은 것을 만드는 기본 재료다.
\cmd{\hbox}\marg{stuff} 는 \meta{stuff} 를 수평 박스에 넣으며,
\cmd{\cleaders}\meta{box} 는 지정된 양만큼의 공백을 채워넣는다.
만약 여러 개의 박스로 채워질 공간이 너무 많다면,
남는 공간은 골고루 분배될 것이다.
\cmd{\hfil} 는 잘 늘어나는 공간이다. 
즉, \cs{motif} 매크로는 \meta{shape} 들로 공간을 채우는 것이다.
%The definition of \cs{motif} is basically taken from \tx, and is part of the
%code for making things like dot leaders. \cmd{\hbox}\marg{stuff} puts
%\meta{stuff} into a horizontal box, and \cmd{\cleaders}\meta{box} fills
%a specified amount of space using whatever number of copies of the 
%\meta{box} as is needed; if there is
%too much space to be filled by a whole number of boxes, the spare space
%is spread around equally. \cmd{\hfil} is stretchy space. The \cs{motif} macro
%essentially says, fill up a space with with copies of \meta{shape}.

%    We also need another macro, \cs{chain}\marg{shape}\marg{length}, 
%where \marg{shape}
%is a shape to be repeated as many times as it takes to fill up a distance
%\meta{length}.
우리는 \cs{chain}\marg{shape}\marg{length} 와 같은 매크로도 필요로 한다.
여기에서 \marg{shape}는 \meta{length} 만큼의 공간을 채우기 위해 필요한 만큼
반복될 문양이다.
\begin{lcode}
\newcommand{\chain}[2]{\leavevmode\hbox to #2{\motif{#1}}}
\end{lcode}
\cmd{\leavevmode} 명령은 우리가 좌우로 식자하고 있음을 보장시켜주며,
\cmd{\hbox} \verb?to? \verb?<length>{stuff}? 는 \meta{stuff} 를
\meta{length} 만큼의 고정된 길이를 가진 수평 박스에 넣어준다.
%The \cmd{\leavevmode} makes sure that we are typesetting horizontally, and
%\cmd{\hbox} \verb?to? \verb?<length>{stuff}? puts \meta{stuff} into a horizontal 
%box with the
%fixed length of \meta{length}. Roughly, what \cs{chain} and \cs{motif}
%do together is typeset enough copies of \meta{shape} to make up
%a distance \meta{length}. 

%    That is what we have been aiming for. All that
%remains is to decide on what shape we might want to use. Here is one
%consisting of diamonds.
우리가 하려고 하던 것이 바로 그것이다.
이제 남은 일은 어떤 모양을 사용할지 결정하는 것이다.
다음은 마름모 문양으로 이루어진 한 예시이다:
\begin{lcode}
\makeatletter
\newcommand{\diamonds}{\m@th$\mkern-.6mu \diamond \mkern-.6mu$}
\makeatother
\end{lcode}
\cmd{\diamond} 기호는 math mode 에서만 사용될 수 있기 때문에
\verb?$...$?로 둘러싸여 있다. 
보통 \tx\ 은 수학 기호 주위에 약간의 공백을 주지만 \cmd{\m@th}은 그를 막아준다.
\cmd{\mkern}는 math mode 에서의 공백을 조정해주며,
이 경우 우리는 마름모 좌우에 있었을 공백을 제거한다.
\footnote{보통 올바른 커닝\label{fn:kerning}을 위한 적정값을 찾는 것은 실험을 해봐야 알게 된다.}
최종적으로는 좌우에 0칸의 공백을 가진 맘름모 기호를 얻게 된다.
%The \cmd{\diamond} symbol can only be used in math mode, hence it
%is surrounded by the shorthand \verb?$...$?. \tx\ usually leaves a little
%space around maths but the \cmd{\m@th} command stops that. \cmd{\mkern}
%adjusts space in math mode, and in this case we are eliminating
%the spaces\footnote{It is usually a matter for experiment to find
%the right values for the kerning.\label{fn:kerning}} 
%that would normally be on either 
%side of the diamond symbol.
%The whole effect gives us a diamond symbol with zero space around it.

%    The fancy break at the start of this discussion was typeset by
이 절의 시작에 보였던 fancy break는 아래와 같이 조판되었다:
\begin{lcode}
% define \motif, \chain, \diamonds and then
\fancybreak{\chain{\diamonds}{0.25\textwidth}}
\end{lcode}
이 코드는 \Lclass{memoir} 클래스에 들어잇지 않다.
나는 이것을 본문 내에서 직접 정의했다.
프리앰플이나 패키지 안에서 이것을 정의해두는 것이 보통일 것이다.
만약 자신의 패턴을 직접 정의하고 싶다면, 가령 \cs{clubs}라 할 경우,
\cmd{\club} 수학 기호를 사용하고 좌우에 약간의 공백을 두는 것이 될 것이다.
%The code is not part of the \Lclass{memoir} class; I defined it just
%as indicated in the body of the book. It would more naturally go
%into the preamble or a package. You might like to try specifying your
%own pattern, say \cs{clubs}, using the \cmd{\club} math symbol but leaving
%some space between them.

\index{번호 없는 장절 구획!스타일링|)}
\index{anonymous division!styling|)}


%\section{Footnotes in division headings}
\section{장절 헤딩 내의 각주}

\index{footnote!in heading|(}
\index{각주!장절 헤딩 내|(}

%    With the sectioning commands the text of the required argument
%\meta{title} is used as the text for the section title in the body
%of the document.
절을 지정하는 명령어에서 필수 인자 \meta{title}의 텍스트는
본문 내에서 그 절의 제목을 지정하는 데에 쓰인다.

선택 인자 \meta{toc-title}가 절 명령어에서 사용될 경우
그는 움직일 수 있게 되므로
모든 fragile\index{fragile command} 명령어들은
\cmd{\protect}ed\index{protect} 되어야 하며,
\meta{title} 인자는 고정되어야 한다.
\meta{toc-title}는 다음과 같이 두 가지 장소에서 쓰인다:
\begin{enumerate}
	\item \toc 내의 장 제목으로 쓰인다.
	\item 본문 내 페이지의 헤더\index{헤더}\index{header}로 쓰인다.
\end{enumerate}

%    When the optional argument \meta{toc-title} is used in a sectioning
%command it is moving and any fragile\index{fragile command} 
%commands must be \cmd{\protect}ed\index{protect},
%while the \meta{title} argument is fixed. The \meta{toc-title} also
%serves double duty:
%\begin{enumerate}
%\item It is used as the text of the title in the \toc;
%\item It is used as the text in page headers\index{header}. 
%\end{enumerate}

만약 선택 인자가 없을 경우 \meta{title}가 본문 내 텍스트,
\toc{}의 제목, 그리고 페이지 헤더\index{헤더}\index{header} 이렇게
세 곳에서 다함께 쓰이게 된다.

% moving 이 뭘까
%    If the optional argument is not present, then the \meta{title} is
%moving and serves the triple duty of providing the text for the body and \toc{}
%titles and for page headers\index{header}.


몇몇 사람들은 장 제목에 각주\index{각주}\index{footnote}를 달려고 하는데 이는 지양되어야 한다.
그들이 충분한 인내와 체력을 갖고 있지 않다면(?)
웬만하면 선택 인자를 사용함과 동시에 \cmd{\footnote}는 필수 인자에만 사용하는 것이 좋다.
만약 선택 인자가 사용되지 않는다면 각주 마크\index{각주!마크}\index{footnote!mark}와
텍스트가 절 페이지, \toc, \meta{title}을 포함하는 모든 페이지의 헤더\index{헤더}\index{header}에
널브러져버릴 가능성이 높다.
어느 독자도 이런 일을 원하지 않을 것이다.
결론적으로, 각주\index{각주}\index{footnote}를 포함한 절 제목은 아래와 같아야 한다:
\begin{lcode}
	\chapter[제목]{제목\footnote{정말 이렇게 해야만 하겠습니까?}}
\end{lcode}
%    Some folk feel an urge to add a footnote\index{footnote} to a sectioning 
%title, which
%should be resisted. If their flesh is weak, then the optional argument must
%be used and the \cmd{\footnote} attached to the required argument only.
%If the optional argument is not used then the footnote mark\index{footnote!mark} and text is
%likely to be scattered all over the place, on the section page, in the \toc,
%on any page that includes \meta{title} in its headers\index{header}. This is 
%unacceptable to any reader. So, a footnoted\index{footnote} title should look like
%this:
%\begin{lcode}
%\chapter[Title]{Title\footnote{Do you really have to do this?}}
%\end{lcode}

\index{footnote!in heading|)} %| emacs

%%%%%%%%%%%%%%%%%%%%%%%%%%%%%%%%%%%%%%%%%%%%%%%%%%%%%%%%%%%%%%%%%

\section{미리 정의된 헤딩 스타일}
%\section{Predefined heading styles}

책이나 보고서를 조판하는 데에 쓰이는 모든 \ltx\ 클래스들은
section 헤딩 스타일을 정의하는 하나의 스타일을 제공한다.
\Mname\ 클래스는 여러 가지 헤딩 스타일을 제공한다는 점에서 특이하다.
각각의 스타일은 구획 헤딩 주변의 공간, 폰트, 글자크기가 다르다.
참조용으로, \tref{tab:secfonts}은 section 헤딩에 사용되는 기본 폰트의 목록을 보여준다.
이 폰트들은 모두 볼드체로 사용되지만 구획 레벨에 따라 다른 크기로 조판된다.

%   All \ltx\ classes for typesetting books and reports provide a particular
%style for sectional headings. The \Mname\ class is unusual in that it provides
%several sets of heading styles. Each set has different spacing around the
%division heads, and different fonts in different sizes. 
%    As a reference, \tref{tab:secfonts} lists the default fonts used
%for the sectional headings. These fonts are all bold but in different
%sizes depending on the division level.

\begin{table}
\centering
%\caption{Default fonts for sectional headings}\label{tab:secfonts}
\caption{Section 헤딩을 위한 기본 폰트의 목록}\label{tab:secfonts}
\begin{tabular}{lll} \toprule
\cmd{\booknamefont}       & \verb?\huge\bfseries? & \huge\bfseries huge \\
\cmd{\booknumfont}        & \verb?\huge\bfseries? & \huge\bfseries huge \\
\cmd{\booktitlefont}      & \verb?\Huge\bfseries? & \Huge\bfseries Huge \\
\cmd{\partnamefont}       & \verb?\huge\bfseries? & \huge\bfseries huge \\
\cmd{\partnumfont}        & \verb?\huge\bfseries? & \huge\bfseries huge \\
\cmd{\parttitlefont}      & \verb?\Huge\bfseries? & \Huge\bfseries Huge \\
\cmd{\chapnamefont}       & \verb?\huge\bfseries? & \huge\bfseries huge \\
\cmd{\chapnumfont}        & \verb?\huge\bfseries? & \huge\bfseries huge \\
\cmd{\chaptitlefont}      & \verb?\Huge\bfseries? & \Huge\bfseries Huge \\
\cmd{\secheadstyle}       & \verb?\Large\bfseries? & \Large\bfseries Large \\
\cmd{\subsecheadstyle}    & \verb?\large\bfseries? & \Large\bfseries Large \\
\cmd{\subsubsecheadstyle} & \verb?\normalsize\bfseries? & \bfseries normal \\
\cmd{\paraheadstyle}      & \verb?\normalsize\bfseries? & \bfseries normal \\
\cmd{\subparaheadstyle}   & \verb?\normalsize\bfseries? & \bfseries normal \\
\bottomrule
\end{tabular}
\end{table}

\begin{syntax}
\cmd{\makeheadstyles}\marg{name}\marg{code} \\
\cmd{\headstyles}\marg{name} \\
\end{syntax}
\glossary(makeheadstyles)%
  {\cs{makeheadstyles}\marg{name}\marg{code}}%
  {새로운 셋의 section 구획 헤딩 스타일을 만든다.
  	\meta{name}와 같은 이름을 갖고 \meta{code}에 의해 정의된다.}
%  {Create a new set of sectional division headstyles called \meta{name} 
%   defined by \meta{code}.}
\glossary(headstyles)%
  {\cs{headstyles}\marg{name}}%
    {section 구획 헤딩 스타일로 \meta{name}을 사용.}
%  {Use the \meta{name} set of headstyles for sectional division heads.}
%The default sectional division head styles provided by \Mname\ form the
%\hstyle{default} headstyles and give the same appearance as the standard
%\Lclass{book} and \Lclass{report} classes. The set is created via the
%\cmd{\makeheadstyles} macro and called for via the \cmd{headstyles} 
%declaration. 
\Mname\ 에서 제공되는 기본 section 구획 헤딩 스타일들은 \hstyle{default} 헤딩 스타일을 구성하며,
\Lclass{book}과 \Lclass{report} 클래스의 것과 동일한 생김새로 만들어준다.
각각의 셋은 \cmd{\makeheadstyles} 매크로로 생성되며 \cmd{headstyles} 선언문을 통해 호출된다.
\begin{lcode}
\makeheadstyles{default}{%
  \renewcommand*{\booknamefont}{\normalfont\huge\bfseries}
  %% and so on down to subparagraph specification
  \renewcommand*{\subparaheadstyle}{\normalfont\normalsize\bfseries}
}
\headstyles{default}
\end{lcode}

%    A somewhat different set of headstyles is used for this manual. When using
%\cmd{\makeheadstyles} you only need to specify things that differ from
%the \hstyle{default}. Within the class the \hstyle{memman} set of headstyles 
%is defined as:
이 매뉴얼에서는 조금 다른 셋의 헤딩 스타일이 사용되었다.
\cmd{\makeheadstyles}을 사용할 때는 \hstyle{default}와 다른 것들만 특정해주면 된다.
이 클래스 내에서 \hstyle{memman} 셋 헤딩 스타일은 다음과 같이 정의된다:
\begin{lcode}
\newcommand*{\addperiod}[1]{#1.}
\makeheadstyles{memman}{%
% book changes
  \renewcommand*{\booknamefont}{\normalfont\huge\sffamily}
  \renewcommand*{\booknumfont}{\normalfont\huge\sffamily}
  \renewcommand*{\booktitlefont}{\normalfont\Huge\sffamily}
  \renewcommand*{\midbookskip}{\par\vskip 2\onelineskip}%
% part changes
  \renewcommand*{\partnamefont}{\normalfont\huge\sffamily}
  \renewcommand*{\partnumfont}{\normalfont\huge\sffamily}
  \renewcommand*{\parttitlefont}{\normalfont\Huge\sffamily}
  \renewcommand*{\midpartskip}{\par\vskip 2\onelineskip}%
% chapter
  \chapterstyle{demo3}
% section
  \setbeforesecskip{-1.333\onelineskip 
                    \@plus -0.5\onelineskip \@minus -.5\onelineskip}%
  \setaftersecskip{0.667\onelineskip \@plus 0.1\onelineskip}%
  \setsecheadstyle{\normalfont\scshape\raggedright}%
% subsection
  \setbeforesubsecskip{-0.667\onelineskip 
                       \@plus -0.25\onelineskip \@minus -0.25\onelineskip}%
  \setaftersubsecskip{0.333\onelineskip \@plus 0.1\onelineskip}%
  \setsubsecheadstyle{\normalfont\bfseries\raggedright}%
% subsubsection
  \setbeforesubsubsecskip{-0.667\onelineskip 
                          \@plus -0.25\onelineskip \@minus -0.25\onelineskip}%
  \setaftersubsubsecskip{0.333\onelineskip \@plus 0.1\onelineskip}%
  \setsubsubsecheadstyle{\normalfont\normalsize\itshape\raggedright}%
% paragraph
  \setbeforeparaskip{1.0\onelineskip 
                     \@plus 0.5\onelineskip \@minus 0.2\onelineskip}%
  \setafterparaskip{-1em}%
  \setparaheadstyle{\normalfont\normalsize\itshape\addperiod}%
% subparagraph
  \setsubparaindent{\parindent}%
  \setbeforesubparaskip{1.0\onelineskip 
                        \@plus 0.5\onelineskip \@minus 0.2\onelineskip}%
  \setaftersubparaskip{-1em}%
  \setsubparaheadstyle{\normalfont\normalsize\itshape\addperiod}}
\end{lcode}
이 문서 전체에서 이것의 영향을 볼 수 있을 것이다.
이 장에서는 보통 내가 사용했던 \cstyle{demo3}와는 다른
\cstyle{pedersen} chapter 스타일을 사용했기에 조금 다르게 보일 것이다.
%    You can see the effect throughout this document. This chapter is
%slightly different in that I have used the \cstyle{pedersen} chapterstyle
%instead of the \cstyle{demo3} chapterstyle that I have normally used.
    
\begin{table}
\centering
%\caption{Fonts used by different headstyles}\label{tab:headfonts}
\caption{각각의 헤딩 스타일에서 사용된 폰트 목록.}\label{tab:headfonts}
\begin{tabular}{llllllll} \toprule
헤딩 스타일 & & chapter & section & subsec & subsubsec & para & subpara \\ \midrule
\hstyle{bringhurst} & & CAPS  & \textsc{s. caps} & \textit{Italic} & \textsc{s. caps} & \textit{Italic} & \textit{Italic} \\
\hstyle{crosshead}  & & \textbf{Bold}   & CAPS & \textbf{Bold} & \textsc{s. caps} & \textit{Italic} & \textsc{s. caps} \\
\hstyle{default}    & & \textbf{Bold}   & \textbf{Bold} & \textbf{Bold} & \textbf{Bold} & \textbf{Bold} & \textbf{Bold}   \\
\hstyle{dowding}    & & \textit{Italic} & CAPS & \textsc{s. caps} & \textit{Italic} & \textit{Italic} & \textit{Italic} \\
\hstyle{komalike}   & & \textsf{Sans} & \textsf{Sans} & \textsf{Sans} & \textsf{Sans} & \textsf{Sans} & \textsf{Sans}   \\
\hstyle{memman}     & & \textsf{Sans}   & \textsc{s. caps} & \textbf{Bold} & \textit{Italic} & \textit{Italic} & \textit{Italic} \\
\hstyle{ntglike}    & & \textbf{Bold}  & \textbf{Bold}  & \textbf{Bold}  & \textsl{Slanted} & \textsl{Slanted} & \textsl{Slanted}  \\
\hstyle{tandh}      & & \textbf{Bold}  & CAPS & \textit{Italic} & \textbf{Bold} & \textit{Italic} & \textit{Italic} \\
\hstyle{wilsondob}  & & \textit{Italic} & CAPS & \textit{Italic} & \textsc{s. caps } & \textit{Italic} & \textit{Italic} \\ \bottomrule
\end{tabular}
\end{table}


%    Several other sets of headstyles are provided as well and the full list
%is below. The different fonts used are given in \tref{tab:headfonts} and
%generally speaking they start off being large for chapter heads but are
%normal size by the time subsubsection heads are reached, or before.

다른 여러 헤딩 셋들도 제공되며, 전체적인 목록은 아래에 나와 있다.
\tref{tab:headfonts}에 나와있듯 서로 다른 폰트가 사용되며,
보통의 경우 chapter 헤딩에서는 크게 나오지만
subsubsection 헤딩이나 그 전 즈음 되어서는 기본 크기로 조판된다.

\begin{itemize}
	\item[\hstyle{bringhurst}] Bringhurst의 \btitle{Elements of
		Typographic Style}~\cite{BRINGHURST99} 책에서 사용되는 셋.
		그 책은 \cstyle{bringhurst} chapterstyle 을 사용한다(\fref{dcbringhurst}).
	\item[\hstyle{crosshead}] 이 셋은 \cstyle{crosshead} chapterstyle을 사용하며
	더 낮은 단계의 구획 제목에서는 crosshead로 지정된다.
%	and the lower level division titles are set as crossheads.
	\item[\hstyle{default}] \ltx\ 의
	\Lclass{book} 클래스에 해당되는 기본 셋.
	\item[\hstyle{dowding}] Dowding의
	\btitle{Finer Points}~\cite{DOWDING96}에 기반한 셋. 그 책은 \cstyle{dowding} 
	chapterstyle 을 사용한다(\fref{dcdowding}).
	\item[\hstyle{komalike}] KOMA \Lclass{scrbook} 클래스의 헤딩 종류에 기반한 셋.
	모든 글자가 bold sans serif font로 식자되며, 
	이는  \cstyle{komalike} chapterstyle 을 사용한다(\fref{dckomalike}).
	\item[\hstyle{memman}] \cstyle{demo3} chapterstyle 을 포함하여 이 문서 내에서 사용된 셋.
	\item[\hstyle{ntglike}] NTG (Dutch TUG)의 \Lclass{boek} 클래스의 헤딩 종류에 기반한 셋. 
	이는 \cstyle{ntglike}
	chapterstyle (\fref{dcntglike})을 사용하며 기본 셋에 비해 조용하다.
	\item[\hstyle{tandh}] Thames 과 Hudson 의
	\btitle{Manual of Typography}~\cite{MCLEAN80} 책에서 사용된 헤딩에 기반한 셋.
	이 책은 \cstyle{tandh}
	chapterstyle (\fref{dctandh}) 을 사용한다.
	\item[\hstyle{wilsondob}] Adrian Wilson 의
	\btitle{Design of Books}~\cite{ADRIANWILSON93} 책에서 사용된 헤딩에 기반한 셋. 
	이 책은 \cstyle{wilsondob}
	chapterstyle (\fref{dcwilsondob}) 을 사용한다.
	
\end{itemize}



%\begin{itemize}
%\item[\hstyle{bringhurst}] A set based on Bringhurst's \btitle{Elements of
%     Typographic Style}~\cite{BRINGHURST99}. It uses the \cstyle{bringhurst}
%     chapterstyle (\fref{dcbringhurst}).
%\item[\hstyle{crosshead}] This set uses the \cstyle{crosshead} chapterstyle
%     and the lower level division titles are set as crossheads.
%\item[\hstyle{default}] The default set corresponding the \ltx\
%     \Lclass{book} class.
%\item[\hstyle{dowding}] A set based on Dowding's 
%     \btitle{Finer Points}~\cite{DOWDING96}. It uses the \cstyle{dowding} 
%     chapterstyle (\fref{dcdowding}).
%\item[\hstyle{komalike}] A set based on the kind of headings used in
%     the KOMA \Lclass{scrbook} class, where there are all in a bold sans serif font.
%     It uses the \cstyle{komalike} chapterstyle (\fref{dckomalike}).
%\item[\hstyle{memman}] The set used in this document, including the 
%     \cstyle{demo3} chapterstyle.
%\item[\hstyle{ntglike}] A set based on the kind of headings used in
%     the NTG (Dutch TUG) \Lclass{boek} class. It uses the \cstyle{ntglike}
%     chapterstyle (\fref{dcntglike}) and the headings are quiter than the 
%     default.
%\item[\hstyle{tandh}] A set based the heads used in Thames \& Hudson 
%     \btitle{Manual of Typography}~\cite{MCLEAN80}. It uses the \cstyle{tandh}
%     chapterstyle (\fref{dctandh})
%\item[\hstyle{wilsondob}] A set based on those used in Adrian Wilson's
%     \btitle{Design of Books}~\cite{ADRIANWILSON93}. It uses the \cstyle{wilsondob}
%     chapterstyle (\fref{dcwilsondob}).
%
%\end{itemize}
%
