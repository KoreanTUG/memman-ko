\chapter{Page notes} \label{chap:mnotes}

   The standard classes provide the \cmd{\footnote} command for notes
at the bottom of the page. The class provides several styles of footnotes
and you can also have several series of footnotes for when the material
gets complicated. The normal \cmd{\marginpar} command puts notes into
the margin, which may float around a little if there are other
\cmd{\marginpar}s on the page. The class additionally supplies commands
for fixed marginal notes and sidebars.


\section{Footnotes}

    A footnote can be considered to be a special kind of float\index{float} 
that is put at the bottom of a page.

\begin{syntax}
\cmd{\footnote}\oarg{num}\marg{text} \\
\end{syntax}
\glossary(footnote)
  {\cs{footnote}\oarg{num}\marg{text}}%
  {Put \meta{text} as a footnote.}
In the main text, the \cmd{\footnote} command puts a marker at the
point where it is called, and puts the \meta{text}, preceded by the same
mark, at the bottom of the page. If the optional \meta{num} is used
then its value is used for the mark, otherwise the \Icn{footnote}
counter is stepped and provides the mark's value. The \cmd{\footnote}
command should be used in paragraph mode where it puts the note at the
bottom of the page, or in a \Ie{minipage} where it puts the note
at the end of the \Ie{minipage}. Results are likely to be peculiar if
it is used anywhere else (like in a \Ie{tabular}).

\begin{syntax}
\cmd{\footnotemark}\oarg{num} \\
\cmd{\footnotetext}\oarg{num}\marg{text} \\
\end{syntax}
\glossary(footnotemark)%
  {\cs{footnotemark}\oarg{num}}%
  {Typesets a footnote mark.}
\glossary(footnotetext)%
  {\cs{footnotetext}\oarg{num}\marg{text}}%
  {Typesets \meta{text} as a footnote at the bottom of the page but does 
   not put a mark in the main text.}
    You can use \cmd{\footnotemark} to put a marker in the main text; the value
is determined just like that for \cmd{\footnote}. Footnote text can be put 
at the bottom of the page via \cmd{\footnotetext}; if the optional \meta{num}
is given it is used as the mark's value, otherwise the value of the
\Icn{footnote} counter is used.
   It may be helpful, but completely untrue, to think of \cmd{\footnote} being
defined like:
\begin{lcode}
\newcommand{\footnote}[1]{\footnotemark\footnotetext{#1}}
\end{lcode}
In any event, you can use a combination of \cmd{\footnotemark} and 
\cmd{\footnotetext} to do footnoting where \ltx\ would normally get upset.

\begin{syntax}
\cmd{\footref}\marg{label} \\
\end{syntax}
\glossary(footref)%
  {\cs{footref}{labstr}}%
  {Reference a labelled footnote.}
    On occasions it may be desireable to make more than one reference
to the text of a footnote\index{footnote!reference}. This can be done by 
putting a \cmd{\label} in the footnote and then using \cmd{\footref} to refer 
to the label; this prints the footnote mark. For example:
\begin{comment}
%%% in memdesign, not memman
\begin{lcode}
...\footnote{... adults or babies.\label{fn:rabbits}}
...
... The footnote\footref{fn:rabbits} on \pref{fn:rabbits} ...
\end{lcode}
In this manual, the last line above prints:
\begin{syntax}
... The footnote\footref{fn:rabbits} on \pref{fn:rabbits} ... \\
\end{syntax}
\end{comment}
\begin{lcode}
...\footnote{...values for the kerning.\label{fn:kerning}} ...
...
... The footnote\footref{fn:kerning} on \pref{fn:kerning} ... \\
\end{lcode}
In this manual, the last line above prints:
\begin{syntax}
... The footnote\footref{fn:kerning} on \pref{fn:kerning} ... \\
\end{syntax}


%%%%%%%%%%%%%%%%%%%%%%%%%%%%%%%%%%%%%%%%%%%%%%%%%%
%%%% from membook

\begin{syntax}
\cmd{\multfootsep} \\
\end{syntax}
\glossary(multfootsep)%
  {\cs{multfootsep}}%
  {Separator between adjacent footnote marks.}
In the standard classes if two or more footnotes are applied 
sequentially\footnote{One footnote}\footnote{Immediately followed by another}
then the markers in the text are just run together. The class, like the
\Lpack{footmisc}~\cite{FOOTMISC} and \Lpack{ledmac} packages, inserts a 
separator\index{footnote!marker separator}
between the marks. In the class the macro \cmd{\multfootsep} is used as
the separator. Its default definition is:
\begin{lcode}
\newcommand*{\multfootsep}{\textsuperscript{\normalfont,}}
\end{lcode}

\begin{syntax}
\cmd{\feetabovefloat} \\
\cmd{\feetbelowfloat} \\
\end{syntax}
\label{interest:feetbelowfloat}
\glossary(feetabovefloat)%
  {\cs{feetabovefloat}}%
  {Typeset footnotes above bottom floats (the default).}
\glossary(feetbelowfloat)%
  {\cs{feetbelowfloat}}%
  {Typeset footnotes below bottom floats.}
In the standard classes, footnotes on a page that has a float at the
bottom are typeset before the float. I think that this looks
peculiar. Following the \cmd{\feetbelowfloat} declaration footnotes 
will be typeset at the bottom of the page below any bottom 
floats;\index{footnote!bottom float} 
they will also be typeset at the bottom of \cmd{\raggedbottom} pages 
as opposed to being
put just after the bottom line of text. The standard positioning is
used following the \cmd{\feetabovefloat} declaration, which is the default.



\LMnote{2018/09/19}{Removed \cs{feetatbottom} from the manual. It does not
  actually do anything. The footnotes are always at the bottom}
% \begin{syntax}
% \cmd{\feetatbottom} \\
% \end{syntax}
% \glossary(feetatbottom)%
%   {\cs{feetatbottom}}%
%   {Place footnotes at the very bottom of the text block whenever we
%     are in a non \cs{flushbottom} context.}
% Then\Added{2015/04/22} we use \cs{raggedbottom} or similar, \LaTeX{}
% will by default attach the footnotes just below the text. In many
% cases it may look better if the footnotes are being build from the
% bottom of the texst block up. Issuing \cs{feetatbottom} does
% this. Please note that \cs{feetatbottom} has no effect whenever
% \cs{flushbottom} is active.



\subsection{A variety of footnotes}

 \begin{syntax}
 \cmd{\verbfootnote}\oarg{num}\marg{text} \\
 \end{syntax}
\glossary(verbfootnote)%
  {\cs{verbfootnote}\oarg{num}\marg{text}}%
  {Like \cs{footnote} except that \meta{text} can contain verbatim material.}
 The macro \cmd{\verbfootnote} is like the normal \cmd{\footnote}
except that its \meta{text} agument\index{footnote!verbatim text} 
can contain verbatim material.
For example, the next two paragraphs are typeset by this code:
 \begin{lcode}
    Below, footnote~\ref{fn1} is a \verb?\footnote? while
footnote~\ref{fn2} is a \verb?\verbfootnote?.

    The \verb?\verbfootnote? command should 
appear\footnote{There may be some problems if color is
		 used.\label{fn1}}
to give identical results as the normal \verb?\footnote?, 
but it can include some verbatim 
text\verbfootnote{The \verb?\footnote? macro, like all 
                  other macros except for \verb?\verbfootnote?, 
                  can not contain verbatim text in its 
                  argument.\label{fn2}}
in the \meta{text} argument.
\end{lcode}

     Below, footnote~\ref{fn1} is a \verb?\footnote? while
 footnote~\ref{fn2} is a \verb?\verbfootnote?.

     The \verb?\verbfootnote? command should 
 appear\footnote{There may be some problems if color is
		 used.\label{fn1}}
 to give identical results as the normal \verb?\footnote?, but it
 can include some verbatim 
 text\verbfootnote{The \verb?\footnote? macro, like all other macros
		   except for \verb?\verbfootnote?, can not contain
		   verbatim text in its argument.\label{fn2}}
 in the \meta{text} argument.



\begin{syntax}
\cmd{\plainfootnotes} \\
\cmd{\twocolumnfootnotes} \\
\cmd{\threecolumnfootnotes} \\
\cmd{\paragraphfootnotes} \\
\end{syntax}
\glossary(plainfootnotes)%
  {\cs{plainfootnotes}}%
  {Typeset footnotes as separate marked paragraphs (the default).}
\glossary(twocolumnfootnotes)%
  {\cs{twocolumnfootnotes}}%
  {Typeset footnotes in two columns.}
\glossary(threecolumnfootnotes)%
  {\cs{threecolumnfootnotes}}%
  {Typeset footnotes in three columns.}
\glossary(paragraphfootnotes)%
  {\cs{paragraphfootnotes}}%
  {Typeset footnotes as a single paragraph.}

Normally, each footnote\index{footnotes!as paragraphs} 
starts a new paragraph. The class provides three
other\index{footnote!styles} styles, making four in all. 
Following the \cmd{\twocolumnfootnotes}\index{footnotes!as two columns}
declaration footnotes will be typeset in two columns, and similarly
they are typeset in three columns\index{footnotes!as three columns} 
after the \cmd{\threecolumnfootnotes}
declaration. Footnotes are run together as a single 
paragraph\index{footnotes!as a paragraph} after the
\cmd{\paragraphfootnotes} declaration. The default style is used after
the \cmd{\plainfootnotes} declaration. 

   The style can be changed at any 
time but there may be odd effects if the change is made in the middle of
a page when there are footnotes before and after the declaration. You may
find it interesting to try changing styles in an article type document 
that uses \cmd{\maketitle} and \cmd{\thanks}, and some footnotes on the 
page with the title:
\begin{lcode}
\title{...\thanks{...}}
\author{...\thanks{...}...}
...
\begin{document}
\paragraphfootnotes
\maketitle
\plainfootnotes
...
\end{lcode}

\begin{syntax}
\cmd{\footfudgefiddle} \\
\end{syntax}
\glossary(footfudgefiddle)%
  {\cs{footfudgefiddle}}%
  {Integer number (default 64) to help when typesetting \cs{paragraphfootnotes}.}
Paragraphed footnotes may overflow\index{footnote!too long} 
the bottom of a page. \tx\ has
to estimate the amount of space that the paragraph will require once
all the footnotes are assembled into it. It then chops off the main text
to leave the requisite space at the bottom of the page, following which
it assembles and typesets the paragraph. If it underestimated the size
then the footnotes will run down the page too far. If this happens then
you can change \cmd{\footfudgefiddle} to make \tx\ be more generous in
its estimation. The default is 64 and a value about 10\% higher should
fix most overruns.
\begin{lcode}
\renewcommand*{\footfudgefiddle}{70}
\end{lcode}
You must use an integer in the redefinition as the command is going to be 
used in a place where \tx\ expects an integer.

\LMnote{2010/09/17}{removed the text below, I see no reason for adding
this to memoir, so it is now removed from the manual}
% \begin{syntax}
% \cmd{\footnoteA}\marg{text} \\
% \cmd{\footnoteB}\marg{text} \\
% \cmd{\footnoteC}\marg{text} \\
% \end{syntax}
% \glossary(footnoteA)%
%   {\cs{footnoteA}\marg{text}}%
%   {A series A footnote.}
% \glossary(footnoteB)%
%   {\cs{footnoteB}\marg{text}}%
%   {A series B footnote.}
% \glossary(footnotec)%
%   {\cs{footnoteC}\marg{text}}%
%   {A series C footnote.}
%
%     In addition to the regular \cmd{\footnote} the class provides 
% three further\index{footnote!series} series 
% of footnotes, namely the `A', `B', and `C' series which are
% distinguished by appending the series' uppercase letter at the end of
% the command, like \cmd{\footnoteB} for the `B' series.
% Perhaps the normal footnotes are required, 
% marked\index{footnote!marker} flagged with arabic numerals, and another 
% kind of footnote flagged with roman numerals. Each series has its own
% \cmd{\footnotemarkB}, \cmd{\footnotetextB} and so on matching the regular
% commands.


\begin{syntax}
\cmd{\newfootnoteseries}\marg{series} \\
\cmd{\plainfootstyle}\marg{series} \\
\cmd{\twocolumnfootstyle}\marg{series} \\
\cmd{\threecolumnfootstyle}\marg{series} \\
\cmd{\paragraphfootstyle}\marg{series} \\
\end{syntax}
\glossary(newfootnoteseries)%
  {\cs{newfootnoteseries}\marg{series}}%
  {Create a new footnote \meta{series}.}
\glossary(plainfootstyle)%
  {\cs{plainfootstyle}\marg{series}}%
  {Set the \meta{series} footnotes to be typeset plain style.}
\glossary(twocolumnfootstyle)%
  {\cs{twocolumnfootstyle}\marg{series}}%
  {Set the \meta{series} footnotes to be typeset in two column style.}
\glossary(threecolumnfootstyle)%
  {\cs{threecolumnfootstyle}\marg{series}}%
  {Set the \meta{series} footnotes to be typeset in three column style.}
\glossary(paragraphfootstyle)%
  {\cs{paragraphfootstyle}\marg{style}}%
  {Set the \meta{series} footnotes to be typeset in single paragraph style.}

    If you need further series you can create you own.
A new footnote series\index{footnote!new series} is
created by the \cmd{\newfootseries} macro, where \meta{series} is an
alphabetic identifier for the series. This is most conveniently a 
single (upper case) letter, for example \texttt{P}. 

    Calling, say, \verb?\newfootnoteseries{Q}? creates a set of macros
equivalent to those for the normal \cmd{\footnote} but with the \meta{series}
appended. These include \cs{footnoteQ}, \cs{footnotemarkQ},
\cs{footnotetextQ} and so on. These are used just like the normal
\cmd{\footnote} and companions.

    By default, a series is set to typeset using the normal style
of a paragraph per note. The series' style can be changed by using one
of the \cs{...footstyle} commands.\index{footnote!style}

    For example, to have a `P' (for paragraph) series using roman numerals 
as markers which, in the main text are superscript with a closing parenthesis
and at the foot are on the baseline followed by an endash, and the text is
set in italics at the normal footnote size:
\begin{lcode}
\newfootnoteseries{P}
\paragraphfootstyle{P}
\renewcommand{\thefootnoteP}{\roman{footnoteP}}
\footmarkstyleP{#1--}
\renewcommand{\@makefnmarkP}{%
              \hbox{\textsuperscript{\@thefnmarkP)}}}
\renewcommand{\foottextfontP}{\itshape\footnotesize}
\end{lcode}
This can then be used like:
\begin{lcode}
.... this sentence\footnoteP{A `p' footnote\label{fnp}} 
includes footnote~\footrefP{fnp}.
\end{lcode}

   The \cmd{\newfootnoteseries} macro does not create series versions
of the footnote-related length commands, such as \lnc{\footmarkwidth}
and \lnc{\footmarksep}, nor does it create versions of \cmd{\footnoterule}.

   At the foot of the page footnotes are grouped according to their series;
all ordinary footnotes are typeset, then all the first series footnotes 
(if any), then the second series, and so on. The ordering corresponds to
the order of \cmd{\newfootnoteseries} commands.

     If you can't specify a particular footnote style using the
class facilities the \Lpack{footmisc}
package~\cite{FOOTMISC} provides a range of styles. 
A variety of styles also comes with the \Lpack{ledmac} package~\cite{LEDMAC} 
which additionally provides several classes of footnotes that can be mixed
on a page.



\subsection{Styling}

\index{footnote!styling|(}
     The parameters controlling the vertical spacing of footnotes are 
 illustrated in \fref{fig:fn}.

 \begin{figure}
 \centering
 \drawparameterstrue
 \setlayoutscale{0.4}
 \drawfootnote
 \caption{Footnote layout parameters}\label{fig:fn}
 \end{figure}

     There is a discussion in \Sref{sec:thanks} starting on
\pref{sec:thanks} about how to style the \cmd{\thanks} command; footnotes
can be similarly styled. 

    The \cmd{\footnote} macro (and its relations) essentially does three 
things:
\begin{itemize}
\item Typesets a marker\index{footnote!marker} at the point where 
      \cmd{\footnote} is called;
\item Typesets a marker\index{footnote!marker} at the bottom of the page 
      on which \cmd{\footnote} is called;
\item Following the marker at the bottom of the page, typesets the 
      text \index{footnote!text} of the footnote.
\end{itemize}

\begin{syntax}
\cmd{\@makefnmark} \\
\cmd{\@thefnmark} \\
\end{syntax}
\glossary(@makefnmark)%
  {\cs{@makefnmark}}%
  {Typesets the footnote marker where \cs{footnote} is called.}
\glossary(@thefnmark)%
  {\cs{@thefnmark}}%
  {Value of the footnote marker.}

The \cmd{\footnote} macro calls the kernel command \cmd{\@makefnmark} to
typeset the footnote marker at the point where \cmd{\footnote} is called
(the value of the marker is kept in the macro \cmd{\@thefnmark}
which is defined by the \cmd{\footnote} or \cmd{\footnotemark} macros). 
The default definition typesets the mark\index{footnote!marker!styling} 
as a superscript and is effectively
\LMnote{2014/05/07}{\cs{@makefnmark} does not take arguments}
\begin{lcode}
\newcommand*{\@makefnmark}{\hbox{\textsuperscript{\@thefnmark}}}
\end{lcode}
You can change this if, for example,
 you wanted the marks to be in parentheses at the baseline.
\LMnote{2014/05/07}{\cs{@makefnmark} does not take arguments}
 \begin{lcode}
 \renewcommand*{\@makefnmark}{{\footnotesize (\@thefnmark)}}
 \end{lcode}
 or, somewhat better to take account of the size of the surrounding text
\LMnote{2014/05/07}{\cs{@makefnmark} does not take arguments}
 \begin{lcode}
 \renewcommand*{\@makefnmark}{\slashfracstyle{(\@thefnmark)}}
 \end{lcode}



\begin{syntax}
\cmd{\footfootmark} \\
\cmd{\footmarkstyle}\marg{arg} \\
\end{syntax}
\glossary(footfootmark)%
  {\cs{footfootmark}}%
  {Typsets the footnote mark at the bottom of the page.}
\glossary(footmarkstyle)%
  {\cs{footmarkstyle}\marg{style}}%
  {Style of the footnote mark at the bottom of the page.}
The class macro for typesetting the marker at the foot of the page is
\cmd{\footfootmark}.  The appearance of the mark is controlled by
\cmd{\footmarkstyle}. The default specification is
\begin{lcode}
\footmarkstyle{\textsuperscript{#1}}
\end{lcode}
where the \verb?#1? indicates the position of \cmd{\@thefnmark} in the style.
The default results in the mark being set as a superscript.
For example, to have the marker set on the baseline 
and followed by a right parenthesis, do
\begin{lcode}
\footmarkstyle{#1) }
\end{lcode}

\begin{syntax}
\lnc{\footmarkwidth} \lnc{\footmarksep} \lnc{\footparindent} \\
\end{syntax}
\glossary(footmarkwidth)%
  {\cs{footmarkwidth}}%
  {Width of footnote mark box.}
\glossary(footmarksep)%
  {\cs{footmarksep}}%
  {Offset from the footnote mark box for lines after the first.}
\glossary(footparindent)%
  {\cs{footparindent}}%
  {Paragraph indent for multiparagraph footnote text.}
The mark is typeset in a box of width \lnc{\footmarkwidth}
If this is negative, the mark is outdented
into the margin, if zero the mark is flush left, and when positive
the mark is indented. The mark is followed by the 
text\index{footnote!text} of the footnote. Second and later lines of the
text are offset by the length \lnc{\footmarksep} from the end of the box.
The first line of a paragraph within a footnote is indented by
\lnc{\footparindent}. 
 The default values for these lengths are:
\begin{lcode}
\setlength{\footmarkwidth}{1.8em}
\setlength{\footmarksep}{-\footmarkwidth}
\setlength{\footparindent}{1em}
\end{lcode}


\begin{syntax}
\cmd{\foottextfont} \\
\end{syntax}
\glossary(foottextfont)%
  {\cs{foottextfont}}%
  {Font for footnote text.}
The text in the footnote\index{footnote!text!font} is typeset using 
the \cmd{\foottextfont} font. The default is \cmd{\footnotesize}.


    Altogether, the class specifies
\begin{lcode}
\footmarkstyle{\textsuperscript{#1}}
\setlength{\footmarkwidth}{1.8em}
\setlength{\footmarksep}{-1.8em}
\setlength{\footparindent}{1em}
\newcommand{\foottextfont}{\footnotesize}
\end{lcode}
to replicate the standard footnote layout. 

    You might like to try the
combinations of \lnc{\footmarkwidth} and \lnc{\footmarksep} listed
in \tref{tab:fnstyle} to see which you might prefer.
Not listed in the \tablerefname, to get the marker flushleft and then 
the text set as a block paragraph you can try:
\begin{lcode}
\setlength{\footmarkwidth}{1.8em}
\setlength{\footmarksep}{0em}
\footmarkstyle{#1\hfill}
\end{lcode}

\begin{table}
\begin{adjustwidth}{-5mm}{-5mm}
\centering
\caption{Some footnote text styles}\label{tab:fnstyle}
\begin{tabular}{cc>{\raggedright\arraybackslash}p{0.5\textwidth}} \toprule
\lnc{\footmarkwidth} & \lnc{\footmarksep} & Comment \\ \midrule
1.8em & -1.8em & Flushleft, regular indented paragraph (the default) \\
1.8em & 0em    & Indented, block paragraph hung on the mark \\
%1.8em & 1.8em  & Indented, outdented paragraph \\
%0em   & -1.8em & Regular indented paragraph, first line flushleft \\
0em   & 0em    & Flushleft, block paragraph \\
%0em   & 1.8em  & Outdented paragraph, first line flushleft \\
%-1.8em & -1.8em & Regular indented paragraph, starting in the margin \\
%-1.8em & 0em & \\
-1.8em & 1.8em & Block paragraph, flushleft, mark in the margin \\
\LMnote{2010/02/05}{added -1sp trick}
-1sp   & 0em   & Block paragraph, flushleft, mark in the margin but
flush against the text
\\
\bottomrule
\end{tabular}
\end{adjustwidth}
\end{table}

    As an example of a rather different scheme, in at least one discipline
the footnoted text in the main body has a 
marker\index{footnote!marker!multiple} at each end. It is possible 
to define a macro to do this:
\begin{lcode}
\newcommand{\wrapfootnote}[1]{\stepcounter\@mpfn%
  %  marks in the text
  \protected@xdef\@thefnmark{\thempfn}% 
  \@footnotemark #1\@footnotemark%
  %  marks at the bottom
  \protected@xdef\@thefnmark{\thempfn--\thempfn}% 
  \@footnotetext}
\end{lcode}
\makeatletter
\newcommand{\wrapfootnote}[1]{\stepcounter\@mpfn%
  % marks in the text
  \protected@xdef\@thefnmark{\thempfn}% 
  \@footnotemark #1\@footnotemark%
  % marks at the bottom
  \protected@xdef\@thefnmark{\thempfn--\thempfn}%
  \@footnotetext}
\makeatother
The macro is based on a posting to \ctt{} by Donald 
Arseneau\index{Arseneau, Donald} in November 2003, 
and is used like this:
\begin{lcode}
Some 
\wrapfootnote{disciplines}{For example, Celtic studies.} 
require double marks in the text.
\end{lcode}
Some 
\wrapfootnote{disciplines}{For example, Celtic studies.} 
require double marks in the text.


\begin{syntax}
\cmd{\fnsymbol}\marg{counter} \\
\cmd{\@fnsymbol}\marg{num} \\
\end{syntax}
\glossary(fnsymbol)%
  {\cs{fnsymbol}\marg{counter}}%
  {Typesets the representation of the footnote marker.}
\glossary(@fnsymbol)%
  {\cs{@fnsymbol}\marg{num}}%
  {Converts \meta{num} to the footnote marker representation.}

    Any footnotes after this point will be set according to:
\begin{lcode}
\setlength{\footmarkwidth}{-1.0em}
\setlength{\footmarksep}{-\footmarkwidth}
\footmarkstyle{#1}
\end{lcode}
\setlength{\footmarkwidth}{-1.0em}
\setlength{\footmarksep}{-\footmarkwidth}
\footmarkstyle{#1}

The \cmd{\fnsymbol} macro typesets the representation of the 
counter \meta{counter} like a footnote\index{footnote!symbol} symbol. 
Internally it uses the kernel \cmd{\@fnsymbol} macro which converts 
a positive integer \meta{num} to a symbol. If you are not fond of
the standard ordering\index{footnote!symbol!order} of the footnote 
symbols, this is the macro to change. Its
original definition is:
\begin{lcode}
\def\@fnsymbol#1{\ensuremath{%
  \ifcase#1\or *\or \dagger\or \ddagger\or 
  \mathsection\or \mathparagraph\or \|\or **\or 
  \dagger\dagger \or \ddagger\ddagger \else\@ctrerr\fi}}
\end{lcode}
\makeatletter
\let\m@mold@fnsymbol\@fnsymbol
\def\@fnsymbol#1{\ensuremath{%
  \ifcase#1\or *\or \dagger\or \ddagger\or 
  \mathsection\or \mathparagraph\or \|\or **\or 
  \dagger\dagger \or \ddagger\ddagger \else\@ctrerr\fi}}
\makeatother
This, as shown by \verb?\@fnsymbol{1},...\@fnsymbol{9}? produces the series:
\begin{center}
 \makeatletter
\@fnsymbol{1},
\@fnsymbol{2},
\@fnsymbol{3},
\@fnsymbol{4},
\@fnsymbol{5},
\@fnsymbol{6},
\@fnsymbol{7},
\@fnsymbol{8}, and
\@fnsymbol{9}.
%\@fnsymbol{10}  % out of bounds
\makeatother 
\end{center}

\makeatletter
\renewcommand*{\@fnsymbol}[1]{\ensuremath{%
  \ifcase#1\or *\or \dagger\or \ddagger\or
  \mathsection\or \|\or \mathparagraph\or **\or \dagger\dagger
  \or \ddagger\ddagger \else\@ctrerr\fi}}
\makeatother
    Robert Bringhurst quotes the following as the traditional 
ordering\index{footnote!symbol!order} (at least up
to \makeatletter\@fnsymbol{6}\makeatother):
\begin{center}
 \makeatletter
\@fnsymbol{1},
\@fnsymbol{2},
\@fnsymbol{3},
\@fnsymbol{4},
\@fnsymbol{5},
\@fnsymbol{6},
\@fnsymbol{7},
\@fnsymbol{8}, and
\@fnsymbol{9}.
\makeatother 
\end{center}
You can obtain this sequence by redefining \cmd{\@fnsymbol} as:
\begin{lcode}
\renewcommand*{\@fnsymbol}[1]{\ensuremath{%
  \ifcase#1\or *\or \dagger\or \ddagger\or
  \mathsection\or \|\or \mathparagraph\or **\or \dagger\dagger
  \or \ddagger\ddagger \else\@ctrerr\fi}}
\end{lcode}
not forgetting judicious use of \cmd{\makeatletter} and \cmd{\makeatother}
if you do this in the preamble\index{preamble}. Other authorities or publishers
may prefer other sequences and symbols.
\makeatletter\let\@fnsymbol\m@mold@fnsymbol\makeatother

    To get the footnote reference marks\index{reference mark} set with
symbols use:
\begin{lcode}
\renewcommand*{\thefootnote}{\fnsymbol{footnote}}
\end{lcode}
or to use roman numerals instead of the regular arabic numbers:
\begin{lcode}
\renewcommand*{\thefootnote}{\roman{footnote}}
\end{lcode}

\begin{syntax}
\cmd{\footnoterule} \\
\end{syntax}
\glossary(footnoterule)%
{\cs{footnoterule}}%
{Rule separating footnotes from the main text.}

    The rule separating footnotes from the main text is specified 
by \cmd{\footnoterule}:
\begin{lcode}
\newcommand*{\footnoterule}{%
  \kern-3pt%
  \hrule width 0.4\columnwidth
  \kern 2.6pt}
\end{lcode}
If you don't want a rule (but you might later), then the easiest method is:
\begin{lcode}
\let\oldfootnoterule\footnoterule
\renewcommand*{\footnoterule}{}
\end{lcode}
and if you later want rules you can write:
\begin{lcode}
\let\footnoterule\oldfootnoterule
\end{lcode} 


\fancybreak{}


In \fref{fig:fn} we see that the footnotes are separated from the text
by \cs{skip}\cs{footins}. We provide a special interface to set this skip:\Added{2015/04/22}
\begin{syntax}
\cmd{\setfootins}\marg{length for normal}\marg{length for minipage} \\
\end{syntax}
\glossary(setfootins)%
{\cs{setfootins}\marg{length for normal}\marg{length for minipage}}%
{Sets \cs{skip}\cs{footins} and its minipage counterpart.}
The default is similar to
\begin{lcode}
  \setfootins{\bigskipamount}{\bigskipamount}
\end{lcode}
Internally \cmd{\setfootins} also sets the skips being used by
\cmd{\twocolumnfootnotes} and friends.



\index{footnote!styling|)}

\index{footnote|)}

 \section{Marginal notes}

\LMnote{2010/01/**}{added a reference back to \cs{setmarginnotes}}
     Some marginalia can also be considered to be kinds of floats. 
The class provides the standard margin notes\index{margin note} 
via \cmd{\marginpar}. Remember that the width of the margin note, the
separation from the text, and the separation from one \cmd{\marginpar}
to another is controlled by \cmd{\setmarginnotes}, see
Section~\ref{sec:head-foot-marg} on
page~\pageref{sec:head-foot-marg}. 

 \begin{syntax}
 \cmd{\marginpar}\oarg{left-text}\marg{text} \\
 \cmd{\marginparmargin}\marg{placement}\\
 \cmd{\reversemarginpar} \\
 \cmd{\normalmarginpar} \\
 \end{syntax}
\glossary(marginpar)%
  {\cs{marginpar}\oarg{left-text}\marg{text}}%
  {Puts \meta{text} into the margin; if given, \meta{left-text} will be
   used instead of \meta{text} for the left margin.}
\glossary(marginparmargin)%
  {\cs{marginparmargin}\marg{placement}}%
  {Provides a more textual
    interface for the user to specify in which margin the margin note
    should be placed.}
\glossary(reversemarginpar)%
  {\cs{reversemarginpar}}%
  {Reverses the normal margins used by \cs{marginpar}.}
\glossary(normalmarginpar)%
  {\cs{normalmarginpar}}%
  {Sets the normal margins used by \cs{marginpar}(the default).}
 Just as a reminder, the \cmd{\marginpar} macro puts \meta{text} into
the margin alongside the typeblock --- the particular margin depends
on the document style and the particular page. 

\LMnote{2010/01/**}{explained why the need for the
  \cs{marginparmargin} macro}
The interface for specifying which margin \cmd{\marginpar} (and
friends) write to, have long been quite cluttered, so we have in 2010 
adopted a more textual and natural interface. For \cmd{\marginpar} the
macro is named \cmd{\marginparmargin}\marg{placement} with possible
placements: \emph{left}, \emph{right}, \emph{outer}, and
\emph{inner}. The interpretation of which is explained in \fref{fig:xmargin}.
The default corresponds to \verb?\marginparmargin{outer}?.


\LMnote{2010/01/**}{instead of explaining the algorithm, each time the
  algorithm is explaine donce and for all in a figure, then we can
  simply refer to it}
\begin{figure}[htbp]
  \centering
  \begin{minipage}{0.9\linewidth}
    \cs{Xmargin}\marg{placement} for possible placements: \emph{left},
    \emph{right}, \emph{outer}, and \emph{inner}\\
    \renewcommand\descriptionlabel[1]{\hspace\labelsep\normalfont\sffamily\bfseries #1}
    \begin{description}
     \small
    \item[Two column document] If the note is placed in the first
      column, to the left, otherwise  to the right, irrespective the
      document being one- or two-side and of the users choices
\item[One sided document] If user specified \emph{left}, notes are
  placed to the left, otherwise to the right. 
\item[Two sided document] depends on whether a recto or verso page:
   \begin{description}
   \item[Recto (odd) page] note is placed on the right if the user
     specified \emph{right} or \emph{outer}, otherwise the note is
     placed on the left.
   \item[Verso (even) page] note is placed on the left if the user
     specified \emph{left} or \emph{outer}, otherwise the note is
     placed on the right.
   \end{description}
\end{description}
  \end{minipage}
  \caption{Interpretation of the arguments to the \cs{Xmargin}
    commands for specifying the side in which to place side note like
    material. \texttt{X} here equals \texttt{marginpar},
    \texttt{sidepar}, \texttt{sidebar}, or \texttt{sidefoot}.} 
  \label{fig:xmargin}
\end{figure}

\fancybreak{}

\LMnote{2010/02/07}{Removed the explanation of the old interface for
  specifying the margin the \cs{marginpar} should go to. The original
  text is still available via subversion.}
The original convoluted methods of specifying the margin for
\cmd{\marginpar} is deprecated, although still supported; if you need
to know what they are then you can read all about them in \texttt{memoir.dtx}.

     Sometimes \ltx\ gets confused near a page break and a note just after
a break may get put into the wrong\index{margin note!wrong margin} margin 
(the wrong margin for the current 
page but the right one if the note fell on the previous page). If this occurs
then inserting the \cmd{\strictpagecheck} declaration before 
any \cmd{\marginpar}
command is used will prevent this, at the cost of at least one additional
\ltx\ run.


\section{Side notes}

    The vertical position of margin notes\index{margin note!moved} 
specified via \cmd{\marginpar}
is flexible so that adjacent notes are prevented from overlapping.

\LMnote{2010/01/**}{side note section extended with \cs{sidenotefont}
  and \cs{sidenotemargin}}
\begin{syntax}
\cmd{\sidepar}\oarg{left}\marg{right} \\
\cmd{\sideparmargin}\marg{placement}\\
\cmd{\sideparfont}\\
\cmd{\sideparform}\\
\lnc{\sideparvshift} \\
\end{syntax}
\glossary(sidepar)%
  {\cs{sidepar}\oarg{left}\marg{right}}%
  {Like \cs{marginpar} except that the note does not move vertically.}
\glossary(sideparvshift)%
  {\cs{sideparvshift}}%
  {Move a \cs{sidepar} up/down by this amount.}
\glossary(sideparmargin)%
{\cs{sideparmargin}\marg{placement}}%
{Specify into which margin the \cs{sidepar} goes.}
\glossary(sideparfont)%
{\cs{siderparfont}}%
{The font which the \cs{sidepar}s are typeset.}
\glossary(sideparform)
{\cs{sideparform}}%
{Macro holding placement like \cs{raggedright} and such.}

    The \cmd{\sidepar} macro is similar to \cmd{\marginpar} except that
it produces side notes\index{side note} that do not float --- 
they may overlap. 

The same spacing is used for both \cmd{\marginpar} and \cmd{\sidepar},
namely the lengths \lnc{\marginparsep} and \lnc{\marginparwidth}. See
\cmd{\setmarginnotes}, in Section~\ref{sec:head-foot-marg} on
page~\pageref{sec:head-foot-marg}.

The length
\lnc{\sideparvshift} can be used to make vertical 
adjustments\index{side note!adjust position} to the
position of \cmd{\sidepar} notes. By default this is set to a value
of 0pt which should align the top of the note with the text line.

The command \cmd{\sideparfont} is used to specify the font used for
the \cmd{\sidepar}, default is \cmd{\normalfont}\cmd{\normalsize}.

\LMnote{2010/02/05}{added description of \cs{sideparform}}
While \cmd{\sideparfont} holds the font settings for the sidepar, the
local adjustment is kept in \cmd{\sideparform}, the default is
\begin{lcode}
  \newcommand*{\sideparform}{%
    \ifmemtortm\raggedright\else\raggedleft\fi}
\end{lcode}
Which is a special construction the makes the text go flush against the
text block on side specified via \cmd{\sideparmargin}. Since the
margin par area is usually quite narrow it might be an idea to use a
ragged setup which enables hyphenation. This can be achieved by
\begin{lcode}
  \usepackage{ragged2e}
  \newcommand*{\sideparform}{%
    \ifmemtortm\RaggedRight\else\RaggedLeft\fi}
\end{lcode}


\LMnote{2010/01/**}{Fixed the typo, such that it IS now stated that
  the default is left, not right as was earlier stated}
The macro \cmd{\sideparmargin}\marg{placement} can be used to specify
which margin the side note should go to. \meta{placement} should be
one of \emph{left}, \emph{right}, \emph{outer}, or
\emph{inner}. Interpretation of which is explained in
\fref{fig:xmargin}. For some now forgotten reason the default
corresponds to \verb?\sideparmargin{left}?.\footnote{As not to change
  existing documents, we have decided to leave it like that.}


    By default the \meta{right} argument is put in the 
%right
left%
\index{side note!text for particular margin} margin. When
the \Lopt{twoside} option is used the \meta{right} argument is put into
the %left 
right 
margin on the verso (even numbered) pages; however, for these pages
the optional \meta{left} argument is used instead if it is present. For
two column text the relevent argument is put into the `outer' margin with 
respect to the column.

\fancybreak{}

\LMnote{2010/02/07}{Removed the explanation of the old interface for
  specifying the margin the \cs{sidepar} should go to. The original
  text is still available via subversion.}  The original convoluted
methods of specifying the margin for \cmd{\sidepar} is deprecated,
although still supported; if you need to know what they are then you
can read all about them in \texttt{memoir.dtx}.


\begin{syntax}
\cmd{\parnopar} \\
\end{syntax}
\glossary(parnopar)%
  {\cs{parnopar}}%
  {Forces a new paragraph, but with no apparent break in the text.}
    When \ltx\  is deciding where to place the side notes it checks whether
it is on an odd or even page and sometimes TeX doesn't realise that it has just
moved onto the next page. Effectively TeX typesets paragraph by paragraph 
(including any side notes) and at the end of each paragraph sees if there
should have been a page break in the middle of the paragraph. If there was
it outputs the first part of the paragraph, inserts the page break, and retains
the second part of the paragraph, without retypesetting it, for eventual
output at the top of the new page. This means that side notes for any given
paragraph are in the same margin, either left or right. A side note at the
end of a paragraph may then end up in the wrong margin. The macro 
\cmd{\parnopar} forces a new paragraph\index{paragraph break!invisible} 
but without appearing to (the first
line in the following paragraph follows immediately after the last element
in the prior paragraph with no line break). You can use \cmd{\parnopar}
to make TeX to do its page break calculation when you want it to, by splitting
what appears to be one paragraph into two paragraphs.

\LMnote{2010/02/09}{Moved the veelo example to the new sniplet chapter}
    Bastiaan Veelo\index{Veelo, Bastiaan} has kindly provided example code 
for another form of a side note, the code is shown in
\snipletref{snip:veelo}. 

    Bastiaan also noted that it provided an example of using the
\lnc{\foremargin} length.
    If you want to try it out, either put the code in your preamble,
or put it into a package (i.e., \file{.sty} file) without the 
\cs{makeat...} commands. 


\section{Sidebars}

    Sidebars\index{sidebar} are typeset in the margin and usually 
contain material that is ancilliary to the main text. They may be long 
and extend for more than one page.\footnote{Donald 
Arseneau's\index{Arseneau, Donald} help has been invaluable in getting
the sidebar code to work.}


\begin{syntax}
\cmd{\sidebar}\marg{text} \\
\end{syntax}
\glossary(sidebar)%
  {\cs{sidebar}\marg{text}}%
  {Typesets \meta{text} in a sidebar.}
The \cmd{\sidebar} command is like \cmd{\marginpar} in that it sets
the \meta{text} in the margin. However, unlike \cmd{\marginpar} the
\meta{text} will start near the top of the page, and may continue onto
later pages if it is too long to go on a single page. If multiple
\cmd{\sidebar} commands are used on a page, the several \meta{text}s
are set one after the other.

\LMnote{2010/02/07}{dropped \cs{ifsidebaroneside} as it is not used anymore}
\begin{syntax}
\cmd{\sidebarmargin}\marg{margin} \\
\end{syntax}
\glossary(sidebarmargin)%
  {\cs{sidebarmargin}\marg{margin}}%
  {Specifies the \meta{margin}(s) for sidebars.}

\LMnote{2010/02/07}{just adapted the explanation from the other
  \cs{Xmargin} macros}
The macro \cmd{\sidebarmargin}\marg{placement} can be used to specify
which margin the side note should go to. \meta{placement} should be
one of \emph{left}, \emph{right}, \emph{outer}, or
\emph{inner}. Interpretation of which is explained in
\fref{fig:xmargin}. The default corresponds to \verb?\sidebarmargin{outer}?.


\begin{syntax}
\cmd{\sidebarfont} \cmd{\sidebarform} \\
\end{syntax}
\glossary(sidebarfont)%
  {\cs{sidebarfont}}%
  {Font used for sidebars.}
\glossary(sidebarform)%
  {\cs{sidebarform}}%
  {Form (e.g., \cs{raggedright}) used for sidebars.}
The sidebar \meta{text} is typeset using the \cmd{\sidebarfont}, whose initial
definition is
\begin{lcode}
\newcommand{\sidebarfont}{\normalsize\normalfont}
\end{lcode}
Sidebars\index{sidebar!styling} are normally narrow so the text is 
set raggedright\index{raggedright} to 
reduce hyphenation\index{hyphenation}
problems and stop items in environments like \Ie{itemize} from overflowing. 
\LMnote{2010/02/05}{added description of \cs{sidebarform}}
\LMnote{2010/02/06}{fixed typo}
More accurately, the text is set according to \cmd{\sidebarform} which is 
defined as:
\begin{lcode}
  \newcommand*{\sidebarform}{%
    \ifmemtortm\raggedright\else\raggedleft\fi}
\end{lcode}
Which is a special construction the makes the text go flush against the
text block on side specified via \cmd{\sideparmargin}. Since the
margin par area is usually quite narrow it might be an idea to use a
ragged setup which enables hyphenation. This can be achieved by
\begin{lcode}
  \usepackage{ragged2e}
  \newcommand*{\sidebarform}{%
    \ifmemtortm\RaggedRight\else\RaggedLeft\fi}
\end{lcode}



% which is ragged right but with less raggedness than \cmd{\raggedright}
% would allow. As usual, you can change \cmd{\sidebarform}, for example:
% \begin{lcode}
% \renewcommand{\sidebarform}{}            % justified text
% \end{lcode}





    You may run into problems if the \cmd{\sidebar} command comes near 
a pagebreak, or if the sidebar text gets typeset alongside main text that
has non-uniform line spacing (like around a \cmd{\section}). Further,
the contents of sidebars may not be typeset if they are too near to the
end of the document.

\begin{syntax}
\lnc{\sidebarwidth} \lnc{\sidebarhsep} \lnc{\sidebarvsep} \\
\end{syntax}
\glossary(sidebarwidth)%
  {\cs{sidebarwidth}}%
  {Width of sidebars.}
\glossary(sidebarhsep)%
  {\cs{sidebarhsep}}%
  {Space between the edge of the main text and sidebars.}
\glossary(sidebarvsep)%
  {\cs{sidebarvsep}}%
  {Vertical space between sidebars that fall on the same page.}
The \meta{text} of a \cmd{\sidebar} is typeset in a column of width 
\lnc{\sidebarwidth} and there is a horizontal gap of \lnc{\sidebarhsep}
between the main text and the sidebar. The length \lnc{\sidebarvsep}
is the vertical gap between sidebars that fall on the same page; it also
has a role in controlling the start of sidebars with respect to the
top of the page.

\begin{syntax}
\lnc{\sidebartopsep} \\
\cmd{\setsidebarheight}\marg{height} \\
\end{syntax}
\glossary(sidebartopsep)%
  {\cs{sidebartopsep}}%
  {Controls the vertical position of a sidebar. The default is 0pt which
   aligns the tops of the typeblock and the sidebar.}
\glossary(setsidebarheight)%
  {\cs{setsidebarheight}\marg{height}}%
  {Sets the height of sidebars. The default is \cs{textheight}.}
The length \lnc{\sidebartopsep} controls the vertical position of the top
of a sidebar. The default is 0pt which aligns it with the top of the
typeblock.
The command \cmd{\setsidebarheight} sets the height of sidebars to
\meta{height}, without making any allowance for \lnc{\sidebartopsep}.
The \meta{length} argument should be equivalent to an integral number 
of lines. For example:
\begin{lcode}
\setsidebarheight{15\onelineskip}
\end{lcode}
The default is the \lnc{\textheight}.

    Perhaps you would like sidebars to start two lines below the top of
the typeblock but still end at the bottom of the typeblock? If so, and
you are using the \Lpack{calc} package~\cite{CALC}, then the following
will do the job:
\begin{lcode}
\setlength{\sidebartopskip}{2\onelineskip}
\setsidebarheight{\textheight-\sidebartopskip}
\end{lcode}


    The alignment of the text in a sidebar with the main text may not
be particularly good and you may wish to do some experimentation
(possibly through a combination of \lnc{\sidebarvsep} and 
\cmd{\setsidebarheight}) to improve matters.

    Although you can set the parameters for your sidebars individually it
is more efficient to use the \cmd{\setsidebars} command; it \emph{must} be 
used if you change the font and/or the height.
\begin{syntax}
\cmd{\setsidebars}\marg{hsep}\marg{width}\marg{vsep}\marg{topsep}\marg{font}\marg{height} \\
\end{syntax}
\glossary(setsidebars)%
  {\cs{setsidebars}\marg{hsep}\marg{width}\marg{vsep}\marg{topsep}\marg{font}\marg{height}}%
  {Sets the several sidebar parameters.}

The \cmd{\setsidebars} command can be used to set the sidebar parameters.
\lnc{\sidebarhsep} is set to \meta{hsep}, \lnc{\sidebarwidth} is set to
\meta{width}, \lnc{\sidebarvsep} is set to \meta{vsep}, \lnc{\sidebartopsep}
is set to \meta{topsep}, \cmd{\sidebarfont} is set to \meta{font}, 
and finally \cmd{\setsidebarheight} is used to set the height to \meta{height}.
The default is:
\LMnote{2010/02/07}{the default was wrong compared to the class}
\begin{lcode}
\setsidebars{\marginparsep}{\marginparwidth}{\onelineskip}%
            {0pt}{\normalsize\normalfont}{\textheight}
\end{lcode}
Any, or all, of the arguments can be a \verb?*?, in which case the parameter
corresponding to that argument is unchanged. Repeating the above example of 
changing the topskip and the height, assuming that the other defaults are 
satisfactory except that the width should be 3cm and an italic font should 
be used:
\begin{lcode}
\setsidebars{*}{3cm}{*}{2\onelineskip}{\itshape}%
            {\textheight-\sidebartopsep}
\end{lcode}

   Changing the marginpar parameters, for example with \cmd{\setmarginnotes},
will not affect the sidebar parameters.

   Note that \cmd{\checkandfixthelayout} neither checks nor fixes any of
the sidebar parameters. This means, for instance, that if you change the
\lnc{\textheight} from its default value and you want sidebars to have 
the same height then after changing the \lnc{\textheight} you have to 
call \cmd{\checkandfixthelayout}  and then call \cmd{\setsidebars} with
the (new) \lnc{\textheight}. For instance:
\begin{lcode}
...
\settypeblocksize{40\baselineskip}{5in}{*}
...
\checkandfixthelayout
\setsidebars{...}{...}{...}{...}{...}{\textheight}
\end{lcode}

    Unfortunately if a sidebar is on a double column page that either includes
a double column float or starts a new chapter then the top of the sidebar
comes below the float or the chapter title. I have been unable to eliminate 
this `feature'.


\section{Side footnotes}
\label{sec:side-footnotes}

Besides three already mentioned macros for writing in the margin
(\cmd{\marginpar}, \cmd{\sidepar}, and \cmd{\sidebar}) \theclass\ also
provide a functionality to add side footnotes. Actually two ways: one
is to internally make \cmd{\footnote} use \cmd{\marginpar} to write
in the margin, the other is to collect all side footnotes bottom up in
the margin.
\begin{syntax}
  \cmd{\footnotesatfoot}\\
  \cmd{\footnotesinmargin}\\
\end{syntax}
\cmd{\footnotesatfoot} (the default) causes \cmd{\footnote} to place
its text at the bottom of the page. By issuing \cmd{footnotesinmargin}
\cmd{\footnote} (and friends like \cmd{\footnotetext})will internally
use \cmd{\marginpar} to write the footnote to the page.



\subsection{Bottom aligned side footnotes}
\label{sec:bottom-aligned-side}

Bottom aligned footnotes works just like regular footnotes, just with
a separate macro \cmd{\sidefootenote}\marg{text}, and here the side
footnotes are placed at the bottom of the specified margin (more or
like as if one had taken the footnotes from the bottom of the page and
moved it to the margin instead). All the major functionality is the
same as for the normal \cmd{\footnote}
command.\footnote{\cs{sidefootnote} does not make sense inside
  minipages\dots}
\begin{syntax}
  \cmd{\sidefootnote}\oarg{num}\marg{text}\\
  \cmd{\sidefootnotemark}\oarg{num}\\
  \cmd{\sidefootnotetext}\oarg{num}\marg{text}\\
\end{syntax}

By default the regular footnotes and the side footnotes use different
counters. If one would like them to use the same counter, issue the
following in the preamble:
\begin{lcode}
  \letcountercounter{sidefootnote}{footnote}
\end{lcode}


\subsection{Setting the layout for
  \texorpdfstring{\cs{sidefootnote}}{sidefootnote}} 
\label{sec:sett-layo-texorpdfst}



There are several possibilities to change the appearance of the
\cmd{\sidefootnote}:

Specifying the margin in which the side footnote should go, is done by 
\begin{syntax}
  \cmd{\sidefootmargin}\marg{keyword}\\
\end{syntax}
where \meta{keyword} can be \emph{left}, \emph{right}, \emph{outer}, and
\emph{inner}, and their meaning is explained in
\fref{fig:xmargin}. The default is \emph{outer}.

\begin{syntax}
  \lnc{\sidefoothsep}\\
  \lnc{\sidefootwidth}\\
  \lnc{\sidefootvsep}\\
\end{syntax}
\cmd{\sidefoothsep} is a length controlling the separation from the
text to the side footnote column, default
\cmd{\marginparsep}. \cmd{\sidefootwidth} is length controlling the
width of the side footnote column, default \cmd{\marginparwidth}, and
\cmd{\sidefootvsep} is the vertical distance between two side
footnotes, default \cmd{\onelineskip}.

\begin{syntax}
  \lnc{\sidefootadjust}\\
  \cmd{\setsidefootheight}\marg{height}\\
  \cmd{\sidefootfont}\\
\end{syntax}
\cmd{\sidefootadjust} is a length which specifies the placement of the
side footnote column in relation to the bottom of the text block, the
default is 0pt. \cmd{\setsidefootheight} sets the maximal height of
the side footnote column, default \cmd{textwidth}. Lastly
\cmd{\sidefootfont} holds the general font setting for the side
footnote,\footnote{There is a similar macro to control the font of the
  text alone.} default \verb?\normalfont\footnotesize?.


The macro
\begin{syntax}
   \cmd{\setsidefeet}\marg{hsep}\marg{width}\marg{vsep}\marg{adj}\marg{font}\marg{height}\\
\end{syntax}
sets the specifications all six settings above in one go.. An `*'
means `use the current value'. So \theclass\ internally use the
following default
\begin{lcode}
  \setsidefeet{\marginparsep}{\marginparwidth}%
  {\onelineskip}{0pt}%
  {\normalfont\footnotesize}{\textheight}%
\end{lcode}
It is recommended to use this macro along with the other macros in the
preamble to specify document layout.

\subsection{Styling 
  \texorpdfstring{\cs{sidefootnote}}{sidefootnote}} 


\begin{syntax}
  \cmd{\sidefootmarkstyle}\marg{code}\\
\end{syntax}
controls how the side footnote counter is typeset in the side
footnote. The default is
\begin{lcode}
  \sidefootmarkstyle{\textsuperscript{#1}}
\end{lcode}

The mark is typeset in a box of width \lnc{\sidefootmarkwidth}
If this is negative, the mark is outdented
into the margin, if zero the mark is flush left, and when positive
the mark is indented. The mark is followed by the 
text\index{side footnote!text} of the footnote. Second and later lines of the
text are offset by the length \lnc{\sidefootmarksep} from the end of the box.
The first line of a paragraph within a footnote is indented by
\lnc{\sidefootparindent}. The default values for these lengths are:
\begin{lcode}
  \setlength{\sidefootmarkwidth}{0em}
  \setlength{\sidefootmarksep}{0em}
  \setlength{\sidefootparindent}{1em}
\end{lcode}


\fancybreak{}

Caveat: It is natural to specify a length as \lnc{\sidefootparindent}
as a \LaTeX\ length, but it has a down side. If, as we do here, set
the value to 1em, then since the size of the em unit changes with the
current font size, one will actually end up with an indent
corresponding to the font size being used when the
\begin{lcode}
  \setlength{\sidefootparindent}{1em}
\end{lcode}
was issued, not when it has used (where the font size most often will
be \cmd{\footnotesize}).

At this point we consider this to be a \emph{feature} not an
error. One way to get pass this problem it the following
\begin{lcode}
\begingroup% keep font change local
\sidefoottextfont
\global\setlength\sidefootparindent{1em}
\endgroup  
\end{lcode}
Then it will store the value of em corresponding to the font being
used. 


\LMnote{2010/02/07}{explained the rest, left the side footnote hook
  out of it for now}
\begin{syntax}
  \cmd{\sidefoottextfont}\\
\end{syntax}
holds the font being used by the side footnote, default
\verb+\normalfont\footnotesize+. 
\begin{syntax}
  \cmd{\sidefootform}\\
\end{syntax}
is used to specify the raggedness of the text. Default
\begin{lcode}
  \newcommand*{\sidefootform}{\rightskip=\z@ \@plus 2em}
\end{lcode}
which is much like \cs{raggedright} but allows some hyphenation. One
might consider using
\begin{lcode}
  \usepackage{ragged2e}
  \newcommand*{\sidefootform}{\RaggedRight}
\end{lcode}
Which does something similar.





\subsection{Side footnote example}
\label{sec:side-footn-example}



In the margin you will find the result of the following code:
\begin{verbatim}
  Testing\sidefootnote{This is test} bottom aligned
  footnotes.\sidefootnote{This is another side 
  footnote, spanning several lines.

  And several paragraphs}\sidefootnote{And number three}
\end{verbatim}
  Testing\sidefootnote{This is test} bottom aligned
  footnotes.\sidefootnote{This is another side 
  footnote, spanning several lines.

  And several paragraphs}\sidefootnote{And number three}


\LMnote{2013/05/02}{Moved here from backmatter.tex}

\section{Endnotes}
\label{sec:endnotes}

\LMnote{2010/10/28}{several \cs{printpagenotes} was spelled wrong}

\reimplemented{December 2010}{The former implementation had some
  difficulties handling certain types of input. A few of the macros
  used to format the output are no longer supported/used in the new
  implementation.}


    Endnotes are often used instead of footnotes so as not to interrupt the
flow of the main text. Although endnotes are normally put at the end of 
the document, they may instead be put at the end of each chapter.

    The \Lpack{endnotes} package already uses the command \cmd{\endnote} for
an endnote, so the class uses \cmd{\pagenote} for an endnote so as not 
to clash if you prefer to use the package. 
\LMnote{2011/01/23}{The implementation has nothing to do with the
  current pagenote package, thus the remark is removed}
% The following was originally supplied as the \Lpack{pagenote}
% package~\cite{PAGENOTE}. 

\begin{syntax}
\cmd{\makepagenote} \\
\cmd{\pagenote}\oarg{id}\marg{text} \\
\cmd{\printpagenotes} \cmd{\printpagenotes*} \\
\end{syntax}
\glossary(makepagenote)%
  {\cs{makepagenote}}%
  {Preamble command for enabling page/end notes.}%
\glossary(printpagenotes)%
  {\cs{printpagenotes}}%
  {Input the pagenote \file{ent} file for printing, then close it to any 
   more notes.}%
\glossary(printpagenotes*)%
  {\cs{printpagenotes*}}%
  {Input the pagenote \file{ent} file for printing, then empty it ready for 
   further notes.}%

   The general principle is that notes are written out to a file which
is then input at the place where the notes are to be printed. The note 
file has an \pixfile{ent} extension, like the table of contents file
has a \pixfile{toc} extension.

    You have to put \cmd{\makepagenote} in your preamble if you want 
endnotes. This will open the \pixfile{ent} note file which is called
\verb?\jobname.ent?.

   In the body of the text use use \cmd{\pagenote} to create an endnote, just
as you would use \cmd{\footnote} to create a footnote. In the books that I have
checked there are two common methods of identifying an endnote:
\begin{enumerate}
\item Like a footnote, put a number in the text at the location 
of the note and use the same number to identify the note when it
finally gets printed.\label{pagenote:id1}
\item Put no mark in the text, but when it is finally 
      printed\pagenote[Put no mark \ldots finally printed]{This manual uses
      both footnotes and endnotes. For identifying the endnotes I have used the
      `words' method for identifying the parent location of an endnote, so as not
      to start confusing the reader with two sets of note marks in the body of the
      text. Typically either footnotes or endnotes are used, not both, so the
      question of distinguishing them does not normally arise.}
use a few words from the text to identify the origin of the note. The page number
is often used as well with this method.\label{pagenote:id2}
\end{enumerate}
The \meta{text} argument of \cmd{\pagenote} is the contents of the note and
if the optional \meta{id} argument is not used the
result is similar to having used \cmd{\footnote} --- a number in the main text
and the corresponding number in the endnotes listing (as 
in \ref{pagenote:id1} above). For the second reference style 
(\ref{pagenote:id2} above) use the 
optional \meta{id} argument for the `few words', and no mark will be put
into the main text but \meta{id} will be used as the identification in
the listing.

   For one set of endnotes covering the whole document put 
\cmd{\printpagenotes} where you want them printed, typically before
any bibliography or index. The \cmd{\printpagenotes} macro inputs the
\pixfile{ent} endnote file for printing and then closes it to any further
notes.

For notes
at the end of each chapter put \cmd{\printpagenotes*}, which inputs
the \pixfile{ent} file for printing then empties it ready for more notes,
at the end of each chapter.

   The simple use is like this:
\begin{lcode}
\documentclass[...]{memoir}
...
\makepagenote
...
\begin{document}
\chapter{One}
...\pagenote{An end note.} ...
...\pagenote{Fascinating information.}
...
\chapter{Last}% chapter 9
...\pagenote{Another note.}% 30th note
...
...
\printpagenotes
...
\end{document}
\end{lcode}
This will result in an endnote listing looking like \fref{fig:endnote}.

\begin{figure}
\centering
\begin{minipage}{0.8\textwidth}
\begin{framed}
\noindent{\bfseries\Large Notes}\\[\baselineskip]
{\bfseries Chapter 1 One} \\[\baselineskip]
1. An end note \\
2. Fascinating information. \\
.............. \\[\baselineskip]
{\bfseries Chapter 9 Last} \\[\baselineskip]
1. Another note \\[\baselineskip]
\end{framed}
\end{minipage}
\caption{Example endnote listing}\label{fig:endnote}
\end{figure}

For notes at the end of each chapter:
\begin{lcode}
\documentclass[...]{memoir}
...
\makepagenote
...
\begin{document}
\chapter{One}
...\pagenote{An end note.} ...
...
\printpagenotes*
\chapter{Last}
...\pagenote{Another note.} ...
...
\printpagenotes*
%%% no more chapters
...
\end{document}
\end{lcode}

\begin{syntax}
\cmd{\continuousnotenums} \\
\cmd{\notepageref} \\
\end{syntax}
\glossary(continuousnotenums)%
  {\cs{continuousnotenums}}%
  {Declaration to make the numbering of endnotes continuous throughout the
   document.}%
\glossary(notepageref)%
  {\cs{notepageref}}%
  {Declaration that page numbers are available to notes in the endnote listing.}
   The \Icn{pagenote} counter is used for the notes. By default the endnotes 
are numbered per chapter. If you want the numbering
to be continuous throughout the document use the \cmd{\continuousnotenums}
declaration. Normally the information on which page a note was created is
discarded but will be made available to notes in the endnote listing
following the \cmd{\notepageref} declaration.  Both
\cmd{\continuousnotenums} and \cmd{\notepageref} can only be used in
the preamble.

\LMnote{2011/01/23}{Because of the new implementation this is no
  longer needed as the writing to file is no longer delayed.}
% Because of how TeX writes information to files, when the
% \cmd{\notepageref} declaration is used there must be no notes on the
% page where \cmd{\printpagenotes} or \cmd{\printpagenotes*} closes the
% \pixfile{ent} file.  If necessary, a \cmd{\clearpage} or similar must
% be used before the print command.

\begin{syntax}
\cmd{\notesname} \\
\cmd{\notedivision} \\
\end{syntax}
\glossary(notesname)%
  {\cs{notesname}}%
  {Name for endnotes, default \texttt{Notes}.}%
\glossary(notedivision)%
  {\cs{notedivision}}%
  {Heading printed by the \cs{printpagenotes} and \cs{printpagenotes*} macros.}

  When \cmd{\printpagenotes} (or \cmd{\printpagenotes*}) is called the
  first thing it does is call the macro \cmd{\notedivision}. By
  default this is defined as:
\begin{lcode}
\newcommand*{\notedivision}{\chapter{\notesname}}
\end{lcode}
with
\begin{lcode}
  \newcommand*{\notesname}{Notes}
\end{lcode}
In other words, it will print out a heading for the notes that will be read
from the \file{ent} file. \verb?\print...? then closes the \pixfile{ent} file 
for writing and after this \verb?\input?s it to get and process the notes.


\subsection{Changing the appearance}

\subsubsection{In the text}
\label{sec:in-the-text}


\begin{syntax}
\cmd{\notenumintext}\marg{num} \\
\end{syntax}
\glossary(notenumintext)%
  {\cs{notenumintext}\marg{num}}%
  {Typesets the number \meta{num} of a page note in the main text.}%
The \Icn{pagenote} counter is used for pagenotes. The macro
\cmd{\notenumintext} is called by \cmd{\pagenote} with the value of the
\Icn{pagenote} counter as the \meta{num} argument to print the
value of the \Icn{pagenote} counter in the main text. By default it is 
printed as a 
superscript, but this can be changed, or even eliminated.
\begin{lcode}
\newcommand*{\notenumintext}[1]{\textsuperscript{#1}}
\end{lcode}

\subsubsection{In the page note list}
\label{sec:page-note-list}

  
\LMnote{2011/01/23}{Added to make the formating code easier to understand}

To better understand how a page note entry is formatted in the page
note list, we start with the following  pseudo code (it is not exactly
what you will see in the \texttt{.ent} file, but macros will end up
being called in this manner)
\begin{syntax}
\cmd{\prenoteinnotes}\\
\cmd{\noteidinnotes}\marg{notenum}\marg{id}\\
\cmd{\pageinnotes}\marg{auto generated note label key}\\
\cmd{\prenotetext}\\
\quad\meta{Page note text}\\
\cmd{\postnotetext}\\
\cmd{\postnoteinnotes}
\end{syntax}
At the start and end we have the two macros \cmd{\prenoteinnotes} and
\cmd{\postnoteinnotes}, they take care of preparing for and ending an
entry in the list. The list is typeset in a manner where each item is
(at least) a paragraph, so the default definition is
\begin{lcode}
\newcommand{\prenoteinnotes}{\par\noindent}
\newcommand{\postnoteinnotes}{\par}
\end{lcode}
\glossary(prenoteinnotes)%
  {\cs{prenoteinnotes}}%
  {Called by \cs{noteentry} to initialise the printing of an endnote.}%
\glossary(postnoteinnotes)%
  {\cs{postnoteinnotes}}%
  {Called by \cs{noteentry} to finish the printing of an endnote.}%
A user could change this to make it look a bit more like a list
construction. For example the following would give a hanging
indentation
\begin{lcode}
\renewcommand{\prenoteinnotes}{\par\noindent\hangindent 2em}
\end{lcode}

\fancybreak{}


The \cmd{\noteidinnotes} calls \cmd{\idtextinnotes} to print the note \meta{id}
if it was given as the optional argument to \cmd{pagenote}, 
otherwise it calls \cmd{\notenuminnotes} to print the note number.
\begin{syntax}
\cmd{\noteidinnotes}\marg{notenum}\marg{id} \\
\cmd{\idtextinnotes}\marg{id} \\
\cmd{\notenuminnotes}\marg{num} \\
\end{syntax}
\glossary(noteidinnotes)%
  {\cs{noteidinnotes}\marg{notenum}\marg{id}}%
  {Prints an endnote's number or id in the endnote listing.}%
\glossary(idtextinnotes)%
  {\cs{idtextinnotes}\marg{id}}%
  {Prints an endnote's \meta{id} text}%
\glossary(notenuminnotes)%
  {\cs{notenuminnotes}\marg{num}}%
  {Typesets the number \meta{num} of a page note in the note listing.}%
These are defined respectively as:
\begin{lcode}
\newcommand*{\idtextinnotes}[1]{[#1]\space}
\newcommand*{\notenuminnotes}[1]{\normalfont #1.\space}
\end{lcode}

\fancybreak{}

Next we execute \cmd{\pageinnotes}\marg{note label key} which does
nothing by default. But if \cmd{\notepageref} is issued in the
preamble two things happen, (1) each page note issues a label such
that we can refer back to its page, and (2) \cmd{\pageinnotes} calls
\cmd{\printpageinnotes} (or if \Lpack{hyperref} is loaded
\cmd{\printpageinnoteshyperref}) 
\begin{syntax}
\cmd{\pageinnotes}\marg{auto generated note label key}\\
\cmd{\printpageinnotes}\marg{auto generated note label key}\\  
\cmd{\printpageinnoteshyperref}\marg{auto generated note label key}\\
\cmd{\pagerefname}
\end{syntax}
\glossary(pageinnotes)%
{\cs{pageinnotes}\marg{pagenum}}%
{Controls the printing of an endnote's page reference number.}%
\glossary(printpageinnotes)%
{\cs{printpageinnotes}\marg{pagenum}}%
{Prints an endnote's page reference number.}%
\glossary(printpageinnoteshypreref)%
{\cs{printpageinnoteshyperref}\marg{pagenum}}%
{Prints an endnote's page reference number whenever the
  \protect\Lpack{hyperref} package is used, it will include a
  hyperlink back to the page in question.}%
\glossary(pagerefname)%
{\cs{pagerename}}%
{Holds the name prefix when referring to a page number, defaults to \texttt{page}.}%
Default definitions
\begin{lcode}
\newcommand*{\printpageinnotes}[1]{%
    (\pagerefname\ \pageref{#1})\space}
\newcommand\printpageinnoteshyperref[1]{%
   (\hyperref[#1]{\pagerefname\ \pageref*{#1}})\space}
\end{lcode}
That is if \Lpack{hyperref} is loaded the entire text \meta{page 3}
will be the text of a hyperlink.

\bigskip

\begin{syntax}
  \cmd{\prenotetext}\\
  \cmd{\postnotetext}\\
\end{syntax}
\glossary(prenotetext)%
  {\cs{prenotetext}}%
  {Called within a page note list entry to initialise the printing
    of the text part of an endnote.}%
\glossary(postnotetex)%
  {\cs{postnotetext}}%
  {Called within a page note list entry to finish the printing of an
    endnote.}%
The actual text part of the page note is enclosed by
\cmd{\prenotetext} and \cmd{postnotetext}. By default they do nothing,
but could easily be redefined such that (only) the entry text would be
in italic: 
\begin{lcode}
\renewcommand\prenotetext{\begingroup\itshape}
\renewcommand\postnotetext{\endgroup}
\end{lcode}



%%%%%%%%%%%


\LMnote{2011/01/25}{Covered above}
% \begin{syntax}
% \cmd{\notenuminnotes}\marg{num} \\
% \end{syntax}
% \glossary(notenuminnotes)%
%   {\cs{notenuminnotes}\marg{num}}%
%   {Typesets the number \meta{num} of a page note in the note listing.}%
% In the note listing \cmd{\notenuminnotes} is used to print the number
% of a note. The default definitions are:

% \begin{lcode}
% \newcommand*{\notenuminnotes}[1]{\normalfont #1.\space}
% \end{lcode}



\LMnote{2011/01/25}{Covered above, \cmd{\noteentry} is not used anymore}
% \begin{syntax}
% \cmd{\noteentry}\marg{notenum}\marg{id}\marg{text}\marg{pagenum} \\
% \cmd{\prenoteinnotes} \\
% \cmd{\postnoteinnotes} \\
% \end{syntax}
% \glossary(noteentry)%
%   {\cs{noteentry}\marg{notenum}\marg{id}\marg{text}\marg{pagenum}}%
%   {Typesets a pagenote with number \meta{notenum}, identifier \meta{id},
%    contents \meta{text}, created on page \meta{pagenum}.}
% \glossary(prenoteinnotes)%
%   {\cs{prenoteinnotes}}%
%   {Called by \cs{noteentry} to initialise the printing of an endnote.}
% \glossary(postnoteinnotes)%
%   {\cs{postnoteinnotes}}%
%   {Called by \cs{noteentry} to finish the printing of an endnote.}
% The \cmd{\pagenote} macro writes \cmd{\noteentry}, with the appropriate 
% values for the arguments, to the \pixfile{ent} file, 
% where \meta{notenum} is the note number (from the \Icn{pagenote} counter),
% \meta{id} and \meta{text} are as supplied to
% \cmd{\pagenote}, and if the \cmd{\notepageref} declaration option is used, 
% \meta{pagenum} is the page number, otherwise it is empty. 
% The \cmd{\noteentry} macro controls the typesetting of the note.

% The default definition of \cmd{\noteentry} is
% \begin{lcode}
% \newcommand{\notentry}[4]{%
%   \prenoteinnotes
%   \noteidinnotes{#1}{#2}\pageinnotes{#4}\noteinnotes{#3}%
%   \postnoteinnotes}
% \end{lcode}
% and the definitions of other macros are:
% \begin{lcode}
% \newcommand{\prenoteinnotes}{\par\noindent}
% \newcommand{\postnoteinnotes}{\par}
% \end{lcode}
% so that (the first paragraph of) each note is printed as a non-indented 
% paragraph.

%     If you would prefer, say, hanging paragraphs try:
% \begin{lcode}
% \renewcommand{\prenoteinnotes}{\par\noindent\hangindent 2em}
% \end{lcode}

% \begin{syntax}
% \cmd{\noteidinnotes}\marg{notenum}\marg{id} \\
% \cmd{\idtextinnotes}\marg{id} \\
% \cmd{\notenuminnotes}\marg{num} \\
% \end{syntax}
% \glossary(noteidinnotes)%
%   {\cs{noteidinnotes}\marg{notenum}\marg{id}}%
%   {Prints an endnote's number or id in the endnote listing.}%
% \glossary(idtextinnotes)%
%   {\cs{idtextinnotes}\marg{id}}%
%   {Prints an endnote's \meta{id} text}%
% The \cmd{\noteidinnotes} calls \cmd{\idtextinnotes} to print the note \meta{id}
% if it was given as the optional argument to \cmd{pagenote}, 
% otherwise it calls \cmd{\notenuminnotes} to print the note number.
% These are defined respectively as:
% \begin{lcode}
% \newcommand*{\idtextinnotes}[1]{[#1]\space}
% \newcommand*{\notenuminnotes}[1]{\normalfont #1.\space}
% \end{lcode}

% \begin{syntax}
% \cmd{\pageinnotes}\marg{pagenum} \\
% \cmd{\printpageinnotes}\marg{pagenum} \\
% \end{syntax}
% \glossary(pageinnotes)%
%   {\cs{pageinnotes}\marg{pagenum}}%
%   {Controls the printing of an endnote's page reference number.}%
% \glossary(printpageinnotes)%
%   {\cs{printpageinnotes}\marg{pagenum}}%
%   {Prints an endnote's page reference number.}%
% The macro \cmd{\pageinnotes} controls the printing of a note's page 
% reference. If the
% \cmd{\notepageref} declaration has been used it calls
% \cmd{\printpageinnotes} to do the actual printing. Its definition is:
% \begin{lcode}
% \newcommand*{\printpageinnotes}[1]{%
%   (\pagerefname\ #1)\space}
% \end{lcode}

% \begin{syntax}
% \cmd{\noteinnotes}\marg{text} \\
% \end{syntax}
% \glossary(noteinnotes)%
%   {\cs{noteinnotes}\marg{text}}%
%   {Prints the text of an endnote.}%
% The macro \cmd{\noteinnotes}\marg{text} is simply:
% \begin{lcode}
% \newcommand{\noteinnotes}[1]{#1}
% \end{lcode}
% and is used to print the text of a note.




\bigskip



\LMnote{2011/01/25}{rewritten a bit}
\begin{syntax}
\cmd{\addtonotes}\marg{text} \\
\end{syntax}
\glossary(addtonotes)%
  {\cs{addtonotes}\marg{text}}%
  {Inserts \meta{text} into the endnotes \file{ent} file.}%
The macro \cmd{\addtonotes} inserts \meta{text} into the \pixfile{ent}
file. 

\begin{note}
  As the argument to \cmd{\pagenote} and \cmd{\addtonotes} is moving
  you may have to \cmd{\protect} any fragile commands. If you get
  strange error messages, try using \cmd{\protect} and see if they go
  away.
\end{note}
\fancybreak{}

Internally in \cmd{\pagenote} \cmd{\addtonotes} is used to provide
chapter devisions into the note list. It will detect both numbered and
unnumbered chapters. The actual text is provided using
\begin{syntax}
\cmd{\pagenotesubhead}\marg{chapapp}\marg{num}\marg{title} \\
\cmd{\pagenotesubheadstarred}\marg{chapapp}\marg{num}\marg{title} \\
\cmd{\pnchap} \cmd{\pnschap} \\
\end{syntax}
\glossary(pagenotesubhead)%
  {\cs{pagenotesubhead}\marg{chapapp}\marg{num}\marg{title}}%
  {Typesets a subheading for notes from chapter or appendix \meta{chapapp} 
   \meta{num} called \meta{title}.}%
\glossary(pagenotesubheadstarred)%
  {\cs{pagenotesubheadstarred}\marg{chapapp}\marg{num}\marg{title}}%
  {Typesets a subheading for notes from unnumbered chapter or appendix
    \meta{chapapp}  \meta{num} called \meta{title}.}%
\glossary(pnchap)%
  {\cs{pnchap}}%
  {Chapter title for \cs{pagenotesubhead}. Defaults to the ToC entry.}
\glossary(pnschap)%
{\cs{pnschap}}%
{Starred chapter title for \cs{pagenotesubhead}. Defaults to the regular title.}
  
The macro \cmd{\pagenotesubhead} typesets the subheadings in an
endnote list.  The \meta{chapapp} argument is normally
\cmd{\chaptername} but if the notes are from an appendix then
\cmd{\appendixname} is used instead. \meta{num} is the number of the
chapter, or blank if there is no number. Lastly, \meta{title} is
\cmd{\pnchap} for regular chapters which defaults to the ToC entry, or
\cmd{\pnschap} for starred chapters which defaults to the normal
title. The default definition of \cmd{\pagenotesubhead} is very
simply:
\begin{lcode}
\newcommand*{\pagenotesubhead}[3]{%
  \section*{#1 #2 #3}}
\newcommand\pagenotesubheadstarred{\pagenotesubhead} % i.e. the same
\end{lcode}
By default this means that the header for starred chapters will be
something like >>Chapter Title<<, which may look odd. In that case
redefine \cmd{\pagenotesubheadstarred} to something similar to 
\begin{lcode}
\renewcommand\pagenotesubheadstarred[3]{\section*{#3}}
\end{lcode}
Just remember that unless you have specified \cmd{\continuousnotenums}
in the preamble the note counter (\Icn{pagenote}) will only be reset
at the start of any numbered chapters (because it is tied to changes
in the chapter counter).

The scheme is set up under the assumption that notes will only be
printed at the end of the document. If you intend to put them at the
end of each chapter, then you will probably want to change the
definitions of the \cmd{\notedivision} and \cmd{\pagenotesubhead}
macros. For example:
\begin{lcode}
\renewcommand*{\notedivision}{\section*{\notesname}}
\renewcommand*{\pagenotesubhead}[3]{}
\end{lcode}
and remember to use \cmd{\printpagenotes*} at each place you want the
current set of notes to be printed.

\fancybreak{}

Say you have written a document with footnotes, but later on decide on
using end notes (page notes) instead. In that case you can use
\cmd{\foottopagenote} to make \cmd{\footnote}, \cmd{\footnotemark} and
\cmd{\footnotetext} works as if it was implemented using end notes. On
the other hand \cmd{\pagetofootnote} makes all  page notes into
footnotes (note that this might not work, because there are places
where page notes can be issued but foot notes cannot).

\begin{syntax}
\cmd{\foottopagenote}\\ \cmd{\pagetofootnote} \\
\end{syntax}
\glossary(foottopagenote)%
  {\cs{foottopagenote}}%
  {Declaration which turns \cs{footnote}s into \cs{pagenote}s.}%
\glossary(pagetofootnote)%
  {\cs{pagetofootnote}}%
  {Declaration which turns \cs{pagenote}s into \cs{footnote}s.}%
  % You can have both footnotes and endnotes in the same
  % document. However you may start with all footnotes and later
  % decide you would have preferred endnotes instead, or
  % \emph{vice-versa}. The \cmd{\foottopagenote} declaration makes
  % \cmd{\footnote}s behave as \cmd{\pagenote}s, and
  % \cmd{\pagetofootnote} has the opposite effect.
In either conversion the optional argument will be
ignored as for \cmd{\pagenote} it can be arbitrary text whereas for
\cmd{\footnote} it must be a number.






%%%%%%%%%%%%%%%%%%%%%%%%%%%%%%%%%%%%%%%%%%%%%%%%%%%%%%%%
%%%% start membook


%#% extend
%#% extstart include decorative-text.tex
