% !TEX root = ../memman-ko.tex
% \chapter{Titles}
\chapter{표제}

%    The standard classes provide little in the way of help in setting
% the title page(s) for your work, as the \cmd{\maketitle} command
% is principally aimed at generating a title for an article in a technical
% journal; it provides little for titles for works like theses, reports or
% books. For these I recommend that you design your own title page
% layout\footnote{If you are producing a thesis you are probably told
% just how it must look.}
% using the regular \ltx\ commands to lay it out, and ignore \cmd{\maketitle}.
표준 클래스들은 표지를 설정하는데 큰 도움이 되지 못하는데, \cmd{\maketitle}
명령은 테크니컬 논문집에서 논문의 제목을 생성하는 것이 주목적이기 때문이다.
이는 학위 논문, 보고서나 서적의 표지를 만들기에 불충분하다.
필자는 이를 위해 \cmd{\maketitle}을 무시하고, 직접 \ltx\ 표준 명령어를
사용해 표지 레이아웃을\footnote{만약 여러분이 학위 논문을 작성 중이라면, 그것이
어떻게 생겨야 하는지 아마도 정해져 있을 것이다.}
디자인할 것을 권장한다.

    % Quoting from Ruari McLean~\cite[p. 148]{MCLEAN80} in reference to the
% title page he says:
Ruari McLean~\cite[p. 148]{MCLEAN80}을 인용하자면, 그는 표지에 대해서 다음과
같이 말한다:
\begin{quotation}
%     The title-page states, in words, the actual title (and sub-title, if
% there is one) of the book and the name of the author and publisher, and
% sometimes also the number of illustrations, but it should do more than that.
% From the designer's point of view, it is the most important page in the
% book: it sets the style. It is the page which opens communication with the
% reader\ldots
  표지는 책의 실제 제목과 (있을 경우 부제목을 포함해) 저자의 이름, 출판사,
  그리고 간혹 삽화의 개수를 포함하지만, 나아가 그것보다 많은 일을 해야 한다.
  디자이너의 관점에서 표지는 책에서 가장 중요한 부분으로, 책의 양식을 정한다.
  표지는 독자와의 소통을 시작한다\ldots

%     If illustrations play a large part in the book, the title-page opening
% should, or may, express this visually. If any form of decoration is used
% inside the book, e.g., for chapter openings, one would expect this to be
% repeated or echoed on the title-page.
  만약 책에서 삽화가 큰 비중을 차지한다면, 표지는 이를 시각적으로 보일 수
  있거나 그래야 한다.
  책에서, 예컨대 장의 시작에서라던지, 어떠한 형태로든 장식이 사용된다면 독자는
  이것이 표지에서도 반복될 것을 기대할 것이다.

%     Whatever the style of the book, the title-page should give a foretaste
% of it. If the book consists of plain text, the title-page should at least
% be in harmony with it. The title itself should not exceed the width of the
% type area, and will normally be narrower\ldots
  책의 형식이 어떻든지간에, 표지는 그것의 맛보기를 보여줘야 한다.
  만약 책이 줄글로 구성되어 있다면, 표지는 적어도 그것과 조화를 이룰 수 있어야
  한다.
  표지 자체는 글 영역의 너비보다 넓어서는 안되며, 일반적으로는 더 좁을
  것이다\ldots

\end{quotation}

    % A pastiche of McLean's title page is shown in \fref{figure:titleTH}.
McLean의 표지를 모방한 것이 \fref{figure:titleTH}에 나와 있다.

\begin{figure}
\centering
\begin{showtitle}
\titleTH
\end{showtitle}
% \caption{Layout of a title page for a book on typography}\label{figure:titleTH}
\caption{타이포그래피에 관한 책의 표지 레이아웃}\label{figure:titleTH}
\end{figure}


\begin{comment}
% Figures~\ref{fig:titleTH} and~\ref{fig:titleDB}, for example,
% show title pages created using normal \ltx, without bothering
% with \cmd{\maketitle}.
  그림~\ref{fig:titleTH}\와~\ref{fig:titleDB}는 \cmd{\maketitle} 명령을 굳이
  쓰지 않고 일반적인 \ltx 명령을 사용해 표지를 만드는 예를 보여준다.
\end{comment}

%    The typeset format of the \cmd{\maketitle} command is virtually fixed
% within the \ltx\ standard classes. This class
% provides a set of formatting commands that can be used to modify
% the appearance of the title\index{title} information; that is, the contents of
% the \cmd{\title}, \cmd{\author} and \cmd{\date} commands.
% It also keeps the values
% of these commands so that they may be printed again later in the
% document.
\cmd{\maketitle} 명령의 조판 형식은 \ltx\ 표준 클래스에서 사실상 고정된
것으로 보아야 한다.
이 클래스는 제목 정보, 즉 \cmd{\title}, \cmd{\author}, \cmd{\date}의 내용이
표시되는 형태를 수정할 수 있는 일련의 형식화 명령들을 제공한다.
또, 이는 위 명령들의 값을 나중에 문서에서 다시 사용할 수 있도록 유지한다.

%    The class also inhibits the normal automatic cancellation of titling
% commands after \cmd{\maketitle}. This means that you can have multiple
% instances of the same, or perhaps different, titles in one document;
% for example on a half title page\index{half-title page} and the
% full title page\index{title page}.
% Hooks are provided so that additional titling elements can be defined
% and printed by \cmd{\maketitle}.
%   The \cmd{\thanks} command is enhanced to provide various configurations
% for both the marker symbol and the layout of the thanks\index{thanks} notes.
나아가 이 클래스는 \cmd{\maketitle} 명령이 사용된 후에 사용된 명령 값이
자동으로 지워지는 것을 방지한다.
그러므로 하나의 문서 안에서 같거나 다른 표지가, 예를 들면 반표지와 전표지에,
여러 번 나오게 할 수도 있다.
\cmd{\thanks} 명령은 기능이 확장되어 감사의 말 주석의 표지 부호와 레이아웃을
다양하게 설정할 수 있게 되었다.

\begin{figure}
\centering
\begin{showtitle}
\titleDS
\end{showtitle}
% \caption{Example of a mandated title page style for a doctoral thesis}\label{figure:titleDS}
\caption{박사 학위 논문의 지정 표지 양식의 예시}\label{figure:titleDS}
\end{figure}

\begin{figure}%\setlength{\unitlength}{1pt}
\centering
%\begin{showtitle}
{\titleRB}
%\end{showtitle}
% \caption{Example of a Victorian title page}\label{figure:titleRB}
\caption{빅토리아 시대의 표지 예시}\label{figure:titleRB}
\end{figure}

\begin{figure}
\centering
\begin{showtitle}
\titleDB
\end{showtitle}
% \caption{Layout of a title page for a book on book design}\label{figure:titleDB}
\caption{책 디자인에 관한 책의 표지 레이아웃}\label{figure:titleDB}
\end{figure}

\begin{figure}
\centering
\begin{showtitle}
\titleGM
\end{showtitle}
% \caption{Layout of a title page for a book about books}\label{figure:titleGM}
\caption{책들에 관한 책의 표지 레이아웃}\label{figure:titleGM}
\end{figure}

%     Generally speaking, if you want anything other than minor variations
% on the \cmd{\maketitle} layout then it is better to ignore \cmd{\maketitle}
% and take the whole layout into your own hands so you can place everthing
% just where you want it on the page.
일반적으로 \cmd{\maketitle} 레이아웃에서 약간의 변화를 넘어선다면
\cmd{\maketitle}을 무시하고 레이아웃 전체를 여러분이 손수 만들어 여러분이
종의의 원하는 곳에 모든 것을 위치시키는 편이 낫다.

% \section{Styling the titling}
\section{표지 양식화}


\index{title!styling|(}

%    The facilities provided for typesetting titles are limited, essentially
% catering for the kind of titles of articles published in technical journals.
% They can also be used as a quick and dirty method for typesetting titles
% on reports, but for serious work, such as a title page for a book or thesis,
% each title page\index{title page} should be handcrafted.
% For instance, a student of mine, Donald Sanderson\index{Sanderson, Donald}
% used \ltx\ to typeset his doctoral thesis, and \fref{figure:titleDS} shows
% the title page style mandated by Rensselear Polytechnic Institute as of 1994.
% Many other examples of title pages, together with the code to create them,
% are in~\cite{TITLEPAGES}.
표지를 조판하기 위해 제공되는 도구들은 국한되어 있는데, 이들은 본질적으로
테크니컬 논문집에 출판되는 논문의 제목 형식을 제공해주는 것이 전부이다.
이들은 보고서의 제목을 조판할 때에도 빠르고 지저분한 방법을 제공해줄 수 있지만,
책이나 학위 논문과 같은 중요한 작업의 표지\tidx{title page,표지}에 대해서는
수작업이 필요하다.
예컨대 나의 학생인 Donal Sanderson은 그의 박사 학위 논문을 조판하기 위해
\ltx\ 을 사용했고, \fref{figure:titleDS}에 Rensselear Polytechnic
Institute에서 1994년에 지정된 표지 양식이 나와 있다.
이외의 다양한 표지의 예시와, 이들을 만드는데 사용된 코드가 \cite{TITLEPAGES}에
있다.

%     Another handcrafted title page\index{title page} from~\cite{TITLEPAGES} is
% shown in \fref{figure:titleRB}. This one is based on an old booklet I found
% that was
% published towards the end of the 19th century and exhibits the love of
% Victorian printers in displaying a variety of types; the rules are
% an integral part of the title page. For the purposes of this manual I
% have used New Century\index{New Century Schoolbook} Schoolbook, which
% is part of the regular
% \ltx\ distribution, rather than my original choice of
% Century Old Style\index{Century Old Style}
% which is one of the commercial FontSite\index{FontSite} fonts licensed from
% the
% SoftMaker/ATF library, supported for \ltx\ through
% Christopher\index{League, Christopher}
% League's estimable work~\cite{TEXFONTSITE}.
\cite{TITLEPAGES}에 나오는 또 다른 수작업된 표지\tidx{title page,표지}가
\fref{figure:titleRB}에 나와 있다.
이것은 필자가 19세기 말 즈음에 출판된 옛 소책자를 기반으로 한 것이며, 다양한
활자를 표시할 수 있는 빅토리아 시대의 프린터에 대한 애정을 확인할 수 있다.
여기서 괘선이 표지의 핵심적인 부분이라고 할 수 있다.
본 매뉴얼을 위해 필자는 \ltx\ 표준 배포판에 포함된 New Century\index{New
Century Schoolbook} Schoolbook
글꼴을 사용했는데, 본래는 SoftMaker/ATF 라이브러리에서 라이선스를 받은, 그리고 
Christoper League\index{League, Christopher}의 공로\cite{TEXFONTSITE}로 \ltx\ 을
지원하는 FontSite\index{FontSite} 글꼴 중 하나인 Century Old
Style\index{Century Old Style}을 선택했었다.

%     The title page in \fref{figure:titleDB} follows the style of
% of \textit{The Design of Books}~\cite{ADRIANWILSON93}
% and a page similar to Nicholas Basbanes \textit{A Gentle Madness: Bibliophiles,
% Bibliomanes, and the Eternal Passion for Books} is illustrated in
% \fref{figure:titleGM}. These are all from~\cite{TITLEPAGES} and handcrafted.
\fref{figure:titleDB}의 표지는 \textit{The Design of
Books}~\cite{ADRIANWILSON93}의 양식을 따르며, 페이지는 Nicholas Basbanes의
\textit{A Gentle Madness: Bibliophiles, Bibliomanes, and the Eternal Passion
for Books}와 유사한 것이 \fref{figure:titleGM}에 나와 있다.
이들 모두는 \cite{TITLEPAGES}에서 가져온 것이며 수작업된 것들이다.

% In contrast the following code produces the standard \cmd{\maketitle} layout.
반면에 다음 코드는 \cmd{\maketitle}의 표준 레이아웃을 생성한다.

\begin{egsource}{eg:maketitle}
\title{MEANDERINGS}
\author{T. H. E. River \and
        A. Wanderer\thanks{Supported by a grant from the 
        R. Ambler's Fund}\\
        Dun Roamin Institute, NY}
\date{1 April 1993\thanks{First drafted on 29 February 1992}}
...
\maketitle
\end{egsource}

% \begin{egresult}[Example \cs{maketitle} title]{eg:maketitle}
\begin{egresult}[예시 \cs{maketitle} 표제]{eg:maketitle}
%\begin{figure}
%\centering
%\caption{Example \cs{maketitle} title} \label{fig:maketitle}
%\rule{\textwidth}{0.4pt}
%
%\vspace*{2ex}
\begin{center}
\vspace{0.5\onelineskip}
\begin{minipage}{0.75\textwidth}
\begin{center}
{\Large MEANDERINGS} \\
\vspace*{2ex}
T. H. E. River and A. Wanderer\textsuperscript{*} \\
Dun Roamin Institute, NY \\
\vspace*{2ex}
1 April 1993\textsuperscript{\dag}
\begin{displaymath}
\vdots
\end{displaymath}
\end{center}
\begin{footnotesize}
\rule{0.3\textwidth}{0.4pt} \\
\noindent
\textsuperscript{*} Supported by a grant from the R. Ambler's Fund \\
\textsuperscript{\dag} First drafted on 29 February 1992
\end{footnotesize}
\end{minipage}
\vspace{0.75\onelineskip}
\end{center}

%\rule{\textwidth}{0.4pt}
%\end{figure}
\end{egresult}

%    This part of the class is a reimplementation of the \Lpack{titling}
% package~\cite{TITLING}.
본 클래스의 이 부분은 \textsf{titling} 패키지\cite{TITLING}를 재구현한 것이다.

% The class provides a configurable \cmd{\maketitle} command.
% The \cmd{\maketitle} command as defined by the class
% is essentially
이 클래스는 설정 가능한 \cmd{\maketitle} 명령을 제공한다.
\cmd{\maketitle} 명령은 이 클래스에서 본질적으로
\begin{verbatim}
\newcommand{\maketitle}{%
   \vspace*{\droptitle}
   \maketitlehooka
   {\pretitle \title \posttitle}
   \maketitlehookb
   {\preauthor \author \postauthor}
   \maketitlehookc
   {\predate \date \postdate}
   \maketitlehookd
   \thispagestyle{title}
}
\end{verbatim}
% where the \pstyle{title} pagestyle is initially the same as the
% \pstyle{plain} pagestyle.
% The various macros used within \cmd{\maketitle} are described below.
와 같이 정의되는데, 이때 \pstyle{title} 페이지 양식은 처음에는 \pstyle{plain}
페이지 양식과 동일하다.
\cmd{\maketitle} 내에서 사용되는 각종 매크로는 아래에 설명되어 있다.


\begin{syntax}
\cmd{\pretitle}\marg{text} \cmd{\posttitle}\marg{text} \\
\cmd{\preauthor}\marg{text} \cmd{\postauthor}\marg{text} \\
\cmd{\predate}\marg{text} \cmd{\postdate}\marg{text} \\
\end{syntax}
\glossary(pretitle)%
  {\cs{pretitle}\marg{text}}%
  {Command processed before the \cs{title} in \cs{maketitle}.} 
\glossary(posttitle)%
  {\cs{postitle}\marg{text}}%
  {Command processed after the \cs{title} in \cs{maketitle}.} 
\glossary(preauthor)%
  {\cs{preauthor}\marg{text}}%
  {Command processed before the \cs{author} in \cs{maketitle}.} 
\glossary(postauthor)%
  {\cs{postauthor}\marg{text}}%
  {Command processed after the \cs{author} in \cs{maketitle}.} 
\glossary(predate)%
  {\cs{predate}\marg{text}}%
  {Command processed before the \cs{date} in \cs{maketitle}.} 
\glossary(postate)%
  {\cs{postdate}\marg{text}}%
  {Command processed after the \cs{date} in \cs{maketitle}.} 
% These six commands each have a single argument, \meta{text},
% which controls the typesetting of the
% standard elements of the document's \cmd{\maketitle}
% command. The \cmd{\title} is effectively processed between the
% \cmd{\pretitle} and \cmd{\posttitle} commands; that is, like:
위 여섯 개의 명령들은 각자 하나의 인자 \meta{text}를 가지며, 이는 문서에서
\cmd{\maketitle} 명령의 표준 요소의 조판을 통제한다.
\cmd{\title}은 본질적으로 \cmd{\pretitle}과 \cmd{\posttitle} 사이에서
처리되는데, 다음과 같이
\begin{lcode}
{\pretitle \title \posttitle}
\end{lcode}
% and similarly for the \cmd{\author} and \cmd{\date} commands. The
% commands are initialised to mimic the normal result of \cmd{\maketitle}
% typesetting in the \Lclass{report} class.
% That is, the default definitions of the commands are:
되며 \cmd{\author}과 \cmd{\date} 명령에 대해서도 유사하다.
이 명령들은 \Lclass{report} 클래스의 일반적인 \cmd{\maketitle}의 조판 결과를
모방하도록 초기화되어 있다.
즉, 명령들의 기본 정의는 다음과 같다.
\begin{lcode}
\pretitle{\begin{center}\LARGE}
\posttitle{\par\end{center}\vskip 0.5em}
\preauthor{\begin{center}
           \large \lineskip 0.5em%
           \begin{tabular}[t]{c}}
\postauthor{\end{tabular}\par\end{center}}
\predate{\begin{center}\large}
\postdate{\par\end{center}}
\end{lcode}

% They can be changed to obtain different effects. For example to get
% a right justified sans-serif title and a left justifed small caps
% date:
이들은 다른 효과를 내기 위해 바꿀 수 있다.
예를 들어 우측 정렬된 sans-serif 제목과 좌측 정렬된 small caps 날짜를 얻으려면
다음과 같이 하라.
\begin{lcode}
\pretitle{\begin{flushright}\LARGE\sffamily}
\posttitle{\par\end{flushright}\vskip 0.5em}
\predate{\begin{flushleft}\large\scshape}
\postdate{\par\end{flushleft}}
\end{lcode}

\begin{syntax}
\lnc{\droptitle} \\
\end{syntax}
\glossary(droptitle)%
  {\cs{droptitle}}%
  {Length controlling the position of \cs{maketitle} on the page (default 0pt).}
%  The \cmd{\maketitle} command puts the title at a particular height on the
% page.
%  You can change the vertical position of the title via the length
% \lnc{\droptitle}. Giving this a positive value will lower the title and a
% negative value will raise it. The default definition is:
\cmd{\maketitle} 명령은 제목을 페이지의 특정 높이에 놓는다.
여러분은 제목의 세로 위치를 \lnc{\droptitle} 길이를 통해 변경할 수 있다.
여기에 양수를 대입하면 제목을 낮출 것이며, 음수를 대입하면 높일 것이다.
기본 정의는 다음과 같다.
\begin{lcode}
\setlength{\droptitle}{0pt}
\end{lcode}

\begin{syntax}
\cmd{\maketitlehooka} \cmd{\maketitlehookb} \\
\cmd{\maketitlehookc} \cmd{\maketitlehookd} \\
\end{syntax}
\glossary(maketitlehooka)%
  {\cs{maketitlehooka}}%
  {Hook into \cs{maketitle} applied before the \cs{title}.}
\glossary(maketitlehookb)%
  {\cs{maketitlehookb}}%
  {Hook into \cs{maketitle} applied between the \cs{title} and \cs{author}.}
\glossary(maketitlehookc)%
  {\cs{maketitlehookc}}%
  {Hook into \cs{maketitle} applied between the \cs{author} and \cs{date}.}
\glossary(maketitlehookd)%
  {\cs{maketitlehookd}}%
  {Hook into \cs{maketitle} applied after the \cs{date}.}
%  These four hook commands are provided so that additional elements may
% be added to \cmd{\maketitle}. These are initially defined to do nothing
% but can be renewed. For example, some publications
% want a statement about where an article is published immediately
% before the actual titling text. The following defines a command
% \cmd{\published} that can be used to hold the publishing information
% which will then be automatically printed by \cmd{\maketitle}.
이 네 후크 명령은 \cmd{\maketitle}에 추가적인 요소를 넣을 수 있도록 제공된다.
이들은 기본적으로 아무것도 하지 않도록 정의되어 있지만, 재정의할 수 있다.
예를 들어, 일부 출판물에서는 논문이 출판된 곳에 대한 문장을 실제 제목 문구
직전에 삽입하는 것을 요구한다.
다음은 \cmd{\published} 명령을 정의하여 출판 정보를 담을 수 있도록 하며,
\cmd{\maketitle}에 의해 자동적으로 출력된다.
\begin{lcode}
\newcommand{\published}[1]{%
   \gdef\puB{#1}}
\newcommand{\puB}{}
\renewcommand{\maketitlehooka}{%
   \par\noindent \puB}
\end{lcode}
% You can then say:
이후
\begin{lcode}
\published{Originally published in 
          \textit{The Journal of ...}\thanks{Reprinted with permission}}
...
\maketitle
\end{lcode}
% to print both the published and the normal titling information. Note
% that nothing extra had to be done in order to use the \cmd{\thanks} command
% in the argument to the new \cmd{\published} command.
와 같이 하여 출판 정보와 일반적인 제목 정보를 모두 출력할 수 있다.
새로운 \cmd{\published} 명령과 함께 \cmd{\thanks} 명령어를 사용하기 위해서
추가적인 조치가 필요 없다는 것에 주의하라.

\tidx{title page|(,표제|(}
% \index{title page|(}

\begin{syntax}
\senv{titlingpage} text \eenv{titlingpage} \\
\senv{titlingpage*} text \eenv{titlingpage*}\\
\cmd{\titlingpageend}\marg{twoside code}\marg{oneside code} 
\end{syntax}
\glossary(titlingpage)%
  {\senv{titlingpage}}%
  {Environment for a title page, resets the page counter to 1 after it}
\glossary(titlingpage*)%
  {\senv{titlingpage*}}%
  {Like \senv{titlingpage}, but does not reset the page counter.}%
\glossary(titlingpageend)%
  {\cs{titlingpageend}}%
  {Can be used to set what kind of page clearing is issued at the end
    of a titling page. The default for the two args are
    \cs{cleardoublepage} and \cs{clearpage}.}
% When one of the standard classes is used with the \Lopt{titlepage}
% option, \cmd{\maketitle} puts the title elements on an unnumbered page
% and then starts a new page numbered page 1.
% The standard classes also provide
% a \Ie{titlepage} environment which starts a new unnumbered page and at the
% end starts a new page numbered 1. You are entirely responsible
% for specifying exactly what and where is to go on this title page.
% If \cmd{\maketitle} is used  within the \Ie{titlepage} environment it
% will start yet another page.
표준 클래스들과 \Lopt{titlepage} 옵션이 함께 사용되면, \cmd{\maketitle}은
제목 요소를 숫자가 붙지 않은 페이지에 넣고 새로운 페이지를 페이지 번호 1로
시작한다.
표준 클래스는 \Ie{titlepage} 환경도 제공하여, 번호가 붙지 않은 새로운 장을
시작하고 이후 다시 페이지 번호 1부터 새로운 페이지를 시작한다.
이 표지에 어떤 내용을 넣고 어디에 놓을지는 전적으로 여러분의 책임에 달려 있다.
만약 \cmd{\maketitle}이 \Ie{titlepage} 환경 안에서 사용된다면 이는 또 다른
페이지를 시작할 것이다.

%    This class provides neither a \Lopt{titlepage} option nor
% a \Ie{titlepage} environment; instead it provides the \Ie{titlingpage}
% environment which falls between the \Lopt{titlepage}
% option and the \Ie{titlepage} environment. Within the \Ie{titlingpage}
% environment you can use the \cmd{\maketitle} command, and any others
% you wish. The \pstyle{titlingpage} pagestyle is used, and
% at the end it starts another ordinary page numbered one
% (\senv{titlingpage*} does note reset the page number).
% The \pstyle{titlingpage} pagestyle is initially defined to be the same
% as the \pstyle{empty} pagestyle.
이 클래스는 \Ie{titlingpage} 옵션이나 \Lopt{titlepage} 환경 둘 중 어느 것도
제공하지 않는다.
대신 이는 \Lopt{titlepage} 옵션과 \Ie{titlepage} 명령 중간쯤 되는
\Ie{titlingpage} 환경을 제공한다.
여러분은 \Ie{titlingpage} 환경에서 \cmd{\maketitle}을 포함한 명령들을 사용할
수 있다.
\pstyle{titlingpage} 페이지 양식이 사용되며, 끝에는 번호 1이 붙은 일반적인
페이지를 시작한다 (\senv{titlingpage*}는 페이지 번호를 재설정하지 않는다).
\pstyle{titlingpage} 페이지 양식은 \pstyle{empty} 페이지 양식과 같게 초기
설정되어 있다.

% At the end of a \Ie{titlingpage} clearing code is issued, which can
% send you to the next page or the next right handed page. Using
% \cmd{\titlingpageend}\marg{twoside code}\marg{oneside code}, you can
% specify what this clearing code should be. The default is
% \cs{cleardoublepage} and \cs{clearpage} respectively.\footnote{Thus
%   this refactorization will not change existing documents, LM, 2018/03/06.} However a
% better choice might be to just let it follow \cs{clearforchapter}:
\Ie{titlingpage}의 끝에는 지우기 코드가 실행되는데, 이를 통해 다음 페이지 혹은
다음 우츨 페이지로 이동할 수 있다.
\cmd{\titlingpageend}\marg{twoside code}\marg{oneside code}를 사용해서
지우기 코드를 지정할 수 있다.
기본값은 각각 \cs{cleardoublepage}와 \cs{clearpage}이다.\footnote{따라서 이
수정은 기존의 문서를 바꾸지 않을 것이다, LM, 2018/03/06.}
그러나 그냥 \cs{clearforchapter}을 따르도록 하는 것이 더 나은 선택이 될 수
있다.
\begin{lcode}
  \titlingpageend{\clearforchapter}{\clearforchapter}
\end{lcode}
% -- using this value, \Ie{titlingpage} will work as expected with \Lopt{openany}.
이 값을 사용하면, \Ie{titlingpage}는 \Lopt{openany}와 함께 예상대로 작동할


%    For example, to put both the title and an abstract\index{abstract}
% on a title page,
% with a \pstyle{plain} pagestyle:
예를 들어, \pstyle{plain} 페이지 양식으로 제목과 요약\tidx{abstract,요약}을 같은
표지에 두고 싶다면 다음과 같이 한다.
\begin{lcode}
\begin{document}
\begin{titlingpage}
\aliaspagestyle{titlingpage}{plain}
\setlength{\droptitle}{30pt} lower the title
\maketitle
\begin{abstract}...\end{abstract}
\end{titlingpage}
\end{lcode}

% However, it is not required to use \senv{titlingpage} to create a
% title page, you can use regular \ltx\ typesetting without any special
% environment.  That is like:
그러나 \senv{titlingpage}를 사용해서 표지를 만드는 것이 필수는 아니므로,
여러분은 특수한 환경 없이 일반적인 \ltx\ 조판을 사용할 수 있다.
그 방법은 다음과 같을 것이다.
\begin{lcode}
\pagestyle{empty}
%%% Title, author, publisher, etc.,  here
\cleardoublepage
...
\end{lcode}

%    By default, titling information is centered with respect to the
% width of the typeblock\index{typeblock}.
%    Occasionally someone asks on the \texttt{comp.text.tex} newsgroup how
% to center the titling information on a title page
% with respect to the width of the physical
% page. If the typeblock\index{typeblock} is centered with respect to the physical page,
% then the default centering suffices. If the typeblock\index{typeblock} is not physically
% centered, then the titling information either has to be shifted
% horizontally or \cmd{\maketitle} has to be made to think that the typeblock\index{typeblock}
% has been shifted horizontally. The simplest solution is to use the
% \cmd{\calccentering} and \Ie{adjustwidth*} command and environment. For
% example:
기본적으로, 제목 정보는 조판 영역\tidx{typeblock,조판 영역} 너비를 기준으로
가운데 정렬된다.
간혹 누군가 \texttt{comp.text.tex} 뉴스 그룹에 표지의 제목 정보를 실물 페이지
기준으로 가운데 정렬하는 방법을 묻고는 한다.
만약 조판 영역\tidx{typeblock,조판 영역}이 실제 페이지 기준으로 가운데에 있다면
기본 가운데 정렬로 충분할 것이다.
만약 조판 영역\tidx{typeblock,조판 영역}이 실제로 가운데가 아니라면, 제목 정보를
가로 방향으로 이동시키던가, \cmd{\maketitle}에게 조판
영역\tidx{typeblock, 조판 영역}이 가로 방향으로 이동되었다고 믿게할 수 있다.
가장 간단한 해결책은 \cmd{\calccentering} 명령과 \Ie{adjustwidth*} 환경을
사용하는 것이다.
예를 들면 다음과 같다.
\begin{lcode}
\begin{titlingpage}
  \calccentering{\unitlength}
  \begin{adjustwidth*}{\unitlength}{-\unitlength}
    \maketitle
  \end{adjustwidth*}
\end{titlingpage}
\end{lcode}

% \index{title page|)}
\tidx{title page|),표지|)}

\begin{syntax}
\cmd{\title}\marg{text} \cmd{\thetitle} \\
\cmd{\author}\marg{text} \cmd{\theauthor} \\
\cmd{\date}\marg{text} \cmd{\thedate} \\
\end{syntax}
\glossary(title)
  {\cs{title}\marg{text}}%
  {Used by \cs{maketitle} to typeset \meta{text} as the document  title.}
\glossary(thetitle)
  {\cs{thetitle}}%
  {Copy of \meta{text} from \cs{title}.}
\glossary(author)
  {\cs{author}\marg{text}}%
  {Used by \cs{maketitle} to typeset \meta{text} as the document author.}
\glossary(theauthor)
  {\cs{theauthor}}%
  {Copy of \meta{text} from \cs{author}.}
\glossary(date)
  {\cs{date}\marg{text}}%
  {Used by \cs{maketitle} to typeset \meta{text} as the document date.}
\glossary(thedate)
  {\cs{thedate}}%
  {Copy of \meta{text} from \cs{date}.}

%    In the usual document classes, the contents (\meta{text}) of
% the \cmd{\title}, \cmd{\author} and \cmd{\date}
% macros used for \cmd{\maketitle} are unavailable once \cmd{\maketitle}
% has been
% issued. The class provides the \cmd{\thetitle},
% \cmd{\theauthor} and \cmd{\thedate} commands that can be used for printing
% these elements of the title later in the document,
% if desired.
일반적인 문서 클래스에서 \cmd{\maketitle}을 위해 사용되는 \cmd{\title},
\cmd{\author}과 \cmd{date} 매크로의 내용들(\meta{text})은 \cmd{\maketitle} 사용
이후 접근할 수 없다.
본 클래스는 \cmd{\thetitle}, \cmd{\theauthor}과 \cmd{\thedate} 명령을 제공하여
원한다면 이후 문서에서 제목 요소를 출력하는데 사용할 수 있도록 한다.

\begin{syntax}
\cmd{\and} \cmd{\andnext} \\
\end{syntax}
\glossary(and)%
  {\cs{and}}%
  {Use within the argument to \cs{author} to separate author's names.}
\glossary(andnext)%
  {\cs{andnext}}%
  {Use within the argument to \cs{author} to insert a newline..}
%    The macro \cmd{\and} is used within the argument to the
% \cmd{\author} command to add some extra space between the author's names.
% The class \cmd{\andnext} macro inserts a newline instead of a space.
% Within the \cmd{\theauthor} macro both \cmd{\and} and \cmd{\andnext}
% are replaced by a comma.
매크로 \cmd{\and}는 \cmd{\author} 명령어의 인자 안에 쓰이며, 저자들의 이름
사이에 추가 공백을 넣어준다.
이 클래스의 \cmd{\andnext} 매크로는 공백 대신에 새로운 줄을 넣어준다.
\cmd{\theauthor} 매크로 안에서 \cmd{\and}와 \cmd{\andnext}는 둘 다 쉼표로
치환된다.

%    The class does not follow the standard classes' habit
% of automatically killing the titling
% commands after \cmd{\maketitle} has been issued. You can have multiple
% \cmd{\title}, \cmd{\author}, \cmd{\date} and \cmd{\maketitle}
% commands in your
% document if you wish. For example, some reports are issued with
% a title page, followed by an executive summary, and then they
% have another, possibly modified, title at the start of
% the main body of the report. This can be accomplished like this:
본 클래스는 표준 클래스처럼 \cmd{\maketitle} 사용 이후에 표지 명령을 자동으로
끄는 관행을 따르지 않는다.
여러분은 원한다면 문서에 여러 개의 \cmd{\title}, \cmd{\author}, \cmd{date},
그리고 \cmd{\maketitle} 명령들을 넣을 수 있다.
예를 들어, 일부 보고서는 표지로 시작해서 뒤이어 요약문이 따르고, 다르게 표현될
수도 있는 또다른 제목을 본문 앞에 가진다.
이는 다음과 같이 구현할 수 있다.
\begin{lcode}
\title{Cover title}
...
\begin{titlingpage}
\maketitle
\end{titlingpage}
...
\title{Body title}
\maketitle
...
\end{lcode}

\begin{syntax}
\cmd{\killtitle} \cmd{\keepthetitle} \\
\cmd{\emptythanks} \\
\end{syntax}
\glossary(killtitle)%
  {\cs{killtitle}}%
  {Makes all aspects of \cs{maketitle} unavailable.}
\glossary(keepthetitle)%
  {\cs{keepthetitle}}%
  {Makes most aspects of \cs{maketitle} unavailable but keeps \cs{thetitle},
  \cs{theauthor} and \cs{thedate}.}
\glossary(emptythanks)%
  {\cs{emptythanks}}%
  {Discards any text from previous uses of \cs{thanks}.}
%     The \cmd{\killtitle} macro makes all aspects of titling, including
% \cmd{\thetitle} etc.,
% unavailable from the point that it is issued (using this command will save
% some macro space if the \cmd{\thetitle}, etc., commands are not required).
% Using this command is the class's manual version
% of the automatic killing performed by the standard classes.
% The \cmd{\keepthetitle} command performs a similar function, except that
% it keeps the \cmd{\thetitle}, \cmd{\theauthor} and \cmd{\thedate} commands,
%  while killing everything else.
\cmd{\killtitle} 매크로 사용 후에 \cmd{\thetitle} 등을 포함한 모든 표지 관련
명령을 사용하지 못하도록  한다 (이 명령을 사용해 \cmd{\thetitle} 등과 같은
명령이 필요하지 않을 경우 매크로 공간을 확보할 수 있다).
이 명령은 본 클래스에서 표준 클래스의 자동 종료 수행의 수동 버전이다.
\cmd{\keepthetitle} 명령은 비슷한 기능을 하지만, 다른 모든 기능은 끄면서
\cmd{\thetitle}, \cmd{\theauthor}과 \cmd{\thedate} 명령을 유지한다는 차이가
있다.

% The \cmd{\emptythanks} command discards any text from prior use of
% \cmd{\thanks}.
% This command is useful when \cmd{\maketitle} is used multiple times ---
% the \cmd{\thanks} commands in each use just stack up the texts for output
% at each use, so each subsequent use of \cmd{\maketitle} will output all
% previous \cmd{\thanks} texts together with any new ones. To avoid this,
% put \cmd{\emptythanks} before each \cmd{\maketitle} after the first.
\cmd{\emptythanks} 명령은 기존 \cmd{\thanks}의 모든 문구를 지운다.
이 명령은 \cmd{\maketitle}이 여러번 사용될 경우 유용하다.
\cmd{\thanks} 명령은 매 사용마다 문구를 쌓아올리므로, \cmd{\maketitle}을 매번
사용할 때 기존의 모든 \cmd{\thanks} 문구가 새 문구와 함께 출력될 것이다.
이를 방지하기 위해서는 \cmd{\maketitle}을 사용하기 전에 매번
\cmd{\emptythanks}를 넣으면 된다.

% \index{title!styling|)}
\tidx{title!styling|),표제!양식화}


% \section{Styling the thanks} \label{sec:thanks}
\section{감사의 말 양식화하기} \label{sec:thanks}


% \index{thanks}
% \index{thanks!styling|(}
\tidx{thanks,감사의 말}
\tidx{thanks!styling|(,감사의 말!양식화|(}

    % The class provides a configurable \cmd{\thanks} command.
본 클래스는 설정 가능한 \cmd{\thanks} 명령을 제공한다.

\begin{syntax}
\cmd{\thanksmarkseries}\marg{format} \\
\cmd{\symbolthanksmark} \\
\end{syntax}
\glossary(thanksmarkseries)%
  {\cs{thanksmarkseries}\marg{format}}%
  {Thanks marks will be printed using \meta{format} series of symbols.}
\glossary(symbolthanksmark)%
  {\cs{symbolthanksmark}}
  {Set the thanks marks to be printed using the footnote series of symbols.}
%  Any \cmd{\thanks} are marked with symbols in the
% titling and footnotes\index{footnote}.
% The command \cmd{\thanksmarkseries}
% can be used to change the marking style. The \meta{format} argument
% is the name of one of the formats for printing a counter. The name
% is the same as that of a counter format but without the backslash.
% To have the \cmd{\thanks} marks as lowercase letters instead of symbols
% do:
모든 \cmd{\thanks}는 표지나 각주\tidx{footnote,각주}에 기호로 표시된다.
명령 \cmd{\thanksmarkseries}는 이러한 표지 양식을 바꾸는데 사용될 수 있다.
\meta{format} 인자는 카운터를 출력하는 형식들 중 하나의 이름이다.
이름은 카운터 형식과 같지만 백슬래시를 포함하지 않는다.
\cmd{\thanks}가 기호 대신에 소문자로 표시되기 위해서는 다음과 같이 하라.
\begin{lcode}
\thanksmarkseries{alph}
\end{lcode}
% Just for convenience the \cmd{\symbolthanksmark} command sets the series
% to be footnote\index{footnote} symbols.
% Using this class the potential names for \meta{format} are:
% \texttt{arabic}, \texttt{roman}, \texttt{Roman}, \texttt{alph},
% \texttt{Alph}, and \texttt{fnsymbol}.
편의를 위해서 \cmd{\symbolthanksmark}는 목록을 각주\tidx{footnote,각주} 기호로
지정한다.
이 클래스를 사용하면 \meta{format}에는 \texttt{arabic}, \texttt{roman},
\texttt{Roman}, \texttt{alph}, \texttt{Alph} 그리고 \texttt{fnsymbol}이
이름으로 올 수 있다.

\begin{syntax}
\cmd{\continuousmarks} \\
\end{syntax}
\glossary(continuousmarks)%
  {\cs{continuousmarks}}%
  {Specifies that the thanks/footnote marker is not zeroed after titling.}
The \cmd{\thanks} command uses the \Icn{footnote} counter, 
and normally the counter
is zeroed after the titling so that the footnote marks\index{footnote!mark} start from 1.
If the counter should not be zeroed, then just specify 
\cmd{\continuousmarks}.
This might be required if numerals are used as the thanks markers.

\begin{syntax}
\cmd{\thanksheadextra}\marg{pre}\marg{post} \\
\end{syntax}
\glossary(thanksheadextra)%
  {\cs{thanksheadextra}\marg{pre}\marg{post}}%
  {Inserts \meta{pre} and \meta{post} before and after thanks markers
   in the titling code.}
The \cmd{\thanksheadextra} command will insert
\meta{pre} and \meta{post} before and after the thanks markers in the
titling block. By default \meta{pre} and \meta{post} are empty.
For example, to put parentheses round the titling markers do:
\begin{lcode}
\thanksheadextra{(}{)}
\end{lcode}


\begin{syntax}
\cmd{\thanksmark}\marg{n} \\
\end{syntax}
\glossary(thanksmark)%
  {\cs{thanksmark}\marg{n}}%
  {Prints a thanks mark identical to the n'th (previously) printed mark.}
It is sometimes desireable to have the same thanks text be applied to,
say, four out of six authors, these being the first 3 and the last one.
The command \cmd{\thanksmark}\marg{n} is similar to 
\cmd{\footnotemark}\oarg{n} in that it prints a thanks mark identical
to that of the \meta{n}'th  \cmd{\thanks} command. No changes are made
to any thanks in the footnotes\index{footnote}. For instance, in the following
the authors Alpha and Omega will have the same mark:
\begin{lcode}
\title{The work\thanks{Draft}}
\author{Alpha\thanks{ABC},
        Beta\thanks{XYZ} and 
        Omega\thanksmark{2}} 
\maketitle
\end{lcode}

%\begin{syntax}
%\cmd{\thanksgap}\marg{length} \\
%\end{syntax}
%The marks in the titling block printed by the 
%\cmd{\thanks} and \cmd{\thanksmark}
%commands take zero space. This is acceptable if they come at the end of
%a line, but not in the middle of a line. Use the 
%\cmd{\thanksgap} command immediately after a mid-line
%\cmd{\thanks} or \cmd{\thanksmark} command to add \meta{length} amount of
%space before the next word. For example, if there are three authors
%from two different institutions:
%\begin{lcode}
%\author{Alpha\thanks{ABC},
%        Omega\thanks{XYZ}\thanksgap{1ex} and 
%        Beta\thanksmark{1}} 
%\end{lcode}

\begin{syntax}
\cmd{\thanksmarkstyle}\marg{defn} \\
\end{syntax}
\glossary(thanksmarkstyle)%
  {\cs{thanksmarkstyle}\marg{defn}}%
  {Sets the style for the thanks marks at the foot.}
By default the thanks mark at the foot is typeset as a superscript. In
the class this is specifed via
\begin{lcode}
\thanksmarkstyle{\textsuperscript{#1}}
\end{lcode}
where \verb?#1? will be replaced by the thanks mark symbol. You can change
the mark styling
if you wish. For example, to put parentheses round the
mark and typeset it at normal size on the baseline:
\begin{lcode}
\thanksmarkstyle{(#1)}
\end{lcode}


\begin{syntax}
\lnc{\thanksmarkwidth} \\
\end{syntax}
\glossary(thanksmarkwidth)%
  {\cs{thanksmarkwidth}}%
  {Width of box for the thanks marks at the foot.}
 The thanks mark in the footnote\index{footnote} is typeset right justified in a box
of width \lnc{\thanksmarkwidth}. The first line of the thanks text starts
after this box. The initialisation is 
\begin{lcode}
\setlength{\thanksmarkwidth}{1.8em}
\end{lcode}
giving the default position.

\begin{syntax}
\lnc{\thanksmarksep} \\
\end{syntax}
\glossary(thanksmarksep)%
  {\cs{thanksmarksep}}%
  {Indentation of second and subsequent thanks text lines at the foot.}
The value of the length
 \lnc{\thanksmarksep} controls the indentation the
second and subsequent lines of the thanks text, with respect to
the end of the mark box. As examples: 
\begin{lcode}
\setlength{\thanksmarksep}{0em}
\end{lcode}
 will align the left hand ends of of a multiline thanks text, while: 
\begin{lcode}
\setlength{\thanksmarksep}{-\thanksmarkwidth}
\end{lcode}
will left justify any second and subsequent lines of the thanks text. 
This last
setting is the initialised value, giving the default appearance.

\begin{syntax}
\cmd{\thanksfootmark} \\
\end{syntax}
\glossary(thanksfootmark)%
  {\cs{thanksfootmark}}%
  {Typesets a thanks mark at the foot.}
    A thanks mark in the footnote\index{footnote} region is typeset by \cmd{\thanksfootmark}.
The code for this is roughly:
\LMnote{2012/07/02}{Text changed to reflect the actual code}
\begin{lcode}
\newcommand{\thanksfootmark}{%
  \hbox to\thanksmarkwidth{\hfil\normalfont%
     \thanksscript{\thefootnote}}}
\end{lcode}
You should not need to change the definition
of \cmd{\thanksfootmark} 
but you may want to change the default definitions of one or more
of the macros it uses.

\begin{syntax}
\cmd{\thanksscript}\marg{text} \\
\end{syntax}
\glossary(thanksfootmark)%
  {\cs{thanksfootmark}}%
  {Handle the inner part of the thanks mark at the foot.}
This is initially defined as: 
\begin{lcode}
\newcommand{\thanksscript}[1]{\textsuperscript{#1}}
\end{lcode}
so that \cmd{\thanksscript} typesets its argument as a superscript, which
is the default for thanks notes.



\begin{comment}

\begin{syntax}
\cmd{\thanksscript}\marg{text} \\
\end{syntax}
Note that the thanks mark, together with the \verb?\...pre? and \verb?\...post?
commands form the \meta{text} argument to the \cmd{\thanksscript} command. 
This is initially defined as: 
\begin{lcode}
\newcommand{\thanksscript}[1]{\textsuperscript{#1}}
\end{lcode}
so that \cmd{\thanksscript} typesets its argument as a superscript, which
is the default for thanks notes. If you would prefer the mark to be
set at the baseline instead, for example, just do: 
\begin{lcode}
\renewcommand{\thanksscript}[1]{#1}
\end{lcode}
 and if you also wanted very small symbols on the baseline you could do:
\begin{lcode}
\renewcommand{\thanksscript}[1]{\tiny #1}
\end{lcode}

Alternatively 
\begin{lcode}
\renewcommand{\thanksscript}[1]{#1}
\thanksfootextra{\bfseries}{.}
\end{lcode}
will give a bold baseline mark followed by a dot.

   Using combinations of these macros you can easily define
different layouts for the thanks footnotes\index{footnote}. Here are some
sample, but to shorten them I have ignored any
\cs{renewcommand}s and \cs{setlength}s, leaving these to be implied
as necessary.
\begin{itemize}
\item Setting \verb?\thanksfootextra{}{\hfill}? left justifies the mark
    in its box:
   \begin{itemize}
   \item \verb?\thanksscript{\llap{#1\space}}}, \verb?\thanksmarkwidth{0em}? and \\
         \verb?\thanksmargin{0em}? puts the baseline mark in the margin\index{margin}
         and the text left justified.
   \item Using \verb?\thanksscript{#1}?, \verb?\thanksmarkwidth{1em}? and \\
         \verb?\thanksmargin{-\thanksmarkwidth}? makes the baseline mark 
         and text left adjusted.
   \item \verb?\thanksscript{#1}?, \verb?\thanksmarkwidth{1em}? and
         \verb?\thanksmargin{0em}? puts the baseline mark left adjusted
         and the text indented and aligned, like this marked
         sentence is typeset.
   \end{itemize}

\item Setting \verb?\thanksfootextra{ }? and \verb?\thanksscript{#1}? 
      right justifies the baseline mark and a space in the mark box:
   \begin{itemize}
   \item The normal style is
         defined by \verb?\thanksmarkwidth{1.8em}? and \\
         \verb?\thanksmargin{-\thanksmarkwidth}? which put the mark 
         indented and the text left adjusted, like a normal indented
         paragraph\index{paragraph!indentation}.
   \item \verb?\thanksmarkwidth{1.8em}? and
         \verb?\thanksmargin{0em}? put the mark indented 
         and the text indented and aligned.
   \end{itemize}

\end{itemize}

%%%%%%%%%%%%%%%%%%%%%
\end{comment}
%%%%%%%%%%%%%%%%%%%%

\begin{syntax}
\cmd{\makethanksmark} \\
\cmd{\makethanksmarkhook} \\
\end{syntax}
\glossary(makethanksmark)%
  {\cs{makethanksmark}}%
  {Typesets the thanks marker and text at the foot.}
\glossary(makethanksmarkhook)
  {\cs{makethanksmarkhook}}
  {Code hook into \cs{makethanksmark}.}
The macro \cmd{\makethanksmark} typesets both the thanks marker (via
\cmd{\thanksfootmark}) and the thanks text. You probably will not need
to change its default definition. Just in case, though, 
\cmd{\makethanksmark}
calls the macro \cmd{\makethanksmarkhook} before it does any typesetting.
The default definition of this is: 
\begin{lcode}
\newcommand{\makethanksmarkhook}{}
\end{lcode}
which does nothing.

   You can redefine \cmd{\makethanksmarkhook} to do something useful. For
example, if you wanted a slightly bigger baseline skip you could do:
\begin{lcode}
\renewcommand{\makethanksmarkhook}{\fontsize{8}{11}\selectfont}
\end{lcode}
where the numbers \texttt{8} and \texttt{11} specify the point size of the font 
and the baseline skip
respectively. In this example 8pt is the normal \cmd{\footnotesize} in
a 10pt document, and 11pt is the baselineskip for \cmd{\footnotesize}
text in an 11pt document (the normal baseline skip for \cmd{\footnotesize}
in a 10pt document is 9.5pt);
adjust these numbers to suit.

\begin{syntax}
\cmd{\thanksrule} \\
\cmd{\usethanksrule} \\
\cmd{\cancelthanksrule} \\
\end{syntax}
\glossary(thanksrule)%
  {\cs{thanksrule}}%
  {The rule to be typeset before the thanks in the foot.}
\glossary(usethanksrule)%
  {\cs{usethanksrule}}%
  {Specifies that the \cs{thanksrule} is to be typeset in the 
   \texttt{titlingpage} environment.}
\glossary(cancelthanksrule)%
  {\cs{cancelthanksrule}}%
  {Specifies that the \cs{footnoterule} is to be used from now on.}
By default, there is no rule above \cmd{\thanks}
text that appears in a \Ie{titlingpage} environment.
If you want a rule in that environment, put \cmd{\usethanksrule} 
before the \cmd{\maketitle} command, which will then print a rule according
to the current definition of \cmd{\thanksrule}.
\cmd{\thanksrule} is initialised to be a copy of \cmd{\footnoterule} as it 
is defined at the
end of the preamble\index{preamble}. The definition of \cmd{\thanksrule} can be changed
after \verb?\begin{document}?. If the definition of \cmd{\thanksrule} is modified
and a \cmd{\usethanksrule} command has been issued, then the redefined
rule may also be used for footnotes\index{footnote}. Issuing the command 
\cmd{\cancelthanksrule} will cause the normal \cmd{\footnoterule} definition
to be used from thereon; another \cmd{\usethanksrule}
command can be issued later
if you want to swap back again.

The parameters for the vertical positioning of footnotes\index{footnote} 
and thanks notes, and the default \cmd{\footnoterule} are the same
(see \fref{fig:fn} on \pref{fig:fn}).
You will have to change one or more of these if the vertical spacings of 
footnotes\index{footnote}
and thanks notes are meant to be different.

\index{thanks!styling|)}

%#% extend
%#% extstart include abstracts.tex
