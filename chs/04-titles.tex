\chapter{Titles}

   The standard classes provide little in the way of help in setting
the title page(s) for your work, as the \cmd{\maketitle} command
is principally aimed at generating a title for an article in a technical
journal; it provides little for titles for works like theses, reports or
books. For these I recommend that you design your own title page 
layout\footnote{If you are producing a thesis you are probably told 
just how it must look.}
using the regular \ltx\ commands to lay it out, and ignore \cmd{\maketitle}.

    Quoting from Ruari McLean~\cite[p. 148]{MCLEAN80} in reference to the 
title page he says:
\begin{quotation}
    The title-page states, in words, the actual title (and sub-title, if 
there is one) of the book and the name of the author and publisher, and
sometimes also the number of illustrations, but it should do more than that.
From the designer's point of view, it is the most important page in the
book: it sets the style. It is the page which opens communication with the
reader\ldots

    If illustrations play a large part in the book, the title-page opening 
should, or may, express this visually. If any form of decoration is used 
inside the book, e.g., for chapter openings, one would expect this to be
repeated or echoed on the title-page.

    Whatever the style of the book, the title-page should give a foretaste
of it. If the book consists of plain text, the title-page should at least 
be in harmony with it. The title itself should not exceed the width of the
type area, and will normally be narrower\ldots
\end{quotation}

    A pastiche of McLean's title page is shown in \fref{figure:titleTH}.

\begin{figure}
\centering
\begin{showtitle}
\titleTH
\end{showtitle}
\caption{Layout of a title page for a book on typography}\label{figure:titleTH}
\end{figure}


\begin{comment}
Figures~\ref{fig:titleTH} and~\ref{fig:titleDB}, for example, 
show title pages created using normal \ltx, without bothering 
with \cmd{\maketitle}.
\end{comment}

   The typeset format of the \cmd{\maketitle} command is virtually fixed
within the \ltx\ standard classes. This class
provides a set of formatting commands that can be used to modify
the appearance of the title\index{title} information; that is, the contents of
the \cmd{\title}, \cmd{\author} and \cmd{\date} commands. 
It also keeps the values
of these commands so that they may be printed again later in the
document.
   The class also inhibits the normal automatic cancellation of titling
commands after \cmd{\maketitle}. This means that you can have multiple
instances of the same, or perhaps different, titles in one document;
for example on a half title page\index{half-title page} and the
full title page\index{title page}.
Hooks are provided so that additional titling elements can be defined
and printed by \cmd{\maketitle}.
  The \cmd{\thanks} command is enhanced to provide various configurations
for both the marker symbol and the layout of the thanks\index{thanks} notes.

\begin{figure}
\centering
\begin{showtitle}
\titleDS
\end{showtitle}
\caption{Example of a mandated title page style for a doctoral thesis}\label{figure:titleDS}
\end{figure}

\begin{figure}%\setlength{\unitlength}{1pt}
\centering
%\begin{showtitle}
{\titleRB}
%\end{showtitle}
\caption{Example of a Victorian title page}\label{figure:titleRB}
\end{figure}

\begin{figure}
\centering
\begin{showtitle}
\titleDB
\end{showtitle}
\caption{Layout of a title page for a book on book design}\label{figure:titleDB}
\end{figure}

\begin{figure}
\centering
\begin{showtitle}
\titleGM
\end{showtitle}
\caption{Layout of a title page for a book about books}\label{figure:titleGM}
\end{figure}

    Generally speaking, if you want anything other than minor variations
on the \cmd{\maketitle} layout then it is better to ignore \cmd{\maketitle}
and take the whole layout into your own hands so you can place everthing 
just where you want it on the page.

\section{Styling the titling}


\index{title!styling|(}

   The facilities provided for typesetting titles are limited, essentially
catering for the kind of titles of articles published in technical journals.
They can also be used as a quick and dirty method for typesetting titles
on reports, but for serious work, such as a title page for a book or thesis,
each title page\index{title page} should be handcrafted. 
For instance, a student of mine, Donald Sanderson\index{Sanderson, Donald}
used \ltx\ to typeset his doctoral thesis, and \fref{figure:titleDS} shows 
the title page style mandated by Rensselear Polytechnic Institute as of 1994.
Many other examples of title pages, together with the code to create them,
are in~\cite{TITLEPAGES}.

    Another handcrafted title page\index{title page} from~\cite{TITLEPAGES} is 
shown in \fref{figure:titleRB}. This one is based on an old booklet I found 
that was 
published towards the end of the 19th century and exhibits the love of 
Victorian printers in displaying a variety of types; the rules are
an integral part of the title page. For the purposes of this manual I 
have used New Century\index{New Century Schoolbook} Schoolbook, which 
is part of the regular
\ltx\ distribution, rather than my original choice of 
Century Old Style\index{Century Old Style}
which is one of the commercial FontSite\index{FontSite} fonts licensed from 
the 
SoftMaker/ATF library, supported for \ltx\ through 
Christopher\index{League, Christopher}
League's estimable work~\cite{TEXFONTSITE}.

    The title page in \fref{figure:titleDB} follows the style of 
of \textit{The Design of Books}~\cite{ADRIANWILSON93}
and a page similar to Nicholas Basbanes \textit{A Gentle Madness: Bibliophiles, 
Bibliomanes, and the Eternal Passion for Books} is illustrated in 
\fref{figure:titleGM}. These are all from~\cite{TITLEPAGES} and handcrafted.

In contrast the following code produces the standard \cmd{\maketitle} layout.

\begin{egsource}{eg:maketitle}
\title{MEANDERINGS}
\author{T. H. E. River \and
        A. Wanderer\thanks{Supported by a grant from the 
        R. Ambler's Fund}\\
        Dun Roamin Institute, NY}
\date{1 April 1993\thanks{First drafted on 29 February 1992}}
...
\maketitle
\end{egsource}

\begin{egresult}[Example \cs{maketitle} title]{eg:maketitle}
%\begin{figure}
%\centering
%\caption{Example \cs{maketitle} title} \label{fig:maketitle}
%\rule{\textwidth}{0.4pt}
%
%\vspace*{2ex}
\begin{center}
\vspace{0.5\onelineskip}
\begin{minipage}{0.75\textwidth}
\begin{center}
{\Large MEANDERINGS} \\
\vspace*{2ex}
T. H. E. River and A. Wanderer\textsuperscript{*} \\
Dun Roamin Institute, NY \\
\vspace*{2ex}
1 April 1993\textsuperscript{\dag}
\begin{displaymath}
\vdots
\end{displaymath}
\end{center}
\begin{footnotesize}
\rule{0.3\textwidth}{0.4pt} \\
\noindent
\textsuperscript{*} Supported by a grant from the R. Ambler's Fund \\
\textsuperscript{\dag} First drafted on 29 February 1992
\end{footnotesize}
\end{minipage}
\vspace{0.75\onelineskip}
\end{center}

%\rule{\textwidth}{0.4pt}
%\end{figure}
\end{egresult}

   This part of the class is a reimplementation of the \Lpack{titling}
package~\cite{TITLING}.

The class provides a configurable \cmd{\maketitle} command.
The \cmd{\maketitle} command as defined by the class 
is essentially
\begin{lcode}
\newcommand{\maketitle}{%
   \vspace*{\droptitle}
   \maketitlehooka
   {\pretitle \title \posttitle}
   \maketitlehookb
   {\preauthor \author \postauthor}
   \maketitlehookc
   {\predate \date \postdate}
   \maketitlehookd
   \thispagestyle{title}
}
\end{lcode}
where the \pstyle{title} pagestyle is initially the same as the
\pstyle{plain} pagestyle.
The various macros used within \cmd{\maketitle} are described below.


\begin{syntax}
\cmd{\pretitle}\marg{text} \cmd{\posttitle}\marg{text} \\
\cmd{\preauthor}\marg{text} \cmd{\postauthor}\marg{text} \\
\cmd{\predate}\marg{text} \cmd{\postdate}\marg{text} \\
\end{syntax}
\glossary(pretitle)%
  {\cs{pretitle}\marg{text}}%
  {Command processed before the \cs{title} in \cs{maketitle}.} 
\glossary(posttitle)%
  {\cs{postitle}\marg{text}}%
  {Command processed after the \cs{title} in \cs{maketitle}.} 
\glossary(preauthor)%
  {\cs{preauthor}\marg{text}}%
  {Command processed before the \cs{author} in \cs{maketitle}.} 
\glossary(postauthor)%
  {\cs{postauthor}\marg{text}}%
  {Command processed after the \cs{author} in \cs{maketitle}.} 
\glossary(predate)%
  {\cs{predate}\marg{text}}%
  {Command processed before the \cs{date} in \cs{maketitle}.} 
\glossary(postate)%
  {\cs{postdate}\marg{text}}%
  {Command processed after the \cs{date} in \cs{maketitle}.} 
These six commands each have a single argument, \meta{text},
which controls the typesetting of the 
standard elements of the document's \cmd{\maketitle}
command. The \cmd{\title} is effectively processed between the 
\cmd{\pretitle} and \cmd{\posttitle} commands; that is, like:
\begin{lcode}
{\pretitle \title \posttitle}
\end{lcode}
and similarly for the \cmd{\author} and \cmd{\date} commands. The 
commands are initialised to mimic the normal result of \cmd{\maketitle}
typesetting in the \Lclass{report} class.
That is, the default definitions of the commands are:
\begin{lcode}
\pretitle{\begin{center}\LARGE}
\posttitle{\par\end{center}\vskip 0.5em}
\preauthor{\begin{center}
           \large \lineskip 0.5em%
           \begin{tabular}[t]{c}}
\postauthor{\end{tabular}\par\end{center}}
\predate{\begin{center}\large}
\postdate{\par\end{center}}
\end{lcode}

They can be changed to obtain different effects. For example to get
a right justified sans-serif title and a left justifed small caps
date:
\begin{lcode}
\pretitle{\begin{flushright}\LARGE\sffamily}
\posttitle{\par\end{flushright}\vskip 0.5em}
\predate{\begin{flushleft}\large\scshape}
\postdate{\par\end{flushleft}}
\end{lcode}

\begin{syntax}
\lnc{\droptitle} \\
\end{syntax}
\glossary(droptitle)%
  {\cs{droptitle}}%
  {Length controlling the position of \cs{maketitle} on the page (default 0pt).}
 The \cmd{\maketitle} command puts the title at a particular height on the 
page. 
 You can change the vertical position of the title via the length
\lnc{\droptitle}. Giving this a positive value will lower the title and a
negative value will raise it. The default definition is: 
\begin{lcode}
\setlength{\droptitle}{0pt}
\end{lcode}

\begin{syntax}
\cmd{\maketitlehooka} \cmd{\maketitlehookb} \\
\cmd{\maketitlehookc} \cmd{\maketitlehookd} \\
\end{syntax}
\glossary(maketitlehooka)%
  {\cs{maketitlehooka}}%
  {Hook into \cs{maketitle} applied before the \cs{title}.}
\glossary(maketitlehookb)%
  {\cs{maketitlehookb}}%
  {Hook into \cs{maketitle} applied between the \cs{title} and \cs{author}.}
\glossary(maketitlehookc)%
  {\cs{maketitlehookc}}%
  {Hook into \cs{maketitle} applied between the \cs{author} and \cs{date}.}
\glossary(maketitlehookd)%
  {\cs{maketitlehookd}}%
  {Hook into \cs{maketitle} applied after the \cs{date}.}
 These four hook commands are provided so that additional elements may
be added to \cmd{\maketitle}. These are initially defined to do nothing
but can be renewed. For example, some publications
want a statement about where an article is published immediately
before the actual titling text. The following defines a command
\cmd{\published} that can be used to hold the publishing information
which will then be automatically printed by \cmd{\maketitle}.
\begin{lcode}
\newcommand{\published}[1]{%
   \gdef\puB{#1}}
\newcommand{\puB}{}
\renewcommand{\maketitlehooka}{%
   \par\noindent \puB}
\end{lcode}
You can then say:
\begin{lcode}
\published{Originally published in 
          \textit{The Journal of ...}\thanks{Reprinted with permission}}
...
\maketitle
\end{lcode}
to print both the published and the normal titling information. Note
that nothing extra had to be done in order to use the \cmd{\thanks} command
in the argument to the new \cmd{\published} command.

\index{title page|(}

\begin{syntax}
\senv{titlingpage} text \eenv{titlingpage} \\
\senv{titlingpage*} text \eenv{titlingpage*}\\
\cmd{\titlingpageend}\marg{twoside code}\marg{oneside code} 
\end{syntax}
\glossary(titlingpage)%
  {\senv{titlingpage}}%
  {Environment for a title page, resets the page counter to 1 after it}
\glossary(titlingpage*)%
  {\senv{titlingpage*}}%
  {Like \senv{titlingpage}, but does not reset the page counter.}%
\glossary(titlingpageend)%
  {\cs{titlingpageend}}%
  {Can be used to set what kind of page clearing is issued at the end
    of a titling page. The default for the two args are
    \cs{cleardoublepage} and \cs{clearpage}.}
When one of the standard classes is used with the \Lopt{titlepage}
option, \cmd{\maketitle} puts the title elements on an unnumbered page
and then starts a new page numbered page 1. 
The standard classes also provide
a \Ie{titlepage} environment which starts a new unnumbered page and at the
end starts a new page numbered 1. You are entirely responsible
for specifying exactly what and where is to go on this title page.
If \cmd{\maketitle} is used  within the \Ie{titlepage} environment it
will start yet another page.

   This class provides neither a \Lopt{titlepage} option nor
a \Ie{titlepage} environment; instead it provides the \Ie{titlingpage}
environment which falls between the \Lopt{titlepage}
option and the \Ie{titlepage} environment. Within the \Ie{titlingpage}
environment you can use the \cmd{\maketitle} command, and any others 
you wish. The \pstyle{titlingpage} pagestyle is used, and 
at the end it starts another ordinary page numbered one
(\senv{titlingpage*} does note reset the page number). 
The \pstyle{titlingpage} pagestyle is initially defined to be the same
as the \pstyle{empty} pagestyle.

At the end of a \Ie{titlingpage} clearing code is issued, which can
send you to the next page or the next right handed page. Using
\cmd{\titlingpageend}\marg{twoside code}\marg{oneside code}, you can
specify what this clearing code should be. The default is
\cs{cleardoublepage} and \cs{clearpage} respectively.\footnote{Thus
  this refactorization will not change existing documents, LM, 2018/03/06.} However a
better choice might be to just let it follow \cs{clearforchapter}:
\begin{lcode}
  \titlingpageend{\clearforchapter}{\clearforchapter}
\end{lcode}
-- using this value, \Ie{titlingpage} will work as expected with \Lopt{openany}.


   For example, to put both the title and an abstract\index{abstract} 
on a title page,
with a \pstyle{plain} pagestyle:
\begin{lcode}
\begin{document}
\begin{titlingpage}
\aliaspagestyle{titlingpage}{plain}
\setlength{\droptitle}{30pt} lower the title
\maketitle
\begin{abstract}...\end{abstract}
\end{titlingpage}
\end{lcode}

However, it is not required to use \senv{titlingpage} to create a
title page, you can use regular \ltx\ typesetting without any special
environment.  That is like:
\begin{lcode}
\pagestyle{empty}
%%% Title, author, publisher, etc.,  here
\cleardoublepage
...
\end{lcode}

   By default, titling information is centered with respect to the
width of the typeblock\index{typeblock}.
   Occasionally someone asks on the \texttt{comp.text.tex} newsgroup how
to center the titling information on a title page 
with respect to the width of the physical 
page. If the typeblock\index{typeblock} is centered with respect to the physical page,
then the default centering suffices. If the typeblock\index{typeblock} is not physically
centered, then the titling information either has to be shifted 
horizontally or \cmd{\maketitle} has to be made to think that the typeblock\index{typeblock}
has been shifted horizontally. The simplest solution is to use the
\cmd{\calccentering} and \Ie{adjustwidth*} command and environment. For
example:
\begin{lcode}
\begin{titlingpage}
  \calccentering{\unitlength}
  \begin{adjustwidth*}{\unitlength}{-\unitlength}
    \maketitle
  \end{adjustwidth*}
\end{titlingpage}
\end{lcode}

\index{title page|)}

\begin{syntax}
\cmd{\title}\marg{text} \cmd{\thetitle} \\
\cmd{\author}\marg{text} \cmd{\theauthor} \\
\cmd{\date}\marg{text} \cmd{\thedate} \\
\end{syntax}
\glossary(title)
  {\cs{title}\marg{text}}%
  {Used by \cs{maketitle} to typeset \meta{text} as the document  title.}
\glossary(thetitle)
  {\cs{thetitle}}%
  {Copy of \meta{text} from \cs{title}.}
\glossary(author)
  {\cs{author}\marg{text}}%
  {Used by \cs{maketitle} to typeset \meta{text} as the document author.}
\glossary(theauthor)
  {\cs{theauthor}}%
  {Copy of \meta{text} from \cs{author}.}
\glossary(date)
  {\cs{date}\marg{text}}%
  {Used by \cs{maketitle} to typeset \meta{text} as the document date.}
\glossary(thedate)
  {\cs{thedate}}%
  {Copy of \meta{text} from \cs{date}.}

   In the usual document classes, the contents (\meta{text}) of
the \cmd{\title}, \cmd{\author} and \cmd{\date}
macros used for \cmd{\maketitle} are unavailable once \cmd{\maketitle} 
has been
issued. The class provides the \cmd{\thetitle},
\cmd{\theauthor} and \cmd{\thedate} commands that can be used for printing
these elements of the title later in the document, 
if desired. 

\begin{syntax}
\cmd{\and} \cmd{\andnext} \\
\end{syntax}
\glossary(and)%
  {\cs{and}}%
  {Use within the argument to \cs{author} to separate author's names.}
\glossary(andnext)%
  {\cs{andnext}}%
  {Use within the argument to \cs{author} to insert a newline..}
   The macro \cmd{\and} is used within the argument to the
\cmd{\author} command to add some extra space between the author's names.
The class \cmd{\andnext} macro inserts a newline instead of a space.
Within the \cmd{\theauthor} macro both \cmd{\and} and \cmd{\andnext}
are replaced by a comma.

   The class does not follow the standard classes' habit
of automatically killing the titling
commands after \cmd{\maketitle} has been issued. You can have multiple
\cmd{\title}, \cmd{\author}, \cmd{\date} and \cmd{\maketitle} 
commands in your
document if you wish. For example, some reports are issued with
a title page, followed by an executive summary, and then they
have another, possibly modified, title at the start of
the main body of the report. This can be accomplished like this:
\begin{lcode}
\title{Cover title}
...
\begin{titlingpage}
\maketitle
\end{titlingpage}
...
\title{Body title}
\maketitle
...
\end{lcode}

\begin{syntax}
\cmd{\killtitle} \cmd{\keepthetitle} \\
\cmd{\emptythanks} \\
\end{syntax}
\glossary(killtitle)%
  {\cs{killtitle}}%
  {Makes all aspects of \cs{maketitle} unavailable.}
\glossary(keepthetitle)%
  {\cs{keepthetitle}}%
  {Makes most aspects of \cs{maketitle} unavailable but keeps \cs{thetitle},
  \cs{theauthor} and \cs{thedate}.}
\glossary(emptythanks)%
  {\cs{emptythanks}}%
  {Discards any text from previous uses of \cs{thanks}.}
    The \cmd{\killtitle} macro makes all aspects of titling, including
\cmd{\thetitle} etc.,
unavailable from the point that it is issued (using this command will save
some macro space if the \cmd{\thetitle}, etc., commands are not required).
Using this command is the class's manual version
of the automatic killing performed by the standard classes.
The \cmd{\keepthetitle} command performs a similar function, except that
it keeps the \cmd{\thetitle}, \cmd{\theauthor} and \cmd{\thedate} commands,
 while killing everything else.

The \cmd{\emptythanks} command discards any text from prior use of 
\cmd{\thanks}.
This command is useful when \cmd{\maketitle} is used multiple times ---
the \cmd{\thanks} commands in each use just stack up the texts for output
at each use, so each subsequent use of \cmd{\maketitle} will output all 
previous \cmd{\thanks} texts together with any new ones. To avoid this,
put \cmd{\emptythanks} before each \cmd{\maketitle} after the first.

\index{title!styling|)}


\section{Styling the thanks} \label{sec:thanks}

\index{thanks}
\index{thanks!styling|(}

    The class provides a configurable \cmd{\thanks} command.

\begin{syntax}
\cmd{\thanksmarkseries}\marg{format} \\
\cmd{\symbolthanksmark} \\
\end{syntax}
\glossary(thanksmarkseries)%
  {\cs{thanksmarkseries}\marg{format}}%
  {Thanks marks will be printed using \meta{format} series of symbols.}
\glossary(symbolthanksmark)%
  {\cs{symbolthanksmark}}
  {Set the thanks marks to be printed using the footnote series of symbols.}
 Any \cmd{\thanks} are marked with symbols in the 
titling and footnotes\index{footnote}.
The command \cmd{\thanksmarkseries} 
can be used to change the marking style. The \meta{format} argument
is the name of one of the formats for printing a counter. The name 
is the same as that of a counter format but without the backslash.
To have the \cmd{\thanks} marks as lowercase letters instead of symbols 
do:
\begin{lcode}
\thanksmarkseries{alph}
\end{lcode}
Just for convenience the \cmd{\symbolthanksmark} command sets the series
to be footnote\index{footnote} symbols.
Using this class the potential names for \meta{format} are:
\texttt{arabic}, \texttt{roman}, \texttt{Roman}, \texttt{alph}, 
\texttt{Alph}, and \texttt{fnsymbol}. 

\begin{syntax}
\cmd{\continuousmarks} \\
\end{syntax}
\glossary(continuousmarks)%
  {\cs{continuousmarks}}%
  {Specifies that the thanks/footnote marker is not zeroed after titling.}
The \cmd{\thanks} command uses the \Icn{footnote} counter, 
and normally the counter
is zeroed after the titling so that the footnote marks\index{footnote!mark} start from 1.
If the counter should not be zeroed, then just specify 
\cmd{\continuousmarks}.
This might be required if numerals are used as the thanks markers.

\begin{syntax}
\cmd{\thanksheadextra}\marg{pre}\marg{post} \\
\end{syntax}
\glossary(thanksheadextra)%
  {\cs{thanksheadextra}\marg{pre}\marg{post}}%
  {Inserts \meta{pre} and \meta{post} before and after thanks markers
   in the titling code.}
The \cmd{\thanksheadextra} command will insert
\meta{pre} and \meta{post} before and after the thanks markers in the
titling block. By default \meta{pre} and \meta{post} are empty.
For example, to put parentheses round the titling markers do:
\begin{lcode}
\thanksheadextra{(}{)}
\end{lcode}


\begin{syntax}
\cmd{\thanksmark}\marg{n} \\
\end{syntax}
\glossary(thanksmark)%
  {\cs{thanksmark}\marg{n}}%
  {Prints a thanks mark identical to the n'th (previously) printed mark.}
It is sometimes desireable to have the same thanks text be applied to,
say, four out of six authors, these being the first 3 and the last one.
The command \cmd{\thanksmark}\marg{n} is similar to 
\cmd{\footnotemark}\oarg{n} in that it prints a thanks mark identical
to that of the \meta{n}'th  \cmd{\thanks} command. No changes are made
to any thanks in the footnotes\index{footnote}. For instance, in the following
the authors Alpha and Omega will have the same mark:
\begin{lcode}
\title{The work\thanks{Draft}}
\author{Alpha\thanks{ABC},
        Beta\thanks{XYZ} and 
        Omega\thanksmark{2}} 
\maketitle
\end{lcode}

%\begin{syntax}
%\cmd{\thanksgap}\marg{length} \\
%\end{syntax}
%The marks in the titling block printed by the 
%\cmd{\thanks} and \cmd{\thanksmark}
%commands take zero space. This is acceptable if they come at the end of
%a line, but not in the middle of a line. Use the 
%\cmd{\thanksgap} command immediately after a mid-line
%\cmd{\thanks} or \cmd{\thanksmark} command to add \meta{length} amount of
%space before the next word. For example, if there are three authors
%from two different institutions:
%\begin{lcode}
%\author{Alpha\thanks{ABC},
%        Omega\thanks{XYZ}\thanksgap{1ex} and 
%        Beta\thanksmark{1}} 
%\end{lcode}

\begin{syntax}
\cmd{\thanksmarkstyle}\marg{defn} \\
\end{syntax}
\glossary(thanksmarkstyle)%
  {\cs{thanksmarkstyle}\marg{defn}}%
  {Sets the style for the thanks marks at the foot.}
By default the thanks mark at the foot is typeset as a superscript. In
the class this is specifed via
\begin{lcode}
\thanksmarkstyle{\textsuperscript{#1}}
\end{lcode}
where \verb?#1? will be replaced by the thanks mark symbol. You can change
the mark styling
if you wish. For example, to put parentheses round the
mark and typeset it at normal size on the baseline:
\begin{lcode}
\thanksmarkstyle{(#1)}
\end{lcode}


\begin{syntax}
\lnc{\thanksmarkwidth} \\
\end{syntax}
\glossary(thanksmarkwidth)%
  {\cs{thanksmarkwidth}}%
  {Width of box for the thanks marks at the foot.}
 The thanks mark in the footnote\index{footnote} is typeset right justified in a box
of width \lnc{\thanksmarkwidth}. The first line of the thanks text starts
after this box. The initialisation is 
\begin{lcode}
\setlength{\thanksmarkwidth}{1.8em}
\end{lcode}
giving the default position.

\begin{syntax}
\lnc{\thanksmarksep} \\
\end{syntax}
\glossary(thanksmarksep)%
  {\cs{thanksmarksep}}%
  {Indentation of second and subsequent thanks text lines at the foot.}
The value of the length
 \lnc{\thanksmarksep} controls the indentation the
second and subsequent lines of the thanks text, with respect to
the end of the mark box. As examples: 
\begin{lcode}
\setlength{\thanksmarksep}{0em}
\end{lcode}
 will align the left hand ends of of a multiline thanks text, while: 
\begin{lcode}
\setlength{\thanksmarksep}{-\thanksmarkwidth}
\end{lcode}
will left justify any second and subsequent lines of the thanks text. 
This last
setting is the initialised value, giving the default appearance.

\begin{syntax}
\cmd{\thanksfootmark} \\
\end{syntax}
\glossary(thanksfootmark)%
  {\cs{thanksfootmark}}%
  {Typesets a thanks mark at the foot.}
    A thanks mark in the footnote\index{footnote} region is typeset by \cmd{\thanksfootmark}.
The code for this is roughly:
\LMnote{2012/07/02}{Text changed to reflect the actual code}
\begin{lcode}
\newcommand{\thanksfootmark}{%
  \hbox to\thanksmarkwidth{\hfil\normalfont%
     \thanksscript{\thefootnote}}}
\end{lcode}
You should not need to change the definition
of \cmd{\thanksfootmark} 
but you may want to change the default definitions of one or more
of the macros it uses.

\begin{syntax}
\cmd{\thanksscript}\marg{text} \\
\end{syntax}
\glossary(thanksfootmark)%
  {\cs{thanksfootmark}}%
  {Handle the inner part of the thanks mark at the foot.}
This is initially defined as: 
\begin{lcode}
\newcommand{\thanksscript}[1]{\textsuperscript{#1}}
\end{lcode}
so that \cmd{\thanksscript} typesets its argument as a superscript, which
is the default for thanks notes.



\begin{comment}

\begin{syntax}
\cmd{\thanksscript}\marg{text} \\
\end{syntax}
Note that the thanks mark, together with the \verb?\...pre? and \verb?\...post?
commands form the \meta{text} argument to the \cmd{\thanksscript} command. 
This is initially defined as: 
\begin{lcode}
\newcommand{\thanksscript}[1]{\textsuperscript{#1}}
\end{lcode}
so that \cmd{\thanksscript} typesets its argument as a superscript, which
is the default for thanks notes. If you would prefer the mark to be
set at the baseline instead, for example, just do: 
\begin{lcode}
\renewcommand{\thanksscript}[1]{#1}
\end{lcode}
 and if you also wanted very small symbols on the baseline you could do:
\begin{lcode}
\renewcommand{\thanksscript}[1]{\tiny #1}
\end{lcode}

Alternatively 
\begin{lcode}
\renewcommand{\thanksscript}[1]{#1}
\thanksfootextra{\bfseries}{.}
\end{lcode}
will give a bold baseline mark followed by a dot.

   Using combinations of these macros you can easily define
different layouts for the thanks footnotes\index{footnote}. Here are some
sample, but to shorten them I have ignored any
\cs{renewcommand}s and \cs{setlength}s, leaving these to be implied
as necessary.
\begin{itemize}
\item Setting \verb?\thanksfootextra{}{\hfill}? left justifies the mark
    in its box:
   \begin{itemize}
   \item \verb?\thanksscript{\llap{#1\space}}}, \verb?\thanksmarkwidth{0em}? and \\
         \verb?\thanksmargin{0em}? puts the baseline mark in the margin\index{margin}
         and the text left justified.
   \item Using \verb?\thanksscript{#1}?, \verb?\thanksmarkwidth{1em}? and \\
         \verb?\thanksmargin{-\thanksmarkwidth}? makes the baseline mark 
         and text left adjusted.
   \item \verb?\thanksscript{#1}?, \verb?\thanksmarkwidth{1em}? and
         \verb?\thanksmargin{0em}? puts the baseline mark left adjusted
         and the text indented and aligned, like this marked
         sentence is typeset.
   \end{itemize}

\item Setting \verb?\thanksfootextra{ }? and \verb?\thanksscript{#1}? 
      right justifies the baseline mark and a space in the mark box:
   \begin{itemize}
   \item The normal style is
         defined by \verb?\thanksmarkwidth{1.8em}? and \\
         \verb?\thanksmargin{-\thanksmarkwidth}? which put the mark 
         indented and the text left adjusted, like a normal indented
         paragraph\index{paragraph!indentation}.
   \item \verb?\thanksmarkwidth{1.8em}? and
         \verb?\thanksmargin{0em}? put the mark indented 
         and the text indented and aligned.
   \end{itemize}

\end{itemize}

%%%%%%%%%%%%%%%%%%%%%
\end{comment}
%%%%%%%%%%%%%%%%%%%%

\begin{syntax}
\cmd{\makethanksmark} \\
\cmd{\makethanksmarkhook} \\
\end{syntax}
\glossary(makethanksmark)%
  {\cs{makethanksmark}}%
  {Typesets the thanks marker and text at the foot.}
\glossary(makethanksmarkhook)
  {\cs{makethanksmarkhook}}
  {Code hook into \cs{makethanksmark}.}
The macro \cmd{\makethanksmark} typesets both the thanks marker (via
\cmd{\thanksfootmark}) and the thanks text. You probably will not need
to change its default definition. Just in case, though, 
\cmd{\makethanksmark}
calls the macro \cmd{\makethanksmarkhook} before it does any typesetting.
The default definition of this is: 
\begin{lcode}
\newcommand{\makethanksmarkhook}{}
\end{lcode}
which does nothing.

   You can redefine \cmd{\makethanksmarkhook} to do something useful. For
example, if you wanted a slightly bigger baseline skip you could do:
\begin{lcode}
\renewcommand{\makethanksmarkhook}{\fontsize{8}{11}\selectfont}
\end{lcode}
where the numbers \texttt{8} and \texttt{11} specify the point size of the font 
and the baseline skip
respectively. In this example 8pt is the normal \cmd{\footnotesize} in
a 10pt document, and 11pt is the baselineskip for \cmd{\footnotesize}
text in an 11pt document (the normal baseline skip for \cmd{\footnotesize}
in a 10pt document is 9.5pt);
adjust these numbers to suit.

\begin{syntax}
\cmd{\thanksrule} \\
\cmd{\usethanksrule} \\
\cmd{\cancelthanksrule} \\
\end{syntax}
\glossary(thanksrule)%
  {\cs{thanksrule}}%
  {The rule to be typeset before the thanks in the foot.}
\glossary(usethanksrule)%
  {\cs{usethanksrule}}%
  {Specifies that the \cs{thanksrule} is to be typeset in the 
   \texttt{titlingpage} environment.}
\glossary(cancelthanksrule)%
  {\cs{cancelthanksrule}}%
  {Specifies that the \cs{footnoterule} is to be used from now on.}
By default, there is no rule above \cmd{\thanks}
text that appears in a \Ie{titlingpage} environment.
If you want a rule in that environment, put \cmd{\usethanksrule} 
before the \cmd{\maketitle} command, which will then print a rule according
to the current definition of \cmd{\thanksrule}.
\cmd{\thanksrule} is initialised to be a copy of \cmd{\footnoterule} as it 
is defined at the
end of the preamble\index{preamble}. The definition of \cmd{\thanksrule} can be changed
after \verb?\begin{document}?. If the definition of \cmd{\thanksrule} is modified
and a \cmd{\usethanksrule} command has been issued, then the redefined
rule may also be used for footnotes\index{footnote}. Issuing the command 
\cmd{\cancelthanksrule} will cause the normal \cmd{\footnoterule} definition
to be used from thereon; another \cmd{\usethanksrule}
command can be issued later
if you want to swap back again.

The parameters for the vertical positioning of footnotes\index{footnote} 
and thanks notes, and the default \cmd{\footnoterule} are the same
(see \fref{fig:fn} on \pref{fig:fn}).
You will have to change one or more of these if the vertical spacings of 
footnotes\index{footnote}
and thanks notes are meant to be different.

\index{thanks!styling|)}

%#% extend
%#% extstart include abstracts.tex
