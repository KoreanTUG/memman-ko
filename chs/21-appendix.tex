\chapter{Packages and macros}

    The \Mname\ class does not provide for everything that you
have seen in the manual. I have used some packages that you are very likely
to have in your \ltx\ distribution, and have supplemented these with some
additional macros, some of which I will show you.

\section{Packages}

    The packages that I have used that you are likely to have, and if
you do  not have them please consider getting them, are:
\begin{itemize}
\item \Lpack{etex} lets you take advantage of e\tx's extended support
      for counters and such.

      Note that from 2015 and onwards, the allocation of extra
      registers have now been build into the LaTeX kernel. Thus in
      most cases the \Lpack{etex} package is no longer
      necessary. There are how ever extra very special features left
      in \Lpack{etex} that \emph{some} users may need. In that case
      please remember to load \Lpack{etex} by placing
      \verb|\RequirePackage{etex}| \emph{before} \cs{documentclass}! 
\item \Lpack{url}~\cite{URL} is for typesetting URL's without worrying
  about special characters or line breaking.
\item \Lpack{fixltx2e}~\cite{FIXLTX2E} eliminates some infelicities
      of the original LaTeX kernel. In particular it maintains the order
      of floats\index{float} on a twocolumn\index{column!double} page 
      and ensures the correct marking
      on a twocolumn\index{column!double} page.

      Note that as of 2015, the functionality of this package has been
      merged into the \LaTeX{} kernel. Loading this package does nothing.

      
\item \Lpack{alltt}~\cite{ALLTT} is a basic package which provides a 
      verbatim-like environment but \verb?\?, \verb?{?, and \verb?}? have their
      usual meanings (i.e., \ltx\ commands are not disabled).
\item \Lpack{graphicx}~\cite{GRAPHICX} is a required package for
      performing various kinds of graphical functions. 
\item \Lpack{color}~\cite{COLOR} is a required package for using color,
       or \Lpack{xcolor}~\cite{XCOLOR} is an enhanced version of \Lpack{color}.
\item \Lpack{latexsym} gives access to some extra symbols.
\item \Lpack{amsmath} for when you are doing anything except the
       simplest kind of maths typeseting.
\item \Lpack{fontenc} for using fonts with anything other than the
      original \texttt{OT1} encoding (i.e., for practically any font).
\item \Lpack{pifont} for typesetting Pifonts 
       (i.e., Symbol\facesubseeidx{Symbol} and 
               Zapf Dingbats\facesubseeidx{Zapf Dingbats})
\end{itemize}

    Apart from the packages that are supplied as part of the \Mname\ 
distribution, the packages that I used and you most likely do not have are:
\begin{itemize}
\item \Lpack{layouts}~\cite{LAYOUTS}. I used it for all the layout diagrams. 
For example, \fref{fig:displaysechead} and \fref{fig:runsechead} 
were drawn simply by:
\begin{lcode}
\begin{figure}
\centering
\setlayoutscale{1}
\drawparameterstrue
\drawheading{}
\caption{Displayed sectional headings} \label{fig:displaysechead}
\end{figure}

\begin{figure}
\centering
\setlayoutscale{1}
\drawparameterstrue
\runinheadtrue
\drawheading{}
\caption{Run-in sectional headings} \label{fig:runsechead}
\end{figure}
\end{lcode}
The package also lets you try experimenting with different layout 
parameters and draw diagrams showing what the results would be in a document.

    The version of \Lpack{layouts} used for this manual is 
v2.4 dated 2001/04/30. Earlier versions will fail when attempting
to draw some figures 
( e.g., to draw \fref{fig:oddstock}).

\item \Lpack{fonttable}~\cite{FONTTABLE}. I used this for the font tables 
(e.g., \tref{tab:symbolglyphs}). 
You must have at least version 1.3 dated April 2009
for processing the manual (earlier versions are likely to produce errors
in the number formatting area with minor, but odd looking, effect on 
the printed result).

\end{itemize}


\section{Macros}

    Originally the preamble\index{preamble} of the manual contained many macro 
definitions, probably more than most documents would because:
\begin{itemize}
\item I am having to typeset many \ltx\ commands, which require
      some sort of special processing;
\item I have tried to minimize the number of external packages needed
      to \ltx\ this manual satisfactorily, and so have copied various
      macros from elsewhere;
\item I wanted to do some automatic indexing\index{index};
\item I wanted to set off the syntax specifications and the code examples
      from the main text.
\end{itemize}
I have since put the majority of these into a package file called 
\file{memsty.sty}. 
To get the whole glory you will have to read the preamble\index{preamble}, 
and the \Lpack{memsty} package file but I show a few of the macros below 
as they may be of more general interest.

\begin{syntax}
\cmd{\Ppstyle}\marg{pagestyle} \cmd{\pstyle}\marg{pagestyle} \\
\end{syntax}
The command \cmd{\Ppstyle} prints its argument in the font used to indicate
pagestyles and the command \cmd{\pstyle} prints its pagestyle argument and also
makes a pagestyle entry in the index\index{index}. Its definition is
\begin{lcode}
\newcommand*{\pstyle}[1]{\Ppstyle{#1}%
  \index{#1 pages?\Ppstyle{#1} (pagestyle)}%
  \index{pagestyle!#1?\Ppstyle{#1}}}
\end{lcode}
The first part prints the argument in the text and the second adds two
entries to the \file{idx} file. The fragment \verb?#1 pages? is what 
the \Lmakeindex\ program will use for sorting entries, and the 
fragment following the \texttt{?} character is what will be put into the 
index\index{index}.

\begin{syntax}
\cmd{\Pcstyle}\marg{chapterstyle} \cmd{\cstyle}\marg{chapterstyle} \\
\end{syntax}
The command \cmd{\Pcstyle} prints its argument in the font used to indicate
chapterstyles and \cmd{\cstyle} prints its chapterstyle argument 
and also makes a chapterstyle entry in the 
index\index{index}. Its definition is
\begin{lcode}
\newcommand*{\cstyle}[1]{\Pcstyle{#1}%
  \index{#1 chaps?\Pcstyle{#1} (chapterstyle)}%
  \index{chapterstyle!#1?\Pcstyle{#1}}}
\end{lcode}
which is almost identical to \cmd{\pstyle}. 

    There is both a \cstyle{companion} chapterstyle and a \pstyle{companion}
pagestyle. The strings used for sorting the index\index{index} entries for 
these are
\texttt{companion chaps} and \texttt{companion pages} respectively, so 
the chapterstyle will come before the pagestyle in the index\index{index}. 
The reason for distinguishing between the string used for sorting and the 
actual entry is
partly to distinguish between different kinds of entries for a single name
and partly to avoid any formatting commands messing up the sorted order.

\begin{syntax}
\senv{syntax} syntax \eenv{syntax} \\
\end{syntax}
The \Ie{syntax} environment is for specifying command and environment
syntax. Its definition is
\begin{lcode}
\newcommand*{\tightcenter}{%
  \topsep=0.25\onelineskip\trivlist \centering\item\relax}
\def\endtigthcenter{\endtrivlist}
\newenvironment{syntax}{\begin{tightcenter}
                        \begin{tabular}{|p{0.9\linewidth}|} \hline}%
                       {\hline
                        \end{tabular}
                        \end{tightcenter}}
\end{lcode}
It is implemented in terms of the \Ie{tabular} environment, centered within
the typeblock, which forms
a box that will not be broken across a pagebreak. The box frame
is just the normal horizontal and vertical lines that you can use with
a \Ie{tabular}. The width is fixed at 90\% of the text width. As it
is a \Ie{tabular} environment, each line of syntax must be ended with
\cmd{\\}. Note that normal \ltx\ processing occurs within the \Ie{syntax}
environment, so you can effectively put what you like inside it.
The \Ie{center} environment is defined in terms of a \Ie{trivlist} and
\cmd{\centering}. I wanted to be able to control the space before and
after the `\cmd{\centering}' so I defined the \Ie{tightcenter} environment 
which enabled me to do this.


\begin{syntax}
\senv{lcode} LaTeX code \eenv{lcode} \\
\end{syntax}
I use the \Ie{lcode} environment for showing examples of \ltx\ code. It
is a special kind of \Ie{verbatim} environment where the font size is
\cmd{\small} but the normal \lnc{\baselineskip} is used, and each line
is indented. 

    At the bottom the environment is defined in terms of a \Ie{list}, 
although that is not obvious from the code; for details see the 
class code~\cite{MEMCODE}. I wanted the environment to be a tight list 
and started off by defining two helper items.
\begin{lcode}
% \@zeroseps sets list before/after skips to minimum values
\newcommand*{\@zeroseps}{\setlength{\topsep}{\z@}
                         \setlength{\partopsep}{\z@}
                         \setlength{\parskip}{\z@}}
% \gparindent is relative to the \parindent for the body text
\newlength{\gparindent} \setlength{\gparindent}{0.5\parindent}
\end{lcode}
The macro \cmd{\@zeroseps} sets the before, after and middle skips in
a list to 0pt (\cmd{\z@} is shorthand for 0pt). The length \lnc{\gparindent}
will be the line indentation in the environment.
\begin{lcode}
% Now we can do the new lcode verbatim environment. 
% This has no extra before/after spacing.
\newenvironment{lcode}{\@zeroseps
  \renewcommand{\verbatim@startline}%
    {\verbatim@line{\hskip\gparindent}}
  \small\setlength{\baselineskip}{\onelineskip}\verbatim}%
  {\endverbatim
   \vspace{-\baselineskip}\noindent}
\end{lcode}

    The fragment \verb?{\hskip\gparindent}? puts \lnc{\gparindent} space at 
the start of each line.

    The fragment \verb?\small\setlength{\baselineskip}{\onelineskip}? sets the
font size to be \cmd{\small}, which has a smaller \lnc{\baselineskip}
than the normal font, but this is corrected for by changing the local
\lnc{\baselineskip} to the normal skip, \lnc{\onelineskip}. At the end
of the environment there is a negative space of one line to compensate
for a one line space that LaTeX inserts.

%%% \Sref{sec:versal} is in the design manual
\begin{comment}
    The two versals in \Sref{sec:versal} were typeset with macros defined
in \Lpack{memsty}. The poorer of the two used the \cmd{\drop}
macro which was written for \ltx\ v2.09 by David Cantor and Dominik Wujastyk 
in 1998. The better used the \cmd{\versal} macro. Now, if you want
to try your hand at this sort of thing there are some more packages
on CTAN. I have found that the \Lpack{lettrine} package~\cite{LETTRINE} 
serves my needs.
\end{comment}


%#% extend
%#% extstart include showcases.tex

\svnidlong
{$Ignore: $}
{$LastChangedDate: 2010-05-13 17:10:00 +0200 (Thu, 13 May 2010) $}
{$LastChangedRevision: 210 $}
{$LastChangedBy: daleif $}

\chapter{Showcases}
\label{cha:showcases}

The \theclass\ memoir class has several features that involve a
\emph{style} and it provide several of these styles. This chapter is
used to showcase these styles.

\begingroup
% because of hyperref warnings


\section{Chapter styles}
\label{sec:chapter-styles}

\index{chapterstyle|(}

For more about defining chapter styles, see
section~\ref{sec:chapter-headings}, page~\pageref{sec:chapter-headings}.

\begin{demochap}[-3\onelineskip]{default}\end{demochap}

\begin{demochap}[2\onelineskip]{section}\end{demochap}

\begin{demochap}[2\onelineskip]{hangnum}\end{demochap}

\begin{demochap}[2\onelineskip]{companion}\end{demochap}

\begin{demochap}[-1\onelineskip]{article}\end{demochap}

\begin{demochap}[-2\onelineskip]{bianchi}\end{demochap}

\begin{demochap}{bringhurst}\end{demochap}

\begin{demochap}[-2\onelineskip]{brotherton}\end{demochap}

\begin{demochap}{chappell}\end{demochap}

\begin{demochap}[-2\onelineskip]{crosshead}\end{demochap}

\begin{demochap}[-\onelineskip]{culver}\end{demochap}

\begin{demochap}[-4\onelineskip]{dash}\end{demochap}

\begin{demochap}[-\onelineskip]{demo2}\end{demochap}

\begin{demochap}[-2\onelineskip]{dowding}\end{demochap}

\begin{demochap}{ell}\end{demochap}

\begin{demochap}[-4\onelineskip]{ger}\end{demochap}

\begin{demochap}[-2\onelineskip]{komalike}\end{demochap}

\begin{demochap*}[-2\onelineskip]{lyhne}\end{demochap*}

\clearpage

\begin{demochap*}[-2\onelineskip]{madsen}\end{demochap*}

\begin{demochap}[-3\onelineskip]{ntglike}\end{demochap}

\begin{demochap}[-\onelineskip]{southall}\end{demochap}

\begin{demochap}[-1\onelineskip]{tandh}\end{demochap}

\begin{demochap}{thatcher}\end{demochap}

\begin{demochap*}[-2\onelineskip]{veelo}\end{demochap*}

\begin{demochap}{verville}\end{demochap}

\FloatBlock


\begin{demochap}[-1\onelineskip]{wilsondob}\end{demochap}


The code for some of these styles is given in below.  For details of
how the other chapter styles are defined, look at the documented class
code. This should give you ideas if you want to define your own style.

Note that it is not necessary to define a new chapterstyle if you want
to change the chapter headings --- you can just change the individual
macros without putting them into a style.


%%%%%%%%%%%%%%%%%%%%%%%%%%%%%%%%%%%%%%%%%%%%%%%%%%%%%%%%%%%%%%%%%%%


\subsection{Chappell}

    A style that includes rules is one that I based on the chapter heads
in~\cite{CHAPPELL99} and which I have called \cstyle{chappell} after the
first author. The style, which is shown in \fref{dcchappell}, can easily form
the basis for general heads in non-technical books.
\begin{lcode}
\makechapterstyle{chappell}{%
  \setlength{\beforechapskip}{0pt}
  \renewcommand*{\chapnamefont}{\large\centering}
  \renewcommand*{\chapnumfont}{\large}
  \renewcommand*{\printchapternonum}{%
    \vphantom{\printchaptername}%
    \vphantom{\chapnumfont 1}%
    \afterchapternum
    \vskip -\onelineskip}
  \renewcommand*{\chaptitlefont}{\Large\itshape}
  \renewcommand*{\printchaptertitle}[1]{%
    \hrule\vskip\onelineskip \centering\chaptitlefont ##1}}
\end{lcode}


The style centers the chapter number, draws a rule across the page under
it, and below that comes the title, again centered. All the fiddling
in the \cs{printchapternonum} macro is to try and ensure that the rule
above the title is at the same height whether or not the chapter is numbered
(the ToC being an example of an unnumbered heading).

\subsection{Demo, Demo2 and demo3}

    I created a \cstyle{demo} chapterstyle quite a time ago and used it 
on occasions in earlier editions of this Manual. Here is the original code.
\begin{lcode}
\makechapterstyle{demo}{%
  \renewcommand*{\printchaptername}{\centering}
  \renewcommand*{\printchapternum}{\chapnumfont \numtoName{\c@chapter}}
  \renewcommand*{\chaptitlefont}{\normalfont\Huge\sffamily}
  \renewcommand*{\printchaptertitle}[1]{%
    \hrule\vskip\onelineskip \raggedleft \chaptitlefont ##1}
  \renewcommand*{\afterchaptertitle}%
                {\vskip\onelineskip \hrule\vskip \afterchapskip}
}% end demo
\end{lcode}

This has one serious failing and what I now believe is a poor design 
decision. The failing is that if you have any appendices that use the
\cstyle{demo} chapterstyle then they are numbered instead of being lettered.
The poor design is that the position of the title with respect to the top
of the page is not the same for numbered and unnumbered chapters.
    The \cstyle{demo2} chapterstyle below fixes both of these at the expense
of simplicity (at least for me).
\begin{lcode}
\makechapterstyle{demo2}{%
  \renewcommand*{\printchaptername}{\centering}
  \renewcommand*{\printchapternum}{\chapnumfont 
     \ifanappendix \thechapter \else \numtoName{\c@chapter}\fi}
  \renewcommand*{\chaptitlefont}{\normalfont\Huge\sffamily}
  \renewcommand*{\printchaptertitle}[1]{%
    \hrule\vskip\onelineskip \raggedleft \chaptitlefont ##1}
  \renewcommand*{\afterchaptertitle}{%
    \vskip\onelineskip \hrule\vskip \afterchapskip}
  \setlength{\beforechapskip}{3\baselineskip}
  \renewcommand*{\printchapternonum}{%
    \vphantom{\chapnumfont One}
    \afterchapternum%
    \vskip\topskip}
  \setlength{\beforechapskip}{2\onelineskip}
}% end{demo2}
\end{lcode}
    You may find it instructive to compare the code for the \cstyle{demo} and 
\cstyle{demo2} chapterstyles.

    Thec \cstyle{demo} chapterstyle is still available in the class for
backward compatibility reasons, but I strongly advise against anyone using it.

    By chance I inadvertantly typest a chapterstyle that was a mixture
of the \cstyle{pedersen} and \cstyle{demo2} styles. As a result there is
now a \cstyle{demo3} chapterstyle as well. The only difference between the
two styles is in the definition of \cs{chapnumfont} which in \cstyle{demo3}
is:
\begin{lcode}
\renewcommand*{\chapnumfont}{\normalfont\HUGE\itshape}
\end{lcode}

\subsection{Pedersen}

    I have modified Troels Pedersen's original code to make it a little more
efficient and flexible. 

\begin{lcode}
\newcommand*{\colorchapnum}{}
\newcommand*{\colorchaptitle}{}
\makechapterstyle{pedersen}{%
  \setlength{\beforechapskip}{-20pt}
  \setlength{\afterchapskip}{10pt}
  \renewcommand*{\chapnamefont}{\normalfont\LARGE\itshape}
  \renewcommand*{\chapnumfont}{\normalfont\HUGE\itshape\colorchapnum}
  \renewcommand*{\chaptitlefont}{\normalfont\huge\itshape\colorchaptitle}
  \renewcommand*{\afterchapternum}{}
  \renewcommand*{\printchaptername}{}
  \setlength{\midchapskip}{20mm}
  \renewcommand*{\chapternamenum}{}
  \renewcommand*{\printchapternum}{%
    \sidebar{\raisebox{0pt}[0pt][0pt]{\makebox[0pt][l]{%
      \resizebox{!}{\midchapskip}{\chapnumfont\thechapter}}}}}
  \renewcommand*{\printchaptertitle}[1]{\chaptitlefont ##1}
}
\end{lcode}
The chapter number is scaled up from its normal size and set in a 
sidebar\index{sidebar}.

\begin{syntax}
\cmd{\colorchapnum} \cmd{\colorchaptitle} \\
\end{syntax}
\glossary(colorchapnum)
  {\cs{colorchapnum}}%
  {Color for the number in the \Pcstyle{pedersen} chapterstyle.}
\glossary(colorchaptitle)
  {\cs{colorchaptitle}}%
  {Color for the title in the \Pcstyle{pedersen} chapterstyle.}
  The title is set with \cmd{colorchaptitle} and the number with
\cmd{colorchapnum}, both of which default to doing nothing. 
Lars Madsen\index{Madsen, Lars}
has suggested an attractive red color for these:
\begin{lcode}
\usepackage{color}
\definecolor{ared}{rgb}{.647,.129,.149}
\renewcommand{\colorchapnum}{\color{ared}}
\renewcommand{\colorchaptitle}{\color{ared}}
\chapterstyle{pedersen}
\end{lcode}

    The uncolored version is used for the chaptersyle for this chapter;
because of setting the number in a sidebar\index{sidebar} it does not
display well anywhere other than as a real chapter head.

\subsection{Southall}

    On 2006/01/08 Thomas Dye\index{Dye, Thomas} posted his \cstyle{southall} 
chapterstyle on \url{comp.text.tex} and kindly gave me permission to 
include it here. It is based on the headings in a Cambridge Press 
book\footnote{Which I haven't seen} by Aidan Southall.
It produces a simple numbered heading with the title set as a block paragraph,
and with a horizontal rule underneath. His original code called for lining
figures for the number but I have commented out that bit. I also changed
the code to eliminate the need for the two new lengths that Thomas used.

\begin{lcode}
\makechapterstyle{southall}{%
  \setlength{\afterchapskip}{5\baselineskip}
  \setlength{\beforechapskip}{36pt}
  \setlength{\midchapskip}{\textwidth}
  \addtolength{\midchapskip}{-\beforechapskip}
  \renewcommand*{\chapterheadstart}{\vspace*{2\baselineskip}}
  \renewcommand*{\chaptitlefont}{\huge\rmfamily\raggedright}
  \renewcommand*{\chapnumfont}{\chaptitlefont}
  \renewcommand*{\printchaptername}{}
  \renewcommand*{\chapternamenum}{}
  \renewcommand*{\afterchapternum}{}
  \renewcommand*{\printchapternum}{%
    \begin{minipage}[t][\baselineskip][b]{\beforechapskip}
      {\vspace{0pt}\chapnumfont%%%\figureversion{lining} 
                   \thechapter}
    \end{minipage}}
  \renewcommand*{\printchaptertitle}[1]{%
    \hfill\begin{minipage}[t]{\midchapskip}
      {\vspace{0pt}\chaptitlefont ##1\par}\end{minipage}}
  \renewcommand*{\afterchaptertitle}{%
    \par\vspace{\baselineskip}%
    \hrulefill \par\nobreak\noindent \vskip\afterchapskip}}
\end{lcode}

The resulting style is shown in \fref{dcsouthall}.


\subsection{Veelo}

    Bastiaan Veelo 
posted the code for a new chapter style to \pixctt\ on 2003/07/22 under the
title \textit{[memoir] [contrib] New chapter style}. His code, which
I have slightly modified and changed the name to \cstyle{veelo},
is below. I have also exercised editorial privilege on his comments.

\begin{quote}
 I thought I'd share a new chapter style to be used with the memoir class. 
 The style is tailored for documents that are to be trimmed to a smaller 
 width. When the bound document is bent, black tabs will appear on the 
 fore side at the places where new chapters start as a navigational aid.
 We are scaling the chapter number, which most DVI viewers
 will not display accurately. \\[0.5\onelineskip]
Bastiaan.
\end{quote}

    In the style as I modified it, \lnc{\beforechapskip} is used as the 
height of the number and \lnc{\midchapskip} is used as the length of the
black bar.
\begin{lcode}
\newlength{\numberheight}
\newlength{\barlength}
\makechapterstyle{veelo}{%
  \setlength{\afterchapskip}{40pt}
  \renewcommand*{\chapterheadstart}{\vspace*{40pt}}
  \renewcommand*{\afterchapternum}{\par\nobreak\vskip 25pt}
  \renewcommand*{\chapnamefont}{\normalfont\LARGE\flushright}
  \renewcommand*{\chapnumfont}{\normalfont\HUGE}
  \renewcommand*{\chaptitlefont}{\normalfont\HUGE\bfseries\flushright}
  \renewcommand*{\printchaptername}{%
    \chapnamefont\MakeUppercase{\@chapapp}}
  \renewcommand*{\chapternamenum}{}
  \setlength{\beforechapskip}{18mm}
  \setlength{\midchapskip}{\paperwidth}
  \addtolength{\midchapskip}{-\textwidth}
  \addtolength{\midchapskip}{-\spinemargin}
  \renewcommand*{\printchapternum}{%
    \makebox[0pt][l]{\hspace{.8em}%
      \resizebox{!}{\numberheight}{\chapnumfont \thechapter}%
      \hspace{.8em}%
      \rule{\midchapskip}{\beforechapskip}%
     }}%
   \makeoddfoot{plain}{}{}{\thepage}}
\end{lcode}


    If you use this style you will also need to use the \Lpack{graphicx} 
package~\cite{GRAPHICX} because of the \cmd{\resizebox} macro. 
The \cstyle{veelo} style works best for chapters that start 
on recto pages.

\index{chapterstyle|)}

\endgroup


%#% extend
%#% extstart include sniplets.tex


\svnidlong
{$Ignore: $}
{$LastChangedDate: 2015-03-05 18:49:59 +0100 (Thu, 05 Mar 2015) $}
{$LastChangedRevision: 516 $}
{$LastChangedBy: daleif $}


\defaultlists


\LMnote{2010/02/09}{Started the sniplet chapter}
\chapter{Sniplets}
\label{cha:sniplets}

\cftinserthook{toc}{start-sniplets}


This chapter is (over time) meant to hold various pieces of code for
\theclass\ that we have gathered over the years or others have
contributed, and which we think might be useful for others.  In some
cases they will have been moved from the text to this place, in order
to make the manual less cluttered.


If you have some \theclass\ related code you would like to share, feel
free to send it to \verb?daleif@math.au.dk?.



\sniplettoc


\begin{sniplet}[Mirroring the output]\label{snip:1}
  The \theclass\ class is not quite compatible with the \Lpack{crop}
  package. This is usually not a problem as we provide our own
  crop marks. But \Lpack{crop} provide one feature that we do not:
  mirroring of the output. The following sniplet was posted on
  \textsc{ctt} by Heiko Oberdiek (2009/12/05, thread \textit{
    Memoir and mirrored pdf output })
  \begin{lcode}
    \usepackage{atbegshi} 
    \usepackage{graphicx}
    \AtBeginShipout{%
      \sbox\AtBeginShipoutBox{%
        \kern-1in\relax
        \reflectbox{%
          \rlap{\kern1in\copy\AtBeginShipoutBox}%
          \kern\stockwidth
        }%
      }%
    } 
  \end{lcode}
\end{sniplet}

\begin{sniplet}[Remove pagenumber if only one page]\label{snip:2}
  Memoir counts all the pages used. You can use this information
  in various ways. For example, say you are preparing a setup to write
  small assignments in, these may or may not be just one page. How do
  we remove the footer automatically if there is only one page?

  Easy, place the following in the preamble (compile at least twice):
  \begin{lcode}
    \AtEndDocument{\ifnum\value{lastsheet}=1\thispagestyle{empty}\fi}
  \end{lcode}
\end{sniplet}

\begin{sniplet}[A kind of draft note]\label{snip:veelo}
  Bastiaan Veelo\index{Veelo, Bastiaan} has kindly provided example code 
  for another form of a side note, as follows.
\begin{lcode}
%% A new command that allows you to note down ideas or annotations in
%% the margin of the draft. If you are printing on a stock that is wider
%% than the final page width, we will go to some length to utilise the
%% paper that would otherwise be trimmed away, assuming you will not be
%% trimming the draft. These notes will not be printed when we are not
%% in draft mode.
\makeatletter
   \ifdraftdoc
     \newlength{\draftnotewidth}
     \newlength{\draftnotesignwidth}
     \newcommand{\draftnote}[1]{\@bsphack%
       {%% do not interfere with settings for other marginal notes
         \strictpagecheck%
         \checkoddpage%
         \setlength{\draftnotewidth}{\foremargin}%
         \addtolength{\draftnotewidth}{\trimedge}%
         \addtolength{\draftnotewidth}{-3\marginparsep}%
         \ifoddpage
           \setlength{\marginparwidth}{\draftnotewidth}%
           \marginpar{\flushleft\textbf{\textit{\HUGE !\ }}\small #1}%
         \else
           \settowidth{\draftnotesignwidth}{\textbf{\textit{\HUGE\ !}}}%
           \addtolength{\draftnotewidth}{-\draftnotesignwidth}%
           \marginpar{\raggedleft\makebox[0pt][r]{%% hack around
               \parbox[t]{\draftnotewidth}{%%%%%%%%% funny behaviour
                 \raggedleft\small\hspace{0pt}#1%
               }}\textbf{\textit{\HUGE\ !}}%
           }%
         \fi
       }\@esphack}
   \else
     \newcommand{\draftnote}[1]{\@bsphack\@esphack}
   \fi
\makeatother
\end{lcode}
Bastiaan also noted that it provided an example of using the
\lnc{\foremargin} length.  If you want to try it out, either put the
code in your preamble, or put it into a package (i.e., \file{.sty}
file) without the \cs{makeat...} commands.
\end{sniplet}


\LMnote{2010/03/12}{Added this sniplet inspired by a thread on ctt}
\begin{sniplet}[Adding indentation to footnotes]

At times a document design calls for a footnote configuration equal to
the default but everything indented more to the right. This can be
achieved via
\begin{lcode}
  \newlength\myextrafootnoteindent
  \setlength\myextrafootnoteindent{\parindent}
  \renewcommand\makefootmarkhook{%
    \addtolength{\leftskip}{\myextrafootnoteindent}} 
\end{lcode}
In this example we indent the footnotes to match the overall paragraph
indentation. We need to save the current value of \verb?\parindent?
since it is reset in the footnotes.
  
\end{sniplet}


\LMnote{2010/03/12}{Added this sniplet inspired by a thread on ctt}
\begin{sniplet}[Background image and trimmarks]
This sniplet comes from another problem described in \textsc{ctt}. If
one use the \Lpack{eso-pic} package to add a background image, this
image ends up on top of the trim marks. To get it \emph{under} the
trim marks Rolf Niepraschk suggested the following trick
\begin{lcode}
\RequirePackage{atbegshi}\AtBeginShipoutInit
\documentclass[...,showtrims]{memoir}
...
\usepackage{eso-pic}
...
\end{lcode}
  
\end{sniplet}


\LMnote{2012/09/21}{Sniplet showcasing autoadjusting numwidths in the toc}
\begin{sniplet}[Autoadjusted number widths in the ToC]
  \label{snip:autotoc}

  When the \toc{} is typeset the chapter, section etc. number is
  typeset within a box of a certain fixed with (one width for each
  sectional type).  If this width is too small for the current
  document, the user have to manually adjust this width.

  In this sniplet we present a method where we automatically record
  the widest.

  It a later \theclass\ version, we may add similar code to the core.

  There are two ways to record the widest entries of the various
  types, either preprocess the entire \toc{} or measure it as a part
  of the \toc{} typesetting, store it in the \texttt{.aux} and reuse
  it on the next run. We will use the later approach. There is one
  caveat: The \texttt{.aux} file is read at \verb|\begin{document}|,
    so we need to postpone our adjustments via \verb|\AtBeginDocument|.

  The following solution use some \toc{} related hooks within the
  class, plus the \Lpack{etoolbox} and \Lpack{calc} packages.

  First we create some macros to store information within the
  \texttt{.aux} file, and retrieve it again.

  \begin{lcode}
    \makeatletter
    \newcommand\mem@auxrestore[2]{\csgdef{stored@value@#1}{#2}}
    \newcommand\memstorevalue[2]{%
      \@bsphack%
      \immediate\write\@mainaux{\string\mem@auxrestore{#1}{#2}}%
      \@esphack}
    \newcommand\RetrieveStoredLength[1]{%
      \ifcsdef{stored@value@#1}{\csuse{stored@value@#1}}{0pt}}%
    \makeatletter
  \end{lcode}
  Here \cs{RetrieveStoredLength} can be used in most \cmd{\setlength}
  cases, at least when the \Lpack{calc} package is loaded. The
  argument will be the name of the variable one asked to be stored. If
  no corresponding value has been found for a given name, 0\,pt is returned.

  Next we need to prepare the hooks. In this case we will show how to
  take care of \cmd{\chapter}, \cmd{\section} and \cmd{\subsection}.
  \cmd{\chapter} is relatively easy:\footnote{In some cases you may
    want to use \texttt{%
      \{%
      \cs{@chapapp@head}%
      \cs{@cftbsnum} 
      \#1%
      \cs{@cftasnum}%
      \}%
    }}
  \begin{lcode}
    \newlength\tmplen        % scratch length
    \newlength\widestchapter % guess, they are zero by default
    \renewcommand\chapternumberlinehook[1]{%
      \settowidth\tmplen{\hbox{\cftchapterfont#1}}%
      \ifdimgreater\tmplen\widestchapter{%
        \global\widestchapter=\tmplen}{}}
  \end{lcode}
  We use an alternative syntax to make the \cs{widestchapter} global.

  Handling \cmd{\section} and \cmd{\subsection} is slightly more
  tricky, as they both use \cmd{numberline}. Instead we rely on the
  local value of the magic macro \cmd{\cftwhatismyname}. 
  \begin{lcode}
    \newlength\widestsection
    \newlength\widestsubsection
    \renewcommand\numberlinehook[1]{%
      % use a loop handler to loop over a list of possible
      % types. \forcsvlist comes from etoolbox
      \forcsvlist{\ToCHookListHandler{#1}}{section,subsection,subsubsection,%
        paragraph,subparagraph,figure,table}}
    % the actual handler.
    \newcommand\ToCHookListHandler[2]{%
      \edef\tmpstr{#2}%
      \ifdefstrequal{\cftwhatismyname}{\tmpstr}{%
        \settowidth\tmplen{\hbox{\csuse{cft\cftwhatismyname font}#1}}%
        \ifcslength{widest#2}{% is this length defined?
          \ifdimgreater\tmplen{\csuse{widest#2}}{%
            \global\csuse{widest#2}=\tmplen}{}}{}}{}}
  \end{lcode}
  Even though the list mention more macros, we only use those we have
  added corresponding lengths for.

  Next we need to store the values at the end of the document
  \begin{lcode}
    \AtBeginDocument{\AtEndDocument{
        \memstorevalue{widestchapter}{\the\widestchapter}
        \memstorevalue{widestsection}{\the\widestsection}
        \memstorevalue{widestsubsection}{\the\widestsubsection}
      }}
  \end{lcode}

  Here is how to get the standard class setup for a three level
  TOC. We also add a little extra padding to the boxes. Remember that
  it may take a few compilations before the \toc{} settles down.
  \begin{lcode}
    \newlength\cftnumpad         % padding
    \setlength\cftnumpad{0.5em}
    \AtBeginDocument{
      \cftsetindents{chapter}{0pt}{%
        \RetrieveStoredLength{widestchapter}+\cftnumpad}
      \cftsetindents{section}{%
        \cftchapterindent+\cftchapternumwidth}{%
        \RetrieveStoredLength{widestsection}+\cftnumpad}
      \cftsetindents{subsection}{%
        \cftsectionindent+\cftsectionnumwidth}{%
        \RetrieveStoredLength{widestsubsection}+\cftnumpad}
    }
  \end{lcode}
\end{sniplet}


\begin{sniplet}[Using class tools to make a chapter ToC]
  \label{snip:chaptertoc}

  By using a few hooks, we will be able to create a simple chapter
  toc. First a few notes:
  \begin{enumerate}[(a)]
    \setlength\itemsep{0.5em}
  \item In this class, the TOC data can be reused, thus we can load
    the TOC data as many times as we would like.
  \item Data in the TOC is stored as arguments the \cs{contentsline}
    macro, say (see also \fref{fig:tocloflotfiles} on
    page~\pageref{fig:tocloflotfiles})
\begin{verbatim}
\contentsline{chapter}{\chapternumberline {1}Test}{3}
\end{verbatim}
   where the first argument determins which macro is used to process
   the data. Each of these macros look at the value of the \Pcn{tocdepth}
   counter to know whether to typeset or not. 
 \item Using some hooks we can insert local changes to \Pcn{tocdepth}
   in order to only typeset the sections from the current chapter.
 \end{enumerate}
 
The idea is to be able to add hooks at key points in the \toc{} data,
and then use these hooks to enable and disable typesetting.

We will need to add hooks just after a chapter line (like the one
above), and we will need to be able to insert hooks just before items
that mark the end of a chapter, that is the next \cmd{\chapter},
\cmd{\part}, \cmd{\book}, plus a macro like \cmd{\appendixpage} which
also write to the \toc{}.





 
  First we define hooks that add hooks into the TOC. We use a counter
  to make each start and end hook unique. We add \emph{end markers}
  above the \toc{} entries for \cs{chapter}, \cs{part} and
  \cs{book}.
\begin{lcode}
\newcounter{tocmarker}
\renewcommand\mempreaddchaptertotochook{\cftinserthook{toc}{end-\thetocmarker}}
\renewcommand\mempreaddparttotochook   {\cftinserthook{toc}{end-\thetocmarker}}
\renewcommand\mempreaddbooktotochook   {\cftinserthook{toc}{end-\thetocmarker}}
\renewcommand\mempreaddapppagetotochook{\cftinserthook{toc}{end-\thetocmarker}}
% start marker
\renewcommand\mempostaddchaptertotochook{%
  \stepcounter{tocmarker}\cftinserthook{toc}{start-\thetocmarker}}
\let\normalchangetocdepth\changetocdepth % for later
\end{lcode}
The hooks inserted into the TOC file, does nothing by default. You
will notice that the line above will now look like:
\begin{verbatim}
\cftinsert {end-0}
\contentsline{chapter}{\chapternumberline {1}Test}{3}
\cftinsert {start-1}
...
\cftinsert {end-1}
\contentsline{chapter}{\chapternumberline {2}Test}{5}
\end{verbatim}
Thus to get a chapter toc command we need to make sure that (1) all
entries are disabled, (2) at \texttt{start-1} we reenable TOC entries,
and (3) at \texttt{end-1} disable TOC entries again. Here is the rest
of the code, explained via comments.
\begin{lcode}
\makeatletter
\newcommand\chaptertoc{
  % make changes local, remember counters a global
  \begingroup
  % store current value, to be restored later
  \setcounter{@memmarkcntra}{\value{tocdepth}}
  % when ever \settocdepth is used, it adds the new value to the 
  % ToC data. This cause problems when we want to disable all
  % entries. Luckily the data is added via a special macro, we we
  % redefine it, remember we stored the original value earlier.
  \let\changetocdepth\@gobble
  % disable all entries (using our copy from above)
  \normalchangetocdepth{-10}
  % enable toc data within our block, we go as far as subsubsection
  \cftinsertcode{start-\thetocmarker}{\normalchangetocdepth{3}}
  % when the block is done, disable the remaining
  \cftinsertcode{end-\thetocmarker}{\normalchangetocdepth{-10}}
  % remove the spacing above the toc title
  \let\tocheadstart\relax
  % remove the toc title itself
  \let\printtoctitle\@gobble
  % remove space below title
  \let\aftertoctitle\relax
  % reformat TOC entries:
  \setlength{\cftsectionindent}{0pt}
  \setlength{\cftsubsectionindent}{\cftsectionnumwidth}
  \setlength{\cftsubsubsectionindent}{\cftsubsectionindent}
  \addtolength{\cftsubsubsectionindent}{\cftsubsectionnumwidth}
  \renewcommand\cftsectionfont{\small}
  \renewcommand\cftsectionpagefont{\small}
  \renewcommand\cftsubsectionfont{\small}
  \renewcommand\cftsubsectionpagefont{\small}
  \renewcommand\cftsubsubsectionfont{\small}
  \renewcommand\cftsubsubsectionpagefont{\small}
  % include the actual ToC data
  \tableofcontents*
  \endgroup
  % restore tocdepth
  \setcounter{tocdepth}{\value{@memmarkcntra}}
  % to indent or not after the chapter toc
  \m@mindentafterchapter
  % space between chapter toc and text
  \par\bigskip
  % handles indentation after the macro
  \@afterheading}
\makeatother
\end{lcode}
\end{sniplet}

Note that if the \cmd{\chapterprecistoc} or \cmd{\chapterprecis} has
been used then that data is also added to the \toc{} data, and we will
need to locally disable it in the chapter \toc{}. This can be done by
adding
\begin{lcode}
  \let\precistoctext\@gobble
\end{lcode}
to the \cmd{\chaptertoc} definition above, just make sure it is added
before calling
before \cmd{\tableofcontents*}.


%%%%%
%%  Appendix ToC
%%%%


\begin{sniplet}[An appendix ToC]
  \label{snip:apptoc}
  Here we assume a structure like
\begin{verbatim}
\tableofcontents*
\chapter
\chapter
\chapter
\appendix
\appendixpage
\appendixtableofcontents
\chapter
\chapter
\chapter
\end{verbatim}
where the first \toc{} should just show until (and including) 
\cmd{\appendixpage}, and \cmd{\appendixtableofcontents} should only
list the appendices. 

We also assume that no \cmd{\settocdepth}'s have been issued after
\cmd{\appendixpage}.


We only need a single hook after \cmd{\appendixpage}.
\begin{lcode}
\renewcommand\mempostaddapppagetotochook{\cftinserthook{toc}{BREAK}}
\cftinsertcode{BREAK}{\changetocdepth{-10}}
\let\normalchangetocdepth\changetocdepth     % needed for later
\end{lcode}
Then the definition of the actual appendix \toc{}:
\begin{lcode}
\makeatletter
\newcommand\appendixtableofcontents{
  \begingroup
  \let\changetocdepth\@gobble
  \normalchangetocdepth{-10}
  \cftinsertcode{BREAK}{\normalchangetocdepth{3}}
  \renewcommand\contentsname{Appendices overview}
  \tableofcontents*
  \endgroup
}
\makeatother
\end{lcode}


\end{sniplet}




\cftinserthook{toc}{end-sniplets}

%#% extend
%#% extstart include pictures.tex

\svnidlong
{$Ignore: $}
{$LastChangedDate: 2014-03-31 11:34:44 +0200 (Mon, 31 Mar 2014) $}
{$LastChangedRevision: 480 $}
{$LastChangedBy: daleif $}

%%%%%%%%%%%%%%%%%%%%%%%%%%%%%%%%%%%%%%%%%%%%%%%%%%%%%%%%%

\LMnote{2010/02/08}{Added this entire chapter from PW sources}


\chapter{Pictures}  \label{chap:lpic}
%%%%%%%%%%%%%%%%%%%%%%%%%%%%%%%%%%%%%%%%%%%%%%%%%%%%%%%%%

\LMnote{2010/02/08}{Added this intro to show the motivation for adding
this}
\LMnote{2010/02/17}{fixed typo from Adriano Pascoletti}
There are many freely available \ltx\ introductions on \ctan\ and
other places. One thing that these apparently are not covering is the
traditional picture environment. It can be very handy in many
applications, though for more complex drawings the reader might be
better of with TiKz/pgf or PSTricks. For the benefit of the general
reader we here provide a lesson in the standard picture environment.

\LMnote{2010/02/08}{Added a writers note as not to change the text too
much, the most important thing about this note is the mentioning of
the picture package by Heiko, which enables us to use something like
\cs{put}(5mm, 5cm)}
\emph{Writers comment:} There are many extensions to the stock picture
environment provided by the \ltx\ kernel. We have chosen not to deal
with them in this chapter but instead concentrate on what you get as
is from the kernel. But there are a few handy packages that the reader
might want to explore: picture (by Heiko Oberdiek) which extends the
\cs{put} syntax to include arbitrary lengths, like 50mm; pict2e which
is mentioned in \cite{GCOMPANION} but just recently was released;
eepic. All packages are available from \ctan.



\fancybreak{}


    This chapter describes how to draw diagrams\index{diagram} and 
pictures\index{picture} using \ltx.
Pictures are drawn in the \Ie{picture} environment. You can draw\index{draw} 
straight
lines, arrows and circles; you can also put text into your pictures.

    Most often pictures are drawn within a \Ie{figure}\index{figure} 
environment, but
they may also be drawn as part of the normal text.




\section{Basic principles}

    The positions of picture elements are specified in terms of a 
two-dimensional cartesian coordinate\index{coordinate system} system. 
A \emph{coordinate}\index{coordinate} is a
number, such as \texttt{7}, \texttt{-21} or \texttt{1.78}. In the cartesian coordinate
system, a pair of coordinates (i.e., a pair of numbers) specifies a
position relative to the position designated as \texttt{(0,0)}. This special
position is called the \emph{origin}. 
The first of the coordinate pair\index{coordinate pair}
gives the value of the horizontal distance from the origin to the position.
A positive coordinate is an offset to the right and a negative number is
an offset to the left. The first value of a coordinate pair is called the
\emph{x coordinate}\index{coordinate}. 
The second value of a coordinate pair is called the
\emph{y coordinate}\index{coordinate} 
and gives the vertical offset from the origin (positive
upwards and negative downwards).

\begin{syntax}
\lnc{\unitlength} \\
\end{syntax}
\glossary(unitlength)
{\cs{unitlength}}{The unit of length in a picture environment. Default 1pt.}
    To draw a picture we also need to specify the units of measurement.
By default, \ltx\ takes the printer's point (there are 72.27 points to
an inch) as the measurement of length. The value of the unit of 
length\index{unit length} measurement
within a \Ie{picture} environment is actually given by the value
of the \lnc{\unitlength} length declaration.
 This can be changed
to any length that you like via the \cmd{\setlength} command. For example, \\
\verb?\setlength{\unitlength}{2mm}? \\
will make the value of \lnc{\unitlength} to be two millimeters.

    Figure~\ref{flpic:coords} shows the positions of some points and their
coordinate values. Coordinate pairs are typed as a pair of numbers, separated
by a comma, and enclosed in parentheses.

\begin{figure}
\setlength{\unitlength}{1mm}
\centering
\begin{picture}(80,70)
  \thicklines
% x axis
  \put(10,30){\begin{picture}(60,10)
    \thicklines \put(-5,0){\vector(1,0){65}}
    \thinlines  \multiput(0,0)(10,0){6}{\line(0,1){3}}
    \put(0,-3){\makebox(0,0)[t]{-20}}
    \put(10,-3){\makebox(0,0)[t]{-10}}
    \put(23,-3){\makebox(0,0)[t]{0}}
    \put(30,-3){\makebox(0,0)[t]{10}}
    \put(40,-3){\makebox(0,0)[t]{20}}
    \put(50,-3){\makebox(0,0)[t]{30}}
    \put(63,0){\makebox(0,0){x}}
    \end{picture}}
% y axis
  \put(30,10){\begin{picture}(10,60)
    \thicklines \put(0,-5){\vector(0,1){60}}
    \thinlines \multiput(0,0)(0,10){6}{\line(1,0){3}}
    \put(-3,0){\makebox(0,0)[r]{-20}}
    \put(-3,10){\makebox(0,0)[r]{-10}}
    \put(-3,30){\makebox(0,0)[r]{10}}
    \put(-3,40){\makebox(0,0)[r]{20}}
    \put(-3,50){\makebox(0,0)[r]{30}}
    \put(0,58){\makebox(0,0)[b]{y}}
    \end{picture}}

  \put(50,50){\begin{picture}(10,10)
    \put(0,0){\circle*{1}}
    \put(2,2){\makebox(0,0)[bl]{\texttt{(20,20)}}}
    \end{picture}}

  \put(15,45){\begin{picture}(10,10)
    \put(0,0){\circle*{1}}
    \put(-2,2){\makebox(0,0)[br]{\texttt{(-15,15)}}}
    \end{picture}}

  \put(15,20){\begin{picture}(10,10)
    \put(0,0){\circle*{1}}
    \put(-2,-2){\makebox(0,0)[tr]{\texttt{(-15,-10)}}}
    \end{picture}}

  \put(40,20){\begin{picture}(10,10)
    \put(0,0){\circle*{1}}
    \put(2,-2){\makebox(0,0)[tl]{\texttt{(10,-10)}}}
    \end{picture}}


\end{picture}
\setlength{\unitlength}{1pt}
\caption{Some points in the cartesian coordinate system}
\label{flpic:coords}
\end{figure}


\begin{syntax}
\cmd{\thinlines} \\
\cmd{\thicklines} \\
\cmd{\linethickness}\marg{len} \\
\end{syntax}
\glossary(thinlines)
{\cs{thinlines}}{Picture declaration for lines to be thin.}
\glossary(thicklines)
{\cs{thicklines}}{Picture declaration for lines to be thick.}
\glossary(linethickness)
{\cs{linethickness}\marg{len}}{Picture declaration for vertical and 
  horizontal lines to be \meta{len} thick.}

   In general, \ltx\ can draw lines of only two 
thicknesses\indextwo{thickness}{line}, 
thin and thick. The required
thickness is specified via either a \cmd{\thicklines} or a \cmd{\thinlines}
declaration, with the latter being the default. 

There is another declaration, \cmd{\linethickness}, which can be used 
to change the thickness of horizontal and vertical lines only.
It sets the thickness of these lines to \meta{len}, but has no effect on
any sloping lines.

    A \Ie{picture} environment has a required size pair argument that
specifies the width and height of the picture, in terms of the value
of \cmd{\unitlength}.
\begin{syntax}
\senv{picture}\parg{width, height} 
\meta{contents} 
\eenv{picture} \\
\senv{picture}\parg{width, height}\parg{llx, lly} 
\meta{contents} 
\eenv{picture} \\
\end{syntax}
\glossary(picture)
{\senv{picture}\parg{width, height}}{Creates a box of \meta{width} times 
  \meta{height} (in terms of \cs{unitlength}) in which you can use drawing
   commands. The origin is at (0,0).}
\glossary(picturex)
{\senv{picture}\parg{width, height}\parg{llx,lly}}{Creates a box of \meta{width} times 
  \meta{height} (in terms of \cs{unitlength}) in which you can use drawing
  commands. The origin is at (\meta{llx},\meta{lly}).}
 The environment creates a box\indextwo{box}{picture} 
of size \meta{width} by \meta{height}, which will not be
split\index{box!unbreakable} across pages. The default position of
the origin in this environment is at the lower left hand corner of the box.
For example,
\begin{lcode}
\begin{picture}(80,160)
\end{lcode}
creates a picture box of width 80 and height 160 whose lower left hand corner
is at \texttt{(0,0)}. There is also an optional coordinate pair argument (which
comes after the required argument) that specifies the coordinates of the
lower left hand corner of the box if you do not want the default origin.
\begin{lcode}
\begin{picture}(80,160)(10,20)
\end{lcode}
specifies a picture box of width 80 and height 160, as before, but with
the bottom left hand corner having coordinates of \texttt{(10,20)}. Thus, the
top right hand corner will have coordinates \texttt{(90,180)}. Note that the
optional argument is enclosed in parentheses not square brackets as is 
ordinarily the case. Typically, the optional argument is used when you
want to shift everything in the picture. \ltx\ uses the required argument
to determine the space required for typesetting the result. 

    You are not limited to drawing within the box, but if you do draw 
outside the box the result might run into other text on the page, or even 
run off the page altogether. \ltx\ only reserves the space you specify
for the picture and does not know if anything protrudes. In particular
\begin{lcode}
\begin{picture}(0,0)
\end{lcode}
creates a zero-sized\index{picture!zero-sized} picture which takes no
space at all. This can be very useful if you need to position
things on the page.


\LMnote{2010/02/17}{fixed typo from Adriano Pascoletti}
    Within the \Ie{picture} environment, \ltx\ is in a special \emph{picture}
mode\indextwo{mode}{picture} (which is a restriced form of LR 
mode\indextwo{mode}{LR}).
The only commands that can appear in picture mode are  \cmd{\put}, 
\cmd{\multiput} and \cmd{\qbezier} commands, and a few
declarations such as the type style and the thickness declarations.
By the way,
you should only change the value of \cmd{\unitlength} outside 
picture mode\indextwo{mode}{picture}
otherwise \ltx\ will get confused about its measurements.

\section{Picture objects}

    In a picture everything is placed and drawn by the \cmd{\put} (or
its \cmd{\multiput} variant) command.
\begin{syntax}
\cmd{\put}\parg{x, y}\marg{object} \\
\end{syntax}
\glossary(put)
{\cs{put}\parg{x,y}\marg{object}}{Drawing command to place \meta{object}
at coordinates (\meta{x,y}). In the simplest case \meta{object} 
is just text.}
\cmd{\put} places \meta{object} in the picture with the 
object's \emph{reference point}\index{reference point}
 at position \parg{x, y}.

    The following sections describe the various picture 
objects\index{picture object}.

\subsection{Text}

    Text\index{picture object!text} is the simplest kind of picture object. 
This is typeset in LR mode\indextwo{mode}{LR} and the 
reference point\index{reference point!text} is at 
the lower left hand corner of the text.
\begin{egsource}{eg:pic1}
\setlength{\unitlength}{1mm}  % measurements in millimeters
\begin{picture}(30,10)        % define size of picture
\put(0,0){\framebox(30,10){}} % draw frame around picture
\put(10,5){Some text}         % place text 
\thicklines
\put(10,5){\vector(1,1){0}}   % mark reference point
\end{picture}
\setlength{\unitlength}{1pt}  % reset measurements to default
\end{egsource}

\begin{egresult}[Picture: text]{eg:pic1}
\vspace*{0.5\onelineskip}
\setlength{\unitlength}{1mm}
\begin{picture}(30,10)
\put(0,0){\framebox(30,10){}}
\put(10,5){Some text}
\thicklines
\put(10,5){\vector(1,1){0}}
\end{picture}
\setlength{\unitlength}{1pt}
\end{egresult}

    In the diagram, and those following, 
the reference point is indicated by an arrow. Also, a
box is drawn round the diagram at the same size as the \Ie{picture}
environment.

\subsection{Boxes} \label{slpic:boxes}

    A box\index{picture object!box} picture object is made with one of 
the box\index{box} commands. When used
in picture mode\indextwo{mode}{picture}, the box commands have a slightly 
different form than when
in normal text. The first argument of a box command is a size pair
that specifies the width and height of the box. The last argument is the
text to be placed in the box. The reference point\indextwo{reference point}{box} 
of a box is the lower left hand corner.
\begin{syntax}
\cmd{\framebox}\parg{width, height}\oarg{pos}\marg{text} \\
\cmd{\makebox}\parg{width, height}\oarg{pos}\marg{text} \\
\end{syntax}
\glossary(framebox)
{\cs{framebox}\parg{width,height}\oarg{pos}\marg{text}}{A picture
object framing \meta{text} (at position \meta{pos}) in a box \meta{width} 
by \meta{height}.}
\glossary(makebox)
{\cs{makebox}\parg{width,height}\oarg{pos}\marg{text}}{A picture
object putting \meta{text} (at position \meta{pos}) in a box \meta{width} 
by \meta{height}.}
    The \cmd{\framebox} command draws a 
framed box\indextwo{box}{framed}\index{picture object!framed box}
of the specified \parg{width, height} dimensions around
the text.

\begin{egsource}{eg:pic2}
\setlength{\unitlength}{1pc}
\begin{picture}(22,5)
\put(0,0){\framebox(22,5){}}             % empty box
\thicklines
\put(2,1){\framebox(5,2.5){center}}      % centered text
\put(2,1){\vector(1,1){0}}               % ref point
\put(9,1){\framebox(5,2)[b]{bottom}}     % bottomed text
\put(9,1){\vector(1,1){0}}               % ref point
\put(16,1){\framebox(5,3)[tl]{top left}} % cornered text
\put(16,1){\vector(1,1){0}}              % ref point
\end{picture}
\setlength{\unitlength}{1pt}
\end{egsource}

\begin{egresult}[Picture: text in boxes]{eg:pic2}
\vspace*{0.5\onelineskip}
\setlength{\unitlength}{1pc}
\begin{picture}(22,5)
\put(0,0){\framebox(22,5){}}
\thicklines
\put(2,1){\framebox(5,2.5){center}}
\put(2,1){\vector(1,1){0}}
\put(9,1){\framebox(5,2)[b]{bottom}}
\put(9,1){\vector(1,1){0}}
\put(16,1){\framebox(5,3)[tl]{top left}}
\put(16,1){\vector(1,1){0}}
\end{picture}
\setlength{\unitlength}{1pt}
\end{egresult}

    The default position\index{picture object!framed box!text position} 
of the \textit{text} is centered in the box. 
However, this can
be changed via an optional argument (which is enclosed in square brackets),
placed between the coordinate pair and the text argument. This argument 
consists of either one or two of the following letters.
\begin{itemize}
\item[\pixposarg{l}] (left) Places the contents at the left of the box.
\item[\pixposarg{r}] (right) Places the contents at the right of the box.
\item[\pixposarg{t}] (top) Places the contents at the top of the box.
\item[\pixposarg{b}] (bottom) Places the contents at the bottom of the box.
\end{itemize}
These place the text in the corresponding position in the box. In a two-letter
argument the order of the letters is immaterial. For example, \verb?[tr]?
and \verb?[rt]? will both result in the text being placed at the top right
hand corner of the box. Unlike the normal \cmd{\framebox} command, a 
\cmd{\framebox} in a \Ie{picture} environment does not add any extra space
around the text.

    Corresponding to the \cmd{\framebox} there is a \cmd{\makebox} 
command\index{picture object!unframed box} which
does not draw a frame around its contents. The \cmd{\makebox} command takes the
same arguments as the \cmd{\framebox}. Particularly interesting is when you
specify a zero sized\index{picture object!unframed box!zero size} 
\cmd{\makebox}. A \verb?\makebox(0,0){text}?
command will make the reference point the center of \texttt{text}. Similarly,
the other positioning arguments\index{picture object!unframed box!text position}
which will adjust the reference point with respect to
the box contents. This can be used for fine-tuning the position of text
in a picture.

\begin{egsource}{eg:pic3}
\setlength{\unitlength}{1pc}
\begin{picture}(16,2)
\put(0,0){\framebox(16,2){}}
\thicklines
\put(3.5,1){\makebox(0,0){center}}      % ref at text center
\put(3.5,1){\vector(0,-1){0}}
\put(7,1){\makebox(0,0)[b]{bottom}}     % ref at text bottom
\put(7,1){\vector(0,1){0}}
\put(11,1){\makebox(0,0)[tl]{top left}} % ref at text top left
\put(11,1){\vector(1,-1){0}}
\end{picture}
\setlength{\unitlength}{1pt}
\end{egsource}

\begin{egresult}[Picture: positioning text]{eg:pic3}
\vspace{0.5\onelineskip}
\setlength{\unitlength}{1pc}
\begin{picture}(16,2)
\put(0,0){\framebox(16,2){}}
\thicklines
\put(3.5,1){\makebox(0,0){center}}
\put(3.5,1){\vector(0,-1){0}}
\put(7,1){\makebox(0,0)[b]{bottom}}
\put(7,1){\vector(0,1){0}}
\put(11,1){\makebox(0,0)[tl]{top left}}
\put(11,1){\vector(1,-1){0}}
\end{picture}
\setlength{\unitlength}{1pt}
\end{egresult}

    You can draw a dashed box\index{picture object!dashed box} 
with the \cmd{\dashbox} command.
\begin{syntax}
\cmd{\dashbox}\marg{len}\parg{width, height}\oarg{pos}\marg{text} \\
\end{syntax}
\glossary(dashbox)
{\cs{dashbox}\marg{len}\parg{width,height}\oarg{pos}\marg{text}}{A picture
object framing \meta{text} (at position \meta{pos}) in a dashed box 
\meta{width} by \meta{height}. The dashes and spaces are each \meta{len} long.}
 The first argument
of this command specifies the length of each dash. The following arguments
are the same as for the other box commands.

\begin{egsource}{eg:pic4}
\setlength{\unitlength}{4mm}
\begin{picture}(7,4)
\put(0,0){\framebox(7,4){}}
\thicklines
\put(1,1){\dashbox{0.5}(5,2)[tr]{top right}}
\put(1,1){\vector(1,1){0}}
\end{picture}
\setlength{\unitlength}{1pt}
\end{egsource}

\begin{egresult}[Picture: dashed box]{eg:pic4}
\vspace{0.5\onelineskip}
\setlength{\unitlength}{4mm}
\begin{picture}(7,4)
\put(0,0){\framebox(7,4){}}
\thicklines
\put(1,1){\dashbox{0.5}(5,2)[tr]{top right}}
\put(1,1){\vector(1,1){0}}
\end{picture}
\setlength{\unitlength}{1pt}
\end{egresult}

    The appearance of the box is best when the width and height of the
box are integer multiples of the dash length. In the example the dash length
has been set to \texttt{0.5} with the width and height set as \texttt{(5,2)}; thus 
the width and height are respectively ten and four times the dash length.

    The \cmd{\frame}\index{picture object!frame} command draws a 
frame around the contents of the
box that exactly fits the contents. 
\begin{syntax}
\cmd{\frame}\marg{contents} \\
\end{syntax}
\glossary(frame)
{\cs{frame}\marg{contents}}{A picture object drawing a frame about \meta{contents}.}
It takes a single required argument
which is the contents.

\begin{egsource}{eg:pic6}
\setlength{\unitlength}{1pc}
\begin{picture}(10,3)
\put(0,0){\framebox(10,3){}}
\thicklines
\put(0.5,2){\frame{$\mathcal{FRAME}$ text}}
\put(0.5,2){\vector(1,1){0}}
\end{picture}
\setlength{\unitlength}{1pt}
\end{egsource}

\begin{egresult}[Picture: framing]{eg:pic6}
\vspace{0.5\onelineskip}
\setlength{\unitlength}{1pc}
\begin{picture}(10,3)
\put(0,0){\framebox(10,3){}}
\thicklines
\put(0.5,2){\frame{$\mathcal{FRAME}$ text}}
\put(0.5,2){\vector(1,1){0}}
\end{picture}
\setlength{\unitlength}{1pt}
\end{egresult}


    The \cmd{\shortstack} command\index{picture object!stack} 
enables you to stack\index{stacking!text} text vertically. It 
produces a box with a single column of text. As with the other boxes, the
reference point\indextwo{reference point}{stack} is at the lower left hand corner,
although no frame is 
drawn around the stack. The \cmd{\shortstack} command is an ordinary
box making command, but it is not often used outside picture 
mode\indextwo{mode}{picture}.
\begin{syntax}
\cmd{\shortstack}\oarg{pos}\marg{text} \\
\end{syntax}
\glossary(shortstack)
{\cs{shortstack}\oarg{pos}\marg{text}}{Vertically stacks each line of 
\meta{text} into a column, normally centered but can be left or
 right aligned via \meta{pos}. Usually used as a picture object, 
but can be used outside the environment.}
    Each line of \meta{text}, except for the last, is 
ended\indextwo{end}{line} by a \cmd{\\} command.
The default is to center each text line within the column. However, there
is an optional positioning argument.
 A value of \pixposarg{l} for \meta{pos} will left align the
text and a value of \pixposarg{r} will right align the text.

\begin{egsource}{eg:pic7}
\setlength{\unitlength}{1mm}
\begin{picture}(75,25)
\put(0,0){\framebox(75,25){}}
\put(3,3){\shortstack{Default \\ short \\ Stack}}
\put(3,3){\vector(1,1){0}}
\put(23,3){\shortstack[l]{Left\\aligned\\short\\Stack}}
\put(23,3){\vector(1,1){0}}
\put(43,3){\shortstack[r]{Right\\aligned\\short\\Stack}}
\put(43,3){\vector(1,1){0}}
\put(63,3){\shortstack{Extra \\[4ex] spaced \\[2ex] Stack}}
\put(63,3){\vector(1,1){0}}
\end{picture}
\setlength{\unitlength}{1pt}
\end{egsource}

\begin{egresult}[Picture: stacking]{eg:pic7}
\vspace{0.5\onelineskip}
\setlength{\unitlength}{1mm}
\begin{picture}(75,25)
\put(0,0){\framebox(75,25){}}
\put(3,3){\shortstack{Default \\ short \\ Stack}}
\put(3,3){\vector(1,1){0}}

\put(23,3){\shortstack[l]{Left\\aligned\\short\\Stack}}
\put(23,3){\vector(1,1){0}}

\put(43,3){\shortstack[r]{Right\\aligned\\short\\Stack}}
\put(43,3){\vector(1,1){0}}

\put(63,3){\shortstack{Extra \\[4ex] spaced \\[2ex] Stack}}
\put(63,3){\vector(1,1){0}}
\end{picture}
\setlength{\unitlength}{1pt}
\end{egresult}

    The rows in a stack\index{stacking!text!vertical spacing} 
are not evenly spaced. The spacing between two
rows can be changed in one of two ways.
\begin{enumerate}
\item Add a strut to a row. A strut\index{strut} is a vertical rule with
      no width.
\item Use the optional argument to the \cmd{\\} command. This optional
  argument is a length value.
  \begin{syntax}
  \cmd{\\}\oarg{len} \\
  \end{syntax}
  It has the effect of adding additional \meta{len}
  vertical space between the two lines that the \cmd{\\} separates.
\end{enumerate}

\begin{syntax}
\cmd{\newsavebox}\marg{box} \\
\cmd{\savebox}\marg{box}\parg{width, height}\oarg{pos}\marg{text} \\
\cmd{\sbox}\marg{box}\marg{text} \\
\cmd{\usebox}\marg{box} \\
\end{syntax}
\glossary(savebox)
{\cs{savebox}\meta{box}\parg{width, height}\oarg{pos}\marg{text}}{Picture
 command to save 
 \meta{text} in a (pre-existing) storage box \meta{box} making it
 size \meta{width} times \meta{height}. The optional 
  argument controls the position of the \meta{text}.}

\LMnote{2010/02/17}{fixed typo from Adriano Pascoletti}
    Just as in normal text you can save and reuse boxes. 
The \cmd{\savebox}\index{box!saved object}
macro in picture mode\indextwo{mode}{picture} is a variant of the normal
text version, but the other three commands are the same in both 
picture\indextwo{mode}{picture} and paragraph\indextwo{mode}{paragraph} modes,
and are described in \Cref{chap:boxes}. In picture mode you have to
specify the size of the storage box when saving it, via the 
\parg{width, height} argument to \cmd{\savebox}.

    A \cmd{\savebox} command can be used within a picture to store a picture
object. The first argument of \cmd{\savebox} is the name of the
storage bin to be used. The following arguments are the same as the
\cmd{\makebox} command.
The result is stored, not drawn. When you have saved 
something it can be drawn in either the same or other pictures via the
\cmd{\usebox}\index{box!using saved object} command. 
This command takes one argument, which is the name
of the storage bin.

\begin{egsource}{eg:pic8}
\setlength{\unitlength}{1pc}
\begin{picture}(18,5)
\put(0,0){\framebox(18,5){}}
\newsavebox{\Mybox}
\savebox{\Mybox}(6,3)[tr]{$\mathcal{SAVED}$}
\thicklines
\put(1,1){\frame{\usebox{\Mybox}}}
\put(11,1){\frame{\usebox{\Mybox}}}
\put(1,1){\vector(1,1){0}}
\put(11,1){\vector(1,1){0}}
\end{picture}
\setlength{\unitlength}{1pt}
\end{egsource}

\begin{egresult}[Picture: saved boxes]{eg:pic8}
\vspace{0.5\onelineskip}
\setlength{\unitlength}{1pc}
\begin{picture}(18,5)
\put(0,0){\framebox(18,5){}}
\newsavebox{\Mybox}
\savebox{\Mybox}(6,3)[tr]{$\mathcal{SAVED}$}
\thicklines
\put(1,1){\frame{\usebox{\Mybox}}}
\put(11,1){\frame{\usebox{\Mybox}}}
\put(1,1){\vector(1,1){0}}
\put(11,1){\vector(1,1){0}}
\end{picture}
\setlength{\unitlength}{1pt}
\end{egresult}

    It can take \ltx\ a long time to draw something. When a box is saved
it actually contains the typeset contents, which then just get 
printed out when the box is
used. It can save processing time if something which appears several times
is saved and then used as and where required. On the other hand, a saved
box can use up a significant amount of \ltx's internal storage space.
The \cmd{\sbox} command with an empty text argument can be used to delete
the contents of a bin. For example, 
\begin{lcode}
\sbox{\Mybox}{}
\end{lcode}
will empty the \verb?\Mybox? box. Note that this does not delete the 
storage box itself.

\subsection{Lines}

    \ltx\ can draw straight 
lines\index{picture object!line|seealso{line}}\index{line|seealso{picture object}}, 
but the range of slopes\index{line!restricted slope} for lines is
somewhat restricted. Further, very short lines\index{line!short} cannot 
be drawn.
\begin{syntax}
\cmd{\line}\parg{i, j}\marg{distance} \\
\end{syntax}
\glossary(line)
{\cs{line}\parg{dx,dy}\marg{distance}}{Picture object of a line, slope
  \meta{dx,dy} and coordinate length \meta{distance}.}
The pair \parg{i, j} specifies the 
\emph{slope}\index{slope}\indextwo{line}{slope} of the line, and \meta{distance} 
is a value that controls the length\index{line!length} of the line. The line
starts at its reference point\indextwo{reference point}{line} 
(i.e., the place where it is \cmd{\put}).
The slope\index{slope} of the line is such that if a point on the line is slid along the
line, then for every $i$ units the point moves in the horizontal direction
it will also have moved $j$ units in the vertical direction. Negative
values for $i$ or $j$ have the expected meaning. A move of -3 units in $i$
means a move of 3 units to the left, and similarly a move of -4 units in
$j$ means a move of 4 units downwards. So, a line sloping up to the right
will have positive values for $i$ and $j$, while a line sloping up to the
left will have a negative value for $i$ and a positive value for $j$.

    The \meta{distance} argument specifies the length of the line in the
$x$ (horizontal) direction. One problem with this may have occured to you:
what if the line is vertical (i.e., $i=0$)? In this case only, \meta{distance}
specifies the vertical length of the line. The \meta{distance} argument must
be a non-negative value. For horizontal and vertical lines only, the actual
length of the line is \meta{distance}. Figure~\ref{flpic:spec}, which is
produced from the code below, diagrams the line 
specification arguments.

\begin{lcode}
\begin{figure}
\centering
\setlength{\unitlength}{1mm}
\begin{picture}(70,60)
\thicklines   % draw line and ref point
  \put(10,20){\line(2,1){40}}
  \put(10,20){\vector(1,-1){0}}
\thinlines    % draw axes
  \put(0,0){\vector(1,0){60}} \put(63,0){x}
  \put(0,0){\vector(0,1){50}} \put(0,53){y}
              % draw i and j vectors
  \put(20,25){\vector(1,0){20}} 
  \put(30,22){\makebox(0,0)[t]{$i$}}
  \put(40,25){\vector(0,1){10}} 
  \put(42,30){\makebox(0,0)[l]{$j$}}
              % draw distance vector
  \put(30,10){\vector(-1,0){20}}
  \put(30,10){\vector(1,0){20}}
  \put(30,8){\makebox(0,0)[t]{\textit{distance}}}
\end{picture}
\setlength{\unitlength}{1pt}
\caption{Specification of a line or arrow}
\label{flpic:spec}
\end{figure}
\end{lcode}

\begin{figure}
\centering
\setlength{\unitlength}{1mm}
\begin{picture}(70,60)
\thicklines
  \put(10,20){\line(2,1){40}}
  \put(10,20){\vector(1,-1){0}}
\thinlines
  \put(0,0){\vector(1,0){60}} \put(63,0){x}
  \put(0,0){\vector(0,1){50}} \put(0,53){y}

  \put(20,25){\vector(1,0){20}} \put(30,22){\makebox(0,0)[t]{$i$}}
  \put(40,25){\vector(0,1){10}} \put(42,30){\makebox(0,0)[l]{$j$}}

  \put(30,10){\vector(-1,0){20}}
  \put(30,10){\vector(1,0){20}}
  \put(30,8){\makebox(0,0)[t]{\textit{distance}}}

\end{picture}
\setlength{\unitlength}{1pt}
\caption{Specification of a line or arrow}
\label{flpic:spec}
\end{figure}

    Only a fixed number of slopes\index{line!restricted slope} 
are available. This is because \ltx\ uses
a special font for drawing lines --- a line actually consists of little bits
of angled rules joined together. Thus, there is only a limited number of
values for $i$ and $j$. They must both be integers and in the range
$-6 \leq i,j \leq 6$. Also, they must have no common divisor other than 1.
In other words, the ratio between $i$ and $j$ must be in its simplest form.
You cannot, for example, have $(3,6)$; instead it would have to be $(1,2)$.
The shortest line that \ltx\ can draw is about ten points 
(1/7 inch approximately) in overall length. You can, though, draw lines that
are too long to fit on the page.

    Figure~\ref{flpic:lslope} shows the lines and arrows slanting
upwards and to the right that can be drawn in \ltx. The slope $(i,j)$
pair are shown to the right of the first set of lines and arrows, together
with the $j/i$ ratio which gives the slope of the line as a decimal number.


\subsection{Arrows}

    As shown in \fref{flpic:lslope} you can also draw a line with an 
arrowhead\index{line!arrowhead|see{vector}}\index{arrow|see{vector}} 
on it. These are specified 
by the \cmd{\vector}\index{vector|seealso{picture object}}\index{picture object!vector|seealso{vector}}
command.
\begin{syntax}
\cmd{\vector}\parg{i, j}\marg{distance} \\
\end{syntax}
\glossary(vector)
{\cs{vector}\parg{dx,dy}\marg{distance}}{Picture object of a line with an
arrowhead at the end, slope   \meta{dx,dy} and coordinate 
length \meta{distance}.}
 This works exactly
like the \cmd{\line} command and the arrowhead is put on the line at the
end away from the reference point\indextwo{reference point}{vector}. 
That is, the arrow points away from the
reference point. If the \meta{distance} argument is too small (zero,
for instance) the arrowhead only is drawn, with its point at the position
where it is \cmd{\put}.

\begin{figure}
\centering
\setlength{\unitlength}{1mm}
\begin{picture}(70,70)
% lines and arrows with slopes <= (1,1)
\thicklines \put(0,0){\vector(1,0){50}}
\thinlines \put(0,0){\line(6,1){50}}
           \put(0,0){\line(5,1){50}}
\thicklines \put(0,0){\vector(4,1){50}}
            \put(0,0){\vector(3,1){50}}
\thinlines \put(0,0){\line(5,2){50}}
\thicklines \put(0,0){\vector(2,1){50}}
\thinlines \put(0,0){\line(5,3){50}}
\thicklines \put(0,0){\vector(3,2){50}}
            \put(0,0){\vector(4,3){50}}
\thinlines \put(0,0){\line(5,4){50}}
           \put(0,0){\line(6,5){50}}
\thicklines \put(0,0){\vector(1,1){50}}
% label vertically
\put(53,0){\begin{picture}(20,60)
  \begin{footnotesize}
  \put(0,0){$(1,0)=0$}
  \put(0,7.333){$(6,1)=0.167$}
  \put(0,10){$(5,1)=0.2$}
  \put(0,12.5){$(4,1)=0.25$}
  \put(0,16.667){$(3,1)=0.333$}
  \put(0,20){$(5,2)=0.4$}
  \put(0,25){$(2,1)= 0.5$}
  \put(0,30){$(5,3)= 0.6$}
  \put(0,33.333){$(3,2)=0.667$}
  \put(0,37.5){$(4,3)=0.75$}
  \put(0,40){$(5,4)=0.8$}
  \put(0,42.667){$(6,5)=0.833$}
  \put(0,50){$(1,1)=1$}
  \end{footnotesize}
  \end{picture}}
% remaining lines and arrows
\thinlines \put(0,0){\line(5,6){41.667}}
           \put(0,0){\line(4,5){40}}
\thicklines \put(0,0){\vector(3,4){37.5}}
            \put(0,0){\vector(2,3){33.333}}
\thinlines \put(0,0){\line(3,5){30}}
\thicklines \put(0,0){\vector(1,2){25}}
\thinlines \put(0,0){\line(2,5){20}}
\thicklines \put(0,0){\vector(1,3){16.667}}
            \put(0,0){\vector(1,4){12.5}}
\thinlines \put(0,0){\line(1,5){10}}
           \put(0,0){\line(1,6){8.333}}
\thicklines \put(0,0){\vector(0,1){50}}
% label horizontally
\put(0,55){\begin{picture}(50,10)
  \begin{footnotesize}
%  \put(41.667,5){\makebox(0,0){$(5,6)=1.2$}}
%  \put(40,0){\makebox(0,0){$(4,5)=1.25$}}
%  \put(37.5,5){\makebox(0,0){$(3,4)=1.333$}}
%  \put(33.333,0){\makebox(0,0){$(2,3)=1.5$}}
%  \put(30,5){\makebox(0,0){$(3,5)=1.667$}}
  \put(25,0){\makebox(0,0){$(1,2)=2$}}
%  \put(20,5){\makebox(0,0){$(2,5)=2.5$}}
%  \put(16.667,0){\makebox(0,0){$(1,3)=3$}}
%  \put(12.5,5){\makebox(0,0){$(1,4)=4$}}
%  \put(10,0){\makebox(0,0){$(1,5)=5$}}
%  \put(8.333,5){\makebox(0,0){$(1,6)=6$}}
  \put(0,0){\makebox(0,0){$(0,1)= \infty$}}
  \end{footnotesize}
  \end{picture}}
\end{picture}
\setlength{\unitlength}{1pt}
\caption{Sloping lines and arrows} \label{flpic:lslope}
\end{figure}

    \ltx\ is even more restrictive in the number of slopes that it can draw
with arrows\index{vector!restricted slope}\indextwo{slope}{vector} 
than it is with lines. 
The $(i,j)$ slope specification pair
must lie in the range $-4 \leq i,j \leq 4$. Also, as with the \cmd{\line}
command, they must have no common divisor.



\subsection{Circles}

    \ltx\ can draw two kinds of circles\index{picture object!circle|seealso{circle}}\index{circle|seealso{picture object}}. 
One is an open circle\index{circle!open} where only
the perimeter is drawn, and the other is a solidly filled 
disk\index{circle!disk}\index{disk|see{circle}}.
\begin{syntax}
\cmd{\circle}\marg{diameter} \\
\cmd{\circle*}\marg{diameter} \\
\end{syntax}
\glossary(circle)
{\cs{circle}\marg{diam}}{Picture object of a open circle 
diameter \meta{diam}.}
\glossary(circle*)
{\cs{circle*}\marg{diam}}{Picture object of a black closed circle 
diameter \meta{diam}.}

    The reference point\indextwo{reference point}{circle} for the open circle, 
drawn by the \cmd{\circle}
command, and the disk, which is drawn by the \cmd{\circle*} command, 
is at the center
of the circle. The argument to the commands is the \meta{diameter} 
of the circle.

\begin{egsource}{eg:pic10}
\setlength{\unitlength}{1pt}
\begin{picture}(200,60)
\put(0,0){\framebox(200,60){}}
\put(30,30){\circle{40}}
\put(30,30){\vector(1,1){0}}
\put(150,50){\circle*{20}}
\end{picture}
\setlength{\unitlength}{1pt}
\end{egsource}

\begin{egresult}[Picture: circles]{eg:pic10}
\vspace{0.5\onelineskip}
\setlength{\unitlength}{1pt}
\begin{picture}(200,60)
\put(0,0){\framebox(200,60){}}
\put(30,30){\circle{40}}
\put(30,30){\vector(1,1){0}}
\put(150,50){\circle*{20}}
\end{picture}
\setlength{\unitlength}{1pt}
\end{egresult}

    Just as with the \cmd{\line} and \cmd{\vector} commands, there is only a 
limited range of circles\index{circle!restricted diameter} that can be drawn.
Typically, the maximum diamter of a \cmd{\circle} is about 40 points, while
for a \cmd{\circle*} the maximum diameter is less, being about 15 points. 
\ltx\ will choose
the nearest sized circle to the one that you specify. Either consult your
local guru to find what sized circles you can draw on your system, or
try some experiments by drawing a range of circles to see what happens.

\subsubsection{Quarter circles and boxes}

    In \ltx\ an \cmd{\oval}\index{picture object!oval|seealso{box, rounded}}%
\index{oval|seealso{picture object}} is a rectangular 
box\index{box!rounded|seealso{oval}} with rounded corners.
\begin{syntax}
\cmd{\oval}\parg{width, height}\oarg{portion} \\
\end{syntax}
\glossary(oval)
{\cs{oval}\parg{width,height}\oarg{portion}}{Picture object of a rectangular
box, size \meta{width} by \meta{height}, with rounded corners. The optional
\meta{portion} argument controls whether and which a quarter or a half of 
the object will be drawn (default is everything).}

    The \cmd{\oval} command has one required argument which specifies the
width and height of the box. The normally sharp corners of the box are
replaced by quarter circles of the maximum possible radius (which \ltx\
figures out for itself). Unlike the boxes discussed earlier, the reference
point\indextwo{reference point}{oval} is at the `center' of the oval.

\begin{egsource}{eg:pic11}
\setlength{\unitlength}{1mm}
\begin{picture}(75,20)
\thicklines
\put(0,0){\framebox(75,20){}}
\put(15,10){\oval(15,10)}     % complete oval
\put(15,10){\vector(1,1){0}}
\put(30,10){\oval(5,5)}       % small oval
\put(30,10){\vector(1,1){0}}
\put(45,10){\oval(15,10)[l]}  % left half
\put(45,10){\vector(1,1){0}}
\put(60,10){\oval(15,10)[bl]} % bottom left quarter
\put(60,10){\vector(1,1){0}}
\end{picture}
\setlength{\unitlength}{1pt}
\end{egsource}

\begin{egresult}[Picture: ovals]{eg:pic11}
\vspace{0.5\onelineskip}
\setlength{\unitlength}{1mm}
\begin{picture}(75,20)
\thicklines
\put(0,0){\framebox(75,20){}}
\put(15,10){\oval(15,10)}
\put(15,10){\vector(1,1){0}}

\put(30,10){\oval(5,5)}
\put(30,10){\vector(1,1){0}}

\put(45,10){\oval(15,10)[l]}
\put(45,10){\vector(1,1){0}}

\put(60,10){\oval(15,10)[bl]}
\put(60,10){\vector(1,1){0}}
\end{picture}
\setlength{\unitlength}{1pt}
\end{egresult}

    The \cmd{\oval} command also has one optional argument, \meta{portion}, 
which comes after
the required argument.
 Use of the optional argument enables either half\index{box!rounded!half} or
a quarter\index{box!rounded!quarter} of the complete rounded box
to be drawn. The argument is a one or two
letter code drawn from the following.
\begin{itemize}
\item[\pixposarg{l}] (left) Draw the left of the oval.
\item[\pixposarg{r}] (right) Draw the right of the oval.
\item[\pixposarg{t}] (top) Draw the top of the oval.
\item[\pixposarg{b}] (bottom) Draw the bottom of the oval.
\end{itemize}
These are similar to the optional positioning argument in the box commands.
A one letter code will draw the designated half of the oval, while a two letter
code results in the designated quarter of the oval being drawn. In all cases
the reference point is at the center of the `complete' oval.



\begin{egsource}{eg:pic12}
\setlength{\unitlength}{1mm}
\begin{picture}(30,10)
\thicklines
\put(15,5){\oval(30,10)}
\put(15,5){\makebox(0,0){Text in oval}}
\end{picture}
\setlength{\unitlength}{1pt}
\end{egsource}

\begin{egresult}[Picture: text in oval]{eg:pic12}
\vspace{0.5\onelineskip}
\setlength{\unitlength}{1mm}
\begin{picture}(30,10)
\thicklines
\put(15,5){\oval(30,10)}
\put(15,5){\makebox(0,0){Text in oval}}
\end{picture}
\setlength{\unitlength}{1pt}
\end{egresult}

    Unlike the boxes described in \Sref{slpic:boxes} there is no \meta{text}
argument for an \cmd{\oval}. If you want the rounded box to contain text, 
then you have to place the text inside the box yourself. 
The code in example~\ref{eg:pic12} shows
one way of doing this; a zero-sized box is used to center the text at
the center of the oval.


\section{Repetitions}

    The \cmd{\multiput} command is a convenient way to place regularly spaced
copies of an object\index{picture object!regular pattern} 
in a picture.
\begin{syntax}
\cmd{\multiput}\parg{x, y}\parg{dx, dy}\marg{num}\marg{object} \\
\end{syntax}
\glossary(multiput)
{\cs{multiput}\parg{x,y}\parg{dx,dy}\marg{num}\marg{object}}{Drawing
command to place \meta{num} copies of \meta{object}, starting at coordinates
\meta{x,y} and stepping \meta{dx,dy} for each copy after the first.}
As you can see, this is similar to the syntax for the \cmd{\put} command, 
except that there are two more required arguments, namely \parg{dx, dy} 
and \textit{num}.

    The \parg{dx, dy} argument is a pair of (decimal) numbers that
specify the amount that the \meta{object} shall be moved at each repetition.
The first of this pair specifies the horizontal movement and the second the
vertical movement. Positive values shift to the right or up, and negative
numbers shift to the left or down. The \meta{num} argument specifies how many
times the \meta{object} is to be drawn.

    The code below produces \fref{flpic:scales}. 
This example
also shows that a \Ie{picture} can be placed 
inside\index{picture object!picture} another \Ie{picture}.
Often it is useful to break a complex diagram up into pieces, with each
piece being a separate \Ie{picture}. The pieces can then be individually
positioned within the overall diagram.

\begin{figure}
\setlength{\unitlength}{1pc}
\centering
%\begin{picture}(21,35)
\begin{picture}(21,26)
%  Draw Pica scale
%\put(2,2){\begin{picture}(5,33)
\put(2,2){\begin{picture}(5,24)
  \put(0,-0.5){\makebox(0,0)[t]{\textbf{Picas}}}
%  \thicklines \put(0,0){\line(0,1){35}}
  \thicklines \put(0,0){\line(0,1){24.0}}
%  \thinlines \multiput(0,0)(0,1){36}{\line(1,0){1}}
%             \multiput(0,0)(0,10){4}{\line(1,0){2}}
  \thinlines \multiput(0,0)(0,1){25}{\line(1,0){1}}
             \multiput(0,0)(0,10){3}{\line(1,0){2}}
  \put(-1,0){\makebox(0,0)[br]{0}}
  \put(-1,10){\makebox(0,0)[br]{10}}
  \put(-1,20){\makebox(0,0)[br]{20}}
%  \put(-1,30){\makebox(0,0)[br]{30}}
  \end{picture}}
%  Draw Points scale
%\put(7,2){\begin{picture}(5,33)
\put(7,2){\begin{picture}(5,24)
  \put(0,-0.5){\makebox(0,0)[t]{\textbf{Points}}}
%  \thicklines \put(0,0){\line(0,1){35}}
%  \thinlines \multiput(0,0)(0,0.8333){42}{\line(1,0){1}}
%             \multiput(0,0)(0,8.333){5}{\line(1,0){2}}
  \thicklines \put(0,0){\line(0,1){24.2}}
  \thinlines \multiput(0,0)(0,0.8333){30}{\line(1,0){1}}
             \multiput(0,0)(0,8.333){3}{\line(1,0){2}}
  \put(-1,0){\makebox(0,0)[br]{0}}
  \put(-1,8.333){\makebox(0,0)[br]{100}}
  \put(-1,16.667){\makebox(0,0)[br]{200}}
%  \put(-1,25){\makebox(0,0)[br]{300}}
%  \put(-1,33.333){\makebox(0,0)[br]{400}}
  \end{picture}}
%  Draw Millimeter scale
%\put(12,2){\begin{picture}(5,33)
\put(12,2){\begin{picture}(5,24)
  \put(0,-0.5){\makebox(0,0)[t]{\textbf{Millimeters}}}
%  \thicklines \put(0,0){\line(0,1){35}}
%  \thinlines \multiput(0,0)(0,0.4742){74}{\line(1,0){1}}
%             \multiput(0,0)(0,2.3711){15}{\line(1,0){2}}
  \thicklines \put(0,0){\line(0,1){24.2}}
  \thinlines \multiput(0,0)(0,0.4742){52}{\line(1,0){1}}
             \multiput(0,0)(0,2.3711){11}{\line(1,0){2}}
  \put(-1,0){\makebox(0,0)[br]{0}}
%  \put(-1,2.371){\makebox(0,0)[br]{10}}
  \put(-1,4.742){\makebox(0,0)[br]{20}}
%  \put(-1,7.113){\makebox(0,0)[br]{30}}
  \put(-1,9.484){\makebox(0,0)[br]{40}}
%  \put(-1,11.855){\makebox(0,0)[br]{50}}
  \put(-1,14,226){\makebox(0,0)[br]{60}}
%  \put(-1,16.597){\makebox(0,0)[br]{70}}
  \put(-1,18.968){\makebox(0,0)[br]{80}}
%  \put(-1,21.339){\makebox(0,0)[br]{90}}
  \put(-1,23.71){\makebox(0,0)[br]{100}}
%  \put(-1,26.081){\makebox(0,0)[br]{110}}
%  \put(-1,28.452){\makebox(0,0)[br]{120}}
%  \put(-1,30.823){\makebox(0,0)[br]{130}}
%  \put(-1,33.194){\makebox(0,0)[br]{140}}
  \end{picture}}
%  Draw Inch scale
%\put(17,2){\begin{picture}(5,33)
\put(17,2){\begin{picture}(5,24)
  \put(0,-0.5){\makebox(0,0)[t]{\textbf{Inches}}}
%  \thicklines \put(0,0){\line(0,1){35}}
%  \thinlines \multiput(0,0)(0,0.60225){59}{\line(1,0){1}}
%             \multiput(0,0)(0,6.0225){6}{\line(1,0){2}}
  \thicklines \put(0,0){\line(0,1){24.1}}
  \thinlines \multiput(0,0)(0,0.60225){41}{\line(1,0){1}}
             \multiput(0,0)(0,6.0225){5}{\line(1,0){2}}
  \put(-1,0){\makebox(0,0)[br]{0}}
  \put(-1,6.0225){\makebox(0,0)[br]{1}}
  \put(-1,12.045){\makebox(0,0)[br]{2}}
  \put(-1,18.0675){\makebox(0,0)[br]{3}}
  \put(-1,24.09){\makebox(0,0)[br]{4}}
%  \put(-1,30.1125){\makebox(0,0)[br]{5}}
  \end{picture}}

\end{picture}
\setlength{\unitlength}{1pt}
\caption{Some measuring scales} \label{flpic:scales}
\end{figure}

\begin{lcode}
\begin{figure}
\setlength{\unitlength}{1pc}
\centering
\begin{picture}(21,26)
%  Draw Pica scale
\put(2,2){\begin{picture}(5,24)
  \put(0,-0.5){\makebox(0,0)[t]{\textbf{Picas}}}
  \thicklines \put(0,0){\line(0,1){24.0}}
  \thinlines \multiput(0,0)(0,1){25}{\line(1,0){1}}
             \multiput(0,0)(0,10){3}{\line(1,0){2}}
  \put(-1,0){\makebox(0,0)[br]{0}}
  \put(-1,10){\makebox(0,0)[br]{10}}
  \put(-1,20){\makebox(0,0)[br]{20}}
  \end{picture}}
%  Draw Points scale
\put(7,2){\begin{picture}(5,24)
  \put(0,-0.5){\makebox(0,0)[t]{\textbf{Points}}}
  \thicklines \put(0,0){\line(0,1){24.2}}
  \thinlines \multiput(0,0)(0,0.8333){30}{\line(1,0){1}}
             \multiput(0,0)(0,8.333){3}{\line(1,0){2}}
  \put(-1,0){\makebox(0,0)[br]{0}}
  \put(-1,8.333){\makebox(0,0)[br]{100}}
  \put(-1,16.667){\makebox(0,0)[br]{200}}
  \end{picture}}
%  Draw Millimeter scale
\put(12,2){\begin{picture}(5,24)
  \put(0,-0.5){\makebox(0,0)[t]{\textbf{Millimeters}}}
  \thicklines \put(0,0){\line(0,1){24.2}}
  \thinlines \multiput(0,0)(0,0.4742){15}{\line(1,0){1}}
             \multiput(0,0)(0,2.3711){11}{\line(1,0){2}}
  \put(-1,0){\makebox(0,0)[br]{0}}
  \put(-1,4.742){\makebox(0,0)[br]{20}}
  \put(-1,9.484){\makebox(0,0)[br]{40}}
  \put(-1,14,226){\makebox(0,0)[br]{60}}
  \put(-1,18.968){\makebox(0,0)[br]{80}}
  \put(-1,23.71){\makebox(0,0)[br]{100}}
  \end{picture}}
%  Draw Inch scale
\put(17,2){\begin{picture}(5,24)
  \put(0,-0.5){\makebox(0,0)[t]{\textbf{Inches}}}
  \thicklines \put(0,0){\line(0,1){24.1}}
  \thinlines \multiput(0,0)(0,0.60225){41}{\line(1,0){1}}
             \multiput(0,0)(0,6.0225){5}{\line(1,0){2}}
  \put(-1,0){\makebox(0,0)[br]{0}}
  \put(-1,6.0225){\makebox(0,0)[br]{1}}
  \put(-1,12.045){\makebox(0,0)[br]{2}}
  \put(-1,18.0675){\makebox(0,0)[br]{3}}
  \put(-1,24.09){\makebox(0,0)[br]{4}}
  \end{picture}}

\end{picture}
\setlength{\unitlength}{1pt}
\caption{Some measuring scales} \label{flpic:scales}
\end{figure}
\end{lcode}


    You can also make regular
two-dimensional\index{picture object!two-dimensional pattern} patterns 
by using a 
\cmd{\multiput} pattern inside another \cmd{\multiput}. As \ltx\ will
process each \cmd{\multiput} every time it is repeated it is often more
convenient to store the results of the first \cmd{\multiput} in a bin
and then use this as the argument to the second \cmd{\multiput}.

\begin{egsource}{eg:pic14}
\setlength{\unitlength}{1mm}
\begin{picture}(32,14)
\put(0,0){\framebox(32,14){}}
\savebox{\Mybox}(8,8){\multiput(0,0)(4,4){3}{\circle*{1}}}
\multiput(4,4)(6,0){4}{\usebox{\Mybox}}
\sbox{\Mybox}{}
\end{picture}
\setlength{\unitlength}{1pt}
\end{egsource}

\begin{egresult}[Picture: repetitions]{eg:pic14}
\vspace{0.5\onelineskip}
\setlength{\unitlength}{1mm}
\begin{picture}(32,14)
\put(0,0){\framebox(32,14){}}
\savebox{\Mybox}(8,8){\multiput(0,0)(4,4){3}{\circle*{1}}}
\multiput(4,4)(6,0){4}{\usebox{\Mybox}}
\sbox{\Mybox}{}
\end{picture}
\setlength{\unitlength}{1pt}
\end{egresult}

   Remember that
a storage bin must have been declared via a \cmd{\newsavebox} command before
it can be used. I originally declared and used
the \cs{Mybox} bin in \Sref{slpic:boxes}.
As the above example shows, you can change the contents of a storage bin
by utilising it in another \cmd{\savebox}. Storage bins can use up a 
lot of \ltx's memory. After
you have finished with a storage bin empty it via the \cmd{\sbox}
command with an empty last argument, as shown in the example.

\section{Bezier curves}

    Standard \ltx\ provides one further drawing command ---
the \cmd{\qbezier}\index{picture object!Bezier curve|seealso{Bezier curve}} 
command. 
This can be used for drawing fairly arbitrary curves.
\begin{syntax}
\cmd{\qbezier}\oarg{num}\parg{Xs, Ys}\parg{Xm, Ym}\parg{Xe, Ye} \\
\end{syntax}
\glossary(qbezier)
{\cs{qbezier}\oarg{num}\parg{Xs,Ys}\parg{Xm,Ym}\parg{Xe,Ye}}{Picture
command to draw a quadratic Bezier curve from \meta{Xs,Ys} to \meta{Xe,Ye}
passing near \meta{Xm,Ym}. If the optional argument is present exactly
\meta{num} segments will be used in drawing the curve.}
    The command will draw what geometers call a \emph{quadratic Bezier
curve}\index{Bezier curve|seealso{picture object}} from the 
point \parg{Xs, Ys} to the point \parg{Xe, Ye}. The
curve will pass somewhere near to the point \parg{Xm, Ym}. 

    Bezier curves are named after Pierre Bezier\index{Bezier, Pierre}
who first used them in 1962. They are widely used in Computer Aided 
Design (CAD)
programs and other graphics and font design systems. Descriptions, with
varying degrees of mathematical complexity, can be found in many places:
when I was a practicing geometer these included \cite{FAUX80},
\cite{MORTENSON85} and \cite{FARIN90}; no doubt there are more recent 
sources available and there is a brief review in~\cite{BEZ123}.

Figure~\ref{lpicf:bez} shows two of these curves. The figure was
produced by the code below.

%\begin{egsource}{eg11.15}
\begin{lcode}
\begin{figure}
\setlength{\unitlength}{1mm}
\centering
\begin{picture}(100,100)

\thicklines % first curve
\qbezier(10,50)(50,90)(50,50)
\thinlines % draw lines joining control points
\put(10,50){\line(1,1){40}}
\put(50,90){\line(0,-1){40}}
% label control points
\put(10,45){\makebox(0,0)[t]{\texttt{(10,50)}}}
\put(50,95){\makebox(0,0)[b]{\texttt{(50,90)}}}
\put(55,50){\makebox(0,0)[l]{\texttt{(50,50)}}}

\thicklines % second curve
\qbezier[25](50,50)(50,10)(90,50)
\thinlines % draw lines joining control points
% \put(50,50){\line(0,-1){40}}
% \put(50,10){\line(1,1){40}}
% label control points
\put(50,5){\makebox(0,0)[t]{\texttt{(50,10)}}}
\put(90,55){\makebox(0,0)[b]{\texttt{(90,50)}}}

\end{picture}
\setlength{\unitlength}{1pt}
\caption{Two Bezier curves}
\label{lpicf:bez}
\end{figure}
\end{lcode}
%\end{egsource}

\begin{figure}
\setlength{\unitlength}{1mm}
\centering
\begin{picture}(100,100)

\thicklines % first curve
\qbezier(10,50)(50,90)(50,50)
\thinlines % draw lines joining control points
\put(10,50){\line(1,1){40}}
\put(50,90){\line(0,-1){40}}
% label control points
\put(10,45){\makebox(0,0)[t]{\texttt{(10,50)}}}
\put(50,95){\makebox(0,0)[b]{\texttt{(50,90)}}}
\put(55,50){\makebox(0,0)[l]{\texttt{(50,50)}}}

\thicklines % second curve
\qbezier[25](50,50)(50,10)(90,50)
\thinlines % draw lines joining control points
% \put(50,50){\line(0,-1){40}}
% \put(50,10){\line(1,1){40}}
% label control points
\put(50,5){\makebox(0,0)[t]{\texttt{(50,10)}}}
\put(90,55){\makebox(0,0)[b]{\texttt{(90,50)}}}

\end{picture}
\setlength{\unitlength}{1pt}
\caption{Two Bezier curves}
\label{lpicf:bez}
\end{figure}

    The three points used to specify the position and shape of the
Bezier curve are called 
\emph{control points}\index{Bezier curve!control points}. 
The curve starts at
the first control point and is tangent to the line joining the first
and second control points. The curve stops at the last control point
and is tangent to the line joining the last two control points.

    In \fref{lpicf:bez} the lines joining the control points for the
first curve have been drawn in. The locations of all the control points
for the two curves are labeled. 

The second Bezier curve is the same shape
as the first one, but rotated 180 degrees. The first control point
of this curve is the same as the last control point of the first curve.
This means that the two curves are joind at this point. The line, 
although it is not drawn,
connecting the first two control points of the second curve is in the
same direction as the line joining the last two control points of the
first curve. This means that the two curves are also tangent at the
point where they join. By stringing together several Bezier curves
you can draw quite complex curved shapes.

\begin{syntax}
\cmd{\qbeziermax} \\
\end{syntax}
\glossary(qbeziermax)
{\cs{qbeziermax}}{The maximum number of segments for drawing a Bezier curve.}
    The Bezier curves are actually drawn as a 
linearized\index{Bezier curve!linearized rendition} form
using a series of rectangular
blobs of ink. Left to itself, \ltx\ will attempt to pick the number
of blobs to give the smoothest looking curve, up to a maximum number.
(Each blob takes up space in \ltx 's internal memory, and it may run
out of space if too many are used in one picture.) The maximum number
of blobs per Bezier curve is set by the \cmd{\qbeziermax} command. This
can be adjusted with the \cmd{\renewcommand} command. For example: 
\begin{lcode}
\renewcommand{\qbeziermax}{250}
\end{lcode}
will set the maximum number of blobs to be 250.

    Another method of controlling the number of blobs is by
the optional \meta{num} argument to the \cmd{\qbezier} command.
If used, it must be a positive integer number which tells \ltx\
exactly how many blobs to use for the curve.


%#% extend
%#% extstart include latex-and-tex.tex

\svnidlong
{$Ignore: $}
{$LastChangedDate: 2014-03-31 11:34:44 +0200 (Mon, 31 Mar 2014) $}
{$LastChangedRevision: 480 $}
{$LastChangedBy: daleif $}


%%%%%%%%%%%%%%%%%%%%%%%%%%%%%%%%%%%%%%%%%%%%%%
\chapter{\ltx\ and \tx} \label{appendix:alltex}

%%%%%%%%%%%%%%%%%%%%%%%%%%%%%%%%%%%%%%%%%%%%%%

    Strictly speaking, \ltx\ is a set of macros built on top of 
the \tx\ program originally developed by 
Donald Knuth~\cite{TEXPROGRAM,TEXBOOK} 
in the early
1980's. \tx\ is undoubtedly one of the most robust computer programs
to date. 

    Leslie Lamport says that most \tx\ commands can be used with 
\ltx\ and lists those that cannot be used~\cite[Appendix E]{LAMPORT94}.
Apart from this he says nothing about any \tx\ commands. I have used
some \tx\ macros in the code examples and so I need to talk a little
bit about these.

    I like to think of the commands and
macros as falling into one of several groups.
\begin{itemize}

\item \tx\ primitives. These are the basic constructs of the \tx\ language.

\item \tx\ commands or macros. These are part of the plain \tx\ system 
      and are
      constructed from the \tx\ primitives.

\item \ltx\ kernel commands or macros. These are defined in the \ltx\ kernel
      and are based on plain \tx\ primitives or commands. In turn, some 
      higher level kernel macros are constructed from more basic aspects
      of the kernel. The kernel does redefine some of the plain \tx\ commands.
      

\item Class command. These are mainly built up on the kernel commands but
      may use some basic \tx.

\item Package commands. These are similar to the class commands but are
      less likely to directly use \tx\ macros.

\item User commands. Typically these are limited to the commands
      provided by the class and any packages that might be called for,
      but more experienced users will employ some kernel commands,
      like \cmd{\newcommand}, to make their authoring more efficient.
\end{itemize}

    Although \tx\ is designed as a language for typesetting it is
also a `Turing complete' 
language\index{Turing complete language}\index{Turing, Alan}
which means that it can perform any function that can be programmed in
any familiar programming language. For example, an interpreter for the
BASIC language has been written in \tx, but writing this kind of program
using \tx\ is something that only an expert\footnote{Probably also a masochist
with plenty of time.} might consider.

    Nevertheless, you may have to, or wish to, write a typesetting function
of your own. This chapter presents a few of the programming aspects that
you may need for this, such as performing some simple arithmetic or comparing
two lengths. For anything else you will have to read one or more of the 
\tx\ books or tutorials.



    In England witnesses at a trial have to swear to `Tell the truth, the
whole truth, and nothing but the truth'. I will try and tell the truth
about \tx\ but, to misquote Hamlet \linenumberfrequency{0}
\settowidth{\versewidth}{There are more things in heaven and TeX, Horatio}
\begin{verse}[\versewidth]
There are more things in heaven and \tx, Horatio, \\
Than are dreamt of in your philosophy.
\end{verse}

\section{The \tx\ process}

    As we are delving deeper than normal and because at the bottom
it is the \tx\ language that does all the work,  it is useful to 
have an idea of how \tx\ processes a source file to produce a 
\pixfile{dvi} file. It is all explained in detail by 
Knuth~\cite{TEXBOOK} and also perhaps more accessibly by 
Eijkhout~\cite{TEXBYTOPIC}; 
the following is a simplified description.
Basically there are four processes involved and the output from one 
process is the input to the following one.

\begin{description}
\item[Input] The input process, which Knuth terms the `eyes', reads
  the source file and converts what it sees into \emph{tokens}\index{token}.
  There are essentially two kinds of token. A token is either a single
  character such as a letter or a digit or a punctuation mark, or 
  a token is a control sequence. 
  A \emph{control sequence}\index{control sequence} consists of a backslash
  and either all the alphabetic characters immediately following it, or
  the single non-alphabetic following it. Control sequence is the general 
  term for what I have been calling a macro or a command.

\item[Expansion] The expansion processor is what Knuth calls `\tx's mouth'.
  In this process some of the tokens from the input processor are expanded.
  Expansion replaces a token by other tokens or sometimes by no token.
  The expandible tokens include macros, conditionals, and a number of
  \tx\ primitives. 

\item[Execution] The execution process is \tx's `stomach'. This handles
  all the tokens output by the expansion processor. Control sequences
  that are not expandible are termed \emph{executable}, and the execution
  processor executes the executable tokens. Character tokens are
  neither expandible nor executable. It handles any macro defintions
  and also builds horizontal, vertical and mathematical lists.

\item[Layout] The layout processor (\tx's `bowels') breaks horizontal 
  lists into paragraphs, mathematical lists into formulae, and 
  vertical lists into pages. The final output is the \pixfile{dvi} file.

\end{description}

    In spite of the sequential nature implied by this description the 
overall process includes some feedback from a later process to an 
earlier one which may affect what that does.

    It is probably the expansion processor that needs to be best understood.
Its input is a sequence of tokens from the input processor and its output
is a sequence of different tokens.

    In outline, the expansion processor takes each input token in turn
and sees if it is expandible; if it is not it simply passes it on to the
output. If the token is expandible then it is replaced by its expansion.
The most common expandible tokens are control sequences that have been 
defined as macros. If the macro takes no arguments then the macro's name
is replaced by its definition. If the macro takes arguments, sufficient 
tokens are collected to get the values of the arguments, and then the 
macro name is replaced by the definition. The expansion processor then
looks at the first token in the replacement, and if that is expandible
it expands that, and so on. 

    Nominally, the eventual output from the expansion
processor is a stream of non-expandible tokens. There are ways,
however of controlling whether or not the expansion processor will actually
expand an expandible token, and to control the order in which things
get expanded, but that is where things get rapidly complicated.

    The layout processor works something like this. Ignoring maths,
\tx\ stores what you type in two kinds of lists, vertical and horizontal.
As it reads your words it puts them one after another in a horizontal list.
At the end of a paragraph it stops the horizontal list and adds it to the
vertical list. At the beginning of the next paragraph it starts a new
horizontal list and adds the paragraph's words to it. And so on. This
results in a vertical list of horizontal lists of words, where each 
horizontal list contains the words of a paragraph.

    It then goes through each horizontal list in turn, breaking it up into
shorter horizontal lists, one for each line in the paragraph. These are
put into another vertical list, so conceptually there is a vertical list
of paragraphs, and each paragraph is a vertical list of lines, and each
line is a horizontal list of words, or alternatively one vertical list
of lines. Lastly it chops up the vertical list of lines into page sized 
chunks and outputs them a page at a time.

    \tx\ is designed to handle arbitrary sized inserts, like those for
maths, tables, sectional divisions and so forth, in an elegant manner. 
It does this by allowing
vertical spaces on a page to stretch and shrink a little so that the
actual height of the typeblock is constant. If a page consists only of
text with no widow or orphan then the vertical spacing is regular, otherwise
it is likely to vary to some extent. Generally speaking, \tx\ is not
designed to typeset on a fixed grid, but against this
other systems are not designed
to produce high quality typeset mathematics. Attempts have been made
to tweak \ltx\ to typeset on a fixed grid but as far as I know nobody
has been completely successful.


    \tx\ works somewhat more efficiently than I have described. Instead
of reading the whole document before breaking paragraphs into lines, it 
does the line breaking at the end of each paragraph. After each paragraph
it checks to see if it has enough material for a page, and outputs a page
whenever it is full. However, \tx\ is also a bit lazy. Once it has broken
a paragraph into lines it never looks at the paragraph again, except perhaps
to split it at a page break. If you want to change, say, the width of the
typeblock on a particular page, any paragraph that spills over from a
previous page will not be reset to match the new measure. This asynchronous
page breaking\index{page break!asynchronous} also has an unfortunate effect
if you are trying to put a note in say, the outside margin, as the outside 
is unknown until after the paragraph has been set, and so the note may end
up in the wrong margin.

%%%%%%%%%%%%%%%%%%%%%%%%%%%%%%%%%%%%%%%%%%%%%%%%%%%%%%%

\section{\ltx\ files} \label{sec:latexfiles}

    The \pixfile{aux} file is the way \ltx\ transfers information from one
run to the next and the process works roughly like this.
\begin{itemize}
\item The \pixfile{aux} file is read at the start of the \Ie{document}
      environment. If \cmd{\nofiles} has not been specified a 
      new empty \pixfile{aux} file is then created which has the side
      effect of destroying the original \pixfile{aux} file.
\item Within the \Ie{document} environment there may be macros that write
      information to the \pixfile{aux} file, such as the sectioning or
      captioning commands. However, these macros will not write their
      information if \cmd{\nofiles} has been specified.
\item At the end of the \Ie{document} environment the contents
      of the \pixfile{aux} file are read.
\end{itemize}
Under normal circumstances new output files are produced each time \ltx\
is run, but when \cmd{\nofiles} is specified only the \pixfile{dvi} and
\pixfile{log} files will be new --- any other files are unchanged.

    In the case of the sectioning commands these write macros into the
\pixfile{aux} file that in turn write information into a \pixfile{toc}
file, and the \cmd{\tableofcontents} command reads the \pixfile{toc}
file which contains the information for the Table of Contents. To make this
a bit more concrete, as \ltx\ processes a new document through the
first two runs, the following events occur.
\begin{enumerate}
\item Initially there is neither an \pixfile{aux} nor a \pixfile{toc} file.
      At the start of the \Ie{document} environment a new empty \pixfile{aux}
      file is created.
\item During the first run the \cmd{\tableofcontents} typesets the
      Contents heading and creates a new empty \pixfile{toc}
      file.

      During
      the run sectional commands write information into the new 
      \pixfile{aux} file. At the end of the \Ie{document} environment
      the \pixfile{aux} file
      is read. Contents information in the \pixfile{aux} file is written
      to the \pixfile{toc} file. Lastly all the output files are closed.

\item For the second run the \pixfile{aux} file from the previous run is 
      read at the start of the \Ie{document} environment; no information can
      be written to a \pixfile{toc} file because the \pixfile{toc} file
      is only made available by the \cmd{\tableofcontents} command.
      The \pixfile{aux} file from the previous run is closed and the new
      one for this run is created.

        This time the \cmd{\tableofcontents} reads \pixfile{toc} file
      that was created during the previous run which contains the typesetting
      instructions for the contents, and then starts a
      new \pixfile{toc} file.

         And so the process repeats itself.
\end{enumerate}

    The \pixfile{aux} file mechanism means that, except for the simplest 
of documents, \ltx\ has to be run at least
twice in order to have all the information to hand for typesetting. 
If sections are added or deleted, two runs are necessary afterwards 
to ensure that everything is up to date. Sometimes three, or even more, 
runs are necessary to guarantee that things are settled.



%%%%%%%%%%%%%%%%%%%%%%%%%%%%%%%%%%%%%%%%%%%%%%%%%%%%%%%



\section{Syntax}

    The \ltx\ syntax that you normally see is pretty regular. 
Mandatory arguments are enclosed in curly braces and optional
arguments are enclosed in square brackets. One exception to this
rule is in the \Ie{picture} environment where coordinate and direction
pairs are enclosed in parentheses.

    The \tx\ syntax is not regular in the above sense. For example, if in
\ltx\ you said
\begin{lcode}
\newcommand*{\cmd}[2]{#1 is no. #2 of}
\cmd{M}{13} the alphabet. % prints: M is no. 13 of the alphabet
\end{lcode}
Then in \tx\ you would say
\begin{lcode}
\def\cmd#1#2{#1 is no. #2 of}
\end{lcode}
and you could then use either of the following calls:
\begin{lcode}
\cmd M{13} the alphabet.  % prints: M is no. 13 of the alphabet
\cmd{M}{13} the alphabet. % prints: M is no. 13 of the alphabet
\end{lcode}

     A simplistic explanation of the first \tx\ call of \verb?\cmd? is as 
follows. A control sequence starts with a backslash, followed by either
a single character, or one or more of what \tx\ thinks of as letters
(normally the 52 lower- and upper-case alphabetic characters);
a space or any non-letter, therefore, ends a multiletter control
sequence. \tx\ and \ltx\ discard any spaces after a macro name. 
If the macro takes any arguments, and \verb?\cmd? takes two, \tx\ will
then start looking for the first argument. An argument is either
something enclosed in braces or a single token. In the example the first
token is the character `M', so that is the value of the first argument.
\tx\ then looks for the second argument, which is the `13' enclosed
in the braces. In the second example, both arguments are enclosed in braces.

    Here are some \tx\ variations.
\begin{lcode}
\cmd B{2} the alphabet. % prints: B is no. 2 of the alphabet.
\cmd B2 the alphabet.   % prints: B is no. 2 of the alphabet.
\cmd N14 the alphabet.  % prints: N is no. 1 of4 the alphabet.
\end{lcode}
The result of \verb?\cmd B{2}? is as expected. The results of \verb?\cmd B2?
and \verb?\cmd N14? should also be expected, and if not take a moment to
ponder why. The `B' and 'N' are the first arguments to \verb?\cmd? in the
two cases because a single character is a token. Having found the first
argument \tx\ looks for the second one, which again will be a token as
there are no braces. It will find `2' and `1' as the second arguments
and will then expand the \verb?\cmd? macro. In the case of \verb?\cmd B2? this
gives a reasonable result. In the case of \verb?\cmd N14?, \tx\ expands
\verb?\cmd N1? to produce `N is in position 1 of', then continues printing
the rest of the text, which is `4 the alphabet', hence the odd looking
result.


\section{\alltx{} commands} \label{sec:alltexcommands}

    I have used some \tx\ commands in the example code and it is now time
to describe these. Only enough explanation is given to cover my use of
them. Full explanations would require a doubling in the size of the book 
and a concomitant increase in the price, so for full details consult
the \textit{\txbook} which is the definitive source, or one of the \tx\ 
manuals listed in the Bibliography. I find \textit{\tx\ by Topic}
particularly helpful.

    I have also used \ltx\ commands that are not mentioned by
Lamport. \ltx\ uses a convention for command names; any name that
includes the \texttt{@} character is an `internal' command and may be 
subject to change, or even deletion. Normal commands are meant to be 
stable --- the code implementing them may change but their effect will 
remain unaltered. In the \ltx\ kernel, and in class and package files 
the character \texttt{@} is automatically
treated as a letter so it may be used as part of a command name. 
Anywhere else you have to use 
\cmd{\makeatletter} to make \texttt{@} be treated as a letter and 
\cmd{\makeatother} to make \texttt{@} revert to its other meaning.
So, if you are defining or modifying or directly using any command 
that includes an \texttt{@}\idxatincode\
sign then this must be done in either a \file{.sty} file or if in the 
document itself it must be surrounded by \cmd{\makeatletter} and 
\cmd{\makeatother}. 

    The implication is `don't use internal commands as they may be dangerous'.
Climbing rocks is also dangerous but there are rock climbers; the live ones
though don't try climbing Half Dome in Yosemite or the North Face of the
Eiger without having first gained experience on friendlier rocks.

The \ltx\ kernel is full of internal commands and a few are mentioned
in Lamport. There is no place where you can go to get explanations of all
the \ltx\ commands, but if you run \ltx\ on the \pixfile{source2e.tex} 
file which is in the standard \ltx\ distribution you will get the commented
kernel code. The index of the commands runs to about 40 double column pages.
Each class and package introduce new commands over and above those in the
kernel. 


\ltx\ includes \cmd{\newcommand}, \cmd{\providecommand} and 
\cmd{\renewcommand} as means of (re-)defining a command, but \tx\ 
provides only one method.
\begin{syntax}
\cmd{\def}\meta{cmd}\meta{arg-spec}\marg{text} \\
\end{syntax}
\cmd{\def} specifies that within the local group\index{group}
the command \verb?\cmd? is defined as \meta{text}, and any previous definitions
of \meta{cmd} within the group are overwritten. Neither the 
\meta{text} nor any arguments can include an end-of-paragraph.
The \ltx\ equivalent to \cmd{\def} is the pair of commands
\cmd{\providecommand*} followed by \cmd{\renewcommand*}.

    The \meta{arg-spec} is a list of the argument numbers 
(e.g., \verb?#1#2?)
 in sequential
order, the list ending at the `\{' starting the \meta{text}. Any
spaces or other characters in the argument list are significant. These
must appear in the actual argument list when the macro is used.

\begin{syntax}
\cmd{\long} \cmd{\global} \\
\cmd{\gdef}\meta{cmd}\meta{arg-spec}\marg{text} \\
\cmd{\edef}\meta{cmd}\meta{arg-spec}\marg{text} \\
\cmd{\xdef}\meta{cmd}\meta{arg-spec}\marg{text} \\
\end{syntax}
If you use the \cmd{\long} qualifier before \cmd{\def} (as \verb?\long\def...?)
then the \meta{text} and arguments may include paragraphs.
The \ltx\ version of this is the unstarred \cmd{\providecommand}
followed by \cmd{\renewcommand}.

    To make a command global instead of local to the current group, 
the \cmd{\global} qualifier can be used with \cmd{\def} 
(as \verb?\global\def...?) when defining it;
\cmd{\gdef} is provided as a shorthand for this common case.

    Normally any macros within the replacement \meta{text} of a command
defined by \cmd{\def} are expanded when the command is called. 
The macro \cmd{\edef} also defines a command but in this case any macros
in the replacement \meta{text} are expanded when the command is defined.
Both \cmd{\long} and \cmd{\global} may be used to qualify \cmd{\edef},
and like \cmd{\gdef} being shorthand for \verb?\global\def?, \cmd{\xdef}
is short for \verb?\global\edef?.

    There is much more to the \cmd{\def} family of commands than I have
given; consult elsewhere for all the gory details.

\begin{syntax}
\cmd{\let}\meta{cmda}=\meta{cmdb} \\
\end{syntax}
The \cmd{\let} macro gives \meta{cmda} the same definition as \meta{cmdb}
\emph{at the time the \cmd{\let} is called}. The \Itt{=} sign is optional.
\cmd{\let} is often used when you want to save the definition of a
command.

    Here is a short example of how some of \cmd{\def} and \cmd{\let} work.
\begin{lcode}
\def\name{Alf}
\let\fred = \name
  \name, \fred.         % prints Alf, Alf.
\def\name{Fred}
  \name, \fred.         % prints Fred, Alf.
\def\name{\fred red}
  \name, \fred.         % prints Alfred, Alf.
\end{lcode}

\begin{syntax}
\cmd{\csname} \meta{string}\cmd{\endcsname} \\
\end{syntax}
If you have ever tried to define commands like \verb?\cmd1?, \verb?\cmd2? you will
have found that it does not work. \tx\ command names consists of either
a single character or a name composed solely of what \tx\ thinks
of as alphabetic characters. However, the \cmd{\csname} \cmd{\endcsname}
pair turn the enclosed \meta{string} into the control sequence \verb?\string?,
which means that you can create \verb?\cmd1? by 
\begin{lcode}
\csname cmd1\endcsname
\end{lcode}
Note that the resulting \verb?\cmd1? is not defined (as a macro).

\begin{syntax}
\cmd{\@namedef}\marg{string} \\
\cmd{\@nameuse}\marg{string} \\
\end{syntax}
The kernel \cmd{\@namedef} macro expands to \verb?\def\<string>?, where 
\meta{string} can contain any characters. You can use this to
define commands that include non-alphabetic characters. There is 
the matching \cmd{\@nameuse} macro which expands to \verb?\<string>?
which then lets you use command names that include non-alphabetic
characters. For example:
\begin{lcode}
\@namedef{fred2}{Frederick~II}
...
\makeatletter\@nameuse{fred2}\makeatother reigned from ...
\end{lcode}

\begin{comment}

    \ltx\ lets you create lengths via \cmd{\newlength}.
\begin{syntax}
\cmd{\newdimen}\meta{cmd} \\
\cmd{\newskip}\meta{cmd} \\
\end{syntax}
\tx\ has two kinds of lengths, called \emph{dimension}\index{dimension}
and \emph{glue}\index{glue}. In \ltx\ these are called 
\emph{fixed length}\index{fixed length}\index{length!fixed}
and
\emph{rubber length}\index{rubber length}\index{length!rubber}
respectively.  A new dimension is created by \cmd{\newdimen} and a new
glue by \cmd{\newskip}. It so happens that \ltx's \cmd{\newlength}
always creates a new skip --- all lengths are created as rubber lengths.

    To set a length in \ltx\ you can use any of
several commands, but \tx\ is more parsimonious.
\begin{syntax}
\meta{dimen} = \meta{length} \\
\meta{skip} = \meta{length} plus \meta{length} minus \meta{length} \\
\end{syntax}
A dimension is set by giving it a length, as in
\begin{lcode}
\mydimen = 20pt % or as the = is optional
\mydimen 20pt
\end{lcode}
A glue is also set by giving it a length, which is possibly followed
by \Itt{plus} a length optionally followed by \Itt{minus} a length.  
\begin{lcode}
\myskip = 20pt plus 5pt minus 2pt
\end{lcode}
The actual length will normally be the first specified length, but
the value is allowed to be no shorter than the given length less 
the \Itt{minus} length. The value is allowed, but only very reluctantly,
to be greater than the given length plus the \Itt{plus} length.
As in setting a dimension value the \Itt{=} sign is optional. In the example,
\verb?\myskip? can vary anywhere between 18pt and 25pt, but may possibly be
strained to be greater than 25pt.

\begin{syntax}
\cmd{\@plus} \cmd{\@minus} \\
\end{syntax}
\ltx\ supplies \cmd{\@plus} and \cmd{\@minus} which expand to \Itt{plus}
and \Itt{minus} respectively. Writing \cmd{\@plus} instead of \Itt{plus}
uses one instead of four tokens, saving three tokens, 
and \cmd{\@minus} in place of \Itt{minus}
saves four tokens --- remember that a \tx\ token is either a control 
sequence (e.g. \cmd{\@minus}) or a single character (e.g., \verb?m?). 
\tx's memory is not infinite --- it can only hold so many tokens --- and
it makes sense for kernel and class or package writers to use fewer 
rather than more to leave sufficient space for any that authors might want
to create.

\begin{syntax}
\cmd{\z@} \\
\Itt{fil} \Itt{fill} \Itt{filll} \\
\end{syntax}
\cmd{\z@} is a very useful \ltx\ command when specifying lengths.
Depending on the context it either stands for the number 0 (zero)
or 0pt (zero length). \tx\ has three kinds of infinitely stretchy 
length units that can be used in the \Itt{plus} or \Itt{minus} 
parts of a skip.
\Itt{fil} is infinitely more flexible than any fixed amount, but
\Itt{fill} is infitely more flexible than \Itt{fil} and \Itt{filll}
is infinitely more flexible than anything else at all. These infinite
glues can be used to push things around.

\begin{syntax}
\cmd{\hskip}\meta{skip} \\
\cmd{\vskip}\meta{skip} \\
\end{syntax}
The \tx\ command \cmd{\hskip} inserts \meta{skip} horizontal space
and likewise \cmd{\vskip} inserts \meta{skip} vertical space.

\begin{syntax}
\cmd{\hfil} \cmd{\hfill} \cmd{\hfilneg} \cmd{\hss} \\
\end{syntax}
These commands are all \tx\ primitives and are equivalent to horizontal 
skips with some kind of infinite glue, as indicated below (note the use
of \Itt{fil} as a length unit, it being preceeded by a number):
\begin{lcode}
\hfil     -> \hskip 0pt plus 1fil
\hfill    -> \hskip 0pt plus 1fill
\hfilneg  -> \hskip 0pt           minus 1fil
\hss      -> \hskip 0pt plus 1fil minus 1fil
\end{lcode}

\begin{syntax}
\cmd{\vfil} \cmd{\vfill} \cmd{\vfilneg} \cmd{\vss} \\
\end{syntax}
These commands are all \tx\ primitives and are equivalent to vertical 
skips with some kind of infinite glue, as indicated below:
\begin{lcode}
\vfil     -> \vskip 0pt plus 1fil
\vfill    -> \vskip 0pt plus 1fill
\vfilneg  -> \vskip 0pt           minus 1fil
\vss      -> \vskip 0pt plus 1fil minus 1fil
\end{lcode}

\end{comment}

    At any point in its processing \tx\ is in one of six 
\emph{modes}\index{mode} which can be categorized into three groups:
\begin{enumerate}
\item horizontal and restricted 
      horizontal;\index{horizontal mode}\index{mode!horizontal}\index{mode!restricted horizontal}
\item vertical and internal 
      vertical;\index{vertical mode}\index{mode!vertical}\index{mode!internal vertical}
\item math and display 
      math.\index{math mode}\index{mode!math}\index{mode!display math}
\end{enumerate}
More simply, \tx\ is in either horizontal, or vertical, or math mode.
In horizontal mode \tx\ is typically building lines of text while in
vertical mode it is typically stacking things on top of each other, 
like the lines making up a paragraph. 
Math gets complicated, and who can do with more complications
at this stage of the game?

\begin{syntax}
\cmd{\hbox} to \meta{dimen}\marg{text} \cmd{\hb@xt@}\meta{dimen}\marg{text} \\
\cmd{\vbox} to \meta{dimen}\marg{text} \\
\end{syntax}
With \cmd{\hbox}, \meta{text} is put into a horizontal box, and similarly
\cmd{\vbox} puts \meta{text} into a vertical box. The sizes of the boxes
depend on the size of the \meta{text}. The optional
\Itt{to}~\meta{dimen} phrase sets the size of the box to the fixed
\meta{dimen} value. If the
\meta{text} does not fit neatly inside a fixed size box then \tx\
will report \Itt{overfull} or \Itt{underfull} warnings. \ltx\ supplies
the \cmd{\hb@xt@} command as a shorthand for \cmd{\hbox}~\Itt{to}.

    Inside a horizontal box \tx\ is in restricted horizontal 
mode\index{mode!horizontal}
which means that everything in the box is aligned horizontally.
Inside a vertical box \tx\ is in internal vertical 
mode\index{mode!vertical} and the contents are stacked up 
and aligned vertically.

\begin{syntax}
\cmd{\dp}\meta{box} \cmd{\ht}\meta{box} \cmd{\wd}\meta{box} \\
\end{syntax}
The depth, height and width of a box are returned by the macros
\cmd{\dp}, \cmd{\ht} and \cmd{\wd} respectively.

\begin{syntax}
\cmd{\leavevmode} \\
\end{syntax}
\tx\ may be in either vertical or horizontal mode and there are
things that can be done in one mode while \tx\ reports an eror if they
are attempted in the other mode. When typesetting a paragraph \tx\
is in horizontal mode. If \tx\ is in vertical mode, \cmd{\leavevmode} 
makes it switch to horizontal mode, but does nothing if \tx\ is already
in horizontal mode. It is often used to make sure that \tx\ is in horizontal
mode when it is unclear what state it might be in.

\begin{comment}

\tx\ has various \emph{conditional}\index{conditional}
constructs of the form:
\begin{lcode}
\if something-is-true
  do-true-stuff
\else  % something is not true (i.e. it is false)
  do-false-stuff
\fi
\end{lcode}
In any of these constructs the \verb?\else do-false-stuff? phrase is
optional, and so is the \verb?do-true-stuff?.

\begin{syntax}
\piif{ifodd} \meta{number} ... \piif{else} ...  \piif{fi} \\
\end{syntax}
The command \piif{ifodd} tests whether \meta{number} is an odd number (true)
or an even number (false).

\begin{syntax}
\piif{ifnum} \meta{numbera} \Itt{>} \meta{numberb} ... \piif{else} ... \piif{fi} \\
\piif{ifnum} \meta{numbera} \Itt{=} \meta{numberb} ... \piif{else} ... \piif{fi} \\
\piif{ifnum} \meta{numbera} \Itt{<} \meta{numberb} ... \piif{else} ... \piif{fi} \\
\end{syntax}
\piif{ifnum} tests if \meta{numbera} is greater than \Itt{>}, equal to
\Itt{=} or less than \Itt{<} \meta{numberb}.
There is a similar command, \piif{ifdim}, for comparing two lengths.

\begin{syntax}
\cmd{\newif}\meta{ifcmd} \\
\end{syntax}
\cmd{\newif} creates a new conditional\index{conditional}, \meta{ifcmd}.
The \meta{ifcmd} must start with the three 
characters \verb?\if?. Two other declarations are created at the same time.
These are called \verb?\cmdtrue? and \verb?\cmdfalse?. 
Using \verb?\cmdtrue? sets the result of \verb?\ifcmd? to be true and 
using \verb?\cmdfalse? sets the result of \verb?\ifcmd? to be false. 
At creation time, \verb?\cmdfalse? is declared
so \verb?\ifcmd? is initially false. Here is an example.
\begin{lcode}
\newif\ifpeter
...
\ifpeter
  My name is Peter.
\else
  Call me Ishmael.
\fi
\end{lcode}

\end{comment}

%%%%%%%%%%%%%%%%%%%%%%%%%%%%%%%%%%%%%%%%%%%%%%%
%%\input{program} % programming chapter \label{chap:program}

%% program.tex    (La)TeX programming


%%%%%%%%%%%%%%%%%%%%%%%%%%%%%%%%%%%%%%%%%%%%%%%%%%%%%%%%%%%%%%%
%%\chapter{Calculation and programming} \label{chap:program}
%%%%%%%%%%%%%%%%%%%%%%%%%%%%%%%%%%%%%%%%%%%%%%%%%%%%%%%%%%%%%%%

%%%%%%%%%%%%%%%%%%%%%%%%%%%%%%%%%
\makeatletter
\newcount\fib
\newcount\fibprev
\newcount\fibprevprev
\newcount\fibtogo

\newcommand*{\fibseries}[1]{%
  \fibprevprev=1\relax
  \fibprev=1\relax
  \ifnum #1>0\relax
    \@fibseries{#1}%
  \fi}

\newcommand*{\gfibseries}[3]{%
  \fibprevprev=#1\relax
  \fibprev=#2\relax
  \ifnum #3>0\relax
    \@fibseries{#3}%
  \fi}

\newcommand*{\@fibseries}[1]{%
  \fibtogo=#1\relax
  \ifnum \fibtogo=1\relax
    \the\fibprevprev
  \else
    \ifnum \fibtogo=2\relax
      \the\fibprevprev{} and \the\fibprev
    \else
      \advance\fibtogo by -2\relax
      \the\fibprevprev, \the\fibprev
      \loop
        \@fibnext
      \ifnum \fibtogo>0\relax
      \repeat
    \fi
  \fi}

\newcommand*{\@fibnext}{%
  \fib=\fibprev
  \advance\fib by \fibprevprev
  \fibprevprev=\fibprev
  \fibprev=\fib
  \printfibterm
  \advance\fibtogo by -1\relax}

\newcommand*{\printfibterm}{%
  \ifnum \fibtogo=1\relax
    \space and \else , \fi
  \the\fib}

\renewcommand*{\@fibseries}[1]{%
  \fibtogo=#1\relax
  \ifcase \fibtogo % ignore 0
  \or  % \fibtogo=1
    \the\fibprevprev
  \or  % \fibtogo=2
    \the\fibprevprev{} and \the\fibprev
  \else % fibtogo > 2
    \advance\fibtogo by -\tw@
    \the\fibprevprev, \the\fibprev
    \@whilenum \fibtogo > 0\do {% % !!! must kill space after the {
      \@fibnext}%
  \fi}

\makeatother
%%%%%%%%%%%%%%%%%%%%%%%%%%%%%


\section{Calculation}

    \ltx\ provides some methods for manipulating numbers and these, of course,
are composed from \tx's more basic methods. Sometimes it is
useful to know what \tx\ itself provides. We have met most, if not all,
of \ltx's macros earlier but I'll collect them all here for ease of reference.

\subsection{Numbers}

    In \ltx\ a counter\index{counter} is used for storing an integer number.
\begin{syntax}
\cmd{\newcounter}\marg{counter} \\
\cmd{\setcounter}\marg{counter}\marg{number} \\
\cmd{\stepcounter}\marg{counter} \cmd{\refstepcounter}\marg{counter} \\
\end{syntax}
A new counter called \meta{counter}, without a backslash, is created using
\cmd{\newcounter}. Its value can be set to a \meta{number} by the
\cmd{\setcounter} command and \cmd{\stepcounter} increases its value by one.
If the counter is to be used as the basis for a \cmd{\label}, its
value must be set using \cmd{\refstepcounter}, neither \cmd{\stepcounter}
nor \cmd{\setcounter} will work as expected in this case.

    Internally, a \ltx\ \emph{counter} is represented by a \tx\ 
\emph{count}\index{count} --- the \cmd{\newcounter} macro creates a
\tx\ count named \cs{c@}\meta{counter}, and the other \cs{...counter}
macros similarly operate on the \cs{c@}\meta{counter} count.


\begin{syntax}
\cmd{\newcount}\meta{count} \\
\end{syntax}
The \tx\ \cmd{\newcount} command creates a new count, \meta{count}, which 
\emph{does} include an initial backslash. For example
\begin{lcode}
\newcount\mycount
\end{lcode}
\tx's method of assigning a number to a count uses nothing like 
\cmd{\setcounter}.
\begin{syntax}
\meta{count} [ \texttt{=} ] \meta{number} \\
\end{syntax}
The [ and ] enclosing the \texttt{=} sign are there only
to indicate that the \texttt{=} sign is optional. For example:
\begin{lcode}
\mycount = -24\relax  % \mycount has the value -24 
\mycount 36\relax     % now \mycount has the value 36
\end{lcode}
I have added \cmd{\relax} after the digits forming the number for safety
and efficiency. When \tx\ is reading a number it keeps on looking until
it comes across something that is not part of a number. There are things 
that \tx\ will treat as part of a number which you might not think of, 
but \cmd{\relax} is definitely not part of a number. See, for example,
\cite[chapter 7]{TEXBYTOPIC} for all the intricate details if you need them.

    There are some numbers that are used many times in the \ltx\ kernel
and class codes. To save having to use \cmd{\relax} after such numbers,
and for other reasons of efficiency, there are commands that can be used 
instead of typing the digits. These are listed in \tref{tab:intmacnum}.
The command \cmd{\z@} can be used both for the number zero and for a
length of 0pt. Do not use the commands to print a number.

\newcolumntype{A}{>{\makeatletter}r<{\makeatother}}
\begin{table}
\centering
\caption{Some internal macros for numbers} \label{tab:intmacnum}
\begin{tabular}{lrclrclr} \toprule
\cmd{\m@ne} & \makeatletter\the\m@ne\makeatother & &
%\cmd{\z@}   & \makeatletter\strip@pt\the\z@\makeatother & & 
\cmd{\z@}   & 0 & & 
\cmd{\@ne}  & \makeatletter\the\@ne\makeatother \\
\cmd{\tw@}  & \makeatletter\the\tw@\makeatother  & & 
\cmd{\thr@@} & \makeatletter\the\thr@@\makeatother & & 
\cmd{\sixt@@n} & \makeatletter\the\sixt@@n\makeatother \\
\cmd{\@xxxii} & \makeatletter\the\@xxxii\makeatother & &
\cmd{\@cclv} & \makeatletter\the\@cclv\makeatother & &
\cmd{\@cclvi} & \makeatletter\the\@cclvi\makeatother \\
\cmd{\@m} & \makeatletter\the\@m\makeatother & &
\cmd{\@Mi} & \makeatletter\the\@Mi\makeatother & &
\cmd{\@Mii} & \makeatletter\the\@Mii\makeatother \\
\cmd{\@Miii} & \makeatletter\the\@Miii\makeatother & &
\cmd{\@Miv} & \makeatletter\the\@Miv\makeatother & &
\cmd{\@MM} & \makeatletter\the\@MM\makeatother  \\
\bottomrule
\end{tabular}
\end{table}


\tx\ has a limited vocabulary for arithmetic. It can add to a count,
and can multiply and divide a count, but only by integers. The result
is always an integer. This may be disconcerting after a division where
any remainder is discarded.
The syntax for these operations is:
\begin{syntax}
\cmd{\advance}\meta{count} [ \pixkey{by} ] \meta{number} \\
\cmd{\multiply}\meta{count} [ \pixkey{by} ] \meta{number} \\
\cmd{\divide}\meta{count} [ \pixkey{by} ] \meta{number} \\
\end{syntax}
The \pixkey{by} is a \tx\ keyword and the brackets are 
just there
to indicate that it can be missed out. Some examples:
\begin{lcode}
\advance\mycount by -\mycount  % \mycount is now 0
\mycount = 15\relax            % \mycount is now 15
\divide\mycount by 4\relax     % \mycount is now 3
\multiply\mycount 4\relax      % \mycount is now 12
\advance\mycount by \yourcount % \mycount is now \yourcount + 12
\end{lcode}

The value of a count can be typeset by prepending the count by the \cmd{\the}
command, e.g., \verb?\the\mycount?.


\subsection{Lengths}

    Every length\index{length} has an associated unit. For convenience I'll use 
`\textit{dimension}'\index{dimension} as shorthand for a number and a length 
unit.
\begin{syntax}
\textit{dimension}: \meta{number}\meta{length-unit} \\
\end{syntax}
For example, a \textit{dimension} may be \texttt{10pt}, or \texttt{23mm}, 
or \texttt{1.3pc}.

    Unlike \ltx, \tx\ distinguishes two kinds of lengths. A \tx\
\cmd{\dimen} is a length that is fixed; in \ltx's terms it is a
\emph{rigid}\index{length!rigid} length. On the other hand a \tx\ \cmd{\skip}
is a length that may stretch or shrink a little; it is what \ltx\ calls 
a \emph{rubber}\index{length!rubber} length.
\begin{syntax}
\cmd{\newdimen}\meta{dimen} \cmd{\newskip}\meta{skip} \\
\end{syntax}
The \tx\ macros \cmd{\newdimen} and \cmd{\newskip} are used for creating
a new \meta{dimen} or a new \meta{skip}. For instance:
\begin{lcode}
\newdimen\mydimen
\newskip\myskip
\end{lcode}
The value of a \cmd{\dimen} is a \textit{dimension} and the value of a 
\cmd{\skip} is what \tx\ calls \textit{glue}\index{glue}. It
so happens that \ltx's \cmd{\newlength} always creates a new skip ---
all \ltx\ lengths are created as rubber\index{length!rubber} lengths. Glue
has at least one and possibly as many as three parts.
\begin{syntax}
glue: \textit{dimension} [ \pixkey{plus} \textit{dimension} ] [ \pixkey{minus} \textit{dimension} ] \\
\end{syntax}
The optional \pixkey{plus} part is the amount that the glue can 
stretch from its normal size and the optional \pixkey{minus} part 
is the amount the glue can shrink below its normal size. 
Both \pixkey{plus} and \pixkey{minus} are \tx\ keywords.
Glue can never shrink more than the \pixkey{minus} 
\textit{dimension} and it normally does not stretch more than the
\pixkey{plus} \textit{dimension}. 

\begin{syntax}
\cmd{\@plus} \cmd{\@minus} \\
\end{syntax}
\ltx\ supplies \cmd{\@plus} and \cmd{\@minus} which expand to \Itt{plus}
and \Itt{minus} respectively. Writing \cmd{\@plus} instead of \Itt{plus}
uses one instead of four tokens, saving three tokens, 
and \cmd{\@minus} in place of \Itt{minus}
saves four tokens --- remember that a \tx\ token is either a control 
sequence (e.g. \cmd{\@minus}) or a single character (e.g., \verb?m?). 
\tx's memory is not infinite --- it can only hold so many tokens --- and
it makes sense for kernel and class or package writers to use fewer 
rather than more to leave sufficient space for any that authors might want
to create.

In \tx, assigning a value to a length (\cmd{\dimen} or \cmd{\skip}) is 
rather different from the way it would be done in \ltx.
\begin{syntax}
\meta{dimen} [ \texttt{=} ] \meta{dimension} \\
\meta{skip} [ \texttt{=} ] \meta{glue} \\
\end{syntax}
The [ and ] enclosing the \texttt{=} sign are there only
to indicate that the \texttt{=} sign is optional. For example:
\begin{lcode}
\newdimen\mydimen
\mydimen = 3pt    % \mydimen has the value 3pt
\mydimen   -13pt  % now \mydimen has the value -13pt
\myskip = 10pt plus 3pt minus 2pt % \myskip can vary between
                                  % 8pt and 13pt (or more)
\myskip = 10pt plus 3pt           % \myskip can vary between
                                  % 10pt and 13pt (or more)
\myskip = 10pt minus 2pt          % \myskip can vary between
                                  % 8pt and 10pt 
\myskip = 10pt                    % \myskip is fixed at 10pt
\end{lcode}

    Like counts, the value of a length can be typeset by prepending the 
length by the \cmd{\the} command, e.g., \verb?\the\myskip?.

    \tx's lengths can be manipulated in the same way as a count, using the
\cmd{\advance}, \cmd{\multiply} and \cmd{\divide} macros. Ignoring some 
details, lengths can be added together but may only be multiplied or divided
by an integer number. 


\newdimen\Wdimen \newskip\Wskip

\begin{center}
\begin{tabular}{l}
$\rhd$  \verb?\Wdimen = 10pt? $\Rightarrow$ \\
\multicolumn{1}{r}{\global\Wdimen = 10pt \texttt{Wdimen} = \the\Wdimen} \\
$\rhd$ \verb?\Wskip = 15pt plus 5pt minus 3pt? $\Rightarrow$ \\
\multicolumn{1}{r}{ \global\Wskip = 15pt plus 5pt minus 3pt \texttt{Wskip} = \the\Wskip} \\
$\rhd$ \verb?\advance\Wskip by \Wskip? $\Rightarrow$ \\
\multicolumn{1}{r}{\global\advance\Wskip by \Wskip  \texttt{Wskip} = \the\Wskip} \\
$\rhd$  \verb?\multiply\Wskip by 3? $\Rightarrow$ \\
\multicolumn{1}{r}{\global\multiply\Wskip by 3 \texttt{Wskip} = \the\Wskip} \\
$\rhd$ \verb?\divide\Wskip by 17? $\Rightarrow$ \\
\multicolumn{1}{r}{\global\divide\Wskip by 17  \texttt{Wskip} = \the\Wskip} \\
$\rhd$ \verb?\advance\Wskip by \Wdimen? $\Rightarrow$ \\
\multicolumn{1}{r}{\global\advance\Wskip by \Wdimen  \texttt{Wskip} = \the\Wskip} \\
$\rhd$ \verb?\advance\Wdimen by \Wskip? $\Rightarrow$ \\
\multicolumn{1}{r}{\global\advance\Wdimen by \Wskip  \texttt{Wdimen} = \the\Wdimen} \\
\end{tabular}
\end{center}

    A length can be multiplied by a fractional number by prepending the
length with the number. For example:

\begin{center}
\begin{tabular}{l}
$\rhd$  \verb?\Wdimen = 0.5\Wdimen?  $\Rightarrow$ \\
  \multicolumn{1}{r}{\Wdimen = 0.5\Wdimen \texttt{Wdimen} = \the\Wdimen } \\
$\rhd$ \verb?\Wskip = 0.5\Wskip?  $\Rightarrow$ \\
\multicolumn{1}{r}{\Wskip = 0.5\Wskip  \texttt{Wskip} = \the\Wskip} \\
\end{tabular}
\end{center}

    When \cmd{\multiply} or \cmd{\divide} is applied to a \cmd{\skip}
all its parts are modified, both the fixed part and any elastic components.
However, if a \cmd{\skip} is multiplied by a fractional number then it
loses any elasticity it might have had. In the same vein, 
if a \cmd{\skip} is added to a \cmd{\dimen} any elasticity is lost. 
A \cmd{\skip} can be coerced into behaving like a \cmd{\dimen} but a
\cmd{\dimen} is always rigid. For example, typing \\
`\verb?\Wdimen = 10pt plus 2pt minus 1pt?' results in: 
`\Wdimen = 10pt plus 2pt minus 1pt'.

\begin{syntax}
\cmd{\newlength}\marg{len} \\
\end{syntax}
    \ltx's \cmd{\newlength} macro creates a new 
rubber length\index{length!rubber} (internally it uses \cmd{\newskip});
there is no \ltx\ specific macro to create a rigid length\index{length!rigid}
(i.e., a \cmd{\dimen}).

\ltx\ has a variety of macros for setting or changing its length values.
\begin{syntax}
\cmd{\setlength}\marg{len}\marg{glue} \\
\end{syntax}
The \ltx\ \cmd{\setlength} macro assigns the value \meta{glue} to the 
rubber length \meta{len}.
Some examples of this are: \newlength{\Wlen}

\begin{center}
\begin{tabular}{l}
$\rhd$ \verb?\setlength{\Wlen}{10pt}?  $\Rightarrow$ \\
\multicolumn{1}{r}{\setlength{\Wlen}{10pt} \texttt{Wlen} = \the\Wlen} \\
$\rhd$ \verb?\setlength{\Wlen}{10pt plus 2pt}?  $ \Rightarrow $ \\
\multicolumn{1}{r}{\setlength{\Wlen}{10pt plus 2pt} \texttt{Wlen} = \the\Wlen} \\
$\rhd$ \verb?\setlength{\Wlen}{10pt minus 1pt}?  $ \Rightarrow $ \\ 
\multicolumn{1}{r}{\setlength{\Wlen}{10pt minus 1pt} \texttt{Wlen} = \the\Wlen} \\
$\rhd$ \verb?\setlength{\Wlen}{10mm plus 2pt minus 1pt}?  $ \Rightarrow $ \\
 \multicolumn{1}{r}{\setlength{\Wlen}{10mm plus 2pt minus 1pt} \texttt{Wlen} = \the\Wlen} \\
\end{tabular}
\end{center}

    As shown in the last example above where both mm and pt are used as a 
length unit, the \cmd{\the} applied to a length always prints the value
in pt units.

\begin{syntax}
\cmd{\settowidth}\marg{len}\marg{text} \\
\cmd{\settoheight}\marg{len}\marg{text} \\
\cmd{\settodepth}\marg{len}\marg{text} \\
\end{syntax}
These put the \meta{text} into a box and then set the \meta{len} to the
width, height and depth respectively of the box.

\begin{syntax}
\cmd{\addtolength}\marg{len}\marg{glue} \\
\end{syntax}
\ltx's \cmd{\addtolength} macro is the equivalent of \tx's \cmd{\advance}
command. There are no equivalents to \tx's \cmd{\multiply} or \cmd{\divide}
but in any case a length can still be multiplied by prepending it with 
a fractional number.


\begin{syntax}
\cmd{\z@} \\
\Itt{fil} \Itt{fill} \Itt{filll} \\
\end{syntax}
\cmd{\z@} is a very useful \ltx\ command when specifying lengths.
Depending on the context it either stands for the number 0 (zero)
or 0pt (zero length). \tx\ has three kinds of infinitely stretchy 
length units that can be used in the \Itt{plus} or \Itt{minus} 
parts of a skip.
\Itt{fil} is infinitely more flexible than any fixed amount, but
\Itt{fill} is infitely more flexible than \Itt{fil} and \Itt{filll}
is infinitely more flexible than anything else at all. These infinite
glues can be used to push things around.

\begin{syntax}
\cmd{\hskip}\meta{skip} \\
\cmd{\vskip}\meta{skip} \\
\end{syntax}
The \tx\ command \cmd{\hskip} inserts \meta{skip} horizontal space
and likewise \cmd{\vskip} inserts \meta{skip} vertical space.

\begin{syntax}
\cmd{\hfil} \cmd{\hfill} \cmd{\hfilneg} \cmd{\hss} \\
\end{syntax}
These commands are all \tx\ primitives and are equivalent to horizontal 
skips with some kind of infinite glue, as indicated below (note the use
of \Itt{fil} as a length unit, it being preceeded by a number):
\begin{lcode}
\hfil     -> \hskip 0pt plus 1fil
\hfill    -> \hskip 0pt plus 1fill
\hfilneg  -> \hskip 0pt           minus 1fil
\hss      -> \hskip 0pt plus 1fil minus 1fil
\end{lcode}

\begin{syntax}
\cmd{\vfil} \cmd{\vfill} \cmd{\vfilneg} \cmd{\vss} \\
\end{syntax}
These commands are all \tx\ primitives and are equivalent to vertical 
skips with some kind of infinite glue, as indicated below:
\begin{lcode}
\vfil     -> \vskip 0pt plus 1fil
\vfill    -> \vskip 0pt plus 1fill
\vfilneg  -> \vskip 0pt           minus 1fil
\vss      -> \vskip 0pt plus 1fil minus 1fil
\end{lcode}



\section{Programming}

    One of the commonest programming operations is to possibly do one thing if
something is true and to possibly do another thing if it is not true. Generally
speaking, this is called an `if-then-else'\index{if-then-else} or
\emph{conditional}\index{conditional} statement.

\begin{syntax}
\cs{if...} \meta{test} \meta{true-text} [ \piif{else} \meta{false-text} ] \piif{fi} \\
\end{syntax}
\tx\ has several kinds of `if-then-else' statements which have the general
form shown above. The statement starts with an \cs{if...} and is finished
by a matching \piif{fi}. As usual, the brackets enclose optional elements, 
so there need be no \cs{else} portion. The \meta{true-text}, it it exists,
is processed if the \meta{test} is \ptrue\ otherwise the 
\meta{false-text}, if both the \piif{else} clause and \meta{false-text}
are present, is processed.

The simplest kind of \cs{if...}
is defined by the \cmd{\newif} macro.
\begin{syntax}
\cmd{\newif}\verb?\if?\meta{name} \\
\end{syntax}
\cmd{\newif} creates three new commands, the \cs{ifname} and the
two declarations, \cs{nametrue} and \cs{namefalse}, for setting the value
of \cs{ifname} to \ptrue\ or \pfalse\ respectively.
In this case the \meta{test} is embedded in the \cs{if...}.
For example:
\begin{lcode}
\newif\ifpeter
...
\ifpeter
  My name is Peter.
\else
  Call me Ishmael.
\fi
\end{lcode}
or a more likely scenario is
\begin{lcode}
\newif\ifmine
  \minetrue % or \minefalse
\newcommand{\whose}{%
  \ifmine It's mine. \else I don't know whose it is. \fi}
\end{lcode}

Here are some of the other more commonly used kinds of ifs.
\begin{syntax}
\piif{ifdim} \meta{dimen1} \meta{rel} \meta{dimen2} \\
\piif{ifnum} \meta{number1} \meta{rel} \meta{number2} \\
\piif{ifodd} \meta{number} \\
\end{syntax}
The \meta{rel} in \piif{ifnum} and \piif{ifdim} is one of the three characters:
\texttt{<} (less than), \texttt{=} (equals), or \texttt{>} (greater than).
\piif{ifdim} results in \ptrue\ if the two lengths are in the stated
relationship otherwise it results in \pfalse. Similarly \piif{ifnum}
is for the comparison of two integers. The \piif{ifodd} test is \ptrue\
if the integer \meta{number} is an odd number, otherwise it results 
in \pfalse.

    Among other things, the \ltx\ class code that organizes the page layout
checks if the length values are sensible. The following code is a snippet
from the layout algorithm. It checks that the sum of the margins and the
width of the typeblock is the same as the width of the page after trimming.
\cmd{\@tempdima} and \cmd{\@tempdimb} are two `scratch' lengths used in many
calculations.
\begin{lcode}
\@tempdimb= -1pt              % allow a difference of 1pt
\@tempdima=\paperwidth              % paperwidth
\advance\@tempdima by -\foremargin  % minus the foremargin
\advance\@tempdima -\textwidth      % minus the textwidth
\advance\@tempdima -\spinemargin    % minus the spinemargin
\ifdim\@tempdima < \@tempdimb       % should be close to zero
  %% error                          % otherwise a problem
\fi
\end{lcode}

    Changing the subject, on the offchance that you might want to see 
how the Fibonacci sequence
progresses, the first thirty numbers in the sequence are: 
\fibseries{30}.
I got \ltx\ to calculate those numbers for me, and it could have 
calculated many more. They were produced by just saying \verb?\fibseries{30}?.
The French mathematician 
\'{E}douard Lucas\index{Lucas, Edouard?Lucas, \'{E}douard} 
(1842--1891) studied sequences
like this and was the one to give it the name Fibonacci. Lucas also
invented the game called the Tower of Hanoi with  
Henri de Parville\protect\index{Parville, Henri?de Parville, Henri} (1838--1909), 
supplying the accompanying fable~\cite{PARVILLE84,ROUSEBALL}:
\begin{quotation}
In the great temple at Benares beneath the dome that marks the center of
the world, rests a brass plate in which are fixed three diamond needles, 
each a cubit high and as thick as the body of a bee. On one of these
needles, at the creation, God placed sixty-four discs of pure gold, the 
largest disc resting on the brass plate, and the others getting smaller 
and smaller up to the top one. This is the tower of Bramah. Day and night
unceasingly the priests transfer the discs from one diamond needle to
another according to the fixed and immutable laws of Bramah, which require
that the priest on duty must not move more than one disc at a time and
that he must place this disc on a needle so that there is no smaller disc
below. When the sixty-four discs shall have been thus transferred from the
needle which at creation God placed them, to one of the other needles, 
tower, temple, and Brahmins alike will crumble into dust and with a
thunderclap the world will vanish.
\end{quotation}

    The number of separate transfers of single discs is $2^{64} - 1$
or just under eighteen and a half million million moves, give or take a few,
to move the pile. At the rate of one disc per second, with no mistakes,
it would take more than 58 million million years before we would have to 
start being concerned.

    In his turn, Lucas has a number sequence named after him. There are many 
relationships between the Fibonacci 
numbers $F_{n}$ and the Lucas numbers $L_{n}$, the simplest, perhaps, being
\begin{eqnarray}
L_{n} & = & F_{n-1} + F_{n+1} \\
5F_{n} & = & L_{n-1} + L_{n+1}
\end{eqnarray}
    The first 15 numbers in the Lucas sequence are:
\gfibseries{2}{1}{15}. These were produced by saying 
\verb?\gfibseries{2}{1}{15}?. The Lucas numbers are produced in the same manner
as the Fibonacci numbers, it's just the starting pairs that differ.

However, it is the definition of the \cmd{\fibseries} and \cmd{\gfibseries}
macros that might be more interesting in this context. 

    First, create four new counts. \cs{fibtogo} is the number of terms to be
calculated, \cs{fib} is the current term, and \cs{fibprev} and \cs{fibprevprev}
are the two prior terms.
\begin{lcode}
\newcount\fib
\newcount\fibprev
\newcount\fibprevprev
\newcount\fibtogo
\end{lcode}
The argument to \cmd{\fibseries} is the number of terms. The counts
\cs{fibprevprev} and \cs{fibprev} are set to the starting pair in the sequence.
Provided the number of terms requested is one or more the macro 
\cmd{\@fibseries} is called to do the work.
\begin{lcode}
\newcommand*{\fibseries}[1]{%
  \fibprevprev=1\relax
  \fibprev=1\relax
  \ifnum #1>0\relax
    \@fibseries{#1}%
  \fi}
\end{lcode}

The macro \cmd{\@fibseries} calculates and prints the terms. 
\begin{lcode}
\newcommand*{\@fibseries}[1]{%
  \fibtogo=#1\relax
\end{lcode}
It's simple if no more than two terms have been asked for --- just print
them out.
\begin{lcode}
  \ifnum \fibtogo=\@ne
    \the\fibprevprev
  \else
    \ifnum \fibtogo=\tw@
      \the\fibprevprev{} and \the\fibprev
    \else
\end{lcode}
Three or more terms have to be calculated. We reduce the number to be 
calculated by 2, and print the first two terms.
\begin{lcode}
      \advance\fibtogo by -\tw@
      \the\fibprevprev, \the\fibprev
\end{lcode}
We now have to calculate the rest of the terms, where each term is the sum of
the two previous terms. 
The macro \cmd{\@fibnext} calculates the next term, prints it out and reduces
the number of terms left to be calculated (\cmd{\fibtogo}) by one. 
If there are terms left to be done then the process is repeated until 
they have all been printed.
\begin{lcode}
      \loop
        \@fibnext
      \ifnum \fibtogo>\z@
      \repeat
    \fi
  \fi}
\end{lcode}

The \cmd{\@fibnext} macro calculates a term in the series, uses 
\cmd{\printfibterm} to print it, and decrements the \cmd{\fibtogo} count.
\begin{lcode}
\newcommand*{\@fibnext}{%
  \fib=\fibprev
  \advance\fib by \fibprevprev
  \fibprevprev=\fibprev
  \fibprev=\fib
  \printfibterm
  \advance\fibtogo \m@ne}
\end{lcode}

The last of the macros, \cmd{\printfibterm}, typesets a term in the sequence. 
If the term is the last one print an `and' otherwise print a `,', 
then a space and the term.
\begin{lcode}
\newcommand*{\printfibterm}{%
  \ifnum \fibtogo=\@ne \space and \else , \fi 
  \the\fib}
\end{lcode}

    You have met all of the macros used in this code except for \tx's
\piif{loop} construct. I find the syntax for this a little unusual.
\begin{syntax}
\piif{loop} \meta{text1} \cs{if...} \meta{text2} \piif{repeat} \\
\end{syntax} 
The construct starts with \piif{loop} and is ended by \piif{repeat};
the \cs{if...} is any conditional test, but without the closing \piif{fi}.
\tx\ processes \meta{text1}, then if the \cs{if...} is \ptrue\
it processes \meta{text2} and repeats the sequence again starting
with \meta{text1}. On the other hand, as soon as the result of the
\cs{if...} is 
\pfalse\ the loop stops (i.e., \tx\ jumps over \meta{text2}
and goes on to do whatever is after the \piif{repeat}).

    The \cmd{\gfibseries} macro that I used for the Lucas numbers is a
generalisation of \cmd{\fibseries}, where the first two arguments are the
starting pair for the sequence and the third argument is the number
of terms; so \verb?\gfibseries{1}{1}{...}? is equivalent to
\verb?\fibseries{...}?.
\begin{lcode}
\newcommand*{\gfibseries}[3]{%
  \fibprevprev=#1\relax
  \fibprev=#2\relax
  \ifnum #3>0\relax
    \@fibseries{#3}%
  \fi}
\end{lcode}
    The calculation of the terms in the Fibonacci and in the generalised 
sequences is the same so \cmd{\@fibseries} can be used again.

    I used the \tx\ \piif{loop} construct in the \cmd{\@fibseries} macro
but \ltx\ has a similar construct.
\begin{syntax}
\cmd{\@whilenum} \meta{ifnum test} \cmd{\do} \marg{body} \\
\cmd{\@whiledim} \meta{ifdim test} \cmd{\do} \marg{body} \\
\end{syntax}
As long as the appropriate \meta{test} is \ptrue\ the \meta{body} is processed.

    In \cmd{\@fibseries} I used \cs{ifnum}s to check for 3 possible values.
There is another \cs{if...} form that can be used for this type of work.
\begin{syntax}
\piif{ifcase} \meta{number} \meta{text for 0} \piif{or} \meta{text for 1} 
  \piif{or} \meta{text for 2} \\
... \\
\piif{or} \meta{text for N} [ \piif{else} \meta{text for anything else} ] \piif{fi} \\
\end{syntax}
If the \meta{number} is 0 then \meta{text for 0} is processed, but if
\meta{number} is 1 then \meta{text for 1} is processed, but if \meta{number}
is \ldots Each \meta{text for ...} is separated by an \piif{or}. If \meta{number}
is anything other than the specified cases (i.e., less than zero or greater
than N) then if the \piif{else} is present \meta{text for anything else} is
processed. 

Here's another version of the \cmd{\@fibseries} macro using \piif{ifcase}
and \cmd{\@whilenum}.
\begin{lcode}
\renewcommand*{\@fibseries}[1]{%
  \fibtogo=#1\relax
  \ifcase \fibtogo % ignore 0
  \or  % \fibtogo=1
    \the\fibprevprev
  \or  % \fibtogo=2
    \the\fibprevprev{} and \the\fibprev
  \else % fibtogo > 2
    \advance\fibtogo by -\tw@
    \the\fibprevprev, \the\fibprev
    \@whilenum \fibtogo > \z@ \do {% must kill space after the {
      \@fibnext}%
  \fi}
\end{lcode}

    \tx\ has more programing constructs than I have shown here and these
will be explained in any good \tx\ book. \ltx\ also has more than I have shown
but in this case the best place to look for further information is in the
\ltx\ kernel code, for example in \file{ltcntrl.dtx}.

%%%%%%%%%%%%%%%%%%%%%%%%%%%%%%%%%%%%%
%%%%%%%\endinput
%%%%%%%%%%%%%%%%%%%%%%%%%%%%%%%%%%%%


%#% extend
%#% extstart include the-terrors-of-errors.tex

\svnidlong
{$Ignore: $}
{$LastChangedDate: 2014-03-31 11:34:44 +0200 (Mon, 31 Mar 2014) $}
{$LastChangedRevision: 480 $}
{$LastChangedBy: daleif $}


%%%%%%%%%%%%%%%%%%%%%%%%%%%%%%%%%%%%%%%%%%%%%%%%%%%%%%%%%%%%%

%%\input{merrors} % errors chapter \label{chap:errors}

% merrors.tex    Chapter on (La)TeX errors/warnings


\chapter{The terrors of errors} \label{chap:errors}

    No matter how conscientious you are a mistake or two will occasionally
creep into your document source. The good news is that whatever happens
\tx\ will not destroy your files --- it may produce some odd looking output,
or even no output at all, but your work is safe. The bad news is that you
have to correct any errors that \tx\ finds. To assist you in this \tx\ stops
whenever it comes across what it thinks is an error\index{error} 
and tells you about it.
If you're not sure what to do it will also provide some possibly helpful 
advice.

    \tx\ underlies \ltx\ which underlies classes and packages. You may get
messages than originate from \tx, or from \ltx, or from the class and any
packages you may be using. I'll describe the \tx, \ltx, and class messages
below.

    In general, you will see a message\index{error message!response} 
on your terminal and \ltx\ will
stop and wait for you to respond. It prints a question mark and is 
expecting you to type one of the following:
\begin{itemize}
\item \meta{return} (or \meta{enter} or what is the equivalent on your
       keyboard): \ltx\ will continue\index{error message!response!continue} 
      processing the document.
\item \texttt{H} (help): the help\index{error message!response!help}  
      message is output and \ltx\
      waits for you to respond again.
\item \texttt{S} (scroll): Continue\index{error message!response!scroll}  
      processing, outputting any
      further error messages, but not stopping.
\item \texttt{Q} (quiet): Continue\index{error message!response!quiet}
      processing without stopping
      and with no further messages.
\item \texttt{R} (run): Like\index{error message!response!run}   
      the \texttt{Q} option but not even
      stopping if your document requires some user input.
\item \texttt{I} (insert): To insert\index{error message!response!insert}
      some material for \tx\ to
      read but no changes are made to the source file.
\item \texttt{E} (edit): This\index{error message!response!edit}   
      may return you to an editor so you can
      change the file. What actually happens is system dependent.
\item \texttt{X} (exit): Stop\index{error message!response!exit}   
      this \ltx\ run.
\end{itemize}
On the system I am used to the case of the characters does not matter.
I must admit that the only ones I have used are \meta{return}, \texttt{q},
\texttt{h} and \texttt{x}, in approximately that order of frequency.

    All messages are output to the \pixfile{log} file so you can study
them later if you need to.

\section{\tx\ messages}

\index{error!TeX?\tx|(}
\index{TeX?\tx!error|(}
\index{warning!TeX?\tx|(}
\index{TeX?\tx!warning|(}

    The following is an alphabetical list of some of \tx's messages,
abbreviated in some cases, together
with their corresponding remarks. As an example of how these appear on your
terminal, if you had a line in your source that read: \\
\verb?resulting in $x^3^4$.?\\
then \tx\ would output this:
\begin{lcode}
! Double superscript
l.102 resulting in $x^3^
                        4^$.
?
\end{lcode}
If you typed \texttt{h} in response to this you would then see:
\begin{lcode}
I treat `x^1^2' essentially like `x^1{}^2'.
\end{lcode}

    \tx's messages start with \verb?!? followed by the particular message 
text. The second line starts \verb?l.?
and a number, which is the number of the line in your file where the error
is. This is followed by the text of the line itself up to the point where
the error was detected, and the next line in the report shows the rest of
the erroneous line. The last line of the report is a \verb+?+ and \tx\
awaits your response.


%% List of TeX messages
\newcommand{\textmess}[1]{\texttt{#1}}
\newcommand{\texthelp}[1]{\textit{#1}}

    In the listing I have used \textmess{this font for the error message}
and \texthelp{this font for the comment message}.
\vspace{\onelineskip}

\begin{plainlist}
%1084 
\item[\textmess{!}]\index{A box was supposed to be here} 
     \textmess{A box was supposed to be here.} \\
     \texthelp{I was expecting to see \cmd{\hbox}{} or \cmd{\vbox}{} or 
            \cmd{\copy}{} or \cmd{\box}{} or something like that. 
             So you might find
             something missing in your output. But keep trying; you can 
             fix this later.}

%395 
\item[\textmess{!}]\index{Argument of ... has an extra \rb} 
    \textmess{Argument of ... has an extra \}.} \\
    \texthelp{I've run across a `\}' that doesn't seem to match anything.
            For example,} \verb?`\def\a#1{...}'? \texthelp{and} \verb?`\a}'?
           \texthelp{would produce this error. If you simply proceed now, 
            the \piif{par} that I've just inserted will cause me to report 
            a runaway argument that might be the root of the problem. But if
            your `\}' was spurious, just type `2' and it will go away.} 

      In \ltx\ terms, the example can be translated into \\
      \verb?`\newcommand{\a}[1]{...}'? and \verb?`\a}'?.

    If you can't find the extra \} it might be that you have used a fragile
command\index{fragile} in a moving\index{moving argument} argument. 
Footnotes\index{footnote!in heading}\index{footnote!in caption} 
or math\index{math!in caption or title} in division titles 
or captions
are a fruitful source for this kind of error. You shouldn't be putting 
footnotes into titles that will get listed in the \toc. For maths, put 
\cmd{\protect} before each fragile command.

%1236 
\item[\textmess{!}]\index{Arithmetic overflow}
     \textmess{Arithmetic overflow.} \\
     \texthelp{I can't carry out that multiplication or division,
       since the result is out of range.} 

    The maximum\index{maximum number} number that \tx\
       can deal with is 2,147,483,647 and it balks at 
    dividing\index{divide by zero} by zero.


%168 \texttt{AVAIL list clobbered at \ldots}

%293/4 \texttt{BAD.}

%961 \texttt{Bad \cmd{\patterns}}
%    \texttt{(See Appendix H.)}

%1244 
%\item[\textmess{!}] \textmess{Bad \cmd{\prevgraf}.} \\
%     \texthelp{I allow only nonnegative values here.}

%432 
%\item[\textmess{!}] \textmess{Bad character code} \\
%      \texthelp{The numeric code for a character must be between 0 and 127.
%              I changed this one to zero.}

%435 
%\item[\textmess{!}] \textmess{Bad character code} \\
%      \texthelp{A character number must be between 0 and 255.
%              I changed this one to zero.}

%437 
%\item[\textmess{!}] \textmess{Bad delimter code} \\
%    \texthelp{A numeric delimeter code must be between 0 and} \verb?2^{27}-1.?
%    \texthelp{I changed this one to zero.}

%170 \texttt{Bad flag at}

%170 \texttt{Bad link, display aborted.}

%436 
%\item[\textmess{!}] \textmess{Bad math code} \\
%    \texthelp{A numeric math code must be between 0 and 32767.
%            I changed this one to zero.}

%434 
%\item[\textmess{!}] \textmess{Bad number.} \\
%    \texthelp{Since I expected to read a number between 0 and 15,
%            I changed this one to zero.}

%433 
%\item[\textmess{!}] \textmess{Bad register code} \\
%    \texthelp{A register number must be between 0 and 255.
%            I changed this one to zero.}

%1243 
%\item[\textmess{!}] \textmess{Bad space factor} \\
%     \texthelp{I allow only values in the range 1..32767 here.}

%1328 \texttt{Beginning to dump on file \ldots}

%484 \texttt{*** (cannot \cmd{\read} from terminal in nonstop modes)} fatal error

%638 \texttt{Completed box being shipped out}

%639 \texttt{Memory usage before: }

%460 
\item[\textmess{!}]\index{Dimension too large}
    \textmess{Dimension too large.} \\
    \texthelp{I can't work with sizes\index{maximum length} bigger 
           than about 19 feet.
            Continue and I'll use the largest value I can.}

%1120 
%\item[\textmess{!}] \textmess{Discretionary list is too long.} \\
%     \texthelp{Wow---I never thought anyone would tweak me here.
%             You can't seriously need such a huge discretionary list?}



%1197 
\item[\textmess{!}]\index{Display math should end with \$\$}
     \textmess{Display math should end with \$\$.} \\
     \texthelp{The `\$' that I just saw supposedly matches a previous `\$\$'.
             So I shall assume that you typed `\$\$' both times.}

    Although \$\$ is one of \tx's methods for starting and ending
display math, do \emph{not} use it in \ltx.

%1177 
\item[\textmess{!}]\index{Double subscript}
     \textmess{Double subscript.} \\
     \texthelp{I treat} \verb?`x_1_2'? \texthelp{essentially like} 
     \verb?`x_1{}_2'.? 

    This would produce $x_1{}_2$. If you were after
      say, $x_{2_{3}}$ instead, type \verb?x_{2_{3}}?.

%1177 
\item[\textmess{!}]\index{Double superscript}
     \textmess{Double superscript.} \\
     \texthelp{I treat} \verb?`x^1^2'? \texthelp{essentially like} 
     \verb?`x^1{}^2'.? 

  This would produce $x^1{}^2$. If you were after
      say, $x^{2^{3}}$ instead, type \verb?x^{2^{3}}?.

%169 \texttt{Double-AVAIL list clobbered at \ldots}

%169 \texttt{Doubly free location at ldots}

%1335 
\item[\textmess{!}]\index{end occurred inside?\cs{end} occurred inside a group ...}
     \textmess{(\cmd{\end} occurred inside a group at level ...).}

    This is message is output at the end of a run. It means that you have not
ended all the groups that you started; a group can be started by
a simple open brace (\{), but there are other starting mechanisms as well,
such as \senv{...}. If the problem is a missing \eenv{...}, \ltx\ is kind 
enough to tell you what the mismatch is.

%1335 
\item[\textmess{!}]\index{end occurred when?\cs{end} occurred when ...}
      \textmess{(\cmd{\end} occurred when ... was incomplete).}

%1335 \texttt{(\cmd{\dump} is performed only by INITEX)}

%963 \texttt{Duplicate pattern}
%    \texttt{(See Appendix H.)}

%93 \texttt{Emergency stop}

%183 \texttt{Unknown node type!}

%292 \texttt{\cs{ETC}.}

%293 \texttt{\cs{CLOBBERED}.}

%293/4 \texttt{\cs{BAD}.}

%510 
\item[\textmess{!}]\index{Extra fi?Extra \cs{fi}}%
                   \index{Extra else?Extra \cs{else}}%
                   \index{Extra or?Extra \cs{or}}
     \textmess{Extra \cs{fi}.} or  \textmess{Extra \cs{else}.} or \textmess{Extra \cs{or}.} \\
    \texthelp{I'm ignoring this; it doesn't match any \cs{if}.}

%1135 
\item[\textmess{!}]\index{Extra endcsname?Extra \cs{endcsname}}
     \textmess{Extra \cmd{\endcsname}.} \\
     \texthelp{I'm ignoring this, since I wasn't doing a \cmd{\csname}.}

%1192 
\item[\textmess{!}]\index{Extra right?Extra \cs{right}}
     \textmess{Extra \cmd{\right}.} \\
     \texthelp{I'm ignoring a \cmd{\right} that had no matching \cmd{\left}.}

%1069 
\item[\textmess{!}]\index{Extra ... or forgotten ...} 
     \textmess{Extra \}, or forgotten \cmd{\endgroup}, \$, or \cmd{\right}.} \\
     \texthelp{I've deleted a group closing symbol because it seems to be 
       spurious, as in `\$x\}\$'. But perhas the \} is legitimate and
       you forgot something else, as in} \verb?`\hbox{$x}'.? %$
       \texthelp{In such cases the way to recover is to insert both the
       forgotten and the deleted material, e.g., by typing `I\$\}'.}

       The braces or math mode delimeters didn't match. You might have
       forgotten a \texttt{\{}, \cmd{\[}, \cmd{\(} or \texttt{\$}.

%1066 
\item[\textmess{!}]\index{Extra ...}
      \textmess{Extra ...} \\
     \texthelp{Things are pretty mixed up, but I think the worst is over.}

%792 
\item[\textmess{!}]\index{Extra alignment tab ...}
    \textmess{Extra alignment tab has been changed to \cmd{\cr}.} \\
    \texthelp{You have given more \cmd{\span} or \& marks than there were
      in the preamble to the \cmd{\halign} or \cmd{\valign} now in progress.
      So I'll asume that you meant to type \cmd{\cr} instead.} 

    Internally, \ltx\ uses
      \cmd{\halign} for its \Ie{array}\index{array} and 
     \Ie{tabular}\index{tabular} environments.
      The message means that you have too many column entries in a row 
      (i.e., too many \texttt{\&} before the end of the row). Perhaps
      you have forgotten to put \cmd{\\} at the end of the preceding row.

%789 
%\item[\textmess{!}] \textmess{(interwoven alignment preambles are not allowed).} (fatal error)

%1303 
%\item[\textmess{!}] \textmess{(Fatal format file error; I'm stymied).}

%338 
\item[\textmess{!}]\index{File ended while scanning ...}%
       \index{Forbidden control sequence found ...}
     \textmess{File ended while scanning \ldots .} or 
\textmess{Forbidden control sequence found while scanning \ldots .}  \\
    \texthelp{I suspect you have forgotten a `\}', causing me
      to read past where you wanted me to stop. I'll try to recover;
      but if the error is serious you'd better type `E' or `X' now
      and fix your file.}


%579 
%\item[\textmess{!}] \textmess{Font \ldots has only \ldots fontdimen parameters.} \\
%    \texthelp{To increase the number of font parameters, you must use
%      \cmd{\fontdimen} immediately after the \cmd{\font} is loaded.}

%561 
\item[\textmess{!}]\index{Font ... not loadable ...}
 \textmess{Font \ldots not loadable: Metric (TFM) file not found.} \\
\item[\textmess{!}]\index{Font ... not loadable ...}
    \textmess{Font \ldots not loadable: Bad metric (TFM) file.} \\
    \texthelp{I wasn't able to read the size data for this font, so I will
      ignore the font specification.
      [Wizards can fix TFM files using TFtoPL/PLtoTF.]
      You might try inserting a different font spec;
      e.g., type} 
      \verb?`I\font<same font id>=<substitute font name>'.?

  \ltx\ can't find a font you have asked for.

%567 
%\item[\textmess{!}] \textmess{Font \ldots not loaded: Not enough room left.} \\
%    \texthelp{I'm afraid I won't be able to make use of this font,
%      because my memory for character-size data is too small.
%      If you're really stuck, ask a izard to enlarge me.
%      Or maybe try} 
%      \verb?`I\font<same font id>=<name of loaded font>'.?

%641 
\item[\textmess{!}]\index{Huge page cannot be shipped out}
    \textmess{Huge page cannot be shipped out.} \\
    \texthelp{The page just created is more than 18 feet tall or
      more than 18 feet wide, so I suspect something went wrong.}

%530 
\item[\textmess{!}]\index{I can't find file ...}
     \textmess{I can't find file `\ldots', please type another.} \\

  \tx\ couldn't find the file you asked it to read. You can also
  get this message with \ltx\ if you have missed the braces around
  the argument to \cmd{\input}.

%95 
\item[\textmess{!}]\index{I can't go on meeting you like this} 
   \textmess{I can't go on meeting you like this.} \\
   \texthelp{One of your faux pas seems to have wounded me deeply...
     in fact, I'm barely conscious. Plase fix it and try again.}
   

%530 
\item[\textmess{!}]\index{I can't write on file ...}
     \textmess{I can't write on file `\ldots', please type another.} \\

  \tx\ couldn't write on a file, you might have mispelled the name
   or not have permission to use it.

%535 
%\item[\textmess{!}] \textmess{I can't write on file `\ldots'.} (fatal error?)

%51 \texttt{! I can't read TEX.POOL}

%288/1258 
%\item[\textmess{!}] \textmess{Illegal magnification has been changed to 1000.} \\
%    \texthelp{The magnification ratio must be between 1 and 32768.}

%1120 
%\item[\textmess{!}] \textmess{Illegal math \cmd{\discretionary}.} \\
%     \texthelp{Sorry: The third part of a discretionary break must be empty,
%       in math formulas. I had to delete your third part.}

%479 
\item[\textmess{!}]\index{Illegal parameter number ...} 
    \textmess{Illegal parameter number in definition of \ldots .} \\
    \texthelp{You meant to type \#\# instead of \#, right?
      Or maybe a \} was forgotten somewhere earlier, and things are
      all screwed up? I'm going to assume that you meant \#\#.}

    This is probably due to a command defining command like \cmd{\newcommand}
 or \cmd{\renewcommand} or \cmd{\providecommand}, or an environment
 defining command like \cmd{\newenvironment} or \cmd{\renewenvironment}, where 
a \verb?#? has been used incorrectly. Apart from the command \cmd{\#},
a \verb?#? can only be used to indicate an argument parameter, like \verb?#3?
which denotes the third argument. You cannot use an argument parameter,
like the \verb?#3? in the last argument of either the \cmd{\newenvironment}
or the \cmd{\renewenvironment} commands.

    You get the same error if you try to include any of the above defining
commands inside another one.


%454 
\item[\textmess{!}]\index{Illegal unit of measure ...}
    \textmess{Illegal unit of measure (replaced by filll).} \\
    \texthelp{I dddon't go any higher than filll.}

    You have tried to use a \texttt{filll} with more than 3 `l's.

%456 
\item[\textmess{!}]\index{Illegal unit of measure ...}
    \textmess{Illegal unit of measure (mu inserted).} \\
    \texthelp{The unit of measurement in math glue must be mu.
      To recover gracefully from this error it's best to delete
      the erroneous units; e.g., type `2' to delete two letters.
      (See Chapter 27 of The TeXbook.)}

     \tx\ was in math mode
      and expecting a length, which must be in \texttt{mu} units.

%459 
\item[\textmess{!}]\index{Illegal unit of measure ...} 
    \textmess{Illegal unit of measure (pt inserted).} \\
    \texthelp{Dimensions can be in units of em, ex, in, pt, pc, cm, mm, dd,
      cc, bp, or sp; but yours is a new one!
      I'll assume you meant to say pt, for printers' points.
      To recover gracefully from this error it's best to delete
      the erroneous units; e.g., type `2' to delete two letters.
      (See Chapter 27 of The TeXbook.)} 

    \tx\ was expecting a length
      but it found just a number without a known length unit. For example
you wrote \verb?2ib? instead of \verb?2in?.

%1121 
%\item[\textmess{!}] \textmess{Improper discretionary list.} \\
%     \texthelp{Discretionary lists must contain only boxes and kerns.
%       The following discretionary sublist has been deleted: \ldots}

%935 
\item[\textmess{!}]\index{Improper \cs{hyphenation} ...}
    \textmess{Improper \cmd{\hyphenation} will be flushed.} \\
    \texthelp{Hyphenation exceptions must contain only letters
      and hyphens. But continue; I'll forgive and forget.}

%288 
%\item[\textmess{!}] \textmess{Incompatible magnification (\ldots); the previous value will
%            be retained.} \\
%    \texthelp{I can handle only one magnification ratio per job. So I've
%      reverted to the magnification you used earlier on this run.}

%336 
\item[\textmess{!}]\index{Incomplete ...}
    \textmess{Incomplete \ldots all text was ignored after line \ldots.} \\
    \texthelp{A forbidden control sequence occurred in skipped text.
      This kind of error happens when you say `\cs{if}...' and forget
      the matching `\cs{fi}'. I've inserted a `\cs{fi}'; this might work.}

%993 
%\item[\textmess{!}] \textmess{Insertions can only be aded to a vbox.} \\
%    \texthelp{Tut tut: You're trying to insert into a
%      box register that now contains an \cmd{\hbox}.
%      Proceed, and I'll discard its present contents.}

%826 
\item[\textmess{!}]\index{Infinite glue shrinkage ...}
    \textmess{Infinite glue shrinkage found in a paragraph.} \\
    \texthelp{The paragraph just ended includes some glue that has
       infinite shrinkability, e.g.,} \verb?`\hskip 0pt minus 1fil'.?
       \texthelp{Such glue doesn't belong there---it allows a paragraph
       of any length to fit on one line. But it's safe to proceed,
       since the offensive shrinkability has been made finite.}

%1232 
%\item[\textmess{!}] \textmess{Invalid code (\ldots) should be in the range 0 to \ldots.}
%or \textmess{Invalid code (\ldots) should be at most \ldots} \\
%     \texthelp{I'm going to use 0 instead of that illegal code value.}

%1159 
\item[\textmess{!}]\index{Limit controls ...}
     \textmess{Limit controls must follow a math operator.} \\
     \texthelp{I'm ignoring this misplaced \cmd{\limits} or \cmd{\nolimits} 
       command.}

%660 
%\item[\textmess{!}] \textmess{Loose \cmd{\hbox} (badness \ldots).}

%674 
%\item[\textmess{!}] \textmess{Loose \cmd{\vbox} (badness \ldots).}


%1195 
%\item[\textmess{!}]\textmess{Math formula deleted: Insufficient extension fonts.} \\
%     \texthelp{Sorry, but I can't typeset math unless \cmd{\textfont} 3
%       and \cmd{\scriptfont} 3 and \cmd{\scriptscriptfont} 3
%       have all the \cmd{\fontdimen} values needed in math extension fonts.}

%1195 
%\item[\textmess{!}] \textmess{Math formula deleted: Insufficient symbol fonts.} \\
%     \texthelp{Sorry, but I can't typeset math unless \cmd{\textfont} 2
%       and \cmd{\scriptfont} 2 and \cmd{\scriptscriptfont} 2
%       have all the \cmd{\fontdimen} values needed in math symbol fonts.}

%1128 
\item[\textmess{!}]\index{Misplaced \&}\index{Misplaced \cs{cr}}\index{Misplaced \cs{span}} 
      \textmess{Misplaced \&.} or \textmess{Misplaced \cmd{\cr}.} or \textmess{Misplaced \cmd{\span}.} \\
     \texthelp{I can't figure out why you would want to use a tab mark
       or \cmd{\cr} or \cmd{\span} here. 
       If you just want an ampersand the remedy is simple: Just type}
       \verb?`I\&'? 
       \texthelp{now. But if some right brace
       up above has ended a previous alignment prematurely,
       you're probably due for more error messages, and you
       might try typing `S' now just to see what is salvageable.}

     In \ltx\ the most likely of these messages is the 
     \textmess{Misplaced \&}. You can only use a naked \texttt{\&} in 
     environments like \Ie{array} and \Ie{tabular} as column separators.
     Anywhere else you have to use \cmd{\&}.


%1129 
\item[\textmess{!}]\index{Misplaced \cs{noalign}} 
     \textmess{Misplaced \cmd{\noalign}.} \\
     \texthelp{I expect to see \cmd{\noalign} only after the \cmd{\cr} of
       an alignment. Proceed, and I'll ignore this case.}

%1129 
\item[\textmess{!}]\index{Misplaced \cs{omit}} 
     \textmess{Misplaced \cmd{\omit}.} \\
     \texthelp{I expect to see \cmd{\omit} only after the tab marks or 
       the \cmd{\cr} of an alignment. Proceed, and I'll ignore this case.}

%1132 
\item[\textmess{!}]\index{Missing \cs{cr} inserted} 
     \textmess{Missing \cmd{\cr} inserted.} \\
     \texthelp{I'm guessing that you meant to end an alignment here.}
    
 You might have missed a \cmd{\\} at the end of the last row
     of a \Ie{tabular} or \Ie{array}.

%503 
\item[\textmess{!}]\index{Missing = inserted ...} 
    \textmess{Missing = inserted for \ldots .} \\
    \texthelp{I was expecting to see `$<$', `$=$', or `$>$'. Didn't.}

%783 
\item[\textmess{!}]\index{Missing \# inserted ...} 
     \textmess{Missing \# inserted in alignment preamble.} \\
    \texthelp{There should be exactly one \# between \&'s, when an 
      \cmd{\halign} or \cmd{\valign} is being set up. In this case you had
      none, so I've put one in; maybe that will work.}

    If you get this in \ltx\ then there are problems with the argument
    to an \Ie{array} or \Ie{tabular}. 

\item[\textmess{!}]\index{Missing \$ inserted}%
    \index{Missing \cs{endgroup} inserted}%
    \index{Missing \cs{right} inserted}%
    \index{Missing \rb inserted} 
    \textmess{Missing \$ inserted.} or 
    \textmess{Missing \cmd{\endgroup} inserted.} or 
    \textmess{Missing \cmd{\right} inserted.} or 
    \textmess{Missing \} inserted.} \\
     \texthelp{I've inserted something that you may have forgotten.
       (See the $<$inserted text$>$ above.)
       With luck, this will get me unwedged, But if you 
       really didn't forget anything, try typing `2' now; then
       my insertion and my current dilemma will both disappear.}

   This is a general response to the above messages. There is also a
more specific response for each of the messages, as listed below.


%1047 
\item[\textmess{!}]\index{Missing \$ inserted} 
     \textmess{Missing \$ inserted.} \\
     \texthelp{I've inserted a begin-math/end-math symbol since I think
       you left one out. Proceed with fingers crossed.}

    Certain commands can only be executed in math mode and there are 
others that cannot be used in math mode. \tx\ has come across a command that
cannot be used in the current mode, so it switches into, or out of, math
mode on the assumption that that was what you had forgotten to do.

%1065 
%373 
\item[\textmess{!}]\index{Missing \cs{endcsname} inserted} 
    \textmess{Missing \cmd{\endcsname} inserted.} \\
    \texthelp{The control sequence marked $<$to be read again$>$ should
      not appear between \cmd{\csname} and \cmd{\endcsname}.}

%403 
\item[\textmess{!}]\index{Missing \lb{} inserted} 
    \textmess{Missing \{ inserted.} \\
    \texthelp{A left brace was mandatory here, so I've put one in.
      You might want to delete and/or insert some corrections
      so that I will find a matching right brace soon.
      If you're confused by all this, try typing `I\}' now.}

%475 
\item[\textmess{!}]\index{Missing \lb{} inserted} 
    \textmess{Missing \{ inserted.} \\
    \texthelp{Where was the left brace? You said something like}
      \verb?\def\a}',?
      \texthelp{which I'm going to interpret as}
      \verb?\def\a{}'.?

     In \ltx\ terms, the example wrongdoing would be \verb?\newcommand{\a}}?

%1127 
\item[\textmess{!}]\index{Missing \lb{} inserted} 
     \textmess{Missing \{ inserted.} \\
     \texthelp{I've put in what seems becessary to fix
       the current column of the current alignment.
       Try to go on, since this might almost work.}

      It seems that a \texttt{\{} might have been missing in a \Ie{tabular}
      or \Ie{array} entry.

%1082 
%\item[\textmess{!}] \textmess{Missing `to' inserted.} \\
%     \texthelp{I'm working on} \verb?`\vsplit<box number> to <dimen>';?
%       \texthelp{will look for the $<$dimen$>$ next.}

%1207 
%\item[\textmess{!}] \textmess{Missing \$\$ inserted.} \\
%     \texthelp{Displays can use special alignments (like \cmd{eqalignno})
%       only if nothing but the alignment itself is between \$\$'s.}

%581 
%\item[\textmess{!}] \textmess{Missing character: there is no \ldots in font \ldots.}

%1215 
\item[\textmess{!}]\index{Missing control sequence inserted} 
   \textmess{Missing control sequence inserted.} \\
     \texthelp{Please don't say} \verb?`\def cs{...}',? \texthelp{say}
       \verb?`\def\cs{...}'.?
       \texthelp{I've inserted an inaccessible control sequence so that your
       definition will be completed without mixing me up too badly.
       You can recover graciously from this error, if you're
       careful; see exercise 27.2 in The TeXbook.}

%1161 
\item[\textmess{!}]\index{Missing delimeter(. inserted).} 
     \textmess{Missing delimeter(. inserted).} \\
     \texthelp{I was expecting to see something like} 
       \verb?`('? 
       \texthelp{or}
       \verb?`\{'? 
       \texthelp{or} 
       \verb?`\}'? 
       \texthelp{here. If you typed, e.g.,} 
       \verb?`{'? 
       \texthelp{instead of}
       \verb?`\{'? 
       \texthelp{you should probably delete the}
       \verb?`{'? 
       \texthelp{by typing `1' now, so that braces don't get unbalanced.
       Otherwise just proceed.
       Acceptable delimeters are characters whose \cmd{\delcode} is
       nonnegative, or you can use `\cmd{\delimeter} $<$delimeter code$>$'.}

%577 
%\item[\textmess{!}] \textmess{Missing font identifier.} \\
%    \texthelp{I was looking for a control sequence whose
%      current meaning has been defined by \cmd{\font}.}

%415/446 
\item[\textmess{!}]\index{Missing number ...}
    \textmess{Missing number, treated as zero.} \\
    \texthelp{A number should have been here; I inserted `0'.
      (If you can't figure out why I needed to see a number,
      look up `weird error' in the index to The TeXbook.)}

    In \ltx\ this is often caused by a command expecting a number or a length
argument but not finding it. You might have forgotten the argument or
an opening square bracket in the text might have been taken as the start
of an optional argument. For example, the \cmd{\\} (newline) command takes 
an optional length argument, so the following will produce this error:
\begin{lcode}
... next line\\
[Horatio:] ...
\end{lcode}
                         
%937 
\item[\textmess{!}]\index{Not a letter} 
    \textmess{Not a letter.} \\
    \texthelp{Letters in \cmd{\hyphenation} words must have \cmd{\lccode}>0.}

    One or more characters in the argument to the \cmd{\hyphenation} command
    should not be there.


%962 
%\item[\textmess{!}] \textmess{Nonletter.} \\
%    \texthelp{(See Appendix H.)}

%445 
\item[\textmess{!}]\index{Number too big} 
    \textmess{Number too big.} \\
    \texthelp{I can only go up to 2147483647 = '17777777777 = "7FFFFFFF,
      so I'm using that number instead of yours.} 

    These all represent the
      same value, firstly in decimal, secondly in octal, and lastly in
      hexadecimal notations.

%1024 
\item[\textmess{!}]\index{Output loop ...}
     \textmess{Output loop--- \ldots consecutive dead cycles.} \\
     \texthelp{I've concluded that your \cmd{\output} is awry; it never does a
       \cs{shipout}, so I'm shipping \cmd{\box255} out myself. Next time
       increase \cmd{\maxdeadcycles} if you want me to be more patient!}

       \tx\ appears to be spinning its wheels, doing nothing.

%1024 
%\item[\textmess{!}] \textmess{Output routine didn't use all of \cmd{\box255}.} \\
%     \texthelp{Your \cmd{\output} commands should empty \cmd{\box255},
%       e.g., by saying `\cs{shipout}\cmd{\box255}'.
%       Proceed; I'll discard its present contents.}

%666 
\item[\textmess{!}]\index{Overfull \cs{hbox} ...} 
    \textmess{Overfull \cmd{\hbox} (\ldots pt too wide).}

    This is a warning that \tx\ couldn't cram some text into the alloted
horizontal space.

%677 
\item[\textmess{!}]\index{Overfull \cs{vbox} ...}
    \textmess{Overfull \cmd{\vbox} (\ldots pt too high).}

    This is a warning that \tx\ couldn't find a good place for a
pagebreak, so it has put too much onto the current page.


%396 
\item[\textmess{!}]\index{Paragraph ended before ...} 
    \textmess{Paragraph ended before \ldots was complete.} \\
    \texthelp{I suspect you've forgotten a `\}', causing me to apply this
      control sequence to too much text. How can we recover?
      My plan is to forget the whole thing and hope for the best.}

      Either a blank line or a \piif{par} command appeared in the
      argument to a macro that cannot handle paragraphs (e.g.,
      a macro that was defined using \cmd{\newcommand*}).

%476 
%\item[\textmess{!}] \textmess{Parameters must be numbered consecutively.} \\
%    \texthelp{I've inserted the digit you should have used after the \#.
%      Type `1' to delete what you did use.}

%1252 \texttt{Patterns can only be loaded by INITEX}

%360 
\item[\textmess{!}]\index{Please type a command ...}  
     \textmess{Please type a command or say `\cmd{\end}'.} 

    This is the message that causes me the most trouble. My computer
always ignores whatever I say to it and even typing \cmd{\end} has
no effect. What I usually do, after having tried a few variations
like \eenv{document}, is to kill the program by whatever means the operating
system provides. Some other possible responses include:
\begin{itemize}
\item Type \cmd{\stop}
\item Type \verb?\csname @@end\endcsname? (\ltx\ stores \tx's version of
      \cmd{\end} as \cmd{\@@end})
\item Type some macro that you think is unknown, perhaps \cs{qwertyuiod},
      then respond to the error message: \textmess{Undefined control sequence.}
\item Sometimes nothing works except killing the program. If you are are sure you
      know how to kill a program, try the following highly contrived code:
\begin{lcode}
\documentclass{article}
  \newif\ifland
  \newif\ifprint
  \newcommand{\Xor}[2]{\ifx #1 #2}
\begin{document}
%  \Xor{\ifland}{\ifprint}% try uncommenting this
  \iffalse
\end{document}
\end{lcode}
\end{itemize}



%1166 
%\item[\textmess{!}] \textmess{Please use \cmd{\mathaccent} for accents in math mode.} \\
%     \texthelp{I'm changing \cmd{\accent} to \cmd{\mathaccent} here; wish 
%       me luck. Accents are not the same in formulas as they are in text.}

%306 
\item[\textmess{!}]\index{Runaway argument}%
                   \index{Runaway definition}%
                   \index{Runaway preamble}%
                   \index{Runaway text}
    \textmess{Runaway argument.} or 
    \textmess{Runaway definition.} or
    \textmess{Runaway preamble.} or 
    \textmess{Runaway text.} 

%524 
%\item[\textmess{!}] \textmess{Sorry, I can't find that format; will try PLAIN.}

%1050 
\item[\textmess{!}]\index{Sorry, but I'm not ...}
     \textmess{Sorry, but I'm not programmed to handle this case.} \\
     \texthelp{I'll just pretend that you didn't ask for it.
       If you're in the wrong mode, you might be able to
       return to the right one by typing `I\}' or `I\$' or
       `I\cs{par}'.}


%94 
\item[\textmess{!}]\index{TeX capacity exceeded ...} 
   \textmess{TeX capacity exceeded, sorry [\ldots].} \\
   \texthelp{If you absolutely need more capacity, you can ask a wizard 
     to enlarge me.}

   This is dealt with in more detail below.

%346 
\item[\textmess{!}]\index{Text line contains ...} 
    \textmess{Text line contains an invalid character.} \\
    \texthelp{A funny symbol that I can't read has just been input.
      Continue, and I'll forget that it ever happened.}

    The input file contains a nonprinting (control) character; only
printing characters should be in the file. Some programs, 
like word processors, insert invisible characters into their output file. If
you have used one of these to prepare your input file, make sure you
save it as a plain text file (also known as an ASCII file).


%82 
\item[\textmess{!}]\index{That makes 100 errors ...} 
     \textmess{That makes 100 errors; please try again.}

%95 
\item[\textmess{!}]\index{This can't happen ...} 
    \textmess{This can't happen (\ldots).} \\
   \texthelp{I'm broken. Please show this to someone who can fix can fix}

    This is the message you should never see!

%667 
%\item[\textmess{!}] \textmess{Tight \cmd{\hbox} (badness \ldots).}

%678 
%\item[\textmess{!}] \textmess{Tight \cmd{\vbox} (badness \ldots).}


%1068 
\item[\textmess{!}]\index{Too many \rb's} 
     \textmess{Too many \}'s.} \\
     \texthelp{You've closed more groups than you opened.
       Such booboos are generally harmless, so keep going.}

    There are more closing braces (\}) than there are opening braces (\{).

%1027 
\item[\textmess{!}]\index{Unbalanced output routine} 
     \textmess{Unbalanced output routine.} \\
     \texthelp{Your sneaky output routine has fewer real \{'s than \}'s.
       I can't handle that very well; good luck.}

     A package or class has done nasty things to one of \ltx's most
     delicate parts --- the output routine.

%1372 
\item[\textmess{!}]\index{Unbalanced write command} 
     \textmess{Unbalanced write command.} \\
     \texthelp{On this page there's a \cmd{\write} with fewer real \{'s 
       than \}'s. I can't handle that very well; good luck.}

%370 
\item[\textmess{!}]\index{Undefined control sequence} 
    \textmess{Undefined control sequence.} \\
    \texthelp{The control sequence at the end of the top line
      of your error message was never \cmd{\def}'ed. If you have
      misspelled it (e.g., `\cs{hobx}'), type `I' and the correct
      spelling (e.g., `I\cs{hbox}'). Otherwise just continue,
      and I'll forget whatever was undefined.}

    \tx\ has come across a macro name that it does not know about.
Perhaps you mispelled it, or it is defined in a package you did not include.
Another possibility is that you used a macro name that included the
\texttt{@} character without enclosing it between \cmd{\makeatletter}
and \cmd{\makeother} (\seeatincode)\idxatincode. 
In this case \tx\ would think that the name was 
just the portion up to the \texttt{@}.


%660 
\item[\textmess{!}]\index{Underfull \cs{hbox} ...} 
    \textmess{Underfull \cmd{\hbox} (badness \ldots).}

    This is a warning. There might be some extra horizontal space. It could
be caused by trying to use two \cmd{\newline} or \cmd{\\} commands
in succession with nothing intervening, or by using a \cmd{\linebreak}
command or typesetting with the \cmd{\sloppy} declaration.

%674 
\item[\textmess{!}]\index{Underfull \cs{vbox} ...}  
   \textmess{Underfull \cmd{\vbox} (badness \ldots).}

    This is a warning that \tx\ couldn't find a good place for a
pagebreak, so it produced a page with too much whitespace on it.

%398 
\item[\textmess{!}]\index{Use of ... doesn't match ...}
    \textmess{Use of \ldots doesn't match its definition.} \\
    \texthelp{If you say, e.g.,}
     \verb?`\def\a1{...}',?
     \texthelp{then you must always put `1' after `\cs{a}', since the control
       sequence names are made up of letters only. 
       The macro here has not been followed by the required stuff,
       so I'm ignoring it.}


%476 
%\item[\textmess{!}] \textmess{You already have nine parameters.} \\
%    \texthelp{I'm going to ignore the \# sign you just used.}



%1304 \texttt{You can't dump inside a group}
%     \texttt{`\{...\cmd{\dump}\}' is a no-no.}

%1099 
%\item[\textmess{!}] \textmess{You can't \cs{insert255}.} \\
%     \texthelp{I'm changing to \cs{insert0}; box 255 is special.}

%1095 
%\item[\textmess{!}] \textmess{You can't use `\cmd{\hrule}' here without leaders.} \\
%     \texthelp{To put a horizontal rule in an hbox or an alignment,
%       you should use \cmd{\leaders} or \cmd{\hrulefill} 
%       (see The TeXbook).}

%1213 
%\item[\textmess{!}] \textmess{You can't use `\cmd{\long}' or `\cmd{\outer}' with \ldots} \\
%     \texthelp{I'll pretend you didn't say \cmd{\long} or \cmd{\outer} here.}

%1212 
%\item[\textmess{!}] \textmess{You can't use a prefix with \ldots} \\
%     \texthelp{I'll pretend you didn't say \cmd{\long} or \cmd{\outer} 
%       or \cmd{\global}.}

%428 
%\item[\textmess{!}] \textmess{You can't use `\ldots' after \cmd{\the}.} \\
%    \texthelp{I'm forgetting what you said and using zero instead.}

%428 
%\item[\textmess{!}] \textmess{You can't use `\ldots' after `\ldots'.} \\
%    \texthelp{I'm forgetting what you said and not changing anything.}

%1049 
\item[\textmess{!}]\index{You can't use ... in ...} 
    \textmess{You can't use `\ldots' in `\ldots'.} \\

 This often manifests itself in the form \\
\textmess{You can't use `\cmd{\spacefactor}' in vertical 
 mode}\index{You can't use `\cs{spacefactor}' in vertical mode} \\
 the cause
is usually trying to use a macro with \texttt{@} in its name, typically
in the preamble (\seeatincode)\idxatincode. 
The solution is to enclose the macro within
\cmd{\makeatletter} and \cmd{\makeatother}. 

    Another version is \\
\textmess{You can't use `macro parameter character \#' in ... 
mode.}\index{You can't use `macro parameter character \#' in ... mode} \\
In this case you have used a naked \texttt{\#} in ordinary text; it can only
be used in the definition of a macro. In ordinary text you have to use 
\cmd{\#}.

%486 
%\item[\textmess{!}] \textmess{File ended within \cmd{\read}.} \\
%    \texthelp{This \cmd{\read} has unbalanced braces.}

\end{plainlist}

\index{TeX?\tx!warning|)}
\index{warning!TeX?\tx|)}

\subsection{\tx\ capacity exceeded}

\index{TeX capacity exceeded ...|(} 
    \tx\ has run out of computer space before it finished processing your
document. The most likely cause is an error in the input file rather than
there really not being enough space --- I have processed documents consisting
of more than 1400 pages without any capacity problems.

    You can very easily make \tx\ run out of space. Try inputting this:
\begin{lcode}
\documentclass{article}
\newcommand*{\fred}{Fred}          % should print `Fred'
% try to make it print `Frederick' instead
\renewcommand{\fred}{\fred erick}  
\begin{document}
  His name is \fred.
\end{document}
\end{lcode}
and \tx\ will tell you that it has run out of stack space:
\begin{lcode}
! TeX capacity exceeded, sorry [input stack size=15000].
\fred ->\fred
             erick
l.5 His name is \fred
                      .
No pages of output.
Transcript written on errors.log.
\end{lcode}

 The offending code above
tries to define \cs{fred} in terms of itself, and \tx\ just keeps chasing 
round and round trying to pin down \cs{fred} until it is exhausted.

    At the end of the \file{log} file for a run, \tx\ prints the memory space
it has used. For example:
\begin{lcode}
Here is how much of TeX's memory you used:
 2432 strings out of 60985
 29447 string characters out of 4940048
 106416 words of memory out of 8000001
 5453 multiletter control sequences out of 10000+65535
 8933 words of font info for 31 fonts out of 1000000 for 1000
 276 hyphenation exceptions out of 1000
 26i,11n,21p,210b,380s stack positions out of 
            15000i,4000n,6000p,200000b,40000s
\end{lcode}

    The error message says what kind of space it exhausted (input stack size
in the example above). The most common are:
\begin{plainlist}

\item[\texttt{buffer size}\index{buffer size}] 
   Can be caused by too long a section or caption title appearing
   in the \toc, \lof, etc. Use the optional argument to produce a 
   shorter entry.

\item[\texttt{exception dictionary}\index{exception dictionary}] There
    are too many words listed in \cmd{\hyphenation} commands. Remove any 
    that are not actually used and if that doesn't work, remove the less 
    common ones and insert \cmd{\-} in the words in the text.

\item[\texttt{hash size}\index{hash size}]
    The document defines too many command names and/or uses too many
    cross-referencing \cmd{\label}s.

\item[\texttt{input stack size}\index{input stack size}]
    Typically caused by a self-referencing macro definition.


\item[\texttt{main memory size}\index{main memory size}]
    There are three main things that  cause \tx\ to run out of main memory:
\begin{itemize}
\item Defining a lot of very long complicated macros.
\item Having too many \cmd{\index} or \cmd{\glossary} commands on a page.
\item Creating such a complicated page that \tx\ cannot hold all it needs
      to process it.
\end{itemize}
The solution to the first two problems is to simplify and eliminate. The
third is more problematic.

    Large \Ie{tabular}s, \Ie{array}s and \Ie{picture}s (the \cmd{\qbezier}
command is a memory hog) can gobble up memory. A queue of floats also demands
memory space. Try putting a \cmd{\clearpage} just before the place where the
error occurs and if it still runs out of room then there may be an error in
your file, otherwise you did exceed the capacity.

    If you have a long paragraph or a long \Ie{verbatim} environment try
breaking it up, as \tx\ keeps these in memory until it is ready to 
typeset them. If you have a queue of floats make sure that you have done
your best to help \ltx\ find a way to output them (see \Sref{sec:floatplace})
and try adding \cmd{\clearpage} at appropriate places to flush the queue.


\item[\texttt{pool size}\index{pool size}]
    Typically caused by having too many characters in command names
    and label names.

    It can also be caused by omitting the right brace that ends the
argument of a counter command (\cmd{\setcounter} or \cmd{\addtocounter})
or of a \cmd{\newenvironment} or \cmd{\newtheorem} command.

\item[\texttt{save stack size}\index{save stack size}]
   This happens if commands or environments are nested too deeply.
For instance a \Ie{picture} that contains a \Ie{picture} that includes
a \cmd{\multiput} that includes a \Ie{picture} that includes a \ldots


\index{TeX capacity exceeded ...|)} 

\end{plainlist}

\index{TeX?\tx!error|)}
\index{error!TeX?\tx|)}


\section{\ltx\ errors}

\index{LaTeX?\ltx!error|(}
\index{error!LaTeX?\ltx|(}

    \ltx\ errors introduce themselves differently from those
that \tx\ finds. For example, if you ever happended to
use the \cmd{\caption} command outside a float, like:
\begin{lcode}
\caption{Naked}
\end{lcode}
you would get the message:
\begin{lcode}
! LaTeX Error: \caption outside float.

See the LaTeX manual or LaTeX Companion for explanation.
Type H <return> for immediate help.
 ...

l.624 \caption
              {Naked}
?
\end{lcode}
If you then typed \texttt{H} in response you would get the following
helpful message:
\begin{lcode}
You're in trouble here. Try typing <return> to proceed.
If that doesn't work, type X <return> to quit.
?
\end{lcode}
The majority of \ltx's help messages follow this formula, so I have
not noted them in the alphabetical listing below.


%%%%\subsection{Errors}

\begin{plainlist}

%LTTAB
\item[]\index{\cs{<} in mid line} 
   \textmess{\cs{<} in mid line}

    A \cmd{\<} appears in the middle of a line in a \Ie{tabbing} environment;
it should only come at the start of a line.

%LTFSSDCL
\item[]\index{... allowed only in math mode} 
   \textmess{... allowed only in math mode}

    You have tried to use a math command in a non-math mode.


%%%% BBBBBBBBBBBBBBBBBB

%LTERROR
\item[]\index{Bad \cs{line} or \cs{vector} argument} 
   \textmess{Bad \cmd{\line} or \cmd{\vector} argument}

    A \cmd{\line} or \cmd{\vector} has a negative length argument or
the slope is not within the allowed range.

%LTERROR
\item[]\index{Bad math environment delimeter} 
   \textmess{Bad math environment delimeter}

    If in math mode there is a start math mode command like \cmd{\(}
or \cmd{\[} or if in LR or paragraph mode there is an end math mode
command like \cmd{\)}or \cmd{\]}. The basic problem is unmatched math
mode delimeters or unbalanced braces.

%LTERROR
\item[]\index{begin{...} ended by end{...}?\senv{...} ended by \eenv{...}} 
   \textmess{\senv{...} ended by \eenv{...}}

    The name of the \cmd{\begin} argument is not the same as the
name of the \cmd{\end} argument. This could be caused by a typo or a missing
\cmd{\end}.

%%%% CCCCCCCCCCCCCCCC

%LTERROR
\item[]\index{Can only be used in the preamble} 
    \textmess{Can only be used in the preamble}

    Some commands can only be used in the preamble\index{preamble}, such
as \cmd{\usepackage}, but
there was one of these after the \senv{document}.


%LTFLOAT
\item[]\index{caption outside float?\cs{caption} outside float} 
   \textmess{\cmd{\caption} outside float}

    You have used the \cmd{\caption} command outside a float, such as
a \Ie{figure} or \Ie{table} environment. 


%LTERROR
\item[]\index{Command \cs{...} already defined ...}
   \textmess{Command \cs{...} already defined or name \cs{end}... illegal}

    This is normally because you have used one of the \cs{new...}
commands to define a command or environment or counter name that has 
already been used; remember also that defining an environment \verb?foo?
automatically defines the macro \cs{foo}. Either choose a new name or
use the appropriate \cs{renew...}; also, see \Sref{sec:nameclash}.
In the unlikely event that you have
tried to define something beginning with \cs{end...}, choose another name.
\label{alreadydefined}



%LTERROR %LTFSSINI
\item[]\index{Command ... invalid ...} 
   \textmess{Command ... invalid in math mode}

    You have used a non-math command in math mode.

%LTFSSDCL
%\item[]\index{Command \cs{..} not defined as a math alphabet}
%     \textmess{Command \cs{..} not defined as a math alphabet} 
%       (use \cmd{\DeclareMathAlphabet} to define it)

%LTFSSINI
\item[]\index{Command ... not provided ...}
     \textmess{Command ... not provided in base LaTeX2e} 

    You have tried to use a symbol that is not part of basic \ltx.
Try loading the \Lpack{latexsym} or \Lpack{amsfonts} package which
might define the symbol.

%LTOUTENC
%\item[] \textmess{Command ... unavailable in encoding ...}

%LTFSSBAS
%\item[] \textmess{Corrupted NFSS tables}


%LTERROR
\item[]\index{Counter too large} 
   \textmess{Counter too large}

    You are using a non-numeric counter representation, such as letters
or footnote symbols, and the counter has exceeded the allowed number
(for example there are only 26 alphabetic characters).


%%%% EEEEEEEEEEEEEEEEEEEE

%LTFSSBAS
%\item[] \textmess{Encoding scheme `...' unknown}

%LTOUTENC
%\item[] \textmess{Encoding file ... not found. You might have misspelt the
%       name of the encoding.}


%LTDEFNS %LTERROR %LTMISCEN
\item[]\index{Environment ... undefined}  
    \textmess{Environment ... undefined} 

    \ltx\ does not know the name of the argument of a \cmd{\begin}.
You have probably misspelled it.


%%%% FFFFFFFFFFFFFFFFFFFFF

%LTFILES
\item[]\index{File not found ...}
     \textmess{File not found. Type X to quit or <RETURN> to proceed
       or enter new name (Default extension: ...)}

    \ltx\ cannot find the file you requested. The extension \file{tex}
results from a problematic \cmd{\input} or \cmd{\include}; the extension
\file{sty} from a \cmd{\usepackage} and an extension \file{cls}
from a \cmd{\documentclass}. 

%LTERROR %LTOUTPUT
\item[]\index{Float(s) lost} 
    \textmess{Float(s) lost}

    Usually caused by having too many \cmd{\marginpar}s on a page.

%LTFSSBAS
%\item[] \textmess{Font family `..+..' unknown}

%LTFSSTRC
%\item[] \textmess{Font ... not found}



%%%% IIIIIIIIIIIIIIIIII

%LTERROR
\item[]\index{Illegal character ...} 
    \textmess{Illegal character in array argument} 

    There is an illegal character in the argument of an \Ie{array} or
\Ie{tabular} environment, or in the second argument of a
\cmd{\multicolumn} command.

%LTFILES
\item[]\index{\cs{include} cannot be nested}
   \textmess{\cmd{\include} cannot be nested}

    A file that is \cmd{\include}d cannot \cmd{\include} any other files.


%%%%% LLLLLLLLLLLLLLLL

%LTCLASS
\item[]\index{\cs{LoadClass} in package file}
  \textmess{\cmd{\LoadClass} in package file} 

    This is an error in a package file you are using 
(you can only use \cmd{\LoadClass} in a class file). Complain to the author.

%LTLISTS
\item[]\index{Lonely \cs{item} ...}
   \textmess{Lonely \cmd{\item} --- perhaps a missing list environment}

    An \cmd{\item} command appears to be outside any list environment.


%%%% MMMMMMMMMMMMMMMM

%LTFSSBAS
%\item[] \textmess{Math alphabet identifier ...is undefined in math version `...'} 
%       (Check the spelling or use the \cmd{\SetMathAlphabet} command)

%LTFSSBAS %LTFSSDCL
%\item[] \textmess{Math version `..' is not defined} (misspelled or need a package)


%LTERROR
\item[]\index{Missing \senv{document}}  
  \textmess{Missing \cs{begin}\{document\}} 

    If you haven't forgotten \senv{document} then there is something
wrong in the preamble as \ltx\ is trying to typeset something before
the document starts. This is often caused by missing the backslash from
a command, misplaced braces round an argument, a stray character, or
suchlike.

\item[]\index{Missing @-exp ...}
  \textmess{Missing @-exp in array argument} 

    The \texttt{@} character is not followed by an \pixatexp{}
in the argument of an \Ie{array} or
\Ie{tabular} environment, or in the second argument of a
\cmd{\multicolumn} command.

\item[]\index{Missing p-arg ...}
  \textmess{Missing p-arg in array argument} 

    There is a \texttt{p} not followed by braces 
in the argument of an \Ie{array} or
\Ie{tabular} environment, or in the second argument of a
\cmd{\multicolumn} command.

%%%% NNNNNNNNNNNNNNNN

%LTERROR
\item[]\index{No counter ... defined}
  \textmess{No counter ... defined} 

    The argument to a \cmd{\setcounter} or \cmd{\addtocounter}
command, or in the optional argument to \cmd{\newcounter}
or \cmd{\newtheorem} is not the name of a counter. Perhaps you 
misspelled the name.
However, if the error occured while an \file{aux} file was being read
then you might well have used a \cmd{\newcounter} in an \cmd{\include}d file.


%LTFSSDCL
%\item[] \textmess{Not a command name: \cs{...}}

%LTFSSTRC
%\item[] \textmess{No declaration for shape ...}

%LTPLAIN
\item[]\index{No room for a new ...}
  \textmess{No room for a new ...}

    \tx\ is limited in the numbers of different things it can handle. You 
might not recognize the thing that the message mentions as some of them are
hidden in \ltx. The \ltx\ \texttt{counter} uses a \tx\ \cmd{\count} for
example, and a length is a \tx\ \cmd{\skip}. Most things are limited to a 
maximum of 256 but there can be no more than 16 files open for reading
and 16 for writing.

%LTSECT
\item[]\index{No title given?No \cs{title} given}
  \textmess{No \cmd{\title} given}

You did not put a \cmd{\title} command before using \cmd{\maketitle}.


%LTERROR
\item[]\index{Not in outer par mode}
   \textmess{Not in outer par mode}

    There is a float (e.g., a \Ie{figure} or a \cmd{\marginpar})
in math mode or in a parbox (e.g., in another float).


%%%% OOOOOOOOOOOOOOOO

%LTHYPHEN
%\item[] \textmess{OOPS! I can't find any hyphenation patterns for US English.
%       Think of getting some otherwise latex2e setup will never succeed.}


%LTCLASS
\item[]\index{Option clash for ...}
  \textmess{Option clash for ...} 

    The same package was used twice but with different options. It is possible
for one package to use another package which might be the cause if you 
can't see anything obvious.

%%%% PPPPPPPPPPPPPPPPPP

%LTOUTPUT
\item[]\index{Page height already too large} 
  \textmess{Page height already too large}

    You are trying to use \cmd{\enlargethispage} when the page is already
too large.


%LTERROR
\item[]\index{pushtabs and poptabs don't match?\cs{pushtabs} and \cs{poptabs} don't match}
   \textmess{\cmd{\pushtabs} and \cmd{\poptabs}  don't match}

    There are unmatched \cmd{\pushtabs} and \cmd{\poptabs} in a
\Ie{tabbing} environment.


%%%% RRRRRRRRRRRRRRRR

%LTCLASS
\item[]\index{RequirePackage or LoadClass in Options Section?\cs{RequirePackage} or \cs{LoadClass} in Options Section}
   \textmess{\cmd{\RequirePackage} or \cmd{\LoadClass} in Options Section}

    This is a problem in a class or package file. Complain to the author.

%%%% SSSSSSSSSSSSSSSS

%LTERROR
\item[]\index{Something's wrong ...}
  \textmess{Something's wrong --- perhaps a missing \cmd{\item}} 

   This can be caused by not starting a list environment, such as \Ie{itemize}
with a \cmd{\item} command, or by omitting the argument to the
\Ie{thebibliography} environment. There are many other non-obvious
causes, such as calling some macro that ends up using \cmd{\addvspace} 
or \cmd{\addpenalty} when not in \texttt{vmode}.

%LTOUTPUT
\item[]\index{Suggested extra height ...}
  \textmess{Suggested extra height (...) dangerously large} 

    \ltx\ is concerned that you a trying to increase the page size
    too much with the \cmd{\enlargthispage} command.


%LTFSSDCL
%\item[] \textmess{Symbol font ... not defined}

%%%% TTTTTTTTTTTTTTTT

%LTERROR %LTTAB
\item[]\index{Tab overflow}
  \textmess{Tab overflow}

    In the \Ie{tabbing} environment a \cmd{\=} has exceeded \ltx's maximum
number of tab stops.


%LTCLASS
\item[]\index{The file needs format ...}
   \textmess{The file needs format ... but this is ...}

    The document uses a document class or package that is not compatible
with the  version of \ltx\ you are using. If you are using only standard
files then there is a problem with your \ltx\ installation.

%LTFNTCMD
%\item[] \textmess{The font size command \cmd{\normalsize} is not defined:
%       there is probably something wrong with the class file}

%LTERROR
\item[]\index{There's no line to end here}
   \textmess{There's no line to end here} 

    A \cmd{\newline} or \cmd{\\} appears in vertical mode, for example 
between paragraphs. Or perhaps you have tried to put \cmd{\\} immediately 
after an \cmd{\item} to start the text on a new line. If this is the case, 
then try this:
\begin{lcode}
\item \mbox{} \\
...
\end{lcode}

%LTERROR
\item[]\index{This may be a LaTeX bug} 
  \textmess{This may be a LaTeX bug} % (in output routine)
 
    This is a message you don't want to see as it is produced by the
output routine --- perhaps the most obscure part of \ltx. It is probably
due to an earlier error. If it is the first error, though, and you can't 
see anything wrong, ask for somebody's help.

%LTFSSDCL
%\item[] \textmess{This NFSS system isn't set up properly} (For encoding scheme ...
%       .../.../... do not form a valid font shape)

%LTFSSDCL
%5\item[] \textmess{This NFSS system isn't set up properly} 
%      (The system manitainer forgot
%       to specify a suitable substitution font shape using the 
%       \cmd{\DeclareErrorFont} command)

%LTERROR
\item[]\index{Too deeply nested} 
   \textmess{Too deeply nested}

    There are too many list environments nested within each other. At least
four levels are usually available but some list environments are not obvious
(for example the \Ie{quotation} environment is actually a list).

%LTMATH
\item[]\index{Too many columns ...}
   \textmess{Too many columns in eqnarray environment}

    An \Ie{eqnarray} environment has three \texttt{\&} column separators
with no \cmd{\\} between.

%LTFSSDCL
%\item[] \textmess{Too many math alphabets used in version ...}

%LTERROR
\item[]\index{Too many unprocessed floats}
   \textmess{Too many unprocessed floats}

    There may be too many \cmd{\marginpar}s to fit on a page, but it's more
likely that \ltx\ hasn't been able to find locations for printing all the
figures or tables. If one float cannot be placed, all later ones are saved 
until \ltx\ runs out of storage space. See \Sref{sec:floatplace} for 
details on how \ltx\ decides to place a float.

%LTCLASS
\item[]\index{Two documentclass commands?Two \cs{documentclass} commands}
   \textmess{Two \cmd{\documentclass} commands} 

    Your document has two \cmd{\documentclass} commands; only one is
permitted.


%LTCLASS
\item[]\index{Two LoadClass commands?Two \cmd{\LoadClass} commands}
  \textmess{Two \cmd{\LoadClass} commands}% (only one allowed)

    This is an error in the class file. Complain to the author.

%%%% UUUUUUUUUUUUUUUUUUUUUUUUUUU

%LTFSSTRC
%\item[] \textmess{Undefined font size function ...}

%LTERROR
\item[]\index{Undefined tab position}
   \textmess{Undefined tab position}

    A \cmd{\>}, \cmd{\+}, \cmd{\-}, or \cmd{\<} tabbing command
is trying to move to a tab position that has not been defined by a 
\cmd{\=} command.

%LTCLASS
\item[]\index{Unknown option ...}
   \textmess{Unknown option ... for class/package ...} 

    You have asked for an option that the class or package does not know about.
Perhaps you have mispelled something, or omitted a comma.

%LTFSSDCL
%\item[] \textmess{Unknown symbol font ...}

%LTCLASS
\item[]\index{usepackage before documentclass?\cs{usepackage} before \cs{documentclass}}
  \textmess{\cmd{\usepackage} before \cmd{\documentclass}} 

    In general, the \cmd{\usepackage} command can only be used in the 
preamble\index{preamble}.

%%%% VVVVVVVVVVVVVVVVVVVVVVVVVVVVV

%LTMISCEN
\item[]\index{verb ended by end of line?\cs{verb} ended by end of line}
  \textmess{\cs{verb} ended by end of line}

     The argument of a \piif{verb} command runs past the end of the line.
Perhaps you forgot to put in the correct ending character.

%LTMISCEN
\item[]\index{verb illegal in command argument?\cs{verb} illegal in command argument}
  \textmess{\cs{verb} illegal in command argument}

    A \piif{verb} cannot be part of the argument to another command.


%%%%%%%%%%%%%%%%%%%%% TBD



%LTVERS
%\textmess{LaTeX source files more than 1 year old!}



%LTFSSCMP
%\textmess{*** What's this? NFSS release 0? ***}
 
%LTFSSCMP
%\textmess{*** NFSS release 1 command ... found 
%       *** Recovery not possible. Use ...}

%LTFSSCMP
%\textmess{*** NFSS release 1 command \cmd{\newmathalphabet} found
%       *** Automatic recovery not possible}

\end{plainlist}

\index{error!LaTeX?\ltx|)}
\index{LaTeX?\ltx!error|)}


\section{\ltx\ warnings}

\index{warning!LaTeX?\ltx|(}
\index{LaTeX?\ltx!warning|(}

    Most warnings are given at the point in the document where
a potential problem is discovered, while others are output
after the document has been processed.

For example, the following code
\begin{lcode}
... \ref{joe}... \cite{FRED96} ...
\end{lcode}
may produce warnings like
\begin{lcode}
Latex Warning: Reference `joe' on page 12 undefined 
               on input line 881.
Latex Warning: Citation `FRED96' on page 12 undefined 
               at lines 890--897.
\end{lcode}
during the document processing, and then at the end there will also
be the warning:
\begin{lcode}
LaTeX Warning: There were undefined references. 
\end{lcode}

Some warning messages pinpoint where a problem might lie, as in the citation
warning above, while others make no attempt to do so. In the alphabetical
listing that follows I have not included such information, even if it is
supplied.

\begin{plainlist}
%%%% AAAAAAAAAAAAAAAAAAAAAAAAAAAAAAAAAA

%LTOUTPUT
%\item[] \textmess{Active ... character found while output routine is active.
%         This may be a bug in a package you are using}


%%%% BBBBBBBBBBBBBBBBBBBBBB


%%%% CCCCCCCCCCCCCCCCCCCCCCC

%LTBIBL
\item[]\index{Citation ... on page ...} 
  \textmess{Citation ... on page ... undefined}

    The key in a \cmd{\cite} command was not defined by any \cmd{\bibitem}.

%LTBIBL
\item[]\index{Citation ... undefined}
   \textmess{Citation ... undefined}

    The key in a \cmd{\cite} command was not defined by any \cmd{\bibitem}.

%LTDEFNS
%\item[] \textmess{Command ... has changed. Check if current package is valid} 
%  (from \cmd{\CheckCommand})

%LTFSSBAS
\item[]\index{Command ... invalid ...}
   \textmess{Command ... invalid in math mode}

    The command is not permitted in math mode but was used there anyway.
Remember that font size commands and \cmd{\boldmath} or \cmd{\unboldmath}
cannot be used in math mode.


%LTFSSTRC
%\item[] \textmess{Command \cmd{\tracingfonts} not provided. Use the 'tracefnt' 
%         package. Command found: ...}

%%%% EEEEEEEEEEEEEEEEEEEEEEEEEEEEEEEE

%LTFSSDCL
%\item[] \textmess{Encoding ... has changed to ... for symbol font ...
%         in the math version ...}


%%%% FFFFFFFFFFFFFFFFFFFFFFFFFFFF

%LTCLASS
%\item[] \textmess{File ... already exits on the system. Not generating it from this source}

%LTFLOAT
\item[]\index{Float too large ...}
   \textmess{Float too large for page by ...}

     A float (table or figure) is too tall to fit properly on a page by
the given amount. It is put on a page by itself.


%LTFSSTRC
\item[]\index{Font shape ...}
   \textmess{Font shape ... in size ... not available size ... substituted}

    You asked for a font size that was not available. The message also 
says what font is being used instead.

%LTFSSBAS
\item[]\index{Font shape ...}
   \textmess{Font shape ... undefined using ... instead}

    You asked for a font shape that was not available. The message also 
says what font is being used instead.

%%%% HHHHHHHHHHHHHHHHHHHHHHHHHHHHHHHHHH

%LTOUTPUT
\item[]\index{h float specifier ...}\index{"!h float specifier ...}
   \textmess{h float specifier changed to ht} or 
        \textmess{!h float specifier changed to !ht}

    A float has an optional \texttt{h} or \texttt{!h} argument but
as it wouldn't fit on the curent page it has been moved to the top
pf the next page.


%%%% IIIIIIIIIIIIIIIIIIIIIIIIII

%LTFILES
%\item[] \textmess{Inputting ... instead of obsolete ...}

%%%% LLLLLLLLLLLLLLLLLLLLLLLLLLL

%LTXREF
\item[]\index{Label ... multiply defined}
   \textmess{Label ... multiply defined}

    Two \cmd{\label} or \cmd{\bibitem} commands have the same argument
(at least during the previous \ltx\ run).

%LTMISCEN
\item[]\index{Label(s) may have changed ...}
   \textmess{Label(s) may have changed. Rerun to get cross-references right}

    This is only output at the end of the run.

One of the numbers printed by \cmd{\cite}, \cmd{\ref}
or \cmd{\pageref} commands might be incorrect because the correct values
have changed since the preceding \ltx\ run.


%%%% MMMMMMMMMMMMMMMMMMMMMMMMMMMMM

%LTOUTPUT
\item[]\index{Marginpar on page ...}
   \textmess{Marginpar on page ... moved}

    A \cmd{\marginpar} was moved down the page to avoid overwriting an earlier
one. The result will not be aligned with the \cmd{\marginpar} call.

%%%% NNNNNNNNNNNNNNNNNNNNNNNNNNNNNNNNNNNN

%LTSECT
\item[]\index{No author given?No \cs{author} given} 
  \textmess{No \cmd{\author} given}

    There is no \cmd{\author} command before calling \cmd{\maketitle}.

%LTOUTPUT
\item[]\index{No positions in optional float specifier ...}
  \textmess{No positions in optional float specifier.
         Default added (so using `tbp')}

    You have used an empty optional argument to a float, for example: \\
\verb?\begin{figure}[]? \\
so it has used \\
\verb?\begin{figure}[tbp]? \\
instead.

%%%% OOOOOOOOOOOOOOOOOOOOO

%LTOUTPUT
\item[]\index{Optional argument of twocolumn ...?Optional argument of \cs{twocolumn} ...}
   \textmess{Optional argument of \cmd{\twocolumn} too tall on page ...}

     The contents of the optional argument to \cmd{\twocolumn} was too
long to fit on the page.

%LTPICTUR
\item[]\index{oval, circle, or line size unavailable?\cs{oval}, \cs{circle}, or \cs{line} size unavailable}
   \textmess{\cmd{\oval}, \cmd{\circle}, or \cmd{\line} size unavailable}

    You have asked for too large (or too small) an oval or circle,
 or too short a line, in a \Ie{picture}.

%%%% RRRRRRRRRRRRRRRRRRRRRRRRRRRRRRRRRR

%LTXREF
\item[]\index{Reference ... on page ...}
   \textmess{Reference ... on page ... undefined}

    The argument of a \cmd{\ref} or \cmd{\pageref} has not been defined
on the preceding run by a \cmd{\label} command.

%%%% SSSSSSSSSSSSSSSSSSSSSS

%LTFINAL
\item[]\index{Size substitutions ....}
     \textmess{Size substitutions with differences up to ... have occured.
         Please check the transcript file carefully and redo the
         format generation if necessary!}

    This is only output at the end of the run.

    Some fonts have had to be used as substitutes for requested ones and
they are a different size.

%LTFSSBAS
\item[]\index{Some shapes ...}
   \textmess{Some shapes were not available, defaults substituted}

    This is only output at the end of the run.

    At least one font had to be substituted.

%%%% TTTTTTTTTTTTTTTTTTTTTTTTTTTTTTT

%LTOUTPUT
\item[]\index{Text page ... contains only floats}
   \textmess{Text page ... contains only floats}

    The page should have included some textual material but there was
no room for it.

%LTXREF
\item[]\index{There were multiply defined labels}
   \textmess{There were multiply defined labels}

    This is only output at the end of the run.

    Two or more \cmd{\label} or \cmd{\cite} commands had the same argument.

%LTXREF
\item[]\index{There were undefined references} 
  \textmess{There were undefined references}

    This is only output at the end of the run.

    There was at least one \cmd{\ref} or \cmd{\pageref} or \cmd{\cite} 
whose argument had not been defined
on the preceding run by a \cmd{\label} or \cmd{\biblabel} command.

%%%% UUUUUUUUUUUUUUUUUUUUUUUUUUUU

%LTFILES
\item[]\index{Unused global option(s) ...} 
  \textmess{Unused global option(s) [...]}

    The listed options were not known to the document class or any packages
you used.

%%%% WWWWWWWWWWWWWWWWWWWWWWWWWWWWWWW

%LTCLASS
%\item[] \textmess{Writing text ... before \cs{end}\{...\} as last line of ...}


%%%% YYYYYYYYYYYYYYYYYYYYYYYYYYYYYY

%LTCLASS
%\item[] \textmess{You have requested class/package ... but the class/package provides ...}

%LTCLASS
\item[]\index{You have requested release ...}
   \textmess{You have requested release ... of LaTeX but only release ... is available}

    You are using a class or package that requires a later release of \ltx\
than the one you are using. You should get the latest release.

%LTCLASS
\item[]\index{You have requested version ...}
   \textmess{You have requested version ... of class/package ... but only version
      ... is available}

    You (or the class or one of the packages you are using) needs a later 
release of a class or package than the one you are using. You should get
the latest release.

%%%%%%%%%%%%%%%%%%%%%%%%%%%%%%%%%%%%%%%%%

%%LTCLASS
%\textmess{... has been converted to Blank ...3e}


%%LTFSSCMP
%\textmess{*** NFSS release 1 command ... found 
%         *** Update by using release 2 command ...}


\end{plainlist}

\index{LaTeX?\ltx!warning|)}
\index{warning!LaTeX?\ltx|)}

\section{Class errors}

%%\index{memoir class!error|(}
\Iclasssub{memoir}{error|(}
\index{error!memoir class|(}

    The class errors introduce themselves differently from those
that \ltx\ finds. Instead of starting with \\
\verb?! LaTeX Error:?  \\
the class errors start with \\
\verb?! Class memoir Error:? \\
After that, it is indistinguishable from a \ltx\ error.
For example, if you ever happened to input the
next line as line 954 in your document you would get the error message
that follows \\
\verb?\sidecapmargin{either}? 
\begin{verbatim}
! Class memoir Error: Unrecognized argument for \sidecapmargin.

See the memoir class documentation for explanation.
Type H <return> for immediate help.
 ...

l.954 \sidecapmargin{either}
?
\end{verbatim}
If you then typed \texttt{H} (or \texttt{h}) in response you would 
get the following helpful message:
\begin{lcode}
Try typing <return> to proceed.
If that doesn't work, type X <return> to quit.
?
\end{lcode}
The majority of the help messages follow this formula, so I have
not noted them in the alphabetical listing below.


%%%%%\subsection{Errors}

\begin{plainlist}

\item[]\index{... is negative}
    \textmess{... is negative} 

    The value is negative. It should be at least zero.

\item[]\index{... is not a counter}
    \textmess{... is not a counter}

    An argument that should be the name of a counter is not.

\item[]\index{... is zero or negative}
    \textmess{... is zero or negative}

    The value must be greater than zero.


\item[]\index{>\lb...\rb at wrong position ...}
   \textmess{>\{...\} at wrong position: token ignored}

    A \verb?>{...}? in the argument to an \Ie{array} or \Ie{tabular}
is incorrectly placed and is being ignored.

\item[]\index{<\lb...\rb at wrong position ...}
   \textmess{<\{...\} at wrong position: changed to !\{...\}} 

    A \verb?<{...}? in the argument to an \Ie{array} or \Ie{tabular}
is incorrectly placed. It has been changed to \verb?!{...}? instead.

\item[]\index{A pattern has not been specified}
   \textmess{A pattern has not been specified}% (\cmd{\getstar@vsindent}

   You are trying to use the \Ie{patverse} or \Ie{patverse*} environment
without having first defined a pattern.

\item[]\index{Argument to \cs{setsidecappos} is not ...}
  \textmess{Argument to \cmd{\setsidecappos} is not t or c or b}

    The argument will be assumed to be \texttt{c}.

\item[]\index{Argument to \cs{overridesidecapmargin} neither  ...}
  \textmess{Argument to \cmd{\overridesidecapmargin} neither left nor right}

    The argument to \cmd{\overridesidecapmargin} must be either
\texttt{left} or \texttt{right}. The attempted override will be ignored.


\item[]\index{Cannot change a macro that has delimited arguments}
  \textmess{Cannot change a macro that has delimited arguments}

   You are using \cmd{patchcmd} on a macro that has delimted arguments.

\item[]\index{Empty preamble: `l' used} 
  \textmess{Empty preamble: `l' used} % (array/tabular)

    The argument to an \Ie{array} or \Ie{tabular} is empty. The
specification \verb?{l}? is being used instead.

\item[]\index{Font command ... is not supported}
   \textmess{Font command ... is not supported} 

    You have tried to use a deprecated font command. Either replace
it with the current font command or declaration or use 
the \Lopt{oldfontcommands} class option.

\item[]\index{footskip is too large ...?\cs{footskip} is too large ...}
   \textmess{\lnc{\footskip} is too large for \lnc{\lowermargin} by ...}

    The \lnc{\footskip} is too large for the \lnc{\lowermargin}. Either
increase the \lnc{\lowermargin} or decrease the \lnc{\footskip}.


\item[]\index{headheight and/or headsep are too large ...?\cs{headheight} and/or \cs{headsep} are too large ...}
   \textmess{\lnc{\headheight} and/or \lnc{\headsep} are too large for
        \lnc{\uppermargin} by ...}

        The sum of the \lnc{\headheight} and the \lnc{\headsep} is
  larger than the \lnc{\uppermargin}. Either increase the \lnc{\uppermargin}
  or reduce the others.


\item[]\index{Illegal pream-token ...}
    \textmess{Illegal pream-token (...): `c' used}

    An illegal character is used in the argument to an \Ie{array}
or \Ie{tabular}. The `c' specifier is being used instead 
(which centers the column).


\item[]\index{Index ... outside limits ...} 
    \textmess{Index ... outside limits for array ...} % (\cmd{\checkarrayindex}

    Trying to access an index for the array data structure that is not between
the low and high indices.


\item[]\index{Limits for array ... }
    \textmess{Limits for array ... are in reverse order} 

    The low index is not less than the high index in \cmd{\newarray}.


\item[]\index{Missing arg: token ignored} 
    \textmess{Missing arg: token ignored} % (array/tabular)

    The argument to a column specifier for a \Ie{array} or \Ie{tabular}
is missing.



\item[]\index{No array called ...}
    \textmess{No array called ...} % (\cmd{\checkarrayindex}

    You have tried to access an unknown array data structure.


\item[]\index{Not defined: ...}
  \textmess{Not defined: ...}

    You are using \cmd{\patchcmd} on a macro that is not defined.

\item[]\index{Not redefinable: ...}
  \textmess{Not redefinable: ...}

    You are using \cmd{\patchcmd} on a macro that it is unable to
modify.



\item[]\index{Only one column-spec. allowed}
    \textmess{Only one column-spec. allowed}% (array/tabular)

    There can only be one column specifier in a \cmd{\multicolumn}.

\item[]\index{Optional argument is not one of: ...}
  \textmess{Optional argument is not one of: classic, fixed, lines,
            or nearest. I will assume the default.}

    You have provided an unknown name for the optional argument to
\cmd{\checkthelayout}. The default \texttt{classic} will be used instead.


\item[]\index{paperheight and/or trimtop are too large ...?\cs{paperheight} and/or \cs{trimtop} are too large ...}
    \textmess{\lnc{\paperheight} and/or \lnc{\trimtop} are too large for
        \lnc{\stockheight} by ...}

        The sum of the \lnc{\paperheight} and the \lnc{\trimtop} is
  larger than the \lnc{\stockheight}. Either increase the \lnc{\stockheight}
  or reduce the others.

\item[]\index{paperwidth and/or trimedge are too large ...?\cs{paperwidth} and/or \cs{trimedge} are too large ...}
   \textmess{\lnc{\paperwidth} and/or \lnc{\trimedge} are too large for
        \lnc{\stockwidth} by ...}

        The sum of the \lnc{\paperwidth} and the \lnc{\trimedge} is
  larger than the \lnc{\stockwidth}. Either increase the \lnc{\stockwidth}
  or reduce the others.

\item[]\index{spinemargin and/or textwidth and/or foremargin are too large ...?\cs{spinemargin} and/or \cs{textwidth} and/or \cs{foremargin} are too large ...}
 \textmess{\lnc{\spinemargin} and/or \lnc{\textwidth} and/or \lnc{\foremargin}
        are too large for \lnc{\paperwidth} by ...}

        The sum of the \lnc{\spinemargin} and the \lnc{\textwidth}
  and the \lnc{\foremargin} is
  larger than the \lnc{\paperwidth}. Either increase the \lnc{\paperwidth}
  or reduce the others.


\item[]\index{The combination of argument values ...} 
   \textmess{The combination of argument values is ambiguous.
        The lengths will be set to zero} 
%    (\cmd{\setrectanglesize}, \cmd{\setfillsize})

    The combination of values in the arguments to one of the commands
for page layout does not make sense.

\item[]\index{The `extrafontsizes' option ...}
  \textmess{The `extrafontsizes' option is required to use the `...pt' option}

    If you want to use a `...pt' class option greater than \Lopt{25pt}
you also have to use the \Lopt{extrafontsizes} option. The class will
use the \Lopt{17pt} option.


\item[]\index{Unknown document division ...}
    \textmess{Unknown document division name (...)}

    You have used an unknown division name in the argument to 
\cmd{\settocdepth} or \cmd{\setsecnumdepth} and friends. If you haven't
mistyped it you will have to use \cmd{\setcounter} instead.

\item[]\index{Unknown mark setting type ...}
    \textmess{Unknown mark setting type `...' for ...mark}

    In \cmd{\createmark} or \cmd{\createplainmark} the mark setting 
type should have been \texttt{left} or \texttt{both} 
or \texttt{right}. The class will use \texttt{both}.

\item[]\index{Unknown numbering type ...}
  \textmess{Unknown numbering type ... for ...mark}

  In \cmd{\createmark} the class expected either \texttt{shownumber} or
\texttt{nonumber} for displaying the number. It will use 
\texttt{shownumber}.

\item[]\index{Unrecognized argument for \cs{sidecapmargin}}
  \textmess{Unrecognized argument for \cmd{\sidecapmargin}}

    The argument to \cmd{\sidecaption} should be \texttt{left} or 
\texttt{right} or \texttt{inner} or \texttt{outer}.

\item[]\index{uppermargin and/or textheight and/or lowermargin are too large ...?\cs{uppermargin} and/or \cs{textheight} and/or \cs{lowermargin} are too large ...}
   \textmess{\lnc{\uppermargin} and/or \lnc{\textheight} and/or 
        \lnc{\lowermargin} are too large for \lnc{\paperheight} by ...}

        The sum of the \lnc{\uppermargin} and the \lnc{\textheight}
  and the \lnc{\lowermargin} is
  larger than the \lnc{\paperheight}. Either increase the \lnc{\paperheight}
  or reduce the others.

\item[]\index{You have used the `*pt' option but file ...}
  \textmess{You have used the `*pt' option but file ... can't be found}

    You have used the \Lopt{*pt} option but the corresponding 
\pixfile{clo} file can't be found. Check your definitions of
\cmd{\anyptfilebase} and \cmd{\anyptsize}. The \pixfile{mem10.clo} 
file will be used instead.

\item[]\index{XeTeX is required to process this document}
  \textmess{XeTeX is required to process this document}

   The document needs to be processed via \pixxetx. Try using 
\texttt{xelatex} instead of \texttt{(pdf)latex}, or try removing 
any \pixxetx\ packages from the document.

    

\end{plainlist}


\index{error!memoir class|)}
%%\index{memoir class!error|)}
\Iclasssub{memoir}{error|)}

\section{Class warnings}

%%\index{memoir class!warning|(}
\Iclasssub{memoir}{warning|(}
\index{warning!memoir class|(}

    These are introduced by \verb?Class memoir Warning:? 

For example
\verb?\addtodef{alf}{\joe}{fred}? will produce a message along the lines of:
\begin{verbatim}
Class memoir Warning: `alf' is not a macro on input line 91.
\end{verbatim}
while 
\verb?\addtodef{\joe}{alf}{fred}? might produce:
\begin{verbatim}
Class memoir Warning: `\joe' is not a macro on input line 97.
\end{verbatim}

    The following is an alphabeticised list of the class warnings.

\begin{plainlist}

%%%%%%%%%%%%%%%%%%%%%%%

\item[]\index{... at index ... in pattern ...}
    \textmess{... at index ... in pattern ... is not a digit}% (\cmd{\get@vsindent}

    The character at the given position in the verse pattern is not a digit.

\item[]\index{... is not a macro}
    \textmess{... is not a macro}

    Using \cmd{\addtodef} or \cmd{\addtoiargdef} you have tried to extend 
the definition of an unknown macro.

\item[]\index{... is not an input stream}
    \textmess{... is not an input stream} 

    You are trying to access a non-existent input stream.

\item[]\index{... is not an output stream}
    \textmess{... is not an output stream}% (\cmd{\outstre@mandopen}, \cmd{\outstre@mandclosed}

    You are trying to access a non-existent output stream.

\item[]\index{Bad \cs{sidebarmargin} argument}
  \textmess{Bad \cs{sidebarmargin} argument}

   The argument to \cs{sidebarmargin} is not recognized. The class
will use \texttt{right}.


\item[]\index{Characters dropped after \eenv{...}}
    \textmess{Characters dropped after \eenv{...}} % (\cmd{\verbatim@rescan}

    At the end of a \Ie{verbatim} environment there should be no characters
after the \eenv{...} on the same line. 

\item[]\index{Column ... is already defined}
    \textmess{Column ... is already defined} % (tabulars

    The column type has been defined by a previous \cmd{\newcolumntype}.


\item[]\index{Counter ... already defined}
    \textmess{Counter ... already defined}

    For information only, the counter in \cmd{\providecounter} 
is already defined.

\item[]\index{Do not use footnote ...?Do not use \cs{footnote} ...}
   \textmess{Do not use \cmd{\footnote} in \cmd{\maketitle}. Use \cmd{\thanks} instead}

    You cannot use \cmd{\footnote} in any of the \cmd{\maketitle} elements
(i.e., \cmd{\title} or \cmd{\author} or \cmd{\date}) but you can use 
\cmd{\thanks}.

\item[]\index{Empty `thebibliography' environment}
    \textmess{Empty `thebibliography' environment}

    There are no \cmd{\bibitem}s in the \Ie{thebibliography} environment.

\item[]\index{Environment ... already defined}
    \textmess{Environment ... already defined}% (\cmd{\m@mprovenv}

    For information only, the environment in \cmd{\provideenvironment} 
is already defined.


\item[]\index{Index ... for pattern ...}
    \textmess{Index ... for pattern ... is out of bounds}% (\cmd{\get@vsindent}

    The index for the verse pattern is either too low or too high.

\item[]\index{Input stream ... is already defined}
    \textmess{Input stream ... is already defined} 

     You are trying to use \cmd{\newinputstream} to create an already existing
input stream.

\item[]\index{Input stream ... is not open}
    \textmess{Input stream ... is not open}% (\cmd{\instre@mandopen}

    You are trying to access or close an input stream that is closed.

\item[]\index{Input stream ... is open}
    \textmess{Input stream ... is open}% (\cmd{\instre@mandclosed}

    You are trying to open an input stream that is already open.

\item[]\index{Length ... already defined}
    \textmess{Length ... already defined}

    For information only, the length in \cmd{\providelength} 
is already defined.



\item[]\index{Marginpar on page ...}
    \textmess{Marginpar on page ... moved by ...}

    A marginal note has been lowered by the given amount to avoid overwriting
a previous note; the moved note will not be aligned with its \cmd{\marginpar}.
(This is a more informative message than the normal \ltx\ one.)

\item[]\index{No more to read from stream ...}
    \textmess{No more to read from stream ...}% (\cmd{\readaline})

    There is nothing left in the stream to be read.

\item[]\index{Optional argument of twocolumn ...?Optional argument of \cs{twocolumn} ...}
   \textmess{Optional argument of \cmd{\twocolumn} too tall on page ...}% (\cmd{\@topnewpage}

     The contents of the optional argument to \cmd{\twocolumn} was too
long to fit on the page.

\item[]\index{Output stream ... is already defined} 
    \textmess{Output stream ... is already defined}

     You are trying to use \cmd{\newoutputstream} to create an already existing
output stream.

\item[]\index{Output stream ... is not open}
    \textmess{Output stream ... is not open}% (\cmd{\outstre@mandopen}

    You are trying to access or close an output stream that is closed.

\item[]\index{Output stream ... is open}
    \textmess{Output stream ... is open}% (\cmd{\outstre@mandclosed}

    You are trying to open an output stream that already open.

\item[]\index{Redefining primitive column ...}
    \textmess{Redefining primitive column ...} % (tabulars

    The argument to \cmd{\newcolumntype} is one of the basic column types.



\item[]\index{Stream ... is not open}
    \textmess{Stream ... is not open}% (\cmd{stre@mverb@input} \cmd{\stre@mbvin}

    You are trying to access a stream, either input or output, that is closed.


\item[]\index{The ... font command is deprecated ...}
    \textmess{The ... font command is deprecated. Use ... or ... instead}

    You are using a deprecated font command. Consider using one of the
alternatives.

\item[]\index{The counter will not be printed ...}
    \textmess{The counter will not be printed. The label is: ...}% (\cmd{\@@enum@}

    The optional \meta{style} argument to the \Ie{enumerate} environment
does not include one of the special characters.

\item[]\index{Undefined index file ...}
    \textmess{Undefined index file ...}% (\cmd{\@index}, \cmd{\@spindex}

    You are trying to add an index entry to an unknown \file{idx} file.


\item[]\index{Unknown toclevel for ...}
  \textmess{Unknown toclevel for ...}

    The division name you have used for \cmd{\settocdepth} is not 
recognized.



\item[]\index{verb may be unreliable ...?\cs{verb} may be unreliable ...}
    \textmess{\cs{verb} may be unreliable inside tabularx} 

    A \cs{verb} in a \Ie{tabularx} may work, but may not.

\item[]\index{X columns too narrow ...} 
    \textmess{X columns too narrow (table too wide)} % (tabulars

    The width of the X columns in a \Ie{tabularx} had to be made too narrow.

\end{plainlist}

\index{warning!memoir class|)}
%%\index{memoir class!warning|)}
\Iclasssub{memoir}{warning|)}

%#% extend
%#% extstart include comments.tex

\svnidlong
{$Ignore: $}
{$LastChangedDate: 2018-04-05 11:07:48 +0200 (Thu, 05 Apr 2018) $}
{$LastChangedRevision: 596 $}
{$LastChangedBy: daleif@math.au.dk $}

\chapter{Comments}
\label{cha:comments}

\section{Algorithms}
\label{sec:algorithms}

Over time we may use this section to explain, or list some of the
algorithms for some of the macros in the class. The information may be
useful to some.

\subsection{Autoadjusting
  \texorpdfstring{\cs{marginparwidth}}{\textbackslash marginparwidth}}
\label{sec:auto-csmarg}

This algorithm is used within \cmd{\fixthelayout} unless the user have
used \cmd{\setmarginnotes}.

\noindent
\begin{framed}
  \vskip-2\baselineskip
  \begin{small}
\begin{verbatim}
if twocolumn then
  marginparwidth = min{inner margin,outer margin}
else
  if twoside then
    if marginpar always left or always right then
      marginparwidth = min{inner margin,outer margin}
    else if marginpar in outer margin then
      marginparwidth = outer margin
    else if marginpar in inner margin then
      marginparmargin = inner margin
    end if
  else
    if marginpar in left margin then
      marginparwidth = inner margin
    else
      marginparwidth = outer margin
    end if
  end if
end if
marginparwidth = marginparwidth - 2marginparsep
if marginparwidth < 1pt then
  marginparwidth = 1pt
end if
\end{verbatim}
  \end{small}
\end{framed}


%#% extend

%#% extstart input backend.tex
