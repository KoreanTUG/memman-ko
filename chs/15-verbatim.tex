%%%%%%%%%%%%%%%%%%%%%%%%%%%%%%%%%%%%%%%%%%%%%%%%%%%%%%%%%
\chapter{Boxes, verbatims and files} \label{chap:bvf}\label{chap:boxes}
%%%%%%%%%%%%%%%%%%%%%%%%%%%%%%%%%%%%%%%%%%%%%%%%%%%%

    The title of this chapter indicates that it deals with three 
disconnected topics, but there is method in the seeming peculiarity.
By the end of the chapter you will be able to write \ltx\ code that
lets you put things in your document source at one place and have them
typeset at a different place, or places. For example, if you are writing
a text book that includes questions and answers then you could write 
a question and answer together yet have the answer typeset at the
end of the book. 

    Writing in one place and printing in another is based on outputting
stuff to a file\index{file!write} and then inputting 
it\index{file!read} for processing at another place 
or time. This is just how \ltx\ produces the \toc. It is often important
when writing to a file that \ltx\ does no processing of any macros, which
implies that we need to be able to write verbatim. One use of verbatim
in \ltx\ is to typeset computer code or the like, and to clearly
distinguish the code from the main text it is often typeset within a box.
Hence the chapter title.

    The class extends the kinds of boxes\index{box} normally provided, 
extends the default verbatims\index{verbatim}, and provides a simple means 
of writing\index{file!write} and reading\index{file!read} files.

    One problem with verbatims\index{verbatim!in argument} is that they 
can not be used as part of
an argument to a command. For example to typeset something in a 
framed\index{frame!minipage}
\Ie{minipage} the obvious way is to use the \Ie{minipage} as the argument
to the \cmd{\fbox} macro:
\begin{lcode}
\fbox{\begin{minipage}{6cm} 
      Contents of framed minipage 
      \end{minipage}}
\end{lcode}
This works perfectly well until the contents includes some verbatim
material, whereupon you will get nasty 
error\index{\cs{verb} illegal in command argument} messages. However this 
particular conundrum is solvable, even if the solution is not particularly
obvious. Here it is.

    We can put things into a box, declared via \cmd{\newsavebox}, and typeset
the contents of the box later via \cmd{\usebox}. The most common way
of putting things into a save box is by the \cmd{\sbox} or \cmd{\savebox}
macros, but as the material for saving is one of the arguments to these 
macros this approach fails. But, \Ie{lrbox} is an environment form of
\cmd{\sbox}, so it can handle verbatim material. The code below,
after getting a new save box, defines a new \Ie{framedminipage} 
environment\index{frame!minipage!verbatim} which is used just 
like the standard \Ie{minipage}. 
The \Ie{framedminipage} 
starts an \Ie{lrbox} environment and then
starts a \Ie{minipage} environment, after which comes the contents.
At the end it closes the two environments and calls \cmd{\fbox} with its
argument being the contents of the saved box \emph{which have already been
typeset}.
\begin{lcode}
\newsavebox{\minibox}
\newenvironment{framedminipage}[1]{%
  \begin{lrbox}{\minibox}\begin{minipage}{#1}}%
  {\end{minipage}\end{lrbox}\fbox{\usebox{\minibox}}}
\end{lcode}

\vspace{\onelineskip}
\noindent\textbf{Question 1.} Can you think of any improvements to
  the definition of the \Ie{framedminipage} environment?

\vspace{\onelineskip}
\noindent\textbf{Question 2.} An answer to question 1 is at the end of this
  chapter. Suggest how it was put there.



\section{Boxes}

\index{box!framed|(}
\index{frame!box|(}

    \ltx\ provides some commands to put a box round some text. The class
extends the available kinds of boxes.


\begin{syntax}
\senv{framed} text \eenv{framed} \\
\senv{shaded} text \eenv{shaded} \\
\senv{snugshade} text \eenv{snugshade} \\
%%%%%\cmd{\frameasnormaltrue} \cmd{\frameasnormalfalse} \\
\end{syntax}
\glossary(framed)%
  {\senv{framed}}%
  {Put a ruled box around the contents of the environment; the box can include
   pagebreaks.}
\glossary(shaded)%
  {\senv{shaded}}%
  {Put a colored background behind the contents of the environment, which
   can include pagebreaks. The color extends into the margins a little.}
\glossary(snugshade)%
  {\senv{snugshade}}%
  {Like \Pe{shaded} but does not bleed into the margins.}
The \Ie{framed}, \Ie{shaded}, and \Ie{snugshade}  
environments, which were created by Donald Arseneau\index{Arseneau, Donald} 
as part of his \Lpack{framed} package~\cite{FRAMED},
put their contents
into boxes\index{box!include pagebreak} that break across pages. 
The \Ie{framed} environment delineates
the box by drawing a rectangular frame. If there is a pagebreak in
the middle of the box, frames are drawn on both pages.

    The \Ie{shaded} environment typesets the box with a 
shaded\index{box!shaded background} or
colored background. This requires the use of the \Lpack{color} 
package~\cite{COLOR}, which is one of the required \ltx\ packages,
or the \Lpack{xcolor} package~\cite{XCOLOR}.
The shading color is \texttt{shadecolor}, which you have to define before
using the environment. For
example, to have a light gray background:
\begin{lcode}
\definecolor{shadecolor}{gray}{0.9}
\end{lcode}
For complete information on this see the documentation for the
\Lpack{color} or \Lpack{xcolor} packages, or one of the \ltx\ books like the
\textit{Graphics Companion}~\cite{GCOMPANION}.
In the \Ie{snugshaded} environment the box clings more closely to its
contents than it does in the \Ie{shaded} environment.

\begin{recommended}
  Since the class was originally written, much have happened in the
  gfx generating capabilities in LaTeX, especially the popularity of
  TikZ has provided many more extensive box and graphics generating
  packages.

  As of 2018 one of the most impressive packages for all sorts of
  boxes is the \Lpack{tcolorbox} package by Thomas~F.~Sturm.
\end{recommended}



%%    Following the declaration \cmd{\frameasnomaltrue}, which is the
%default, normal paragraphing is used for the framed text. On the
%%other hand, following the declaration \cmd{\frameasnormalfalse}
%the paragraphing follows the \Ie{minipage} style layout (i.e.,
%%no indentation of the first line).

Be aware that the boxes we present in this manual are somewhat
delicate; they do not work in all circumstances. For example they will
not work with the \Lpack{multicol} package~\cite{MULTICOL}, and any
floats or footnotes in the boxes will disappear.

\begin{syntax}
\lnc{\FrameRule} \lnc{\FrameSep} \lnc{\FrameHeightAdjust} \\
\end{syntax}
\glossary(FrameRule)%
  {\cs{FrameRule}}%
  {Thickness of the rules around an \Pe{framed} environment.}
\glossary(FrameSep)%
  {\cs{FrameSep}}%
  {Separation between the surrounding box and text in a \Pe{framed} or
   \Pe{shaded} environment.}
\glossary(FrameHeightAdjust)%
  {\cs{FrameHeightAdjust}}%
  {Height of the top of a frame in a \Pe{framed} environment
  above the baseline at the top of a page.}
The \Ie{framed} environment puts the text into an `\cmd{\fbox}' with
the settings:
\begin{lcode}
\setlength{\FrameRule}{\fboxrule}
\setlength{\FrameSep}{3\fboxsep}
\end{lcode}
The macro \cmd{\FrameHeightAdjust} specifies the height of the top of the frame
above the baseline at the top of a page; its initial definition is:
\begin{lcode}
\providecommand*{\FrameHeightAdjust}{0.6em}
\end{lcode}

\index{frame!box!styling}
\begin{syntax}
\cmd{\MakeFramed}\marg{settings} \cmd{\endMakeFramed} \\
\cmd{\FrameCommand} \cmd{\FrameRestore} \\
\end{syntax}
\glossary(MakeFramed)%
  {\cs{MakeFramed}\marg{settings}}%
  {The \Pe{MakeFramed} environment is the workhorse for the
   \Pe{framed}, \Pe{shaded}, etc., environments.
   The \meta{settings} argument controls the final appearance and
   should include a \cs{FrameRestore} to reset things back to normal.}
\glossary(endMakeFramed)%
  {\cs{endMakeFramed}}%
  {Ends the \Pe{MakeFramed} environment.}
\glossary(FrameCommand)%
  {\cs{FrameCommand}}%
  {Draws a `frame'.}
\glossary(FrameRestore)%
  {\cs{FrameRestore}}%
  {Restores settings after a `frame'.}
Internally, the environments are specified using the \Ie{MakeFramed}
environment. The \meta{setting} should contain any adjustments to the 
text width
(applied to \lnc{\hsize} and using the \lnc{\width} of the frame itself)
and a `restore' command, which is normally the provided \cmd{\FrameRestore}
macro. The frame itself is drawn via the 
\cmd{\FrameCommand}, which can be changed to obtain other boxing styles. The
default definition equates to an \cmd{\fbox} and is:
\begin{lcode}
\newcommand*{\FrameCommand}{%
  \setlength{\fboxrule}{\FrameRule}\setlength{\fboxsep}{\FrameSep}%
  \fbox}
\end{lcode}
For example, the \Ie{framed}, \Ie{shaded} and \Ie{snugshade} environments 
are defined as
\begin{lcode}
\newenvironment{framed}{% % uses default \FrameCommand
  \MakeFramed{\advance\hsize -\width \FrameRestore}}%
  {\endMakeFramed}
\newenvironment{shaded}{% % redefines \FrameCommand as \colorbox
  \def\FrameCommand{\fboxsep=\FrameSep \colorbox{shadecolor}}%
  \MakeFramed{\FrameRestore}}%
  {\endMakeFramed}
\newenvironment{snugshade}{% A tight version of shaded
  \def\FrameCommand{\colorbox{shadecolor}}%
  \MakeFramed{\FrameRestore\@setminipage}}%
  {\par\unskip\endMakeFramed}
\end{lcode}

    If you wanted a narrow, centered, framed\index{frame!narrow box} 
environment you could do something like this:
\begin{lcode}
\newenvironment{narrowframed}{%
  \MakeFramed{\setlength{\hsize}{22pc}\FrameRestore}}%
  {\endMakeFramed}
\end{lcode}
where \texttt{22pc} will be the width of the new framed environment.


\begin{syntax}
\senv{leftbar} text \eenv{leftbar} \\
\end{syntax}
\glossary(leftbar)%
  {\senv{leftbar}}%
  {Draws a thick vertical line in the left margin alongside the contents 
   of the environment.}

The \Ie{leftbar} environment draws a thick vertical line at the 
left\index{rule!in margin} of the text. It is defined as
\begin{lcode}
\newenvironment{leftbar}{%
  \def\FrameCommand{\vrule width 3pt \hspace{10pt}}%
  \MakeFramed{\advance\hsize -\width \FrameRestore}}%
  {\endMakeFramed}
\end{lcode}

    By changing the \meta{setting} for \cmd{\MakeFramed} and the
definition of \cmd{\FrameCommand} you can obtain a variety of framing
styles. For instance, to have rounded corners to the 
frame\index{frame!rounded corners} instead of
the normal sharp ones, you can use the \Lpack{fancybox} 
package~\cite{FANCYBOX} and the following code:
\begin{lcode}
\usepackage{fancybox}
\newenvironment{roundedframe}{%
  \def\FrameCommand{%
    \cornersize*{20pt}%
    \setlength{\fboxsep}{5pt}%
    \ovalbox}%
  \MakeFramed{\advance\hsize-\width \FrameRestore}}%
  {\endMakeFramed}
\end{lcode}

\index{frame!title|(}
     A framed environment is normally used to distinguish its contents
from the surrounding text. A title for the environment may be useful, and
if there was a pagebreak in the middle, a title on the continuation could
be desireable. Doing this takes a bit more work than I have shown so far.
This first part was inspired by a posting to \ctt\ by 
Donald Arseneau\index{Arseneau, Donald}.\footnote{On 2003/10/24 in the thread
\textit{framed.sty w/heading?}. The particulars are no longer applicable as 
the framing code in question then has since been revised.}.
\PWnote{2009/06/25}{Rewrote section and code for titled frames.}
\begin{comment}
This first part is from a posting to \ctt\ by 
Donald Arseneau\index{Arseneau, Donald}.\footnote{On 2003/10/24 in the thread
\textit{framed.sty w/heading?}}.

\begin{lcode}
\newcommand{\FrameTitle}[2]{%
  \fboxrule=\FrameRule \fboxsep=\FrameSep
  \fbox{\vbox{\nobreak \vskip -0.7\FrameSep
    \rlap{\strut#1}\nobreak\nointerlineskip% left justified
    \vskip 0.7\FrameSep
    \hbox{#2}}}}
\newenvironment{framewithtitle}[2][\FrameFirst@Lab\ (cont.)]{%
  \def\FrameFirst@Lab{\textbf{#2}}%
  \def\FrameCont@Lab{\textbf{#1}}%
  \def\FrameCommand##1{%
    \FrameTitle{\FrameCurrent@Lab}{##1}%
    \global\let\FrameCurrent@Lab\FrameNext@Lab
    \global\let\FrameNext@Lab\FrameCont@Lab
  }%
  \global\let\FrameCurrent@Lab\FrameFirst@Lab
  \global\let\FrameNext@Lab\FrameFirst@Lab
  \MakeFramed{\advance\hsize-\width \FrameRestore}}%
  {\endMakeFramed}
\end{lcode}
\end{comment}

\begin{lcode}
\newcommand{\FrameTitle}[2]{%
  \fboxrule=\FrameRule \fboxsep=\FrameSep
  \fbox{\vbox{\nobreak \vskip -0.7\FrameSep
    \rlap{\strut#1}\nobreak\nointerlineskip% left justified
    \vskip 0.7\FrameSep
    \hbox{#2}}}}
\newenvironment{framewithtitle}[2][\FrameFirst@Lab\ (cont.)]{%
  \def\FrameFirst@Lab{\textbf{#2}}%
  \def\FrameCont@Lab{\textbf{#1}}%
  \def\FrameCommand##1{%
    \FrameTitle{\FrameFirst@Lab}{##1}}%
  \def\FirstFrameCommand##1{%
    \FrameTitle{\FrameFirst@Lab}{##1}}%
  \def\MidFrameCommand##1{%
    \FrameTitle{\FrameCont@Lab}{##1}}%
  \def\LastFrameCommand##1{%
    \FrameTitle{\FrameCont@Lab}{##1}}%
  \MakeFramed{\advance\hsize-\width \FrameRestore}}%
  {\endMakeFramed}
\end{lcode}

The \Ie{framewithtitle} environment, which is the end goal of this
exercise, acts like the \Ie{framed} environment except that it puts
a left-justified title just after the top of the frame box and before the
regular contents.
\begin{syntax}
\senv{framewithtitle}\oarg{cont-title}\marg{title} text \\
 \eenv{framewithtitle} \\
\end{syntax}
The \meta{title} is set in a bold font. If the optional \meta{cont-title}
argument is given then \meta{cont-title} is used as the title on
any suceeding pages, otherwise the phrase `\meta{title} (cont.)' is used
for the continuation title.

    If you would like the titles centered, replace the line 
marked `left justified' in the code for \cmd{\FrameTitle} with the line:
\begin{lcode}
\rlap{\centerline{\strut#1}}\nobreak\nointerlineskip% centered
\end{lcode}

    The code for the \Ie{frametitle} environment is not obvious. The difficulty
in creating the environment was that the underlying framing code goes through
the `stuff' to be framed by first trying to fit it all onto one page 
(\cs{FrameCommand}). If it does not fit, then it takes as much as will fit
and typesets that using \cs{FirstFrameCommand}, then tries to typeset the 
remainder on the next page. If it all fits then it uses \cs{LastFrameCommand}.
If it doesn't fit, it typesets as much as it can using \cs{MidFrameCommand}, 
and then tries to set the remainder on the following page. The process repeats
until all has been set.

    If you would prefer to have the title at the top outside the frame the 
above code needs adjusting.
\begin{comment}
\begin{lcode}
\newcommand{\TitleFrame}[2]{%
  \fboxrule=\FrameRule \fboxsep=\FrameSep
  \vbox{\nobreak \vskip -0.7\FrameSep
    \rlap{\strut#1}\nobreak\nointerlineskip% left justified
    \vskip 0.7\FrameSep
    \noindent\fbox{#2}}}
\newenvironment{titledframe}[2][\FrameFirst@Lab\ (cont.)]{%
  \def\FrameFirst@Lab{\textbf{#2}}%
  \def\FrameCont@Lab{\textbf{#1}}%
  \def\FrameCommand##1{%
    \TitleFrame{\FrameCurrent@Lab}{##1}
    \global\let\FrameCurrent@Lab\FrameNext@Lab
    \global\let\FrameNext@Lab\FrameCont@Lab
  }%
  \global\let\FrameCurrent@Lab\FrameFirst@Lab
  \global\let\FrameNext@Lab\FrameFirst@Lab
  \MakeFramed{\hsize\textwidth
              \advance\hsize -2\FrameRule
              \advance\hsize -2\FrameSep
              \FrameRestore}}%
  {\endMakeFramed}
\end{lcode}
\end{comment}

\begin{lcode}
\newcommand{\TitleFrame}[2]{%
  \fboxrule=\FrameRule \fboxsep=\FrameSep
  \vbox{\nobreak \vskip -0.7\FrameSep
    \rlap{\strut#1}\nobreak\nointerlineskip% left justified
    \vskip 0.7\FrameSep
    \noindent\fbox{#2}}}
\newenvironment{titledframe}[2][\FrameFirst@Lab\ (cont.)]{%
  \def\FrameFirst@Lab{\textbf{#2}}%
  \def\FrameCont@Lab{\textbf{#1}}%
  \def\FrameCommand##1{%
    \TitleFrame{\FrameFirst@Lab}{##1}}
  \def\FirstFrameCommand##1{%
    \TitleFrame{\FrameFirst@Lab}{##1}}
  \def\MidFrameCommand##1{%
    \TitleFrame{\FrameCont@Lab}{##1}}
  \def\LastFrameCommand##1{%
    \TitleFrame{\FrameCont@Lab}{##1}}
  \MakeFramed{\hsize\textwidth
              \advance\hsize -2\FrameRule
              \advance\hsize -2\FrameSep
              \FrameRestore}}%
  {\endMakeFramed}
\end{lcode}

\begin{syntax}
\senv{titledframe}\oarg{cont-title}\marg{title} text \eenv{titledframe} \\
\end{syntax}
The \Ie{titledframe} environment is identical to \Ie{framewithtitle}
except that the title is placed just before the frame. Again, if
you would like a centered title, replace the line marked `left justified'
in \cmd{\TitleFrame} by
\begin{lcode}
\rlap{\centerline{\strut#1}}\nobreak\nointerlineskip% centered
\end{lcode}

    You can adjust the code for the \Ie{framewithtitle} and \Ie{titledframe}
environments to suit your own purposes, especially as they are not 
part of the class so you would have to type them in yourself anyway
if you wanted to use them, using whatever names you felt suitable.

\index{frame!title|)}

    The class provides two further environments in addition to those
from the \Lpack{framed} package.
\begin{syntax}
\senv{qframe} text \eenv{qframe} \\
\senv{qshade} text \eenv{qshade} \\
\end{syntax}

   When used within, say, a \Ie{quotation} environment, the \Ie{framed}
and \Ie{shaded} environments do not closely box the indented text. The
\Ie{qframe} and \Ie{qshade} environments do provide close 
boxing.\footnote{Donald Arseneau has said that he may put something similar
in a later version the \Ie{framed} package.}
The difference can be seen in the following \Ie{quotation}.

\begin{quotation}
This is the start of a \Ie{quotation} environment. It forms the basis showing
the difference between the \Ie{framed} and \Ie{qframe} environments.

\begin{qframe}
This is the second paragraph in the \Ie{quotation} environment and in turn it 
is within the \Ie{qframe} environment.
\end{qframe}

\begin{framed}
This is the third paragraph in the \Ie{quotation} environment and in turn it 
is within the \Ie{framed} environment.
\end{framed}

This is the fourth and final paragraph within the \Ie{quotation} environment
and is not within either a \Ie{qfame} or \Ie{framed} environment.
\end{quotation}

    If you want to put a frame inside an \Ie{adjustwidth} environment
then you may well find that \Ie{qframe} or \Ie{qshade} meet your
expections better than \Ie{framed} of \Ie{shaded}. Of course, it does
depend on what your expectations are.

\index{frame!box|)}
\index{box!framed|)}

\section{Long comments}

    The \% comment character can be used to comment out (part of) a
line of \tx\ code, but this gets tedious if you need to comment out
long chunks of code.

\begin{syntax}
\senv{comment} text to be skipped over \eenv{comment} \\
\end{syntax}
\glossary(comment)%
  {\senv{comment}}%
  {Skip over the environment.}
\index{comment out text}
As an extreme form of font changing, although it doesn't actually work that 
way, anything in a \Ie{comment} environment will not appear in the
document; effectively, \ltx\ throws it all away. This can be useful
to temporarily discard chunks of stuff instead of having to mark each line
with the \% comment character. 

\begin{syntax}
\cmd{\newcomment}\marg{name} \\
\cmd{\commentsoff}\marg{name} \\
\cmd{\commentson}\marg{name} \\
\end{syntax}
\glossary(newcomment)%
  {\cs{newcomment}\marg{name}}%
  {Define a new comment environment called \meta{name}.}
\glossary(commentsoff)%
  {\cs{commentsoff}\marg{name}}%
  {Process contents of the \meta{name} comment environment.}
\glossary(commentson)%
  {\cs{commentson}\marg{name}}%
  {Skip contents of the \meta{name} comment environment.}
The class lets you define your own comment environment via the 
\cmd{\newcomment} command which defines a comment environment called
\meta{name}. In fact the class itself uses \verb?\newcomment{comment}? to
define the \Ie{comment} environment. A comment environment \meta{name}
may be switched off so that its contents are not ignored by using the
\cmd{\commentsoff} declaration. It may be switched on later by the
\cmd{\commentson} declaration. In either case \meta{name} must have
been previously declared as a comment environment via \cmd{\newcomment}.

    Suppose, for example, that you are preparing a draft document for 
review by some others and you want to include some notes for the reviewers.
Also, you want to include some private comments in the source for yourself.
You could use the \Ie{comment} environment for your private comments and
create another environment for the notes to the reviewers. These notes
should not appear in the final document. Your source might then look like:
\begin{lcode}
\newcomment{review}
\ifdraftdoc\else
  \commentsoff{review}
\fi
...
\begin{comment}
Remember to finagle the wingle!
\end{comment}
...
\begin{review}
\textit{REVIEWERS: Please pay particular attention to this section.}
\end{review}
...
\end{lcode}

    Comment environments cannot be nested, nor can they overlap. 
The environments in the code below will not work in the manner that might
be expected:
\begin{lcode}
\newcomment{acomment} \newcomment{mycomment}
\begin{comment}
  \begin{acomment} %% comments cannot be nested
  ...
  \end{acomment}
  ...
  \begin{mycomment}
  ...
\end{comment}
...
  \end{mycomment}  %% comments cannot overlap
\end{lcode}

    More encompassing \Ie{comment} environments are available if you
use Victor Eijkhout's \Lpack{comment} package~\cite{COMMENT}.


\section{Verbatims}

\index{verbatim|(}

    Standard \ltx\ defines the \cmd{\verb} and \cmd{\verb*} commands
for typesetting short pieces of text verbatim, short because they
cannot include a linebreak. For longer verbatim texts the
\Ie{verbatim} or \Ie{verbatim*} environments can be used. The star forms
indicate spaces in the verbatim text by outputing a \verb*? ? mark for 
each space. The class extends the standard verbatims in various ways.

\index{verbatim!short}
    If you have to write a lot of \cmd{\verb} text, as I have had to do for
this book, it gets tedious to keep on typing this sort of thing:
\verb?\verb!verbatim text!?. Remember that the character immediately after
the \cmd{\verb}, or \cmd{\verb*}, ends the verbatim processing.
\begin{syntax}
\cmd{\MakeShortVerb}\marg{backslash-char} \\
\cmd{\DeleteShortVerb}\marg{backslash-char} \\
\end{syntax}
\glossary(MakeShortVerb)%
  {\cs{MakeShortVerb}\marg{backslash-char}}%
  {Makes \meta{char} a shorthand for \cs{verb}\meta{char}.}
\glossary(DeleteShortVerb)%
  {\cs{DeleteShortVerb}\marg{backslash-char}}%
  {Returns \meta{char} to its normal meaning instead of being a shorthand 
   for \cs{verb}\meta{char}.}
The \cmd{\MakeShortVerb} macro takes a character preceded by a backslash
as its argument, say \verb?\!?, and makes that character equivalent to 
\verb?\verb!?. Using the character a second time will stop the verbatim 
processing.
Doing, for example \verb?\MakeShortVerb{\!}?, lets you then use 
\verb?!verbatim text!?
instead of the longer winded \verb?\verb!verbatim text!?. 

    You have to pick
as the short verb character one that you are unlikely to use; a good choice
is often the \verb?|? bar character as this rarely used in normal text.
This choice, though may be unfortunate if you want to have any tabulars with
vertical lines, as the bar character is used to specify those. The
\cmd{\DeleteShortVerb} macro is provided for this contingency; give it the
same argument as an earlier \cmd{\MakeShortVerb} and it will restore
the short verb character to its normal state.

    The \cmd{\MakeShortVerb} and \cmd{\DeleteShortVerb} macros come from the
\Lpack{shortvrb} package which is part of the \ltx\ base system, but I 
have found them so convenient that I added them to the class.

\begin{syntax}
\cmd{\setverbatimfont}\marg{font-declaration} \\
\end{syntax}
\glossary(setverbatimfont)%
  {\cs{setverbatimfont}\marg{fontspec}}%
  {Sets the font to be used for verbatim text.}
The default font\index{verbatim!changing font} for verbatims is the normal 
sized monospaced font. The declaration
\cmd{\setverbatimfont} can be used to specify a different font.
The class default is 
\begin{lcode}
\setverbatimfont{\normalfont\ttfamily}
\end{lcode}
To use a smaller version simply say 
\begin{lcode}
\setverbatimfont{\normalfont\ttfamily\small}
\end{lcode}

    A monospaced font is normally chosen as verbatim text is often used 
to present program code or typewritten text. If you want a more exotic
font, try this
\begin{lcode}
\setverbatimfont{\fontencoding{T1}\fontfamily{cmss}\selectfont}
\end{lcode}
and your verbatim text will then look like %%
\setverbatimfont{\fontencoding{T1}\fontfamily{cmss}\selectfont}
\begin{verbatim}
We are no longer using the boring old typewriter font
for verbatim text. We used the T1 encoding 
to make sure that characters that are often ligatures
like ``, or '', or ---, or <, or >, print as expected.
After this we will switch back to the default verbatim font via
\setverbatimfont{\normalfont\ttfamily}
\end{verbatim}
\setverbatimfont{\normalfont\ttfamily}
In the normal way of things with an OT1 fontencoding, 
typesetting the ligatures mentioned above
in the sans font produces: 
{\fontencoding{OT1}\fontfamily{cmss}\selectfont ligatures like ``, or '', or ---, or <, or >}, 
which is not what happens in the \cmd{\verbatim} environment.


\begin{syntax}
\senv{verbatim} anything \eenv{verbatim} \\
\senv{verbatim*} anything \eenv{verbatim*} \\
\end{syntax}
\glossary(verbatim)%
  {\senv{verbatim}}%
  {Typeset the contents verbatim.}
In the \Ie{verbatim} environment\footnote{This version of the \Ie{verbatim}
environment is heavily based on the \Lpack{verbatim} package~\cite{VERBATIM}
but does provide some extensions.}
 you can write anything you want (except
\eenv{verbatim}), and it will be typeset exactly as written. The \Ie{verbatim*}
environment is similar except, like with \cmd{\verb*}, spaces will be
indicated with a \verb*? ? mark.

\begin{syntax}
\cmd{\tabson}\oarg{number} \\
\cmd{\tabsoff} \\
\end{syntax}
\glossary(tabson)%
  {\cs{tabson}\oarg{number}}%
  {Set \meta{number} of spaces in a verbatim for a TAB character;
   default 4.}
\glossary(tabsoff)%
  {\cs{tabsoff}}%
  {Ignore extra TAB spaces in a verbatim.} 
\index{verbatim!with tab spaces}
The standard \Ie{verbatim} environment ignores any TAB characters; with
the class's environment after calling the \cmd{\tabson} declaration 
the environment will handle TAB characters. By default 4 spaces are used
to represent a TAB; the optional \meta{number} argument to the declaration
will set the number of spaces for a TAB to be \meta{number}.
Some folk like to use 8 spaces for a TAB, in which case they would need
to declare \verb?\tabson[8]?. Unremarkably, the declaration \cmd{\tabsoff}
switches off TABs. The class default is \cmd{\tabsoff}.

\begin{syntax}
\cmd{\wrappingon} \\
\cmd{\wrappingoff} \\
\lnc{\verbatimindent} \\
\cmd{\verbatimbreakchar}\marg{char} \\
\end{syntax}
\glossary(wrappingon)%
  {\cs{wrappingon}}%
  {Wrap overlong verbatim lines.}
\glossary(wrappingoff)%
  {\cs{wrappingoff}}%
  {The normal behaviour of not wrapping overlong verbatim lines.}
\glossary(verbatimindent)%
  {\cs{verbatimindent}}%
  {Indent for wrapped overlong verbatim lines.}
\glossary(verbatimbreakchar)%
  {\cs{verbatimbreakchar}\marg{char}}%
  {Character indicating a verbatim line is being wrapped.}
As noted, whatever is written in a \Ie{verbatim} environment is output
just as written, even if lines are too long\index{verbatim!wrap long lines} 
to fit on the page. The
declaration \cmd{\wrappingon} lets the environment break lines so that they
do not overflow. The declaration \cmd{\wrappingoff} restores the normal
behaviour.

    The following is an example of how a wrapped verbatim line looks. In
the source the contents of the \Ie{verbatim} was written as a single line.
\wrappingon
\begin{verbatim}
This is an example of line wrapping in the verbatim environment. It is a single line in the source and the \wrappingon declaration has been used.
\end{verbatim}
\wrappingoff

   The wrapped portion of verbatim lines are indented from the left margin
by the length \lnc{\verbatimindent}. The value can be changed by the usual
length changing commands. The end of each line that has been wrapped is marked
with the \meta{char} character of the \cmd{\verbatimbreakchar} macro.
The class default is \verb?\verbatimbreakchar{\char`\%}?, so that lines are 
marked with \verb?%?.
To put a `/' mark at the end of wrapped lines you can do
\begin{lcode}
\setverbatimbreak{\char'\/}
\end{lcode}
or similarly if you would like another character. Another possibility
is
\begin{lcode}
\setverbatimchar{\char'\/\char'\*}
\end{lcode}
which will make `/*' the end marker.

\subsection{Boxed verbatims}

    Verbatim environments are often used to present program code or, as
in this book, \ltx\ code. For such applications it can be useful to
put the code in a box, or to number the code lines, or perhaps both.

\begin{syntax}
\senv{fboxverbatim} anything \eenv{fboxverbatim} \\
\end{syntax}
\glossary(fboxverbatim)%
  {\senv{fboxverbatim}}%
  {Puts a frame around the verbatim material. Page breaks are not allowed.}
The \Ie{fboxverbatim} environment\index{frame!verbatim}\index{verbatim!frame} 
typesets its contents verbatim and
puts a tightly fitting frame around the result; in a sense it is similar
to the \cmd{\fbox} command.

\begin{syntax}
\senv{boxedverbatim} anything \eenv{boxedverbatim} \\
\senv{boxedverbatim*} anything \eenv{boxedverbatim*} \\
\end{syntax}
\glossary(boxedverbatim)%
  {\senv{boxedverbatim}}%
  {May put a box around the verbatim material; lines may be numbered and page
   breaks are allowed.}
\glossary(boxedverbatim*)%
  {\senv{boxedverbatim*}}%
  {May put a box around the verbatim* material; lines may be numbered and page
   breaks are allowed.}
The \Ie{boxedverbatim} and \Ie{boxedverbatim*} environments are like
the \Ie{verbatim} and \Ie{verbatim*} environments except that a box,
allowing page breaks, may be put around the verbatim text and the lines
of text\index{line number} may be numbered.\index{boxed verbatim}\index{numbered lines}
The particular format of the output can be 
controlled as described below.
\begin{syntax}
\cmd{\bvbox} \cmd{\bvtopandtail} \cmd{\bvsides} \cmd{\nobvbox} \\
\lnc{\bvboxsep} \\
\end{syntax}
\glossary(bvbox)%
  {\cs{bvbox}}%
  {Rectangular boxes will be drawn for \Pe{boxedverbatim} environments.}
\glossary(nobvbox)%
  {\cs{nobvbox}}%
  {\Pe{boxedverbatim} environments will not be framed in any way.}
\glossary(bvtopandtail)%
  {\cs{bvtopandtail}}%
  {Draw horizontal rules before and after \Pe{boxedverbatim} environments.}
\glossary(bvsides)%
  {\cs{bvsides}}%
  {Draw vertical rules on each side of \Pe{boxedverbatim} environments.}
\glossary(bvboxsep)%
  {\cs{bvboxsep}}%
  {Separation between text and framing in \Pe{boxedverbatim} environments.}
Four styles of boxes are provided and you can extend these. Following
the \cmd{\bvbox} declaration, a box is drawn round the verbatim text, breaking
at page boundaries if necessary; this is the default style. Conversely,
no boxes are drawn after the \cmd{\nobvbox} declaration. With the
\cmd{\bvtopandtail} declaration horizontal lines are drawn at the start and 
end of the verbatim text, and with the \cmd{\bvsides} declarations, vertical
lines are drawn at the left and right of the text. The separation between
the lines and the text is given by the length \lnc{\bvboxsep}.

    The following hooks are provided to set your own 
boxing\index{frame!verbatim!styling}\index{verbatim!frame!styling} style.
\begin{syntax}
\cmd{\bvtoprulehook} \cmd{\bvtopmidhook} \cmd{\bvendrulehook} \\
\cmd{\bvleftsidehook} \cmd{\bvrightsidehook} \\
\end{syntax}
\glossary(bvtoprulehook)%
  {\cs{bvtoprulehook}}%
  {Called at the start of a \Pe{boxedverbatim} environment and after a pagebreak.}
\glossary(bvtopmidhook)%
  {\cs{bvtopmidhook}}%
  {Called after \cs{bvtoprulehook} at the start of a \Pe{boxedverbatim} environment.}
\glossary(bvendrulehook)%
  {\cs{bvendrulehook}}%
  {Called at the end of a \Pe{boxedverbatim} environment, and before a pagebreak.}
\glossary(bvleftsidehook)%
  {\cs{bvleftsidehook}}%
  {Called before each line in a \Pe{boxedverbatim} environment.}
\glossary(bvrightsidehook)%
  {\cs{bvrightsidehook}}%
  {Called after each line in a \Pe{boxedverbatim} environment.}
The macros \cmd{\bvtoprulehook} and \cmd{\bvendrulehook} are called at
the start and end of the \Ie{boxedverbatim} environment, and before and after
page breaks. The macros
\cmd{\bvleftsidehook} and \cmd{\bvrightsidehook} are called at the start
and end of each verbatim line. The macro \cmd{\bvtopmidhook} is
called after \cmd{\bvtoprulehook} at the start of the environment.
It can be used to add some space if \cmd{\bvtoprulehook} is empty.

\begin{syntax}
\cmd{\bvperpagetrue} \cmd{\bvperpagefalse} \\
\cmd{\bvtopofpage}\marg{text} \cmd{\bvendofpage}\marg{text} \\
\end{syntax}
\glossary(bvperpagetrue)%
  {\cs{bvperpagetrue}}%
  {Visibly break a \Pe{boxedverbatim} at a page break using \cs{bvtopofpage}
   and \cs{bvendofpage}.}
\glossary(bvperpagefalse)%
  {\cs{bvperpagefalse}}%
  {Do not mark page breaks in a \Pe{boxedverbatim}.}
\glossary(bvtopofpage)%
  {\cs{bvtopofpage}\marg{text}}%
  {Use \meta{text} as the \Pe{boxedverbatim} page break marker at the top of 
   a page.}
\glossary(bvendofpage)%
  {\cs{bvendofpage}\marg{text}}%
  {Use \meta{text} as the \Pe{boxedverbatim} page break marker at the bottom 
   of a page.}
The command \cmd{\bvperpagetrue} indicates
that a box should be visibly broken at a pagebreak, while there should
be no visible break for \cmd{\bvperpagefalse}. 
If the box continues on to another page then it may be advantageous
to place some sort of heading before the verbatim continues. Following
the declaration \cmd{\bvperpagetrue} the \meta{text} argument to
\cmd{\bvtopofpage} will be typeset after any pagebreak. For example you
could set:
\begin{lcode}
\bvtopofpage{continued}
\end{lcode}
to print `continued' in the normal text font. 

By default, the class sets
\begin{lcode}
\bvendofpage{\hrule\kern-.4pt}
\end{lcode}
which causes the \cmd{\hrule} to be drawn at the end of a page as the
visible break (the rule is 0.4pt thick and the kern backs up
that amount after the rule, so it effectively takes no vertical space).
This is not always suitable. For instance, if there will be
a `continued' message at the top of the following page it may seem odd
to draw a line at the bottom of the previous page. In this case, setting
\begin{lcode}
\bvendofpage{}
\end{lcode}
will eliminate the rule.

As examples of the use of
these hooks, here is how some of the boxed verbatim styles are defined.

The default style is \cmd{\bvbox}, 
which puts separate full boxes on each page. 
\begin{lcode}
\newcommand{\bvbox}{%
  \bvperpagetrue
  \renewcommand{\bvtoprulehook}{\hrule \nobreak \vskip-.1pt}%
  \renewcommand{\bvleftsidehook}{\vrule}%
  \renewcommand{\bvrightsidehook}{\vrule}%
  \renewcommand{\bvendrulehook}{\hrule}%
  \renewcommand{\bvtopmidhook}{\rule{0pt}{2\fboxsep} \hss}%
}
\end{lcode}
The \cmd{\nobvbox} turns off all boxing, and is defined as
\begin{lcode}
\newcommand{\nobvbox}{%
  \bvperpagefalse
  \renewcommand{\bvtoprulehook}{}%
  \renewcommand{\bvleftsidehook}{}%
  \renewcommand{\bvrightsidehook}{}%
  \renewcommand{\bvendrulehook}{}%
  \renewcommand{\bvtopmidhook}{\rule{0pt}{2\fboxsep} \hss}%
}
\end{lcode}
The definitions of the other styles, \cmd{\bvtopandtail} and \cmd{\bvsides},
are intermediate between \cmd{\bvbox} and \cmd{\nobvbox} in the obvious
manner.


\begin{syntax}
\cmd{\linenumberfrequency}\marg{nth} \\
\cmd{\resetbvlinenumber} \\
\cmd{\setbvlinenums}\marg{first}\marg{startat} \\
\cmd{\linenumberfont}\marg{font declaration} \\
\end{syntax}
\glossary(linenumberfrequency)%
  {\cs{linenumberfrequency}\marg{nth}}%
  {Number every \meta{nth} line in a \Pe{boxedverbatim} or a \Pe{verse}.}
\glossary(resetbvlinenumber)%
  {\cs{resetbvlinenumber}}%
  {Resets the \Pe{boxedverbatim} line number to zero.}
\glossary(setbvlinenums)%
  {\cs{setbvlinenums}\marg{first}\marg{startat}}%
  {The first line of the following \Pe{boxedverbatim} is number \marg{first} 
   and the
   first printed line number should be \meta{startat}.}
\glossary(linenumberfont)%
  {\cs{linenumberfont}\marg{fontspec}}%
  {Specify the font for line numbers.}

The command \cmd{\linenumberfrequency} controls the 
numbering\index{line number!frequency} of lines in
a \Ie{boxedverbatim} --- every \meta{nth} line will be numbered. 
If \meta{nth} is 0 or less, 
then no lines are numbered, if \meta{nth} is 1 then each line is numbered,
and if \meta{nth} is \texttt{n}, where \texttt{n} is 2 or more, then 
only every \texttt{n}th line is numbered. Line numbering is continuous 
from one instance
of the \Ie{boxedverbatim} environment to the next. Outside the environment
the line numbers\index{line number!reset} can be reset at any time by the 
command \cmd{\resetbvlinenumber}.

The \cmd{\setbvlinenums} macro can be
used to specify that the number of the first line of the following 
\Ie{boxedverbatim}
shall be \meta{first} and the first printed number shall be \meta{startat}.

The \cmd{\linenumberfont} declaration sets
\meta{font declaration} as the font\index{line number!font} for the 
line numbers. The default specification for this is:
\begin{lcode}
\linenumberfont{\footnotesize\rmfamily}
\end{lcode}
Line numbers\index{line number!position} are always set at the left of 
the lines because there
is no telling how long a line might be and it might clash with a line number
set at the right.
\begin{syntax}
\cmd{\bvnumbersinside} \\
\cmd{\bvnumbersoutside} \\
\end{syntax}
\glossary(bvnumbersinside)%
  {\cs{bvnumbersinside}}%
  {Line numbers typeset inside a \Pe{boxedverbatim} box.}
\glossary(bvnumbersoutside)%
  {\cs{bvnumbersoutside}}%
  {Line numbers typeset outside a \Pe{boxedverbatim} box.}
Line numbers are typeset inside the box after the declaration 
\cmd{\bvnumberinside} and are typeset outside the box after the
declaration \cmd{\bvnumbersoutside}. The default is to print
the numbers inside the box.

    Verbatim tabbing, but not wrapping, applies to the \Ie{boxedverbatim}
environment.

\begin{recommended}
  Again the \Lpack{tcolorbox} package offers boxes vs verbatim text.
\end{recommended}


\subsection{New verbatims}

\index{verbatim!new|(}
    The class implementation of verbatims lets you define your
own kind of verbatim environment. Unfortunately this is not quite
as simple as saying
\begin{lcode}
\newverbatim{myverbatim}{...}{...} 
\end{lcode}
as you can for defining normal environments. Instead, the general scheme
is
\begin{lcode}
\newenvironment{myverbatim}%
{<non-verbatim stuff> \verbatim <more non-verbatim stuff>}%
{\endverbatim}
\end{lcode}
In particular, you cannot use either the \cmd{\begin} or \cmd{\end}
macros inside the definition of the new verbatim environment. For example,
the following code will not work
\begin{lcode}
\newenvironment{badverbatim}%
  {NBG\begin{verbatim}}{\end{verbatim}}
\end{lcode}
and this won't work either
\begin{lcode}
\newenvironment{badverbatim}%
  {\begin{env}\verbatim}{\endverbatim\end{env}}
\end{lcode}
And, as with the standard \Ie{verbatim} environment, you cannot use
the new one in the definition of a new command.

    For an example of something that does work, this next little piece of 
typesetting was done in a new verbatim environment I have called 
\texttt{verbexami}, which starts and ends with a horizontal rule, and it
shows the definition of \texttt{verbexami}.
\newenvironment{verbexami}%
  {\par\noindent\hrule The verbexami environment \verbatim}%
  {\endverbatim\hrule}

\vspace{0.5\onelineskip}
\begin{verbexami}
\newenvironment{verbexami}%
  {\par\noindent\hrule The verbexami environment \verbatim}%
  {\endverbatim\hrule}
\end{verbexami}
\vspace{0.5\onelineskip}

    And this is a variation on the theme, with the environment again being
enclosed by horizontal rules.
\newenvironment{verbexamii}%
  {\vspace{0.5\baselineskip}\hrule 
   \vspace{0.2\baselineskip} Verbexamii \verbatim \textsc{Is this fun?}}%
  {\endverbatim\hrule\vspace{0.3\baselineskip}}

\vspace{0.5\onelineskip}
\begin{verbexamii}
\newenvironment{verbexamii}%
  {\vspace{0.5\baselineskip}\hrule \vspace{0.2\baselineskip}
    Verbexamii \verbatim \textsc{Is this fun?}}%
  {\endverbatim\hrule\vspace{0.3\baselineskip}}
\end{verbexamii}
\vspace{0.5\onelineskip}

    As no doubt you agree, these are not memorable examples of
the typesetter's art but do indicate that you can define your own
verbatim environments and may need to take a bit of care to get something
that passes muster.

    I will give some more useful examples, but mainly based on environments
for writing verbatim files as I think that these provide a broader
scope. 


\subsection{Example: the \texttt{lcode} environment}

    In this manual all the example \ltx\ code has been typeset in
the \Ie{lcode} environment; this is a verbatim environment defined
especially for the purpose. Below I describe the
code for defining my \Ie{lcode} environment, but first here 
is a simple definition of a verbatim environment, which I will 
call \texttt{smallverbatim},
that uses the \cmd{\small} font\index{verbatim!font} instead of the 
normalsize font.
\begin{lcode}
\newenvironment{smallverbatim}%
  {\setverbatimfont{\normalfont\ttfamily\small}%
   \verbatim}%
  {\endverbatim}
\end{lcode}

    The \Ie{verbatim} environment is implemented as a kind of \Ie{trivlist},
and lists usually have extra vertical space before and after them. For
my environment I did not want any extra spacing\index{list!spaces} 
so I defined the
macro \cmd{\@zeroseps} to zero the relevant list spacings. I also wanted
the code lines to be inset a little, so I defined a new length
called \lnc{\gparindent} to use as the indentation.
\begin{lcode}
\makeatletter
\newcommand{\@zeroseps}{\setlength{\topsep}{\z@}%
                        \setlength{\partopsep}{\z@}%
                        \setlength{\parskip}{\z@}}
\newlength{\gparindent} \setlength{\gparindent}{\parindent}
\setlength{\gparindent}{0.5\parindent}
% Now, the environment itself
\newenvironment{lcode}{\@zeroseps
  \renewcommand{\verbatim@startline}{%
                \verbatim@line{\hskip\gparindent}}
  \small\setlength{\baselineskip}{\onelineskip}\verbatim}%
  {\endverbatim
   \vspace{-\baselineskip}%
  \noindent
 }
\makeatother
\end{lcode}

    Unless you are intimately familiar with the inner workings of the
\Ie{verbatim} processing you deserve an explanation of the \Ie{lcode}
definition.

    Extremely roughly, the code for \cmd{\verbatim} looks like this:
\begin{lcode}
\def\verbatim{%
  \verbatim@font
  % for each line, until \end{verbatim}
    \verbatim@startline 
    % collect the characters in \verbatim@line 
    \verbatim@processline{\the\verbatim@line\par}
    % repeat for the next line
}
\end{lcode}
The code first calls \cmd{\verbatim@font} to set the font to be used.
Then, for each line it does the following:
\begin{itemize}
\item Calls the macro \cmd{\verbatim@startline} to start
      off the output version of the line.
\item Collects all the characters comprising the line 
      as a single token called \cmd{\verbatim@line}.
\item If the characters are the string `\verb?\end{verbatim}?' it finishes
      the verbatim environment.
\item Otherwise it calls the macro \cmd{\verbatim@processline} whose 
      argument is the characters in the 
      line, treated as a paragraph. It then starts all over again with
      the next line.
\end{itemize}

    I configured the \cmd{\verbatim@startline}
macro to indent the line of text using a horizontal skip of \lnc{\gparindent}.
The rest of the initialisation code, before calling \cmd{\verbatim}
to do the real processing, just sets up the vertical spacing. 


\index{verbatim!new|)}

\index{verbatim|)}




\section{Files}

\index{file|(}

    \ltx\ reads and writes various files as it processes a document.
Obviously it reads the document source file, or files, and it writes
the \pixfile{log} file recording what it has done. It also reads and writes
the \pixfile{aux} file, and may read and write other files like a 
\pixfile{toc} file. 

    On occasions it can be useful to get \ltx\ to read and/or write 
other files of your own choosing. Unfortunately standard \ltx\ does
not provide any easy method for doing this. The \Mname\ class
tries to rectify this.

\begin{syntax}
\cmd{\jobname} \\
\end{syntax}
\glossary(jobname)%
  {\cs{jobname}}%
  {The name of the document's main source file.}
When you run \ltx\ on your source file, say \texttt{fred.tex}, \ltx\
stores the name of this file (\texttt{fred}) in the macro \cmd{\jobname}.
\ltx\ uses this to name the various files that it writes out --- the
\pixfile{dvi} or \pixfile{pdf} file, the \pixfile{log} file, the
\pixfile{aux} file, etc.

\index{stream|(}

    \tx\ can read from 16 input streams\index{stream!limited number} 
and can write to 16 output
streams. Normally an input stream\index{stream!input} is allocated for each 
kind of file that will be read\index{file!read} and an 
output\index{stream!output} stream for each kind of file that will
be written\index{file!write}. On the input side, then, at least two 
streams are allocated, one for the source \pixfile{tex} file and 
one for the \pixfile{aux} file. 
On the output side again at least two streams are allocated, one for 
the \pixfile{log} file and one for the \pixfile{aux} file. 
When \pixfile{toc} and other similar
files are also part of the \ltx\ process you can see that many of the
16 input and output streams may be allocated before you can get to use one
yourself.

\begin{syntax}
\cmd{\newoutputstream}\marg{stream} \\
\cmd{\newinputstream}\marg{stream} \\
\end{syntax}
\glossary(newoutputstream)%
  {\cs{newoutputstream}\marg{stream}}%
  {Creates a new output stream called \meta{stream}.}
\glossary(newinputstream)%
  {\cs{newinputstream}\marg{stream}}%
  {Creates a new input stream called \meta{stream}.}
The macros \cmd{\newoutputstream} and \cmd{\newinputstream} respectively
create a new output\index{stream!new output} and input\index{stream!new input} 
stream called \meta{stream}, where \meta{stream}
should be a string of alphabetic characters, like \texttt{myout} or 
\texttt{myin}.
The \meta{stream} names must be unique, you cannot use the same name 
for two streams even if one is a input stream and the other is an output 
stream. If all the 16 streams of the given type have already been allocated
\tx\ will issue a message telling you about this, of the form:%
\index{No room for a new write}\index{No room for a new read}
\begin{lcode}
No room for a new write   % for an output stream
No room for a new read    % for an input stream
\end{lcode}

    The two \cs{new...stream} commands also provide two empty macros called
\verb?\atstreamopen<stream>? and \verb?\atstreamclose<stream>?. 
If these macros already exist then they are left undisturbed. 
For example if you do:
\begin{lcode}
\newcommand{\atstreamopenmyout}{...}
\newoutputstream{myout}
\newinputstream{myin}
\end{lcode}
Then you will find that three new commands have been created like:
\begin{lcode}
\newcommand{\atstreamclosemyout}{}
\newcommand{\atstreamopenmyin}{}
\newcommand{\atstreamclosemyin}{}
\end{lcode}
You can use \cmd{\renewcommand} to change the definitions of these if you
wish.

\begin{syntax}
\cmd{\IfStreamOpen}\marg{stream}\marg{true-code}\marg{false-code} \\
\end{syntax}
\glossary(IfStreamOpen)%
  {\cs{IfStreamOpen}\marg{stream}\marg{yes}\marg{no}}%
  {If \meta{stream} is open then the \meta{yes} argument is processed
   otherwise the \meta{no} argument is processed.}
The macro \cmd{\IfStreamOpen} checks whether or not the \meta{stream}
stream\index{stream!check open} is open. If it is then 
the \meta{true-code} argument is processed,
while when it is not open the \meta{false-code} argument is processed.

\subsection{Writing to a file}

\index{file!write|(}

    One stream may be used for writing to several different files, although not
simultaneously.

\begin{syntax}
\cmd{\openoutputfile}\marg{filename}\marg{stream} \\
\cmd{\closeoutputstream}\marg{stream} \\
\end{syntax}
\glossary(openoutputfile)%
  {\cs{openoutputfile}\marg{filename}\marg{stream}}%
  {Attaches the file \meta{filename} to the output \meta{stream}.}
\glossary(closeoutputstream)%
  {\cs{closeoutputstream}\marg{stream}}%
  {Detaches and closes the file associated with the output \meta{stream}.}
The command \cmd{\openoutputfile} opens\index{file!open} the file 
called \meta{filename},
either creating it if it does not exist, or emptying it if it already exists.
It then attaches the file to the output\index{stream!output} 
stream called \meta{stream} so that
it can be written to, and then finally calls the macro 
named \verb?\atstreamopen<stream>?. 

    The command \cmd{\closeoutputstream} firstly calls the macro named
\verb?\atstreamclose<stream>? then closes\index{stream!close output} the
output stream \meta{stream}, and finally detaches and 
closes\index{file!close} the associated file.

\begin{syntax}
\cmd{\addtostream}\marg{stream}\marg{text} \\
\end{syntax}
\glossary(addtostream)%
  {\cs{addtostream}\marg{stream}\marg{text}}%
  {Adds \meta{text} to the file associated with the output \meta{stream}.}
The \cmd{\addtostream} command writes \meta{text} to the output stream
\meta{stream}, and hence to whatever file is currently attached to the
stream. The \meta{stream} must be open. Any commands within the \meta{text}
argument will be processed before being written. To prevent command
expansion, precede the command in question with \cmd{\protect}.

    Writing\index{file!write!verbatim} verbatim text to a file is 
treated specially as it is likely
to be the most common usage.
\begin{syntax}
\senv{verbatimoutput}\marg{file} anything \eenv{verbatimoutput} \\
\senv{writeverbatim}\marg{stream} anything \eenv{writeverbatim} \\
\end{syntax}
\glossary(verbatimoutput)%
  {\senv{verbatimoutput}\marg{file}}%
  {The contents of the environment are written verbatim to the \meta{file} 
   file, overwriting anything previously in the file.}
\glossary(writeverbatim)%
  {\senv{writeverbatim}\marg{stream}}%
  {The contents of the environment are written verbatim to the \meta{stream} 
   stream.}
The text within a \Ie{verbatimoutput} environment is written verbatim
to the \meta{file} file. Alternatively, the contents of the
\Ie{writeverbatim} environment are written verbatim to the \meta{stream} 
stream. 

    Specifically, \Ie{verbatimoutput} opens the specified file, writes
to it, and then closes the file. This means that if \Ie{verbatimoutput}
is used more than once to write to a given
file, then only the contents of the last of these outputs is captured 
in the file.
On the other hand, you can use \Ie{writeverbatim} several times to write
to the file attached to the stream and, providing the stream has not
been closed in the meantime, all will be captured.

\index{file!write|)}

\subsection{Reading from a file}

\index{file!read|(}

   One stream may be used for reading from several files, although not
simultaneously.

\begin{syntax}
\cmd{\openinputfile}\marg{filename}\marg{stream} \\
\cmd{\closeinputstream}\marg{stream} \\
\end{syntax}
\glossary(openinputfile)%
  {\cs{openinputfile}\marg{filename}\marg{stream}}%
  {Attaches the file \meta{filename} to the input \meta{stream}.}
\glossary(closeinputstream)%
  {\cs{closeinputstream}\marg{stream}}%
  {Detaches and closes the file associated with the input \meta{stream}.}
The command \cmd{\openinputfile} opens\index{file!open} the file 
called \meta{filename}
and attaches it to the input\index{stream!input} stream called 
\meta{stream} so that
it can be read from. Finally it calls the macro named 
\verb?\atstreamopen<stream>?.
It is an error if \meta{filename} can not be found.

    The command \cmd{\closeinputstream} calls the macro named
\verb?\atstreamclose<stream>?, closes\index{stream!close input} the
output stream \meta{stream}, and then detaches and closes\index{file!close} 
the associated file.

\begin{syntax}
\cmd{\readstream}\marg{stream} \\
\end{syntax}
\glossary(readstream)%
  {\cs{readstream}\marg{stream}}%
  {Reads the entire contents of the file associated with the input \meta{stream}.}
The command \cmd{\readstream} reads the entire contents of the file
currently associated with the input stream \meta{stream}. This
provides the same functionality as \cmd{\input}\marg{filename}.

\begin{syntax}
\cmd{\readaline}\marg{stream} \\
\end{syntax}
\glossary(readaline)%
  {\cs{readaline}\marg{stream}}%
  {Reads a single line from the file associated with the input \meta{stream}.}
The \cmd{\readaline} reads\index{file!read!single line} what \tx\ 
considers to be one line from
the file that is currently associated with the input stream \meta{stream}.

Multiple lines can be read by calling \cmd{\readaline} multiple times.
A warning is issued if there are no more lines to be read (i.e., the
end of the file has been reached).

Just as for writing, reading files\index{file!read!verbatim} 
verbatim is treated specially.
\begin{syntax}
\cmd{\verbatiminput}\marg{file} \cmd{\verbatiminput*}\marg{file} \\
\cmd{\boxedverbatiminput}\marg{file} \cmd{\boxedverbatiminput*}\marg{file} \\
\cmd{\readverbatim}\marg{stream} \cmd{\readverbatim*}\marg{stream} \\
\cmd{\readboxedverbatim}\marg{stream} \cmd{\readboxedverbatim*}\marg{stream} \\
\end{syntax}
\glossary(verbatiminput)%
  {\cs{verbatiminput}\marg{file}}%
  {Acts like \Pe{verbatim} except the contents is read from the \meta{file} file.}
\glossary(verbatiminput*)%
  {\cs{verbatiminput*}\marg{file}}%
  {Acts like \Pe{verbatim*} except the contents is read from the \meta{file} file.}
\glossary(boxedverbatiminput)%
  {\cs{boxedverbatiminput}\marg{file}}%
  {Acts like \Pe{boxedverbatim} except the contents is read from the \meta{file} file.}
\glossary(boxedverbatiminput*)%
  {\cs{boxedverbatiminput*}\marg{file}}%
  {Acts like \Pe{boxedverbatim*} except the contents is read from the \meta{file} file.}
\glossary(readverbatim)%
  {\cs{readverbatim}\marg{stream}}%
  {Acts like \Pe{verbatim} except the contents is read from the file 
   associated with the input \meta{stream}.}
\glossary(readverbatim*)%
  {\cs{readverbatim*}\marg{stream}}%
  {Acts like \Pe{verbatim*} except the contents is read from the file 
   associated with the input \meta{stream}.}
\glossary(readboxedverbatim)%
  {\cs{readboxedverbatim}\marg{stream}}%
  {Acts like \Pe{boxedverbatim} except the contents is read from the file 
   associated with the input \meta{stream}.}
\glossary(readboxedverbatim*)%
  {\cs{readboxedverbatim*}\marg{stream}}%
  {Acts like \Pe{boxedverbatim*} except the contents is read from the file 
   associated with the input \meta{stream}.}
The commands \cmd{\verbatiminput} and 
\cmd{\boxedverbatiminput},\index{frame!verbatim}\index{verbatim!frame} 
 and their
starred versions, act like the \Ie{verbatim} and \Ie{boxedverbatim}
environments, except that they get their text from the \meta{file} file.
It is an error if \meta{file} cannot be found.
Similarly, \cmd{\readverbatim} and \cmd{\readboxedverbatim} get their
text from the file currently attached to the \meta{stream} input stream.
It is an error if \meta{stream} is not open for input.

\index{file!read|)}

\subsection{Example: endnotes}

\LMnote{2010/10/28}{It is confusing for users that the manual contain
  an example as to how one manual provide endnotes, when memoir
  actually provide the functionality in a later section.}
\index{endnotes}
\begin{itshape}
  In an earier version of the manual, this section contained an
  example as to how one could make endnotes. The example is now
  irrelevant, since \theclass\ contain something similar to end notes
  called page notes, see section~\ref{sec:endnotes} on
  page~\pageref{sec:endnotes}. 

  Those interested in the code from the old example, can find it in
  the manual source (it has just been commented out).
\end{itshape}

\begin{comment}
\index{endnotes|(}

    Books like biographies often quote sources for quotations by the subject,
or sources for statements of fact and so on, at the end of the book or chapter.
These are often like a collected set of footnotes. The example shows
a somewhat rough and ready approach to implementing endnotes.

    Typically endnotes come in one of two forms: they are like a normal
footnote, except that the note text is on another page, or; there is
no mark in the body of the text and the note is identified via a small
quote from the text and its page number. 
The example is for the footnote-like form and for endnotes collected
at the end of the document, with an appropriate heading to distinguish
notes from different chapters.

     We have to be careful in choosing names for the macros we will be
defining for endnotes. Remember, you cannot use \cmd{\newcommand}
to define a new command whose name starts \cs{end...}, so \cs{endnote} 
appears to be out.
However, the \tx\ primitive \cmd{\def} command does let you define
a command starting with \cs{end...}. The syntax of the \cmd{\def} command, 
like 
that of many of \tx\ macros, looks strange to \ltx\ eyes. The major
disadvantage in using \cmd{\def} is that it will merrily overwrite any
previous definition with the same name (the \ltx\ \cmd{\newcommand} won't
let you do that). I could use \cmd{\def} for an \cs{endnote} macro, like
\begin{lcode}
\long\def\endnote#1{...}
\end{lcode}
I won't do that, though, as there is at least one \ltx\ class that includes
a \texttt{note} environment and that means that \cs{endnote} is already defined
in that class. To avoid potential pitfalls like that I'll use \cs{enote}
rather than the more evocative \cs{endnote}.

    We need a new counter for the endnotes, starting afresh with each chapter,
and to print in arabic numerals. 
\begin{lcode}
\newcounter{enote}[chapter]
  \renewcommand{\theenote}{\arabic{enote}}
\end{lcode}
And we need a macro to typeset the text of the note. This will take two
arguments, the number of the note, and the text.
\begin{lcode}
\DeclareRobustCommand{\enotetext}[2]{%
  \par\noindent \textsuperscript{#1} #2\par
      \vspace{\baselineskip}}
\end{lcode}
This makes sure that it starts a new non-indented paragraph, then typesets
the first argument (the number) as a superscript and then processes the
second argument (the text of the note). After that it makes sure that 
any paragraph is ended and puts some vertical space in case there is
another note following.

    The basic idea is to define a command, \cs{enote}\marg{text}, like
\cmd{\footnote}, that will write \meta{text} to a file\index{file!write} 
which will be read in later to typeset the \meta{text}. 

    To this end, we need an output stream, and we will use a file with 
extension \pixfile{ent}, the first
part of the file name being the name of the \ltx\ source file; this is
available via the \cmd{\jobname} macro.
\begin{lcode}
\newoutputstream{notesout}
  \openoutputfile{\jobname.ent}{notesout}
\newcommand{\printendnotes}{%
  \closeoutputstream{notesout}%
  \input{\jobname.ent}}
\end{lcode}
The \cmd{\printendnotes} macro can be called at the appropriate place in the
document to print any endnotes. It closes the output file and 
then inputs it\index{file!read} to print the endnotes.

    As well as putting the notes into the file we are also going to
add a heading indicating the chapter. Rather than invent a completely
new kind of heading I'll simply use \cmd{\subsection*} --- the starred
form so that there will be no \prtoc{} entry.
\begin{lcode}
\DeclareRobustCommand{\enotehead}[1]{%
  \subsection*{Notes for chapter #1}}
\end{lcode}
The argument to the \cs{enotehead} macro is the number of a chapter. Also
needed is a method for determining when this heading should be added to 
the endnote file. One simple way is using a counter holding the chapter
number. Initialise the counter to something that is an invalid
chapter number.
\begin{lcode}
\newcounter{savechap}
  \setcounter{savechap}{-1000}
\end{lcode}

    We have the pieces ready, and all that remains is to define
the \cs{enote} macro, which will take one argument --- the text of the
note.
\begin{lcode}
\newcommand{\enote}[1]{%
  \refstepcounter{enote}%      increment the counter
  \textsuperscript{\theenote}% typeset it as a superscript
  \ifnum\value{savechap}=\value{chapter}\else % in a new chapter
    \setcounter{savechap}{\value{chapter}%      save the number
    \addtostream{notesout}{\enotehead{\thechapter}}% the heading
  \fi
  \addtostream{notesout}{\enotetext{\theenote}{#1}}}
\end{lcode}
\cs{enote}, which is used just like \cmd{\footnote}, increments the 
counter for endnotes, typesets that as a superscript, and then writes
the \cs{enotetext} command to the endnotes file. Entries in the \pixfile{ent}
file will look like:
\begin{lcode}
...
\enotehead{3}     % for chapter 3
\enotetext{1}{First end note in chapter 3.}
\enotetext{2}{The next end note.}
...
\end{lcode}

    You can try this, perhaps changing the definition of \cs{enotetext}
to give a better looking presentation of an endnote. There is, however,
a caveat if you use \cs{enote}. 

\vspace{\onelineskip}
\noindent\textbf{Question 3.} What is the caveat?

If you can't
think what it might be, don't worry as it will be dealt with in another
example. 

\index{endnotes|)}
\end{comment}

\subsection{Example: end floats}

\index{end floats|(}

    There are some documents where all figures are required to be grouped
in one place, for instance at the end of the document or perhaps at the
end of each chapter. Grouping at the end of a document with 
chapters is harder, so we'll tackle that one.

   The basic idea is to write out verbatim\index{verbatim!write} 
each figure environment and then read them all back in at the end. 
We will use a stream,\index{stream} let's call
it \texttt{tryout}, and call our file for figures \file{tryout.fig}.
\begin{lcode}
\newoutputstream{tryout}
\openoutputfile{tryout.fig}{tryout}
\end{lcode}

    If all were simple, in the document we could then just do
\begin{lcode}
\begin{writeverbatim}{tryout}
\begin{figure} ... \end{figure}
\end{writeverbatim}
...
\closeoutputstream{tryout}
\input{tryout.fig}
\end{lcode}

    So, what's the problem?

    By default figure captions are numbered per chapter, and are preceeded
by the chapter number; more precisely, the definition of a figure number
is 
\begin{lcode}
\thechapter.\arabic{figure}
\end{lcode}
If we simply lump all the figures at 
the end, then they
will all be numbered as if they were in the final chapter. 
For the sake of argument assume that the last chapter is number 10.
The nth figure will then be numbered 10.n.
One thing that we
can do rather simply is to change the definition of the figure by using
another counter, let's call it \texttt{pseudo}, instead of the chapter.
\begin{lcode}
\newcounter{pseudo}
  \renewcommand{\thepseudo}{\arabic{pseudo}}
\renewcommand{\thefigure}{\thepseudo.\arabic{figure}}
\end{lcode}
Now, all we should have to do is arrange that the proper value of 
\texttt{pseudo}
is available before each figure is typeset at the end. The code around
the \Ie{figure} environments might then look like this
\begin{lcode}
\addtostream{tryout}{\protect\setcounter{pseudo}{\thechapter}}
\begin{writeverbatim}{tryout}
\begin{figure}...
\end{lcode}
and a part of the file might then look like
\begin{lcode}
...
\setcounter{pseudo}{4}
\begin{figure}...
\end{lcode}
The \cmd{\protect} before the \cmd{\setcounter} command will stop it
from expanding before it is written to the file, while the \cmd{\thechapter}
command \emph{will} be expanded to give the actual number of the current 
chapter. This looks better as now at least the figure will be numbered 4.n 
instead of 10.n.

    There is one last snag --- figure numbers are reset at the start of each
chapter --- but if we just dump the figures at the end of the document
then although the chapter part of the number will alter appropriately
because of the \texttt{pseudo} process,
the second part of the number will just increase continuously. It looks
as though we should write out a change to the chapter counter at the start
of each chapter. If we do that, then we should be able to get rid of the
\texttt{pseudo} counter, which sounds good. But, and this is almost the 
last but,
what if there are chapters after we have read in the figure file? To
cater for this the chapter number of the last chapter before the file must
be saved, and then restored after the figures have been processed.

    Finally, wouldn't it be much better for the user if everything was
wrapped up in an environment that handled all the messy stuff?

    Here is the final code that I am going to produce which, by the way,
is displayed in the \Ie{boxedverbatim} environment\index{line number} 
with line numbers and the following settings, just in case there is
a page break in the middle of the box.
\begin{lcode}
\nobvbox
\bvperpagetrue
\bvtopofpage{\begin{center}\normalfont%
             (Continued from previous page)\end{center}}
\bvendofpage{}
\resetbvlinenumber
\linenumberfrequency{1}
\bvnumbersoutside
\linenumberfont{\footnotesize\rmfamily}
\begin{boxedverbatim}
...
\end{lcode}

\nobvbox
\bvperpagetrue
\bvtopofpage{\begin{center}\normalfont%
             (Continued from previous page)\end{center}}
\bvendofpage{}
\resetbvlinenumber
\linenumberfrequency{1}
\bvnumbersoutside
\linenumberfont{\footnotesize\rmfamily}
\begin{boxedverbatim}
\newoutputstream{tryout}
\openoutputfile{\jobname.fig}{tryout}
\newcounter{pseudo}
\renewcommand{\thefigure}{\thepseudo.\arabic{figure}}
\newenvironment{writefigure}{%
  \ifnum\value{chapter}=\value{pseudo}\else
    \setcounter{pseudo}{\value{chapter}}
    \addtostream{tryout}{\protect\stepcounter{chapter}}
    \addtostream{tryout}{\protect\addtocounter{chapter}{-1}}
    \addtostream{tryout}{%
      \protect\setcounter{pseudo}{\thechapter}}
  \fi
  \addtostream{tryout}{\protect\begin{figure}}
  \writeverbatim{tryout}}%
 {\endwriteverbatim\finishwritefigure}
\newcommand{\finishwritefigure}{%
  \addtostream{tryout}{\protect\end{figure}}}
\newcommand{\printfigures}{%
  \closeoutputstream{tryout}%
  \input{\jobname.fig}%
}
\end{boxedverbatim}
\linenumberfrequency{0}
    The above code should be either put in the preamble\index{preamble} 
or in a separate package\index{package} file.

   The first four lines of the code perform the initial setup described
earlier. Lines 1 and 2 set up for outputting\index{file!write} to a file 
\verb?\jobname.fig?, which
is where the figures will be collected. Lines 3 and 4 create the 
new counter\index{new!counter}
we need and change the construction of the figure number. The rest of the code
defines a new environment\index{new!environment} \Ie{writefigure} 
which is to be used instead 
of the \Ie{figure} environment. It writes its content out to the 
\texttt{tryout} stream.

    In line 6 a check is made to see if the current values of the 
\Icn{chapter} and \Icn{pseudo} counters are the same; 
nothing is done if they are. If they are
different, it means that this is the first figure in the chapter and we have
to put appropriate information into the figure file. Line 7 sets the
\Icn{pseudo} counter to the value of the \Icn{chapter} counter 
(if there is another \Ie{writefigure} in the chapter it will then 
skip over the code in lines 7 to 11).
The next lines put (where N is the number of the current chapter):
\begin{lcode}
\stepcounter{chapter}
\addtocounter{chapter}{-1}
\setcounter{pseudo}{N}
\end{lcode}
into the figure file. Stepping the chapter number (by one) resets the 
following figure number, and then subtracting one from the stepped number
returns the chapter number to its original value. 
Finally the counter \Icn{pseudo} is set to the number of the 
current chapter.

    Line 13 puts
\begin{lcode}
\begin{figure}
\end{lcode}
into the figure file, and line 14 starts the 
\Ie{writeverbatim}\index{verbatim!write} environment.

    For the end of the \Ie{writefigure} environment (line 15), the 
\Ie{writeverbatim} environment is ended and after that the 
\cmd{\finishwritefigure} macro
is called. This is defined in lines 16 and 17, and simply writes
\begin{lcode}
\end{figure}
\end{lcode}
out to the figure file. The \cmd{\endwriteverbatim}, and any other kind of
\cs{end...verbatim}, command is very sensitive to anything that follows it,
and in this case did not like to be immediately followed by an
\verb?\addtostream{...}?, but did not mind it being wrapped up in 
the \cmd{\finishwritefigure} macro.

    The \cmd{\printfigures} macro defined in the last three lines of the code
simply closes the output stream\index{stream!output} and then inputs the 
figures\index{file!read} file.

    As an example of how this works, if we have the following source code:
\begin{lcode}
\chapter{The fifth chapter}
...
\begin{writefigure}
%% illustration and caption
\end{writefigure}
...
\begin{writefigure}
%% another illustration and caption
\end{writefigure}
\end{lcode}
then the figure file will contain the following (shown verbatim in the 
\Ie{fboxverbatim}\index{framed!verbatim} environment).

\begin{fboxverbatim}
\stepcounter{chapter}
\addtocounter{chapter}{-1}
\setcounter{pseudo}{5}
\begin{figure}
%% illustration and caption
\end{figure}
\begin{figure}
%% another illustration and caption
\end{figure}
\end{fboxverbatim}

\index{end floats|)}


\subsection{Example: questions and answers}

\index{questions and answers|(}

    Text books often have questions at the end of a chapter. Sometimes answers
are also provided at the end of the book, or in a separate teachers guide.
During the draft stages of such a book it is useful to keep the
questions and answers together in the source and paper drafts, only removing
or repositioning the answers towards the end of the writing process.

    This example provides an outline for meeting these desires. For 
pedagogical purposes I use a \cmd{\label} and \cmd{\ref} technique although
there are better methods. The example also shows that not
everything works as expected --- it is a reasonably accurate rendition
of the process that I actually went through in designing it.

    First we need a counter for the questions and we'll use an 
environment\index{environment!new}
for questions as these may be of any complexity. The environment takes one
argument --- a unique key to be used in a \cmd{\label}.
\begin{lcode}
\newcounter{question} \setcounter{question}{0}
\renewcommand{\thequestion}{\arabic{question}}
\newenvironment{question}[1]%
  {\refstepcounter{question}
   \par\noindent\textbf{Question \thequestion:}\label{#1}}%
  {\par}
\end{lcode}
I have used \cmd{\refstepcounter} to increment\index{counter!increment} 
the counter so that
the \cmd{\label} will refer to it, and not some external counter.

    We will use a file, called \verb?\jobname.ans? to collect the answers
and this will be written\index{file!write} to by a stream.\index{stream} 
There is also a convenience
macro, \cmd{\printanswers}, for the user to call to print the answers.
\begin{lcode}
\newoutputstream{ansout}
\end{lcode}


    A matching environment\index{environment!new} for answers is required. 
The argument to the environment is the key of the question.

   In \Lopt{draft} mode it is simple, just typeset the answer and no need to
bother with any file printing (remember that \piif{ifdraftdoc} is \ptrue\ for a 
\Lopt{draft} mode document). 
\begin{lcode}
\ifdraftdoc                       % when in draft mode
\newenvironment{answer}[1]%
  {\par\noindent\textbf{Answer \ref{#1}:}}%
  {\par}
\newcommand{\printanswers}{}
\else                             % when not in draft mode
\end{lcode}

   In \Lopt{final} mode the \Ie{answer} environment must write its contents 
verbatim to the \pixfile{ans} file for printing by \cmd{\printanswers}.
Dealing with these in reverse order, this is the definition of
\cmd{\printanswer} when not in \Lopt{draft} mode.
\begin{lcode}
\newcommand{\printanswers}{%
  \closeoutputstream{ansout}
  \input{\jobname.ans}}
\end{lcode}
 
    Now for the tricky bit, the \Ie{answer} environment. First define an
environment\index{environment!new} that makes sure our 
output\index{stream!output} stream is open, and which then
writes the answer title to the stream.
\begin{lcode}
\newenvironment{@nswer}[1]{\@bsphack
  \IfStreamOpen{ansout}{}{%
    \openoutputfile{\jobname.ans}{ansout}%
  }%
  \addtostream{ansout}{\par\noindent\textbf{Answer \ref{#1}:}}%
  }{\@esphack}
\end{lcode}
The macros \cmd{\@bsphack} and \cmd{\@esphack} are \ltx\ kernel macros
that will gobble\index{space!gobble} extraneous spaces around the 
environment. In other words,
this environment will take no space in the typeset result. The
\cmd{\IfStreamOpen} macro is used to test whether or not the stream is open, 
and if it isn't then it opens it. The answer title is then written
out to the stream. Now we can define the \Ie{answer} environment so that
its contents get written out\index{write!verbatim} verbatim.
\begin{lcode}
\newenvironment{answer}[1]%
  {\@bsphack\@nswer{#1}\writeverbatim{ansout}}%
  {\par\endwriteverbatim\end@nswer\@esphack}
\fi                               % end of \ifdraftdoc ...\else ...
\end{lcode}

    When I was testing this code I had a surprise as I got nasty error messages
from \ltx\ the first time around, but it worked fine when I processed the
source a second time! The problem lies in the code line
\begin{lcode}
\addtostream{ansout}{\par\noindent\textbf{Answer \ref{#1}:}}%
\end{lcode}  

    The first time around, \ltx\ processed the \cmd{\ref} command and of
course it was undefined. In this case \cmd{\ref} gets replaced by the
code to print the error message, which involves macros that have \texttt{@}
in their names, which \ltx\ only understands under special circumstances.
The second time around \cmd{\ref} gets replaced by the question number
and all is well. I then remembered that some commands need 
protecting\index{protect}
when they are written out, so I tried (I've wrapped the line to fit)
\begin{lcode}
\addtostream{ansout}{\par\noindent
  \protect\makeatletter\textbf{Answer 
  \protect\ref{#1}:}\protect\makeatother}%
\end{lcode}  
which did work but seemed very clumsy.

    I then took another line of attack, and looked at the definition
of \cmd{\ref} to see if I could come up with something that didn't
expand into \texttt{@} names. The result of this was
\begin{lcode}
\addtostream{ansout}{\par\noindent\textbf{Answer 
                                          \quietref{#1}:}}%
\end{lcode}  
In the kernel file \file{ltxref.dtx} I found the definition of \cmd{\ref}
and it used a macro \cmd{\@setref} (shown below) to do its work.
My \cmd{\quietref} locally changes the definition of \cmd{\@setref} 
and then calls \cmd{\ref}, which will then use the modified \cmd{\@setref}.
\begin{lcode}
\def\@setref#1#2#3{%        %% kernel definition
  \ifx#1\relax
    \protect\G@refundefinedtrue
    \nfss@text{\reset@font\bfseries ??}%
    \@latex@warning{Reference `#3' on page \thepage \space
                    undefined}%
  \else
    \expandafter#2#1\null
  \fi}

\DeclareRobustCommand{\quietref}[1]{\begingroup
  \def\@setref##1##2##3{%
    \ifx##1\relax ??\else
      \expandafter##2##1\null
    \fi
  \ref{#1}\endgroup}
\end{lcode}

    Having gone all round the houses, the simplest solution was actually
one that I had skipped over
\begin{lcode}
\addtostream{ansout}{\par\noindent\textbf{Answer 
                                          \protect\ref{#1}:}}%
\end{lcode}  

    The advantage of using the \cmd{\label} and \cmd{\ref} mechanism is that
a question and its answer need not be adjacent in the source; I think that
you have seen some of the disadvantages. Another disadvantage is that it
is difficult to use, although not impossible, if you want the answers in
a separate document.

    The real answer to all the problems is force an answer to come immediately
after the question in the source and to use the \Icn{question} counter
directly, as in the endnotes\index{endnotes} example. In the traditional manner,
this is left as an exercise for the 
reader.

\index{questions and answers|)}

\index{stream|)}

\index{file|)}

\section{Answers}

\noindent\textbf{Question 1.} As a convenience, the 
argument\index{argument!optional} to the 
environment could be made optional, defaulting, say, to the current
line width. If the default width is used the frame will be wider
than the line width, so we really ought to make the width argument
specify the width of the frame instead of the minipage. This 
means calculating a reduced width for the minipage based on
the values of \lnc{\fboxsep} and \lnc{\fboxrule}.
\begin{lcode}
\newsavebox{\minibox}
\newlength{\minilength}
\newenvironment{framedminipage}[1][\linewidth]{%
  \setlength{\minilength}{#1}
  \addtolength{\minilength}{-2\fboxsep} 
  \addtolength{\minilength}{-2\fboxrule}
  \begin{lrbox}{\minibox}\begin{minipage}{\minilength}}%
  {\end{minipage}\end{lrbox}\fbox{\usebox{\minibox}}}
\end{lcode}


\vspace{\onelineskip}
\noindent\textbf{Question 2.} There are at least three reasonable answers.
In increasing or decreasing order of probability (your choice) they are:
\begin{itemize}
\item I took Sherlock Holmes' advice and followed the methods outlined
   in the chapter;
\item I used a package, such as the \Lpack{answer} package which is designed
  for the purpose;
\item I just wrote the answers here.
\end{itemize}

\LMnote{2010/10/28}{Removed this answer as we have removed the question}
% \vspace{\onelineskip}
% \noindent\textbf{Question 3.} If \ltx\ writes text out to an external
% file which will be read by \ltx\ at some time, any 
% fragile\index{fragile} commands 
% in the text must be \cmd{\protect}ed.\index{protect}

%%%%%%%%%%%%%%%%%%%%%%%%%%%%%%%%%%%%%%%%%%%%%%%%%%%%%%%%%%%%%%%%%%
%%%%%%%%% mbook

%#% extend
%#% extstart include cross-referencing.tex

\svnidlong
{$Ignore: $}
{$LastChangedDate: 2013-04-24 17:14:15 +0200 (Wed, 24 Apr 2013) $}
{$LastChangedRevision: 442 $}
{$LastChangedBy: daleif $}
