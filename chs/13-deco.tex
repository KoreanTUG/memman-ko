\chapter{Decorative text} \label{chap:signposts}


\newcommand{\tepi}[2]{\epigraph{#1}{#2}}

% \tepi{My soul, seek not the life of immortals; but enjoy to the full
%       the resources that are within your reach}
%      {\textit{Pythian Odes \\ Pindar}}
\tepi{Too servile a submission to the books and opinions of the ancients
      has spoiled many an ingenious man, and plagued the world with an
      abundance of pedants and coxcombs.}
     {James Puckle (1677?--1724)}


    By now we have covered most aspects of typesetting. As far as
the class is concerned this chapter describes the slightly more fun 
task of typesetting epigraphs\index{epigraph}.

     Some authors like to add an interesting quotation\index{quotation} 
at either the start or end of a chapter. The class provides commands
to assist in the typesetting of a single epigraph. Other authors like to 
add many such quotations\index{quotation} and the class provides 
environments to cater for these as well.
Epigraphs can be typeset at either the left, the center or the right of 
the typeblock\index{typeblock}. A few example epigraphs are exhibited here, and
others can be found in an article by
Christina Thiele~\cite{TTC199} where she reviewed the \Lpack{epigraph}
package~\cite{EPIGRAPH} which is included in the class.

 \section{Epigraphs} 
\index{epigraph|(}

% \tepi{The whole is more than the sum of the parts.}
%      {\textit{Metaphysica \\ Aristotle}}

 %\subsection{The \texttt{epigraph} command}

  The original inspiration for \cmd{\epigraph} was Doug Schenck's
 for the epigraphs in our book~\cite{EBOOK}. That was hard wired for
 the purpose at hand. The version here provides much more flexibility.


\begin{syntax}
\cmd{\epigraph}\marg{text}\marg{source} \\
\end{syntax}
\glossary(epigraph)%
  {\cs{epigraph}\marg{text}\marg{source}}%
  {Typesets the \meta{text} and \meta{source} of an epigraph.}
 The command \cmd{\epigraph}  typesets
 an epigraph using \meta{text} as the main text of the epigraph and
 \meta{source} being the original author (or book, article, etc.)
 of the quoted text. By default the epigraph is placed at the right hand 
 side of the typeblock\index{typeblock}, and the \meta{source} is typeset at the bottom
 right of the \meta{text}.


\begin{syntax}
\senv{epigraphs}  \\
  \cmd{\qitem}\marg{text}\marg{source} \\
  ... \\
\eenv{epigraphs} \\
\end{syntax}
\glossary(epigraphs)%
  {\senv{epigraphs}}%
  {Environment for several epigraphs.}
\glossary(qitem)%
  {\cs{qitem}\marg{text}\marg{source}}%
  {Typesets the \meta{text} and \meta{source} of an epigraph in an 
   \texttt{epigrpahs} environment.}
 The \Ie{epigraphs} environment typesets a list of epigraphs, and by default
 places them at the right hand side of the typeblock\index{typeblock}.
  Each epigraph in an \Ie{epigraphs} environment is specified by a 
 \cmd{\qitem} (analagous to the \cmd{\item}
 command in ordinary list environments).
 By default, the \meta{source} is typeset at the bottom right of the
 \meta{text}. 

 
 \section{General}

 \tepi{Example is the school of mankind, and they will learn at no other.}
      {\textit{Letters on a Regicide Peace}\\ \textsc{Edmund Burke}}

   The commands described in this section apply to both the \cmd{\epigraph}
command and the \Ie{epigraphs} environment. But first of all, note that an
epigraph immediately after a heading\index{heading} will cause the 
first paragraph\index{paragraph!indentation}
of the following text to be indented. If you want the initial paragraph
to have no indentation, then start it with the \cmd{\noindent} command.

\begin{syntax}
  \lnc{\epigraphwidth} \\
 \cmd{\epigraphposition}\marg{flush} \\
\end{syntax}
\glossary(epigraphwidth)%
  {\cs{epigraphwidth}}%
  {Textwidth for epigraphs.}
\glossary(epigraphposition)%
  {\cs{epigraphposition}\marg{flush}}%
  {Sets the horizontal placement of epigraphs.}
  The epigraphs are typeset in a minipage of width \lnc{\epigraphwidth}. 
The default value for this can be changed using the \cmd{\setlength} command. 
Typically,  epigraphs are typeset in a measure much less than the width of 
the typeblock\index{typeblock}. The horizontal position of an epigraph
in relation to the main typeblock is controlled by 
the \meta{flush} argument to the \cmd{\epigraphposition} declaration.
The default value is \texttt{flushright}, so that epigraphs are set
at the right hand side of the typeblock. This can be changed
to \texttt{flushleft} for positioning at the left hand side or to
\texttt{center} for positioning at the center of the 
typeblock\index{typeblock}.

\begin{syntax}
\cmd{\epigraphtextposition}\marg{flush} \\
\end{syntax}
\glossary(epigraphtextposition)%
  {\cs{epigraphtextposition}\marg{flush}}
  {Sets the justification for epigraph text.}
 In order to avoid bad line breaks, the epigraph \meta{text} is normally 
typeset raggedright. 
    The \meta{flush} argument to the \cmd{\epigraphtextposition} 
declaration controls the \meta{text} typesetting style. By default this is 
\texttt{flushleft} (which produces raggedright text). The sensible values
are \texttt{center} for centered text, \texttt{flushright} for raggedleft
text, and \texttt{flushleftright} for normal justified text.

    If by any chance you want the \meta{text} to be typeset in some
other layout style, the easiest
way to do this is by defining a new environment which sets the paragraphing
parameters to your desired values. For example, as the \meta{text} is
typeset in a minipage, there is no paragraph indentation. If you
want the paragraphs to be indented and justified then define
a new environment like:
\begin{lcode}
\newenvironment{myparastyle}{\setlength{\parindent}{1em}}{}
\end{lcode}
 and use it as: 
\begin{lcode}
\epigraphtextposition{myparastyle}
\end{lcode}

\begin{syntax}
 \cmd{\epigraphsourceposition}\marg{flush} \\
\end{syntax}
\glossary(epigraphsourceposition)%
  {\cs{epigraphsourceposition}\marg{flush}}%
  {Sets the placement of the source within an epigraph.}
 The \meta{flush} argument to the \cmd{\epigraphsourceposition}
declaration controls the position 
of the \meta{source}.
 The default value is \texttt{flushright}. It can be changed to
 \texttt{flushleft}, \texttt{center} or \texttt{flushleftright}.

 For example, to have epigraphs centered with the \meta{source} at the left,
 add the following to your document.
 \begin{lcode}
 \epigraphposition{center}
 \epigraphsourceposition{flushleft}
 \end{lcode}

\begin{syntax}
\cmd{\epigraphfontsize}\marg{fontsize} \\
\end{syntax}
\glossary(epigraphfontsize)%
  {\cs{epigraphfontsize}\marg{fontsize}}%
  {Font size to be used for epigraphs.}
 Epigraphs are often typeset in a smaller font than the main text. The
\meta{fontsize} argument to the \cmd{\epigraphfontsize}
declaration sets the font size to be used.
 If you don't like the default value (\cmd{\small}), you can easily change
it to, say \cmd{\footnotesize} by:
\begin{lcode}
\epigraphfontsize{\footnotesize}
\end{lcode}
		
\begin{syntax}
 \lnc{\epigraphrule} \\
\end{syntax}
\glossary(epigraphrule)%
  {\cs{epigraphrule}}%
  {Thickness of the rule drawn between the text and source of an epigraph.}
 By default, a rule is drawn between the \meta{text} and \meta{source},
 with the rule thickness being given by the value of \lnc{\epigraphrule}.
 The value can be changed by using \cmd{\setlength}.
 A value of \texttt{0pt} will eliminate the rule. Personally, I dislike
 the rule in the list environments.

\begin{syntax}
 \lnc{\beforeepigraphskip} \\
 \lnc{\afterepigraphskip} \\
\end{syntax}
\glossary(beforeepigraphskip)%
  {\cs{beforeepigraphskip}}%
  {Vertical space before an epigraph.}
\glossary(afterepigraphskip)%
  {\cs{afterepigraphskip}}%
  {Vertical space after an epigraph.}
 The two \verb?...skip? commands specify the amount of vertical space inserted
 before and after typeset epigraphs. Again, these can be changed by
 \cmd{\setlength}. It is desireable that the sum of their values should be an 
 integer multiple of the \lnc{\baselineskip}.

 Note that you can use normal LaTeX commands in the \meta{text} and
 \meta{source} arguments. You may wish to use different fonts for the
 \meta{text} (say roman) and the \meta{source} (say italic).

 The epigraph at the start of this section was specified as:
 \begin{lcode}
 \epigraph{Example is the school of mankind,
           and they will learn at no other.}
  {\textit{Letters on a Regicide Peace}\\ \textsc{Edmund Burke}}
 \end{lcode}

 \section{Epigraphs before chapter headings}

\begingroup
 \epigraphsourceposition{flushleft}
 \tepi{If all else fails, immortality can always be assured by spectacular
       error.}
      {\textsf{John Kenneth Galbraith}}
\endgroup

    The \cmd{\epigraph} command and the \Ie{epigraphs} environment typeset
 an epigraph at the point in the text where they are placed. The
 first thing that a \cmd{\chapter} command does is to start off a new page,
 so another mechanism is provided for placing an epigraph just before
 a chapter heading\index{heading!chapter}.
    
\begin{syntax}
 \cmd{\epigraphhead}\oarg{distance}\marg{text} \\
\end{syntax}
\glossary(epigraphhead)%
  {\cs{epigraphhead}\oarg{distance}\marg{text}}%
  {Stores \meta{text} for printing at \meta{distance} below the page header.}
  The \cmd{\epigraphhead} macro  stores \meta{text} 
 for printing at \meta{distance} below the header\index{header} on a page.
 \meta{text} can be ordinary text or, more likely, can be either an
 \cmd{\epigraph} command or an \Ie{epigraphs} environment. By default, the 
 epigraph will be typeset at the righthand margin\index{margin!right}.
 If the command is immediately preceded by a \cmd{\chapter} or \cmd{\chapter*} 
 command, the epigraph is typeset on the chapter title page.

    The default value for the optional \meta{distance} argument is set so
 that an \cmd{\epigraph} consisting of a single line of 
quotation\index{quotation} and a single
line denoting the source is aligned with the bottom of the `Chapter X'
line produced by the \cmd{\chapter} command using the \cstyle{default} 
chapterstyle. In other cases you will
have to experiment with the \meta{distance} value. The value for
\meta{distance} can be either a integer or a real number. The units
are in terms of the current value for \lnc{\unitlength}. A typical value
for \meta{distance} for a single line quotation\index{quotation} and 
source for a \cmd{\chapter*} might be about 70 (points). A positive value
of \meta{distance} places the epigraph below the page heading and a negative
value will raise it above the page heading.

    Here's some example code:
 \begin{lcode}
 \chapter*{Celestial navigation}
 \epigraphhead[70]{\epigraph{Star crossed lovers.}{\textit{The Bard}}}
 \end{lcode}
 The \meta{text} argument is put into a minipage of width \lnc{\epigraphwidth}.
 If you use something other than \cmd{\epigraph} or \Ie{epigraphs} for the
 \meta{text} argument, you may have to do some positioning of the text
 yourself so that it is properly located in the minipage. For example
 \begin{lcode}
 \chapter{Short}
 \renewcommand{\epigraphflush}{center}
 \epigraphhead{\centerline{Short quote}}
 \end{lcode}

 The \cmd{\epigraphhead} command changes the page style for the page on
 which it is specified, so there should be no text between the
\cmd{\chapter} and the \cmd{\epigraphhead} commands. The page style
is identical to the \pstyle{plain} page style except for the inclusion of
the epigraph.
    If you want a more fancy style for epigraphed chapters you will have
to do some work yourself.

\begin{syntax}
\cmd{\epigraphforheader}\oarg{distance}\marg{text} \\
\cmd{\epigraphpicture} \\
\end{syntax}
\glossary(epigraphforheader)%
  {\cs{epigraphforheader}\oarg{distance}\marg{text}}%
  {Puts \meta{text} into a zero-sized picture (\cs{epigraphpicture})
   at the coordinate position (0, -meta{distance}).}
\glossary(epigraphpicture)%
  {\cs{epigraphpicture}}%
  {A zero-sized picture holding the result of \cs{epigraphforheader}.}
The \cmd{\epigraphforheader} macro takes the same arguments as
\cmd{\epigraphhead} but puts \meta{text} into a zero-sized picture at
the coordinate position \verb?(0,-<distance>)?; the macro 
\cmd{\epigraphpicture}
holds the resulting picture. This can then be used as part of a 
chapter pagestyle, as in
\begin{lcode}
\makepagestyle{mychapterpagestyle}
...
\makeoddhead{mychapterpagestyle}{}{}{\epigraphpicture}
\end{lcode}
    Of course the \meta{text} argument for \cs{epigraphforheader} need not
be an \cs{epigraph}, it can be arbitrary text.

\begin{syntax}
 \cmd{\dropchapter}\marg{length} \\
 \cmd{\undodrop} \\
\end{syntax}
\glossary(dropchapter)%
  {\cs{dropchapter}\marg{length}}%
  {Lowers subsequent chapter heads by \meta{length}.}
\glossary(undodrop)%
  {\cs{undodrop}}%
  {Following a \cs{dropchapter} restores subsequent chapter heads to their 
   normal position.}
 If a long epigraph is placed before a chapter title it is possible that the
 bottom of the epigraph may interfere with the chapter title. The command
 \cmd{\dropchapter} will lower any subsequent chapter titles by 
 \meta{length}; a negative \meta{length} will raise the titles.
 The command \cmd{\undodrop} restores subsequent chapter titles to their default
 positions. For example:
 \begin{lcode}
 \dropchapter{2in}
 \chapter{Title}
 \epigraphhead{long epigraph}
 \undodrop
 \end{lcode}

\index{epigraph|)}

\begin{syntax}
 \cmd{\cleartoevenpage}\oarg{text} \\
\end{syntax}
\glossary(cleartoevenpage)%
  {\cs{cleartoevenpage}\oarg{text}}%
  {Clears the current page and moves to the next verso page; the optional
   \meta{text} is put on the skipped page (if there is one).}
 On occasions it may be desirable to put something (e.g., an epigraph, a map,
 a picture) on the page facing the start
 of a chapter, where the something belongs to the chapter that is about to 
 start rather than the chapter that has just ended. In order to do this 
 in a document that is going to be printed
 doublesided, the chapter must start on an odd numbered page and the 
 pre-chapter material put on the immediately preceding even numbered page.
 The \cmd{\cleartoevenpage} command is like \cmd{\cleardoublepage} except
 that the page following the command will be an even numbered page, and the
 command takes an optional argument 
 which is applied to the skipped page (if any).

    Here is an example:
\begin{lcode}
 ... end previous chapter.
 \cleartoevenpage
 \begin{center}
 \begin{picture}... \end{picture}
 \end{center}
 \chapter{Next chapter}
\end{lcode}
 If the style is such that chapter headings\index{heading!chapter} are put at the top of the pages,
 then it would be advisable to include \verb?\thispagestyle{empty}? 
(or perhaps \texttt{plain})
 immediately after \cmd{\cleartoevenpage} to avoid a heading related to the
 previous chapter from appearing on the page. 

 If the something is like a figure\index{figure} with a numbered caption and the numbering
 depends on the chapter numbering, then the numbers have to be hand set (unless
 you define a special chapter command for the purpose). For example:
\begin{lcode}
 ... end previous chapter.
 \cleartoevenpage[\thispagestyle{empty}] % a skipped page to be empty
 \thispagestyle{plain}
 \addtocounter{chapter}{1} % increment the chapter number
 \setcounter{figure}{0}    % initialise figure counter
 \begin{figure}
 ...
 \caption{Pre chapter figure}
 \end{figure}

 \addtocounter{chapter}{-1} % decrement the chapter number
 \chapter{Next chapter}     % increments chapter & resets figure numbers
 \addtocounter{figure}{1}   % to account for pre-chapter figure
\end{lcode}
 

 \subsection{Epigraphs on book or part pages}

\index{epigraph|(}

    If you wish to put an epigraphs on \cmd{\book} or \cmd{\part}
pages you have to do a little more work than in other cases. This
is because these division commands do some page flipping before and
after typesetting the title.

   One method is to put the epigraph into the page header as for epigraphs
before \cmd{\chapter} titles. By suitable adjustments the epigraph can be
placed anywhere on the page, independently of whatever else is on the page.
     A similar scheme may be used for epigraphs on other kinds of pages. 
 The essential
 trick is to make sure that the \pstyle{epigraph} pagestyle is used for
 the page.
    For an epigraphed bibliography\index{bibliography} 
or index\index{index}, the macros \cmd{\prebibhook}
or \cmd{\preindexhook} can be appropriately modified to do this.

    The other method is to subvert the \cmd{\beforepartskip} command 
for epigraphs before the title, or the \cmd{\afterpartskip} command
for epigraphs after the title (or the equivalents for \cmd{\book} pages).

    For example:
\begin{lcode}
\let\oldbeforepartskip\beforepartskip % save definition
\renewcommand*{\beforepartskip}{%
  \epigraph{...}{...}% an epigraph
  \vfil}
\part{An epigraphed part}
...
\renewcommand*{\beforepartskip}{%
  \epigraph{...}{...}% another epigraph
  \vfil}
\part{A different epigraphed part}
...
\let\beforepartskip\oldbeforepartskip % restore definition
\part{An unepigraphed part}
...
\end{lcode}


\index{epigraph|)}

%#% extend
%#% extstart include poetry.tex
